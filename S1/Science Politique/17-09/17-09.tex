\documentclass[12pt, a4paper, openany]{book}

\usepackage[latin1]{inputenc}
\usepackage[T1]{fontenc}
\usepackage[francais]{babel}
\author{Jérémy B.}
\date{}
\title{Cours de Science Politique (UFR Amiens)}
\pagestyle{plain}

\begin{document}
\maketitle

\chapter{Introduction en Science Politique}

La science politique constitue une certaine manière d'analyser, d'étudier la politique. C'est une discipline dont le but dont l'objectif est de comprendre sur des bases scientifiques l'organisation, le fonctionnement politique des sociétés.

Étudier la science politique, c'est essayer de comprendre qui gouverne, pourquoi, comment sont prises les décisions etc...

C'est une discipline qui se situe au carrefour de plusieurs autres (droit, économie, histoire, sociologie) et dont les méthodes d'analyse sont les mêmes que celles utilisés par les sciences sociales.

\chapter{Analyser scientifiquement les phénomènes politiques}

	\section{La science politique est une science sociale}

Les sciences sociales cherchent à comprendre les comportements humains en partant du principe que les comportements ne relèvent pas du hasard ni de la biologie mais du social. L'objectif de la recherche est d'identifier des régularités statistiques mais aussi de les expliquer.

Durkheim et son étude sur le suicide\footnote{"Le suicide" 1897}: habituellement interprété comme un acte relevant exclusivement de la psychologie individuelle, mais, pour Durkheim, il a des causes sociales (il est imposé du dehors, à l'individu, par la société). Durkheim va donc balayer les explications d'ordre naturel. Il va démontrer que si on se suicide d'avantage en été qu'en qu'en hiver, ce n'est pas parce qu'il fait plus chaud mais parce que l'intensité de la vie sociale varie selon les saisons.

La société entraine donc des contraintes (des coercitions) plus ou moins explicite sur l'individu, ce sont les faits sociaux.

Durkheim nous explique donc qu'au 19e siècle, on se suicide plus en ville qu'à la campagne, les hommes plus que les femmes etc...\footnote{Aujourd'hui, le taux de suicide le plus élevé se situe chez les agriculteurs} Cependant, l'analyse de Durkheim va au dela de ces constats car il a une idée en tête: c'est la question de l'intégration sociale dans les nouvelles sociétés industrialisés. Pour Durkheim, le suicide constitue donc un dysfonctionnement de cette intégration. Il ne réfute pas non plus les questions psychologiques, ethniques, biologique, il les dépasse.

Si un individu est déprimé, celà peut venir du fait que l'individu est isolé socialement: pas de familles, amis etc...

L'intégration sociale est donc un facteur déterminant chez l'individu.

Selon Durkheim, l'homme n'agit pas librement, mais son comportement dépend d'un contexte social qui le fait agir. Si le suicide est un fait social, son explication doit lui aussi être recherché dans un autre fait social. La statistique permet de mettre en évidence la contrainte extérieur sur l'individu qui suit des règles à son insu.

Un fait social obéit à trois caractéristiques:
\begin{itemize}
\item Extérieur à l'individu
\item Pouvoir de contrainte (la coercition)
\item S'explique par un autre fait social
\end{itemize}

Dans un ouvrage "La démocratie de l'abstention", des professeurs de Science Politique ont étudié la participation éléctorale en banlieu et ont prouvé que si les personnes seules votaient moins, c'est parce qu'il manquait un "rappel" au devoir civique. 

On ne peut pas penser le politique sans le social et le sociologue doit adopter une démarche scientifique, ce qui suppose que les faits sociaux seraient des choses.

La science du social étant justement sur le social, le sociologue travaille sur ses semblable, sur un ensemble dont il fait lui même parti, d'où la difficulté à être objectif par rapport à un scientifique classique travaillant sur de la matière\footnote{Bonne chance aux sociologues qui travaillent sur le vote FN}

La Science Politique est une science sociale car elle est une science qui étudie un certain type de faits sociaux: les faits politiques.

La Science Politique s'est développé en empruntant des concepts, des outils d'analyse aux différentes sciences sociales, à l'histoire, à l'économie, à l'anthropologie, à la sociologie (surtout à celle là): la Science Politique est une discipline pluridisciplinaire.

La Science Politique s'intéresse moins aux règles formelles du jeu politique mais se focalise sur le jeu politique lui même, donc aussi tout ce qui peut se passer en dehors des règles. Le droit constitutionnel et la science politique sont donc complémentaires. Il serait incohérent de ne pas chercher à comprendre les règles constitutionnelles pour comprendre le système politique. 

Inversement, le constitutionnaliste ne peut se passer du politique car il serai absurde de s'intéresser aux règles mais pas à leur mise en pratique.

	\section{Les objets de la science politique}



	\section{Le métier de politiste}

	\section{Pratiquer la science politique}

\end{document}
