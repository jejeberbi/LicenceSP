\documentclass[12pt, a4paper, openany]{book}

\usepackage[latin1]{inputenc}
\usepackage[T1]{fontenc}
\usepackage[francais]{babel}
\author{J�r�my B.}
\date{}
\title{Cours de Droit constit (UFR Amiens)}
\pagestyle{plain}

\begin{document}

\subsubsection{La fonction ex�cutive}

Le S�nat am�ricain donne un "avis conforme" pour la nomination des ambassadeurs, des ministres publics, les consuls, les juges � la cour supr�me etc. C'est donc le pr�sident qui nomme mais le S�nat doit donner son consentement. \\
La ratification des trait�s internationaux doivent aussi obtenir la confiance du S�nat, mais aux deux tiers des S�nateurs. \\
Le Parlement (le congr�s) a le pouvoir de d�clarer la guerre: Art. premier, section VIII.

\subsubsection{La fonction juridictionnelle}

L'essentiel de la fonction juridictionnelle appartient aux tribunaux et aux juges. Mais le pr�sident a le "pouvoir d'accorder des sursis et des gr�ces pour crimes contre les �tats contre les �tats-Unis" Art 2, section 2. \\
Le congr�s est juge �lectoral (Art premier, section 5) mais aussi juge du pr�sident, des juges f�d�raux, et des fonctionnaires dans le cadre de "l'impeachment".


\subsection{Les rapports inter-organiques}

\subsubsection{L'ind�pendance absolue du Congr�s}

Il n'existe pas de dissolution des chambres, ni de proc�dure de destitution individuelle par une autre institution f�d�rale. 



\end{document}
