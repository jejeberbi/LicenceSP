\documentclass[12pt, a4paper, openany]{book}

\usepackage[latin1]{inputenc}
\usepackage[T1]{fontenc}
\usepackage[francais]{babel}
\author{Jérémy B.}
\date{}
\title{Cours de droit civil (UFR Amiens)}
\pagestyle{plain}

\begin{document}
\maketitle

\section{Les différents textes légaux}

\subsection{La constitution (bloc constitutionnel)}

L'étude de la constitution relève du droit public. La constitution actuellement en vigueur est la constitution de la Ve république du 4 octobre 1958.

Si l'on parle de la constitution, on doit se référer au préambule: la partie la préliminaire de la constitution qui consiste en une proclamation solennelle de principes fondamentaux. 

Il y a quelque chose d'essentiel dans ce préambule: "le peuple Français proclame son attachement à la DDHC de 1789 confirmée et complétée par le préambule de la constitution de 1946".

La DDHC de 1789 et le préambule de 1946 ont donc valeur constitutionnel.

Bloc constitutionnel: 
\begin{itemize}
\item constitution de 1958
\item DDHC de 1789+préambule de 1946
\item les "principes fondamentaux reconnus par les lois de la république"\footnote{Par exemple, en 1971, le CC a rendu une décision: a considéré que le principe de la liberté d'association était fondamental et reconnu par les lois de république. Il a donc valeur constitutionnel.} (principes dégagés par le conseil constitutionnel quand il est saisi)
\item la charte de l'environnement de 2004. 
\end{itemize}

Le tout s'applique.

\subsection{Les textes internationaux}

Cela correspond aux traités, aux accords, aux conventions ratifiés.

La constitution de 1958 donne une place à ces textes. L'article 55 dispose: "Les traités ou accords régulièrement ratifiés ou approuvés ont, dès leur publication, une autorité supérieur à celle des Lois."

Puisque les textes internationaux sont devant la loi, on parlera de droit supranationale. 

Un texte international important est la CEDH\footnote{Convention de sauvegarde des droits de l'homme et des libertés fondamentales donc Convention Européenne des Droits de l'Homme} signée le 4 novembre 1950. Ce texte n'a rien à voir avec l'Union Européenne. Ce texte est important car proclame un certain nombre de droits comme à l'article 8 qui dispose: "Toutes personnes a droit au respect de sa vie privée et familiale, de son domicile et de sa correspondance".

C'est la CEDH, la Cour Européenne des Droits de l'Homme qui statue contre les États qui ne respecte pas la convention. 

S'agissant de l'UE, tout est parti du traité de Rome du 25 Mars 1957, qui a créé la CEE\footnote{Communauté Économique Européenne}, l'ancêtre de l'UE. Puis Maastricht du 7 février 1992, puis enfin le traité de Lisbonne du 13 décembre 2007. Ceci, avec deux corollaires, créer un grand marché avec deux principes, libre circulation des marchandises et des personnes (Schengen).

Il y a un certain nombre de texte légaux qui viennent de l'organisation Européenne. Il y a le traité de fonctionnement de l'UE (TFUE). À côté, il y a le traité sur l'UE (le TUE). À partir de ces traités, il est habituel de parlé des droits de l'UE primaire (droit issu des traités), ainsi que du droit de l'UE dérivé (droit qui découle des organes de l'UE).

Il y a dans les droits primaires et dérivés les règlements de l'UE ainsi que les directives. Les règlements sont des textes légaux Européens qui contiennent des règles de droit obligatoire et directement applicable dans les pays membres de l'UE. Article 288 du TFUE qui dispose: "Le règlement a une portée générale. Il est obligatoire dans tous ses éléments et il est directement applicable dans tout État membre."

Les règlements de l'UE proviennent soit du Conseil Européen soit du parlement Européen.

Les directives de l'UE sont, quant à elles, obligatoire mais elles ne sont pas directement applicable. Chaque État membre de l'UE est tenu de transposer dans ses textes de droit internes les objectifs assignés par la directive. Article 288 du TFUE qui dispose "La directive lie tout État membre destinataire quant aux résultats à atteindre tout en laissant aux instances nationales la compétence quant à la forme et aux moyens."

Si les règlements ne sont pas respectés, la CJUE peut condamner les États membres. Concernant les directives, il y a des objectifs à atteindre, souvent, il y a un calendrier de prévu. Les États membres peuvent aller dans les généralités prévus ou aller plus loin encore. On dit qu'un État transpose une directive dans son droit national.

Si le délai de la directive est écoulé et que la transposition n'a pas été faite en droit national, il y a condamnation.

Au fil du temps, plus il y a de règles légales qui émanent de l'UE. Les domaines gérés par l'UE sont essentiellement de l'ordre de l'économie, cela va de plus en plus loin, jusqu'à sortir parfois de ce domaine économique.

On parle parfois de bloc conventionnel. 

\subsection{Les Lois et les ordonnances}

La Loi est un texte légal voté par le parlement Français, composé de l'AN et du Sénat. Il y a les lois organiques, fixant dans le cadre de la constitution, les règles relatives aux pouvoirs publiques.

Les lois ordinaires, posent un premier problème: les parlementaires devaient-ils traiter toutes les questions d'ordre juridique ? Avec un système juridique romano-germanique (beaucoup d'écrit), il y aurai énormément de domaines sur lesquels légiférer. Il y a donc une liste dans la constitution des domaines dans lesquels les parlementaires peuvent légiférer. Les autres questions sont réglés par le gouvernement directement, par décret.

L'article 34 dispose donc: "La Loi fixe les règles concernant:
\begin{itemize}
\item La nationalité
\item La capacité des personnes
\item Les régimes matrimoniaux
\item Les successions
\item [...]
\item la détermination des crimes et délits et les peines qui leurs sont applicables
\item [...]
\end{itemize}
La loi détermine les principes fondamentaux du régime de la propriété[...]du droit du travail, du droit syndical et de la sécurité sociale etc..."

Le législtateur est l'autorité qui établit les Lois. En France, le législateur est le parlement.


Une ordonnance est un texte légal provisoire exceptionnellement pris par le gouvernement dans le domaine de l'article 34. L'ordonnance a valeur équivalente d'une loi ordinaire. L'ordonnance, quand elle est ratifiée par le parlement, devient une vraie Loi.

Le gouvernement doit être autorisé par le parlement pour prendre l'ordonnance. Mais le parlement aura le dernier mot quoiqu'il arrive car c'est à lui de ratifier ou non l'ordonnance. Celle-ci a une durée de vie programmé au terme duquel elle sera ratifiée ou non.

\subsection{Les règlements administratifs}


\subsection{Les circulaires administratives}




\end{document}
