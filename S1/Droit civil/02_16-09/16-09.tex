\documentclass[12pt, a4paper, openany]{book}

\usepackage[latin1]{inputenc}
\usepackage[T1]{fontenc}
\usepackage[francais]{babel}
\author{Jérémy B.}
\date{}
\title{Cours de Droit civil (UFR Amiens)}
\pagestyle{plain}

\begin{document}
\maketitle

\section{Caractéristique du droit intermédiaire}

\subsection{Laïcité}
Il y a dans le code canonique qui s'appliquait à tout le monde. Le nouveau principe de la laïcité venu avec la révolution est donc important. Toutes les règles du droit intermédiaire sont donc laïque, l'Église n'a donc pas à interférer dans les dispositions qui s'appliquent aux citoyens.

Le signe de cette évolution est le mariage à la mairie. Les églises qui tenaient donc les registres de l'état civil n'en sont plus responsable, cette responsabilité allant désormais aux mairies.

\subsection{Propriété}
Article 17 de la DDHC: "La propriété étant un droit inviolable et sacré, nul ne peut en être privé".

\subsection{Égalité}
Enfin, il y a égalité des citoyens entre eux, article 1 de la DDHC: "Les hommes naissent et demeurent libres et égaux en droit".


Ces principes rompent avec l'ancien régime. Les révolutionnaires ont passés beaucoup de temps à légiférer et batir un nouvel ordre juridique. D'où leur attachement à la loi dont l'objectif est de batir une société égalitaire.

\section{Le code civil}
De 1790 à 1799, il va y avoir quatre tentatives pour donner à la France un code civil. Ce qui montre qu'il n'était pas facile d'élaborer et faire passer un seul code civil.

La cinquième tentative sera la bonne parce que le premier consul Bonaparte\footnote{qui a pris le pouvoir le 9 novembre 1799} va charger une commission le 14 aout 1800 de préparer un code civil.

Cette commission est dirigée par Cambacérès, qui était un avocat ayant été ministre de la justice. Avec lui, il y a Tronchet, qui était président du tribunal de cassation (ancêtre de la cour de cassation), et sénateur de la Somme en 1801. Mais aussi Bigot de Préameneu, juge au tribunal de cassation. Maleville, juge également au tribunal de cassation. Le dernier est Portalis. 

Ces cinq juristes vont, en quatre mois, rédiger le code civil. Or, le code civil ne va être promulgué qu'en 1804.

Il y a un discours préliminaire au code civil que l'on attribue à Portalis. Ce discours permet d'expliquer ce qu'est le code civil.

Le projet de code civil a été envoyé à tous les magistrats de France, afin d'être commenté et critiqué afin d'amélioré encore les choses. Ceci a pris un peu de temps.

Mais en fait, ce qui a pris le plus de temps est le fait que le projet a été présenté devant plusieurs assemblées et devant la difficulté législative, le projet a trainé en longueur.

Quand Bonaparte est devenu Napoléon, qu'il est devenu empereur, il l'a fais passé en "tapant sur la table" en 1804.

De par ses nombreuses participations aux séances de travail de rédaction du code civil, il déclarera que c'est l'oeuvre de sa vie.

C'est donc grâce à la volonté politique de Napoléon que le code civil sera promulgué le 21 Mars 1804.

Il faut savoir que le code civil a été découpé en 36 lois qui ont été fusionnés. Au moment de sa promulgation, le code civil a 2281 articles. Il est entendu que toutes les anciennes dispositions datant soit de l'ancien régime soit du droit intermédiaire sont abrogés.

En soumettant tous les Français aux mêmes règles, Napoléon crée l'unité, ce qui, pour lui, a une forte valeur politique.

\section{Quels sont les idées directrices du code civil ?}

Ce qui est important dans le code civil, c'est bien sûr l'uniformisation des règles. 

Dans la grande diversité des règles avant le code civil, il y a eu une prise des meilleurs textes qui existaient. Le code civil n'a pas été construit à partir de rien. En cela, les rédacteurs du code civil ont expliqués que le code était la réunion de toutes les règles ayant pré existés.

S'agissant d'un domaine particulier, le droit de la famille, qui dans l'ancien droit était régit par l'église, dans le code civil, il devient un modèle de société. Ce modèle de société est basé sur le mariage. Mariage conçu comme patriarcal avec le mari comme maitre et seigneur. La femme était juridiquement considéré comme incapable (le mari devait donner son accord sur tout).

Le code civil est d'inspiration bourgeoise et libéral (l'individuel étant mis en avant sur le collectif). Art 544 du code civil:\footnote{Créé en 1804, toujours en vigueur} "La propriété est le droit de jouir et de disposer des choses de la façon la plus absolue"

\section{Quel est le destin du code de 1804 ?}

\subsection{Qu'est-devenu le code civil de 1804 par rapport au code civil d'aujourd'hui ?}

La table des matières est le plan du code civil.

Le code civil de 1804 est composé de quatres parties, le titre préliminaire et trois livres. Le titre préliminaire en 1804 est "De la publication, des effets et de l'application des lois en général", le libre Ier : "Des personnes", livre IIème: "Des biens et des différentes modifications de la propriété", livre IIIème: "Des différentes manières dont on acquiert la propriété". Le tout en 2281 articles.

Aujourd'hui, il y a toujours un titre préliminaire, mais 5 livres. Le titre préliminaire est le même qu'en 1804. Les trois premiers livres du code version 2015 sont les mêmes qu'en 1804. Les deux nouveaux livres sont: Livre IVème\footnote{Apparu en 2006}: "Des suretés" ; Livre Vème\footnote{2002-2006}: "Dispositions relatives à Mayotte". Le code civil de 2015 comporte 2534 articles. 

On pourrai se dire que les deux tiers n'ont pas changés, donc que le code civil de Napoléon s'appliquerai encore. Ceci est à moitié juste. Ou à moitié faux...

En effet beaucoup de choses ont changés comme si, dans un batiment ancien, on gardait la façade puis on changeait l'intérieur. Le droit des personnes et le droit de la famille ont profondément changés, notamment le côté patriarcal du droit ou encore le mariage pour tous, le PACS...

En revanche, il y a des dispositions qui sont toujours les mêmes, qui n'ont pas changés depuis 1804. Par exemple, environ 80\% des textes concernant les biens n'ont jamais changés, ce qui montre la grande qualité du code civil écrit en 1804.

Article 1382 qui s'applique toujours: "Tout fait quelquonque de l'homme qui cause à autrui un dommage oblige celui de la faute... à le réparer" Cet article a franchi le temps. Un code bien fait permet aux juges d'appliquer des principes anciens à la réalité du monde moderne. Cet article est par exemple utilisé dans les accidents de voiture. 


\subsection{En sortant des frontières, est-ce que le code civil de Napoléon a influencé d'autres pays ?}

Il faut distinguer deux périodes sur le destin internationnal du code civil: une période qui est l'age d'or du code civil (au 19e siècle), à cet age d'or a succédé un déclin au 20e siècle.

Le succès du code civil résulte d'abord du fait des victoires militaires de Napoléon. Il appliquait le code civil aux pays conquis. Napoléon a conquis la Belgique, le Luxembourg et certains cantons Suisses. Quand Napoléon est défait, ces pays vont se débarrasser du code civil. 

En revanche, en dehors des conquêtes militaires, le code civil est un exemple dans le monde entier, le Brésil (en 1831) le prendra en exemple comme le Chili (en 1835), le Quebec (en 1870) ou encore la Roumanie (en 1865).

Pourtant, au 20ème siècle va suivre une période de déclin car à côté du code civil Français, il va y avoir le code civil Allemand (le BGB) qui lui aussi va unifier une nation (en 1900). Celui ci sera donc aussi une référence dans d'autres pays du monde. Le Brésil va adopter le code civil allemand en 1916 comme le Mexique en 1928 ainsi que le Japon.

Il y a aussi le code civil Suisse de 1912 qui va influencer certains pays.

Au 20ème siècle, on voit apparaitre deux grands systèmes juridiques, des pays de droit écrit. Ce sont essentiellement les pays du vieux continent mais surtout les pays de l'Europe du Sud. Ces pays s'opposent au monde Anglo-Saxon qui sont des pays de la common-law. 

Il s'oppose au système romano-germanique car il y a pas ou peu de code, pas ou peu de lois. Tout repose sur les décisions des juges qui palient aux manques de la loi. Ils appliquent la règle du précédent. 

Le code civil va pérycliter car les pays anglo-saxons vont se developper et donc imposer le système de la common-law.

À l'occasion du bi-centenaire du code civil, un bilan va pouvoir être dressé. Le code civil est une oeuvre de cohérence et donc regardé à l'étranger quand il n'y a pas de solutions dans le droit nationnal. Le code civil est comparable à la langue Française. 

\chapter{Les règles de droit objectif}

\section{Droit objectif}

On définit le droit objectif comme l'ensemble de toutes les règles juridiques applicables. Si on le définit en France, c'est l'ensemble des règles applicables en France.

Quand on parle de droit positif, on sous entend droit objectif positif. C'est l'ensemble des règles juridiques applicables à un moment donné. 

Quand on parle du droit positif Français, on parle du droit objectif qui s'applique maintenant en France.

\section{Règle de droit}

\subsection{Caractère de la règle de droit}

Art 212 du code civil: "Les époux se doivent mutuellement respect, fidélité, secours, assistance". 

L'article 212 a un caractère général et impersonnel, il s'applique à tous les époux sans en désigner un en particulier. 

La loi utilise le verbe "devoir", la loi a donc un caractère d'obligation. Parfois, ce caractère découle de l'interdiction d'empêcher une possibilité offerte par la loi. 

Art 218: "Un époux peut donner mandat à l'autre de le représenter dans l'exercice des pouvoirs que le régime matrimonial lui attribue". Cette loi a un caractère obligatoire tout de même. Elle impose de laisser la possibilité de. On ne peut pas empêcher que cette possibilité existe.

Caractère coercitif: il y a un aspect de contrainte. Ce qui caractérise la règle de droit, c'est qu'elle est coercitive. Si les règles ne sont pas respectés, il y a dommages et intérêts. En pénal, la contrainte se traduit par l'emprisonnement, les amendes... Tout cela aboutit à ce que la règle de droit soit une contrainte et que cette contrainte mette en jeu l'état.

\subsection{Distinction entre règles de droit et règles de vie (dont règles morales)}

Le but de la règle morale tend au perfectionnement intérieur de l'homme et à l'épanouissement de sa conscience.

Le but de la règle de droit est le maintient de l'ordre social, éviter l'anarchie. 

Lorsque l'on parle de règle morale, sa violation n'est pas sanctionné par l'État. La sanction est intérieur à la personne. Il y a cependant parfois quelques passerelles, des cas où la règle de droit fait appel à la morale. Par exemple, l'article 1133 dit que si un contrat ne respecte pas les règle de bonne moeurs, ce contrat devra être annulé. 

Dans le cas d'une règle religieuse, la finalité est le salut de ses fidèles en relation avec Dieu (un ou plusieurs selon la religion). La règle de droit est d'expression laïque comme nous l'avons déjà expliqués. Dans le système Français, ces deux règles sont séparés par principe. Les deux sont donc indépendantes.

Il est possible de se marier à la fois non religieusement et religieusement, indépendamment l'un de l'autre. Si il y a mariage religieux (qui est facultatif car laissé à la liberté de conscience de chacun) mais intervient forcément après le mariage non religieux. Le "ministère du culte" ne peut marier que si l'on lui montre l'acte d'état civil de mariage. Si il ne le fait pas, il encoure une sanction pénale.

L'ordre civil et l'ordre moral sont bien séparés et fonctionnent plus ou moins harmonieusement chacun dans sa sphère, dans un respect mutuel.

Il existe cependant des pays où il n'existe pas de frontières entre règles religieuses et règles civil.

\section{Les sources directes des règles de droits objectif}

\subsection{Sources principales}

\subsection{Sources secondaires}

\section{Les sources indirectes des règles de droits objectif}

\part{Les règles de droits subjectif}


\end{document}
