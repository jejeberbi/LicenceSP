\subsection{L'empire temporel du texte légal}

Lorsqu'une loi nouvelle remplace une loi ancienne, il y a une zone d'empiétement. On ne sait donc pas si c'est la loi ancienne ou nouvelle qui s'applique. Il peut ne pas avoir y avoir cette zone.

\subsubsection{Si il n'y a pas empiétement}

"La loi ne dispose que pour l'avenir" \\ 
À travers l'article 1er in limine, la Loi nouvelle intervient sur un domaine où il n'y a pas eu de Lois antérieures. C'est une nouveauté législative. Il n'y a donc aucun problème, on applique la nouvelle loi dans l'avenir.


Si une Loi nouvelle succède à une Loi ancienne, la Loi ancienne s'arrête au moment où la Loi nouvelle entre en vigueur. Ces deux Lois traitent bien sûr du même sujet. Ici, on parle d'une réforme ou d'une modification administrative. \\
Pour que la loi ancienne cesse d'exister, celle-ci doit être abrogée. Le législateur, dans la Loi nouvelle, déclare, de façon expresse, que la Loi ancienne est abrogée. \\
L'abrogation peut être tacite. Dans ce cas là, le législateur n'a pas bien fait son travail. On va donc considérer que la nouvelle Loi abroge de façon tacite l'ancienne Loi si les dispositions de la nouvelle Loi vont dans un sens complètement différend que dans la Loi ancienne. \\
Si, d'aventures, une Loi ne serai plus appliquée, et pourtant il n'y as pas d'abrogation ; on dit que cette loi est tombée en désuétude. Même si une loi est tombée en désuétude, tant qu'elle n'est pas abrogée de façon expresse ou tacite, elle peut être appliquée. \\
La Loi nouvelle ne dispose que pour l'avenir, la Loi ancienne ne disposait que pour le passé.


"[La loi] n'a point d'effet rétroactif" \\
Si l'on admet de façon générale la rétroactivité, nous serions dans une sorte d'insécurité permanente. Car ce que nous faisons en fonction des lois en vigueur pourrait être remis en question par une loi future que nous ne connaissions pas au moment des faits.


Si la loi nouvelle ne relève pas du droit pénal, nous verrons d'abord le droit civil puis le droit fiscal. Dans le code civil, l'article 2 dispose que la Loi n'a pas d'effet rétroactif. Cependant, aucune convention n'interdit cette rétroactivité. Au niveau constitutionnel, le Conseil Constitutionnel a déclaré qu'on ne trouvait pas cette règle de non rétroactivité dans le bloc constitutionnel (décision du 7 Novembre 1997 #97-391): "considérant que le principe de non-rétroactivité des Lois n'a valeur constitutionnel[...]qu'en matière répressive\footnote{Relevant du droit pénal}." \\
C'est un raisonnement "à contrario", par lequel on déduit "en sens contraire" une règle d'une autre règle posée. Ici, seulement en matière pénale la rétroactivité a valeur constitutionnel. La décision sera confirmé par la cour de Cassation à la première chambre civil (Civ 1ère, 20 Juin 2000 ; D.200,p699) "En matière civile, le législateur n'est pas lié par le principe de non-rétroactivité des lois" ; c'est à dire que l'article 2 du code civile n'est pas confirmé par un texte supérieur et donc ne s'impose pas au législateur. \\
Même si la théorie l'autorise, en pratique, la loi rétroactive est très rare car elle créerai une grande insécurité juridique, un grand désordre. Pour qu'une loi civile puisse être rétroactive, il faut que le législateur indique expressément que la loi est rétroactive.


S'agissant des lois fiscales, l'enjeu est de savoir si une loi peut, de façon rétroactive supprimé un avantage fiscale ou imposer une nouvelle taxe. Ici, le Conseil Constitutionnel a eu à se prononcé le 18 Décembre 1998 (décision 98-404): "considérant que le principe de non-rétroactivité des loi n'a valeur constitutionnel qu'en matière répressive, que néanmoins, si le législateur a la faculté d'adopter des dispositions fiscales rétroactives, il ne peut le faire qu'en considération d'un motif d'intérêt général suffisant...". \\ 
Dans cette décision, une partie de la loi de financement de la sécurité sociale de 1998 avait été censuré partiellement car une disposition ne comportait pas un motif d'intérêt général suffisant.


Si la loi nouvelle relève du droit pénal, on s'intéresse au code pénal. Code qui dédie un article, le 112-1 à la rétroactivité qui dispose "Sont seuls punissables les faits constitutifs d'une infraction à la date à laquelle ils ont été commis." ; "Peuvent seuls être prononcées les peines légalement applicable à la même date." ; "Toutefois, les dispositions nouvelles s'appliquant aux infractions commises avant leur entrée en vigueur et n'ayant pas donné lieux à une condamnation passé en force de chose jugée lorsqu'elles sont moins sévère que les dispositions anciennes" \\
Il y a donc un principe de non rétroactivité, mais il y a une exception sur l'alinéa 3: rétroactivité de la loi pénale plus douce.


Cette non rétroactivité de la loi pénale, à la différence de la non rétroactivité de la loi civile, apparaît au niveau des conventions mais aussi de la constitution. \\
Au niveau conventionnel, il faut se reporter à l'article 7, alinéa premier de la CEDH. Au niveau constitutionnel, il y a l'article 8 de la DDHC qui dispose "Nul ne peut être puni qu'en vertu d'une loi établi et promulguée antérieurement au délit". \\ 
On en déduit que contrairement au droit civil où la règle de non rétroactivité pouvait être dérogé, ce n'est ici pas le cas. 


Définition légale des infractions que sont les crimes et délits, la détermination des crimes et délits ainsi que leurs peines applicables relèvent de la loi. Contravention (tribunal de police) ; délit (tribunal correctionnel) ; crime (cour d'assises), les crimes et délits relèvent de la loi. L'alinéa 1 définit donc les faits, et l'alinéa 2 la légalité des peines. \\


La rétroactivité par exception de la loi pénale plus douce: si un fait a été commis à un moment et si la procédure pénale dure, que l'affaire n'est pas définitivement jugée (elle n'a pas force de choses jugées: un jugement a force de choses jugées quand il n'est plus susceptible d'aucuns recours en droit interne), et qu'une nouvelle loi remplace l'ancienne en réduisant la peine ; on considère dans ce cas là en application de l'article 112-1 que la Loi va s'appliquer de manière rétroactive. \\
C'est la rétroactivité "in mitius". Cette rétroactivité est consacrée dans les conventions internationales mais aussi dans la constitution. Au niveau conventionnel, ceci est consacré dans le pacte international de New York de l'ONU le 19 Décembre 1966 qui le prévoit dans son article 15 in fine. Au niveau constitutionnel, il faut se référer à une décision du 20 Janvier 1981, la n°80-127: "Sont contraires à la Constitution les dispositions d'une Loi tendant à limiter les effets de la rétroactivité [in mitius], en effet, le fait de ne pas appliquer aux infractions commises sous l'empire de la loi ancienne la loi pénale nouvelle, plus douce, revient à permettre aux juges de prononcer les peines prévues par la loi ancienne et qui, selon l'appréciation même du législateur, ne sont plus nécessaires."


Exception générale au principe de non rétroactivité: rétroactivité de la loi interprétative, la loi interprétative est toujours considéré comme rétroactive, et rétroactive à la loi qu'elle interprète. \\
Une loi interprétative est, selon un arrêt de la Civ. 3eme du 27 Février 2002, Bull.civ.III, n°53: "Une loi ne peut être considérée comme interprétative qu'autant qu'elle se borne à reconnaître sans rien innover un état de droit pré-existant qu'une définition imparfaite a rendu susceptible de controverse". La loi interprétative ne fait qu'expliquer, définir des points obscurs de la loi qu'elle interprète (donc forcément postérieur à elle même). Si la loi "interprétative" crée de nouvelles notions, c'est une nouvelle loi qui modifie la loi ancienne. Elle n'ajoute donc rien, elle ne fait qu'interpréter. \\
Celle-ci sera donc rétroactive car elle fait corps avec la loi qu'elle interprète. Ce sont les juges qui jugent si la loi considérée comme interprétative est vraiment interprétative et donc si elle est rétroactive.

