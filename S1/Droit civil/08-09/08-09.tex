\chapter{Introduction au droit civil}

L'idée que le/la juriste a une grande mémoire est fausse. Le code civil est conséquent. Le raisonnement, l'argumentation, et la rigueur sont les qualités d'un juriste. \newline

\part{Le droit civil}

/* Définir le droit */

Gérard Cornu a été doyen (le plus agé, ou celui qui dirige), disait dans "Vocabulaire Juridique" : "Partie fondamentale du droit privé comprenant les règles relatives aux personnes, aux biens, à la famille, aux obligations, et plus spécialement aux divers contrats... /* trouver fin de citation */ ". \newline

Le droit civil n'est qu'une partie du droit privé. Le droit commercial, le droit des sociétés font partie du droit privé mais ne font pas partie du droit civil par exemple. \newline

Le droit à l'image, le mariage, le PACS... relèvent du droit des personnes et donc du droit civil. \newline

Mr Cornu nous parle aussi du droit des biens : mobilier et immobilier ; qui relève du droit civil lui aussi. \newline

Le doyen Cornu parle aussi des contrats, qui est un accord de volonté entre deux personnes (au moins) qui par cet accord convienne de réaliser une opération quelquonque (~par exemple, contrat de vente entre un acheteur et un vendeur~).\newline

Enfin, les obligations font partie du droit civil. En droit civil, on parle d'obligation quand une personne donnée (~un débiteur~) doit de l'argent à une personne (~le créancier~). Exemple : une personne faisant du vélo ne voit pas un piéton qu'il renverse. Ce piéton finit blessé. Dans ce cas là, en droit français, le cycliste est responsable de l'accident qu'il a commis à l'égard de ce piéton, ce cycliste est resonsable du dommage causé. Celui-ci doit donc indemnisé ce piéton. Il y est tenu et donc obligé. Cycliste~: débiteur ; piéton~: créancier. \footnote{Il est à noter qu'il y a des cas désagravants~: piéton suicidaire par exemple, fort vent donc plus le contrôle du véhicule dans ce cas ci} \newline

Les assurances dites "responsabilité civile" indemnisera à ma place les dommages que je cause.

"Le droit civil est donc une partie fondamental du droit privé" Cornu. \newline

\begin{itemize}
\item Droit privé~: le droit privé s'oppose au droit public. On parle de droit privé dès l'instant où l'on s'intérresse aux règles qui concernent le rapport des individus entre eux. Exemple~: les règles qui concernent les rapports entre un propriétaire d'un appartement et un locataire sont des règles de droit privé.\footnote{Contrat entre le locataire et le propriétaire}.
\item Droit public~: On s'intérresse aux rapports entre les citoyens et l'État, les citoyens et les collectivités locales. Mais aussi entre l'État et les collectivités.\footnote{Les réformes régionales de 2015 sont du droit public}\footnote{Le droit constitutionnel est du droit public}
\end{itemize}



\part{Le code civil}

L'essentiel du droit civil est dans le code civil. Le fait qu'en droit Français, l'essentiel du droit civil soit dans le code civil est une spécificité française. Qui dit code dit organisation rationnelle et pédagogique des matières traitées dans ce code. Si il n'y a pas de code, il n'y a pas cette organisation rationnelle et donc des magistrats faisant jurisprudence.\footnote{qui dit code dit à priori facilité} \newline

Derrière le code civil, il y a Napoléon qui aurait dis "C'est l'oeuvre de ma vie". En effet, il a eu la volonté politique qu'il y ait en France un code civil depuis 1804. \newline

\section{Élaboration du code civil}

État des lieux avant 1804~: Il faut distinguer deux périodes historiques, avant la révolution française, puis de la révolution jusqu'à 1804. Avant la révolution~: ancien droit français. De la révolution jusqu'à 1804~: droit intermédiaire. \newline

Ce qui caractérise l'ancien droit français concernant les règles du droit civil, c'est la diversité des règles concernant leur origine. Diversité entraine divisions. Le droit est un éclatement.

Dans l'ancien droit français, il y avait des provinces qui relevait du droit écrit. 

Ces provinces était au sud de la France, le midi, où l'on parlait la langue d'Oc, leurs règles de droit civil qui s'y appliquaient étaient donc des règles qui avaient été mises par écrit et influencés par le droit romain.\footnote{Ces règles sont écrites en latin}

Derrière ces textes, il y avait l'idée que les textes romains avaient une rationalité supérieure. 

Au nord et au centre de la France, les pays aux langues d'Oil ont un droit civil dit coutumier. Un droit qui correspond à une pratique répétée dans le temps (~la coutume\footnote{La coutume est une loi qui s'impose d'elle même par sa répétion, personne n'est là pour la faire appliqué} donc~). Ces coutumes se transmettent oralement.\footnote{Par nécessité pratique, ces règles orales se sont retrouvés à l'écrit. La coutume a été rapporté, non rationalisé~: c'est un coutumier.}

Les règles écrites s'opposent donc aux règles orales.

On distingue des coutumes principales et des coutumes secondaires. Voltaire, avant le début de la révolution avait dénombré près de 60 coutumes principales et de 300 à 700 coutumes locales. Sur le même sujet, il pouvait y avoir plusieurs réponses possibles, radicalement différentes. Voltaire~: "On change de lois\footnote{coutume} en voyageant aussi souvent que de chevaux".

Il y a aussi des règles de droit civil qui viennent de l'église. Ce sont des règles venant du "droit canoniques".\footnote{il existe un code canonique qui possède des canons, et non des articles} L'église régit donc, à l'époque, un certain nombre de domaines, comme le mariage.

Encore une dernière source de droit~: le droit royal, qui provient des ordonances royales, mais de manière assez minimes.

Bilan~: À cette époque c'est le bordel~:
\begin{itemize}
\item Lois locales~: orales/coutumes au centre et au nord de la France, écrites au sud
\item Lois nationales~: Église ou lois royales
\end{itemize}

L'exploit de Napoléon est donc de produire une seule source de lois, ce que la révolution Française n'as jamais réussi à faire.

Il est à noter que dans certains pays de par le monde, les lois coutumières existent encore.







