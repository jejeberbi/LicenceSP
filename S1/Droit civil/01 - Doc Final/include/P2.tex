\part{Les sources indirectes de droit objectif}

\chapter{La jurisprudence}
\section{Définitions possibles de la jurisprudence}
Le terme jurisprudence est susceptible d'avoir deux sens, un sens large et un précis. \\
Dans un sens large, nous dit Cornu dans son vocabulaire juridique "La jurisprudence, c'est l'ensemble des décisions de justice". \\
Dans un sens plus précis, toujours en citant Cornu: "la tendance habituelle d'une juridiction déterminée ou d'une catégorie de juridictions à juger dans tel sens (plutôt que dans tel autre)". Par exemple, dans un cas précis, il faudrait que l'avocat dise à son client "habituellement, tel tribunal juge dans tel sens". Dans un sens étroit, on va chercher des décisions qui illustrent un point de vue important de l'application d'un texte.

\section{Jurisprudence et décisions de justice considérés isolément}

\subsection{Une décision de justice est rendue par un magistrat disant le droit}

On utilise l'expression "dire le droit" car le rôle de juger implique nécessairement pour le magistrat d'appliquer de façon concrète et particulière au cas qui lui est soumis les conditions légales qui elles, sont abstraites et générales. \\
Ainsi, si le juge est saisi d'une action en divorce pour faute par un conjoint. Le juge, après avoir examiné les faits, devra dire si les conditions légales applicables au divorce pour fautes sont remplies concrètement. Voir art 242 du code civil. \\
Le juge dit le droit, c'est à dire qu'il met en oeuvre le droit, il l'applique.

\subsubsection{Les deux obligations du magistrat disant le droit: juger et juger en droit}

Interdiction du déni de justice: si l'on dit que le juge a une obligation de juger dans les litiges qui lui sont soumis, c'est qu'il ne peut commettre un déni de justice. Au terme de l'article 4 du Code Civil,  "Le juge qui refusera de juger (...) pourra être poursuivi comme coupable de déni de justice". Le juge doit donc juger, c'est son obligation. \\
Le déni de justice est un délit pénal et passible du tribunal correctionnel. Ceci est affirmé dans l'article 434-7-1 du Code Pénal qui dispose "Le fait par un magistrat (...) de dénier de rendre la justice après en avoir été requis et de persévérer dans son déni après avertissement ou injonction de ses supérieurs est puni de 7500€ d'amende et de l'interdiction de l'exercice des fonctions publiques pour une durée de 5 à 20 ans." \\
Sur le plan pénal, c'est le juge qui doit refuser de juger, or sur le plan civil, la solution donnée par le juge peut être apparentée à un déni de justice.


Décision de la C.Cas, Civ 2ème du 21 Janvier 1993: "Le juge ne peut refuser de statuer en se fondant sur l'insuffisance des preuves qui lui sont fournies par les parties." \\
Pour refuser de rendre un jugement, le juge n'a pas le droit d'utiliser pour motif l'insuffisance ou le silence de la loi (il n'existe pas de texte légal), ou l'obscurité d'un texte (la loi n'est pas claire). \\
Or dans l'affaire cité ci-dessus, le juge devait faire justice en se basant sur les preuves de chacune des parties, celui-ci a statué qu'il ne pouvait pas rendre un jugement car les preuves fournies étaient insuffisantes. La C.Cas va donc statué que implicitement, le juge a commis un déni de justice, le juge doit rendre un jugement avec les éléments, même si ils sont insuffisants. 


La Civ.3ème a rendu une décision le 16 Avril 1970. Dans cette décision, la C.Cas a considérée qu'il y avait déni de justice d'un juge dans un procès en revendication de la propriété d'un immeuble entre deux parties. Un procès en revendication est un procès dans lequel un propriétaire dépossédé intente une action pour retrouver sa propriété. \\ 
Le juge, dans son jugement, avait reconnu que l'immeuble appartenait nécessairement à l'un ou à l'autre ; il avait considéré qu'il ne pouvait pas juger car aucune des deux parties n'avait prouvé sa supériorité de son droit.


En France, le juge est soumis au droit et non à l'équité. \\
L'alinéa premier de l'article 12 du code de procédure civil dispose "Le juge doit trancher le litige conformément aux règles de droits qui lui sont applicables." Le juge doit donc se soumettre au droit. \\
Juge, justice et équité semble corrélé, cependant, dans la tradition du Code Civil, nous avons écarté l'équité. \\
L'équité est défini pour les juriste par Cornu comme "manière de résoudre les litiges en dehors des règles de droits selon des critères tel que la raison, l'utilité, l'amour de la paix, la morale...". Ce qui est gênant dans l'équité, c'est que ce n'est pas objectivement universel, on serai donc subjectif et le juge devrai être subjectif. Or la règle de droit est une règle impersonnel et générale et donc par définition objective. \\
Sous l'ancien droit (avant la révolution), l'équité jouait un rôle important au point qu'un proverbe de 1626 disait "Dieu nous garde de l'équité des parlements !". Les parlements, dans l'ancienne France, était des cours supérieures de justice régionales. Dans celles-ci, les plaideurs avaient l'impression que les cours rendaient une justice de classe. L'équité invoqué pouvait conduire à l'inégalité et était donc profondément injuste: le juge faisait ce qu'il voulait. \\
La jurisprudence va dans le sens du respect du principe visant à écarter l'équité. La C.Cas dans sa Chambre Civile, le 6 Mars 1876\footnote{DP 1856, P 1193} relatif à l'article 1134 du Code Civil\footnote{Point 18 du Code Civil Dalloz}, sur la seconde affaire du canal de Craponne indique que "dans aucun cas, il n'appartient aux tribunaux quelque équitable que puisse paraître leur décision, de prendre en considération le temps et les circonstances pour modifier les conventions des parties". \\ 
Cette jurisprudence a été renouvelé dans une décision du 23 janvier 1948\footnote{JCP 1949, deuxième partie, n°4229} "En se bornant à trancher le différend en équité alors que le législateur n'a pas abandonné aux juges le soin de rechercher le prix équitable en dehors des règles qu'il a lui même fixé, il y a eu violation de la Loi."


La première exception est ce qu'on appelle l'amiable composition. L'amiable composition relève de l'article 1474 du Code de procédure civile. Cet article dispose "L'arbitre tranche le litige conformément aux règles de droit à moins que dans la convention d'arbitrage, les parties ne lui aient conférées mission de statués comme amiable compositeur." \\
L'amiable compositeur est, pour Cornu "un arbitre, auquel la convention d'arbitrage donne pour mission de trancher le litige en équité sans être tenu de suivre [...] les règles du droit". Dans certains cas, en matière de droit privé, les parties, au lieu d'aller en justice et de demander au juge de trancher leur litige, les parties demandent de faire appel à un tiers qui n'est pas un magistrat mais un arbitre. Les parties vont demander à cet arbitre de trancher leur différend. \\
Il faut que les deux parties soient d'accord pour faire appel à un arbitre. Les parties vont signer alors une convention d'arbitrage. Dans cette convention d'arbitrage, les parties vont exprimer leur accord pour respecter la décision prise par l'arbitre. Les parties peuvent décider que l'arbitre n'aura pas à suivre les règles de droit ou qu'il pourra juger en équité. L'arbitre est donc un aimable compositeur si il peut trancher le différend en se basant sur l'équité.


Un arbitrage est beaucoup plus rapide à prendre qu'un jugement classique, voilà pourquoi l'arbitrage est très utilisé dans les affaires internationales comme dans le domaine financier/commercial. 

\subsubsection{La nécessité pour le magistrat disant le droit d'interpréter les règles de droits}

Quand le juge interprète les règles de droit, il a, à son service des méthodes d'interprétation mais aussi des maximes d'interprétation.


On retiendra trois méthodes d'interprétation. \\
La méthode exégétique ou méthode littérale, consiste à dire que tout est dans la loi, tout est dans la lettre de la loi, rien que dans la loi. Ce qui revient à dire qu'il doit résoudre le problème qui lui est posé avec le texte de loi qu'il a. La bonne intention de cette méthode est que l'on veut que la loi soit mise sur un piédestal, le "culte" de la loi car c'est censé être la volonté du peuple et le juge doit respecter la volonté du peuple. Le mauvais côté est que, parfois, la loi ne s'applique pas facilement dans un cas précis. \\
La méthode historique ou évolutive, le point de départ de cette méthode est le même que précédemment: la lettre du texte. Mais cette lettre du texte, il ne va pas la suivre aveuglément, il va l'adapter aux réalités du moment. Aspect positif: le juge adapte le texte aux réalités ; aspect négatif: on donne le pouvoir aux juges de compléter voir de poser de nouvelles règles de lois et donc de devenir un législateur bis. \\
La méthode téléologique: teleos en Grec veut dire but. Dans cette méthode, on met en avant le but, l'objectif du législateur au travers de la loi qu'il a voté: à quel résultat le législateur voulait-il parvenir ? L'on arrive à déterminer les objectifs du législateur en travaillant sur les travaux parlementaires: il faut relire tout ce que a été écrit ou dit à l'AN ou au Sénat. C'est ce que l'on appelle rechercher l'esprit de la loi (au delà de la loi, il y a l'esprit). \\
Les magistrats vont d'abords s'attacher à la loi donc utiliser la première méthode, puis la deuxième quand la seule lecture de la loi ne suffit pas, puis enfin, la dernière méthode. \\
Attention, ces méthodes ne sont applicables qu'au droit privé et n'est pas applicable directement en droit pénal.


Il existe un grand nombre de maximes d'interprétations. On en retiendra deux à titre d'exemple \\
"Les exceptions sont d'interprétation stricte". On veut dire que lorsqu'on est face à un texte, soit le texte pose un principe, soit une exception. Dès l'instant où il y a exception, cette maxime nous dit que l'exception doit toujours être interprété restrictivement. \\
"Il n'y a pas lieu de distinguer là où la loi ne distingue pas". Dès l'instant où dans un principe, on ne pose pas de distinctions ou d'exceptions, le juge n'a pas lui même à restreindre le domaine du principe.

\subsection{Une décision de justice est rendue par un magistrat ne faisant pas le droit}

En droit Français, il est habituel de dire que les juges ne peuvent pas prononcer des arrêts de règlements. Le juge, quand il juge, ne doit pas se comporter comme une autorité qui édicterai des règles de droit ; le juge ne peut se comporter comme le législateur ou le gouvernement. Voir article 5 du C.Civ. Les juges doivent se cantonner aux éléments qu'ils ont dans leur dossier à traiter. 

\section{Jurisprudence et décisions de justice considérés dans un ensemble}

\subsection{La jurisprudence est une source indirecte du droit objectif}

Ici, l'on va parler essentiellement de la tendance d'une juridiction dans tel sens plutôt qu'un autre, on va se cantonner à la Cour de Cassation. On dit qu'un arrêt fait ou fera jurisprudence quand la doctrine qualifie cet arrêt d'arrêt de principes. Cela veut dire qu'il y a, quand on mesure la portée des arrêts de la Cour de Cassation, un arrêt de principe et un arrêt d'espèce. L'arrêt d'espèce est important pour les parties mais peu important pour le droit en lui même. \\
Généralement (ce n'est pas systématique), un arrêt de principes est donné en assemblée plénière ou en chambre mixte. Un arrêt de principe est souvent un arrêt de cassation. \\
Un arrêt qui fait jurisprudence est un arrêt dont la solution a été reconduite, non seulement par la C.Cas elle même mais aussi par les juridictions du fond.



\subsection{La jurisprudence est une source incertaine du droit objectif}

Contrairement à d'autres systèmes de par le monde, notamment le système anglo-saxon où il existe la règle du précédent, le système Français ne comporte pas cette règle. Alors même qu'une jurisprudence serait établi, rien n'empêche en théorie à ce qu'un TGI refuse d'appliquer la décision de la C.Cas. La C.Cas peut en effet changer d'avis et y avoir des revirements de jurisprudences. 

\chapter{La doctrine}

\section{La définition de la doctrine}

La doctrine peut se définir de façon différente: un sens large et un sens étroit. \\
Dans un sens large, la doctrine désigne, selon Beignier "la doctrine est l'ensemble des opinions émises par les juristes, magistrats, avocats, et surtout professeurs de droit que l'on appelle les auteurs". \\
Dans un sens plus étroit, la doctrine aura trait aux opinions de ces auteurs sur une question précise. (Ex: Doctrine relative aux méthodes d'interprétation de la règle de droit). \\
Ces auteurs rédigent des ouvrages, des manuels, des commentaires dans des revues. Les auteurs en doctrine aiment bien commenter les arrêts de la C.Cas.

\section{Le rôle de la doctrine}

La doctrine a un rôle double.

\subsection{Inspirer le juge}

Les auteurs de la doctrine peuvent montrer aux juges que l'interprétation qui a été retenu est une interprétation qui peut être contestée. En critiquant ces décisions, cela peut influencer les juges sur les jugements à venir. \\ 
Mais la doctrine aide la justice aussi en faisant des commentaires d'une loi nouvelle et aide donc à l'application de celle-ci. 

\subsection{Influencer le législateur}

Cette influence de la doctrine sur le législateur a aussi une double nature. \\
D'abord, la doctrine peut proposer aux législateurs de prendre de nouvelles lois ou d'améliorer des lois existantes. Sur certaines questions, on voit apparaître très nettement l'influence de membres de la doctrine. \\
La deuxième influence est dans le cadre des travaux préparatoires (soit de l'AN soit du Sénat), où la commission consulte des experts, parmi lesquels il y a des membres de la doctrine. Ces experts sont auditionnés et les parlementaires posent des questions très précises.


\part{Les droits subjectifs}

\chapter{La preuve des droits subjectifs}

\section{Les préalables à toutes les preuves}

\subsection{La charge de la preuve}

Dans un procès civil entre un demandeur et un défendeur, à qui incombe la charge de la preuve initiale (les auteurs parlent du fardeau de la preuve) ? \\
Si on se reporte au Code de procédure civil, il ne nous donne pas vraiment de réponses. En revanche, l'article 1315 §1er, du C.Civ nous donne des réponses: "Celui qui réclame l'exécution d'un obligation doit la prouver." C'est donc au demandeur d'apporter la preuve initiale de ce qu'il demande. C'est ensuite au défendeur d'apporter la preuve de sa non culpabilité, art. 1315 §2 "Réciproquement, celui qui se prétend libéré, doit justifier le payement ou le fait qui a produit l'extinction de son obligation".

\subsection{La recherche de la preuve}

La recherche de la preuve et des éléments relatifs à cette preuve sont placés sous l'autorité du juge. Le juge civil a des pouvoirs d'instruction (ne pas confondre le juge civil qui a des pouvoirs d'instruction avec le juge d'instruction qui est un juge pénal). \\
L'article 10 du code de procédure civil nous dit "Le juge a le pouvoir d'ordonner d'office toutes les mesures d'instruction légalement admissible, son pouvoir d'instruction s'étend d'abord aux partis au procès. Mais ce pouvoir s'étend aussi aux tiers." S'agissant des pouvoirs d'instruction du juge, il faut se reporter à l'article 11, §1 "Les parties sont tenus d'apporter leurs concours aux mesures d'instruction sauf au juge attiré de toutes conséquences d'une abstention ou d'un refus". Si le défendeur ne donne pas son concours à l'instruction, le juge pourra en tirer des conséquences et condamner sur les seuls arguments du demandeur. \\
La recherche des preuves obéit à des règles légales strictes et en particulier quand est en jeu la vie privée. La recherche est autant à l'initiative du juge que des parties. 


\section{La preuve selon la nature du droit subjectif envisagé}

On distingue les droits subjectifs qui découlent d'un fait juridique ainsi que d'un acte juridique. 

\subsection{La preuve d'un acte juridique}

\subsubsection{L'acte juridique}

L'acte juridique, à la différence d'un fait juridique, est une manifestation de volonté créant des effets juridiques recherchés. Ainsi, un contrat de vente est la manifestation de deux volontés, la volonté d'un vendeur de vendre ainsi que la volonté d'un acheteur d'acheté. Le vendeur recherche le fait d'acquérir un prix et l'acheteur d'acquérir la propriété d'un bien. \\
Un acte juridique est le contrat (voir article 1101). \\
S'agissant des actes juridiques, le C.Civ a prévu des modalités de la preuve, ce sont des preuves nommés. Les preuves sont nommés par le code civil. Une preuve par écrit est une preuve nommé.

\subsubsection{Le règne de la preuve par écrit}

Quand on parle de la preuve par écrit, on doit tout de suite mettre en avant deux preuves par écrit les plus importantes: les actes authentiques et les actes sous seing privé.


Acte authentique: on va s'intéressé essentiellement aux actes notariés. L'article 1317 définit l'acte authentique: "L'acte authentique est celui qui a été reçu par officiers publics ayant le droit d'instrumenter dans le lieu où l'acte a été rédigé, et avec les solennités requises." Il faut que l'acte (premier sens: instrumentaire, instrumentum) soit à l'écrit car il constate de l'opération envisagé. Cet acte doit répondre à trois conditions: acte qui ait reçu (dressé) par un officier public (notaire, huissier, officier de l'état civil etc.). \\ 
L'acte notarié doit être reçu par un notaire ayant la compétence territoriale pour réaliser cet acte, c'est se demander dans quelle limite géographique le notaire peut exercer, voir article 8 à 10 du décret modifié du 26 Novembre 1971. Pour certaines opérations, les notaires ont compétence générales en France Métropolitaine mais pour d'autres opérations, leur compétence territoriale est donnée par la limite de la Cour d'Appel dans laquelle ils sont. \\
La troisième condition est que l'acte doit être accompagné des solennités requises: ce sont des formalités spéciales, particulièrement importante. Dans le cas d'un acte notarié, la formalité est qu'il faut un notaire de présent en personne, il signera lui même l'acte authentique. Les parties à cette acte doivent signer aussi. L'acte doit être rédigé en Français sur du papier timbré. Les ratures doivent être approuvées et donc paraphées etc. Il faut que toutes ces formalités soient réalisées pour que l'on puisse dire qu'un acte a été valablement reçu en la forme notariée. \\
L'acte peut être un contenant (instrument, instrumentum) ou un contenu ("l'affaire", le negotium). 


Nous allons mesuré l'importance attaché à un acte juridique notarié. Dans la mesure où le notaire vérifie l'identité, les capacités (juridiques) et les pouvoirs des intéressés et dans la mesure où le notaire signe aussi l'acte en qualité d'officier publique, il va conféré de la sorte à cet acte notarié un caractère authentique en s'engageant (le notaire) sur le contenu et la date de cet acte. On dira alors que cet acte authentique fait pleine foi et qu'il a force exécutoire. \\
Pour la pleine foi, il faut se reporter à l'article 1319 du Code Civil. Il y a pleine foi entre les parties contractantes eux même mais aussi entre leurs héritiers. La pleine foi est la pleine valeur de l'acte juridique. Il faut distinguer deux séries d'éléments dans l'acte, une série que le notaire doit vérifier, qui engage donc sa responsabilité et font pleine foi. Il y a quelques éléments sur lesquels le notaire n'a pas à les vérifier et qui donc, ne feront pas pleine foi. \\
L'acte authentique fait pleine foi jusqu'au déclenchement éventuel d'une action judiciaire en inscription de faux. Une telle action, c'est contester la contenu de l'acte authentique. C'est une procédure complexe rarement mise en oeuvre (voir article 303 à 316 du code de procédure civile). \\
L'alinéa 2 de l'article 1319 évoque l'exception de la pleine foi lorsqu'il y a mise en accusation. \\
Lorsque l'action initiale principale en inscription de faux intervient devant le TGI, en se reportant à la première partie de l'alinéa 2, "l'exécution de l'acte argué de faux sera suspendue". Première conséquence: le juge va, après vérification, conclure que l'acte authentique est un vrai acte, la suspension se termine et l'acte reprend force exécutoire. Deuxième conséquence: le juge conclut que l'acte est un faux, l'acte authentique devient totalement nul et perd donc sa force exécutoire. \\
Lorsque l'action en inscription de faux intervient incidemment (en cours d'instance, dans un procès engagé) devant la Cour d'Appel, on regarde la deuxième partie du deuxième alinéa de l'article 1319, "les tribunaux pourront suspendre provisoirement l'exécution de l'acte". Les deux conséquences possibles du jugement seront les mêmes que ci dessus. \\
Dans certains cas, le fait d'avoir aidé, en toutes connaissances de choses, à établir un acte authentique faux, il peut y avoir des poursuites pénales allant jusqu'à 10 ans de prison et 150 000€ d'amendes (Art 441-4 du code pénal). 


Lorsque le débiteur n'exécute pas ses obligations, l'acte notarié évite au créancier d'avoir à obtenir une décision de justice si il veut obtenir le paiement de sa dette. En revanche, lorsque l'acte est réalisé sous seing privé, il faut une décision de justice. Un acte authentique a donc force exécutoire automatiquement. 


Un acte sous seing privé original n'a pas de définition dans le Code Civil contrairement à l'acte authentique. De façon commune, un acte sous seing privé peut se définir comme un écrit établi par les parties elles mêmes sous leurs seules signatures donc sans l'intervention d'un officier public. \\
Ce qui est paradoxal, c'est qu'il n'y a pas de définition des actes sous seing privé mais le Code Civil pose des effets et des conditions. Pour qu'un acte fasse pleine foi, il faut se reporter à l'article 1325 (règle du double, triple etc.). Une convention synallagmatique est une convention qui contient des obligations réciproques, bilatérales. Cela s'oppose à une convention unilatérale.  Il faut autant d'originaux qu'il y a de parties ("d'intérêts distincts"). \\ 
La Cour de Cassation a admis une exception, non prévu par la loi, dans un arrêt rendu par la troisième Chambre Civile le 15 Avril 1992\footnote{Bulletin civil III N°131 ; Dalloz 1992 sommaire. p398} qui dit "La remise à un tiers d'un acte sous seing privé établi en un seul exemplaire constitue une exception au principe posé à l'alinéa premier de l'article 1325 du C.Civ". Le tiers est quelqu'un qui n'est pas partie à la convention. Concrètement, dans une convention, les parties ne décident de faire qu'un seul exemplaire qu'ils confient à un tiers, cette convention est valide et constitue donc une exception au principe posé par l'article 1325. \\
L'alinéa trois pose comme condition que le nombre d'originaux soit renseigné. Si il n'y a qu'un exemplaire, il faut le renseigner. Si il y a plusieurs exemplaires, il faudra indiquer le nombre d'exemplaires établis. L'alinéa 4 pose une exception: si la formalité visant à indiquer le nombre d'exemplaires n'a pas été effectuée, on ne peut pas utiliser ce défaut pour aller à l'encontre de la convention si on l'a déjà exécutée. On peut se demander si l'on parle d'exécution complète ou partielle: dans un arrêt rendu le 20 Octobre 1981 par la Civ 1ère\footnote{Bull. Civ. I, n°300} a considéré qu'une exécution même partielle de la convention permet d'écarter l'opposition dont pouvait se prévaloir celui qui a exécuté partiellement la convention. \\
Dans un engagement unilatéral, c'est une hypothèse particulière (article 1326). 


1. Pleine foi dans un acte de seing privé qui respecte les conditions \\
L'acte sous seing privé a entre ceux qui l'ont souscrit la même foi que l'acte authentique. Donc dès que cet acte est valablement établi, il a, comme l'acte authentique, pleine foi entre les parties. Comme pour l'acte authentique, on étend la règle de la pleine foi des contractants à leurs héritiers, l'acte a donc force probante. \\ 
La fragilité de l'acte sous seing privé est qu'il ne vaut pleine foi que si il est reconnu par celui auquel on l'oppose. Celui-ci peut donc désavouer sa signature/son écriture. \\
Si il avoue sa signature, cela veut dire qu'il reconnaît qu'il a bel et bien signé cet acte sous seing privé et le reconnaît comme valable et donc qu'il l'a bien accepté. \\
Si il désavoue sa signature, celui-ci ne doit certainement pas exécuter la convention et engagé une action en justice (art. 1324). C'est une action judiciaire en vérification d'écriture, cela peut aussi être une action en faux (art. 287 à 315 du code de procédure civile). Soit le tribunal fait droit à sa demande, c'est que le tribunal va considéré que l'acte sous seing privé que l'on a opposé à cette personne n'est pas l'acte qu'il a signé, voire même qu'il n'a jamais existé. En faisant droit à la demande, l'acte perd alors toute valeur, "c'est un faux". \\ 
Soit, le tribunal ne fait pas droit à la demande et va donc considéré que l'acte n'est pas un faux et donc la convention sera opposable mais une sanction est aussi prévue "Celui qui avait désavoué l'acte sous seing privé peut être condamné à une amende de 3000€" mais il y a aussi possibilité d'avoir des dommages et intérêts pour le créancier. \\
Dans les cas les plus graves, des poursuites pénales peuvent être engagés (tribunal correctionnel, c'est un délit).  


2. Force exécutoire ? \\
Alors qu'un acte authentique est immédiatement exécutoire, un acte sous seing privé suppose d'engager une action en justice si le débiteur ne souhaite pas payer. 

\subsubsection{L'écrit produit et la copie d'un acte authentique ou la copie d'un acte sous seing privé original}

Le titre original existe encore. \\
Dans cette hypothèse, il suffit de se reporter à l'article 1334 du Code Civil: "Les copies, lorsque le titre original subsiste, ne font foi que de ce qui est contenu au titre, dont la représentation peut toujours être exigée". La copie lorsque le titre original subsiste, existe encore, cette copie fera foie à valeur probante dès l'instant où l'on peut montrer, présenter l'original.


Le titre original n'existe plus. \\
Ici, on va se demander si il existe des règles de "secours". Il faut déjà noter qu'il n'existe pas de "règles de secours" concernant les actes sous seing privé dont les originaux auraient étés perdus. \\ 
En revanche, pour les actes notariés authentique, il y a une voie de secours prévu à l'article 1335 du Code Civil: "Les grosses ou premières expéditions font la même foi que l'original". "Grosse" et "les premières expéditions" sont des copies établis par le notaire lui même et certifié par le notaire comme étant conforme à l'original (original que l'on pourrait appeler "la minute"). Ces copies certifiées par le notaire comme conforme à l'original font donc foie. 

\subsubsection{Le domaine des actes authentiques et des actes sous seing privé}

L'article 1341 est la matière de ce que nous allons voir maintenant.


Les deux principes de l'article 1341. \\
Cet article appartient à une section qui s'appelle "De la preuve testimoniale" et le paradoxe est que le terme testimonial est un faux ami car on parle de preuve par écrit. La preuve évoquée dans l'article 1341 est par écrit: soit un acte authentique notarié, soit sous seing privé. Il faut regarder le décret modifié du 15 Juillet 1980 (1500€). \\
In limine de cet article, soit les intérêts en jeu sont supérieur à 1500 soit les intérêts sont inférieur ou égal à 1500€. Dès l'instant où les intérêts sont supérieur, la règle s'applique: il doit être réalisé un acte (authentique ou sous seing privé). \\
Si les intérêts en jeu sont inférieur ou égaux à 1500€: on applique la règle à contrario, il n'est pas obligatoire de réalisé un acte authentique ou sous seing privé. Si cela a tout de même été fait, les parties ne peuvent pas faire "marche arrière" à ce qui a été établi par un acte. Les parties sont donc tenus de faire ce qui a été établi et ne peuvent pas contester cet acte en considérant qu'elles n'étaient pas obligés de faire un acte. \\
L'article 1341 ne concerne que la matière civile et non la matière commerciale (deuxième alinéa). \\
Un texte impératif est un texte que les parties ne peuvent écarter. Un texte supplétif de volonté est un texte qui s'applique dès l'instant où les parties ne l'ont pas écarté. L'article 1341 a toujours été considéré comme non impératif, il est supplétif de volonté. Il s'applique quand les intéressés n'ont pas décidés de l'écarter. Une décision rendue par la Civ. 3ème le 16 Novembre 1977 qui dit: "Les dispositions de l'article 1341 quoi que n'étant pas d'ordre publique, s'impose aux juges dès que les parties n'ont pas expressément ou tacitement écarté cet article." Elle s'impose aux juges car elle est supplétive de volonté. Lorsque des parties dans un acte, veulent écarter l'article 1341, les parties peuvent déclarer ne pas appliquer cet article.


Il y a quatre exceptions à envisagé à l'article 1341. \\
Le premier groupe d'exceptions est donné par l'article 1347, le deuxième groupe à l'article 1348. \\
L'article 1347 dispose: "On appelle ainsi tout acte par écrit qui est émané de celui contre lequel la demande est formée, ou de celui qu'il représente, et qui rend vraisemblable le fait allégué." Il y a ici trois conditions cumulatives pour qu'il y ait commencement de preuve par écrit: il faut qu'il y ait tout autre écrit qu'un acte parfait, cet écrit doit émaner de celui contre lequel la demande est formée (le défendeur), il faut que le défendeur rende vraisemblable le fait allégué. Les juges du fond apprécient souverainement si un écrit rend vraisemblable le fait allégué. Leur appréciation échappe donc à la Cour de Cassation, c'est ce que dit une décision rendu par la Civ. 1ère la 1er décembre 1965\footnote{Bull. Civ. I N°670}. Les juges ont souveraineté d'appréciation quand la C.Cas ne peut pas contrôler la chose. Une simple lettre peut constituer un commencement de preuve par écrit mais elle doit respecter les précédentes conditions (doit émaner du défendeur et rendre vraisemblable le fait allégué). \\
Une photocopie peut constituer un commencement de preuve par écrit, cela découle d'une décision rendu par la Civ 1ère, le 14 Février 1995\footnote{Dalloz 1995, P340}. De même, un fax, une télécopie, peut constituer un commencement de preuve par écrit: chambre commerciale le 2 Décembre 1997\footnote{Dalloz 1998, P192}. 


L'article 1348 du C.Civ donne trois exceptions sont l'impossibilité matérielle et morale ; la force ; la copie fidèle et durable. 


Il y a exception quand une des parties n'a pas eu la possibilité d'établir un acte authentique ou un acte sous seing privé (impossibilité physique/matérielle ou morale). Pour l'impossibilité matérielle, nous avons un exemple rendu par la Civ 1ère, le 13 Mai 1964\footnote{Bull. Civ, I, N°251}: "Une partie incapable d'écrire est dans l'impossibilité matérielle de produire une preuve littérale." Une partie analphabète est dans l'impossibilité de produire une preuve résultant d'un acte écrit authentique ou sous seing privé. \\
L'idée générale qui préside lorsque l'on parle d'impossibilité morale est le fait que moralement, en raison des relations qui existaient entre les parties, l'une, voire les deux ont un blocage d'ordre psychologique visant à ce qu'ils ne se sentent pas en mesure vers la demande d'établir cet acte authentique ou sous seing privé: rapports de familles, en mariage etc. Mais aussi les rapports cordiaux: "Attendu qu'ayant constaté que les travaux litigieux avaient été sollicités par Mr. Y dans un contexte d'un lien de voisinage et d'une entente cordiale né d'une passion commune pour les voitures anciennes, la CA en a déduit que le garagiste s'était trouvé dans l'impossibilité morale de se procuré une preuve écrite dans la commande de ses travaux..." La CA a pu valablement rendre cette décision et ce dans le cadre de son appréciation souveraine. Une autre décision du 24 Octobre 1972, par la Civ.3ème: "Les juges du fond apprécient souverainement le point de savoir si une partie s'est trouvé dans l'impossibilité morale d'exiger un écrit". \\


Un événement de force majeure, c'est un événement qui est irrésistible, imprévisible et insurmontable. Il s'agit donc de situations extra-ordinaire. L'on pourra considérer donc que l'on a perdu l'acte authentique ou l'acte sous seing privé et donc déroger à l'article 1341 car le cas de force majeure m'a fais perdre l'original/les copies.


La dernière exception ne concerne que les actes sous seing privés. Dans le cas où l'original n'existe plus, la loi dit que l'on peut présenter une copie qui reproduit l'original fidèle et aussi durable. Une photocopie peut elle être fidèle et durable au sens de l'alinéa 2 de l'article 1348 ? Civ.1ère le 30 Mai 2000: "Mais attendu que la CA appréciant souverainement la valeur et la porté de la photocopie qui lui a été soumise à jugé que celle-ci ne révélait aucune trace de falsification par montage de plusieurs documents, et permettait de constater que les caractéristiques d'ordre général de l'écriture du document photocopié présentait de grandes similitudes avec celle de documents antérieurs constituait une copie sincère et fidèle du document du 21 Mai 1992 au sens de l'article 1348, alinéa 2 du C.Civ." \\
Lorsqu'on est sur le terrain de l'exception au sens de l'article 1348, ce sont les juges du fond qui apprécient souverainement si la copie est fidèle et durable. En général, ils font appel à des experts qui confirment la non présence de traces de montages mais regarde aussi l'écriture.


\subsection{Le règne partagé de la preuve par écrit}

On parlera ici d'autres preuves que écrites. 

\subsubsection{Au titre des exceptions de l'article 1341 du Code Civil}

Le commencement de preuve par écrit ne suffit pas pour établir la preuve de ce que le demandeur invoque. La jurisprudence a ajouté deux conditions: il faut que le demandeur apporte une preuve complémentaire ; il faut apporter une autre preuve, complémentaire, qui doit être extérieure (extrinsèque) à l'acte lui même. \\
Cette preuve complémentaire peut être faite par témoignage ou présomption (indices). Cette jurisprudence constante dit "Il appartient au demandeur qui a rapporté un commencement de preuve par écrit de le parfaire par d'autres éléments te que témoignages ou indices ; et les juges du fond apprécient souverainement si ce complément de preuve a été fourni."\footnote{Civ. 1ère 12 Juillet 1972, Bull Civ I n°185} \\
"Pour compléter un commencement de preuve par écrit, les juges du fond doivent se fonder sur des éléments extérieurs à l'acte lui même tel que témoignages, indices, ou présomption."\footnote{Civ. 1ère 16 Janvier 1985, Bull. Civ. I n°24} \\
Indice: élément de preuve consistant en un fait, un événement, un objet, une trace etc. dont la constatation fait présumé l'existence du fait à démontrer. 


La présomption est une preuve nommée (dite nommée quand elle est défini par le Code Civil, innomé quand elle ne l'est pas). Une présomption est une conséquence d'un fait connu à un fait inconnu. Par exemple, dès qu'un enfant a été conçu ou est né pendant un mariage aura pour père présumé le mari (art. 312). \\
Le fait connu est la situation générale, habituelle. 


On distingue les présomptions simples ou réfragable des présomptions irréfragables ou absolus. \\
La présomption simple ou réfragable peut être combattu par la preuve contraire. \\
La présomption ou irréfragable elle, ne peut être combattu par aucune preuve contraire. \\
Concernant les présomptions prévus par la loi, on parle de présomption légale. Elles sont prévus par les article 1350 et suivants du C.Civ. \\

Pour les présomptions établis par les juges, on a une définition à l'article 1353 du C.Civ. Ces présomptions vont permettre aux juges de déliers les affaires qui n'aurait pas de solution sinon. 

\subsubsection{Au titre des exceptions de l'article 1348}




\subsubsection{L'aveu}











\end{document}
