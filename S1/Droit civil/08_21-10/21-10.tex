\documentclass[12pt, a4paper, openany]{book}

\usepackage[latin1]{inputenc}
\usepackage[T1]{fontenc}
\usepackage[francais]{babel}
\author{Jérémy B.}
\date{}
\title{Cours de droit civil (UFR Amiens)}
\pagestyle{plain}

\begin{document}
\chapter{Les coutumes}


Si la source principale du droit est les texte légaux, la source secondaire est la coutume.

\section{Les éléments constitutifs de la coutume}
\subsection{Un usage ancien}
Pour qu'il y ait coutume, il faut qu'il y ait un usage établi, répété au fil du temps. Ce n'est qu'au bout d'un certain temps, que parce que cet usage s'est répété qu'on dira qu'il y a coutume. \\
"Il faut que l'on puisse dire: tout le monde fait comme cela" Cornu.


\subsection{Une conviction juridique}
Au facteur temps, qui est un élément objectif, s'ajoute un facteur subjectif, un élément psychologique. Pour qu'on puisse parler de coutume, il faut que celle et ceux qui appliquent la coutume, aient le sentiment qu'ils doivent appliquer cette coutume. \\
Celui qui applique une coutume est persuadé que la règle coutumière doit avoir la conviction qu'il applique une règle de droit obligatoire. \\
La coutume est un fait social.


L'adhésion de ces deux facteurs créent la coutume.


\section{Le rôle de la coutume}
Quelle place occupe la coutume en droit français ? Les auteur distingue trois hypothèses.

\subsection{"Secundum Legem" Selon la Loi}

Dans cette hypothèse, il n'y a pas de conflit entre la Loi et la coutume. La coutume intervient à la demande de la Loi. C'est la Loi elle même qui renvoie à la coutume. Elle peut le faire de façon explicite ou implicite. \\
Explicite comme dans l'article 1135 du code civil où le mot usage est utilisé. \\
Implicite comme dans l'article 6 du code civil où l'expression bonne moeurs est utilisé, et donc, renvoie à la morale et aux coutumes. Baudelaire a par exemple été condamné pour atteinte aux bonne moeurs pour "les fleurs du mal".

\subsection{"Praeter Legem" À côté de la Loi}

Ce sont des coutumes qui interviendraient dans des domaines où la Loi ne dirait rien. On peut admettre qu'il y ait des domaines où la Loi ne dirai rien et donc qu'il y aurai coutume. \\
Cependant, d'une part, il faudrait que la coutume soit assez importante, et d'une autre part, que si une tel coutume existait, c'est que le législateur le veut bien. \\
Un exemple traditionnel est la règle coutumière permettant à la femme mariée de porter le nom de famille de son mari. Désormais les textes du code civil sous entendent un certain nombre de chose et donc l'exemple est moins vrai.

\subjectif{"Contra Legem" Contre la Loi}

Une coutume ne peut aller à l'encontre de la Loi. La coutume serai écarté dans le cadre d'une procédure.


Si la coutume a une faible importance dans le droit interne (pas pour tous les pays), elle peut être très importante dans les relations internationales, notamment dans le domaine du commerce international.

% ------------------------------------------------------------------------------------------------------------------------------- %

\part{Les sources indirectes de droit objectif}

\chapter{La jurisprudence}
\section{Définitions possibles de la jurisprudence}
Le terme jurisprudence est susceptible d'avoir deux sens, un sens large et un précis. \\
Dans un sens large, nous dit Cornu dans son vocabulaire juridique "La jurisprudence, c'est l'ensemble des décisions de justice". \\
Dans un sens plus précis, toujours en citant Cornu: "la tendance habituelle d'une juridiction déterminée ou d'une catégorie de juridictions à juger dans tel sens (plutôt que dans tel autre)". Par exemple, dans un cas précis, il faudrait que l'avocat dise à son client "habituellement, tel tribunal juge dans tel sens". Dans un sens étroit, on va chercher des décisions qui illustrent un point de vue important de l'application d'un texte.

\section{Jurisprudence et décisions de justice considérés isolément}

\subsection{Une décision de justice est rendue par un magistrat disant le droit}

On utilise l'expression "dire le droit" car le rôle de juger implique nécessairement pour le magistrat d'appliquer de façon concrète et particulière au cas qui lui est soumis les conditions légales qui elles, sont abstraites et générales. \\
Ainsi, si le juge est saisi d'une action en divorce pour faute par un conjoint. Le juge, après avoir examiné les faits, devra dire si les conditions légales applicables au divorce pour fautes sont remplies concrètement. Voir art 242 du code civil. \\
Le juge dit le droit, c'est à dire qu'il met en oeuvre le droit, il l'applique.



\subsection{Une décision de justice est rendue par un magistrat ne faisant pas le droit}




\section{Jurisprudence et décisions de justice considérés dans un ensemble}







\chapter{La doctrine}








\end{document}
