\chapter{Introduction à l'Histoire du droit}

	\part{Pourquoi l'histoire du droit ?}

L'histoire a une forte présence dans le droit constitutionnel. L'histoire du droit s'intéresse au fait juridique. 

Pour connaître une institution juridique, la comprendre, il est important de connaître leur évolution historique.

L'histoire du droit permet de le replacer dans son contexte. Le droit n'est pas une génération spontanée. La loi est le produit d'une société à un moment donné. Le droit est un fait social. Les règles juridiques entretiennent des relations avec l'état intellectuel, économique, social et politique de la société. Le droit agit la société pour la maintenir, mais la faire évoluer également.

Le droit est inhérent à la société, on a pas de société sans droit, mais pas non plus de droit sans société.

Deux grands types de droit : 

Dans une société plus ou moins organisé, on a le droit qui désigne des règles obligatoire (donc assortis de sanctions), ces règles émanent d'une autorité légitime. Ces règles régissent la conduite des Homme. C'est le droit objectif.

Le droit qui définit le statut des personnes, des individus qui composent la société, comme la liberté, le droit de propriété etc... Ces droits n'ont pas toujours existé. C'est le droit subjectif.

Ce droit (l'ensemble des deux) est en perpétuelle évolution.

Si on ne s'intéresse pas à l'histoire du droit, nous aurions qu'une vision purement technique du droit. Le droit est aussi une science : la science juridique, qui possède donc son propre langage.

Il existe une pratique, une technique, un savoir faire juridique, que l'histoire du droit aide à maîtriser.

Le droit est aussi une culture juridique, que l'on acquiert également par son histoire.
