\documentclass[12pt, a4paper, openany]{book}

\usepackage[latin1]{inputenc}
\usepackage[T1]{fontenc}
\usepackage[francais]{babel}
\author{Jérémy B.}
\date{}
\title{Cours de Relations Internationales (UFR Amiens)}
\pagestyle{plain}

\begin{document}
\maketitle

\section{Rapports de la discipline avec le droit}

\subsection{Droit et société}

Là où il y a société, il y a du droit. Droit et société ne sont cependant pas la même chose. Le droit doit être distingué des autres règles sociales, règle de la morale, de politesse etc... Voir cours d'introduction au droit.

Le droit permet d'apprécier la situation que l'on constate. Le jugement juridique nous permet de dire si c'est valide ou invalide. Un fait est valide au moment où elle correspond aux règles posés par le droit.

Il y a trois fonctions du droit:
\begin[itemize}
\item Pose des règles qui doivent être respéctés
\item Apprécier si des faits donnés sont conforme aux règles posées
\item Tirer les conséquences du respect ou de l'irrespect des règles par les faits
\end{itemize}

Le droit a un caractère fondamental mais limité sur la société. Fondamental car le droit organise la société, évalue les comportements et sanctionne les manquements.

Le droit n'est cependant pas le seul facteur de l'organisation. La moral, la religion, tout fait social en général participe à organiser la société.

\subsection{Droit international et société international}

\subsubsection{L'existence du droit international}
Certains auteurs affirment qu'il n'existe pas de droit qui permet de remplir les fonctions précédemment cités en droit international. Pour certains, le droit international n'existe pas, pour d'autres, il existe mais n'est pas efficace.

Un professeur d'Oxford écrit que l'anarchie est fondamental dans les relations international. Herbert LA Hart a écrit "The concept of law" où il pose la question de ce qu'est le droit, dont un chapitre est dédié au droit international. Il dit qu'il n'existe pas de droit international mais une morale international "Le droit international ne mérite pas son appellation de droit".

Premier argument de Hart: il n'existe pas d'organe centralisé qui permet de créer des règles.

Deuxième argument: il n'existe pas de système juridictionnel obligatoire permettant d'évaluer la conformité des États par rapport aux règles.

Troisième argument: il n'existe pas de police international permettant de sanctionner les manquements au droit.

Le "droit" international manque donc à ces trois fonctions, d'où la conclusion de Hart.


Hart et les autres auteurs qui renient le droit international commet une confusion élémentaire. En effet, ses arguments reposent sur une centralisation de création des règles, d'évaluation et d'application.

Deux auteurs vont le contredire, et donc affirmé l'existence du droit international. Ces auteurs sont Hans Kelser notamment qui a écrit "La théorie pure du droit". Il va affirmer qu'il existe du droit international car pour lui, il existe des moyens de créer des règles, de les évaluer et de sanctionner.

Il va expliquer qu'il y a moyen de créer des règles grâce aux traités. L'évaluation est décentralisé, chaque État évalue tous les autres. Les sanctions sont aussi décentralisés, notamment via des contres mesures.

Les fonctions sont donc remplis, mais le droit international est un droit "primitif" selon Kelser. Le droit interne est largement en avance par rapport au droit international. Le droit international a tendance à évoluer comme le droit interne.

Un autre auteur, Michel Virally, va prendre aussi la défense du droit international mais en contestant Kelser. Pour Virally, le droit étatique et le droit international sont différents. Ce que repprochent les opposants au droit international, c'est qu'il ne ressemble pas au droit interne. Or, pour lui, c'est une erreur que de penser que le droit interne doit être un modèle. La centralisation et la séparation des pouvoirs sont différentes formes de droit possible.

Le droit international est donc bel et bien un droit car il assure les mêmes fonctions mais ne le fait pas de la même façon.

Le droit international est donc bien anarchique dans le sens où il n'y a pas de pouvoir (à ne pas confondre avec l'anomie où il n'y a pas de règles). 

\subsubsection{L'efficacité du droit international}

Le droit international est-il donc efficace ? Il est sûrement moins efficace que le droit interne car la société interne et la société international n'ont pas du tout la même structure.

Les sociétés interne sont dites verticales de par la centralisation des pouvoirs. L'État est juridiquement supérieur aux individus. 

Par opposition, la société international est horizontal. Il n'y a pas de pouvoir supérieur aux États, et un État ne peut pas unilatéralement imposer quelque chose à un autre État.

Toutefois, il faut constater que cette différence d'efficacité est seulement relative. Il n'y a pas de différence de nature, mais de degrés. Le droit interne n'est pas pleinement efficace dans l'absolu, certains crimes restent non élucidés, des infractions non constatés.

Dans le droit international, les règles qui prohibent la guerre sont globalement respectés par exemple. 

Les rapports qui existent entre société interne et droit interne sont globalement les mêmes qu'entre le droit international et la société international. Le droit international peut donc être utilisé comme point de vue pour étudier les relations internationales. 
\end{document}
