\documentclass[12pt, a4paper, openany]{book}

\usepackage[utf8]{inputenc}
\usepackage[T1]{fontenc}
\usepackage[francais]{babel}
\author{Jérémy B.}
\date{}
\title{Cours de Relations internationales (UFR Amiens)}
\pagestyle{plain}

\begin{document}
\maketitle

\chapter{Introduction à l'introduction aux relations internationales par l'actualité}
4 à 6 millions de réfugiéés partant de la Syrie:
Deux accueils, compassion ou hostile.

L'hostilité s'explique sur trois plans: économique (emploi), culturel (l'europe traverse une crise d'identité), "stratégique" (crainte d'arrivé de terroristes).

En ce qui concerne la communauté universitaire, nous allons essayer de prendre nos distances de ces deux réactions.

L'Allemagne a rétabli ses contrôles aux frontières: la compassion a ses limites.

\section{Rappel géopolitique}

La Syrie est au Moyen Orient et victime d'une guerre civile comportant quatre protagonistes: le gouvernement de Bachar, des rebelles, mais aussi Daech et les Kurdes.

La Syrie est un ancien pays sous mandat français, placé sous l'autorité française par la SDN. Ce mandat a duré 26ans (1920-1946). 

La crise des migrants s'inscrit dans la continuité du printemps Arabe. L'occident a été un soutien à ce printemps Arabe. La crise des migrants n'a donc rien de nouveau. La situation est classique, on doit donc être capable de la gérer.

\section{La crise des migrants syriens}

Si l'on regarde seulement les journaux TV, on pourrait croire que l'Europe accueille toute la misère du monde, c'est faux. Près de deux millions ont fuis vers la Turquie, un million vers le Liban, 600 mille en Jordanie, 133 mille en Égypte...

Il n'y a pas de volonté pour les populations de venir en Europe, d'où leur fuite vers les pays limitrophes.

80\% des réfugiés dans le monde sont accueillis dans des pays en voie de développement. 

Le Liban est une terre d'accueil des réfugiés historiquement (notamment par la guerre israelo-arabe), de plus le Liban est historiquement lié à la Syrie car sous mandat français aussi.

Le terme de migrants est problématique. Le terme migrant n'est pas le terme juridiquement correct. En effet, si les migrants sont considérés comme réfugiés, ils ne peuvent être renvoyés dans leur pays. Être accueilli quand on est réfugié est un droit reconnu par la convention de Genève.

La convention de Genève est placé sous la surveillance l'UNHCR. Celui-ci signal que le terme migrant est inadapté.  

\section{Conclusion}

Il n'existe plus aujourd'hui de problèmes seulement nationaux. Les problèmes sont imbriqués. 

Si on fait la synthèse des protagonistes de la crise syrienne, on y trouve des groupes privés, non étatiques, des États, des instances supra-nationales... Il y a diversité des acteurs.


Il y a, aujourd'hui, en droit, plus une seule situation où l'on n'invoque pas du droit international, d'où l'importance d'étudier les relations internationales. Il existe une grande pluralité des droits. 

\chapter{Introduction aux relations internationales}

	\section{Objet de la discipline}
	
"Le champs des relations internationales est indéfini et mobile" Serge Sur

Il existe des situations qui sont évidemment internationales comme la guerre. Mais toutes les situations ne sont pas ainsi. Des situations qui peuvent paraître purement nationales mais qui en fait relèvent aussi du droit international. Par exemple les ECTS dans les diplômes nationaux.

On ne peut donc pas définir le champ du droit international, puisqu'une situation en dépend suivant comment on la considère.

Le champ du droit international est mobile. Par exemple, la question de l'environnement était originairement national, désormais, après certains accords signés, c'est désormais une question encadré par le droit international. 

La notion de relation est étendue, et la notion d'international est ambigu.

		\subsection{L'étendu de la notion de relation}

Une relation sociale est une relation entre deux personnes. Il y a deux catégories de personnes en droit, la personne physique (les individus) et la personne morale (une entité).

L'État est une personne morale, la plus puissante de tous. FH qui serre la main a BO est une relation entre deux personnes physiques qui symbolises une relation entre deux États, c'est une relation internationale.

Il existe plusieurs relations, toutes différentes.

Les relations commerciales sont des accords relatif aux échanges. Par exemple, la France et le Pakistan ont signé un accord (bilatéral) le 26 mai 1955 régissant un secteur de commerce: le coton.

La multiplication de ces accords a amené à la création de l'Organisation Mondiale du Commerce qui régit de manière général les échanges commerciaux. Le simple fait pour un État de devenir membre de l'OMC permet d'échanger librement avec les autres membres. L'OMC a été créé en 1994.


Les relations culturelles:
\begin{itemize}
\item Accord sur la reconnaissance académique des diplômes et des périodes d'études supérieurs signé entre la France et la Roumanie en 2012
\item Accord d'un processus d'échanges culturels entre la France et l'Inde de Février 2013
\item Convention pour savoir qui va accueillir une coupe du monde, les JO...
\end{itemize}

Les relations dans le domaine de l'environnement:
\begin{itemize}
\item Convention internationale pour la régulation de la chasse à la baleine, 1946
\item Union pour la conservation de la nature, 1948
\item Convention de Vienne sur la protection de la couche d'ozone, 1985
\item Convention cadre des Nations Unies sur le réchauffement climatique, 1992\footnote{Puis protocole de Kyoto en 1997 puis COP21 pour le remplacer, Kyoto expirant}
\end{itemize}

Les relations dans le domaine de la paix et la sécurité. C'est le domaine le plus visible des relations international, mais aussi le plus ancien. On peut considérer que la guerre est la première des relations internationales\footnote{Il faut être prudent sur cette affirmation}.

Il existe énormément de traités concernant ce domaine. Un classique est le traité d'alliance.

Quelques exemples de traités:
\begin{itemize}
\item Le premier traité écrit connu est un traité de paix entre deux provinces de l'actuel Irak, en 2300 av JC.
\item Le traité de Munster entre le Saint Empire Romain et la France, 1648
\item Traité de paix ente les puissances alliés et l'Italie, 1947
\item Traité de Washington entre l'Égypte et Israël, 1979\footnote{L'Égypte étant le premier pays Arabe à reconnaitre Israël}
\item Accord cadre général de Dayton conclu entre la Serbie, la Bosnie-Hérségovine et la Croatie, 1995
\end{itemize}

Depuis la fin de la seconde guerre mondiale, il y a une forte baisse des traités d'alliance et de paix. La raison est simple, dans la charte des Nations Unis interdit l'emploi de la force armé entre États\footnote{La guerre entre États est donc illégal}, et donc chaque alliance est illégitime.

		\subsection{Nature des relations}
		
Il existe plusieurs relations:
\begin{itemize}
\item Bilatéral: seulement deux parties
\item Multilatéral: plus de deux (régionnal et universel)
\end{itemize}

Relation bilatérale: crée des effets entre seulement deux pays, deux États. Ces relations sont extremements nombreuses. Chaque État conclut des accords avec tous les autres États, il en existe donc des million (chacun des 200 pays ayant des accords avec chacun des autres pays dans tous les domaines possibles).

Les relations bilatérales sont la base des relations.

Relation multilatérale: crée des droits entre de nombreux États. On distingue le régional (un continent) de l'universel (la planète). Accords multilatérals notables: UE, ALENA, ASEAN, MERCOSUR, APEC.

Plusieurs conventions des droits de l'Homme: CESDH (1949), CIADH(Amérique), CADH (Afrique).

Tout n'est pas si simple: les organisations s'imbriquent.


Les organisations à vocation universels essayent de réunir le plus d'États possible. La plus importante est l'ONU. Tous les États sont présents à l'ONU sauf ceux dont le statut d'État est contesté (comme Taïwan, le Kossovo). 

L'OMC est aussi une de ces oranisations. 


La scène internationale est à la fois un lieu de coopération mais aussi de compétition. Les deux se passent en même temps. On cherche donc à distinguer deux types de relation, les relation irémiques (tournés vers la paix), les relations polémiques (où chacun lutte pour défendre ses propres intérêts).

Sur la dimension irémique: pour de nombreux auteurs, l'histoire des relations internationales, c'est l'histoire de la pacifisation du monde, c'est une conception idéaliste. Les relations ont pour but des idéaux: la paix, la justice...\footnote{Kant, Paix perpetuelle}

Projet de Jean Monnet: souhaite fonder l'UE dans le but d'initier une union toujours plus étroite entre les peuples européens. 

Pendant longtemps, les seules coopérations entre États étaient des coopérations militaires. Cela devient faux à partir de 1850. Il existe des questions universels qui émergent et qui sont la question de tous comme la protection de l'environnement ou la protection des droits de l'homme.

Nous sommes passés d'une simple société internationale dans laquelle il existait des rapports sociaux à une communauté internationale destinés à protéger un bien commun (UNESCO par exemple).


Sur la dimension polémique, des auteurs contestent la dimension précédente. Ils se disent réalistes, par opposition aux idéalistes. Machiavel écrit qu'il ne faut s'attacher qu'au résultat (la survie ou la mort), qu'il ne faut rechercher dans la politique international que la conservation au pouvoir. "Le prince doit savoir perséverer dans le bien lorsqu'il n'y voit aucun inconvénient". Avec une politique ainsi, on justifie tous crimes internationaux.

Dans l'État de nature, les hommes jouissent dans la liberté , y compris de celle de porter atteinte aux autres.

Dans chaque État il existe une sorte de contrat social, cependant, il n'y en a pas en droit international. Il n'existe pas d'idéaux international.

Il existe encore aujourd'hui énormément de violence. Chaque État cherche à survivre. D'où l'idée de la compétitivité, qui prend place non dans une idée de coopération mais de compétition.

Il existe une forme de violence militaire dans les relations internationales, en témoigne les nombreuses interventions militaires récentes. Cependant, les raisons de la violence mutent fortement. Quand la Russie protège la Serbie de ses exactions, l'OTAN intervient au nom de certains idéaux. Malgré le blocage à l'ONU.

Le gros problème de la communauté internationale est que les différents États ne partagent pas les mêmes idéaux. La volonté de paix crée la guerre. Napoléon écrit qu'il fait la guerre car il souhaite la paix. On ne peux pas envisager les relations séparément dans leurs aspects idéalistes ou réalistes.

La disparition de violence est elle même une cause de violence.

		\subsection{L'ambiguïté du terme d'internationalité}
		
Le mot "international" apparaît en 1761 en Anglais dans une sorte de manuel de droit écrit par Bentam. Il est traduit en France en 1801. "Les relations internationales sont les relations qui s'établissent entre souverains en tant que tel" Jérémy Bentam, 1761. Or le terme international fait débat car les relations de monarque à monarque ne sont pas des relations de nation à nation mais d'État à État (ce qui est donc une relation inter-étatique).

L'État et la nation sont deux choses différentes. Il n'y a pas nécessairement de nation avec État comme d'État avec une seule nation. Les Indiens sont par exemple une nation qui n'a pas d'État, ils sont Américain. D'autres États comportent plusieurs nations, comme beaucoup de pays Africains.

International est donc réducteur puisqu'il représente les relations entre deux groupes précis.

Il existe deux grandes théories pour définir la nation, la théorie allemande et la théorie française.

La théorie allemande est fondée sur l'étymologie du mot nation, qui vient du mot latin nacio qui signifie naître. C'est donc en naissant dans un groupe social donné qu'on fait partie de ce groupe social. Cette conception apparaît au 19e siècle sous la plume de Fichte dans "Discours à la nation allemande". Cette conception est exclusive.

La conception Française est opposée. Ernest Renan explique à la Sorbonne en 1882 "Qu'est-ce qu'une nation ?". Il explique que ce n'est pas en naissant dans une communauté que l'on y est intégré, mais parce qu'on le veut. Cette conception est inclusive.

La conception allemande est objective puisqu'elle se base sur des faits auquel on ne peut rien. Alors que la conception française est subjective car elle se base sur la volonté.

Ces deux conceptions ont un point commun: on désigne un certain type de groupe social, opposé à tous les autres.


Les acteurs des relations internationales vont changer. Par exemple, au 18e siècle, les relations internationales sont essentiellement familiales: une dynastie se battant contre une autre. GoT représente très bien l'idée de famille.

En 1701, le Roi d'Espagne meurt, dont l'héritage irait au duc d'Anjou qui est aussi le successeur de Louis XIV. C'est pour éviter cette situation que la guerre de succession d'Espagne commencera. Ce sont donc des problèmes de famille qui provoquent des conflits internationaux. Étudier les relations internationales médiéval, c'est donc étudier les relations familiales.

Aujourd'hui, on étudie un nombre d'acteurs bien plus conséquent dont la nature est complètement nouvelle. Certains acteur n'étant pas des États réussissent à les concurrencer.

On a donc dans les relations internationales, des relations inter-étatiques mais aussi des relations trans-étatiques (qui passent à travers les États).

Les relations internationales en tant que discipline naît en même temps que l'État au 17e siècle d'où l'étude d'abord des relations inter-étatiques. Les États sont alors très important mais les autres acteurs vont commencer à se développer, notamment au 20e siècle.

Avant la deuxième moitié du 20e siècle, il n'existe aucun groupe, qu'il soit paramilitaire ou économique qui puisse rivaliser avec l'État. En revanche, aujourd'hui, l'État est fortement concurrencé, par exemple, la valeur en bourse total d'Apple représente un budget annuel Français.

Les grands groupes privés ont donc réussi à s'affranchir de la domination des États. Ce qui est vrai pour les grands groupes économiques est aussi vrai pour d'autres groupes comme Greenpeace par exemple ou encore Amnesty International ou Human Right Watch. Leur métier à ces organisations est même de surveiller les États. Enfin, il y a les groupements terroristes qui essayent d'être présent aux quatre coins du monde.

Il faut donc autant prendre en compte les relations inter-étatique que les relations trans-étatique.
		

Une question est donc internationale dès que ses enjeux traversent les frontières d'un unique état et dès que plusieurs acteurs sont impliqués. Jean-Claude Zarka: "Les relations internationales sont l'ensemble des rapports pouvant s'établir entre des groupes sociaux et qui traversent les frontières".
		
	\section{Rapports de la discipline avec le droit}

	\section{Objectifs du cours}

\end{document}
