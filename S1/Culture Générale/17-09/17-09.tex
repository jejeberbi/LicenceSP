\documentclass[12pt, a4paper, openany]{book}

\usepackage[latin1]{inputenc}
\usepackage[T1]{fontenc}
\usepackage[francais]{babel}
\author{Jérémy B.}
\date{}
\title{Cours de Culture Générale (UFR Amiens)}
\pagestyle{plain}

\begin{document}
\maketitle

\part{Culture générale en relation avec la famille}

\chapter{Introduction}

La question du mariage pour tous en 2013 et la virulences des différentes manifestations et la sur-médiatisation de certaines personnalités montrent que la question de la famille est, aujourd'hui encore, au centre des questions de la société contemporaine.

Pour définir la famille, il va falloir distinguer tous les points de vue différents et les analyser, essayer de les comprendre. Les remettre dans leurs contextes historiques et culturels.

Les questions posés sont donc en rapport à l'ordre social, le rapport à la nature et à la culture.

L'idée de l'ordre social est la stabilité de la société. En touchant à la famille, on bouscule l'ordre social, en effet, pour beaucoup, "la" famille est la base de l'ordre social.

Rappel: la nature et la culture sont opposés.

Une question à se poser est donc y-a-t-il "la" ou "les" familles ?

Selon l'INSEE, il y a deux notions: le ménage et la famille. Le ménage est l'ensemble des individus vivant dans un même foyer, peu importe la nature de leurs relations. Ceci intéresse plus l'INSEE que la famille.

La famille a la particularité d'être un ménage fonctionnant de manière collective, communautaire.

La famille, selon l'INSEE est un ménage avec des liens de parenté (couple, enfant, parents, grand-parents).

La convention des ménages permet de dénombrer la population mais aussi de prendre en compte la diversité des situations (familles mono-parentales etc...) mais ne prenaient jusqu'à quelques années que les couples de sexe opposés comme une famille.

La première reconnaissance des familles homosexuelles s'est faite avec le PACS à la fin des années 1990. Mais rien d'aussi transcendant que le mariage pour tous.

Il se pose alors la question des normes, des conventions sociales. Les normes sociales sont des prescriptions qui sont formulés dans la société et tendent à s'imposer aux autres malgré qu'elles ne soit pas dans le droit.

Ces normes peuvent être véhiculés par de nombreux acteurs, de l'Église aux médias.

La première définition sur la famille nous vient de la biologie. L'idée de nature, donc. 

Mais il y aura des définitions plus sociologiques, comme Émile Zola et les Rougon-Maquart.

Dans la criminalité, jusqu'au début du XXe siècle, on cherchait des explications biologiques, aujourd'hui, la tendance est à l'analyse psycho-sociologique.\footnote{Comme le représente les séries policières américaines}

Les Sciences Humaines qui se détachent donc de la biologie sont:
\begin{itemize}
\item Ethnologie/Anthropologie (étude des cultures et des sociétés)
\item Histoire (étude des sociétés dans l'histoire)
\item Sociologie (diversité de la société dans laquelle on vit)
\item Psychologie (étudie l'affect, les relations, l'identification, les souffrances, le comportement)
\end{itemize}


% Plan %

\part{La famille dans l'actualité sociale, politique et culturelle}

\chapter{La presse}

\section{La place des quotidiens en France}

Les quotidiens se vendent de moins en moins, à cause notamment de supports beaucoup plus courant comme Internet ou encore la télévision. Cependant, les presses papiers régionales restent énormément consultés, plus que grand nombre de journaux nationaux.

Les journaux nationaux ont cependant une place importante dans l'expression de l'opinion. 

Il y a une opposition entre les différents journaux, comme entre partis politiques. Cependant, même si "Libération" par exemple est un journal dit de gauche, il n'est pas forcément toujours d'accord avec le PS. 

Il y a des relations entre les organes de presse qu'il faut essayer de comprendre.

Le journal le plus diffusé est "Le Parisien/ Aujourd'hui en France". 

La presse crée les opinions, les structure. Elle a pour fonction de raconter ce qu'elle voit en la commentant.

3 dates à retenir pour les plus grandes manifestations: 1968, 1984, 2013.

\section{Historique des quotidiens}

Le Figaro: créé en 1826, plus vieux quotidien, créé pendant la restauration. Sous la monarchie de Juillet, le Figaro se dira ouvertement républicain, donc de gauche, à l'époque. Sous la IIIe république, ce sera un journal républicain tant lu par la bourgeoisie que par les ouvriers. Il sera cependant contre la commune de Paris. Sous la 2ème guerre mondiale, le Figaro prendra parti pour le camp du Général de Gaulle. Le Figaro se définit lui même comme républicain, libéral (sur le plan politique comme le plan économique) et conservateur.


Le Parisien/Aujourd'hui en France: créé à la libération (le Parisien libéré à l'époque), sous forme de coopérative (pour être indépendant). Le groupe qui crée le journal est plutôt centriste (pas mal de Gaullistes mais aussi des maréchaliste non collaborationniste). Le journal se définit comme populaire et de qualité (qui va croisé actualité local et national/international). À partir des années 1960, il va se marquer à droite. Dans les années 1970, il va connaître une importante crise, dans les années 80, une crise de succession. En 1986, le journal revoit sa ligne éditoriale (nom, photos, article etc..) et veut devenir un journal absolument neutre (suppression des éditorial).


Le Monde: créé en 1944, prend la suite de "Le Temps". Le journal est soutenu par De Gaulle qui veut un grand journal de qualité, avec des analyses et traitant des rapports internationaux. Bref, un journal de référence. Cela supposait d'être le plus indépendant possible et Le Monde a été pendant longtemps une référence en matière d'indépendance, ce qui est moins vrai depuis les années 1990.


Libération: créé en 1973 dans la foulée des événements de 1968. Au départ fondé par des Maoïste, il finira par se recentré au centre-gauche à la suite de crises successives dans les années 70. À la fin des années 70, le journal se définit comme libéral libertaire. Dans les années 80, avec la gauche au pouvoir, Libé se fera le porte voix d'une nouvelle gauche, une gauche libertaire, moins sociale. À partir des années 2000, le journal a d'énormes difficultés qu'il possède encore aujourd'hui.


L'Humanité: créé en 1904 par Jean Jaurès. Le journal deviendra en 1920 la propriété du  PCF




\chapter{Les mouvements sociaux}

\chapter{Les religions}

\chapter{Les partis politiques}

\chapter{Le cinéma et la télévision}


\part{Les approches scientifiques de la famille}

\chapter{L'approche ethnologique}

\chapter{L'approche historique}

\chapter{L'approche sociologique}

\chapter{L'approche psychologique}

\chapter{L'approche philosophique}


\end{document}
