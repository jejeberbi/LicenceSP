\documentclass[12pt, a4paper, openany]{book}

\usepackage[utf8]{inputenc}
\usepackage[T1]{fontenc}
\usepackage[francais]{babel}
\author{Jérémy B.}
\date{}
\title{Cours de sociologie (UFR Amiens)}
\pagestyle{plain}

\begin{document}
\maketitle

\chapter{Introduction}

	\section{Qu'est-ce que la sociologie ?}

On peut caractériser la sociologie comme l'étude scientifique du social, c'est à dire de la société et des hommes vivant et agissant en société. (Raymond Aron) %Rechercher le nom exact... %

L'objet propre de la sociologie c'est la société et le social. La sociologie essaye d'étudier la morphologie de la société, sa forme, son organisation, sa composition. Mais aussi les comportements collectifs (impliquant un grand nombre d'individus), mais aussi les comportements individuel commun à un grand nombre d'individus.

La sociologie s'intéresse à l'influence du social sur les comportements individuels. Beaucoup de comportements qui nous semble personnel, naturel peuvent être étudiés par la sociologie car relève finalement du social.

Le fait de mourir ne fait pas partie de la sociologie, en revanche, étudier quand et comment on meurt dans tel ou tel groupe social en relève. Ce sont des faits qui relèvent des habitudes sociales. \newline

La sociologie a une forte ambition scientifique. Une science se définit toujours par un ensemble de connaissance qui se caractérise par un objet d'étude, une méthode (respectant certains principes fondé sur l'objectivité et la raison) ainsi qu'un projet explicatif concernant les phénomènes étudiés.

En philosophie, la réflexion de la science est l'épistémologie.

Norbert Helias: "Les scientifiques sont des chasseurs de mythes. En s'appuyant sur l'observation de faits, ils s'éforcent les images subjectives des complexes évenementiels, les mythes, les croyances et les spéculations méta-physique par des théories, c'est à dire par des modèles de relation que l'observation des faits peut vérifier, coroborer, et corriger." Dans "Qu'est-ce que la sociologie ?"

La sociologie propose de mieux comprendre et mieux connaitre le monde social.

	\section{Pourquoi apprendre la sociologie ?}

La sociologie a une utilité réformatrice: mieux connaitre la société pour mieux la réformé. C'est donc un usage politique. 

Chez certains sociologues, il y a une vocation critique de la sociologie, ils cherchent à percer à jour les phénomènes de la domination par exemple.

Bourdieu a montré que les classes dominantes se servaient du système scolaire pour asseoir leur domination, le justifiant par la méritocratie scolaire.

La sociologie n'a pas forcément d'usages pratiques, elle peut avoir valeur en soi en établissant des vérités scientifiques créant ainsi des connaissances.

\part{Qu'est-ce que la sociologie?}
\chapter{Histoire de la sociologie}

Les théories qui guident le sociologue sont le fruit de travaux antérieurs qui offrent des modèles d'analyse de la société. Pour faire un travail sociologique, on utilise des techniques qui ont déjà été éprouvés.

La sociologie est à la fois une démarche individuel mais aussi collective: les travaux sont partagés, commentés et donc confirmés ou infirmés par d'autres travaux. Il y a une relation de contrôle réciproque.

Pour comprendre les travaux contemporains, il faut connaitre l'histoire de la sociologie.

\section{Fondateurs et précurseurs}

Le terme de sociologie apparait dans la première moitié du 19e siècle, avec Auguste Comte, en 1832. Marx aussi s'interresse à ce domaine, sans le nommer, en effet, le domaine n'existe pas à cette époque. Ils ne participent pas à la fondation de la sociologie, ce sont des précurseurs. Ils ne recourent pas aux méthodes d'investigations et aux méthodes d'explication qui seront ceux de la sociologie moderne.

Ils sont précurseurs car par leurs travaux, ils constituent une source d'inspiration importante pour les sociologues. Marx, Thocqueville et Comte propose des réflexions importantes sur les transformations sociales, économiques.

Ceux que l'on considère comme fondateur: Durkeim(1895, règle de la méthode sociologique), Weber(1903, revue sociologique), l'école de Chicago(1895, fondation du département sociologique). La sociologie apparait au même moment en Allemagne, en France et aux États-Unis. C'est désormais une discipline autonôme car une communauté sociologue se forme.

\section{Facteurs d'émergence}

La sociologie est la fille de la révolution politique et industrielle du 19e siècle. La révolution industrielle provoque des bouleversements sociaux.

D'un point de vue politique, on a un certain nombre de révolutions qui amènent à la démocratie. Il y a un pouvoir nouveau comparé à la monarchie, les hommes sont désormais considérés égaux. Le pouvoir tient maintenant sa légitimité de la société à contrario des monarchies qui tenaient leur pouvoir de Dieu.

La question du pouvoir, de l'autorité, du peuple, du pouvoir sont donc des questions nouvelles qui vont générer un objet d'interrogation et donc rendre possible le projet sociologique.

La révolution industrielle va contribuer à modifier la morphologie de la société notamment les catégories dominantes. Le 19e siècle voit l'émergence de la bourgeoisie où ils possèdent leurs richesses et leur pouvoir du capitalisme. On passe d'une domination du propriétaire terrien sur ses paysans à une domination du patron sur ses ouvriers. 

Il y a aussi le développement de profession plus modeste profitant du capitalisme. Il n'y a pas qu'une grande bourgeoisie d'affaire et industrielle mais aussi une petite et moyenne bourgeoisie. Dans ces nouvelles professions, on a les ingénieurs, les professions juridiques, les fonctionnaires dont le nombre augmente.

Il y a un lien entre essor des sciences et essor de la bourgeoisie: ceux ci tenant leur légitimité de leur capacité reconnu par un diplôme. Il y a une valorisation du savoir.

Il y aussi l'émergence d'un prolétariat urbain qui fait peur aux élites dominantes malgré que ce ne soit pas une majorité de la population en France au 19e. Cela va donc être un thème de réflexion fondamental pour Marx ou Weber. 

\end{document}
