\documentclass[10pt, a4paper, openany]{book}

\usepackage[utf8x]{inputenc}
\usepackage[T1]{fontenc}
\usepackage[francais]{babel}
\usepackage{bookman}
\setlength{\parskip}{5px}
\author{Jérémy B.}
\date{}
\title{Cours de Sociologie (UFR Amiens)}
\pagestyle{plain}

\begin{document}
\maketitle
\tableofcontents

\chapter{Introduction}

\section{Qu'est-ce que la sociologie ?}

On peut caractériser la sociologie comme l'étude scientifique du social, c'est à dire de la société et des hommes vivant et agissant en société. (Raymond Aron) %Rechercher le nom exact... %

L'objet propre de la sociologie c'est la société et le social. La sociologie essaye d'étudier la morphologie de la société, sa forme, son organisation, sa composition. Mais aussi les comportements collectifs (impliquant un grand nombre d'individus), mais aussi les comportements individuel commun à un grand nombre d'individus.

La sociologie s'intéresse à l'influence du social sur les comportements individuels. Beaucoup de comportements qui nous semble personnel, naturel peuvent être étudiés par la sociologie car relève finalement du social.

Le fait de mourir ne fait pas partie de la sociologie, en revanche, étudier quand et comment on meurt dans tel ou tel groupe social en relève. Ce sont des faits qui relèvent des habitudes sociales. \newline

La sociologie a une forte ambition scientifique. Une science se définit toujours par un ensemble de connaissance qui se caractérise par un objet d'étude, une méthode (respectant certains principes fondé sur l'objectivité et la raison) ainsi qu'un projet explicatif concernant les phénomènes étudiés.

En philosophie, la réflexion de la science est l'épistémologie.

Norbert Helias: "Les scientifiques sont des chasseurs de mythes. En s'appuyant sur l'observation de faits, ils s'éforcent les images subjectives des complexes évenementiels, les mythes, les croyances et les spéculations méta-physique par des théories, c'est à dire par des modèles de relation que l'observation des faits peut vérifier, coroborer, et corriger." Dans "Qu'est-ce que la sociologie ?"

La sociologie propose de mieux comprendre et mieux connaitre le monde social.

	\section{Pourquoi apprendre la sociologie ?}

La sociologie a une utilité réformatrice: mieux connaitre la société pour mieux la réformé. C'est donc un usage politique. 

Chez certains sociologues, il y a une vocation critique de la sociologie, ils cherchent à percer à jour les phénomènes de la domination par exemple.

Bourdieu a montré que les classes dominantes se servaient du système scolaire pour asseoir leur domination, le justifiant par la méritocratie scolaire.

La sociologie n'a pas forcément d'usages pratiques, elle peut avoir valeur en soi en établissant des vérités scientifiques créant ainsi des connaissances.

\part{Qu'est-ce que la sociologie?}
\chapter{Histoire de la sociologie}

Les théories qui guident le sociologue sont le fruit de travaux antérieurs qui offrent des modèles d'analyse de la société. Pour faire un travail sociologique, on utilise des techniques qui ont déjà été éprouvés.

La sociologie est à la fois une démarche individuel mais aussi collective: les travaux sont partagés, commentés et donc confirmés ou infirmés par d'autres travaux. Il y a une relation de contrôle réciproque.

Pour comprendre les travaux contemporains, il faut connaitre l'histoire de la sociologie.

\section{Fondateurs et précurseurs}

Le terme de sociologie apparait dans la première moitié du 19e siècle, avec Auguste Comte, en 1832. Marx aussi s'interresse à ce domaine, sans le nommer, en effet, le domaine n'existe pas à cette époque. Ils ne participent pas à la fondation de la sociologie, ce sont des précurseurs. Ils ne recourent pas aux méthodes d'investigations et aux méthodes d'explication qui seront ceux de la sociologie moderne.

Ils sont précurseurs car par leurs travaux, ils constituent une source d'inspiration importante pour les sociologues. Marx, Thocqueville et Comte propose des réflexions importantes sur les transformations sociales, économiques.

Ceux que l'on considère comme fondateur: Durkeim(1895, règle de la méthode sociologique), Weber(1903, revue sociologique), l'école de Chicago(1895, fondation du département sociologique). La sociologie apparait au même moment en Allemagne, en France et aux États-Unis. C'est désormais une discipline autonôme car une communauté sociologue se forme.

\section{Facteurs d'émergence}

La sociologie est la fille de la révolution politique et industrielle du 19e siècle. La révolution industrielle provoque des bouleversements sociaux.

D'un point de vue politique, on a un certain nombre de révolutions qui amènent à la démocratie. Il y a un pouvoir nouveau comparé à la monarchie, les hommes sont désormais considérés égaux. Le pouvoir tient maintenant sa légitimité de la société à contrario des monarchies qui tenaient leur pouvoir de Dieu.

La question du pouvoir, de l'autorité, du peuple, du pouvoir sont donc des questions nouvelles qui vont générer un objet d'interrogation et donc rendre possible le projet sociologique.

La révolution industrielle va contribuer à modifier la morphologie de la société notamment les catégories dominantes. Le 19e siècle voit l'émergence de la bourgeoisie où ils possèdent leurs richesses et leur pouvoir du capitalisme. On passe d'une domination du propriétaire terrien sur ses paysans à une domination du patron sur ses ouvriers. 

Il y a aussi le développement de profession plus modeste profitant du capitalisme. Il n'y a pas qu'une grande bourgeoisie d'affaire et industrielle mais aussi une petite et moyenne bourgeoisie. Dans ces nouvelles professions, on a les ingénieurs, les professions juridiques, les fonctionnaires dont le nombre augmente.

Il y a un lien entre essor des sciences et essor de la bourgeoisie: ceux ci tenant leur légitimité de leur capacité reconnu par un diplôme. Il y a une valorisation du savoir.

Il y aussi l'émergence d'un prolétariat urbain qui fait peur aux élites dominantes malgré que ce ne soit pas une majorité de la population en France au 19e. Cela va donc être un thème de réflexion fondamental pour Marx ou Weber. 

\chapter{Émile Durkheim et la sociologie holiste \newline La fondation de la sociologie comme science}

Étudier les débuts de la sociologie en France revient à étudier Émile Durkheim.

\section{Les étapes d'une carrière}

\subsection{Repères}
\begin{itemize}
\item Né à Épinal le 15 Avril 1858
\item 1887: chaire de "pédagogie et de science sociale" à l'Université de Bordeaux
\item 1902: Sorbonne
\item mort en 1917
\end{itemize}

Durkheim naît dans une famille juive traditionaliste qui le destine à être rabbin, mais il se détachera de la religion et fera des études prestigieuses. Il sera agrégé de philosophie par l'ENS en 1882. Il sera recruté par l'Université de Bordeaux puis par la Sorbonne.

C'est entre 1893 et 1912 que Durkheim publie l'essentiel de son oeuvre sociologique:
\begin{itemize}
\item "De la division du travail social" 1893
\item "Les règles de la méthode sociologique" 1895
\item "Le suicide" 1897
\item "Les formes élémentaires de la vie religieuse" 1912
\item Ainsi qu'un recueil d'articles et de conférences (comme "Leçons de sociologie")
\end{itemize}

\subsection{Durkheim et l'université}
\begin{itemize}
\item Durkheim fait entrer la sociologie à l'université (Bordeaux puis Sorbonne)
\item Un intellectuel engagé
\item Proche du gouvernement républicain: conseiller du ministre de l'éducation
\end{itemize}

Durkheim est fondamentalement républicain, fortement engagé, il est partisan d'un socialisme humaniste et démocratique, réformiste. C'est un ami de Jean Jaurès.

C'est cet engagement qui lui fera avoir la confiance de certains responsables politique, notamment Louis Liard, directeur de l'enseignement supérieur au ministère de l'instruction. Ces soutiens lui permettront de faire entrer la sociologie à l'université.

En 1900, lors d'une réforme de l'université et de l'enseignement, Durkheim est sollicité en qualité de conseiller. 

\subsection{La constitution d'une École}
Durkheim a cherché, très tôt, à constituer une école, une "équipe" de chercheurs qui va travailler sous sa direction. Il va ainsi, chercher à répandre la sociologie.

Son premier collaborateur est Marcel Mauss, qui va l'assisté notamment sur la constitution des statistiques utilisées dans "Le suicide". C'est son plus proche collaborateur. Après la mort de Durkheim, Mauss continuera à s'occuper de la revue créé par Durkheim. Il faut noter que Mauss est le fondateur de l'ethnologie en France. Il est connu notamment pour son texte sur le don/contre don.

Durkheim a d'autres collaborateur qui agiront aussi pour la postérité de Durkheim, comme Célestin Bouglé, François Simiand, Maurice Halbwachs. Ils ont tous plus ou moins lancé des domaines de la sociologie.


\section{Fonder une nouvelle science, définir son objet}

\subsection{Le fait social (Les règles de la méthode)}
Dans "Les règles de la méthode sociologique", Durkheim cherche à définir son objet d'étude, un objet propre à la sociologie. Ce que peut apporter la sociologie de nouveau, c'est le fait social dont il dira que toutes les autres disciplines l'ont ignoré.

Les faits sociaux sont "des types de pensée ou de conduite" qui sont "extérieurs à l'individu" mais qui sont "doués d'une puissance impérative et coercitive en vertu de laquelle ils s'imposent à lui, qu'il le veuille ou non".

Durkheim montre que des conduites durent dans le temps, et ce, malgré la disparition des individus, il y a donc bien des normes, extérieurs aux individus qui agissent sur eux. Les faits sociaux ne résultent pas de la volonté humaine, mais qui sont imposés par la société.

Chaque individu naît dans une société structuré qui prescrit ce qu'il faut faire et comment le faire. Les individus, qu'il le veuille ou non, intègres ces faits au cours de la socialisation. 

Durkheim dit que la société n'est pas la simple somme des individus qui la composent, elle est quelque chose d'autre. Une identité qui a une existence propre, une autonomie et un pouvoir de contrainte.

Ces contraintes ne se ressentent pas, mais c'est quand on enfreint ces contraintes qu'on perçoit le caractère contraignant des faits sociaux car on encourt une sanction sociale. Ces sanctions peuvent être la moquerie, l'exclusion, le harcèlement, le ridicule, etc...

Le social ne peut s'expliquer que par le social. D'où le détachement des autres sciences (notamment la psychologie). C'est en comprenant comment la société produit des normes de comportement, comment elle agit sur les individus, qu'il est possible de rendre compte des actions individuelles.

Exemple: Norbert Elias, "La civilisation des moeurs" qui analyse la naissance de la sensibilité occidentale (toutes les règles de conduite en société). Il étudie ce sujet en étudiant les manuels de savoir-vivre du 16e au 20e siècle. Au début, ces règles sont clairement imposés à partir de contraintes extérieurs, la contrainte sociale est forte et visible.

La contrainte devient, au bout d'un certain moment, intériorisé et ne se ressent donc plus, on en a plus conscience. 

\section{La sociologie comme démarche scientifique}

\subsection{Observer des faits sociaux comme des choses}

La sociologie est une science car elle a des règles d'observation des faits sociaux.

"La première règle et la plus fondamentale est de considérer les faits sociaux comme des choses[...]il nous faut considérer les phénomènes sociaux en eux-mêmes, détachés des sujets conscients qui se les représentent ; il faut les étudier du dehors comme des choses extérieures".

Les faits sociaux doivent être considérés comme des réalités dont nous n'avons pas connaissance avant de les étudier.

\subsection{Écarter les prénotions}
"Il faut que le sociologue[...]s'affranchisse des fausses évidences qui dominent l'esprit du vulgaire, qu'il secoue une fois pour toute le joug de ces catégories empiriques qu'une longue accoutumance finit par rendre tyrannique".

\subsection{Expliquer par la causalité}

Le social s'explique par le social: "Toutes les fois que, pour expliquer un fait social on se référera à des motivations individuelles, on pourra être assuré que l'explication est fausse".

"Le mariage n'est pas du à la peur de vivre seul, mais c'est parce que le mariage est une norme que l'on ressent de la solitude."

"La monogamie ne s'explique pas par la jalousie, mais c'est parce que l'on est monogame que l'on est jaloux". 

Pour expliquer le social par le social, la méthode, c'est le rationalisme expérimental. La première étape est de définir le fait social et on le caractérise par des faits extérieurs observables. Ensuite, on va chercher à établir des lois qui dérivent de l'observation empirique.

Expliquer de manière scientifique, c'est expliquer par des liens de causalités ce qui relie deux phénomènes. La méthode pour expliquer est l'expérimentation. Expérimenter c'est montrer que lorsqu'on produit un phénomène, il est cause du second.

Si l'on ne peut pas expérimenter les phénomènes que l'on souhaite observer, on lui substitue une autre méthode: la comparaison. Pour Durkheim, la comparaison porte sur les variations réciproques des phénomènes étudiés: "La méthode des variations concomitantes", montrer qu'il y a des corrélations donc. 

\section{L'étude du suicide}


Durkheim souhaite montrer que le suicide n'est pas quelque chose d'individuel mais relève plutôt du fait social. Aborder le suicide de cet angle là, permet de comprendre le phénomène du suicide et ses variations. 


La démarche de Durkheim va se faire en trois temps: il va déjà montrer qu'il existe des régularités statistiques qui montrent que le suicide n'est pas un phénomène imprévisible, les facteurs qui sont couramment évoqués à l'époque ont en fait un effet quasi nul. \\
Dans un second temps, il va montrer que d'autres facteurs sont beaucoup plus important comme la religion, le sexe, l'age, le lieu de résidence. \\
Dans un troisième temps, il va proposer une analyse théorique sur ces variations afin de les comprendre au regard de l'organisation sociale.


Durkheim va commencer par faire une collecte statistiques sur le suicide à partir des registres de l'état civil. \\
Ensuite, il va définir préalablement ce sur quoi il travaille, c'est à dire le suicide: "on appelle suicide tout cas de mort qui résulte directement ou indirectement d'un acte positif ou négatif accompli par la victime elle même et qu'elle savait devoir produire ce résultat. \\
Il s'attaque ensuite au coeur de la démonstration: il doit montrer que le suicide est un fait social. "Si on considère l'ensemble des suicides commis dans une société donnée pendant une unité de temps donnée, on constate que le total ainsi obtenu n'est pas une simple somme d'unités indépendantes, mais qu'il constitue par lui-même un fait nouveau et sui generis\footnote{Par lui même}, qui a son unité et son individualité, sa nature propre par conséquent, et que, de plus, cette nature est éminemment sociale." C'est à dire que, pris un à un, les suicides ne sont pas des faits sociaux, mais dans son ensemble, le phénomène est bel et bien social car:
\begin{itemize}
\item 1er constat: régularité des taux de suicide d'une année sur l'autre pour une même société
\item 2ème constat: les taux de suicide sont très différents d'une société à l'autre 
\end{itemize}

"Chaque société a donc, à chaque moment de son histoire, une aptitude définie pour le suicide." \\
S'ensuit une réfutation des théories antérieures, Durkheim montre que le suicide n'est pas lié au climat ou à la température, ni à l'hérédité familiale ou génétique. Il expliquera que c'est une question d'intensité de la vie sociale. Il réfute aussi le facteur imitation de Gabriel Tarde.


Durkheim va essayer de saisir statistiquement si le taux de suicide varie en fonction d'un certains nombre de facteurs comme le lieu de résidence, le sexe, la religion, le statut matrimonial. Il recherche un lien systématique entre des variables, il appelle ça des "variations concomitantes" que l'on appelle aujourd'hui corrélation. Pour cela, une relation doit être régulière et vérifiée dans son ensemble. \\
Il s'agit de vérifier si la relation entre le suicide et le sexe subsiste si on la considère à age égal, même lieu de résidence, même statut matrimonial etc. On fait ça pour toutes les autres variables.


Durkheim met donc en évidence que:
\begin{itemize}
\item Le taux de suicide s'accroît avec l'age
\item il est supérieur chez les hommes que chez les femmes
\item supérieur à Paris qu'en province
\item il varie également avec la religion, les protestants se suicident plus que les catholiques qui se suicident plus que les juifs
\item la famille protège du suicide, plus elle est nombreuse, moins le taux de suicide est élevé
\end{itemize}


\subsection{L'analyse théorique de ces causes sociales}

Il reste à Durkheim la tâche de proposer une explication générale du suicide. Après avoir mis en lumière plusieurs corrélations, il va pourvoir expliquer que: \\
Le taux de suicide varie en fonction du degré d'insertion de l'individu dans la communauté à laquelle il appartient.


Il isole donc trois types de suicide, le suicide égoïste par défaut d'intégration, le suicide altruiste (une forme de fanatisme, mais ce terme ne serai pas correct car porte un jugement de valeur) par excès d'intégration et le suicide anomique (défaut de règles, la société n'offre pas suffisamment de cadres intégrateurs). Ce que Durkheim appelle l'anomie, c'est un état de désorganisation conjoncturelle de la société causée par des bouleversements sociaux ou économique de grande ampleur. En clair, en crise économique, le taux de suicide est en forte augmentation, mais ce n'est pas la pauvreté qui favorise le suicide, mais le fait que la pauvreté s'abatte soudainement sur une société. \\
Le taux de suicide augmente aussi en période de prospérité économique. Le trait commun entre la période de prospérité et la période de crise, c'est le décalage conjoncturelle entre ce que l'on attend et ce que l'on a réellement. Ces deux périodes entraînent donc une dérégulation des règles sociales et donc des cadres intégrateurs d'où le nom d'anomie. \\ 
Le suicide est donc un symptôme de défaut d'intégration.


\subsection{Actualité des thèses de Durkheim}

Les données statistiques de Durkheim étaient imparfait de par les moyens de l'époque. On est donc aujourd'hui beaucoup plus précis. Mais la méthode de Durkheim reste une référence. \\
Aujourd'hui, au niveau du suicide, beaucoup de choses constatées par Durkheim restent vrais, comme la famille qui protège toujours du suicide. Cependant, désormais, la ville protège plus du suicide que la campagne.

\chapter{Max Weber et la sociologie compréhensive}

\section{Introduction}

La sociologie allemande est le deuxième grand foyer de la pensée sociologie allemande. George Zimmel, Max Weber, Ferdinand Tönnies, Werner Sombart sont des sociologues allemands qui ont contribué à la création du domaine. \\
En Allemagne, l'institution de la sociologie est moins importante qu'en France. Une association notable est celle de Weber et Tönnies.


Il y a trois grandes différences entre la sociologie allemande et la sociologie de Durkheim. En France, le projet est positiviste: créer une science sociale comme branche des sciences classiques. \\
En revanche, l'Allemagne est de tradition philosophique "dualiste" qui postule qu'il n'y a pas une science, mais deux: les sciences de la nature et les sciences de l'homme. Les spécificités sont que la sociologie allemande ne consiste pas à reproduire les méthodes de la science exacte ; l'étude de l'homme doit être contextualisée, c'est à dire qu'elle doit prendre en compte les contextes particuliers ; la sociologie ne peut se fonder exclusivement sur une observation extérieur des actions humaines, c'est à dire que l'homme a un but, et qu'il donne une signification à ses actions, donc la sociologie ne peut être objectif. \\
La sociologie Allemande ouvre une voie alternative à l'étude du social où il donne plus d'importance à l'acteur et à son ambition. \\
On oppose souvent Durkheim et Weber, mais le but reste le même, fonder des méthodes de compréhension de l'homme: les sciences sociales.


\section{Éléments biographiques}

Il est né en 1864 dans une famille de la bourgeoisie intellectuelle de confession calviniste. \\
Il va étudier le droit, l'histoire, la philosophie et l'économie. Il passera son doctorat en 1889. \\
Il sera professeur d'économie en 1896. \\
En 1903 il fonde une revue de science sociale et publiera en 1905 "L'éthique protestante et l'esprit du capitalisme" ; en parallèle il publie des essais sur les sciences sociales \\
Il s'engage en politique en fondant le parti démocratique allemand. Il sera d'abord favorable à la grande guerre, mais deviendra petit à petit pacifiste. De par son engagement en politique, il sera expert pour négocier le traité de Versailles. \\
Il obtient ensuite une chaire de sociologie à Munich, commencera une oeuvre qui restera inachevé car il meurt en 1920.


Les oeuvres de Weber:
\begin{itemize}
\item Économie et société
\item L'éthique protestante et l'esprit du capitalisme
\item Essaie sur la théorie de la science
\item Le savant et le politique
\end{itemize}


Weber fonde une sociologie de la compréhension, c'est à dire où l'acteur est au centre de l'étude et non la société. Il va accordé aux croyances, aux représentations, bref à la subjectivité, une place fondamentale.


\section{Comprendre et interpréter l'activité sociale}

\subsection{L'activité sociale}

Weber comme Durkheim tente d'isolé l'objet de la sociologie, quels phénomènes peuvent être étudiés par la sociologie. Sa meilleure réponse se trouve dans "Économie et Société" où Weber décrit sa démarche.

% Récupérer la citation ; si pas de diapos en ligne: Économie et société p 28 "Nous appellerons sociologie etc."

Il isole trois critères:
\begin{itemize}
\item Comportement des agents sociaux
\item Mus par du sens, des intentions
\item Orientés vers et en fonction d'autrui
\end{itemize}

Les actes qui fondent donc la sociologie pour Weber, ce sont les actes/comportement auquel les individus (les agents) donnent un sens par une conscience des acteurs ou par les habitudes/coutumes et dans ce cas, cela revêt un sens particulier. \\
Ces actions doivent être orientés relationnellement pour que cela soit de l'activité sociale.

\subsection{Comprendre pour expliquer}

L'objectif est de rendre compte des comportements humains en s'attachant à comprendre les intentions des acteurs et le sens qu'ils donnent à leurs actes ou les valeurs qui les motivent. Prendre en compte la subjectivité ne veut pas dire qu'on renonce à toute prétention scientifique. \\
Il le fait en interprétant: c'est à dire "organiser en concepts le sens subjectif" des agents. C'est à dire construire des modèles qui valent pour des groupes ou des catégories d'individus. 


Exemple: la typologie des logiques de l'action
\begin{itemize}
\item L'action rationnelle en finalité
\item L'action rationnelle en valeur
\item L'action traditionnelle
\item L'action affective
\end{itemize}
Weber formule des idéaux-types: c'est à dire des modèles de comportement.


\subsection{Une science modeste}

La réalité sociale est inépuisable et les outils dont on dispose pour en rendre scientifiquement compte sont limités. Il n'existe pas de possibilité de représentation théorique globale. Les concepts théoriques ne peuvent être totalement adéquats à la réalité qu'ils sont sensés représenter. \\
Voir la définition de l'idéal-type: c'est un système ou modèle de traits considérés comme essentiels et susceptibles de rendre intelligible une réalité qui est forcément plus complexe. \\
On ne peut prévoir le sens de l'histoire à partir d'une représentation théorique.


L'objectivité du chercheur ne peut être que relative. Ce chercheur vivant lui même dans la société, il y aura toujours un jugement de valeur: une prise de position en termes de bien ou de mal. Il y a aussi un rapport aux valeurs: tout ce qui oriente la démarche scientifique, à savoir les présupposés qui fondent sa curiosité. \\
L'objectivité absolue n'est pas possible, il ne peut y avoir qu'une objectivité relative, pour peu que le chercheur respecte la neutralité axiologique.

\section{L'éthique protestante et l'esprit du capitalisme}

L'enjeu de cet ouvrage est de montrer en quoi la foi protestante a permis le développement du capitalisme. Marx et Hegel avaient déjà observé ce lien, mais dans l'autre sens, où c'était l'économie qui expliquait la religion. Le coup de force de Weber est donc d'inverser la relation de causalité.


Weber définit d'abord ce qu'il entend par capitalisme et Église protestante (corps constitué en vue de la grâce qui administre les biens religieux du salut). La secte en revanche est une communauté de plus petite dimension, qui accepte les grands principes d'une religion mais qui affirme une spécificité sur un point de doctrine particulier. \\
Le capitalisme est un système économique centré sur la recherche pacifique et rationnelle du profit.

Il a trois points de démonstration: \\
Il démontre une corrélation statistique: le capitalisme s'est implanté le plus précocement et est le plus développé dans les zones de force de la religion réformé. \\
Définir les croyances qu'il étudie, quelles sont les dispositions d'esprits nécessaire au développement du capitalisme. L'esprit capitaliste renvoie à l'idée de profit, il doit être souhaité et souhaitable, mais c'est aussi la propension à calculer et à penser rationnellement l'activité économique. \\
Montrer comment certaines croyances protestantes (en particulier celles des calvinistes) ont pu favoriser le développement du capitalisme en favorisant l'émergence de ces dispositions d'esprit.

\chapter{L'école de Chicago}
\section{La naissance de la sociologie américaine: sociologie empirique}

\subsection{3 Spécificités de cette sociologie naissantes}

\begin{itemize}
\item Fort et rapide ancrage et développement universitaire \\
1894: Création de l'université de Columbia ; 1892: Université de Chicago ; 1919: 850 sociologues ; 1929: environ 2850
\item Discipline empirique (primat à la collecte des faits sur la théorie, forte influence de l'ethnographie)
\item Un soucis de réformisme social (racisme, violence etc.)
\end{itemize}

\subsection{Naissance et développement d'une sociologie empirique}

Les quartiers de Chicago étaient séparés par origine (sans que ce soit des ghettos), mais après la seconde guerre mondiale et l'arrivée d'affro-américain, les ghettos ont commencés à apparaître.

1910-1920: W. Thomas sort "The Polish peasant"


\subsection{Critiques d'une tradition et discussion philosophique}

À partir des années 1930: critique des méthodes de l'école de Chicago, beaucoup de données collectées mais peu de théorisation, peu de généralisation. 

À partir de la fin de la seconde guerre mondiale, début de la seconde école de Chicago qui se renouvelle par rapport à l'ancienne, avec beaucoup d'entretiens. 

% Gauffman et Becker

\subsection{Gauffman et la prise en compte des interactions}

\subsubsection{Éléments de biographie et bibliographie}
1922: naissance au Canada ; il rejoint l'université de Chicago en 1945 où il fera une thèse sur les îles Shettland où il va en 1949. Gauffman vit en ermite, travaille en ermite, car il considère qu'il faut vivre comme les enquêtés. Il soutient sa thèse en 1953 et donne lieu à la publication d'un ouvrage: "La présentation de soi dans la vie quotidienne". \\
En 1956, il va travailler pendant un an dans un hôpital psychiatrique en faisant une enquête incognito. Il publiera "Asile". \\
Il fera de nombreuses études dont une sur les casinos.

\subsubsection{Interaction comme objet de la sociologie}

Gauffman considère l'interaction en face à face comme un authentique objet sociologique, car pour lui, lors d'un face à face, il y a une certaine autonomie par rapport à la structure sociale. \\ 
Il essaye de trouver des règles générales qui ordonne les situations d'interaction de face à face. Il va dégager certains principes généraux. \\
Dès lors que nous sommes en présence de quelqu'un d'autre, on est sous son regard, notre comportement a une signification que l'autre interprète, on transmet une image de soi. C'est une sociologie compréhensive, la question de l'interprétation est centrale. Les interactants partagent un sens commun. 

\subsubsection{Anatomie de l'interaction de face à face}

Deux ouvrages notables de Gauffman: "La présentation de soi dans la vie quotidienne" ; "Les rites d'interactions".


Dans le premier, il présente un modèle théâtrale. Il va se demander dans quel mesure l'acteur social joue un rôle. Le point de départ étant la métaphore du théâtre avec l'acteur et son rôle. Il va donc essayer de caractériser les modalités de ce jeu social qu'est l'interaction. \\
Quels sont les contraintes, les impératifs qui s'imposent aux acteurs dans ses relations de face à face ? \\ 
"On peut supposer que toute personne placé en présence des autres a de multiples raisons d'essayer de contrôler l'impression qu'il reçoive de la situation. On s'intéresse ici à certaines des techniques couramment employées pour produire ces impressions, et à certaines des circonstances le plus souvent associées à ces techniques". "On s'occupera uniquement aux problèmes dramaturgiques qui s'imposent aux acteurs". \\
En clair, il faut être crédible dans le rôle qu'on tient lors d'une interaction. "Il s'agit pour un acteur de donner l'impression qu'il a toujours possédé son autorité et sa compétence actuelle et qu'il n'a jamais eu à tâtonné tout au long d'une période d'apprentissage". \\
Pour être crédible dans son rôle, l'acteur a des instruments symboliques, assez diversifiés, ce que Gauffman appelle la façade personnelle: manière de parler, de s'habiller, gestuelle etc. La combinaison de toutes ces ressources symboliques produit des impressions qui ont un impératif
commun: éviter les dissonances. Ce travail sur les impressions est de l'ordre du charme, il ne faut pas rompre la cohérence de la représentation. Une seule note dissonante peut rendre une interaction complètement fausse "une seule fausse note peut provoquer une rupture de thon qui affecte la représentation toute entière".


"Les rites d'interaction" s'intéresse plus au contenu des interaction alors que "la présentation de soi" s'intéressais davantage à la forme. Lors d'une interaction de face à face, le principe fondamental est de maintenir la face: la valeur sociale positive qu'une personne revendique à travers la ligne d'action que les autres supposent qu'elle a adopté au cours d'un contact particulier ; l'identité que chaque personne doit adopté pour se conformé à l'attente des autres. \\
L'impératif dans chaque situation d'interaction c'est de préservé ce qu'on est censé être aux yeux des autres, maintenir une stabilité symbolique. Pour la face individuelle, il s'agit d'amour propre (tenue), pour autrui, de la considération (déférence). \\
Si Gauffman parle de rite d'interaction, c'est parce que ce travail de maintien de la face est codifié. Ces manières de se comporté dans les rôles sont définis dans le collectif. Sortir de ces rôles, manquer de tenue, de considération sont de l'ordre du sacrilège. 

\subsubsection{Becker et ses conceptualisations}

D'après Becker, dans "Outsider: étude de la sociologie de la déviance", la déviance est une construction sociale par laquelle certains individus deviennent déviants car qualifiés comme tel par les autres. La déviance n'est pas une qualité d'un fait, mais une désignation comme tel par les autres. \\
C'est une catégorie qui se construit au cours des interactions entre ceux qui sont qualifiés de déviants et ceux qui se chargent de faire respecter les normes. "Les groupes sociaux créent la déviance en instituant des normes dont la transgression constitue la déviance, en appliquant ces normes à certains individus, et en les étiquetant comme des déviants" Le déviant est celui à qui on a collé une étiquette de déviant. \\
L'étiquetage est un processus interactif. Ce sont certains groupes sociaux, des "entrepreneurs de moral" qui créent la déviance en instituant des normes dont la transgression constitue la déviance. \\
On va être amené à être caractérisé comme déviant (ou non) au cours d'une "carrière".


Becker se propose de s'intéressé aux activités qui conduisent à un classement comme "fumeur" ou "déviant". Il s'intéresse aux formes d'activités addictives auxquelles s'adonne les individus et à leurs dynamiques. C'est leur "carrière". Il cherche à démontrer que la déviance est un processus. \\
La déviance a donc des phases de changements, la carrière déviante, qui a plusieurs étapes. \\
Becker distingue quatre phases de la carrière déviante:
\begin{enumerate}
\item la transgression occasionelle de la norme ;
\item l'engagement (nouveau mode de vie, changement d'identité, sous-culture organisée autour d'une activité déviante) ;
\item la désignation publique (le moment où on est reconnu publiquement comme déviant): conséquences importantes sur la vie sociale et l'image de soi "la manière dont on traite les déviants équivaut à leur refuser les moyens ordinaires d'accomplir les activités routinières de leur vie quotidienne. En raison de ce refus, le déviant doit mettre en oeuvre des pratiques routinières illégitimes." ;
\item l'adhésion à un groupe déviant, dans le but de légitimer leurs pratiques déviantes, afin de penser positivement sa différence. 
\end{enumerate}
Cette notion de carrière a de nombreux avantages. \\ 
Le caractère séquentiel de cette notion permet de penser la dynamique, la manière dont chaque séquence joue sur les autres. Cela permet d'envisager la déviance comme le résultat d'une histoire et non d'une cause. \\
Révéler le travail effectué pour devenir quelque chose. \\
Appréhender la dimension structurelle des interactions en jeux dans l'engagement, articuler les logiques individuelles et institutionnelle. La question de l'apprentissage et de la socialisation est importante.




\end{document}
