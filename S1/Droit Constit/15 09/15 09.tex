\documentclass[12pt, a4paper, openany]{book}

\usepackage[utf8]{inputenc}
\usepackage[T1]{fontenc}
\usepackage[francais]{babel}
\author{Jérémy B.}
\date{}
\title{Cours de droit constitutionnel (UFR Amiens)}
\pagestyle{plain}

\begin{document}
\maketitle

\chapter{Introduction générale}

	\section{Qu'est-ce que le droit constitutionnel ?}

L'expression est apparue en France aux alentours de 1775-1777. 

Le terme droit est polysémique, il désigne:
\begin{itemize}
\item soit une faculté d'accomplir une action garanti par une norme juridique
\item soit un ensemble de normes juridiques
\item soit la discipline qui étudie les normes juridiques
\end{itemize}

L'adjectif constitutionnel exprime l'idée que le droit constitutionnel est un droit relatif à la constitution.

Deux approches de la constitution entendue comme norme juridique:
\begin{itemize}
\item Approche matérielle: la constitution désigne l'ensemble de normes juridiques caractérisées par leur contenu, à savoir les institutions politiques et les relations entre les gouvernés et les gouvernants.
\item Approche formelle: la constitution désigne l'ensemble de normes juridiques caractérisées par leur place dans l'ordre juridique, à savoir qu'elles se situent au plus haut niveau de cet ordre et qu'elles ne peuvent être modifiées que par des normes du même niveau.
\end{itemize}

	\section{Le droit constitutionnel comme "l'ensemble de normes juridiques relatives au pouvoir politique"}

		\subsection{Un ensemble de normes juridiques...}

La norme est la signification d'un énoncé prescrivant un modèle de conduite. Celà implique que la norme est le sens de l'énoncé, le sens que l'on cherche à donné à un mot. Ce sens prescrit une action, elle oriente les conduites, elle ne les décrit pas.

La spécificité de la norme juridique réside dans la source de l'émission de la norme : la norme juridique est celle adoptée, directement ou indirectement par l'État. On parle à cet égard de "droit positif" pour désigner l'ensemble des normes juridiques en vigueur dans un État.

La norme juridiqque est posée directement par l'État lorsqu'elle est adoptée par une institution politique habilité par la constitution. Ex: La loi est la norme juridique adoptée par le Parlement.

La norme juridique est posée indirectement par l'État lorsqu'elle est adoptée par une ou plusieurs personnes privées qui ont été habilitées par une norme de l'État. Ex: "Les conventions légalement formées tiennent lieu de loi à ceux qui les ont faites" (Art 1134 du Code Civil)

		\subsection{...relatives au pouvoir politique}

On peut définir le pouvoir politique comme le phénomène qui lie les gouvernés aux gouvernants pour la conduite des affaires publiques.
\begin{itemize}
\item Le pouvoir politique est la source de normes juridiques.
\item Le pouvoir politique est, en principe, soumis aux normes juridiques.
\ned{itemize}

	\section{Le droit constitutionnel comme une discipline étudiant les normes}

		\subsection{La recherche}

Les acteurs dans la recherche en droit constitutionnel sont:
\begin{itemize}
\item Les universitaires (prrofesseurs et maîtres de conférences en droit public, docteurs en droit public et étudiants en thèse de doctorat de droit public,...)
\item Les non universitaires (juges constitutionnels et administratifs, personnels politiques, hauts fonctionnaires, avocats...)
\end{itemize}

La "doctrine" désigne la pensée et l'ensemble des auteurs ayant un regard savant sur le droit.

Il y a plusieurs discours sur le droit constitutionnel:
\begin{itemize}
\item Le discours du personnel politique et des militants ;
\item Le discours des philosophes ;
\item Le discours des médias.
\end{itemize}

Le discours savant sur le droit constitutionnel a une utilité spécifique dans la société: sa logique est celle de l'élucidation. Il vise à permettre une meilleure connaissance et compréhension des interactions entre le droit et le pouvoir politique.

Il n'y a pas une seule façon de faire de la recherche. Il y a une hétérogénéité de la recherche.
Les normes juridiques sont la traduction d'un choix politique qui va régir, de façon impérative, l'ensemble de la population (y compris les membres de la doctrine). En tant qu'individu, chaque personne a des convictions politiques et morales. La question se pose alors au chercheur de savoir s'il doit mettre de côté ses convictions personnelles dans l'accomplissement de son travail.

L'approche positiviste repose sur l'idée d'une séparation aussi rigoureuse que possible entre l'analyse savante et le jugement de valeur (neutralité axiologique).

Cette posture du chercheur, qui apparait comme un observateur extérieur, signifie que la science juridique doit se borner à analyser les phénomènes juridiques tels qu'ils sont, et non tels à prescrire ce qu'ils devraient être.

L'approche dogmatique renvoie à une étude savante du droit visant à trouver la solution souhaitable, au regard de valeurs prédéterminées. 

Cette approche refuse l'idée que le chercheur en droit soit un observateur extérieur. Elle l'érige en acteur du droit, qui participe à son élaboration et à son application.

	\section{L'enseignement du droit constitutionnel}

Sous l'ancien régime, aucun cours de droit constitutionnel n'existe. On peut relever deux principales explications:
\begin{itemize}
\item Le droit constitutionnel moderne, comme ensemble de normes juridiques, est véritablement né à la fin du 18e siècle, aux États-Unis puis en France et en Pologne
\item Les autorités politiques craignent que l'enseignement du droit constitutionnel n'incite à la critique du régime, et, in fine, à l'insurrection.
\end{itemize}

Le premier cours de droit constitutionnel a été créé en 1834 en France à la Faculté de droit de Paris.

Le cours est confié à un professeur d'origine italienne Pellegrino Rossi\footnote{Premier cours de droit constitutionnel en Italie créé en 1797 (université de Ferrare)}

La fin du Second Empire marque la renaissance de la chaire de droit constitutionnel, en 1871, à l'initiative des républicains. Mais, comme au temps de Rossi, le cours n'est dispensé qu'aux élèves de doctorat et seulement en option jusqu'en 1882.

Le 24 Juillet 1889 est créé en première année de licence un cours semestriel obligatoire d'Element de droit constitutionnel et organisation des pouvoir publics, qui sera confié, à la Faculté de droit de Paris, au professeur Adhémar Esmein.

	\section{Organisation du cours}
		\subsection{Théorie}
La théorie étudie les concepts fondamentaux et généraux du droit (savoir théorique). Ex: l'Etat, la constitution, le régime parlementaire, les droits fondamentaux...

		\subsection{Pratique}
La pratique étudie directement le droit applicable à un ordre juridique déterminé (savoir pratique). Ex: le droit français, le droit américain, le droit allemand, le droit chinois...




\part{Théorie Constitutionnelle}

\chapter{L'État}

Il y a deux questions à se poser pour définir l'État:
\begin{itemize}
\item La première est celle qui porte les juristes, les politistes ou les décideurs publics à savoir si telle ou telle entité politique constitue un État. Cette question est celle des faits générateurs ou éléments constitutifs de l'État. \newline

Traditionnellement, l'État s'entend comme "un groupement humain, fixé sur un territoire déterminé et sur lequel une autorité politique exclusive s'exerce" (Gicquel, Droit constitutionnel et...) \newline % Retrouver la référence exacte

Éléments constitutifs de l'État:
\begin{itemize}
\item Territoire
\item Puissance publique
\item Population
\end{itemize}


\item La deuxième question est celle de nature ou de l'essence juridique de l'État, des attributs juridiques de cette forme politico-juridique qu'est l'État. Cette question renvoie au concept juridique de l'État.

	\section{La structure de l'État}

L'État est un ordre juridique autonome doté de la personnalité juridique. % À la place d'autonome : autarcie ? Aristote...

		\subsection{Les attributs de l'État}
		
		
		

\end{document}
