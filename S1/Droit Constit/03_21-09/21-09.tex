\documentclass[12pt, a4paper, openany]{book}

\usepackage[latin1]{inputenc}
\usepackage[T1]{fontenc}
\usepackage[francais]{babel}
\author{Jérémy B.}
\date{}
\title{Cours de Droit Constitutionnel (UFR Amiens)}
\pagestyle{plain}

\begin{document}
\maketitle

\subsection{Les attributs de l'État}

L'État est un ordre juridique autonome doté de la personnalité juridique.


Hans Kelsen, juriste autrichien est un théoricien du droit. Il est le père de l'approche positiviste du droit. Il est connu aussi pour l'écriture de la constitution Autrichienne de 1920 qui est une des premières à créer une cour constitutionnelle. 

Pour Kelsen "l'État est un ordre juridique relativement centralisé, limité dans son domaine de validité spatiale et temporelle".

Kelsen défend l'idée que l'État et le droit ne font qu'un. Dès lors, définir l'État, c'est définir le droit, ou plus exactement l'ordre juridique (c'est à dire l'ensemble des normes juridiques en vigueur).

Selon Kelsen, les 3 éléments constitutifs de l'État ne peuvent être définis que par le droit:
\begin{itemize}
\item la population est l'ensemble des individus soumis aux normes appartenant à l'ordre juridique ;
\item le territoire est l'espace sur lequel ces normes sont applicables
\item la puissance publique est celle qui s'exerce sur le fondement et à l'aide de ces normes.
\end{itemize}

Il y a deux types d'ordre juridique pour Kelsen::

\begin{itemize}
\item OJ décentralisés dans lesquels ce ne sont pas des organes spécialisés qui créent et appliquent les normes juridiques mais les sujets eux mêmes\footnote{Comme la société internationale (cf le cours de relations internationales} ;
\item OJ centralisés, dans lesquels l'ordre juridique institue des organes spécialisés pour créer et appliquer les normes juridiques.
\end{itemize}

L'ordre juridique étatique est hiérarchisé: une norme juridique n'est valide que si elle est conforme à la norme ou aux normes juridiques qui lui sont supérieures.


De façon générale, la notion de personne morale est une construction juridique  destinée à prendre en charge de façon permanente les intérêts d'un groupe humain indépendamment des personnes physiques qui agissent en son nom.

L'État est "la personnification juridique d'une Nation" Adhémar Esmein, Éléments de droit constitutionnel français et comparé.

La nation renvoie à un groupement humain dans lequel les individus se sentent unis les uns aux autres par des liens, à la fois matériels et spirituels, et se conçoivent différents des individus qui composent les autres groupements nationaux. 

L'État se définit comme "l'être de droit en qui se résume abstraitement la collectivité nationale ou la personnification de cette dernière" Raymond Carré de Malberg, Contribution à la théorie générale de l'État, 1920.

L'idée de personnalité juridique de l'État permet de remplir plusieurs fonctions:
\begin{itemize}
\item elle assure l'unité de l'État
Assure l'unité face à l'hétérogénéité des individus qui composent la Nation. En outre, elle permet d'expliquer que l'État dispose d'une volonté unique, distincte des individus.
\item elle assure la continuité de l'État face à l'alternance des régimes politiques et l'évolution de la population
\item elle assure l'autonomie de l'État en permettant l'existence de moyens propre à l'État, puisque, en tant que personne juridique, il dispose d'un patrimoine propre dont le budget constitue l'élément essentiel.
\end{itemize}


La notion de souveraineté est apparue à partie du 16e siècle dans les travaux d'auteurs portant sur le concept d'État:
\begin{itemize}
\item Jean Bodin, Les six livres de la République, 1576
\item Thomas Hobbes, Le Léviathan, 1651
\end{itemize}

"La souveraineté est la puissance absolue et perpétuelle d'une République" Jean Bodin.

Pour la doctrine allemande, par exemple Georg Jellinek (1851-1911), être souverain, c'est avoir la "compétence de sa compétence".

Problématique: la souveraineté est-elle un critère de définition de l'État ? Autrement dit, peut-il exister des États non souverain ?

Autonomie de l'État: La caractéristique essentielle de la puissance étatique réside dans son pouvoir d'auto-organisation, autrement dit dans son pouvoir d'adopter une constitution qui va habiliter les institutions politiques (Staatsgewalt).

Par exemple, la Californie est un État car cette collectivité est dotée de sa propre constitution et a choisi ses institutions politiques (Gouverneur de Californie, Chambre de représentants de Californie, Sénat de Californie, Cour Suprême de Californie, etc...).

En revanche, les départements français n'ont pas de constitutions, leurs institutions sont prévus par les lois nationales, ce ne sont donc pas des États.

Pour le cas de la Catalogne, ce n'est pas un État car elle ne dispose que d'un statut dont l'existence est prévue par la constitution espagnole et qui est soumis au contrôle préalable du Tribunal constitutionnel espagnol.

Il existe dont des États souverains (France, Allemagne, États-Unis...) et des États non souverains (Californie, Land de Bavière, Canton de Vaud).

		\subsection{Les formes de l'État}

La confédération n'est pas un État: c'est une association d'États qui demeurent indépendants qui ont, par traité, délégué l'exercice de certaines compétences à une institution commune (Confédération Américaine de 1777-1787).

Attention, la Confédération Helvétique (Suisse) n'est pas une confédération au sens de la théorie constitutionnelle.


Il existe donc deux formes d'État:
\begin{itemize}
\item L'État unitaire (France, Royaume-Uni...)
\item L'État fédéral (États-Unis, Allemagne, Suisse, Inde, Canada)
\end{itemize}

L'État unitaire en comprend un seul: il a une seule constitution habilitant l'ensemble des pouvoirs nationaux et locaux. L'État unitaire n'a qu'une seule structure étatique qui peut être centralisé ou décentralisé.

L'État fédéral est un État composé de plusieurs États:
\begin{itemize}
\item La fédération qui est la structure étatique nationale
\item Les États fédérés qui sont les structures étatiques locales
\end{itemize}

L'État fédéral se caractérise ainsi par une pluralité de constitutions (la constitution fédérale de la Fédération et les constitutions locales des entités fédérées).

À la différence de la confédération, l'État fédéral est un véritable État car la Fédération dispose d'une capacité d'auto-organisation politique (autonomie). En plus, c'est un État souverain.


Dans l'État unitaire centralisé, toutes les normes juridiques sont prises par des autorités nationales, dites aussi centrales.

Dans l'État unitaire décentralisé, les normes locales sont émises par des collectivités locales, juridiquement distinctes de l'État, et dotées d'organes propres, en règle générale élus par les citoyens concernés.

La décentralisation se définit comme le transfert de compétences de l'État vers des autorités locales dites décentralisés, c'est à dire des entités non étatiques dotées d'une personnalité juridique distincte de celle de l'État.

La décentralisation ne doit pas être confondue avec la déconcentration, qui est un processus d'aménagement territorial du pouvoir consistant à implanter dans des circonscriptions administratives locales des autorités représentant le pouvoir.

La déconcentration existe aussi bien dans les États unitaires (comme le préfet en France) et dans les États fédéraux.


Il existe 4 principes pour caractériser l'État fédéral:
\begin{itemize}
\item Le principe de supériorité du droit fédéral sur le droit fédéré\footnote{Ce principe n'est pas propre à l'État fédéral}
\item Le principe de séparation constitutionnelle des compétences\footnote{Il n'est pas non plus propre à l'État fédéral}
\item Le principe de participation des États fédérés à l'exercice du pouvoir fédéral
Il y a la participation organique qui se réalise, en principe par le moyen d'une seconde chambre au sein du Parlement fédéral qui représente spécialement les États fédérés et dont la composition assure la prise en considération des intérêts des États fédérés (ex: Sénat aux États-Unis, Bundesrat en Allemagne).

Mais il y a aussi la participation normative qui se réalise par la participation des États fédérés à la procédure législative et à la procédure de révision de la Constitution fédérale (Aux États-Unis, pour que la constitution soit modifié, il faut la majorité du congrès aux trois quart).

\item Le principe d'autonomie constitutionnelle des États fédérés
\end{itemize}

\section{Les organes de l'État}



\end{document}
