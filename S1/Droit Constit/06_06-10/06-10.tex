\documentclass[12pt, a4paper, openany]{book}

\usepackage[latin1]{inputenc}
\usepackage[T1]{fontenc}
\usepackage[francais]{babel}
\author{Jérémy B.}
\date{}
\title{Cours de Droit Constitutionnel (UFR Amiens)}
\pagestyle{plain}

\begin{document}

\section{Les rapports institutionnels}

"J'appelle faculté de statuer, le droit d'ordonner par soi même, ou de corriger ce qui a été ordonné par un autre. J'appelle faculté d'empêcher, le droit de rendre nulle une résolution prise par quelque autre" Montesquieu, De l'esprit des lois, 1748. \\
Nous distinguerons les rapports de sujétion et les rapports de collaboration. Seuls les rapports juridiques entre les organes de l'État, c'est à dire ceux qui s'exercent sur le fondement d'une norme et par le moyen d'un pouvoir normatif, seront pris en considération.

\subsection{Les rapports de sujétion}

Sujétion: "état de celui/ce qui est assujetti à quelqu'un ou à quelque chose". Il s'agit donc d'un état de dépendance, de soumission plus ou moins forte. Le mot soumission n'est cependant pas adapté car trop fort. \\
Une institution est assujettie à une autre institution si cette dernière peut agir sur elle.

\subsubsection{L'action sur la structure de l'institution}

L'action sur la structure d'une institution regroupe trois cas possible:
\begin{itemize}
\item l'action sur l'existence d'une institution
\item l'action sur le statut de l'institution
\item l'action sur les ressources de l'institution
\end{itemize}

\subsubsection{L'action sur la composition de l'institution}

Cette action consiste à avoir la main sur la composition du personnel d'une institution. C'est à dire pouvoir désigner ou révoquer la ou les personnes physiques qui la compose. \\
La portée du pouvoir de révocation est beaucoup plus forte que celle du pouvoir de nomination car le statut de l'organe peut rendre ses membres indépendants après la nomination. \\
Le pouvoir de révocation est un pouvoir juridique, c'est à dire qu'il s'exerce en application d'une norme juridique et par le moyen d'une norme juridique. Il ne doit pas être confondu avec un pouvoir factuel, une pression politique ou une influence sociale.

\subsubsection{L'action sur les actes de l'institution}

Il s'agit de l'hypothèse dans laquelle une institution peut modifier ou supprimer un acte juridique adopté par une autre institution.


\subsection{Rapports de collaboration}

Une institution collabore avec une autre institution lorsqu'elles accomplissent une tache commune que l'une ne peut pas réaliser sans l'intervention ou l'accord de l'autre. \\
Il existe deux types de rapports de collaboration: la collaboration entre les institutions politique porte sur l'exercice des fonctions législatives, exécutive, et juridictionnelle ; la collaboration entre les institutions porte sur l'exercice des rapports de sujétions à l'encontre d'une tierce institution.


\part{La constitution}

Le mot "constitution" débute par une majuscule lorsqu'il désigne la constitution d'un État en particulier ("la Constitution tunisienne de 2014") mais pas lorsqu'il désigne le concept général de constitution.


L'idée de la constitution remonte très loin dans l'histoire, en Grèce antique notamment où Aristote a écrit "Constitution d'Athènes", environ 325 avant J.C. \\
C'est au besoin de fixer par l'écriture l'état des relations entre gouvernants et gouvernés que correspondent les premiers documents politiques de caractère constitutionnel en Europe, tels que la Grande Charte (Magma Carta) de 1215 en Angleterre. \\
Il faut attendre la fin du 18e siècle pour assister à l'apparition des premières constitutions écrites modernes, c'est à dire des premières normes juridiques intitulés "Constitution" dont l'objet était d'organiser les institutions politiques et de limiter l'exercice du pouvoir. \\
Les premières constitutions moderne sont:
\begin{itemize}
\item Constitution de Virginie (1776)
\item Constitution des États-Unis d'Amérique (1787)
\item Constitution de Pologne (mai 1791)
\item Constitution de la France (septembre 1791)
\end{itemize}


Il existe deux conceptions de l'idée de la constitution. Il y a la constitution, entendue comme un ensemble de faits politiques, désigne les pratiques des acteurs politiques observables dans un pays donné. La constitution, entendue comme un ensemble de normes juridiques, désigne les normes juridiques qui sont soit matériellement soit formellement constitutionnelles.


Constitution matérielle: la constitution désigne l'ensemble des normes caractérisées par leur contenu, à savoir les institutions politiques et les relations entre les gouvernés et les gouvernants. \\
Constitution formelle: la constitution désigne l'ensemble de normes caractérisées par leur place dans l'ordre juridique, à savoir qu'elles se situent au plus haut niveau de cet ordre et qu'elles ne peuvent être modifiées que par des normes du même niveau.


\chapter{La constitution formelle}

Au sens formel, sont constitutionnelles toutes les normes, quel que soit leur objet, qui sont énoncées dans la forme constitutionnelle: elles sont en général contenues dans un document spécial mais, surtout, elles ont une valeur supérieure à celle de toutes les autres normes et ne peuvent être modifiées que conformément à une procédure spéciale.

\section{L'élaboration de la constitution}

L'élaboration de la constitution renvoie à deux interventions possibles du pouvoir constituant: adopter une nouvelle constitution ; modifier, amender, ou réviser la constitution en vigueur. \\
Le pouvoir constituant est le pouvoir d'adopter les normes. \\
Le pouvoir constituant est dit originaire lorsqu'il a pour objet l'adoption d'une nouvelle constitution. \\
Le pouvoir constituant est dit dérivé lorsque son objet consiste à modifier ou réviser ou amender la constitution en vigueur. 

\subsection{L'adoption de la nouvelle constitution}

\subsubsection{L'origine de l'adoption d'une constitution}

Il existe de nombreuses raisons pour un pays d'adopter une nouvelle constitution. Mais dans les plus fréquentes, nous avons les révolutions, pacifiques ou violente (Portugal en 1974-1976) ; une crise politique majeure (France en 1958) ; un coup d'État (France en décembre 1851) ; la création d'un nouvel État, souverain (Tunisie en 1956-1959) ou non (la République de Crimée en 2014).

\subsubsection{Le processus d'adoption d'une constitution}

Le processus d'adoption d'une constitution est juridiquement libre, dans le sens où il n'est soumis à aucune règle juridique préexistante. On dit que le pouvoir constituant originaire est souverain. \\
Il existe tout de même deux grands processus: un processus non démocratique d'adoption d'une nouvelle constitution: la constitution octroyée à la population. Processus démocratique d'adoption d'une nouvelle constitution: la constitution consentie, directement ou indirectement par le corps électoral.


Dans un processus démocratique, on a deux étapes, la préparation d'abord, où l'on écrit la constitution. La constitution est écrite soit par un organe élu par le corps électoral (1946) soit par un organe non élu par le corps électoral mais désigné par une assemblé qui lui, l'a été (1958).\\
L'adoption peut se faire de deux façons, soit directement par le corps électoral par le référendum, soit indirectement par un organe élu par le corps électoral, qui prend, du coup, le nom d'assemblée constituante (comme la constitution allemande du 31 Juillet 1919).


\subsection{La révision de la constitution en vigueur}

Le constituant a deux exigences contradictoires: Assurer la stabilité et la protection de la constitution et ne pas figer la constitution et permettre son évolution dans le temps. \\
Le pouvoir de révision ou pouvoir constituant dérivé est un pouvoir institué par le pouvoir constituant originaire: dès lors, il n'a pas la même autorité que lui et doit lui être subordonné...mais la subordination du pouvoir constituant dérivé au pouvoir constituant originaire n'est pas la même que la subordination du pouvoir législatif au pouvoir constituant car: le pouvoir législatif ne peut pas modifier la constitution ; le pouvoir de révision peut modifier la constitution (c'est sa raison d'être). \\
Double nature de la constitution initiale par rapport au pouvoir de révision: la constitution initiale habilite le pouvoir de révision mais elle entend le limiter pour se protéger.

\subsubsection{L'habilitation constitutionnelle du pouvoir de révision}

Parfois, la constitution ne prévoit pas de procédure de révision (Charte du 4 Juin 1814). On donc se demander: que déduire du silence de la constitution ? \\
Soit on estime que la révision est impossible, seule l'adoption d'une nouvelle constitution est envisageable. Soit on estime que la révision est possible, mais qu'elle ne peut être réalisée que par l'organe doté du pouvoir constituant originaire. Soit on estime que la révision est possible et que les organes constitués peuvent déterminer eux même ses modalités.


Le plus souvent, la constitution prévoit une procédure de révision. \\
L'article (ou les articles) de la constitution fixant la procédure de révision est communément appelé "clause de révision". Il existe des pays qui prévoient plusieurs procédures de révisions.


Dans le cas d'une habilitation explicite, se pose une question cruciale: le pouvoir de révision peut-il modifier la clause de révision ? \\
Pour certains, cela est impossible car ce serait illogique: une clause de révision ne peut pas autoriser sa propre révision, sans quoi elle ne sert à rien.  \\
Pour d'autres, cela est possible car ce n'est pas interdit par la norme constitutionnelle. \\
Deux thèses politiques s'affrontent:
\begin{itemize}
\item La constitution doit être préservée contre les modifications qui pourraient la dénaturer en portant atteinte à ses valeurs fondamentales (thèse conservatrice).
\item La constitution ne doit pas être intangible car une génération n'a pas le droit d'assujettir à ses règles les générations futures (thèse progressiste).
\end{itemize}


\end{document}
