\chapter{Introduction au droit constitutionnel}

Le droit constitunionel fait partie du droit public, il a deux objets~:
\begin{itemize}
\item Premier objet~: entre citoyens et l'État.
\item Deuxième objet~: entre les collectivités entre elles.
\end{itemize}
C'est l'étude de la constitution que l'on peut définir comme étant la norme fondamentale dans un ordre juridique.

Premier angle d'étude: théorique et historique~: qu'est-ce que l'État ? L'État n'existe que parcequ'il y a une constitution. Qu'est-ce que la norme juridique ? Qu'est-ce que l'histoire du phénomène constitutionnel ?

La première constitution écrite date de 1791~: révolution française.

Second angle d'étude~: étude de la constitution en vigueur, ses effets. C'est le droit constitutionnel positif.

\section{Qu'est-ce que le droit ?}

Le droit est un ensemble de règles de comportements qui conditionne la vie en société (qui implique donc un rapport à l'autre et au pouvoir) dont la violation est sanctionnée par une autorité publique.\footnote{Une autorité disposant du monopole de la violence publique légitime, Max Weber.}

Trois critères donc du droit~:
Ensemble de règles, phénomène donc normatif.\footnote{Le droit est donc un système normatif} Le droit est normatif au sens où il impose des obligations. Deux catégories d'obligation~: obligation de faire une action (comme payer les impôts) ; obbligation de ne pas faire (cf: droit pénal, qui comporte + de 12 000 infractions).

Le droit reconnait des prégoratives\footnote{Capacité d'agir} que la philosophie reconnait comme les droits de l'Homme.\footnote{Liberté, Propriété, Expression...}\newline

Pour qu'il y ai du droit, il faut qu'il y ait sanction par autorité publique. Quand il y a sanction par autorité privé, ce n'est pas du droit.

En france, les autorités publiques sanctionnant sont le juge et l'autorité administrative. \newline

Le troisième critère est le rapport entre le droit et la société. Le droit est un phénomène social. La fonction du droit est d'organiser et conditionner la vie en société. Il existe donc un rapport très étroit entre société et droit, chaque société a un droit différent. \newline

Trois fonctions/objectifs du droits~:
\begin{itemize}
\item Maintenir la paix civile (sûreté de l'individu ainsi que sa tranquilité)
\item Garantir les libertés \footnote{garantit par la constitution justement}
\item Garantir un lien social (éviter que les individus s'isolent sur eux même)\footnote{D'où le mot Fraternité dans la devise républicaine}
\end{itemize}


\section{Qu'est-ce que la constitution dans le droit ?}

La France est un État de droit. C'est à dire que les comportements sociaux, politiques, individuels etc.. Sont soumis à des règles juridiques. Ces règles appartiennent à un système unifié et hierarchisé. Unifié car les règles appartiennent toutes à l'ordre juridique Français. Hérarchisé dans la mesure où les règles de rang inférieur, pour être valide (et produire des effets) doivent respecter les normes de rang supérieur.\newline

Au sommet de cette hierarchie, il y a donc la norme constitunionelle. Plus le niveau est elevé, moins il y a de normes, plus le niveau est bas dans la hierarchie, plus les normes sont nombreuses.\newline

La constitution de 5ème république, on retrouve les règles de dévolution du pouvoir politique. On y retrouve aussi les déclarations des droits de l'Homme. Il y en a trois~: la première et la plus importante est la DDHC du 26 aout 1789. La deuxième est une déclaration qui actualise la première après la seconde guerre mondiale (cette déclaration donne des droits sociaux, grêve, égalité homme/femme...).\footnote{préambule de la constituion de la IVème république} La troisième déclaration de 2004 concerne l'environnement.

Il existe un rang en dessous de la constitution, qui a une origine exogène~: le droit international et, notamment, européen. Donc, dans la perspective hierarchique, ce droit est conforme à la constitution. Si ce n'est pas le cas, la constitution peut être révisée.

C'est le conseil constitunionel qui juge de la conformité.\newline

Un rang en dessous, on a la législation, qui est fait par le législateur. Ces lois doivent être conforme aussi au droit international et européen. Toutes juridictions ordinaires en France peut juger de cette conformité (juridiction judiciaire comme juridiction administrative).\newline

Cour de cassation~: niveau le plus élevé en judiciaire ; conseil d'État~: plus haute juridiction administrative. 

Rang encore en dessous, la règlementation, qui comprend les décrets présidentiels et du premier ministre. Il y aussi les arrêtés ministériels, mais aussi les arrêtés locaux (préfectauraux, municipaux, des établissements publiques..).\newline

Il y a trois ressorts pour juger des conformités~:
\begin{itemize}
\item Le tribunal administratif
\item Cour Administrative d'Appel (CAA)
\item Le conseil d'État
\end{itemize}

Il existe des millions d'actes et faits juridiques privés adoptés chaque jour. Il existe aussi des faits juridiques non écrit (comme l'achat d'un bien). Les infractions au code pénal sont aussi des faits juridiques. On est dans le fait juridique dès que l'on a un rapport au social. \newline

Il existe aussi des actes administratifs individuel, 5ème échelon. L'attribution d'un permis est un acte administratif individuel (de conduire, de construire...). \newline

Juridiction en première instance~: très nombreuse selon le fait ; tribunal d'instance ou de grande instance, Prud'hommes, tribunal des affaires de sécurité sociale (TASS), tribunal de police, tribunal de correctionnel, cour d'assises.

Si premier apppel~: cour d'appel puis cour de cassation en troisième appel.


\section{Comment envisager la constitution dans le cadre d'un régime politique démocratique et libéral ?}

Premier aspect~: la constitution est un acte qui a été adopté par le peuple en tant que pouvoir souverain.

La constitution doit garantir les droits et libertés.

La constitution doit garantir que le gouvernement soit modéré pour empêcher un gouvernement absolutiste ou totalitaire. Ce qui est fait par la séparation des pouvoirs. Il y a séparation des pouvoirs quand des organes différents ont les différents pouvoirs.

Selon la relation entre le pouvoir exécutif et législatif, on distinguera plusieurs régimes, parlementaires, présidentiel...


\section{Comment envisager la constitution actuellement en vigueur ?}
