\documentclass[10pt, a4paper, openany]{book}

\usepackage[utf8x]{inputenc}
\usepackage[T1]{fontenc}
\usepackage[francais]{babel}
\usepackage{bookman}
\usepackage{fullpage}
\setlength{\parskip}{5px}
\date{\today}
\title{Cours de Politique économique (UFR Amiens)}
\pagestyle{plain}
\begin{document}
\maketitle
\tableofcontents

\chapter{Introduction}

Les politiques économiques peuvent se définir de façon assez large: c'est le domaine d'intervention des pouvoirs publics dans le domaine de la régulation de l'économie. Ce que l'on appelle aujourd'hui économie renvoie à l'économie marchande capitaliste. On peut voir dans les formes concrètes de ces politiques économiques les dépenses publiques, la fiscalité, les réglementations, les incitations diverses, la gestion du service public, la politique monétaire. \\
L'objet de ce cours est d'analyser les formes de ces interventions pour en déduire les enjeux. Cette analyse se heurte à des difficultés particulières, propre au champ économique. Même si elle est abondamment commenté, la politique économique fait l'objet de nombreux débats et l'on peut entendre tout et son contraire. Un des principaux débats porte sur l'efficacité des politiques économiques. Pour certains, le pouvoir politique est le dieu de la croissance et de l'emploi (deux des principaux objectifs que se donnent les politiques économiques). Les politiques économiques suscitent donc de grandes attentes, celles-ci ne sont elles pas trop grande ? \\
À l'opposé, pour d'autres économistes, le politique n'a pas d'influence, et ne peut pas grand chose. Ce serait donc pour eux une affaire d'experts, qui chercheront à mettre en évidence l'inefficacité des politiques économiques et donc l'impuissance des politiques. Leur principal argument est celui de l'avènement de la mondialisation commerciale, qui restreint la puissance du pouvoir politique. \\
Plus fondamentalement, ce débat renvoie au fait que l'évaluation des politiques économiques ne fait pas consensus. \\
Les politiques économiques modernes sont nés dans un certain contexte historique, celui des années 30 et 40, un contexte de crise et de guerre. On a tendance à oublier que les politiques économiques sont nés dans ces contextes, face à la crise où il fallait faire quelque chose et face à la guerre où il fallait coordonner voir planifier l'économie. Le PIB est d'ailleurs un indicateur née pendant cette période, il a été créé par Kuznets dans un contexte où aucun indicateur direct n'existait. On découvre grâce à cet indicateur que le revenu national américain a chuté de 40\%. 


L'économie a une place spéciale dans l'épistémologie. En effet, l'économie n'est pas une science, c'est une pratique, un art ou une technique éventuellement, mais certainement pas une science. L'économie s'appuie cependant sur un certain nombre de savoir que l'on nommait avant l'économie politique, ce que d'autres appellent aujourd'hui la science économique. \\
Les débats économiques ne sont jamais définitivement tranchés, ils sont cycliques, contrairement aux sciences naturelles où les débats sont vites tranchés. \\
Pour certains, il n'existe que la théorie économique. Cette vision implique que les théories antérieurs soient obsolètes et que les paradigmes extérieurs soient non reconnus. \\
Les rapports de force évoluent entre les différentes écoles économiques, mais il n'y a jamais d'éliminations définitives. \\
D'autre part, de nouveaux courants ont vu le jour comme les post-keynésien ou les régulationniste. Les théories ne se succèdent pas, elles coexistent. \\
Pour certains, la coexistence des théories est insupportables et propose d'imposer des vérités qui ne s'imposent pas d'elle même. Le but de l'économie politique est de mettre en évidence des relations causales, elle y parvient parfois de façon grossière et incertain. \\
Les variables économiques ont une influence avec toutes les autres, c'est pourquoi les économistes utilisent la clause "ceteris paribus", ou "toute chose égale par ailleurs", qui vise à poser des hypothèses d'école en disant que si tel variable change, mais pas toutes les autres, alors tel variable changera comme ceci. \\
L'économétrie est un domaine qui essaye de voir les relations entre les variables, en distinguant les variables exogènes (donnés par l'extérieur) et endogènes (donnés par le modèle). Dans la réalité, aucune variable n'est réellement exogène. \\
Une autre difficulté provient de l'idéologie en science sociale. Un chercheur en science naturelle n'est pas impliqué de la même manière dans ses recherches qu'un chercheur en science sociale. Le chercheur en science sociale, ici, en économie, ne va pas voir les mêmes choses, ne va pas regarder aux mêmes endroits que d'autres, ne va pas poser les mêmes questions. Les économistes expriment donc une croyance, une vision du monde. 


Dans le premier chapitre, on va se demander comment les pouvoirs politiques ont acquis une capacité d'intervention économique et comment les formes de cette intervention ont évolués dans le temps. \\
Dans le deuxième chapitre, on se posera la question de pourquoi intervenir dans la régulation de l'économie marchande capitaliste, quelles sont les fonctions des politiques économiques, quels sont les grands débats qui les concerne. \\
Dans le troisième chapitre, on verra ce qu'est la politique budgétaire, et dans le quatrième ce qu'est la politique monétaire. \\
Le cinquième chapitre concernera une analyse d'une économie fermée en autarcie. \\
Le sixième chapitre concernera la modification de l'économie fermée dès que les flux de marchandises et de capitaux s'internationalisent. C'est donc une analyse d'une économie ouverte. \\
Le dernier chapitre concernera les débats des politiques de l'emploi. 

\section{Notions importantes}

\subsection{PIB}

D'abord, le PIB. C'est un indicateur de l'activité économique, qui mesure une partie de la production, celle qui est le résultat d'un travail rémunéré. Cet indicateur ne dit rien de la qualité de vie dans un pays, ni des problèmes sociaux, ni des problèmes environnementaux. On peut dire que le PIB représente la somme des valeurs ajoutés. Le PIB est un flux de production, ce n'est pas un stock. On peut résumer: Valeur Ajouté=Chiffre d'affaire-Consommation intermédiaire. \\
Deux unités produisent de la valeur ajoutée, la production de valeur marchande et la production de valeur non marchande (comme celle des services publics). Dans le secteur non marchand, la valeur ajoutée est considéré comme étant le total du salariat. \\
Il est à noté que le PIB est calculé en volume et non en valeur, permettant d'écarter l'inflation. Une augmentation du volume est une croissance, une augmentation de prix est une inflation.

\subsection{Revenu primaire}

Le revenu primaire est le revenu de l'activité productive et de la propriété, sous forme de salaire, de dividende, de profits, etc. \\
Dans une économie comme la notre, le revenu est monétaire et non composé de service ou de biens. La monnaie est donc un "droit de tirage" sur la richesse créée. La monnaie donne droit à une partie du gâteau. \\
Une autre partie du gâteau n'est pas acquise contre de la monnaie: c'est de la production non marchande. Celle-ci est soumise à des conditions qui ne sont pas économique: l'éducation, la santé, etc. \\
Répartition primaire: Le revenu des facteurs est le revenu du travail et du capital. Les impôts sur la production (TVA). L'amortissement. Les bénéfices non distribuées. 


\subsection{Revenu disponible et son usage}
La répartition primaire se traduit par l'apparition de fortes inégalités. Ces inégalités sont à l'origine de l'intervention de l'État providence, qui intervient dans le but de réduire les inégalités. \\
L'État le fait pour des raisons de justice sociale. Ces motivations socio-politique repose sur une idée principale: dans une société relativement prospère, tout individu doit bénéficier d'un minimum de sécurité (au sens large). Une deuxième motivation est économique, Keynes par exemple, pense que les économies de marché fonctionnent mieux si il y a une relative égalité. Les capitaux qui sortent des circuits par l'épargne des plus riches ne stimulent pas l'économie, donc si ils sont redistribués, ils permettent une consommation.

\subsection{Emploi et chômage}

La population active représente ceux qui travaille et ceux qui cherche un emploi. \\
Le taux de chômage se fait par rapport à la population active et non par rapport à la population totale. 

\chapter{Genèse et formes historiques de la politique économique}

Comme toute politique publique, la politique économique présuppose l'existence d'une autorité qui décide et qui ait le pouvoir de décider, donc présuppose une forme d'État au sens large. L'émergence de l'État repose sur un double processus: l'apparition d'une relation politique de pouvoir et celle d'une division sociale du travail. \\
Dans les sociétés primitives, où règne une certaine égalité des conditions et une absence de classe sociale, existe néanmoins des intérêts communs au groupe, à la tribu, au clan, etc. Ces intérêts communs doivent être confiés à la garde de certains individus. Ces intérêts sont liés à la survie du groupe. Ces individus constituent les prémices du pouvoir d'État. Ils vont gagner en autonomie, soit du fait de l'hérédité de la charge, soit du fait de l'impossibilité de se passer d'eux à mesure qu'augmente les conflits avec d'autres groupes. \\
Après s'être rendu indépendante vis à vis de la société, la fonction politique va dominer la société, se placer au dessus du groupe. F. Hengels "Le serveur primitif s'est métamorphosé peu à peu en maître".


Ce processus politico-social se double d'un processus économique: l'apparition de la division sociale du travail. La forme première est l'esclavage où il y a division entre esclave et maître. \\
A. Smith, sur la division du travail, dit que cette division est limitée par l'étendue du marché. Il existe deux manières d'étendre le marché, soit en interne, soit en externe. En externe, en commerçant avec l'extérieur en inventant de nouvelles formes de transport. La croissance démographique en interne permet de l'étendre aussi. \\
Ces deux facteurs résultent eux même de la croissance de la production, formant donc un cercle vertueux.


Dans l'antiquité, les finances publiques des grands empires reposent sur le versement de tributs, sur le pillage, et sur l'impôt. Le budget de l'empire romain repose sur l'impôt et les confiscations. L'effondrement interne de l'empire se combine avec les grandes invasions. Ce mouvement de fragmentation de l'espace politique et militaire de l'ancien empire romain d'occident s'inverse à partir du VIIe siècle pour aboutir à la formation au cours du VIIIe siècle, d'une nouvelle entité politique très étendu, celui du Saint-Empire Romain. \\
Charlemagne distribue ce qu'on appelle des "bénéfices", c'est à dire des terres à ceux dont il voulait s'assurer la fidélité. En échange, les bénéficiaires lui prête serment, c'est la naissance de la vassalité. Avec l'hérédité, ces vassaux vont se constituer en classe sociale. \\
Malgré tout ses efforts, les francs n'avaient pas le sens de l'État, les rois francs voyaient leur royaume comme leur propriété, et l'empire va dépérir au fil des successions. \\
Durant cette période de morcellement de l'empire carolingien, le pouvoir de lever l'impôt est aussi morcelé. 

\section{L'impôt et la genèse de l'État moderne}

Durant plus de deux siècles, la monarchie ne prélève plus d'impôt général en France. Pendant cette période, les droits coutumiers forment l'essentiel des ressources du Roi et des seigneurs. Cette période débute en 924 et s'achève en 1147 lorsque Louis VII ordonne la collecte du 20ème (ou 5\%) des revenus du Royaume pour payer les dépenses de la deuxième croisade. 



\section{L'évolution fiscale et sociale}


\section{Évaluation des dépenses publiques}



\chapter{La politique économique: pour quoi et pour qui ?}

Pourquoi des politiques économiques sont elles mise en place ? Une réponse simple est qu'elles remplissent des fonctions, c'est une approche fonctionnaliste. R. Musgrave a écrit "The theory of public finance" (1959), dans lequel il donne trois fonctions des politiques économiques: allocation (affectation), redistribution, stabilisation. \\
Il écrit aussi en 1959 que le capitalisme est une économie mixte. Il n'y a pas d'un côté une économie entièrement privée et de l'autre une économie entièrement publique. L'État joue lui même un rôle important. Musgrave est Keynésien, il est plutôt pour l'intervention publique. \\
Quels sont les effets de la politique économique ? \\
La fonction d'allocation permet de savoir ce qui relève du domaine couvert par l'État en matière de biens publics et de fixation des règles concernant l'allocation des biens privés, ce qu'il autorise donc en production à l'espace privée. \\
Concernant la redistribution, Musgrave note d'abord le caractère récent de cette fonction qui vise à accorder des correctifs de la redistribution en vigueur. Une fois écoulée la production et établi une distribution primaire des revenus, l'État peut chercher à infléchir la répartition. \\
Enfin, la fonction de stabilisation (dite conjoncturelle), vise à éviter de trop fortes fluctuations et à favoriser l'accroissement régulier de l'investissement et de la consommation. C'est l'idée qu'en l'absence d'intervention de l'État, le marché ne mène pas à une situation optimale. Les auteurs classique ne pensent pas que l'État doit intervenir dans la stabilisation, mais plutôt dans autre chose comme l'infrastructure économique qui doit selon eux, être pris en charge par l'État. Musgrave sur la stabilisation "Maintenir un haut niveau d'utilisation des ressources et la stabilité monétaire[...] cette fonction a été sous les feux de la rampe au cours des 20 dernières années". \\
Une fois noté cette typologie, on doit rappeler que le contenu précis de ces trois fonctions vont faire débat. On peut aussi se demander si les trois fonctions sont endogènes, ou si elles sont exogènes. \\
Si absence d'allocation, il y a désordre, si absence de redistribution, il y a favorisation de la crise économique et risque de crise sociale et politique, si il n'ya pas de stabilisation, il y a crise de chômage. 


Keynes a largement contribué à légitimer l'augmentation des dépenses publiques du XXe siècle. Sa Théorie Générale (1936), avec la rapport Beveridge (1942), ont aussi contribué à légitimer cette redistribution, ce sont les jalons du processus d'édification de l'État providence ou welfare State. 

\section{La justification Keynésienne de la politique économique}

Sur le plan de la pensée économique, les universités anglaises sont des lieux où vont naître des théories décisives dans un triangle Cambridge, Oxford, Londres, qui est au coeur d'un bouillonnement intellectuel dans les années 20 et 30. Keynes est de Cambridge. Hayek est de Londres, et sera contre Keynes. \\
Le Keynésianisme est né à Cambridge et provient des idées de Keynes sur la production du revenu national et l'analyse macro économique. La Révolution Keynésienne ne propose pas seulement un nouveau cadre de pensée mais dérive de celui-ci des conclusions pratiques, en particulier la nécessité d'une intervention de l'État. 

\subsection{Keynes, éléments biographiques}

Keynes est né en 1883, l'année de la mort de Marx, il décède en 1946. Il est devenu célèbre en dénonçant de façon très lucide et prémonitoire le traité de Versailles signé le 28 Juin 1919. Alors âgé de 35 ans, Keynes était un des principaux représentants du trésor anglais à Paris pour les négociations. Très vite, il se dit que ce qu'on va infliger à l'Allemagne est exorbitant. Il démissionne de la délégation Britannique. \\
En 1919, il publie "Les conséquences économiques de la paix". Il explique que ce qu'on demande aux Allemands est impossible à tenir et qu'on prépare une nouvelle guerre. \\
Il a aussi participé de 1942 à 1944 aux négociations entre le RU et les US concernant les bases du nouveau système monétaire international. Les propositions de Keynes ne seront pas retenus, au profit du plan White, adopté à Brettan Woods. \\
Sur un plan politique, Keynes est membre du parti libéral. C'est un parti qui est entre les travaillistes et les conservateurs. Il explique que son but c'est de sauver la propriété privée, où il y aurai une forte intervention de l'État, etc. \\
Keynes a participé à faire que le principal objectif de la politique économique soit l'emploi. Cependant, pour assurer le plein emploi, il faut comprendre les déterminants de l'emploi. C'est pourquoi il écrit en 1936 "Théorie générale de l'emploi, de l'intérêt et de la monnaie". \\
Il explique que la monnaie, via les taux d'intérêts ont une influence sur l'emploi. Il cherche à se distinguer des classiques et des néo-classiques. 

\subsection{La faillite de l'économie classique}

On peut, en s'appuyant sur la théorie classique, prétendre expliquer pourquoi un individu refuse de travailler pour un certain salaire. On pourrai aussi expliquer pourquoi certains individus sont sans emploi de façon transitoire dans une économie dynamique où des secteurs refluent tandis que de nouveaux apparaissent. Aucune de ces considérations ne permet de rendre compte d'un chômage massif et durable. \\
Hoover, le nouveau président, en 1929, ne fait strictement rien car il est persuadé de tout cela. Cela ne marche bien sûr pas du tout. Entre l'élection de Roosevelt et son investiture, les US sont au bord de l'implosion. \\
Keynes va répondre à cette question politique en écartant la solution classique, et expliquer que la solution ne peut pas être une baisse des salaires. Keynes va expliquer qu'il faut stimuler la demande effective, en réduisant le taux d'intérêt et en acceptant un déficit budgétaire. \\
C'est dans les années 1920, dans les contextes du débat politique anglais, que Keynes va commencer à critiquer l'orthodoxie. Les anglais sont à l'époque endetté à 140\% de PIB, le chômage est important. Dès 1924, Keynes publie un article pour soutenir Lloyd George, chef du parti libéral, qui veut lancer une politique de grands travaux. \\
En 1925, Keynes écrit "The economic consequences of Mr Churchill" pour critique sa politique d'austérité. Churchill avait même rétabli l'étalon-or, avec le ratio d'avant-guerre. Keynes va qualifier cet étalon-or de relique barbare car c'est vieux et cela va avoir des conséquences néfastes, la livre sera surestimé, la production britannique ne sera pas compétitive. De fait, la GB va subir une crise toute la décennie: les prix diminuent, causant une baisse d'activité. \\
La GB monte aussi les taux d'intérêts de sorte à maintenir artificiellement l'étalon-or, mais freine du coup l'investissement.


Churchill va mettre en place une politique d'austérité. Il va dégager un solde budgétaire primaire positif (Recettes fiscales-dépenses hors intérêt>0). Il cherche à dégager un excédent. Il dégage d'ailleurs un excédent massif, de 7\% du PIB. L'État "pompe" 7\% de l'activité en vue de se désendetter. \\
Cependant, la conjugaison d'une politique monétaire orthodoxe avec une politique d'austérité contribue à aggraver la déflation, L'ironie c'est que l'endettement ne diminue pas au final, mais augmente, car augmente l'intérêt réel (Intérêt réel=taux d'intérêt-inflation). 


Aux US, dans les années 30, on passe de 3,2\% en 1929 à 25\% en 1933. Entre temps, le PNB a chuté de 30\%. Sur un plan théorique, ce chômage de masse constituait pour la théorie orthodoxe une anomalie. C'en est une à deux égards, d'abord parce que le chômage était devenu un phénomène mondial, d'autre part, les salaires monétaires diminuent sans que l'emploi reparte. 

\subsection{La théorie générale}

Pour Keynes, le niveau de l'emploi et le salaire ne sont pas déterminé sur le marché du travail. Pour lui, le travail n'est pas une marchandise comme une autre, et il existe une asymétrie entre les agents économiques. Il y a une asymétrie entre les entrepreneurs et les autres. Le niveau de l'emploi provient d'une décision unilatérale des entrepreneurs, ce qui ne veut pas le dire qu'ils le font librement. \\
Il dit ensuite que les phénomènes économiques sont marqués par l'incertitude. Ils ne sont pas même probabilisables. \\
Pour Keynes, le salaire nominal est rigide. Il ajoute que la productivité du travail est décroissante: plus l'emploi augmente et plus la productivité ralentit. Cela accroît le prix des marchandises, car la production est moins efficace. salaire réel= w/p (w, salaire nominal, p, pouvoir d'achat). \\
Keynes ajoute qu'aucun mécanisme marchand ne peut assurer le plein emploi, il va même montrer que l'économie peut être en situation d'équilibre de sous emploi. Cela suggère que l'économie peut être stable avec du sous emploi, il dit donc que dans ce cas de figure, l'économie classique ne peut rien faire car voit le chômage comme un déséquilibre. 


I et S (Investissement et épargne). Pour les classiques, I et S sont des fonds prêtables et son prix est le taux d'intérêt. Donc, si I<S->baisse r->I=S. (r, taux d'intérêt). \\
Keynes dit qu'on a forcément l'égalité I=S en fin de période, cela découle de Y=C+I (Revenu=Consommation+Investissement). Y=C+S, car l'épargne est par définition ce qu'il reste du revenu une fois consommé. \\
La consommation dépend du revenu net. C=c.Y (c étant la propension à consommer). Il appelle cela la loi psychologique fondamentale, et nous dit que c est d'autant plus faible que Y est élevé. Donc, on a Y=cY+I et donc Y=I/(1-c) où ici Y est le revenu national, c'est la demande effective d'après Keynes. \\
Keynes ajoute de suite que dans un univers incertain, l'investissement fixé par les entrepreneurs n'a aucune raison de correspondre au plein emploi, donc cela débouche sur une crise. \\
Comment les entrepreneurs fixent le niveau d'investissement concrètement ? Les entrepreneurs prennent leurs décisions en comparant deux variables: r, le taux d'intérêt, et e, l'efficacité marginale du capital (les profits futurs anticipés).  e, c'est, selon Keynes, des "animal spirits", c'est l'état d'esprit des entrepreneurs. Plus e est élevé, plus les entrepreneurs vont investir, mais r, c'est le coût de l'investissement, donc si il est élevé, l'entrepreneur investira pas tant que ça. Pour Keynes donc, tant que e>r, il y a des opportunités d'investissements rentables, donc il y a investissement. Mais plus d'investissement il y a, plus e baisse jusqu'à e=r.


Pour les classiques, r est le prix de renonciation à la consommation présente, cela veut dire que n'importe qui a le choix de consommer ou d'épargner, pour les classiques, c'est le taux d'intérêt qui détermine la répartition entre les deux. \\
Keynes dit non, il dit que le paramètre est psychologique, avec c. Il dit que les plus riches épargnent plus. Mais le taux d'intérêt r peut intervenir sur la forme que l'épargne va prendre, soit des titres financiers (ce qui n'était que ça pour les classiques), soit sous forme de monnaie, où les gens thésaurisent. C'est ici que r entre en jeu pour Keynes. \\
Il y aurai trois motifs de thésaurisation: transaction M^d(Y) (admis par les classiques car temporaire), précaution, spéculation. L'argent liquide selon Keynes, calme l'angoisse. \\
Y=I(r)A/(1-c) où A sont les anticipations. 










\section{Les critiques du }



















\end{document}
