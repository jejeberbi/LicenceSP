\documentclass[10pt, a4paper, openany]{book}

\usepackage[utf8x]{inputenc}
\usepackage[T1]{fontenc}
\usepackage[francais]{babel}
\usepackage{bookman}
\usepackage{fullpage}
\setlength{\parskip}{5px}
\title{Cours de Politique internationale (UFR Amiens)}
\pagestyle{plain}


\begin{document}
\maketitle
\tableofcontents

\chapter{Introduction générale}

\section{L'étude des relations internationales: du Droit à la Science Politique}

Le concept de relation internationale est vaste: il renvoie aux relations de toute nature qui traversent les frontières. Ces relations ont pour principale caractéristique d'échapper au cadre d'un seul État. \\
Si on retient ce champ très large, le domaine de la politique internationale est illimité puisqu'aujourd'hui, aucun État ne peut plus vivre en autarcie, sans subir des contraintes extérieurs (même la Corée du Nord subit certaines contraintes extérieures). \\
Les frontières ne sont jamais complètement hermétiques: flux de personnes, de biens, de capitaux, d'idées, d'informations, etc. Les menaces aussi traversent les frontières, et comme elles les traversent, elles ne peuvent être réglées que par de la coopération internationale. \\
Le champ des relations internationales a cependant des caractéristiques propres. La principale caractéristique du système international est qu'il n'existe pas d'autorité centrale qui pourrai adopter des normes universelles et les faire appliquer. \\
Les relations entre les États se sont institutionnalisés: diplomatie, traités, organisations internationales. Ces organisations édictent d'ailleurs des normes, créant des organes juridictionnels. Cependant, il n'existe pas de pouvoir législatif universel, ni de pouvoir judiciaire obligatoire et universel (la CPI dépend du consentement des États et ne regroupent pas du tout tous les États du monde, les US, la Chine ne sont pas membre de la CPI).


Cependant, on a toujours cherché à comprendre les relations internationales. Les chercheurs ne se posent pas toujours les mêmes questions, ne suivent pas les mêmes méthodes, etc. Les premiers à s'intéresser aux relations internationales étaient les philosophes, d'Aristote à Rousseau en se posant les questions de la guerre et de la paix. Ils ont donc une approche très spéculative et abstraite, mais ont donné un apport non négligeable. \\
Les historiens aussi s'intéressent aux RI, et on peut remonter jusqu'à Hérodote, qui avait étudié la guerre entre les Grecs et les Perses. L'histoire permet d'éclairer un certain nombre d'événements, et contrairement à la philosophie, elle se base sur des données matérielles, ce qui en fait aujourd'hui une discipline à part entière: l'histoire de la diplomatie. \\
Ensuite, on a une approche juridique, plus récente. Hugo de Groot (dit Grotius), un juriste, a permis l'existence de traités sur les droits de la mer et a contribué au XVIe siècle à faire du droit international un droit positif. Entre la fin du XIXe et le XXe siècle, quand on a vu les premières OI se créer, le droit international s'est institutionnalisé. Cependant, le droit international est une science des normes, et non des faits. Le DI cherche à identifier les règles, à les interpréter, si elles sont respectées. Le DI ne s'intéresse pas au pourquoi.


C'est pourquoi le politiste a cherché à expliquer les phénomènes internationaux. C'est l'approche la plus récente, le premier enseignement de politique internationale était en Grande Bretagne après la première guerre mondiale. La politique internationale s'est surtout imposée après la deuxième guerre mondiale, surtout aux USA, parce que c'est à la fin de cette guerre que les US se sont acceptés comme superpuissance, et, avec la guerre froide, les US avaient une réflexion à faire, le gouvernement investissait d'ailleurs énormément dans les universités. \\
Les fac américaines venaient justement d'accueillir un grand nombre d'intellectuels européens qui ont massivement investi le domaine de la PI, se sentant concerné. \\
La PI s'est tout de même diffusée, des recherches ont étés lancées un peu partout sur des thèmes divers: le tiers monde, les mobilisations transnationales, etc.


Dans la PI, il s'agit avant tout de chercher des relations de causes à effet. On va donc se demander qui sont les principaux acteurs, quels sont les facteurs qui déterminent leur comportement. 

\chapter{La nature du système international: une vieille controverse théorique}

La question de la nature du système international a toujours fait l'objet d'une querelle entre les chercheurs mais aussi entre les acteurs politiques. Cette querelle a eu des répercussions pratiques. De fait, aux US, les liens entre les milieux universitaires et le milieu gouvernemental est très étroit, le gouvernement US investit massivement, lui permettant de pousser la recherche vers les questions qui l'intéresse. \\
Aux US, les universitaires sont souvent amenés à exercer des responsabilités publiques: conseiller voire membre du gouvernement. De plus, les Think Tanks sont très nombreuses aux US, ce sont des structures de droit privés, non lucratives, dont l'objet est de développer des idées et des connaissances pour guider l'action publique. \\
Ces débats théoriques exercent donc une influence sur la gestion des problèmes concrets. 


La question qui divise est de savoir si il existe un certain ordre dans le fonction du système international ou si il est par nature anarchique. \\
L'approche qui a longtemps dominé et qui exerce toujours une place de premier plan est celle de l'anarchie. 

\section{L'approche réaliste: un cadre d'analyse dominant}

C'est une approche qui a le mérite d'être simple et rationnelle, c'est une approche qui veut voir le monde tel qu'il est et non tel qu'on voudrait qu'il soit. Cela implique de se démarquer de l'idéalisme, du moralisme et de mettre l'accent sur les rapports de force. \\
Cette approche va de pair avec un certain conservatisme, un statut quo, et a donc exercé une très forte influence sur les Républicains aux US, notamment Bush père et surtout Nixon. Elle a aussi trouvé quelques adeptes du côté démocrate, Obama, en pratique, par sa prudence avait un côté réaliste par sa volonté de ne pas trop bousculer l'ordre établi. Joe Biden et d'autres conseillers d'Obama se revendiquaient réalistes. \\
Dans le cas de la France, tous les Présidents depuis le début de la Vème République ont menés une politique réaliste.

\subsection{Les précurseurs}

Les premiers précurseurs de l'approche réaliste apparaissent dès le XVIe-XVIIe siècle avec Machiavel d'abord qui cherchait à séparait l'analyse de la réalité et la morale et qu'il voyait la société internationale dominé par les rapports de force et les intérêts des États. \\
Ensuite, on a Thomas Hobbes dont le postulat de base est que chaque homme est le concurrent de l'homme. Il développe l'idée que pour échapper à l'état de nature libre et dangereux, l'homme aurait passé un contrat social pour créer un pouvoir politique en échange de la paix et de la sécurité. Il ajoute que dans l'ordre international, il n'y a pas de contrat social possible, car cela signifierait de renoncer à la souveraineté et voit donc le système international comme un état de nature éternel et anarchique. \\
Cette approche réaliste s'est perpétuée. Au tout début du XXème siècle, il s'est imposé une conception idéaliste et pacifiste des relations internationales, qui a eu du mal à résister à la chute de la SDN et à la montée du nazisme.

\subsection{Le réalisme depuis le XXème siècle}

Le réalisme a triomphé après la seconde guerre mondiale. Au milieu du XXème siècle, les travaux des américains ont servi de justifications de la politique internationale des US. \\
Hans Morgenthau est considéré comme le principal héritier de Machiavel ou Hobbes. Il voulait créer une vraie science des relations internationales. Pour lui, les États sont tous en quête de puissance, à cause de la nature humaine. \\
Henry Kissinger, c'est le réaliste par excellence. Il voulait débarrasser la PI de toutes les valeurs idéologiques et morales pour ne prendre en compte que les logiques de puissance. Il n'a plus aucune fonction officielle mais continue d'être très actif en prenant position, en faisant des conférences, etc. \\
Raymond Aron, un français, a écrit "Paix et guerre entre les Nations". C'est lui aussi un réaliste qui dit du système international qu'il est anarchique. Il dit que la société internationale est une société asociale. 


Les réalistes estiment que le système international et le système national sont complètement différent, le système international étant par nature anarchique. \\
Pour les réalistes, les seules acteurs des RI sont les États et négligent les autres acteurs comme les ONG, les entreprises, etc. ou les voient seulement comme des instruments de leur État de référence. \\
Les États sont rationnels selon les réalistes, et sont aussi unitaires, agissent comme un seul homme. L'État agirait selon un calcul coût/avantage, et sont vus comme foncièrement égoïste. L'intérêt national serait de conserver ou croître sa puissance. L'État serait donc par définition amorale. \\
Les facteurs économiques et culturels seraient de moindre importance selon les réalistes par rapport à la stratégie. \\
Par nature, le système international est conflictuel car les intérêts s'opposent. \\
Pour les réalistes, l'histoire n'a pas de fin, n'a pas d'issue: c'est un éternel recommencement où le progrès n'est pas possible. Quelques règles sont apparus si quelques États, les plus puissants, y consentent car ce serait dans leur intérêt. \\
Pour eux, la paix n'est pas possible, il n'existe que des trêves. Un seul moyen existe pour conserver cette trêve: l'équilibre des puissances. \\
On résume les réalistes par le fait que les RI sont comparables à un jeu de billard où les États sont les boules de billard: elles sont lisses et on ne prend pas en compte ce qu'il y a dedans. 


Une grande critique de cette approche est qu'elle est très rigide car s'intéresse toujours aux États, et a donc eu beaucoup de difficultés à expliquer le terrorisme. Face aux critiques, il y a eu un néo-libéralisme. \\
Ce néo-libéralisme, dont l'un des principaux penseurs est Kenneth Waltz, s'intéresse davantage à l'économie et aux acteurs privés. 

\subsection{Une théorie à l'épreuve du réel: l'interprétation réaliste de l'intervention américaine en Irak de 2003 et de ses suites}

Officiellement, l'objectif de Bush fils était de libérer le peuple Irakien et de propager la démocratie. Les réalistes ne peuvent pas accepter un tel argument, sauf si c'est juste un habillage pour obtenir le soutien des autres puissances. \\
Certains disaient que l'intervention était pour le pétrole. Si c'est le cas, c'est tout à fait envisageable pour les réalistes car les gisements pétroliers peuvent être vus comme une ressource hautement stratégique. Cependant, ce ne peut qu'être un seul facteur, les US produisant déjà beaucoup de pétrole. \\
Une autre idée est que les US devaient s'attendre à avoir des rivaux, et qu'ils devaient les éliminer, pour dissuader les autres rivaux. Les attentats du 11 Septembre ont montrés d'ailleurs que la sécurité des US était menacée. L'intervention armée permettrait donc aux US de ne pas assurer sa sécurité seulement par du défensif, ainsi que de donner un certain exemple en attaquant soi disant le mal à la racine. \\
Une question se pose: pourquoi l'Irak ? Pourquoi pas l'Iran, la Corée du Nord ? En 2003, l'Irak était assez isolé, avait peu d'alliés. Les américains connaissaient le terrain grâce à la guerre du Koweit de 1991 et l'Irak était faible à cause de cette défaite ainsi que des embargos que le pays avait subi. \\
De plus, le Moyen-Orient a toujours été une zone stratégique, mais aussi une zone fortement instable. Une autre donnée était que les US n'étaient plus sûr de leur allié dans la région (l'Arabie Saoudite) et aurait pu compter sur l'Irak sans Sadam. \\
Un réaliste critiquera la décision si il la juge inopportun, et, dès le départ, les réalistes étaient très divisés sur la décision. \\
Au final, l'intervention Américaine en Irak a été largement contre productive, ternissant l'image des US, se mettant à dos ses alliés et permettant des groupes terroristes de s'installer en Irak. \\
Les réalistes peuvent dire de cette opération que les calculs de base étaient erronés, les moyens engagés n'étaient pas les bons. 

\section{Des approches alternatives}

\subsection{L'approche libérale}

Cette approche contient plusieurs variantes, plusieurs courants. Ces courants tendent à relativiser le rôle des États pour prendre en compte les rôles des acteurs non étatiques. Ils mettent l'accent sur les facteurs de solidarité qui existent. Pour eux, un ordre pourrait émerger, l'anarchie internationale n'est pas absolue. \\
Les variantes sont le fait que les auteurs divergent sur les facteurs, certains parle de la Raison, pour d'autres, c'est le commerce, d'autres, les normes et les OI. 


L'approche libérale vient du premier libéralisme qui a inspiré la DDHC. Des auteurs croyaient au progrès et à la raison. Pour ces auteurs, les premiers acteurs sont les sociétés civiles composés d'individus. Pour eux, l'État n'est qu'un intermédiaire entre ces sociétés et la société internationale. \\
Pour d'autres, la progression des droits peut arriver à la création d'un ordre mondial, pacifique. \\
Pour certains, le libéralisme économique intervient. Pour ces auteurs, si les échanges sont amplifiés, alors les États n'ont pas intérêt à se faire la guerre. \\
C'est Kant qui parle de la raison qui doit gagner sur la force brute (Traité pour une paix perpétuelle). Kant proposait une fédération mondiale, qui permettrait de lier tous les États par des règles de droit commune. Kant comptait aussi sur la diffusion de la démocratie, car, pour lui, les démocraties ne peuvent pas se faire la guerre entre elles. 


Ces idées ont connu leur plus grand succès dans l'entre deux guerres. Pendant quelques années, l'approche libérale a réussi à s'imposer dans l'analyse des RI, mais dans sa version la plus idéaliste, et donc probablement la plus naïve. \\
Concrètement, cette tendance a influencée le Président US Wilson, père de la SDN. \\
Avec la montée du fascisme, l'éclatement de la SDN, et la seconde guerre mondiale, cette approche libérale a explosé.\\
Le transnationalisme est un renouvellement de cette approche libérale, débarrassée d'une bonne part de son idéalisme. Ce transnationalisme a eu une certaine influence, notamment sur Bill Clinton qui voulait revitaliser les OI, le commerce, pour assurer la paix et conserver une domination Américaine. \\
Jean Monnet et son entourage ne croyait pas aux utopies fédéralistes à court terme. Par contre, il pensait que la mise en place d'une coopération technique permettrait de dépasser les intérêts nationaux et de lancer un engrenage de fédéralisme. \\
Dans cette approche, les États ne sont pas les seuls acteurs des RI. Il y a tout un tas d'autres acteurs: des entreprises, des organisations, des OI, qui, justement, sont transnationales. \\
Dans cette approche, il faut une analyse plus fine, qui cherche à analyser les réseaux publics/privés. \\
Robert Keohane et Joseph Nye ont étés les premiers à insister sur le rôle que jouent ces acteurs non étatiques, mais aussi sur les influences mutuelles que va jouer la politique interne, qui, à contrario de la pensée des réalistes, jouent un rôle pour les transnationalistes (Linkage). \\
Pour les transnationalistes, les facteurs économiques, culturels, identitaires, sont aussi important que le facteur de la puissance militaire. \\
Les transnationalistes rejoignent un peu les libéraux car pensent que le monde peut sortir de cette anarchie perpétuelle. Ils disent donc que la multiplication des réseaux transnationaux créent une interdépendance qui est censé les amener à coopérer. \\


De fait, la mondialisation a accéléré ces relations d'interdépendance. C'est pour cela qu'on a créée la notion de "gouvernance mondiale". Ce mot, gouvernance, désigne la fonction de gouverner sans qu'il existe un véritable organe dont la fonction est de gouverner (le gouvernement désignant cet organe). \\
Il existe des variables d'ajustement qui peuvent s'imposer aux acteurs, ces variables jouant le rôle de la gouvernance. Ces variables résultent de l'interaction de tous ces acteurs. \\
L'idée est donc qu'il y a une toile de relations qui relie tous les acteurs de la planète.


Cette vision est très neutre et dépolitisé des rapports. En cherchant à démontrer la complexité des choses, en cherchant à démontrer la gouvernance, les transnationaux oublient ce qui relève de la coercition, de la domination, des inégalités. \\
En gros, tous les acteurs se valent dans cette approche. On retrouve un peu l'idée de la main invisible pour les libéraux économiques. 

\subsection{L'approche marxiste}

Cette approche prend le contre-pied de l'approche libérale, sans revenir au réalisme. Aujourd'hui, cette approche est un peu plus marginale que les précédentes, elle a été très forte dans les années 60-70. Elle reste assez présente dans les partis communistes et connaît un certain regain dans les approches altermondialistes.


Cette approche a ses racines dans les oeuvres de Marx. Pour Marx, le champ politique n'a pas de réelle autonomie, ce champ reflétant l'état du champ économique. Le mode de production économique (relations de travail et mode d'existence matériel) détermine tout le reste: les institutions, la vie politique, le droit, les idées. \\
Selon Marx, l'évolution économique s'est manifesté par une évolution de la division du travail donc une création de classes sociales. Pour Marx, il y a toujours eu une classe dominante qui possède les moyens de production. Marx pensait que le système capitaliste allait être victime de ses contradictions, et conduirait à une disparition des classes, et donc à l'État. \\
Pour Marx, les bourgeoisies nationales allaient petit à petit, mondialiser leur action et que cela allait amener leurs États respectifs à des conflits armés. \\
Certains successeurs de Marx ont repris certain des postulats pour les prolonger. Lénine a créé la notion d'impérialisme, stade ultime du capitalisme, qui se caractérise par une concentration à l'échelle internationale des moyens de production dans les mains de quelques groupes industriels et financiers qui rivalisent pour se partager le monde. Pour Lénine, les États seraient à la solde de ces groupes, créant donc des conflits mondiaux, créant donc une révolution commune des prolétaires de tous les pays. \\
Il faut attendre le XXe siècle et la décolonisation pour que ces idées se renouvellent. Il y a eu des néo-marxistes. Samir Amin, un auteur franco-égyptien, a écrit "Le développement inégal", où il explique le système international, c'est un centre et une périphérie. Le centre, ce sont les pays développés et la périphérie, les autres. Il explique que le centre a toujours eu tendance à exploiter la périphérie. Du coup, la périphérie est maintenu dans une dépendance. \\
Pour Samir Amin, le sous développement s'explique par cette exploitation. Une division du travail international obligerait les pays des périphéries à exploiter les matières premières.


Les Marxistes ont une vision qui est beaucoup plus dé-centrés. Le problème de la guerre et de la paix n'est qu'un sous problème de la division des richesses dans ce point de vue. \\
Les Marxistes s'accordent avec les réalistes: les RI sont conflictuelles. Les néo-marxistes ne pensent pas que les États soient les principaux acteurs. Pour eux, ce sont des instruments. Les acteurs seraient donc la bourgeoisie capitaliste, qui s'allient souvent avec les élites des pays en voie de développement. D'autres part, il y aurait les peuples, qui sont exploités de toutes part, qui ont du mal à s'allier entre eux, pensant leurs intérêts divergeant.

\subsection{Deux relectures de l'intervention américaine en Irak}

L'argument de propagation de la démocratie peut être sensible pour les libéraux. Cependant, sans autorisation de l'ONU, sans alliés de poids, sans soutien d'aucune OI, un libéral dénoncerait le caractère unilatéral de l'action militaire. \\
Pour les libéraux, la force armée n'est pas la meilleure façon de propager la démocratie. Dans les faits, les penseurs libéraux étaient fortement divisés. \\
Pour un transnationaliste, il faut observer la pression des acteurs économiques, etc. Le lobby pétrolier était pour cette intervention, comme le lobby Israélien. Un autre acteur international, ce sont les opposants au régime Irakien, en exil aux US. \\
Pour un libéral, l'État n'est pas un acteur unitaire. On peut donc s'intéresser aux rapports de force dans le cabinet américain: le gouvernement n'était pas unanime. George W. Bush n'était pas décidé au départ, mais l'échéance électoral, le contexte, etc, est déterminant dans l'approche libérale. 


Dans une approche Marxiste, on va dire que les US sont la puissance impérialiste par excellence, c'est le centre du centre. L'Irak est un État du Sud, convoité. C'est un pays de la périphérie. \\
Le principal objectif des US pendant longtemps, c'était le pétrole Irakien. Dans cette situation, la force militaire est un instrument permettant d'asseoir une domination économique. \\
Les US s'appuient toujours sur les Kurdes en Irak. Le Kurdistan est une région très riche en pétrole. \\
Si on est Marxiste, le fondamentalisme religieux est l'opium du peuple, détournant de la lutte des classes. \\
À priori, toutes les interventions militaires américaine en Irak sont condamnables, la victime étant le peuple. 


Aucune approche ne permet d'expliquer complètement les faits internationaux. Aujourd'hui on voit une approche constructiviste des RI, qui dit que dans l'absolu, la réalité n'existe pas, que la réalité n'a pas de sens, mis à part celui qu'on lui donne. 

\chapter{L'évolution des ordres internationaux: à la recherche de l'équilibre des puissance}

Comment définir la puissance d'un État ? Raymond Aron parle de sa capacité à imposer sa volonté aux autres. D'autres spécialistes s'accordent pour dire qu'il y a d'autres capacités: la capacité de faire des choses, la capacité de faire faire des choses aux autres, capacité de refuser de faire, et capacité d'empêcher de faire. \\
La puissance s'exerce donc avec les autres, dans une relation. Les réalistes classiques pensaient qu'on pouvait prendre des critères et établir un classement. Morgenthaw identifiait trois critères: la géographie (ressources liées au territoire de l'État) ; la démographie (les ressources liées à la population de l'État) ; le potentiel militaire. Morgentghaw pensait que ces facteurs conditionnaient la puissance de chaque État, c'est une approche qui a suscité toutes sortes de critiques. \\
La puissance ne se mesure pas à la force des armes. Même avec un potentiel militaire évident, on peut ne pas gagner (Vietnam, Irak, Afghanistan). Le potentiel militaire peut être même contre productif. Certains auteurs néo-réalistes parlent du dilemme de sécurité: quand un État augmente sont potentiel militaire, les autres États vont y voir une menace pour leur propre sécurité, et vont donc être incité à faire de même, lancer une course à l'armement et augmenter le climat d'insécurité. \\
Morgenthaw place l'économie dans la géographie et dans la démographie, ce qui n'est peut être pas suffisant. Les néo-réalistes continuent de penser que la puissance économique va de pair avec la puissance militaire. \\
Aujourd'hui, on pense qu'on peut évaluer la puissance d'un pays dans un domaine, mais ces domaines sont autonomes. 


Les transnationalistes montrent qu'il n'y a pas que l'économie et le militaire. Ils montrent que d'autres formes de puissance peuvent conditionner les puissances précédentes. Ils développent donc l'idée d'un hard power (puissance qui peut s'exprimer par la coercition) et d'un soft power (puissance non coercitive, persuasion, séduction, capacité d'influencer indirectement le comportement des autres) (Nye). \\
Il faut ensuite noter que la puissance est quelque chose d'éphémère. Toutes les grandes puissances ont connu un âge d'or et puis un déclin. Le monde a connu plusieurs configurations de puissance avec un, deux, ou plusieurs pôles de puissances: le monde peut être unipolaire, bipolaire ou multipolaire. Un pôle est quelque chose vers lequel toutes les forces sont attirés. \\
Chez les réalistes, la configuration de la puissance détermine les actions des États. C'est de cette configuration que tiendrait la stabilité, car, en fonction du nombre de pôle, il peut exister un certain équilibre. \\
Cette notion d'équilibre, est aussi empruntée aux sciences physiques, c'est l'idée de forces égales qui s'annulent en s'opposant. D. Hume a écrit sur l'équilibre des puissances (au XVIIIe, c'était un diplomate britannique). Il explique que la recherche d'équilibre est une tendance constante dans le développement des RI. Les États, pour développer leur sécurité, mais surtout pour empêcher que l'un d'eux dominent et s'emparent des autres, ils ont tendance à créer des alliances de forces égales. Dès qu'un pays a trop de puissance, les autres États ont tendance à se liguer contre lui. \\
Morgenthaw a été séduit par cette idée d'équilibre: il y a intérêt, pour un État, de se rallier au moins puissant. 

\section{De la fin du Moyen-Âge aux deux guerres mondiales: l'équilibre multipolaire européen}

La fin du Moyen-Âge correspond à l'émergence des premiers États moderne et aux grandes découvertes et la colonisation. Au XIXe, le congrès de Vienne (1814-1815), réorganise l'Europe à la fin de l'empire Napoléonien. Après ceci, l'Europe domine, via la colonisation la majorité du monde connu. \\
Les deux guerre font passer d'un monde multipolaire à un monde bipolaire. 

\subsection{Les caractéristiques de l'équilibre européen}

On a pu constater une première application du principe d'équilibre des puissances. À partir du congrès de Vienne, il y avait un équilibre à cinq puissances: Grande Bretagne, France, Prusse, Autriche, Russie. À chaque fois qu'un État a commencé à s'imposer, à dominer l'Europe, les autres ont toujours fini par se liguer pour briser ses prétentions. \\
La Grande Bretagne a joué un rôle un peu à part. À partir du moment où la GB a renoncée à ses possessions sur le continent, la GB s'est recentrée sur l'outre mer et est devenu une sorte de gardienne de l'équilibre: elle n'a jamais subi de coalitions mais les a toujours supportés, voire menés. \\
Cet équilibre obligeait à nouer des alliances, à créer des règles communes. Cela n'a pas pour autant empêché les guerres, bien au contraire. \\
Les alliances entre États sont à géométrie variable, ce qui est normal dans un système multipolaire. \\
Cet équilibre a assuré des périodes de trêve mais jamais de paix. Par définition, dans un monde multipolaire, la paix ne peut se conserver que par la force, contre celui qui essaye de dominer les autres. \\
Pour être certain que l'équilibre existe toujours, il y a besoin qu'il y ait des confrontations régulières. 

\subsection{De l'âge d'or au déclin}

La puissance de l'Europe a commencée à décliner assez tôt, sans que les acteurs n'en prennent conscience. Dès la fin du XIXe siècle, on a vu émerger de nouvelles puissances en dehors de l'Europe: les États-Unis vers 1870 qui sont devenus industrialisés et qui étendent leur influence vers l'Amérique du sud. \\
Le Japon aussi, un des rares pays à avoir éviter la colonisation. À partir des années 1860, le Japon cherche à se moderniser, ainsi qu'à étendre son influence, en étant expansionniste du côté de l'Asie notamment. \\
À partir du moment où les européens ne dominent plus le monde, leur équilibre ne peut suffire pour équilibrer les RI. \\
Avec la première guerre mondiale, les européens se sont rendus compte que leur équilibre ne marchait plus. Pour mettre fin aux hostilités, les européens ont dû laisser des puissances étrangères arbitrer leurs propres intérêts (intervention militaire US). \\
De plus, à ce moment, il y a la révolution Bolchevique qui bouscule les RI. \\
À partir de la première guerre mondiale, les Européens ont cessés d'être sujet de leur histoire et son devenus des objets d'enjeux extérieurs. 


Dans l'entre deux guerres, il y a eu un système un peu trompeur. Les européens s'accrochaient à leur ambition. L'équilibre n'était plus multipolaire, la confrontation US/URSS s'annonçait déjà. Il y avait en quelque sorte trois pôles, un camp occidental, un camp soviétique, un camp fasciste. \\
Ces trois camps s'équilibraient plus ou moins, et c'est l'alliance de deux de ces camps qui permettent de détruire le dernier. 


\section{1945-1989: l'ordre bipolaire}

Marqué par la disparition du multipolaire. Opposition entre deux superpuissances. L'opposition est totale: idéologique, économique, militaire... \\
Pendant toute cette période, tout ce qui se passait dans les RI étaient interprétés dans la logique des deux blocs. 

\subsection{La politique des blocs et la confrontation Est/Ouest}

La guerre froide dure de 1947 à 1962, suivie de la détente. La fin de la seconde guerre mondiale annonçait la guerre froide. En 1947, sont arrivés au pouvoir les communistes en Pologne, en Roumanie, etc. \\
C'est pendant ces années que les deux superpuissances ont pris conscience de cette bipolarité et de ses conséquences. Ils ont, pendant cette première période défini des "règles du jeu". 


\subsubsection{Les règles du jeu de la bipolarité}

La première est la création de blocs. Pendant la multi-polarité, les alliances étaient à géométrie variable. Ce n'est plus le cas dans un système bipolaire. Chacune des superpuissance a intérêt à ce que ses alliés forment un bloc, qui sont loyaux, ou du moins la superpuissance doit veiller à ce que ce soit le cas. \\
Les tiers doivent choisir leur camp, surtout si ils sont dans une "zone vitale". Dans la guerre froide, les deux superpuissances se donnaient le droit d'intervenir pour maintenir leur cohésion, tout en fournissant l'aide nécessaire pour éviter l'attirance de l'allié pour son ennemi. \\
Doctrine Truman, c'est la doctrine du containment qui consiste à endiguer la progression du communisme dans les limites issus de la seconde guerre mondiale. Le plan Marshall est dans le même état d'esprit. \\
Dans le camp soviétique, on a la même doctrine: la doctrine Jdanov. Pour lui, chaque pays devait forcément choisir son camp et tout ceux qui exprimeraient des divergences avec l'URSS seraient contre elle. Doctrine Brejnev: l'URSS se donne le droit d'intervenir dans tout pays membre de son bloc qui souhaiterait le quitter.


Autre règle: chaque superpuissance ne devait pas intervenir dans le camp adverse. Dans un monde bipolaire, pour maintenir malgré tout un équilibre, il faut qu'il y ait une entente tacite entre les deux superpuissances pour que chacune laisse l'autre libre d'agir à sa guise dans son bloc. Ils ont d'ailleurs chacun un intérêt en commun, maintenir sa suprématie sur toutes les autres. Tant que la victoire n'est pas possible, il vaut mieux garder la bipolarité. Concrètement, il ne faut pas prendre de risques inutiles. \\
En 1947, ce principe n'était pas clair, jusqu'en 1956, des troupes Soviétiques avaient étés envoyées en Hongrie pour éviter la transition démocratique et malgré l'appel à l'aide des Hongrois, les US n'y sont jamais allés. 


Troisième et dernière guerre: la non participation des superpuissances dans les querelles de leurs alliés. Les superpuissances n'ont pas à aller risquer l'affrontement direct si ce n'est pas vital pour eux. \\
En 1956, avec la deuxième guerre Israelo-Arabe, la France et la GB s'engage avec Israël contre l'Égypte, allié à l'URSS. Les deux superpuissances n'ont pas intervenus l'un contre l'autre. 

\subsubsection{Une stabilité très relative}

Pour les auteurs réalistes, la bipolarité offre le meilleur équilibre. Il y a de nombreux avantages: nombre de joueurs réduits dont on réduit les incertitudes. L'intelligibilité du monde est augmenté. Tout cela est très satisfaisant pour des auteurs réalistes. \\
Il y a donc une certaine stabilité. "Paix impossible mais guerre improbable", Raymon Aron. Si les pays considérés comme vitaux n'ont pas connus de guerre, on a vu une multiplication des guerres de périphérie: la guerre de Corée par exemple. Tout ces conflits n'étaient pas uniquement dans logiques Est/Ouest, notamment les guerres d'indépendance, les guerres israelo-arabe. Cela dit, à chaque fois, les protagonistes ont reçu le soutien de l'une ou l'autre des superpuissances. \\
À la suite de la crise des missiles de Cuba, la guerre froide a cédé à une période légèrement différente: la détente. La confrontation est devenue plus indirecte. Leurs dirigeants se rencontraient et co-géraient à deux l'équilibre du monde. Cela n'a pas changé pour autant la nature de l'équilibre du monde. Aucune des deux superpuissances n'avaient abandonnés l'idée de dominer l'autre. La guerre du Vietnam ou d'Afghanistan sont les exemples symétriques. \\
Les superpuissances ont continués de s'ingérer dans les pays de son bloc car cela semblait encore moins dangereux qu'avant à cause du dialogue qui s'était mis en place entre les deux pays. \\
Le processus de détente s'est quand même essoufflé vers les années 70-80, car les superpuissances avaient l'impression que chaque bloc avançait, les rendant alors plus agressifs. Raegan avait donc remis en oeuvre l'endiguement en se faisant élire. La tension remonte donc.


La bipolarité a été fortement remise en cause. On a vu s'esquisser un nouvel axe. 

\subsection{Les incidences de la décolonisation et la formation d'un axe Nord/Sud}

Dans les années qui ont suivi la révolution Chinoise, la Chine s'est retrouvée alliée avec l'URSS. Cependant la Chine n'appréciait pas la tutelle de l'URSS, et a donc forgé une alliance avec les US dans les années 60. La logique des blocs s'est trouvé relativisé, un changement d'alliance a été possible. \\
Le changement le plus marquant a été l'irruption sur la scène internationale de pays nouvellement créés à la suite de la décolonisation. 1960: on a vu une douzaine d'États apparaître d'un seul coup. Les pays nouvellement créés sont devenus majoritaires à l'AG de l'ONU. Ces nouveaux États ont cherchés à échapper à la logique bipolaire. 

\subsubsection{Des pays en voie de développement au tiers monde, diagnostics et solutions}

Il fallait trouver un nom pour trouver ce que représentaient ces nouveaux pays. On a d'abord parlé de pays sous développé, puis pays en voie de développement. \\
Une expression s'est imposé dans les années 60-70: pays du Nord, pays du Sud. Cela ne correspond pas tout à fait à la réalité mais a le mérite d'être neutre. Cela permet de dépasser clairement la ligne bipolaire: il existe un autre clivage que l'est/ouest. \\
Ensuite, est apparu l'expression "tiers monde", qui fait référence au tiers État de la Révolution française, coincé entre deux ordres qui possèdent le monopole politique mais ne représente pas grand chose. \\
Les pays du tiers monde ont envisagé leur développement et développé leurs propres diagnostics. Le principal obstacle au développement de ces pays est le système économique international car rend dépendant les pays du tiers monde aux autres systèmes économiques. \\
L'analyse dominante est que les pays du tiers monde ont besoin d'exporter pour investir, mais les exportations de ces pays reposent sur les matières premières et sont obligés d'importer les produits industrialisés. Or, le cours des matières premières est soumis à fortes fluctuation, les pays du tiers monde ne contrôlent donc pas leurs prix. Il a sorti de tout ça qu'il était recommandé aux États du tiers monde de mener une économie Keynésienne, de développer le service public en développant l'industrialisation. \\
Les pays du tiers monde revendiquaient un NOEI. Cela reposait sur une nouvelle conception de la souveraineté étatique en mettant l'accent sur une idée de souveraineté économique. Ils cherchaient à pouvoir nationaliser des transnationales sur leur territoire, à contrôler le cours des matières premières, etc. Ils réclamaient une aide financière, matérielle, mais non conçu comme une forme de charité mais comme une créance, une dette de l'histoire. \\
Le relatif succès qu'a eu le tiers monde tient du fait qu'ils étaient d'abord majoritaire, puis les superpuissances cherchaient aussi à les séduire, à lâcher du lest. CNUCED, créé en 1964, devait être le cadre où serait renégocié le commerce sur l'axe Nord/Sud. \\
En 1974, l'AG de l'ONU a adopté un programme d'action pour NOEI. \\
La banque mondiale a participé à ce dynamisme tiers-mondiste. Sa mission était d'accorder des prêts à très long terme aux pays du tiers monde tout en fournissant des conseils d'expert. 

\subsubsection{Le mouvement des non alignés}

C'est le mouvement le plus sérieux pour remettre en cause la bipolarité. Il date de la conférence de Bandoeng en 1955, pour affirmer des principes communs. En 1961, le sommet de Belgrade à lieu en Yougoslavie, c'est là qu'a été créé le mouvement des non alignés sous l'impulsion de Nasser, Neru, et Tito. \\
La Chine a voulu s'imposer comme leader ou tuteur des non alignés, elle a participé à la création du mouvement mais a perdu en influence avec le grand bon en avant. \\
Ce mouvement cherchait à lutter contre la ségrégation, contre le colonialisme, soutenait l'idée du NOEI. Il s'agissait avant tout toute dépendance par rapport aux deux blocs. Ils ne prétendaient pas pour autant être neutre. \\
Ce mouvement s'est vite effrité, et ses membres se sont rapprochés de l'URSS ou des US. 

\subsubsection{La fin des années 70 et la remise en cause du tiers-mondisme}

On s'est rendu compte que le modèle de développement ne donnait pas tous les résultats escomptés. Les aides extérieurs s'étaient traduites par du gaspillage, du double emploi, pour enrichir des dictateurs. \\
Beaucoup d'intellectuels ont avancés que l'analyse de l'époque était entièrement fausses. Ils disaient que le colonialisme n'était pas la cause des problèmes de développement dans les pays du tiers monde mais la corruption, ainsi que d'autres causes endogènes. \\
Au cours des années 70, les pays du tiers monde se sont énormément endettés. Les banques privées ont eu énormément d'argent à placer grâce aux pétrodollars, mais la montée du pétrole a créé une crise économique, obligeant les US à augmenter le taux d'intérêt, triplant les dettes des pays du tiers monde. \\
Le FMI s'est impliqué dans l'aide aux pays du tiers monde en lançant des programmes d'ajustement structurel. \\
Dans beaucoup de pays, ces mesures ont eu des effets désastreux: déclin de la production, des infrastructures, etc. Quelques pays ont connus un essor remarquable en attirant sur leurs territoires des entreprises délocalisés: Taïwan, Hong-Kong, Corée du Sud. 

\section{Le monde depuis 1990: ordre ou anarchie ?}

\subsection{L'effondrement du bloc soviétique et ses conséquences}

Contrairement à ce qui était prévu, le bloc soviétique s'est effondré de l'intérieur de manière étonnement rapide et pacifique. \\
Gorbatchev arrive au pouvoir en 1985 et lance des réformes économiques et politiques. En 1988, les autorités soviétiques abandonnent la théorie de la souveraineté limitée et annonce qu'elle n'interviendra plus pour soutenir des régimes contestés. Les transitions se sont enchaînés, avec la Pologne qui a commencé en 1989, puis chute du mur, transitions dans les autres pays pour finir par la dissolution du pacte de Varsovie en 1991. \\
En Août 1991, une tentative de putsch fait que le PC a été dissout, Gorbatchev a démissionné et fait disparaître l'URSS en tant qu'État. \\
Le système soviétique étant en crise depuis longtemps, cet effondrement était inévitable. L'URSS n'arrivait pas à assurer la cohésion de ses différentes nationalités et connaissait des crises économiques de plus en plus difficile. 

\subsubsection{L'annonce d'un "nouvel ordre mondial"}

Il y a une courte période qui commence lorsque Gorbatchev arrive au pouvoir et qui finit vers 1993. On a pensé pendant cette période que le monde devenait plus sûr, plus cohérent, plus harmonieux. À partir du moment où l'URSS a relâché la pression, les US étaient incités à faire de même. On a donc pensé que les conflits périphériques allaient s'éteindre. \\
Lorsque les pays de l'est ont annoncés leurs transitions, il y avait l'idée que la démocratie et l'économie de marché allait s'imposer partout. \\
On se disait que l'ONU allait pouvoir enfin jouer un vrai rôle sans être paralysé par la Russie ou les US. \\
La vague de démocratisation a en effet touché certains pays, notamment en Afrique et en Asie. À l'échelon international, il y a eu une conférence Israelo-Arabe réuni à Madrid, ce qui a abouti à la reconnaissance d'une autorité Palestinienne. \\
Gorbatchev a été le premier à lancer la formule du nouvel ordre mondial, mais elle a été connu lorsque Bush père l'a utilisé. \\
Fukuyama parlait de "La fin de l'histoire et le dernier homme". Il constate que la démocratie libérale et l'économie de marché sont devenus les seules solutions possibles dans les sociétés modernes. Ces solutions sont déjà en train de triompher d'où cette idée de fin de l'histoire, au sens où on est à la fin d'un processus. Il défend aussi l'idée d'une homogénéisation des cultures.

\subsubsection{Le désenchantement des années 1993-1994}

Le clivage idéologique dominant avait certes disparu, mais cela allait faire place à d'autres choses. Huntington s'est opposé à Fukuyama. En 1993 et en 1996, il parlait du choc des civilisations. Il voit le monde comme conflictuel. \\
Il dit que la mondialisation et la diffusion des produits de la mondialisation n'a rien à voir avec la diffusion des valeurs. \\
Son idée c'est que avant, les clivages étaient idéologiques, économique, politique, mais que désormais, ils sont culturels. Il identifie 7 grandes civilisations humaines. Il les définit essentiellement avec des critères religieux qu'il recoupe avec la géographie. \\
Il explique que les conflits les plus durs ne seront pas ceux entre toutes mais entre certaines. \\
Il n'est donc pas sûr que la configuration post-bipolaire soit plus stable, mais pas sûr non plus qu'elle soit aussi conflictuelle. \\
Ce qui est sûr, c'est que le monde est plus complexe, moins intelligible. Entre autres parce qu'on ne sait pas où trouver l'équilibre qu'on avait pendant la guerre froide. 

\subsection{L'hégémonie américaine et ses limites}

Militaire: US loin devant tous les autres. Économie: US devant, en tête à tête avec la Chine, cela dépend de comment on calcule le PIB (nominal ou PPA). De plus, dollar a une hégémonie. Soft power: US loin devant. \\
Les US sont ils menacés dans leur hégémonie ? 

\subsubsection{Un ordre unipolaire ou multipolaire ?}

Les États-Unis sont le seul État qui cumule encore tous les attributs de la puissance. Si il a des concurrents sérieux, aucun ne l'égal dans tous les domaines. 

































\end{document}
