\documentclass[10pt, a4paper, openany]{book}

\usepackage[utf8x]{inputenc}
\usepackage[T1]{fontenc}
\usepackage[francais]{babel}
\usepackage{bookman}
\usepackage{fullpage}
\setlength{\parskip}{5px}
\title{Cours de Politique internationale (UFR Amiens)}
\pagestyle{plain}


\begin{document}
\maketitle
\tableofcontents

\chapter{Introduction générale}

\section{L'étude des relations internationales: du Droit à la Science Politique}

Le concept de relation internationale est vaste: il renvoie aux relations de toute nature qui traversent les frontières. Ces relations ont pour principale caractéristique d'échapper au cadre d'un seul État. \\
Si on retient ce champ très large, le domaine de la politique internationale est illimité puisqu'aujourd'hui, aucun État ne peut plus vivre en autarcie, sans subir des contraintes extérieurs (même la Corée du Nord subit certaines contraintes extérieures). \\
Les frontières ne sont jamais complètement hermétiques: flux de personnes, de biens, de capitaux, d'idées, d'informations, etc. Les menaces aussi traversent les frontières, et comme elles les traversent, elles ne peuvent être réglées que par de la coopération internationale. \\
Le champ des relations internationales a cependant des caractéristiques propres. La principale caractéristique du système international est qu'il n'existe pas d'autorité centrale qui pourrai adopter des normes universelles et les faire appliquer. \\
Les relations entre les États se sont institutionnalisés: diplomatie, traités, organisations internationales. Ces organisations édictent d'ailleurs des normes, créant des organes juridictionnels. Cependant, il n'existe pas de pouvoir législatif universel, ni de pouvoir judiciaire obligatoire et universel (la CPI dépend du consentement des États et ne regroupent pas du tout tous les États du monde, les US, la Chine ne sont pas membre de la CPI).


Cependant, on a toujours cherché à comprendre les relations internationales. Les chercheurs ne se posent pas toujours les mêmes questions, ne suivent pas les mêmes méthodes, etc. Les premiers à s'intéresser aux relations internationales étaient les philosophes, d'Aristote à Rousseau en se posant les questions de la guerre et de la paix. Ils ont donc une approche très spéculative et abstraite, mais ont donné un apport non négligeable. \\
Les historiens aussi s'intéressent aux RI, et on peut remonter jusqu'à Hérodote, qui avait étudié la guerre entre les Grecs et les Perses. L'histoire permet d'éclairer un certain nombre d'événements, et contrairement à la philosophie, elle se base sur des données matérielles, ce qui en fait aujourd'hui une discipline à part entière: l'histoire de la diplomatie. \\
Ensuite, on a une approche juridique, plus récente. Hugo de Groot (dit Grotius), un juriste, a permis l'existence de traités sur les droits de la mer et a contribué au XVIe siècle à faire du droit international un droit positif. Entre la fin du XIXe et le XXe siècle, quand on a vu les premières OI se créer, le droit international s'est institutionnalisé. Cependant, le droit international est une science des normes, et non des faits. Le DI cherche à identifier les règles, à les interpréter, si elles sont respectées. Le DI ne s'intéresse pas au pourquoi.


C'est pourquoi le politiste a cherché à expliquer les phénomènes internationaux. C'est l'approche la plus récente, le premier enseignement de politique internationale était en Grande Bretagne après la première guerre mondiale. La politique internationale s'est surtout imposée après la deuxième guerre mondiale, surtout aux USA, parce que c'est à la fin de cette guerre que les US se sont acceptés comme superpuissance, et, avec la guerre froide, les US avaient une réflexion à faire, le gouvernement investissait d'ailleurs énormément dans les universités. \\
Les fac américaines venaient justement d'accueillir un grand nombre d'intellectuels européens qui ont massivement investi le domaine de la PI, se sentant concerné. \\
La PI s'est tout de même diffusée, des recherches ont étés lancées un peu partout sur des thèmes divers: le tiers monde, les mobilisations transnationales, etc.


Dans la PI, il s'agit avant tout de chercher des relations de causes à effet. On va donc se demander qui sont les principaux acteurs, quels sont les facteurs qui déterminent leur comportement. 

\chapter{La nature du système international: une vieille controverse théorique}

La question de la nature du système international a toujours fait l'objet d'une querelle entre les chercheurs mais aussi entre les acteurs politiques. Cette querelle a eu des répercussions pratiques. De fait, aux US, les liens entre les milieux universitaires et le milieu gouvernemental est très étroit, le gouvernement US investit massivement, lui permettant de pousser la recherche vers les questions qui l'intéresse. \\
Aux US, les universitaires sont souvent amenés à exercer des responsabilités publiques: conseiller voire membre du gouvernement. De plus, les Think Tanks sont très nombreuses aux US, ce sont des structures de droit privés, non lucratives, dont l'objet est de développer des idées et des connaissances pour guider l'action publique. \\
Ces débats théoriques exercent donc une influence sur la gestion des problèmes concrets. 


La question qui divise est de savoir si il existe un certain ordre dans le fonction du système international ou si il est par nature anarchique. \\
L'approche qui a longtemps dominé et qui exerce toujours une place de premier plan est celle de l'anarchie. 

\section{L'approche réaliste: un cadre d'analyse dominant}

C'est une approche qui a le mérite d'être simple et rationnelle, c'est une approche qui veut voir le monde tel qu'il est et non tel qu'on voudrait qu'il soit. Cela implique de se démarquer de l'idéalisme, du moralisme et de mettre l'accent sur les rapports de force. \\
Cette approche va de pair avec un certain conservatisme, un statut quo, et a donc exercé une très forte influence sur les Républicains aux US, notamment Bush père et surtout Nixon. Elle a aussi trouvé quelques adeptes du côté démocrate, Obama, en pratique, par sa prudence avait un côté réaliste par sa volonté de ne pas trop bousculer l'ordre établi. Joe Biden et d'autres conseillers d'Obama se revendiquaient réalistes. \\
Dans le cas de la France, tous les Présidents depuis le début de la Vème République ont menés une politique réaliste.

\subsection{Les précurseurs}

Les premiers précurseurs de l'approche réaliste apparaissent dès le XVIe-XVIIe siècle avec Machiavel d'abord qui cherchait à séparait l'analyse de la réalité et la morale et qu'il voyait la société internationale dominé par les rapports de force et les intérêts des États. \\
Ensuite, on a Thomas Hobbes dont le postulat de base est que chaque homme est le concurrent de l'homme. Il développe l'idée que pour échapper à l'état de nature libre et dangereux, l'homme aurait passé un contrat social pour créer un pouvoir politique en échange de la paix et de la sécurité. Il ajoute que dans l'ordre international, il n'y a pas de contrat social possible, car cela signifierait de renoncer à la souveraineté et voit donc le système international comme un état de nature éternel et anarchique. \\
Cette approche réaliste s'est perpétuée. Au tout début du XXème siècle, il s'est imposé une conception idéaliste et pacifiste des relations internationales, qui a eu du mal à résister à la chute de la SDN et à la montée du nazisme.

\subsection{Le réalisme depuis le XXème siècle}

Le réalisme a triomphé après la seconde guerre mondiale. Au milieu du XXème siècle, les travaux des américains ont servi de justifications de la politique internationale des US. \\
Hans Morgenthau est considéré comme le principal héritier de Machiavel ou Hobbes. Il voulait créer une vraie science des relations internationales. Pour lui, les États sont tous en quête de puissance, à cause de la nature humaine. \\
Henry Kissinger, c'est le réaliste par excellence. Il voulait débarrasser la PI de toutes les valeurs idéologiques et morales pour ne prendre en compte que les logiques de puissance. Il n'a plus aucune fonction officielle mais continue d'être très actif en prenant position, en faisant des conférences, etc. \\
Raymond Aron, un français, a écrit "Paix et guerre entre les Nations". C'est lui aussi un réaliste qui dit du système international qu'il est anarchique. Il dit que la société internationale est une société asociale. 


Les réalistes estiment que le système international et le système national sont complètement différent, le système international étant par nature anarchique. \\
Pour les réalistes, les seules acteurs des RI sont les États et négligent les autres acteurs comme les ONG, les entreprises, etc. ou les voient seulement comme des instruments de leur État de référence. \\
Les États sont rationnels selon les réalistes, et sont aussi unitaires, agissent comme un seul homme. L'État agirait selon un calcul coût/avantage, et sont vus comme foncièrement égoïste. L'intérêt national serait de conserver ou croître sa puissance. L'État serait donc par définition amorale. \\
Les facteurs économiques et culturels seraient de moindre importance selon les réalistes par rapport à la stratégie. \\
Par nature, le système international est conflictuel car les intérêts s'opposent. \\
Pour les réalistes, l'histoire n'a pas de fin, n'a pas d'issue: c'est un éternel recommencement où le progrès n'est pas possible. Quelques règles sont apparus si quelques États, les plus puissants, y consentent car ce serait dans leur intérêt. \\
Pour eux, la paix n'est pas possible, il n'existe que des trêves. Un seul moyen existe pour conserver cette trêve: l'équilibre des puissances. \\
On résume les réalistes par le fait que les RI sont comparables à un jeu de billard où les États sont les boules de billard: elles sont lisses et on ne prend pas en compte ce qu'il y a dedans. 


Une grande critique de cette approche est qu'elle est très rigide car s'intéresse toujours aux États, et a donc eu beaucoup de difficultés à expliquer le terrorisme. Face aux critiques, il y a eu un néo-libéralisme. \\
Ce néo-libéralisme, dont l'un des principaux penseurs est Kenneth Waltz, s'intéresse davantage à l'économie et aux acteurs privés. 

\subsection{Une théorie à l'épreuve du réel: l'interprétation réaliste de l'intervention américaine en Irak de 2003 et de ses suites}

Officiellement, l'objectif de Bush fils était de libérer le peuple Irakien et de propager la démocratie. Les réalistes ne peuvent pas accepter un tel argument, sauf si c'est juste un habillage pour obtenir le soutien des autres puissances. \\
Certains disaient que l'intervention était pour le pétrole. Si c'est le cas, c'est tout à fait envisageable pour les réalistes car les gisements pétroliers peuvent être vus comme une ressource hautement stratégique. Cependant, ce ne peut qu'être un seul facteur, les US produisant déjà beaucoup de pétrole. \\
Une autre idée est que les US devaient s'attendre à avoir des rivaux, et qu'ils devaient les éliminer, pour dissuader les autres rivaux. Les attentats du 11 Septembre ont montrés d'ailleurs que la sécurité des US était menacée. L'intervention armée permettrait donc aux US de ne pas assurer sa sécurité seulement par du défensif, ainsi que de donner un certain exemple en attaquant soi disant le mal à la racine. \\
Une question se pose: pourquoi l'Irak ? Pourquoi pas l'Iran, la Corée du Nord ? En 2003, l'Irak était assez isolé, avait peu d'alliés. Les américains connaissaient le terrain grâce à la guerre du Koweit de 1991 et l'Irak était faible à cause de cette défaite ainsi que des embargos que le pays avait subi. \\
De plus, le Moyen-Orient a toujours été une zone stratégique, mais aussi une zone fortement instable. Une autre donnée était que les US n'étaient plus sûr de leur allié dans la région (l'Arabie Saoudite) et aurait pu compter sur l'Irak sans Sadam. \\
Un réaliste critiquera la décision si il la juge inopportun, et, dès le départ, les réalistes étaient très divisés sur la décision. \\
Au final, l'intervention Américaine en Irak a été largement contre productive, ternissant l'image des US, se mettant à dos ses alliés et permettant des groupes terroristes de s'installer en Irak. \\
Les réalistes peuvent dire de cette opération que les calculs de base étaient erronés, les moyens engagés n'étaient pas les bons. 

\section{Des approches alternatives}

\subsection{L'approche libérale}

Cette approche contient plusieurs variantes, plusieurs courants. Ces courants tendent à relativiser le rôle des États pour prendre en compte les rôles des acteurs non étatiques. Ils mettent l'accent sur les facteurs de solidarité qui existent. Pour eux, un ordre pourrait émerger, l'anarchie internationale n'est pas absolue. \\
Les variantes sont le fait que les auteurs divergent sur les facteurs, certains parle de la Raison, pour d'autres, c'est le commerce, d'autres, les normes et les OI. 


L'approche libérale vient du premier libéralisme qui a inspiré la DDHC. Des auteurs croyaient au progrès et à la raison. Pour ces auteurs, les premiers acteurs sont les sociétés civiles composés d'individus. Pour eux, l'État n'est qu'un intermédiaire entre ces sociétés et la société internationale. \\
Pour d'autres, la progression des droits peut arriver à la création d'un ordre mondial, pacifique. \\
Pour certains, le libéralisme économique intervient. Pour ces auteurs, si les échanges sont amplifiés, alors les États n'ont pas intérêt à se faire la guerre. \\
C'est Kant qui parle de la raison qui doit gagner sur la force brute (Traité pour une paix perpétuelle). Kant proposait une fédération mondiale, qui permettrait de lier tous les États par des règles de droit commune. Kant comptait aussi sur la diffusion de la démocratie, car, pour lui, les démocraties ne peuvent pas se faire la guerre entre elles. 


Ces idées ont connu leur plus grand succès dans l'entre deux guerres. Pendant quelques années, l'approche libérale a réussi à s'imposer dans l'analyse des RI, mais dans sa version la plus idéaliste, et donc probablement la plus naïve. \\
Concrètement, cette tendance a influencée le Président US Wilson, père de la SDN. \\
Avec la montée du fascisme, l'éclatement de la SDN, et la seconde guerre mondiale, cette approche libérale a explosé.\\
Le transnationalisme est un renouvellement de cette approche libérale, débarrassée d'une bonne part de son idéalisme. Ce transnationalisme a eu une certaine influence, notamment sur Bill Clinton qui voulait revitaliser les OI, le commerce, pour assurer la paix et conserver une domination Américaine. \\
Jean Monnet et son entourage ne croyait pas aux utopies fédéralistes à court terme. Par contre, il pensait que la mise en place d'une coopération technique permettrait de dépasser les intérêts nationaux et de lancer un engrenage de fédéralisme. \\
Dans cette approche, les États ne sont pas les seuls acteurs des RI. Il y a tout un tas d'autres acteurs: des entreprises, des organisations, des OI, qui, justement, sont transnationales. \\
Dans cette approche, il faut une analyse plus fine, qui cherche à analyser les réseaux publics/privés. \\
Robert Keohane et Joseph Nye ont étés les premiers à insister sur le rôle que jouent ces acteurs non étatiques, mais aussi sur les influences mutuelles que va jouer la politique interne, qui, à contrario de la pensée des réalistes, jouent un rôle pour les transnationalistes (Linkage). \\
Pour les transnationalistes, les facteurs économiques, culturels, identitaires, sont aussi important que le facteur de la puissance militaire. \\
Les transnationalistes rejoignent un peu les libéraux car pensent que le monde peut sortir de cette anarchie perpétuelle. Ils disent donc que la multiplication des réseaux transnationaux créent une interdépendance qui est censé les amener à coopérer. \\


De fait, la mondialisation a accéléré ces relations d'interdépendance. C'est pour cela qu'on a créée la notion de "gouvernance mondiale". Ce mot, gouvernance, désigne la fonction de gouverner sans qu'il existe un véritable organe dont la fonction est de gouverner (le gouvernement désignant cet organe). \\
Il existe des variables d'ajustement qui peuvent s'imposer aux acteurs, ces variables jouant le rôle de la gouvernance. Ces variables résultent de l'interaction de tous ces acteurs. \\
L'idée est donc qu'il y a une toile de relations qui relie tous les acteurs de la planète.


Cette vision est très neutre et dépolitisé des rapports. En cherchant à démontrer la complexité des choses, en cherchant à démontrer la gouvernance, les transnationaux oublient ce qui relève de la coercition, de la domination, des inégalités. \\
En gros, tous les acteurs se valent dans cette approche. On retrouve un peu l'idée de la main invisible pour les libéraux économiques. 

\subsection{L'approche marxiste}

Cette approche prend le contre-pied de l'approche libérale, sans revenir au réalisme. Aujourd'hui, cette approche est un peu plus marginale que les précédentes, elle a été très forte dans les années 60-70. Elle reste assez présente dans les partis communistes et connaît un certain regain dans les approches altermondialistes.


Cette approche a ses racines dans les oeuvres de Marx. Pour Marx, le champ politique n'a pas de réelle autonomie, ce champ reflétant l'état du champ économique. Le mode de production économique (relations de travail et mode d'existence matériel) détermine tout le reste: les institutions, la vie politique, le droit, les idées. \\
Selon Marx, l'évolution économique s'est manifesté par une évolution de la division du travail donc une création de classes sociales. Pour Marx, il y a toujours eu une classe dominante qui possède les moyens de production. Marx pensait que le système capitaliste allait être victime de ses contradictions, et conduirait à une disparition des classes, et donc à l'État. \\
Pour Marx, les bourgeoisies nationales allaient petit à petit, mondialiser leur action et que cela allait amener leurs États respectifs à des conflits armés. \\
Certains successeurs de Marx ont repris certain des postulats pour les prolonger. Lénine a créé la notion d'impérialisme, stade ultime du capitalisme, qui se caractérise par une concentration à l'échelle internationale des moyens de production dans les mains de quelques groupes industriels et financiers qui rivalisent pour se partager le monde. Pour Lénine, les États seraient à la solde de ces groupes, créant donc des conflits mondiaux, créant donc une révolution commune des prolétaires de tous les pays. \\
Il faut attendre le XXe siècle et la décolonisation pour que ces idées se renouvellent. Il y a eu des néo-marxistes. Samir Amin, un auteur franco-égyptien, a écrit "Le développement inégal", où il explique le système international, c'est un centre et une périphérie. Le centre, ce sont les pays développés et la périphérie, les autres. Il explique que le centre a toujours eu tendance à exploiter la périphérie. Du coup, la périphérie est maintenu dans une dépendance. \\
Pour Samir Amin, le sous développement s'explique par cette exploitation. Une division du travail international obligerait les pays des périphéries à exploiter les matières premières.


Les Marxistes ont une vision qui est beaucoup plus dé-centrés. Le problème de la guerre et de la paix n'est qu'un sous problème de la division des richesses dans ce point de vue. \\
Les Marxistes s'accordent avec les réalistes: les RI sont conflictuelles. Les néo-marxistes ne pensent pas que les États soient les principaux acteurs. Pour eux, ce sont des instruments. Les acteurs seraient donc la bourgeoisie capitaliste, qui s'allient souvent avec les élites des pays en voie de développement. D'autres part, il y aurait les peuples, qui sont exploités de toutes part, qui ont du mal à s'allier entre eux, pensant leurs intérêts divergeant.

\subsection{Deux relectures de l'intervention américaine en Irak}

L'argument de propagation de la démocratie peut être sensible pour les libéraux. Cependant, sans autorisation de l'ONU, sans alliés de poids, sans soutien d'aucune OI, un libéral dénoncerait le caractère unilatéral de l'action militaire. \\
Pour les libéraux, la force armée n'est pas la meilleure façon de propager la démocratie. Dans les faits, les penseurs libéraux étaient fortement divisés. \\
Pour un transnationaliste, il faut observer la pression des acteurs économiques, etc. Le lobby pétrolier était pour cette intervention, comme le lobby Israélien. Un autre acteur international, ce sont les opposants au régime Irakien, en exil aux US. \\
Pour un libéral, l'État n'est pas un acteur unitaire. On peut donc s'intéresser aux rapports de force dans le cabinet américain: le gouvernement n'était pas unanime. George W. Bush n'était pas décidé au départ, mais l'échéance électoral, le contexte, etc, est déterminant dans l'approche libérale. 


Dans une approche Marxiste, on va dire que les US sont la puissance impérialiste par excellence, c'est le centre du centre. L'Irak est un État du Sud, convoité. C'est un pays de la périphérie. \\
Le principal objectif des US pendant longtemps, c'était le pétrole Irakien. Dans cette situation, la force militaire est un instrument permettant d'asseoir une domination économique. \\
Les US s'appuient toujours sur les Kurdes en Irak. Le Kurdistan est une région très riche en pétrole. \\
Si on est Marxiste, le fondamentalisme religieux est l'opium du peuple, détournant de la lutte des classes. \\
À priori, toutes les interventions militaires américaine en Irak sont condamnables, la victime étant le peuple. 


Aucune approche ne permet d'expliquer complètement les faits internationaux. Aujourd'hui on voit une approche constructiviste des RI, qui dit que dans l'absolu, la réalité n'existe pas, que la réalité n'a pas de sens, mis à part celui qu'on lui donne. 

\chapter{L'évolution des ordres internationaux: à la recherche de l'équilibre des puissance}

Comment définir la puissance d'un État ? Raymond Aron parle de sa capacité à imposer sa volonté aux autres. D'autres spécialistes s'accordent pour dire qu'il y a d'autres capacités: la capacité de faire des choses, la capacité de faire faire des choses aux autres, capacité de refuser de faire, et capacité d'empêcher de faire. \\
La puissance s'exerce donc avec les autres, dans une relation. Les réalistes classiques pensaient qu'on pouvait prendre des critères et établir un classement. Morgenthaw identifiait trois critères: la géographie (ressources liées au territoire de l'État) ; la démographie (les ressources liées à la population de l'État) ; le potentiel militaire. Morgentghaw pensait que ces facteurs conditionnaient la puissance de chaque État, c'est une approche qui a suscité toutes sortes de critiques. \\
La puissance ne se mesure pas à la force des armes. Même avec un potentiel militaire évident, on peut ne pas gagner (Vietnam, Irak, Afghanistan). Le potentiel militaire peut être même contre productif. Certains auteurs néo-réalistes parlent du dilemme de sécurité: quand un État augmente sont potentiel militaire, les autres États vont y voir une menace pour leur propre sécurité, et vont donc être incité à faire de même, lancer une course à l'armement et augmenter le climat d'insécurité. \\
Morgenthaw place l'économie dans la géographie et dans la démographie, ce qui n'est peut être pas suffisant. Les néo-réalistes continuent de penser que la puissance économique va de pair avec la puissance militaire. \\
Aujourd'hui, on pense qu'on peut évaluer la puissance d'un pays dans un domaine, mais ces domaines sont autonomes. 


Les transnationalistes montrent qu'il n'y a pas que l'économie et le militaire. Ils montrent que d'autres formes de puissance peuvent conditionner les puissances précédentes. Ils développent donc l'idée d'un hard power (puissance qui peut s'exprimer par la coercition) et d'un soft power (puissance non coercitive, persuasion, séduction, capacité d'influencer indirectement le comportement des autres) (Nye). \\
Il faut ensuite noter que la puissance est quelque chose d'éphémère. Toutes les grandes puissances ont connu un âge d'or et puis un déclin. Le monde a connu plusieurs configurations de puissance avec un, deux, ou plusieurs pôles de puissances: le monde peut être unipolaire, bipolaire ou multipolaire. Un pôle est quelque chose vers lequel toutes les forces sont attirés. \\
Chez les réalistes, la configuration de la puissance détermine les actions des États. C'est de cette configuration que tiendrait la stabilité, car, en fonction du nombre de pôle, il peut exister un certain équilibre. \\
Cette notion d'équilibre, est aussi empruntée aux sciences physiques, c'est l'idée de forces égales qui s'annulent en s'opposant. D. Hume a écrit sur l'équilibre des puissances (au XVIIIe, c'était un diplomate britannique). Il explique que la recherche d'équilibre est une tendance constante dans le développement des RI. Les États, pour développer leur sécurité, mais surtout pour empêcher que l'un d'eux dominent et s'emparent des autres, ils ont tendance à créer des alliances de forces égales. Dès qu'un pays a trop de puissance, les autres États ont tendance à se liguer contre lui. \\
Morgenthaw a été séduit par cette idée d'équilibre: il y a intérêt, pour un État, de se rallier au moins puissant. 

\section{De la fin du Moyen-Âge aux deux guerres mondiales: l'équilibre multipolaire européen}

La fin du Moyen-Âge correspond à l'émergence des premiers États moderne et aux grandes découvertes et la colonisation. Au XIXe, le congrès de Vienne (1814-1815), réorganise l'Europe à la fin de l'empire Napoléonien. Après ceci, l'Europe domine, via la colonisation la majorité du monde connu. \\
Les deux guerre font passer d'un monde multipolaire à un monde bipolaire. 

\subsection{Les caractéristiques de l'équilibre européen}

On a pu constater une première application du principe d'équilibre des puissances. À partir du congrès de Vienne, il y avait un équilibre à cinq puissances: Grande Bretagne, France, Prusse, Autriche, Russie. À chaque fois qu'un État a commencé à s'imposer, à dominer l'Europe, les autres ont toujours fini par se liguer pour briser ses prétentions. \\
La Grande Bretagne a joué un rôle un peu à part. À partir du moment où la GB a renoncée à ses possessions sur le continent, la GB s'est recentrée sur l'outre mer et est devenu une sorte de gardienne de l'équilibre: elle n'a jamais subi de coalitions mais les a toujours supportés, voire menés. \\
Cet équilibre obligeait à nouer des alliances, à créer des règles communes. Cela n'a pas pour autant empêché les guerres, bien au contraire. \\
Les alliances entre États sont à géométrie variable, ce qui est normal dans un système multipolaire. \\
Cet équilibre a assuré des périodes de trêve mais jamais de paix. Par définition, dans un monde multipolaire, la paix ne peut se conserver que par la force, contre celui qui essaye de dominer les autres. \\
Pour être certain que l'équilibre existe toujours, il y a besoin qu'il y ait des confrontations régulières. 

\subsection{De l'âge d'or au déclin}

La puissance de l'Europe a commencée à décliner assez tôt, sans que les acteurs n'en prennent conscience. Dès la fin du XIXe siècle, on a vu émerger de nouvelles puissances en dehors de l'Europe: les États-Unis vers 1870 qui sont devenus industrialisés et qui étendent leur influence vers l'Amérique du sud. \\
Le Japon aussi, un des rares pays à avoir éviter la colonisation. À partir des années 1860, le Japon cherche à se moderniser, ainsi qu'à étendre son influence, en étant expansionniste du côté de l'Asie notamment. \\
À partir du moment où les européens ne dominent plus le monde, leur équilibre ne peut suffire pour équilibrer les RI. \\
Avec la première guerre mondiale, les européens se sont rendus compte que leur équilibre ne marchait plus. Pour mettre fin aux hostilités, les européens ont dû laisser des puissances étrangères arbitrer leurs propres intérêts (intervention militaire US). \\
De plus, à ce moment, il y a la révolution Bolchevique qui bouscule les RI. \\
À partir de la première guerre mondiale, les Européens ont cessés d'être sujet de leur histoire et son devenus des objets d'enjeux extérieurs. 


Dans l'entre deux guerres, il y a eu un système un peu trompeur. Les européens s'accrochaient à leur ambition. L'équilibre n'était plus multipolaire, la confrontation US/URSS s'annonçait déjà. Il y avait en quelque sorte trois pôles, un camp occidental, un camp soviétique, un camp fasciste. \\
Ces trois camps s'équilibraient plus ou moins, et c'est l'alliance de deux de ces camps qui permettent de détruire le dernier. 


\section{1945-1989: l'ordre bipolaire}

Marqué par la disparition du multipolaire. Opposition entre deux superpuissances. L'opposition est totale: idéologique, économique, militaire... \\
Pendant toute cette période, tout ce qui se passait dans les RI étaient interprétés dans la logique des deux blocs. 

\subsection{La politique des blocs et la confrontation Est/Ouest}

La guerre froide dure de 1947 à 1962, suivie de la détente. La fin de la seconde guerre mondiale annonçait la guerre froide. En 1947, sont arrivés au pouvoir les communistes en Pologne, en Roumanie, etc. \\
C'est pendant ces années que les deux superpuissances ont pris conscience de cette bipolarité et de ses conséquences. Ils ont, pendant cette première période défini des "règles du jeu". 


\subsubsection{Les règles du jeu de la bipolarité}

La première est la création de blocs. Pendant la multi-polarité, les alliances étaient à géométrie variable. Ce n'est plus le cas dans un système bipolaire. Chacune des superpuissance a intérêt à ce que ses alliés forment un bloc, qui sont loyaux, ou du moins la superpuissance doit veiller à ce que ce soit le cas. \\
Les tiers doivent choisir leur camp, surtout si ils sont dans une "zone vitale". Dans la guerre froide, les deux superpuissances se donnaient le droit d'intervenir pour maintenir leur cohésion, tout en fournissant l'aide nécessaire pour éviter l'attirance de l'allié pour son ennemi. \\
Doctrine Truman, c'est la doctrine du containment qui consiste à endiguer la progression du communisme dans les limites issus de la seconde guerre mondiale. Le plan Marshall est dans le même état d'esprit. \\
Dans le camp soviétique, on a la même doctrine: la doctrine Jdanov. Pour lui, chaque pays devait forcément choisir son camp et tout ceux qui exprimeraient des divergences avec l'URSS seraient contre elle. Doctrine Brejnev: l'URSS se donne le droit d'intervenir dans tout pays membre de son bloc qui souhaiterait le quitter.


Autre règle: chaque superpuissance ne devait pas intervenir dans le camp adverse. Dans un monde bipolaire, pour maintenir malgré tout un équilibre, il faut qu'il y ait une entente tacite entre les deux superpuissances pour que chacune laisse l'autre libre d'agir à sa guise dans son bloc. Ils ont d'ailleurs chacun un intérêt en commun, maintenir sa suprématie sur toutes les autres. Tant que la victoire n'est pas possible, il vaut mieux garder la bipolarité. Concrètement, il ne faut pas prendre de risques inutiles. \\
En 1947, ce principe n'était pas clair, jusqu'en 1956, des troupes Soviétiques avaient étés envoyées en Hongrie pour éviter la transition démocratique et malgré l'appel à l'aide des Hongrois, les US n'y sont jamais allés. 


Troisième et dernière guerre: la non participation des superpuissances dans les querelles de leurs alliés. Les superpuissances n'ont pas à aller risquer l'affrontement direct si ce n'est pas vital pour eux. \\
En 1956, avec la deuxième guerre Israelo-Arabe, la France et la GB s'engage avec Israël contre l'Égypte, allié à l'URSS. Les deux superpuissances n'ont pas intervenus l'un contre l'autre. 

\subsubsection{Une stabilité très relative}

Pour les auteurs réalistes, la bipolarité offre le meilleur équilibre. Il y a de nombreux avantages: nombre de joueurs réduits dont on réduit les incertitudes. L'intelligibilité du monde est augmenté. Tout cela est très satisfaisant pour des auteurs réalistes. \\
Il y a donc une certaine stabilité. "Paix impossible mais guerre improbable", Raymon Aron. Si les pays considérés comme vitaux n'ont pas connus de guerre, on a vu une multiplication des guerres de périphérie: la guerre de Corée par exemple. Tout ces conflits n'étaient pas uniquement dans logiques Est/Ouest, notamment les guerres d'indépendance, les guerres israelo-arabe. Cela dit, à chaque fois, les protagonistes ont reçu le soutien de l'une ou l'autre des superpuissances. \\
À la suite de la crise des missiles de Cuba, la guerre froide a cédé à une période légèrement différente: la détente. La confrontation est devenue plus indirecte. Leurs dirigeants se rencontraient et co-géraient à deux l'équilibre du monde. Cela n'a pas changé pour autant la nature de l'équilibre du monde. Aucune des deux superpuissances n'avaient abandonnés l'idée de dominer l'autre. La guerre du Vietnam ou d'Afghanistan sont les exemples symétriques. \\
Les superpuissances ont continués de s'ingérer dans les pays de son bloc car cela semblait encore moins dangereux qu'avant à cause du dialogue qui s'était mis en place entre les deux pays. \\
Le processus de détente s'est quand même essoufflé vers les années 70-80, car les superpuissances avaient l'impression que chaque bloc avançait, les rendant alors plus agressifs. Raegan avait donc remis en oeuvre l'endiguement en se faisant élire. La tension remonte donc.


La bipolarité a été fortement remise en cause. On a vu s'esquisser un nouvel axe. 

\subsection{Les incidences de la décolonisation et la formation d'un axe Nord/Sud}

Dans les années qui ont suivi la révolution Chinoise, la Chine s'est retrouvée alliée avec l'URSS. Cependant la Chine n'appréciait pas la tutelle de l'URSS, et a donc forgé une alliance avec les US dans les années 60. La logique des blocs s'est trouvé relativisé, un changement d'alliance a été possible. \\
Le changement le plus marquant a été l'irruption sur la scène internationale de pays nouvellement créés à la suite de la décolonisation. 1960: on a vu une douzaine d'États apparaître d'un seul coup. Les pays nouvellement créés sont devenus majoritaires à l'AG de l'ONU. Ces nouveaux États ont cherchés à échapper à la logique bipolaire. 

\subsubsection{Des pays en voie de développement au tiers monde, diagnostics et solutions}

Il fallait trouver un nom pour trouver ce que représentaient ces nouveaux pays. On a d'abord parlé de pays sous développé, puis pays en voie de développement. \\
Une expression s'est imposé dans les années 60-70: pays du Nord, pays du Sud. Cela ne correspond pas tout à fait à la réalité mais a le mérite d'être neutre. Cela permet de dépasser clairement la ligne bipolaire: il existe un autre clivage que l'est/ouest. \\
Ensuite, est apparu l'expression "tiers monde", qui fait référence au tiers État de la Révolution française, coincé entre deux ordres qui possèdent le monopole politique mais ne représente pas grand chose. \\
Les pays du tiers monde ont envisagé leur développement et développé leurs propres diagnostics. Le principal obstacle au développement de ces pays est le système économique international car rend dépendant les pays du tiers monde aux autres systèmes économiques. \\
L'analyse dominante est que les pays du tiers monde ont besoin d'exporter pour investir, mais les exportations de ces pays reposent sur les matières premières et sont obligés d'importer les produits industrialisés. Or, le cours des matières premières est soumis à fortes fluctuation, les pays du tiers monde ne contrôlent donc pas leurs prix. Il a sorti de tout ça qu'il était recommandé aux États du tiers monde de mener une économie Keynésienne, de développer le service public en développant l'industrialisation. \\
Les pays du tiers monde revendiquaient un NOEI. Cela reposait sur une nouvelle conception de la souveraineté étatique en mettant l'accent sur une idée de souveraineté économique. Ils cherchaient à pouvoir nationaliser des transnationales sur leur territoire, à contrôler le cours des matières premières, etc. Ils réclamaient une aide financière, matérielle, mais non conçu comme une forme de charité mais comme une créance, une dette de l'histoire. \\
Le relatif succès qu'a eu le tiers monde tient du fait qu'ils étaient d'abord majoritaire, puis les superpuissances cherchaient aussi à les séduire, à lâcher du lest. CNUCED, créé en 1964, devait être le cadre où serait renégocié le commerce sur l'axe Nord/Sud. \\
En 1974, l'AG de l'ONU a adopté un programme d'action pour NOEI. \\
La banque mondiale a participé à ce dynamisme tiers-mondiste. Sa mission était d'accorder des prêts à très long terme aux pays du tiers monde tout en fournissant des conseils d'expert. 

\subsubsection{Le mouvement des non alignés}

C'est le mouvement le plus sérieux pour remettre en cause la bipolarité. Il date de la conférence de Bandoeng en 1955, pour affirmer des principes communs. En 1961, le sommet de Belgrade à lieu en Yougoslavie, c'est là qu'a été créé le mouvement des non alignés sous l'impulsion de Nasser, Neru, et Tito. \\
La Chine a voulu s'imposer comme leader ou tuteur des non alignés, elle a participé à la création du mouvement mais a perdu en influence avec le grand bon en avant. \\
Ce mouvement cherchait à lutter contre la ségrégation, contre le colonialisme, soutenait l'idée du NOEI. Il s'agissait avant tout toute dépendance par rapport aux deux blocs. Ils ne prétendaient pas pour autant être neutre. \\
Ce mouvement s'est vite effrité, et ses membres se sont rapprochés de l'URSS ou des US. 

\subsubsection{La fin des années 70 et la remise en cause du tiers-mondisme}

On s'est rendu compte que le modèle de développement ne donnait pas tous les résultats escomptés. Les aides extérieurs s'étaient traduites par du gaspillage, du double emploi, pour enrichir des dictateurs. \\
Beaucoup d'intellectuels ont avancés que l'analyse de l'époque était entièrement fausses. Ils disaient que le colonialisme n'était pas la cause des problèmes de développement dans les pays du tiers monde mais la corruption, ainsi que d'autres causes endogènes. \\
Au cours des années 70, les pays du tiers monde se sont énormément endettés. Les banques privées ont eu énormément d'argent à placer grâce aux pétrodollars, mais la montée du pétrole a créé une crise économique, obligeant les US à augmenter le taux d'intérêt, triplant les dettes des pays du tiers monde. \\
Le FMI s'est impliqué dans l'aide aux pays du tiers monde en lançant des programmes d'ajustement structurel. \\
Dans beaucoup de pays, ces mesures ont eu des effets désastreux: déclin de la production, des infrastructures, etc. Quelques pays ont connus un essor remarquable en attirant sur leurs territoires des entreprises délocalisés: Taïwan, Hong-Kong, Corée du Sud. 

\section{Le monde depuis 1990: ordre ou anarchie ?}

\subsection{L'effondrement du bloc soviétique et ses conséquences}

Contrairement à ce qui était prévu, le bloc soviétique s'est effondré de l'intérieur de manière étonnement rapide et pacifique. \\
Gorbatchev arrive au pouvoir en 1985 et lance des réformes économiques et politiques. En 1988, les autorités soviétiques abandonnent la théorie de la souveraineté limitée et annonce qu'elle n'interviendra plus pour soutenir des régimes contestés. Les transitions se sont enchaînés, avec la Pologne qui a commencé en 1989, puis chute du mur, transitions dans les autres pays pour finir par la dissolution du pacte de Varsovie en 1991. \\
En Août 1991, une tentative de putsch fait que le PC a été dissout, Gorbatchev a démissionné et fait disparaître l'URSS en tant qu'État. \\
Le système soviétique étant en crise depuis longtemps, cet effondrement était inévitable. L'URSS n'arrivait pas à assurer la cohésion de ses différentes nationalités et connaissait des crises économiques de plus en plus difficile. 

\subsubsection{L'annonce d'un "nouvel ordre mondial"}

Il y a une courte période qui commence lorsque Gorbatchev arrive au pouvoir et qui finit vers 1993. On a pensé pendant cette période que le monde devenait plus sûr, plus cohérent, plus harmonieux. À partir du moment où l'URSS a relâché la pression, les US étaient incités à faire de même. On a donc pensé que les conflits périphériques allaient s'éteindre. \\
Lorsque les pays de l'est ont annoncés leurs transitions, il y avait l'idée que la démocratie et l'économie de marché allait s'imposer partout. \\
On se disait que l'ONU allait pouvoir enfin jouer un vrai rôle sans être paralysé par la Russie ou les US. \\
La vague de démocratisation a en effet touché certains pays, notamment en Afrique et en Asie. À l'échelon international, il y a eu une conférence Israelo-Arabe réuni à Madrid, ce qui a abouti à la reconnaissance d'une autorité Palestinienne. \\
Gorbatchev a été le premier à lancer la formule du nouvel ordre mondial, mais elle a été connu lorsque Bush père l'a utilisé. \\
Fukuyama parlait de "La fin de l'histoire et le dernier homme". Il constate que la démocratie libérale et l'économie de marché sont devenus les seules solutions possibles dans les sociétés modernes. Ces solutions sont déjà en train de triompher d'où cette idée de fin de l'histoire, au sens où on est à la fin d'un processus. Il défend aussi l'idée d'une homogénéisation des cultures.

\subsubsection{Le désenchantement des années 1993-1994}

Le clivage idéologique dominant avait certes disparu, mais cela allait faire place à d'autres choses. Huntington s'est opposé à Fukuyama. En 1993 et en 1996, il parlait du choc des civilisations. Il voit le monde comme conflictuel. \\
Il dit que la mondialisation et la diffusion des produits de la mondialisation n'a rien à voir avec la diffusion des valeurs. \\
Son idée c'est que avant, les clivages étaient idéologiques, économique, politique, mais que désormais, ils sont culturels. Il identifie 7 grandes civilisations humaines. Il les définit essentiellement avec des critères religieux qu'il recoupe avec la géographie. \\
Il explique que les conflits les plus durs ne seront pas ceux entre toutes mais entre certaines. \\
Il n'est donc pas sûr que la configuration post-bipolaire soit plus stable, mais pas sûr non plus qu'elle soit aussi conflictuelle. \\
Ce qui est sûr, c'est que le monde est plus complexe, moins intelligible. Entre autres parce qu'on ne sait pas où trouver l'équilibre qu'on avait pendant la guerre froide. 

\subsection{L'hégémonie américaine et ses limites}

Militaire: US loin devant tous les autres. Économie: US devant, en tête à tête avec la Chine, cela dépend de comment on calcule le PIB (nominal ou PPA). De plus, dollar a une hégémonie. Soft power: US loin devant. \\
Les US sont ils menacés dans leur hégémonie ? 

\subsubsection{Un ordre unipolaire ou multipolaire ?}

Les États-Unis sont le seul État qui cumule encore tous les attributs de la puissance. Si il a des concurrents sérieux, aucun ne l'égal dans tous les domaines. \\
Le Japon est très bon commercialement, mais sa constitution lui interdit la guerre, et sa démographie n'est pas dynamique, sa population étant vieillissante. L'Afrique, pris dans sa globalité est très dynamique. \\
Les BRICS représentent aussi un grand dynamisme, coopérant pour leur développement, créant d'ailleurs une banque d'investissement, conçue comme une alternative, une rivale au FMI et à la banque mondiale. Les cinq pays des BRICS, c'est la moitié de la population mondiale et un cinquième du PIB mondial, qui augmenterai à 40\% du PIB mondial dans quelques années. Cependant, les économistes se fondent toujours sur la loi de la croissance infini, et aujourd'hui, les BRICS traversent des situations difficiles, sauf l'Inde. \\
D'autres parlent des MINT, Mexique, Indonésie, Nigeria, Turquie qui ont les plus fortes croissances.


Ces pays émergents peuvent-ils constituer un pôle face aux États-Unis ? N'étant pas uni, il est difficile de dire qu'ils pourraient être un pôle face aux US. Les BRICS sont des puissances, certes, qui toutes ne sont pas émergente d'ailleurs, elles ne sont pas unis, elles ont des intérêts divergeant. Elles sont aussi très différentes, on y voit des démocraties, des régimes autoritaires, à la limite de la dictature. \\
Ces pays émergeant souhaitent modifier l'équilibre international actuel, car cet équilibre a été créé par les US à la fin de la seconde guerre mondiale, ils ne souhaitent pas remettre en question le libéralisme et la mondialisation. C'est pourquoi ces pays se mettent souvent d'accord concernant leur vote à l'ONU. 


La Chine est vu comme le challenger des US. Son atout le plus évident serai sa démographie, étant le pays le plus peuplé du monde (l'Inde rattrape cependant la Chine). Cette démographie permet un grand nombre de mains d'oeuvre, un grand nombre de soldats, etc. Cependant, cela implique aussi de les nourrir. \\
Le modèle Chinois est aussi en contestation, qui pourrait déboucher sur une explosion sociale. \\
La population Chinoise n'est pas homogène, on peut distinguer des mouvements séparatistes ou encore des régionalismes. Au Xinjiang, avec les Ouïghours, on note une des plus grandes mouvances indépendantistes, qui se radicalisent de plus en plus. \\
La Chine a besoin de l'économie américaine et européenne soit forte pour maintenir leur économie forte, c'est un lien d'interdépendance. Depuis 2015, la croissance Chinoise est tombée en dessous de 7\%, d'après les chiffres du Gouvernement et  non d'organisme indépendant. \\
La Chine est aussi une puissance militaire. Leur budget de la défense augmente largement en dépit de la baisse de la croissance. \\
Xi Jinping, au pouvoir depuis 2013, sera bientôt renouvelé. \\
Depuis quelques années, les instituts confucius cherchent à promouvoir la culture chinoise. \\
La Chine est un acteur incontournable dans sa zone géographique. \\
En Afrique, la Chine investit beaucoup dans les infrastructures. Ils recherchent essentiellement les matières premières qu'ils n'ont pas eux mêmes. \\
La Chine est aussi présente au Moyen-Orient, car elle a besoin de pétrole. \\
Il y a aussi un projet de nouvelle route de la soie, l'idée étant de faire deux routes vers l'ouest, qui impliquent de créer tout un tas de routes, de ports, etc, sur toute l'Asie centrale pour rejoindre l'Europe ; la deuxième route serait maritime. \\
Cela inquiète beaucoup les US qui seraient marginalisé. L'Inde aussi s'inquiète et essaye donc de créer sa nouvelle route des épices. \\
Les relations Chine-Russie se sont développés après la fin de la guerre froide, qui se sont confirmés avec la crise de 2008. Ils ont compris qu'ils avaient intérêt à échanger entre eux quand le marché occidental était moins porteur. Il y a un intérêt commun à faire contrepoids aux occidentaux. Certains parlent d'un axe Pékin-Moscou. \\
Deux analyses. La première est que ce qui caractérise la Chine est qu'elle mène une diplomatie prudente ; au fond, la Chine ne cherche pas à dominer le monde, donc la Chine chercher à défendre ses intérêts, sa souveraineté, mais elle n'est pas agressive pour autant ; la concurrence pour l'accès aux matières premières créent des collisions qui ne sont pas le but initial de la Chine qui a pour mot d'ordre une émergence pacifique. La deuxième analyse est que la Chine a dépassé ce stade, qu'elle cherche à rentrer dans le rôle de deuxième superpuissance, qu'elle commencerai à se sentir plus forte, plus sûre d'elle, et donc peut avoir une certaine volonté expansionniste et agressive. \\
La Chine a cherché à se pacifier avec ses ennemis traditionnels, ses voisins. Un enjeu pour la Chine est la mer de Chine méridionale qu'elle revendique, car il y a des ressources (pêche, pétrole, etc) et c'est une voie de passage, qu'elle essaye de contrôler en construisant des îles artificielles. \\
Jusqu'à présent, la Chine a cherché à ne jamais s'opposer aux US. Cependant, depuis quelques années, des contentieux font surface, notamment à cause des US qui sont alliés au Japon et doivent les défendre en cas de conflit armé. Beaucoup de réalistes disent qu'une guerre froide va éclaté entre la Chine et les US. L'objectif des US devraient donc être de conjurer cette menace avant qu'ils soient trop tard. 


Concernant la Russie, elle n'a rien à voir avec la Chine. Ce n'est pas une puissance émergente, c'est une superpuissance déchue, le rapport au monde est donc complètement différent. Tout le monde semble s'accorder pour dire que la Russie n'a pas les moyens de redevenir une superpuissance. \\
Elle ne peut pas devenir leader d'un bloc. Cela dit, elle a des ressources non négligeable qui lui permettent de défier de temps en temps les US. \\
La première puissance de la Russie est sa puissance militaire et nucléaire. Ils cherchent à moderniser leur armée. C'est un des premiers exportateurs d'armes au monde. \\
Sur le plan économique, la Russie avait eu une forte reprise au début des années 2000 mais se dégrade maintenant. Cette dégradation est due aux sanctions économiques à cause de l'Ukraine, mais aussi le coût du pétrole qui a chuté. Cela dit, l'économie Russe a de graves faiblesses car reposent essentiellement sur la rente de la vente de pétrole et de gaz, ce qui n'est pas normal pour une puissance dite développé. Le domaine de l'armement et l'énergie sont les seuls domaines qui tournent en Russie. De plus la population est vieillissante. \\
On note aussi des mouvements islamistes armées qui remettent en cause l'intégrité du territoire par leurs ambitions indépendantistes, mais menacent aussi la sécurité. Tchétchénie, Ingouchie, Daghestan sont des républiques fédérés du Cocase, qui sont fortement indépendantistes. \\
Aujourd'hui, la Russie cherche à reconquérir sa place, ou du moins, une place puissante. Il y a une idée de lever l'humiliation de la chute de l'URSS. \\
La priorité numéro une de la Russie, c'est de maintenir une forme de droit de regard sur ce qu'elle appelle son étranger proche, c'est à dire les ex Républiques soviétiques. Aujourd'hui, la Russie cherche à rebâtir une véritable sphère d'influence autour d'elle. La Russie maintient beaucoup de relations avec la Biélorussie, l'Arménie, le Kazakhstan et l'Asie centrale en général. À terme, le gouvernement Russe cherche à former une Union Eurasiatique. Or, des pays proches de la Russie se sont rapprochés de l'occident comme l'Ukraine ou la Géorgie. 


Unipolaire ? Multipolaire ? \\
On est dans l'entre deux, dans une période de transition. Pour les libéraux, le monde est unipolaire militairement et multipolaire sur le plan économique. \\
Les réalistes sont divisées. À un moment, ils ont émis l'idée selon laquelle les US allaient rester globalement la puissance hégémonique. Ils soutiennent même que c'est ce qui pourrait le mieux assurer la stabilité car ils sont éloignés, pouvant se tenir à l'écart de la compétition. \\
La grande majorité des réalistes pensent que le monde unipolaire est intenable car viole l'équilibre des puissances. La plupart pense que les US ne peuvent pas être des garants de l'extérieur car ils ne sont pas à l'extérieur, ils sont des acteurs à part entière, comme les autres. \\
Beaucoup de réalistes ne pensent pas qu'on va vers un monde multipolaire mais parle plutôt d'une nouvelle bipolarité Chine/US. La récente crise a beaucoup joué à affaiblir les US ainsi que les erreurs stratégiques que les US a commises, comme l'Irak en 2003 ou encore la Syrie aujourd'hui. Ils sont donc moins aimé mais surtout ils font moins peur.

\subsubsection{L'évolution de la politique étrangère américaine}

Tout au long de leur histoire, les US ont connu des phases alternatives d'isolationnisme et d'interventionnisme. Depuis la fin de la guerre froide, les dimensions ont changés. N'ayant plus de contrepoids depuis la chute de l'URSS, les US sont incités à intervenir partout. \\
Depuis la fin de la bipolarité, les présidents des US ont tous annoncés pendant leur campagne qu'ils allaient arrêter d'intervenir. Ils l'ont tous dit et ont tous essayés de le faire au début de leur mandat. En pratique, tous les gouvernements américains ont recommencés les interventions à l'étranger. \\
Par exemple, la tentation et la volonté isolationniste était très forte au début des années 90. Cela s'est manifesté très clairement à l'arrivé au pouvoir de Clinton. Le discours dominant consistait à dire "qu'il était temps de toucher les dividendes de la paix". Clinton a fait une priorité de la politique intérieur de son pays. Il a fermé des centaines de bases militaires à l'étranger, a démobilisé des soldats, etc. Il s'est quand même investi dans les RI en privilégiant les cadres multilatéraux. \\
Les attitudes isolationnistes ne sont pas tenables pour maintenir la suprématie. À la fin du mandat de Clinton, en 1995, il a décidé d'envoyer une délégation en ex-Yougoslavie et a réglé l'affaire en quelques semaines.


La même chose a été constatée avec Bush, qui disait aussi se recentrer sur les vrais problèmes intérieurs. Pendant les premiers mois, il s'est intéressé de très loin à l'international, jusqu'au 11 septembre où il a fait son revirement. Les néo-conservateurs ont eu une large influence sur la politique étrangère de Bush, ils étaient très présent dans son administration. Wolfowitz était un représentant de cette tendance, en tant que secrétaire adjoint à la défense. \\
Aujourd'hui, ces néo-conservateurs sont moins influents mais représentent toujours une partie du parti républicain. Ils sont majoritairement contre Trump. \\
Au départ, les néo-conservateurs étaient plutôt proche du parti démocrate voir de l'extrême gauche. Ces néo-conservateurs ont commencés à se droitiser dans les années 60-70, contre les droit civils et la discrimination positive. Ces néo-conservateurs se sont rapprochés du pouvoir quand Reagan s'est fait élire. Wolfowitz était trotskyste quand il était étudiant, il a enseigné à la fac, il continué au Pentagone, puis s'est rapproché du pouvoir avec Reagan. \\
Les néo-conservateurs défendent trois grandes idées: la première est qu'il faut conforter la puissance américaine, ils ont une obsession du déclin, et pour ça, il faut revalorisé le budget militaire ainsi que prendre des distances par rapport aux instances multilatérales comme l'ONU ; la deuxième est que les US doivent profiter de leur puissance pour changer le monde, le remodeler à l'image des États-Unis, l'idée est d'assumer les responsabilités qu'impliqueraient une telle puissance ; la troisième idée est que les US doivent mener une politique étrangère active, ils ne doivent pas hésiter à imposer des changements de régime quitte à mener des actions préemptives (pour empêcher une offensive imminente). \\
Pour les néo-conservateur, les menaces, ce sont les "rogue states", les États "voyous". C'est un État qui ne respecte pas les règles internationales, qui ne respectent pas non plus les droits de l'Homme et qui est menaçant car soutient les terroristes ou cherchent à se procurer des armes de destruction massive, et de plus, l'État doit être hostile aux US. Cette notion impliquerai que les États soient unis autour des US, et qu'ils n'ont pas d'ennemis, seulement des États marginaux, au banc de la société internationale. On compte parmi ces États la Libye jusqu'en 2003, l'Irak jusqu'au renversement de Sadam. La liste a toujours englobé l'Iran et la Corée du Nord. Il n'y a jamais eu de puissances importantes dans cette liste comme la Russie ou l'Inde. 


La présidence d'Obama a été elle aussi marqué par une certaine hésitation entre isolationnisme et interventionnisme. IL y a eu un effort pour sortir de ce dilemme en intervenant autrement dans les affaires du Monde. Il a toujours été réticent aux interventions extérieures. Au début de son deuxième mandat, il a insisté sur le fait que les US devaient se sortir d'une décennie de guerres. \\
De fait, il a plus intervenu à l'extérieur, car les défis du passé n'ont pas disparus, coincés dans les choix de ses prédécesseurs. On peut citer par exemple le fait qu'il a envoyé des troupes en Irak suite à l'apparition de l'EI. \\
Obama a cherché à intervenir autrement, en mettant l'accent sur le multilatéralisme et en mettant l'accent sur le soft power. Sur le multilatéralisme, cela correspond à ce que prône l'approche libérale, c'est une tendance assez constante au parti démocrate. Sous Obama, ça a pris une véritable dimension: il cherchait à partager le fardeau, la gestion du monde, en s'appuyant sur ses alliés. Ceci afin de permettre d'alléger le fardeau. \\
On peut noter une amélioration des relations entre l'ONU et les US. Obama a cherché à revitaliser les relations transatlantiques. Obama s'est appuyé sur l'OTAN à chaque fois qu'il ne voulait pas que les US ne soient pas en première ligne. L'exemple le plus clair est l'intervention en Libye en 2011. \\
Sous Obama, l'Europe n'était déjà plus une priorité pour les US depuis qu'elle n'est plus menacée par l'URSS. Obama reprochait aux européens de ne pas supporter assez le coût de l'OTAN. \\
La priorité d'Obama, c'était les nouveaux partenariats, avec les pays d'Asie pacifique essentiellement. Ils ont une stratégie de pivot: de l'atlantique vers le pacifique. Il y a une idée de contenir la Chine. On note un renforcement des alliances en Japon et Corée du Sud, mais aussi de nouvelles relations avec le Vietnam ou l'Inde. \\
Une partie de l'Asie pacifique a signée le TTP, un traité de libre échange, qui a permis d'isoler la Chine, pour la coincer entre deux fronts. D'où la négociation d'un accord avec les européens au même moment: le TAFTA. Le TTP n'a pas pu être ratifié, et Trump a décidé de s'en retiré, pareil pour le TAFTA sur lequel Trump est défavorable. \\
Obama n'a pas totalement réussi sa stratégie de pivot. Il a été obligé de se réinvestir en Europe et au Moyen-Orient à cause de la crise Ukrainienne et l'EI. 


Obama a beaucoup utilisé le soft power. Pour conforter la sécurité des US, il a cherché à améliorer son image. Il pense que la plupart des problèmes ne peuvent pas se régler par la guerre, notamment les problèmes environnementaux, etc. \\
Son premier décret après son élection était l'interdiction de toutes formes de torture, et la fermeture du centre de Guantánamo. Ce dernier n'a pas pu être fermé. Dès sa campagne électorale, il a pris l'engagement de mener des dialogues avec des pays considérés comme hostiles aux US. On note d'abord l'Iran, avec son accord du 14 Juillet 2015, mais aussi Cuba. \\
Obama souhaitait aussi améliorer ses relations avec la Russie. Il a aussi cherché à rapproché les US du monde musulman, cela n'a pas duré car il y a eu une ambiguïté dans la réaction des US face au printemps Arabe. Le conflit Israëlo-palestinien a aussi été un frein à cet objectif. \\
À la fin de son mandat, Obama parlait de smart power, puissance astucieuse. L'idée est la juste combinaison entre le soft et le hard. 


Sur Trump, on ne peut pas avoir de certitudes. Il a surpris tout le monde en appliquant ses promesses de campagne même les plus polémistes. Trump est très imprévisible, il semble naviguer à vue. Ce qui est intéressant avec Trump, c'est de mesurer les contraintes structurelles: les équilibres interne (les check and balances) et l'équilibre international. \\
La première question qui se pose est toujours isolationnisme ou interventionnisme. Trump a beaucoup répété qu'il fallait limiter au maximum les interventions extérieures, les interventions morales, etc. Trump va-t-il se faire rattraper par la réalité ou non ? \\
Pour ce qui est du soft power, il n'a pas l'air d'être à l'ODJ en ce moment. Il se dégrade en Europe, par les déclarations pro-brexit de Trump, les critiques qu'il a fait sur la politique allemande. Son image s'est dégradé avec le "muslim ban" dans les pays musulmans. \\
Le multilatéralisme n'est plus non plus à l'ordre du jour, il a quitté le TPP, il veut renégocier l'ALENA, il pense à quitter l'OMC. \\
Trump a une certaine incohérence dans ses propos. Trump avait annoncé une volonté de rapprochement avec la Russie et la Chine, cela s'est confirmé après son élection. Il a nommé parmi ses ministres des proches du Gouvernement Russe. Cependant, après certains scandales, les nouveaux nommés ont fait preuve de plus de prudence. \\
Trump est très dur vis à vis de la Chine. Il a eu un coup de téléphone de la présidente de Taïwan. Trump a dit que tout était négociable, y compris le principe de la Chine unique. 

\subsubsection{"L'impuissance de la puissance"}

Cela résume assez bien le paradoxe de la position américaine. De fait, les spécialistes des relations internationales se divisent sur la question de la polarité. La majorité des auteurs disent cependant que la puissance américaine n'est pas viable à long terme, un des arguments étant: le monde n'a jamais connu d'empire universel. \\
Benjamin Constant écrivait que la nation qui se voulait universel ne pouvait pas durer car serait contre toutes les autres. Badie reprend l'idée et montre que s'opposer au système international, c'est s'opposer aux États-Unis. Il parle d'une nouvelle puissance: la capacité de nuire. Selon Badie, cela est la stratégie du pauvre. Plus on descend dans la puissance plus la stratégie du désordre voire du chaos devient alléchante, car combattre frontalement le géant n'a aucune chance de victoire. Depuis une décennie, l'anti-américanisme s'est beaucoup développé. \\
Un autre auteur parle du risque de la sur-expansion: à partir d'un certain point, trop de conquêtes mènent à une impasse, cela devient trop coûteux pour cette puissance de maintenir son contrôle partout. Les grandes puissances hégémoniques ont du mal à se remettre en question, elles finissent par croire qu'elles peuvent tout faire, ne se rendent pas compte de leurs limites. \\
Certains pensent que les US doivent s'auto-limiter avant qu'ils soient obligés à le faire par les autres. Ces auteurs soutiennent que le repli total est à éviter, cela risquerai d'être interprété comme un signe de faiblesse, tant par les ennemis que par les alliés. \\
La première limite est US est donc qu'elle peut être autodestructrice. 


Paradoxalement, le principal problème des US aujourd'hui, est qu'ils n'ont plus d'ennemis. Ils ont des concurrents de plus en plus sérieux, des opposants, mais n'ont pas d'ennemis étatiques à leur mesure pour jouer le rôle que jouait l'URSS. \\
Les candidats ne manquent pas selon Badie, mais n'ont pas l'envergure de l'ex URSS. Par exemple, le terrorisme n'est pas un ennemi comme l'URSS. \\
Un ennemi est nécessaire pour mobiliser la population, pour justifier la politique menée, pour fidéliser les alliés. C'est l'idée que la puissance nourrit la puissance. C'est l'idée qu'une puissance a besoin d'un double. Certains stratèges aiment à penser que la prochaine guerre froide sera contre la Chine. \\
D'autres pensent que l'hégémonie américaine est d'ores et déjà terminée, car n'est pas adaptée aux défis du monde d'aujourd'hui. Sur le plan strictement militaire, les US ont trouvé une limite. Cette stratégie s'appuie sur des technologies sophistiqués, mais cela ne suffit pas à vaincre des troupes qui utilisent la guérilla ou le terrorisme. \\
Les arguments qui convergent tendent à dire que l'hégémonie américaine n'est pas tenable. Ils n'ont plus les moyens d'assurer la stabilité du monde. Peut-être que le monde n'est plus gouvernable, ou du moins, que ce point de vue ne semble plus suffire.


Depuis la fin de la bipolarité, il y a une idée selon laquelle l'État vivrait une forme de crise. C'est peut être ça la véritable caractéristique depuis les années 90. 

\subsection{La diversification des acteurs internationaux et la remise en cause du rôle de l'État}

La fin de la bipolarité s'est traduit par un affaiblissement des États, quels qu'ils soient. Ces principes semblent de plus en plus inadaptés au monde actuel. Le rôle de l'État dans les RI se heurtent de plus en plus à la concurrence d'acteurs non étatiques. On constate que l'État est attaqué à la fois de l'intérieur et de l'extérieur. Il est remis en cause par la mondialisation et le repli identitaire.

\subsubsection{De la dilution du monde étatique à la "fin des territoires"}

Cela est paradoxal, mais la principale dilution du modèle étatique, c'est probablement son succès à travers le monde. Cela s'explique par la grande vague de décolonisation qui fait apparaître énormément d'États. \\
Ce phénomène s'est poursuivi par la suite avec l'éclatement de l'URSS, de la Yougoslavie, etc. Depuis la fin des années 90, d'autres États ont annoncés leurs indépendances. Le dernier apparu est le Soudan du Sud en 2011, qui était une région de la République du Soudan qui a fait sécession après une longue guerre civile. On compte aussi le Kossovo en 2008, le Monténégro en 2006, le Timor Oriental en 1999. Ces indépendances débouchent sur des situations assez problématiques, certains refusant de les reconnaître. \\
Quand l'ONU a été créé, il y avait 50 États membres. Il y a aujourd'hui 193 États membres. Cela pourrait encore croître, on peut penser aux Palestiniens, aux Kurdes, aux Tchétchènes, etc. \\
À première vue, cela peut prouver les avantages qu'auraient l'État. Cela dit, cette prolifération a eu pour effet de fragmenter un peu plus la planète et a eu des conséquences sur le modèle étatique lui même. Toutes les cultures ne sont pas forcément adaptés à la forme étatique. La greffe n'a pas toujours bien pris, notamment dans les pays du tiers monde. Les nouveaux pays issus de la décolonisation ont du mettre en place des structures étatiques du jour au lendemain alors que le processus a pris des siècles en Europe. \\
Dans les États issus de la décolonisation, il n'y avait pas forcément de coïncidence entre l'État et la Nation. Si l'idée nationaliste a permis de soulever des peuples contre le colonialisme, il a ensuite posé des problèmes dans des régions où les frontières étaient artificielles. \\
En Libye, le pays repose essentiellement sur des tribus, qui ont survécus à la colonisation Italienne mais aussi à la formation récente d'un État. Quand Kadhafi prend le pouvoir, il ne casse pas ce système tribal, au contraire, il cherche à s'appuyer dessus. En 2011, plusieurs chefs de tribu ont pris position contre Kadhafi, et aujourd'hui encore, les tribus sont le principal moteur de la politique Libyenne. Après Kadhafi, le pouvoir en place n'arrive pas à s'imposer face aux milices qui se sont coalisés soit autour de mouvements islamistes soit autour d'anciens militaires anciennement proche du colonel. En 2014, le pouvoir s'est effondré, le Gouvernement s'est effondré, les milices ont mis en place un nouveau parlement, face à l'ancien qui s'est réfugié à Toubrouk. À partir de 2014, plusieurs personnalités, notamment en France, ont plaidés pour retourner en Libye. En Janvier 2016, un gouvernement d'union nationale a été mis en place, avec Fayez el Sarraj comme Premier Ministre. Cependant, le général Haftar, ancien proche de Khadafi, contrôle une partie du pays, notamment les installations pétrolières. 


Actuellement, le monde des États est très hétérogènes, avec des États-nations européens. D'autres fonctionnent qu'en apparence comme un État. Les logiques de la décolonisation ont participé à créer des micro-États, qui n'ont que l'apparence de l'État. Le plus petit est Nauru. \\
Certains États se sont effondrés institutionnellement: plus de justice, plus d'armées, plus de police avec des territoires sous contrôle d'organisations criminelles. Dans ces cas là, on parle de "failed states". On peut appliquer cela à l'Afghanistan, à la Somalie par exemple. 


Il y a des entités qui fonctionnent de facto comme des États, ils ont un territoire, l'indépendance, des populations et des vrais gouvernements. Cependant, leurs gouvernements ne sont reconnus par personne ou presque. Certains parlent de quasi-État, voire d'États fantômes. En étant non reconnu, ils ont beaucoup de problèmes, ils ne peuvent pas conclure de traités, ne peuvent pas adhérer à des OI, etc. \\
Par exemple, on a l'Ossétie du Sud Abkhazie ou encore le Kanabakh, ou bien la moitié Nord de l'île de Chypre. Il y a aussi le Somaliland. \\
Qu'ils soient grand ou petit, les États se définissent surtout par une assise territoriale. Cela est surtout essentiel d'un point de vue juridique: c'est à la fois un titre et une limite à l'exercice des compétences étatiques: chaque État exerce une autorité sur les biens et personnes présentes sur son territoire. En même temps, le contrôle de ces territoires échappe de plus en plus aux États. Aujourd'hui, cette logique territoriale, se heurte à tout un tas de réseaux sans territoires. Ce sont des organisations non étatiques. 

\subsubsection{L'essor des organisations non étatiques}

Quels sont les conséquences de l'essor des organisations non étatiques depuis la fin de la guerre froide ? Pour les transnationalistes, les conséquences sont forcément très importantes. Ils disent qu'il y aurai d'un côté la scène étatique, où le nombre augmente, certes, mais n'est pas infini. À côté de ça, il y aurai scène transnationale, se composant d'un nombre infini d'acteurs, n'ayant aucun pôle, en reconstitution perpétuelle. Cette scène échapperai au contrôle étatique. Chaque scène pourrai faire contrepoids à l'autre, mais chacune obéit à sa propre logique, il y a donc possibilité d'interférence. \\
Pour les réalistes, il n'y a qu'une scène, celle des États. 


FMN: Firme Multi-Nationale, pas de définition officielle, ni de personnalité juridique officiel. Ces firmes sont soumises au droit des différents pays où elles opèrent. C'est en gros une entreprise qui exerce des activités dans plusieurs pays, soit parce qu'elle y a créé des filières à l'étranger, ou parce qu'elle a acheté des entreprises à l'étranger ou par des investissements direct. \\
Au XIXe siècle, ces FMN étaient surtout européenne, surtout dans le domaine de l'agricole. Elle cherchait surtout à se rapprocher des gisements de matière première. Cependant, ces firmes se sont multipliés dans la deuxième moitié du XXe siècle. Elles couvrent toute la planète, touchent tous les secteurs, y compris le tertiaire. \\
Ces firmes s'implantent à l'étranger pour tout, pas seulement pour se rapprocher des matières premières. L'idée est de faire du bénéfice. On compte plus de 100 000 FMN aujourd'hui dans le monde. Les 500 plus grosses FMN ont leur siège social dans à peine 30 pays, ce qui nous montre qu'elles sont plutôt concentrées. 


Relation entre les États et ces FMN ? \\
Les relations sont très compliquées, les FMN ont des puissances financières bien supérieures à beaucoup d'États. Total par exemple, a un chiffre d'affaire annuel de 145 milliards de dollars, soit 10 fois plus que le PIB du Gabon. Du coup, ces firmes peuvent en arriver à constituer un État dans l'État. Ces firmes n'ont pas pour but de privilégier le développement, ou fournir des emplois, elles profitent donc des disparités juridiques en s'installant là où le droit leur est le plus favorable. Il peut arriver que ces firmes ingèrent directement dans les affaires de certains États comme en faisant élire ou partir des gouvernements: par exemple, un FMN Américaine dans l'agro-alimentaire a soutenu des coups d'État. \\
Face à ces risques, les États d'accueil ont un certain nombre d'armes, pour forcer les FMN à respecter les règles, au risque de les voir disparaître. Ils peuvent aussi nationaliser ou expropriés. L'ONU reconnaît le droit aux États de nationaliser ou exproprier les entreprises présentes sur leur territoire à la condition de donner une indemnité correcte. L'UE a des moyens d'interdire les abus de position dominantes. \\
La plupart des pays du monde cherchent à attirer des FMN dans leurs pays. Ils recherchent les capitaux, les investissements, les emplois, les transferts de technologie. \\
Par rapport à leurs pays d'origine, les FMN peuvent être des instruments au service de leur pays d'origine. Cela peut se faire de façon diffuse, même si cela n'est délibéré. On peut se demander si les États les plus puissant utilisent leur FMN pour appuyer leurs relations internationales ou est-ce l'inverse pour accroître les profits ? \\
En général, les FMN font concurrence aux États, mais ne sont pas forcément des outils de leur État d'origine. De plus, les FMN ont besoin des États, pour soutenir la R\&D. En tout cas, les États sont les seuls à pouvoir assurer à ces entreprises la sécurité matérielle dont les FMN ont besoin pour pouvoir se développer. 


Concernant les OING (abrégés ONG), elles font l'objet d'une perception beaucoup plus positives, présentant une expression de la solidarité internationale, de la société civile, etc. \\
Ces ONG bénéficient d'une reconnaissance officielle et internationale. La charte de l'ONU prévoit une consultation obligatoire des ONG travaillant dans les domaines de sa compétence. \\
Dans le Conseil de l'Europe, il y a eu une mise en place d'une conférence des OING qui sont consultés et qui peuvent directement travailler sur les projets du Conseil. On définit ONG comme "Des associations à but non lucratives et d'utilité internationale, qui exerce une activité effective dans au moins deux États différents". \\
Ces ONG sont très diverses: sportives, culturelles, religieuse, représentant des métiers, vocation syndicale, etc. Les plus intéressantes pour nous sont les ONG de défense des droits de l'Homme, ou encore de l'environnement, comme par exemple la croix rouge, Amnesty International, Green Peace, etc. \\
Les ONG sont de plus en plus importantes dans les RI. Elles ont fini par s'imposer comme interlocutrices face aux États. Elles ont des atouts, leur but non lucratif, leur capacité à mobiliser l'opinion. Elles ont mêmes participé à l'écriture de certains textes: CPI, Convention sur les mines anti-personnelles, etc.  \\
Certaines ONG ont la capacité de défier même la souveraineté de certains États, typiquement Green Peace. Certaines se revendiquent même un droit d'ingérence. C'est une initiative venu de médecins français, en 1968, qui travaillaient à la Croix Rouge, où ils devaient travailler en Afrique, fortement toucher par une famine et une guerre civile. Les médecins, témoins de massacre et de détournement de biens humanitaires, ne pouvaient pas faire grand chose, la croix rouge imposant une stricte neutralité, discrétion ou réserve. Ils ont décidés de prendre parti, et donc de quitter la croix rouge pour fonder médecins sans frontière (MSF), qui accorde une priorité au sauvetage des victimes, quitte à agir dans la clandestinité et à passer des frontières en fraude. \\
Deux résolutions à l'ONU ont étés votées sous l'impulsion de MSF, qui demandait de ne pas entraver le passage de l'humanitaire. \\
Les relations entre les ONG et les États restent très ambigus. Même les plus grosses ONG n'ont pas les mêmes ressources qu'une FMN. Les ONG, aujourd'hui, vivent du lobying, de la médiatisation, de l'expertise, des campagnes de communication, de la maîtrise du marketing. Plus les ONG sont nombreuses, plus elles sont dépendantes de l'argent public, les dons privés ne suffisant pas.


On constate donc que les deux scènes dont parlent les transnationalistes sont en interdépendance, elles sont toujours en train de communiquer, d'échanger. 

\subsubsection{Les défis de la mondialisation}

C'est une notion apparu dans les années 60, s'est répandu dans les années 80. Elle peut prendre plusieurs significations. \\
Au sens étroit, cela concerne essentiellement l'économie: "l'interdépendance économique croissante de l'ensemble des pays du monde, provoquée par l'augmentation des transactions transfrontalières de biens et de services, par l'augmentation des flux internationaux de capitaux, et par la diffusion accélérée de la technologie", définition du FMI. Au sens large, c'est l'augmentation planétaire de tous les échanges, quels qu'ils soient. \\
Plusieurs causes à la mondialisation: le rôle des FMN, qui est à la fois une cause et un effet, sauf qu'elles ne l'ont pas créées à elles seule. La mondialisation a surtout été favorisé par des mesures de dérégulation, notamment aux US, à l'OMC, etc. La mondialisation résulte donc de choix politiques. On compte comme causes aussi l'entrée dans le marché mondial des pays ex membres de l'URSS ainsi que la progression technologique qui a complètement modifié le rapport à l'espace et au temps, diminuant les coûts des transport et de communication par rapport aux coûts de production. \\
La mondialisation a d'abord profité aux pays développés. Aujourd'hui, elle lui apporte encore des bénéfices, mais cela dépend pour qui: la mondialisation profite surtout aux détenteur de capitaux, ou aux consommateurs, mais il y a des risques de délocalisations, donc dans les pays riches, le risque est la désindustrialisation et le chômage. Il est donc impossible de dire quelle perte a provoquée la mondialisation, par contre, il est clair que ceux qui en sont gagnant, réinvestissent peu d'argent pour dédommager les perdants. \\
Pour les pays du Sud, les conséquences de la mondialisation sont tout aussi contrastée. Les pays émergents eux, ont globalement bénéficié de la mondialisation, ils ont réussi à s'insérer dans le marché mondial, ont réussi à attirer des investissements, etc. Concernant les PMA, il y a là aussi des différences: certains sont très dynamiques économiquement, mais en même temps, il y en a qui restent très à l'écart de tout ça. \\
La mondialisation a rendu plus vulnérable les pays qui en ont profité. Plus un pays est inséré dans le commerce international, plus il risque de subir les soubresauts de l'économie mondiale. Il faut noter qu'à partir du moment où les pays ont adaptés leurs économies à la mondialisation, ils sont vulnérables aux ralentissements des échanges.


Pour certains, nous sommes entrés dans une période de démondialisation. Plusieurs signaux d'après les tenants de cette théorie: la croissance mondiale diminue, le commerce international augmente moins vite que le PIB mondial. Ils évoquent aussi ce qu'on appelle la détérioration de la chaîne de valeur: les FMN avaient l'habitude de faire une certaine division internationale du travail, mais cela est moins rentable depuis quelques temps, on assiste donc à une forme de relocalisation. \\
Concernant les flux de personnes, cela s'accélère quoi qu'on en dise. \\
Le modèle de l'État semble menacé de l'extérieur, par un ensemble de phénomènes, venus concurrencer l'État qui ne semble plus approprié pour gérer par ces phénomènes. L'État est aussi menacée de l'extérieur par la résurgence de tout un tas de revendications identitaires. 

\subsubsection{La question des "replis identitaires"}

L'expression doit être mise entre guillemets car elle est assez discutable. Elle s'est imposée dans les années 90. C'est discutable car c'est péjoratif, et permet de stigmatiser certaines revendications, comme si c'était un retour en arrière. \\
Ben Barber dans "Djihad versus Mcworld", explique qu'il y aurai d'un côté une tendance au renforcement particulariste religieux, ethnique ou culturel, ce qui suppose selon lui la "re-tribalisation de larges pans de l'humanité". Cette tendance là est au fond dirigée contre la modernité elle même selon lui. De l'autre côté, il y a ce qu'il appelle le "McWorld", une tendance à l'uniformisation des modes de vie, à la globalisation de l'économie et au développement de réseaux d'informations planétaires. Pour lui, les deux sont critiquables, et les deux ont une conception complètement antagoniste. \\
Ces tendances sont intimement liés, elles se nourrissent mutuellement. Les replis identitaires peuvent être vus comme une résistance face à la mondialisation, mais pas seulement. Certains mouvements identitaires profitent totalement de la mondialisation, profitent de la perméabilité des frontières. Certains chercheurs américains ont inventés un mot pour décrire ça: la "glocalization". 


Ces replis sont un défi pour les États car cela peut déboucher sur des troubles, sur des violences, sur des conflits armés. De façon générale, cela révèle la logique sur laquelle les États se sont construits. Les États, la France compris, se sont construit contre les fiefs locaux, contre les langues locales, etc. Ils se sont construits d'autre part en affirmant leur autonomie par rapport à l'Église. \\
Les réveils identitaires peuvent donc remettre en cause ces logiques. Ils peuvent prendre plusieurs formes. Ces réveils peuvent avoir des conséquences plutôt graves. \\
Les acteurs politiques se soucient surtout du réveil des identités religieuses. Cela est sans doute plus profond, car donne un sens, a une portée plus universelle que la notion d'ethnie. Cette religiosité dans les RI a été masquée pendant la guerre froide. \\
Le retour des religieux date des années 70 mais a pris son ampleur après la fin de la bipolarité pour remplir le vide qu'a mis la fin de l'URSS. La religion a pu donner un sens à la révolte de certaines populations. Ce réveil est un phénomène qu'on peut observer dans l'histoire, par périodes. Par exemple, l'islamisme est né avec la disparition du dernier empire Musulman, dans les années 20. Dans les années 70, l'islamisme a connu une grande résurgence à cause de l'arabisme, une idéologie plutôt laïque, cherchant à fédérer les peuples arabes, y compris chrétiens, incarnés par exemple par Nasser. \\
À la fin de la guerre froide, l'islamisme a connu un essor sans précédent, à cause de contextes internes aux pays musulmans car les gouvernements en place, issus des mouvements de libération nationale, qui monopolisaient le pouvoir. L'islamisme s'enracine surtout dans des pays musulmans qui connaissent de forts clivage sociaux avec une alternance politique inexistante. 


L'Algérie n'a pas été touchée en profondeur par le printemps Arabe. La contestation a eu du mal à s'organiser, sans doute traumatisé par la violence de la précédente décennie. Comme beaucoup de pays issus de la décolonisation, l'Algérie a été pendant longtemps gouverné par un parti unique, le FLN, dépassé ensuite par l'état major militaire. \\
Le pays a connu des problème divers: emploi, logement, etc. Des mouvements islamistes sont donc apparus et se sont renforcés petit à petit, mobilisant une partie de la jeunesse, les exclus. En 1989, le Gouvernement met en place des réformes pour permettre une certaine pluralité, ce qui permet aux islamistes de créer le FIS (Front Islamiste du Salut), qui prône l'instauration de la charia et la mise en place d'un islamisme social. Il obtiendra un très bon score, mais entre deux tours de l'élection législative, l'état major de l'armée a annulé l'élection, dissout le FIS et mit en place un gouvernement militaire. \\
À partir de là, fin 1991, l'Algérie a sombré dans la guerre civile, les islamistes ont opté pour la lutte armée, et ce qui restait du FIS a été dépassé par des plus extrêmes: le GIA. La violence s'est étendue au pays entier. On compte de cette décennie 200 000 morts environ. \\
La situation s'est apaisé courant des années 2000. Le Président Bouteflika, arrivé au pouvoir en 1999, a lancé une politique de réconciliation nationale, permettant à des islamistes de rendre les armes contre l'amnistie. Dans un contexte de vide de pouvoir (Bouteflika étant très malade), l'Algérie se tourne de nouveau vers une période d'incertitude. \\
Les héritiers du FIS sont désormais éclatés, affaiblis. Les islamistes armés sont beaucoup moins nombreux mais néanmoins toujours actifs. Ils ont d'ailleurs adoptés une nouvelles stratégie, visant à amplifier leur emprise au delà des frontières algériennes. Ils ont donc créés un nouveau groupe qui opère dans tout le Saël. Aujourd'hui, ce groupe est AQMI.


L'Islamisme a pu se présenter comme le seul moyen de résistance face à l'impérialisme, aux colonisateurs ou à la culture occidentale en général. Un des principaux ressorts de la montée de l'islamisme, c'est la quête de puissance de certains États: la religions est aussi utilisée dans un but international. Cela se manifeste très bien par la guerre d'influence entre l'Iran et l'Arabie Saoudite, le premier soutenant les Chiites, le deuxième les Sunnites. Ce clivage date de la mort du prophète, du choix de son successeur. \\
Il est à noter que les Chiites reconnaissent l'existence d'un clergé spécialisé, hiérarchisé, professionnel alors que ce n'est pas le cas pour les Sunnites. Ces derniers sont très largement dominant, mais les Chiites sont majoritaires dans certains pays, notamment en Iran, en Irak, en Azerbaïdjan, en Bahrein. \\
Dans cette guerre d'influence, l'Arabie Saoudite finance de nombreuses organisations sunnites. Ils ont notamment financés les frères musulmans, mais ont aujourd'hui arrêtés. Ils soutiennent donc plutôt les salafistes (officiellement que les modérés).


L'Iran cherche aussi à étendre son influence, ce que l'Arabie Saoudite ne supporte pas. L'Iran a toujours cherché, officiellement, à apporter sa révolution dans les autres pays musulmans. Le principal relais de l'Iran internationalement, c'est le Hezbollah et a donc une influence directe sur le Liban et le conflit Israelo-palestinien. Le Hezbollah a été fondé en 1982, en réaction à l'invasion Israélienne du Sud du Liban. \\
La lutte AS/Iran s'est intensifiée ses dernières années, les relations diplomatiques ont mêmes étés rompus depuis début 2016. Ils s'affrontent par pays interposés, notamment au Yémen, où il y a de nombreuses victimes.


Aujourd'hui, le monde connaît un réveil des identités ethniques ou religieuses, remettant en cause la centralisation du pouvoir ou remettant encore en cause la différenciation politique/religieux. \\
Le système international depuis la fin des années 1990, connaît tout un tas de phénomène qui remet en cause la puissance des États. Phénomène déclenché par les États eux mêmes souvent. Les réveils des identités ont toujours été stimulés par certains États, que ce soit l'AS, l'Iran, le Qatar ou même les USA eux mêmes. \\
Même les États les plus puissants semblent être désarmés aujourd'hui face à la mondialisation. Les OI autour de l'ONU connaissent elles mêmes une sorte de crise, les autres OI (FMI, OMC), sont elles de plus en plus controversé. L'ONU elle même pourrait être au dessus de tout ça, sauf que sa légitimité et son efficacité sont elle même controversé, puisqu'elle a montré son inefficacité à régler des conflits. L'ONU devrait donc se réformer en profondeur, changer notamment les mécanismes de prise de décision notamment au sein du conseil de sécurité. \\
Une nouveauté depuis la fin de la guerre froide: une augmentation des OI régionales. Certains pensent donc que ces OI pourraient servir de substitut à l'inefficacité de l'ONU. Parmi ces nouvelles OI, on a le MERCOSUR, la CELAC, l'UA, la CEDEAO. 

\chapter{Omniprésence et métamorphose de la violence dans les relations internationales}

Pour les réalistes, la violence est la normalité dans les relations internationales. Raymond Aron écrivait "Les RI se déroulent à l'ombre de la guerre". \\
Il existait une conception romantique de la guerre qui a beaucoup évolué à partir de la première guerre mondiale, guerre qui a engendré un nouveau système de valeurs. La deuxième guerre mondiale a contrebalancée ce qui s'est déduit de la première, le pacifisme n'étant pas ce qu'il fallait faire. \\
Aujourd'hui, les réalistes voient la guerre comme quelque chose de non souhaitable mais inévitable. \\
La violence internationale a subi des transformations profondes, où la frontière entre le national et international est flou. L'élaboration des armes de destructions massives a rendu flou la distinction entre paix et guerre. Surtout, la violence internationale n'est plus seulement le fait d'État, mais elle est aussi privée, transfrontalière, institutionnalisée et privée. \\
Le droit international a lui même connu des développements majeurs, cependant la paix par le droit s'heurte à des obstacles très sérieux. 

\section{Vers une dissémination de la violence internationale}

La violence est une forme de coercition brutale exercée contre des personnes, des biens, voire contre des symboles qui pour effet ou pour objectif de porter atteinte à leur intégrité. Il y a dans la violence l'idée de "brutale", donc ce qui est excessif, abusif. C'est une perception qui dépend du contexte dans lequel on se trouve. Le seuil de sensibilité à la violence est très évolutif, il peut s'élever ou se baisser selon les pays et les lieux, etc. \\
Il existe un lien entre violence et politique, puisque l'État se définit par son monopole de la violence légitime: "Nous entendons par État une entreprise politique de caractère institutionnelle lorsque et tant que sa direction revendique avec succès le monopole de la violence physique légitime" M. Weber. \\
L'idée dominante est que la violence est inhérente à la nature humaine (instinct de survie), mais cette violence naturelle aurait finit par être enrayé, canalisé, par l'institution d'un pouvoir politique (voir Hobbes). \\
D'autres auteurs vont contredire cela, Rousseau notamment. Dans un autre ordre d'idée, René Girard, un historien français, soutient lui aussi que la violence n'est pas inhérente à l'homme (ni bon ni mauvais) mais que la plupart des sociétés se construit dessus, se définit des victimes, pour souder le groupe et assurer la cohésion interne. \\
Milgram avec son expérience de soumission à l'autorité montre qu'on peut infliger des violences sous l'influence d'une autorité. \\
Certaines institutions ont réservés la violence, ou ont du moins, canalisé celle-ci vers des cibles désignés. \\
Tout cela, à l'échelle internationale, n'existe plus tellement.


\subsection{Quelques hypothèses sur les sources de la violence dans les relations internationales}

Il y a eu beaucoup d'études statistiques sur la guerre. On a essayé de mesurer les variations de violence dans l'espace et le temps, et si il y avait des leçons à tirer de toute ça. \\
D. Singer, Correlates of war project, a cherché d'accumuler un maximum de donnés sur les conflits internationaux ayant eu lieu depuis 1815. Ils ont essayés de mettre au jour des corrélations ou du moins des régularités statistiques pour observer si des circonstances favorisait la guerre. \\
Ce qu'ils ont réussi à démontrer c'est que les États neutre participent moins à la guerre, qu'elles débutes surtout au printemps ou à l'automne. Si ces affirmation semblent loufoques, il faut préciser qu'aucune hypothèse n'avait été posée et aucune problématique non plus. Il n'y a pas eu de vrai réflexion au départ pour savoir ce qu'était la guerre. Ce qu'ils ont posés, c'est prendre tout conflit armé engageant au moins deux États, mobilisant au moins 1000 soldats et ayant fait au moins 1000 morts.


À partir de 1945, un projet semblable a été lancé en France par G. Bouthoul, qui exerce en polémologie qui consiste à étudier le phénomène guerrier. Il dit "les guerres existent car ont une certaine rationalité, elles remplissent une certaine fonction d'ordre démographique", pour lui, la guerre joue le rôle d'un contrôleur démographique. \\
Il ne croit pas au droit pour éviter la guerre, ni le désarmement. La seule solution pour lui est de contrôler la population. Sa thèse a bien entendu été très critiquée. 


Des chercheurs ont fondés l'irénologie, la science de la paix, en 1958, on note Galtung dans les fondateurs. Sa thèse est surtout que la cause première de la violence dans le monde, ce sont les inégalités. Il a inventé le concept de la violence structurelle qui n'implique pas forcément la violence armée mais qui peut prendre la forme de la violence économique, alimentaire, etc. \\
Derrière l'idée de structure, c'est une idée de violence plus subtile, mais tout aussi destructeur, où ce n'est pas un individu qui exerce la violence mais une structure, qui peut engendrer une révolter et donc une contre violence. 


D'autres recherches ont concernés le régime, lesquels pouvaient engendrer des guerres, cherchant donc à vérifier ce que Kant disait à propos des démocraties qui ne se faisaient pas la guerre. Selon Kant, il y a un lien entre démocratie et paix car le pouvoir politique repose sur le respect des libertés, et est supporté par la société civile qui, justement, supporte le coût de la guerre si guerre il y a. \\
Dans un régime non démocratique, les dirigeants n'ont pas à assumer leur choix électoralement. \\
La réflexion de Kant repose sur le fait que les citoyens seraient pacifistes par nature, ce qui n'est pas forcément le cas. De 1945 à nos jours, les pays les plus engagés dans des guerres sont les US, la Russie, la GB et la France. \\
La thèse de Kant garde malgré tout des adeptes, notamment M. Doyle qui a montré que Kant avait raison sur un point: les démocraties ne se combattent pas entre elles. Il montre que parmi tous les conflits internationaux depuis 1945, aucun n'avait opposé deux démocraties entre elles. Beaucoup estiment que l'absence de guerre entre démocraties s'approche d'une loi empirique des relations internationales. Pour Doyle, la tendance pour les démocraties de partir en guerre ensemble permette une relation de confiance mutuelle. Cependant, c'est une corrélation sans lien de cause à effet, la notion de démocratie est subjective à Doyle.


Cela montre qu'il est difficile d'isoler des facteurs déterminants de la guerre ou de la paix. \\
La guerre au sens strict n'est d'ailleurs qu'une pratique de la violence. 

\subsection{De la guerre aux conflits armés}

\subsubsection{Qu'est-ce que la "guerre" ?}

Au sens strict, c'est un "conflit violent, mené aux moyens d'armes, et entraînant un effet homicide, c'est une activité collective organisée, opposant deux ou plusieurs États et entreprise sur décisions des autorités étatiques dans le but de contraindre la volonté de l'adversaire". Il est à noter que la guerre a des objectifs qui sont toujours de nature politique. Clauswitz disait que la "guerre c'est la poursuite de la politique par d'autres moyens".


L'ampleur des guerres s'étaient accrus à l'époque même de Clauswitz. Elles étaient jusqu'à la fin du XVIIIe siècle, limités, liés aux contraintes. On avait des guerres plutôt patrimoniales, avec des militaires de métier, dirigés par des aristocrates. L'issue des guerres se décidaient lors des bataille ou des sièges, et les forces ne cherchaient pas à se détruire. \\
À l'époque de la révolution française, cela a changé. On note l'appel des citoyens aux armes en 1793, ce qui est nouveau. On invente ensuite la conscription. À partir du XVIII, la guerre est devenue l'affaire des peuples, ont étés conduites au nom d'idées politiques. \\
Au XXe siècle, une nouvelle étape a été franchie, avec la possibilité de guerre mondiale où toutes les ressources d'un pays pouvaient être mobilisées. \\
La guerre au sens strict peut donc prendre plusieurs formes. Mais par définition, c'est une affaire d'États. On note aussi la guerre civile, la guerre de libération nationale. Aujourd'hui, la grande majorité des conflits sont asymétriques, ce qui explique que les stratégies de guérillas sont beaucoup utilisées. \\
Aujourd'hui, les guerres au sens strict sont très rare, ce qui amène certains auteurs penser qu'elles sont obsolètes. 

\subsubsection{La guerre est-elle obsolète ?}

Il y a eu très peu de guerres au sens strict. L'une des dernières est la guerre Iran-Irak, une guerre très classique, qui opposait des armées régulières, avec des lignes de fronts, etc. Une autre guerre est entre l'Éthiopie et l'Érythrée. Il y a eu aussi trois guerres entre l'Inde et le Pakistan, depuis, c'est une guerre indirecte. Il y a eu aussi de nombreuses guerres entre Israël et les pays Arabes, mais là aussi une grande transformation a eu lieu depuis les années 90, où les États ne participent plus. \\
La dernière vrai guerre serait l'intervention en Irak de 2002. Mais les opérations proprement dites ont durés à peine quelques semaines. \\
Les derniers conflits, meurtriers, sont plutôt des guérillas, des guerres civiles, ou encore des guerres de libération nationale. Depuis 1945, deux tiers des conflits étaient des guerres civiles ou des guerres de libération nationale. \\
Certains auteurs ont donc conclus que la guerre inter-étatique était devenu quelque chose d'obsolète (J. Mueller). Pour Mueller, les guerres inter-étatiques sont vouées à disparaître, en partie grâce à la dissuasion nucléaire, grâce aux changements de valeurs dus à la seconde guerre mondiale. Mais son principal argument est que la guerre ne paye plus, en tout cas pour les États disposants d'équipements et d'armées régulière. Cela ne vaudrait pas le coup de mettre en péril la prospérité existante. Selon Mueller, il n'est plus indispensable de faire la guerre pour acquérir une influence. Cela ne veut pas dire que tout va mieux, pour Mueller, la guerre sera infra-étatique ou transnationale. 


On note une corrélation très élevé entre l'IDH et les guerres. On constate que l'existence de conflit armée est là où l'IDH est le plus faible, l'un renforçant l'autre. Les protagonistes de ces conflits locaux reçoivent très souvent le soutien de grands États voisins, ou de grands réseaux transnationaux. Les OI comme l'ONU peuvent parfois s'en mêler, dans des missions de maintien de la paix ou peuvent même commettre des exactions. \\
Dans ce type de conflit, les États perdent souvent le contrôle des événements, les plus grandes violences étant l'oeuvre des armées non régulières. \\
La RDC a été frappée par une série de conflits depuis 1996. Il y a eu beaucoup de phases: des accords de paix, des changements de gouvernement, etc. Aujourd'hui les violences sont dans l'est du pays. En 1997, les États environnants ont aidés à renverser le dictateur, dans le but de s'emparer des ressources du pays. Les compagnies minières sont aussi impliquées. L'ONU aussi s'est impliquée. Aujourd'hui, ce conflit a provoqué 4-5 millions de victimes. \\
Ces conflits d'un nouveau type sont d'autant plus difficiles à résoudre, où on sait pas trop qui se bat contre qui, qui contrôle qui. Cela est embêtant, puisque si on veut négocier la paix, il faut bien avoir quelqu'un avec qui discuter. Ces conflits ne sont d'ailleurs jamais de façon interne, ce qui ne sert à rien d'essayer de les résoudre isolément, surtout que les intervenants extérieurs le font de manière officieuse. 

\subsection{Un défi pour le droit}

Le droit internationale essaye de réguler la guerre. Des OI sont chargés d'appliquer ces règles, l'ONU par exemple a des outils dans le but de maintenir la paix. \\
Le dispositif de sécurité collective est prévue par le chapitre 7 de la charte de l'ONU, ce dispositif permet au Conseil de sécurité d'adopter des mesures de coercitions contre un État qui mettrait en cause la sécurité et la paix internationale. Les mesures de sécurité vont de l'embargo économique jusqu'à l'intervention armée en passant par des gels financiers ciblés. \\
Les opérations de maintien de la paix ne sont pas prévus par la charte, mais ont étés inventés par un secrétaire général dans les années 1950. Ces OMP n'impliquent pas de mesures de coercition. Elles ne sont pas dirigés contre un État déterminé. Ces OMP ont pour but de geler les conflits. L'idée est de s'interposer entre les belligérants. Les troupes n'ont pas le droits d'utiliser leurs armes sauf en cas de légitime défense ou si le conseil de sécurité les y autorise. 


Il y a deux  corps de règles différents. Le "jus ad bellum" et le "jus in bello". Le premier limite le droit de déclencher un conflit. Le deuxième vise à restreindre les effets de la violence, une fois les hostilités engagées. 

\subsubsection{La réglementation du recours à la force armée ("jus ad bellum"): une solution irréaliste ?}

L'idée est d'interdire le recours à la force armée. Pendant longtemps, cette idée est restée purement utopique. Un concept inventé et développé dès le Moyen-âge, c'est le concept de guerre juste, qui visait à concilier les exigences politiques et la morale catholique. L'idée est de dire que la guerre est condamnable mais qu'elle peut se justifier. \\
D'après Saint-Thomas d'Aquin, trois conditions peuvent justifier la guerre: à juste titre (une autorité légitime) ; une juste cause (ce qu'on combat doit s'être rendu coupable d'une faute, ou chercher à réparer une grave injustice) ; une intention droite (le but ultime doit être de procurer un bien ou d'éviter un mal). \\
Cette doctrine a inspirée le discours de plusieurs gouvernants. Ce sont les US, en 2002, qui ont remis au goût du jour cette doctrine: 60 intellectuels américains avaient écrit une lettre ouverte "What we are fighting for", où ils cherchaient à savoir quand quelles conditions une riposte américaine était justifiée.


La doctrine de la guerre juste relève de la morale, n'est pas consacrée en tant que tel dans le droit. L'ONU s'en est cependant inspiré récemment pour forger un concept: la responsabilité de protéger. Ce principe a été reconnu par l'AG de l'ONU en 2005. C'est une version modernisée du droit d'ingérence. Cette responsabilité peut s'ajouter à l'idée de juste cause. \\
Ce n'est pas la première fois que des règles existent. Lors des monarchies absolus, elles se devaient de se déclarer la guerre avant de la faire. À partir du XVIIIe siècle, les déclarations de guerre nécessitaient l'approbation des parlements. C'est vraiment au XXe siècle qu'on a cherché à interdire la règle, après la première guerre mondiale. En 1919, le pacte de la SDN interdisait les guerres de conquête et interdisait les guerres tant qu'il n'y avait pas eu de tentatives de régler le différend de façon pacifique. \\
Le pacte Briand-Kellog, signé par la plupart des membres de la SDN en 1928, les États se sont engagés à renoncer à la guerre comme instrument de leurs politiques respectives. \\
Aujourd'hui, on a la charte de l'ONU qui pose le principe dans l'article 2§4: "Les membres de l'organisation s'abstiennent dans leurs relations internationales de recourir à la menace ou à l'emploi de la force, soit contre l'intégrité territoriale ou l'indépendance politique de tout État, soit de toute autre manière incompatible avec le but de l'ONU". 


Pour que le conseil de sécurité de l'ONU accepte une coercition au titre du chapitre 7, il faut que 9 votants soient en faveur de la mesure, dont au moins les 5 membres permanents (sur 15 membres au total). \\
Concernant la CIJ, un organe de l'ONU, les États ne sont pas obligés de la reconnaître. Lorsque deux États signent un traité, ils peuvent choisir de rendre la CIJ compétent pour traiter d'un différend sur ce traité. Les États peuvent reconnaître ou non la compétence générale de la CIJ. \\
Concernant la CPI, une juridiction non raccordée à l'ONU, elle est compétente que lorsque les dirigeants font partis d'un État qui ont ratifiés le statut. \\
Concernant la légitime défense, c'est dans l'article 51 de la charte de l'ONU "Aucune disposition de la présente charte ne porte atteinte au droit naturel de légitime défense, individuelle ou collective dans le cas où un membre des Nations Unis est l'objet d'une agression armée, jusqu'à ce que le conseil de sécurité ait pris les mesures nécessaires pour garantir la paix et la sécurité". On note que le conseil de sécurité a un certain monopole de définition de ce qu'est une agression armée ou non. Le statut de la CPI définit depuis peu le crime d'agresse comme "l'emploi par un État de la force armée contre un autre État", on tourne donc en rond avec l'article 2 de la charte de l'ONU. \\
Si on lit l'article 2§4, on voit que ça concerne les États et que les États, donc en cas de guerre civile ou de libération nationale, on ne sait pas quel droit appliquer. 


\subsubsection{La codification du droit de la guerre ("jus in bello") et ses failles}

Ce droit a pour but d'éviter aux combattants et aux non combattants des "souffrances inutiles". Organiser un tel droit, c'est avouer l'impuissance du droit à la guerre. On peut voir là du pragmatisme car il est toujours possible d'abuser du droit précédent, et avoir des gardes fous supplémentaires peut être vu comme une bonne chose. \\
On a des règles là qui ont longtemps étés coutumières, qui ont commencés à être codifiés au XIXe siècle. En 1864 est signée la première convention de Genève qui obligeait les États-partis à protéger les blessés de guerre et les équipes médicales, peu importe les nationalités. \\
Il y a eu une dizaine de convention signé à La Haye, entre la fin du XIXe et le début du XXe. Ce sont des règles à observer par les combattants: statut de neutralité, emploi de certaines armes, etc. Surtout, cette convention affirme que les belligérants n'ont pas un droit illimité concernant les moyens pour nuire à l'ennemi.


Ce sont surtout les quatre nouvelles conventions de Genève de 1949 qui constituent le coeur actuel du jus in bello. Cette fois, il s'agit de protéger les personnes qui ne participent pas ou plus aux combats. La première convention réactualise la toute première et porte sur le sort des blessés ou des malades. \\
La deuxième complète la première en spécifiant les statuts des combattants lors des batailles navales et l'obligation de secourir les naufragés. \\
La troisième concerne les prisonniers de guerre qui doivent être traités dignement, qui doivent être suivis médicalement, libérés à la fin des hostilités, etc. \\
La quatrième concerne les personnes civiles en temps de guerre. Il est interdit toute attaque intentionnelle qui porterait atteinte aux civils ou à leurs biens. Une puissance occupant un territoire étranger a des obligations: protéger les populations et ont l'interdiction de déporter les habitants. 


Comme les guerres au sens strict existent de moins en moins, cela peut être inneficace. \\
Deux protocoles additionnels ont étés adoptés en 1977. Ces protocoles parlent de conflits armés en général. Le premier s'applique aux conflits armés internationaux, ce qui englobe les guerres au sens strict mais aussi les conflits dans lesquelles les peuples luttent contre une forme de domination. Le deuxième s'applique aux conflits armés non internationaux, si les groupes en présence sont organisés, conduit par un commandement responsable. Aucun de ces protocoles ne s'appliquent aux violences sporadiques internes, comme les émeutes par exemple. À partir de quand a-t-on franchi le seuil nécessaire pour qu'une tension interne devienne un conflit ? \\
L'interdiction du recours à la force ou l'encadrement du recours à la forcé armée constitue certes un progrès indéniable, mais est très difficile à appliquer, surtout lorsque les façons même de faire la guerre changent. Le droit international a la tendance d'être en retard d'une guerre: les normes prises marchent pour la situation précédente mais pas pour la suivante. 

\section{Le fait nucléaire et ses conséquences}

Le nucléaire fait que la guerre n'est plus un outil comme un autre. L'objectif ne peut plus être de gagner, mais de prévenir une guerre. Ça donc été une grande nouveauté. En théorie, ces armes sont faites pour ne pas être utilisées, mais seulement pour être dissuadées. L'utilisation est d'ailleurs fortement condamnée, mais la possession donne de la puissance. \\
Le club des puissances nucléaires a tendance à s'accroître. Or, plus il y a de monde, moins la dissuasion ne peut marcher. 

\subsection{Les armes de destruction massives et le cas particulier des armes nucléaires: un état des lieux provisoire}

Les armes conventionnelles ou tactiques servent à infliger à l'ennemi des pertes suffisantes pour qu'il capitule. Les armes dites de destruction massives ont le pouvoir d'anéantir entièrement l'adversaire voire de lui infliger des dommages irréparables. Elles sont faites en principe pour dissuader. \\
À l'heure actuelle, certaines armes de destruction massive peuvent être utilisés comme des armes tactiques si la puissance est réduite et que la miniaturisation est faite. \\
On compte parmi les armes de destruction massive les armes biologiques, chimiques et bien sûr nucléaire. \\
En principe, les programmes de recherche d'armes chimiques et biologiques ont cessés à la fin de la guerre froide, mais de facto, ce ne serait pas du tout le cas. \\
Le droit international interdit formellement l'emploi, la construction ou la recherche de tels armes (biologique/chimique). Ces conventions ont étés ratifiés par presque la totalité des pays du monde. 


Pour les armes nucléaires, cela est différents. Les US ont mis au point cette arme et ont étés les seuls à les utiliser. L'URSS a suivi de peu les US. De là a commencé la course à l'armement. En 1952, la GB a testée sa première bombe, puis la France en 1960. Enfin, en 1964, la Chine. \\
À partir de là, la porte du club nucléaire s'est en quelque sorte refermé. En 1968 a été adoptée le TNP (Traité de Non Prolifération nucléaire). Dans ce traité, on a les EDAN (États Dotés de l'Arme nucléaire), ceux là ont le droit de conserver leurs armes mais s'engagent à ne pas les utiliser contre les autres parties au TNP, à ne pas diffuser leurs technologies et s'engagent à poursuivre les négociations en but d'un désarmement. Les autres, les ENDAN, s'engagent à ne jamais en fabriquer et ne jamais chercher à s'en procurer. En échange, ceux qui l'ont doivent les aider à se doter du nucléaire civil. \\
À noter que le TNP prévoit des zones où les armes nucléaires ne peuvent pas être entreposées comme l'Antarctique, l'espace, l'Amérique du Sud, etc. \\
Des traités interdisent les essais nucléaires, notamment celui de 1996 qui interdit entièrement les essais nucléaires. \\
Il y a des politiques internationales mis en place pour sécuriser la matière fissile. 


L'AIEA contrôle le respect du TNP. Elle a été créée en 1957 pour favoriser l'usage du nucléaire civil. À partir du TNP, elle a du contrôler l'usage de l'énergie nucléaire civile pour éviter qu'elle soit utiliser de manière militaire. Cependant, ses inspections exigent le consentement des États, sauf si le conseil permanent adopte une résolution. \\
Tous les 5 ans est organisée une conférence pour suivre l'application du TNP. Il y a un désaccord de fond entre ceux qui possèdent l'arme et ceux qui ne la possèdent pas. Pour les premiers, la non prolifération est l'important, pour les deuxième, c'est le désarmement. \\
À noter que le TNP ne dit rien sur l'utilisation de l'arme nucléaire. La CIJ a rendu un avis en 1996 disant que l'article 2§4 de la charte de l'ONU n'interdit aucune arme mais que cependant, l'usage des armes frappant sans distinction les civils comme les militaires est interdit et la CIJ a déclaré qu'elle ne pourrait pas se prononcer définitivement car les bombes nucléaires pourraient être miniaturisées et être très précises. 

\subsubsection{Définitions et aperçu historique}

Russie, Chine, US, GB, France, possèdent la puissance nucléaire. Ce sont aussi les cinq membres permanents du conseil de l'ONU. \\
En 2016, les US et la Russie, à eux deux, avaient approximativement 14 000 têtes nucléaires. Les US ont cependant un avantage technologique et stratégique car leurs têtes nucléaires sont déployés aussi en dehors de leur territoire. Ils ont 90\% de l'armement nucléaire à eux deux. La France a 300 têtes, la GB, 200, la Chine, 250. \\
Dans les faits, d'autres pays possèdent la bombe nucléaire: le Pakistan et l'Israël. Ils doivent avoir une centaine de têtes chacun. Les autorités Israéliennes n'ont jamais annoncés cela officiellement. Le Pakistan, lui, l'a annoncé officiellement. Cela n'est pas rassuré, étant deux pays dans des foyers de conflit important. On note aussi que plus le nombre de pays possédant la bombe augmente, plus le risque de fuites augmentent. \\
Une autre puissance nucléaire de facto existe: la Corée du Nord. Elle avait signée et ratifiée le TNP, puis a décidée de s'en retirer en 2003 après l'intervention militaire en Irak. La Corée du Nord a testé sa première bombe nucléaire en 2006. Elle a quelques capacités en matière de missiles balistiques, et prétend être sur le point de créer des missiles intercontinentaux. \\
Depuis 2003, les US ont réussi à faire adopté par l'ONU des résolutions de sanction financière contre la Corée du Nord. En 2007, la Corée du Nord s'était engagée à arrêter leur programme nucléaire, en échange de la levée des sanctions financières ; cet accord n'a pas été respecté. \\
L'Allemagne et le Japon, même si ils ont tout pour développer des bombes nucléaires, ils ont décidés de ne pas la développer. \\
D'autres pays avaient lancés des programmes nucléaires, bien avancés, mais ont abandonnés. Des pays de l'ex-URSS ont par exemple accepté de rétrocéder les armes nucléaires à la Russie. L'Argentine, le Brésil et l'Afrique du Sud ont abandonnés leurs programmes nucléaires lors de leurs transitions démocratiques. L'Irak possédait aussi un programme assez avancé, mais qui a été démantelé après la première guerre du Golfe. La Libye avait aussi un programme, mais, après avoir négocié, en 2003, a accepté de tout démanteler.

\subsubsection{Les risques de prolifération actuel}

On soupçonne des pays de chercher à se procurer des armes, comme l'Iran notamment. L'Iran a toujours nier vouloir fabriquer des armes nucléaires. Il revendique le droit d'accéder au nucléaire civil, cela dit, il a quand même développé un programme très avancé. Il est probable que la décision n'ait jamais été prise, car le système de décisions iraniens est très complexe. Certains disent qu'ils cherchaient à avoir un programme, pour être en mesure de fabriquer une arme mais sans franchir le dernier pas. \\
En 2002, l'AIEA a eu des preuves et des témoignages que l'Iran cherchait à enrichir de l'uranium en grande quantité. L'Iran a accepté de suspendre son programme et d'entamer des négociations. La France, la GB et l'Allemagne ont été les principaux négociateurs. \\
En 2005, les négociations s'interrompent et le programme reprend avec l'arrivé au pouvoir de Ahmedinejad. Waltz, un réaliste, pensait qu'il serait pas mal de laisser s'établir un équilibre nucléaire au Moyen-Orient avec l'idée qu'appartenir au club nucléaire a un effet modérateur. \\
Les sanctions ont étés la voie privilégié dès la reprise du programme en 2005. Le conseil de sécurité de l'ONU a pris plusieurs résolutions qui n'allaient pas jusqu'à l'intervention armée ni jusqu'à l'embargo total. Au départ, ces sanctions n'ont pas eu tellement d'effets, elles ont d'abord pénalisées surtout la population. Elles ont tout de même fini par faire effet, puisque l'économie Iranienne a fini asphyxiée. \\
En 2013, avec l'arrivée d'un nouveau président, Rohani, une nouvelle fenêtre d'opportunité s'est ouverte. Le but de Rohani était d'obtenir la levée des sanctions. Un accord a été conclu le 14 Juillet 2015. L'accord prévoit l'arrêt des centrifugeuses qui enrichissent l'uranium avec des inspections de l'AIEA en échange de la levée des sanctions. L'objectif est de ne pas laisser à l'Iran les ressources nécessaires à fabriquer rapidement une arme nucléaire. Aujourd'hui, avec le stock qu'a l'Iran, il lui faudrait plus d'un an pour fabriquer une arme. \\
Les voisins de l'Iran craignent que ce soit sur un gain de temps avant que l'Iran ne devienne véritablement une puissance nucléaire. Si l'Iran aurait un tel arsenal, ses voisins pourraient chercher aussi à avoir un arsenal nucléaire. On peut donc penser que l'Arabie Saoudite est plus prête que jamais à acquérir des armes nucléaires. 


\subsection{Dissuasion et doctrines nucléaires: quelques études de cas}

La dissuasion a toujours existé, un vieil adage romain dit d'ailleurs "Pour avoir la paix, il faut préparer la guerre". L'important, en principe avec l'arme nucléaire, est son existence. La dissuasion est tout de même quelque chose de complexe, il y a donc toute une doctrine. Chaque Nation doit voir dans quel cas elle utiliserait ou non l'arme nucléaire. Il ne faut pas non plus s'interdire de l'utiliser sinon on dissuade personne. \\
Ensuite, chacun doit faire connaître sa doctrine à son adversaire, tout en préservant un certain flou. C'est pourquoi la décision de lancer la bombe nucléaire est toujours concentrée dans les mains d'une seule personne. \\
Si frappe il y a, il faut savoir quelle cible sera frappée. Il existe deux stratégies: l'anti-ville et l'anti-force. Dans la première consiste à viser les infrastructures civiles, dans la deuxième, on va cible des objectifs militaires. Le premier est bien sûr bien plus dissuasif et le deuxième nécessite de la précision. \\
L'essentiel pour une puissance nucléaire est de posséder une force de seconde frappe. C'est à dire une force suffisante pour continuer le conflit après avoir subi une attaque nucléaire. Cela n'est pas à la portée de tout le monde. 


Quand l'URSS et les US se sont dotés de l'arme, ils n'ont pas pris conscience de cela. Tout a changé lors de la guerre de Corée, où les US envisageaient l'usage de la bombe. Le président Truman a dit non et à partir de là, les US et l'URSS ont commencés à développer leurs doctrines.

\subsubsection{Les doctrines nucléaires à l'époque de la bipolarité}

Les US ont énoncés leur première doctrine après la guerre de Corée. Les US se disaient prêt à utiliser l'arme nucléaire pour n'importe quelle attaque soviétique, même conventionnelle. L'idée était de prévenir toutes formes d'attaques conventionnelles et d'éviter des batailles traditionnelles coûtant cher. \\
Le problème de cette doctrine était le tout ou rien, ce qui est très dangereux. Si l'URSS ne prenait pas la menace au sérieux, les US étaient obligés de recourir à l'arme nucléaire. \\
La doctrine américaine et soviétique se sont développé en miroir. 


Après la crise de Cuba, il y a un changement de doctrine pour une "riposte graduée". Il y aurait toujours représailles massives au cas où l'URSS attaquait des villes américaines. Les US se donnaient comme options de moduler leur riposte. Cela suppose une diversification de l'armement, ainsi que de la communication pour éviter tout malentendu. \\
Pour les alliés des US, cette doctrine n'est pas forcément rassurante. Car les US ne prendraient pas forcément le risque d'une confrontation nucléaire. \\
À partir des années 70, des accords URSS-US ont eu lieu. Les deux puissances ont compris que la course aux armements n'était pas tenable. Pour maintenir l'équilibre, il fallait que chaque puissance soit certaine de pouvoir annihiler l'autre et d'être soi même détruit (MAD, Mutual Assured Destruction). \\
En 1972, plusieurs traités sont signés, notamment les SALT (Stratégic Arms Limitation Talks), qui fixaient un plafond maximum d'armes nucléaires à ne pas dépasser. Le traité ABM (Anti-Balistic Missiles) avait pour but d'empêcher les deux super-puissances de développer des systèmes anti-missiles car la première à réussir à développer ce système serait à l'abri de l'autre. 


Les petites puissances nucléaires ont aussi développé leurs doctrines. Celle de la France a été élaborée dans les années 60. \\
Elle est conçue comme strictement défensive et dissuasive. L'idée d'armes tactiques ou de ripostes gradués sont donc écartés. L'idée est de protéger les intérêts vitaux. La question est donc de savoir quels sont les intérêts vitaux. Il est clair que cela concerne notamment l'intégrité du territoire national. \\
La doctrine française reposait sur la dissuasion du faible au fort. L'idée est de pouvoir infliger à tout agresseur des dommages inacceptables, irréparables, même si l'agresseur est bien plus fort. La doctrine des petites puissances nucléaires est nécessairement anti-ville, car il n'y a pas les moyens de faire autrement. 


On peut dire que la dissuasion fonctionne globalement car depuis 1945, l'arme n'a plus jamais été utilisé. Cependant, la dissuasion n'explique pas tout. Si la dissuasion fonctionnerait tant, il n'y aurait jamais d'agression contre les puissances nucléaire, ce qui n'est pas vrai (cf, les agressions contre Israël ou l'Argentine contre la GB). \\
On pourrait dire que les pays qui ont aujourd'hui l'arme nucléaire n'ont jamais eu l'occasion de s'en servir. On peut aussi dire l'inverse. 

\subsubsection{Les révisions et les aléas de l'après guerre froide}

Quand le club nucléaire s'est agrandi, la dissuasion s'est compliquée. Il y a eu un changement de doctrines. \\
Les anciennes puissances nucléaires ont cherchés à se prémunir d'armes nucléaires possédés par des individus n'ayant pas la même rationalité. Il y a là dedans une part de préjugé qui consiste à dire que ceux qui cherchent à obtenir l'arme n'ont pas la même rationalité. \\
Il existe aujourd'hui une dissuasion du fort au fou. Ce doctrine est dangereux car rend le passage à l'acte possible. La nouvelle doctrine américaine prend en compte la fin de l'URSS. \\
Après la fin de l'URSS on note une diminution des arsenaux nucléaires. On note que chaque pays peut envoyer des observateurs pour vérifier qu'il y a bien une destruction effective des armes. C'est l'objet des accords START (Strategic Arms Reduction Talks). Le New Start de 2010, impose aux deux grandes puissances de conserver uniquement 1550 têtes nucléaires actives. \\
À noter: chaque président américain doit présenter un document sur sa doctrine nucléaire. Sous George W. Bush, les armes nucléaires était considérés comme dans l'ensemble des armes utilisables, y compris contre des États non nucléaires. Les US se sont concentrés sur la mise au point d'armes nucléaires de très faible puissance, susceptible d'être utilisé sur le terrain, et conçu pour frapper en profondeur. Obama a stoppé le développement de ces bombes. Pour lui, les US ont une énorme avance en termes d'armes conventionnelles, ce qui, en soi, est déjà suffisamment dissuasif. Obama s'est engagé à ne pas utiliser d'armes nucléaires contre les États non nucléaires, même si utilisation d'armes chimiques ou biologiques mais à condition qu'ils respectent le TNP. Obama a aussi dit que le but fondamental de la dissuasion est de dissuader contre une attaque nucléaire. Les armes nucléaires ne peuvent donc pas être utilisés de manière préventives. Mais c'est selon Obama un but fondamental, pas unique. \\
Trump avait annoncé sur Twitter que les US allaient accroître leur capacité nucléaire tant que le monde n'aura pas retrouvé la raison. Apparemment, il y aurai une forte pression dans l'armée pour reprendre le développement des armes tactiques de faible puissance.


Un bouclier anti-missile se développe. L'idée avait été lancée par Raegan. Tous les présidents y ont globalement contribués. Le système est censé intercepter tous les missiles cherchant à atteindre les US. Cela est contraire au traité ABM et Bush a annoncé le retrait du traité ABM, qui n'aurait plus de raison d'être étant donné que l'URSS n'existe plus. \\
En 2007, Bush a annoncé qu'il allait développer un système anti-missile en Europe de l'Est. Obama a revu le projet à la baisse, en l'étalant dans le temps. Les anti-missiles ont d'abord été déployés dans la méditerranée. Au final, il y aura un anti-missile en Pologne. \\
Obama a innové: les alliés européens sont entièrement impliqués, les pays membres de l'OTAN sont entièrement protégés. Obama a même proposé à la Russie de participer au programme. Ils n'ont jamais réussi à s'entendre sur les modalités de leur participation. \\
La Chine aussi se sent menacé par un tel dispositif qui a été installé en Corée du Sud. 


La France aussi a modifiée sa doctrine. Depuis les années 90, la France a réduit le nombre de ses têtes. Les crédits consacrés au nucléaire ont été réduit de moitié. Des installations ont étés démantelés et le TNP a été ratifié. Un accord a été conclu avec la GB en 2010, qui prévoit de mutualiser certaines capacités, notamment des installations de test et d'entretien. Les patrouilles en mer sont mutualisés. \\
Comme les US, la France fait reculer sa stratégie anti-ville et vise globalement les centres de commandement des ennemis potentiels. Les frappes préventives sont exclues. \\
Depuis la fin de la guerre froide, la France n'a jamais envisagée d'abandonner son armement. 


Le fait nucléaire a contribué à bouleverser les notions de guerre et de paix. La notion de dissuasion perd de sa rationalité au fil du temps. Quel que soit les précautions prises, il reste toujours quatre hypothèses de guerre nucléaire: celle par accident, à cause d'une erreur technique ; à cause d'un malentendu sur les intentions de l'adversaire ; la guerre par escalade ; la guerre par attention. \\
La solution du désarmement est difficile à envisager. On ne peut pas "désinventer" la bombe nucléaire, en effet, la connaissance existera toujours pour fabriquer de tels armes. Le désarmement n'a pas pour but de faire disparaître la violence en elle même. Les pires atrocités ont étés d'ailleurs commises avec des armes conventionnels, même sous l'ère nucléaire. \\
Certaines armes conventionnelles ont étés interdites comme les mines anti-personnelles. En 2013, l'AG de l'ONU interdit le commerce d'armes si l'objectif est de contourner un embargo. 

\section{Une tendance à la privatisation de la violence}

Le monopole de la violence publique légitime qui définit l'État tend à être de plus en plus reniée par une forme de violence privée. Cette violence a toujours existé, notamment parce que les États les tolérait. Or, aujourd'hui, ces violences privées revendique une certaine légitimité, ce qui correspond d'ailleurs à la crise de légitimité de l'État. \\
D'un côté, on a des acteurs qui défient ouvertement le monopole étatique: les mafias et les mouvements terroristes. Les États luttent en pratique contre ces acteurs mais c'est de plus en plus difficile. \\
D'un autre côté, on a d'autres acteurs qui sont, au contraire, soutenu par les États. Ce sont les mercenaires et les SMP.

\subsection{Mafias et mouvements terroristes: la renaissance d'une violence transnationale}

Bohuer définit les mafias et les mouvements antiterroristes comme des transnationales comme des autres. Il souligne la spécificité de ces organisations: l'élimination physique de la concurrence. Les activités de ces acteurs sont prohibés par le droit, interne comme internationale. \\
Différence fondamentale: à priori, les mafias cherchent des profits économiques alors que les mouvements terroristes recherchent le pouvoir. 

























\end{document}
