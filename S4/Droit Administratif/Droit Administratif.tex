\documentclass[10pt, a4paper, openany]{book}

\usepackage[utf8x]{inputenc}
\usepackage[T1]{fontenc}
\usepackage[francais]{babel}
\usepackage{bookman}
\usepackage{fullpage}
\setlength{\parskip}{5px}
\date{}
\title{Cours de Droit Administratif (UFR Amiens)}
\pagestyle{plain}

\begin{document}
\maketitle
\tableofcontents


\part{Les actes de l'administration}

\chapter{L'acte administratif unilatéral}

Cette question de l'acte unilatéral a évolué sous l'influence du législateur. Le CRPA est entré en vigueur en 2016, un Code des Relations entre le Public et l'Administration. Celui-ci vient codifier toute la jurisprudence administrative, qui n'était pas très claire alors.

\section{Définitions}

La première innovation de ce code est de donner en droit positif une définition de l'acte unilatéral de l'administration. L200-1 CRPA, "Les actes administratifs unilatéraux contiennent les actes décisoires ainsi que les actes non décisoires". Ce n'est donc pas vraiment une définition mais une énumération. \\
Un acte décisoire est un acte qui vient sanctionner l'utilisation d'une prérogative de puissance publique, puisque via ces actes, la puissance publique peut imposer sa volonté, agir contre la volonté des intéressés. Jusqu'en 2016, tous les actes décisoires étaient des actes unilatéraux, aujourd'hui, ce sont aussi des actes non décisoires, qui n'ont aucun effet juridique, ne produit aucun droit et aucune obligation pour les administrés. \\
Dans le CRPA, il est rappelé qu'il existe trois catégories d'actes décisoires, précisés à l'article L200-1, alinéa 2, on y compte les actes réglementaires (celui qui a pour destinataire des personnes désignés abstraitement). Ensuite, on y compte les actes individuels, qui, à l'inverse, a pour destinataire des personnes nominativement désignées ; ces actes peuvent être des actes individuels collectifs comme des décrets de nomination qui visent plusieurs personnes. La troisième catégorie sont les autres actes décisoires non réglementaires ni individuel, qu'on nomme en fait des décisions d'espèce qui se rapporte à une situation donné comme la dissolution d'un conseil municipal, une déclaration d'utilité publique, etc. \\
Une autre distinction à faire est celle des décisions qui viennent créer des droits pour les administrés. Ces décisions ont un régime plus protecteur pour les administrés. \\
Enfin, on distingue les décisions explicites de l'administration, formalisé dans un acte des décisions implicites, qui naissent du silence de l'administration suite à une demande d'un administré. Au bout d'un certain délai suite à la demande, une réponse est supposée ; alors que la réponse supposée était un rejet, la réponse supposée est aujourd'hui l'acceptation de la demande. 

\subsection{Les catégories d'actes administratifs unilatéraux}

Pendant longtemps, il y avait un lien direct entre le caractère décisoire d'un acte et la possibilité de faire un recours contre cet acte. On considérait en effet que seul les actes décisoires pouvaient faire grief au requérant. Cette solution était certaine: CE, 1950, Dame Lamotte, qui fait du REP un principe général du DA, et le juge indique que le REP est possible contre "toutes décisions administratives". \\
Aujourd'hui, il faut dissocier le fait qu'un acte soit décisoire pour qu'il soit attaqué. On constate depuis quelques années que certains actes, pourtant décisoires, ne sont jamais susceptible de recours contentieux, alors qu'à l'inverse, certains actes non décisoires, peuvent faire l'objet depuis quelques années de recours juridictionnel. \\
Soft Law: le droit souple n'était pas susceptible de recours, aujourd'hui, c'est possible. \\
Le JA n'a jamais tenté de définir ce qu'était l'acte administratif unilatéral. 

\subsubsection{Les actes décisoires de l'administration}

En principe, l'acte décisoire fait grief et est susceptible de recours. Pourtant, certains actes décisoires ne font jamais grief et ne sont jamais susceptibles de recours contentieux. Ce sont les mesures d'ordre intérieur (MOI). 


Les actes décisoires sont susceptible de recours contentieux. On doit donc se demander quel est le critère de l'acte décisoire pour le JA. Il l'a dégagé à l'examen de certains actes de l'administration comme des circulaires.


Concernant les circulaires, elles s'inscrivent dans le pouvoir d'instruction appartenant au chef de service qui dispose d'un pouvoir réglementaire pour gérer leur service. Les chefs de services disposent aussi d'un pouvoir d'instruction, leur permettant de donner des ordres aux subordonnés. Ce pouvoir d'instruction appartient au chef de service mais aussi au PM qui peut donner des instruction aux membres du gouvernement, leur demandant d'agir dans un sens donné (CE, Sect., 26 Décembre 2012, Association Libérez les Mlle). \\
Ce pouvoir d'instruction permet au chef de service d'édicter des instructions ou des circulaires. Ces dernières sont souvent longues, techniques, denses. C'est la communication d'un supérieur hiérarchique à l'attention d'un subordonné, où le chef de service va à la fois donner des ordres, rappeler le droit positif, la JP, viennent interpréter le droit positif. Par principe, ces circulaires sont donc interne aux services, et c'est pour cette raison que les administrés ne pouvaient pas contester ces circulaires. Cela posait deux séries de difficultés: une circulaire peut s'avérer illégale, car une instruction ou une interprétation pouvait être contraire à une norme supérieure ; la deuxième concerne le contenu même car pouvait concerner directement les administrés, car les circulaires pouvaient contenir de nouvelles règles. C'est pour ces deux raisons que le JA a accepté de contrôler certaines circulaires.


Trois temps de la JP: CE, Ass., 29 Janvier 1954, Notre Dame du Kreisker, dans laquelle le juge accepte la recevabilité des recours dirigés contre les circulaires réglementaires de l'administration. Le JA distingue deux circulaires: la réglementaire et l'interprétative, la première ajoute quelque chose au droit positif, modifie l'ordonnancement juridique, posait une nouvelle règle de droit, etc. Le JA a considéré cette première comme décisoire. Concernant les circulaires interprétatives, le JA refusait de les contrôler. Deux difficultés de cette JP: la distinction entre les circulaires réglementaires et les interprétatives: la JA nous dit que la réglementaire ajoute quelque chose au droit positif alors que l'interprétative, non. Cela est curieux car l'interprétation est un acte de volonté de celui qui interprète, où une méthode est choisie, il y a donc le choix d'une interprétation parmi d'autres. Une circulaire interprétative ajoute donc aussi quelque chose au droit positif, indirectement. \\
La seconde difficulté concernait les circulaires réglementaires. Cependant, en France, le pouvoir réglementaire est cantonné à certaines autorités: le PM, les autorités dont la loi donne un pouvoir réglementaire. Des autorités incompétentes ont tout de même édictés ce genre de circulaires, et le juge, quand il jugeait au fond, annulait ce genre de circulaires. Avec les circulaires interprétatives, que le juge refuse de contrôler, celle là pouvait être illégale.


Deuxième temps: le JA va accepter de contrôler certaines circulaires interprétatives, notamment celles qui sont illégales. CE, 15 Mai 1987, Ordre des avocats à la Cour de Paris ; CE, Sect., 1993, IFOP. Dans ces deux cas, le JA accepte de contrôler des circulaires interprétatives illégales. \\
Difficulté: suivant cette jurisprudence, le juge confond la recevabilité du recours et le contrôle au fond de l'acte. 


Troisième temps: CE, Sect., 18 Décembre 2002, Mme Duvigneres, cet arrêt étend les possibilités de recours contre les circulaires. Le JA nous indiques que les circulaires impératives à caractère général sont désormais susceptible de recours. \\
Une circulaire impérative est une circulaire qui tend à imposer quelque chose à ses destinataires, comme les circulaires réglementaires qui sont toujours impératives. \\
En reconnaissant ce genre de recours, le JA admet implicitement que toute autorité dispose d'un pouvoir d'édicter de telles circulaires. \\
Le JA n'est pas du tout regardant quant à la forme de l'acte qu'il examine. Le juge évoque notamment les circulaires ou les instructions. En pratique, ces actes peuvent trouver des dénominations très variées, comme des notes de services, des recommandations, ce qui importe, c'est le caractère impératif de l'acte. \\
Pour savoir si l'acte est impératif ou non, le juge va s'attacher à la rédaction de l'acte. Si la circulaire se contente de préconiser, de recommander un comportement, la circulaire n'est pas impérative. À l'inverse, si la circulaire indique des conditions, alors l'acte est considéré comme impératif et susceptible d'un recours contentieux. \\
Il existe des actes qui en apparence ne sont pas impérative, mais qui le sont en fait. Certains actes intitulés "recommandation", comme celle du CSA par exemple, sont impératives malgré leur apparence. CE, Ass, 2009, Mr Hollande, était en cause des règles de propagandes électorales, les recommandations donnés par le CSA étaient bien impératives et susceptible de recours. \\
Dans le cas d'un communiqué de presse d'une autorité, il peut être considéré comme un acte et susceptible d'un recours. CE, 16 Mars 2009, EMM, le CE admet un recours dirigé contre un communiqué de presse qui contenait une interprétation du droit positif. \\
On peut prendre en troisième exemple un tweet du ministère de l'intérieur, qui imposait aux supporters de "bien se comporter aux abords des stades" et disait aussi qu'ils ne devaient pas tenir de propos politiques, idéologique, raciste, ou xénophobe. Ce tweet est impératif car impose un comportement aux supporters, et il est anticonstitutionnel car il est en contradiction avec la liberté d'expression. Si ce tweet avait été contesté en justice, le recours aurait été certainement accepté.


Depuis un décret du 8 Décembre 2008, il existe des règles de publication des circulaires, celles-ci, quand elles sont ministérielles, doivent être publiées sur le site internet du PM. Si cela n'a pas été fait, la circulaire n'est pas opposable à l'administré. Le décret dispose que si les circulaires antérieures n'ont pas été publiées, elles sont considérés comme implicitement abrogées. CE, 23 Février 2011, Association La Cimade. \\
Le CE a étendu le champ d'application de ce texte, CE, 16 Avril 2010, Mr A, le CE considère que les obligations de publications concernent aussi les instructions non écrites. Cette JP supprime la catégorie des instructions non écrites puisqu'elles doivent toutes être écrites. \\
Un décret du 6 Septembre 2012 admet que les circulaires puissent être publiées sur un autre site internet que celui du PM, notamment les sites des autres ministères. \\
Depuis 2016, dans le CRPA, il est codifié l'obligation de publication: L312-2, qui dispose que doivent être publiées les "instructions, circulaires, notes et réponses ministérielles, seulement celle qui contiennent une interprétation du droit positif ou une description des procédures administratives". On constate deux lacunes: le champ d'obligation de publication qui ne concerne que les circulaires interprétatives, les circulaires réglementaires ayant étés totalement étés écartés. En 2016, dans le CRPA, il n'y a plus de sanctions au manque de publications, on peut donc penser que le JA va tenter de ré-appliquer sa JP.


Les actes décisoires qui ne font jamais, ou presque jamais grief, ce sont les MOI. Ces mesures sont internes aux services, ce sont souvent des sanctions prononcés par un chef de service à l'encontre d'un agent, d'un usager. \\
Ces MOI sont décisoires. Le JA refuse de contrôler ces actes au contentieux, il considère qu'une MOI ne fait jamais grief au requérant. Cela s'explique pour deux raisons, déjà par l'adage que le juge ne s'occupe pas des affaires mineures par crainte d'être encombré. La seconde raison est que le contrôle de ces mesures auraient une incidence sur la discipline du service. \\
Ce refus du JA a posé des difficultés. La première tient à ce que ces MOI ne sont pas toujours mineur pour l'administré comme le renvoi d'un élève d'un collège, la mise en isolement d'un détenu dans une prison, etc. La seconde difficulté est celle de l'influence de la CEDH (art. 13, droit au recours effectif devant un juge), la CEDH a considérée pendant les années 90 que l'absence de recours pouvait être vu comme une atteinte à ce droit (Ramirez-Sanchez c/ France). \\
Le CE a progressivement réduit la catégorie des MOI. Le CE considère aujourd'hui que certaines mesures qui étaient alors considérés comme des MOI ne le sont plus et peuvent donc faire l'objet d'un recours contentieux.


On peut noter quatre temps d'évolutions de ces MOI. \\
D'abord, pendant les années 90 où le JA va accepter de contrôler des mesures qui étaient auparavant des MOI. CE, Kherouaa, 1992, le JA accepte de contrôler une mesure d'exclusion d'une élève, mesure qui n'est donc pas considéré comme un MOI. \\
CE, Ass, 17 Février 1995, arrêt Hardouin et deuxième arrêt Marie (du même jour), le JA accepte de contrôler des sanctions infligées à Mr Hardouin, un militaire et des sanctions infligés à Mr Marie, un détenu. \\
La nature de la mesure mais aussi la gravité, et les effets de la mesure sur l'administré peuvent faire qu'une MOI n'en est pas une. Le contrôle se fait donc nécessairement au cas par cas. \\
En 2007, trois décisions rendues le même jour, le 14 décembre: CE, Ass, Boussouar, concernant les mesures de transfert d'un détenu, posant un caractère subsidiaire d'identification des MOI. Même si les critère de nature et de gravité ne sont pas remplis, la mesure n'est pas une MOI selon le critère qu'il pose: l'atteinte aux droits et libertés de la personne concernée. Le juge distingue le transfert d'un établissement pour peine à une maison d'arrêt, qui n'est pas une MOI, eu égard à la nature et la gravité de la mesure. Lorsque c'est l'inverse, puisqu'il n'aggrave pas la situation du détenu, cela semble être une MOI eu égard à la nature et à la gravité, sauf si le critère subsidiaire est rempli (le transfert peut porter atteinte au droit à une vie familiale normale, à l'accès aux soins, etc). \\
Cette évolution s'est poursuivie au sein de la prison car le JA a accepté progressivement de contrôler les refus d'emploi, les rétentions de correspondance, etc. \\
CE, 21 Mai 2014, Garde des Sceaux, qui concernait un avertissement infligé à un détenu, avertissement qui est la sanction la plus légère et le CE a dit que celui-ci n'était pas une MOI. Il a émergé aujourd'hui un contentieux pénitentiaire, avec des spécialistes de ce droit. \\
Dans un quatrième temps, CE, Sect., 25 Septembre 2015, Mme Bourjolly, qui concernant une mesure de réaffectation imposée à un agent. Deux apports à retenir: le JA donne une position de principe, pour lui une mesure de ré-affectation est une MOI, par principe. Cependant, si les critères sont remplis, une action peut être menée contre cet acte. Si la ré-affectation est une mesure de sanction, alors une procédure doit être respectée, et la mesure sera généralement annulé, l'administration ne respectant pas cette procédure. Cette décision évoque des MOI discriminatoire et le juge semble nous dire qu'une MOI discriminatoire est susceptible de recours contentieux. Il affirme donc qu'une MOI peut être susceptible de recours. \\
En plus du développement du contentieux pénitentiaire, on peut noter le développement du contentieux disciplinaire (collège/université, etc.). Cela a tendance à encombré le JA. 

\subsubsection{Les actes non décisoires}

En principe, un acte non décisoire ne fait pas grief et n'est donc pas susceptible d'un recours au contentieux. Trois catégories d'actes sont considérés traditionnellement comme non décisoires: les actes préparatoires, CE, Ass., 15 Avril 1996, Syndicat CGT des hospitaliers de Bédarieux, le JA pose une limite, le préfet admet que si l'acte n'est pas décisoire, le préfet est le seul à pouvoir former un recours ; les actes confirmatifs, des actes dont le contenu est identique à un acte précédent ; les déclarations d'intention qui ne révèlent l'existence d'aucune décision, CE, 5 Octobre 2015, Comité d'entreprise du siège de l'IFREMER, était en cause un discours du PM et non une décision. \\
Il existe aujourd'hui deux catégories d'actes non décisoires mais qui sont pourtant susceptible d'être l'objet de recours au contentieux. On a d'abord les lignes directrices de l'administration et le droit souple de l'administration. 


Concernant le droit souple, il a été pendant longtemps une notion doctrinale, assez confuse. On désignait parfois les normes informelles de l'administration, les normes concertés de l'administration (qui sont édictés après consultation ou concertation avec les administrés), ou encore ce qu'on entendait être le droit mou ou le droit doux, celui qui n'en est pas un, c'est à dire une norme non contraignante. Cette troisième conception a été prise par le CE. \\
On peut définir très simplement le droit souple comme une norme n'étant pas juridique, une norme de comportement qui ne produit pas d'effets juridique. Le droit souple, ce sont toutes ces normes qui ne sont pas contraignantes, qui n'impose aucun comportement mais qui en inspire. Par exemple, une recommandation du ministre de l'intérieur d'allumer des feux de route en journée n'est pas contraignante, elle tend seulement à influencer le comportement des administrés. \\
À priori, le droit souple n'est pas du droit. C'est une norme, certes, mais elle n'a pas d'effets juridiques. Pourtant, cette norme est très souvent respecté, produit des effets pratiques certains. Une théorie sociologique dit que ces actes ne sont pas respectés en fonction de leur nature mais en fonction de l'autorité qui les a édités. C'est parce que telle autorité l'a édicté que tel administré va la respecter. \\
Le droit souple sont des normes apparus dans l'ordre juridique internationale. Dans le droit international, le droit souple est très utile pour inviter les États à adopter tel ou tel comportement, ainsi qu'en témoigne les recommandations de l'ONU. On peut aussi penser aux recommandations de la commission européenne. \\
Ce droit souple a toujours existé dans l'ordre interne, on peut par exemple assimilé les plans de développement avant les années 80 comme du droit souple. Cependant, il a évolué quantitativement dans le domaine de la régulation économique notamment, géré par les AAI, AAI qui font un usage important de ce genre de normes. En effet, pour les AAI, cela leur permet de faire évoluer rapidement les normes applicables, très utile dans des normes très techniques, le tout sans aucun formalisme. Parfois, il permet de se substituer au droit dur dans le sens où les AAI n'ont pas la compétence d'édicter des normes de droit dur. Un dernier avantage, qui est aussi un inconvénient, c'est que n'étant pas impératif, il n'est pas contrôlé.  C'est pour cette raison que le JA a évolué et a accepté, en 2016, de contrôler le droit souple.


CE, Ass., 21 Mars 2016, deux arrêts du même jour: Société Fairvesta | Numéricable. Dans les deux étaient en cause une prise de position de l'autorité de la concurrence et un communiqué de l'AMF. Le JA va ouvrir un nouveau REP contre ces actes de droit souple. \\


Régime contentieux du droit souple. \\
Condition de recevabilité: le juge va distinguer deux séries d'actes de droit souple, dégagé par CE, 11 octobre 2012, Casino Guichard Perrachon, dans laquelle le juge parle des "actes qui énoncent des prescriptions individuelles dont l'autorité pourrait censurer ultérieurement la méconnaissance". Le critère de recevabilité du recours est donc le critère de sanction. \\
Deuxième catégorie d'acte de droit souple: ceux qui produisent des effets concrets, qui ne sont pas des effets juridiques. Dans les deux décisions de 2016, il était question d'une prise de position d'une autorité, dans l'autre affaire, il était en cause un communique de l'AMF qui mettait en garde les investisseurs contre certains investissements. Si recours contentieux il peut y avoir, c'est en raison de ce qu'implique ces actes. Le juge considère que le recours est recevable quand l'acte "produit des effets notables, notamment de nature économique", il prend en compte le fait que l'acte puisse influencer les destinataires de l'acte. \\
Première remarque: le juge est indifférent quant à la dénomination de l'acte, le juge prend de la même manière les avis, les communiqués, les recommandations, les prises de position, etc. Le juge n'est pas regardant quant à la forme de l'acte, ce qui compte, ce sont les effets de l'acte. \\
Deuxième remarque: en prenant en compte les effets de l'acte, il analyse ses conséquences, qui détermine si le recours est recevable ou non. Les effets concrets sont souvent des effets économiques, mais pas que. Dans les deux arrêts ci-dessus de 2016, les effets économiques sont évidents. Le juge prend aussi en compte que l'acte influe en pratique sur les comportements. Les critères sont donc alternatifs mais peuvent se cumuler. \\
Troisième: cette méthode, d'analyser les effets, n'est pas révolutionnaire, c'est ce que fait le juge aussi lorsqu'il contrôle les MOI, où il contrôle les effets des mesures. CE, 30 Janvier 2015, Région PACA, était en cause une prise de position de l'ARAFER, le juge a considéré que cet acte de droit souple pouvait faire l'objet d'un contentieux (cet arrêt annonce ceux de 2016). \\
Quatrième: ce critère des effets sera difficile à manier pour plusieurs raisons. D'abord à cause des termes employés par le juge qui parle de "l'influence significative des actes de droit souple" ou encore les "effets notables", où on peut se demander ce que cela veut dire quand on sait que les actes de droit souple sont globalement respectés, et que donc le champ de recours peut être large. Dans un REP, le juge apprécie l'acte au jour de son édiction, il n'est pas censé prendre en compte les changements postérieurs à l'acte. Le droit souple fait partie du non droit, en principe ; or, depuis 2016, un recours est possible, donc, depuis 2016, ces actes produisent de nouveaux effets juridiques puisqu'on l'associe à un régime juridique, ces actes ne sont donc plus du non droit ; on pourrai donc parler de "semi-droit". \\
Dernière remarque: la solution rendue en 2016 est restée cantonnée aux actes de droit souple rendu par les autorités de régulation. À noter qu'il n'existe pas de définition de la régulation en droit positif, ce qui pose une difficulté. En pratique, cela renvoie à des AAI, mais ce ne sont pas les seules autorités de régulation. Rien ne justifie de cantonner cette décision à la régulation, même si en pratique, ces autorités édictent beaucoup d'actes de droit souple. 


La construction d'un contentieux du droit souple. \\
Concernant les autres conditions de recevabilité. À l'instar de tout acte susceptible de recours, d'autres conditions classiques: l'intérêt à agir du requérant et le délai d'action. Tout administré n'a pas forcément intérêt à agir, qui doit être direct et certain. Les destinataires de l'acte de droit souple ont un intérêt direct et certain, ces destinataires peuvent être des opérateurs ou des consommateurs. D'autres requérants sont possible: les concurrents, les co-contractants des opérateurs concernés. Le juge apprécie très souplement l'intérêt à agir: CE, Novembre 2016, concernant un communiqué du CSA qui portait sur une publicité particulière, concernant les enfants trisomiques, la publicité a été jugée comme une entrave à l'avortement par le CSA, et la personne physique, trisomique qui avait tourné dans le publicité, son intérêt à agir est difficile à cerner, mais a été admis. \\
Le délai de recours: CE, Sect., 13 Juillet 2016, Société GDF-Suez, le juge considère que le délai de recours est un délai classique de deux mois, qui court à compter de la publication de l'acte sur le site internet de l'autorité. Ce délai s'applique sauf réglementation contraire comme un texte législatif pourrait le prévoir. Si un requérant laisse passer le délai, il peut quand même contester indirectement l'acte de droit souple en demandant dans un premier temps l'abrogation de l'acte de droit souple à l'autorité, et contester ensuite le refus de l'autorité, qui permettra au juge de contrôler l'acte. Le requérant pourra aussi contester le refus d'édicter un acte de droit souple (CE, Sect., 2007, Tiniez). \\
L'office du juge du droit souple: on se place dans l'hypothèse où le recours est recevable. On se pose donc la question de l'étendue du contrôle du juge de droit souple. Contrôle adapté est un terme qui ressort des deux arrêts de 2016, "en prenant compte de la nature et des caractéristiques de l'acte contesté, en prenant en compte le pouvoir d'appréciation dont dispose l'autorité". On a donc l'impression qu'il y a un recours particulier alors que non, c'est bien un REP. Ce REP semble très classique, le juge va contrôler d'une part la légalité externe et la légalité interne, comme il fait pour tous les actes administratifs. Externe: compétence de celui qui édicté, si il y a un vice de procédure ou un vice de forme. Interne: l'erreur de fait, l'erreur de droit, l'erreur dans la qualification juridique. Deux spécificités de ce recours: concernant le contrôle de la qualification juridique des faits, cela est variable suivant les domaines, même si le contrôle normal est annoncé, le juge opère en réalité un contrôle assez superficiel (ce qui est difficile de faire bien dans un acte qui ne relève normalement pas du domaine du droit) ; le recours en annulation peut être cumulé avec un recours en responsabilité de la puissance publique (un acte illégal constitue nécessairement une faute de la puissance publique), les arrêts de 2016 confirment cela. 


Les lignes directrices de l'administration. \\
Identification des lignes directrices. Elles avaient un autre nom jusqu'en 2014: directives de l'administration, un terme qui était assez ambigu avec les directives de l'UE. Ce nom de ligne directrice, on le trouve dès CE, 19 Septembre 2014, Jousselin. \\
Ces lignes directrices émanent d'autorités disposant d'un pouvoir discrétionnaire. Cette autorité va précisé les critères ou conditions sur lesquelles elle va exercer son ou ses pouvoirs discrétionnaires. Ces lignes directrices permettent à une autorité de s'auto-encadré. Au demeurant, ces lignes ne sont pas impératives dès lors que les conditions posées ne doivent pas être strictement respectés. En d'autres termes, l'autorité qui édicte ces conditions doit toujours pouvoir s'en écarter pour apprécier les demandes individuelles. \\
CE, Sect., 4 Février 2015, Mme Cortes-Ortiz, le JA poste quatre éléments de définition des lignes directrices. Le juge rappelle le pouvoir discrétionnaire de l'autorité qui les édicte, quand un texte prévoit l'attribution d'un avantage sans avoir défini l'ensemble des conditions permettant son attribution. Par ses lignes directrices, l'autorité vient s'auto-encadré, ou encadré des services qui sont sous son autorité. Le juge précise que dans ses lignes directrices, l'autorité va donner des critères pour mettre en oeuvre le texte, le juge précise que l'autorité n'édicte aucune conditions nouvelle. Le juge nous rappelle qu'il est toujours possible de déroger à ces lignes, pour tenir compte des situations particulières, mais aussi pour des motifs d'intérêt général. 


Le régime contentieux des lignes directrices. \\
N'étant pas impératives, elles ne sont normalement pas susceptible de recours contentieux: CE, Sect., 1970, Crédit Foncier de France. On peut cependant les contester de manière détourné, par la voie de l'exception d'illégalité. \\
À l'occasion d'un recours, un requérant va contester la décision d'application de ces lignes directrices. À l'occasion du recours, le juge peut contrôler ces lignes, au regard des normes supérieures, et si elles se révèlent illégales, le juge va écarter ces lignes du règlement du litige. \\
Depuis 2015, ces lignes directrices deviennent opposable (arrêt Cortes-Ortiz). Il y a la possibilité de se prévaloir de ces normes devant le juge, qui considère qu'elles sont invocables dès lors qu'elles ont étés publiées par l'administration. Le juge fait désormais la distinction entre les lignes directrices et les simples orientations générales de l'administration. Le juge considère qu'il y a lignes directrices lorsque l'administré peut prétendre un avantage, et que les lignes ont étés publiées. Quand l'administré n'a pas de droit à prétendre à un avantage, les lignes sont en fait des orientations générales, ne pouvait pas être invoqués. 


Les chartes de l'Administration. \\
Ce sont des documents qui se développent de plus en plus en matière administrative (par exemple, les chartes de la laïcité dans le service public, une charte de doctorat). Une charte est un document conventionnel, dans lequel il y a eu un consentement des parties. Elles sont non impératives, elles sont destinées à orienter l'action des pouvoirs publics. Les chartes produisent quand même parfois certains effets juridiques, qu'à l'égard des signataires de cette charte. Elle ne produira donc jamais des effets à l'égard des tiers et donc des usagers du service public: CE, Sect., 8 Février 2012, UICMARA. Certaines chartes n'en sont pas vraiment et ne contiennent pas vraiment d'obligation réciproque comme les chartes de la laïcité qui sont présentes dans les collèges, hôpitaux, etc. Ces chartes sont susceptibles de recours, si elles sont impératives ou non.

\section{Le caractère administratif de l'acte unilatéral}

\subsection{La détermination des actes administratif unilatéraux}

Principal critère: un critère organique, l'auteur de l'acte. 


Les actes des personnes publiques. \\
Il existe un principe et deux dérogations. \\
Principe: les actes édictés par une personne publique sont présumées être administratif. La présomption est simple et peut être renversée dans deux cas. Les actes relatifs à la propriété privé de l'administration sont des actes de droit privé, cette propriété privée ne sont pas rattachés au service public, la personne publique gère donc ces propriétés comme un propriétaire ordinaire. \\
Seront aussi de droit privé les actes non réglementaires relatif à la gestion d'un SPIC. 


Les actes des personnes privées. \\
C'est le principe inverse, il y a là une présomption du caractère privée des actes des personnes privées. Deux dérogations. \\
D'abord, les décisions d'une personne privée gérant un SPA et mettant en oeuvre des prérogatives de puissance publique. \\
Avant la seconde guerre mondiale, le CE a posé seulement le critère du SP en considérant que les actes d'une personne privée gérant un service public sont administratif (CE, 1942, Montbeurt). Deuxième temps: CE, Sect., 1961, Magnier, qui a un triple apport, il précise le critère du SP, il cantonne l'arrêt Montbeurt au SPA, seule les décisions individuelles d'une personne privée gérant un SPA sont administratifs. Il faut enfin que l'acte mette en oeuvre des prérogatives de puissance publique. Troisième temps: l'arrêt Magnier est étendu: CE, Sect., 15 Mai 1991, Association Girondins de Bordeaux Football Club, le JA va considéré que le règlement de la ligue est bien un acte administratif, la ligue étant un SPA organisant des compétitions officielles. \\
CE, 17 Octobre 2012, Mr Singa, concernant une décision prise par un évêque pour l'organisation du culte catholique, il voulait utiliser des biens du service public. Le JA va considéré que cet acte est un acte de droit privé en application des trois précédents arrêts. Cela est étonnant dans la mesure où il existe bien un service public concernant ce culte, car l'affaire se déroulait en Alsace-Moselle, où il y a un service public du culte. \\
Actes réglementaires de personnes privées rattachées à un SPIC: TC, 15 Janvier 1968, Air France c/ époux Barbier. Concernant le règlement d'un SPIC, comme il touche à l'organisation d'un SP, c'est un acte administratif alors même qu'il est édicté par une personne privée. 

\subsection{Les actes de gouvernement}

Ces actes sont édictés par des personnes publiques, des autorités administratives. Ils ne relèvent pas l'exercice d'une fonction administrative, ils relèvent d'une fonction gouvernementale. Ces actes sont insusceptible de recours. \\
Il n'existe pas aujourd'hui, de définition d'actes de gouvernement alors qu'il existait au XIXe siècle, il se définissait par son mobile politique: c'est parce qu'il était pris pour un motif politique que c'était un acte de gouvernement. Cela a été abandonné suite à CE, 18 Février 1875, Prince Napoléon. \\
Le juge considère qu'il existe aujourd'hui deux catégories d'actes de gouvernement: les actes de relations entre les pouvoirs publics constitutionnels ; et les actes relatifs aux relations diplomatiques de la France. \\
La tendance est à la réduction des actes de gouvernement, notamment la deuxième catégorie. Le JA le fait sous l'influence de l'art. 13 de la CEDH (le droit au recours effectif). La doctrine propose depuis longtemps la disparition de cette catégorie, qui décline depuis une vingtaine d'années. 


Les actes relatifs aux relations entre les pouvoirs publics constitutionnels. \\
Parmi ces actes, on trouve un certain nombre de décision concernant la procédure législative, comme la décision du Gouvernement refusant le dépôt d'un PJL (CE, 29 Novembre 1968, Tallagrand). \\
L'acte de promulgation de la Loi n'est pas détachable de la procédure législative: CE, 29 Octobre 2015, Fédération démocratique Alsacienne. \\
On peut inclure dans cette catégorie l'ensemble des pouvoirs propres du PR: nommer un membre du CC (CE, Ass., 9 Avril 1999, Mme Ba), faire un référendum (CE, 1962, Brocas), recourir à l'article 16 (CE, 2 Mars 1962, Rubin de Cervens). \\
Ces actes peuvent concerner les rapports au sein d'un même pouvoir comme le décret de composition du Gouvernement: CE, 29 Décembre 1999, Lemaire.


Les actes relatifs aux relations diplomatiques de la France. \\
C'est dans cette deuxième catégorie que le juge réduit le plus les actes de gouvernement. Si l'acte est détachable des relations internationales, il sera considéré comme un acte de gouvernement. \\
Acte ayant un lien direct avec les RI: la décision de suspendre l'application d'un traité (TC, 2 Février 1950, Radio Andorre), à l'inverse, la décision de ratifier un traité peut être contrôlée au regard de l'article 53 de la Constitution. \\
Est considéré comme un acte de gouvernement la décision de reprendre les essais nucléaires, car, selon le juge, la décision est préalable à la négociation d'un traité: CE, 29 Décembre 1995, Greenpeace. \\
CE, Sect., 28 Mars 2014, Mr de Baynast, était en cause la décision d'un groupe français qui a refusée une candidature à la cour permanente d'arbitrage. Pour le juge, cet acte n'est pas détachable de la procédure conduite devant la CPI donc est un acte de gouvernement, insusceptible de recours. \\
À l'inverse, ne sont pas des actes de gouvernement les actes qui sont détachables des RI. Deux illustrations: n'est pas un acte de Gouvernement la décision d'octroi d'un permis de construire à une ambassade étrangère (CE, 22 Décembre 1978, Vothanhnghia). La décision par laquelle la France autorise ou refuse une extradition est susceptible de recours, et n'est donc pas un acte de gouvernement: CE, Ass., 17 Octobre 1993, Royaume-Uni de Grande Bretagne et d'Irlande du Nord.


\section{Les effets de l'acte administratif unilatéral}

\subsection{Le caractère exécutoire des décisions administratifs}

Les actes de l'administration font l'objet d'une présomption de légalité. C'est ce qu'on appelle le "privilège du préalable", ou encore "autorité de chose décidée". C'est de ce privilège que va découler le caractère exécutoire de la décision. L'administré est tenu d'exécuter l'acte de l'administration, et elle peut même le contraindre à s'exécuter sans nécessairement passer par un juge. Le JA a considéré en 1982 que ce caractère exécutoire est selon le juge, la règle fondamentale du droit public: CE, Ass., 2 Juillet 1982, Mr Huglo. \\
La formule est sans doute excessive mais elle permet de marquer l'importance de cette règle. 

\subsection{L'exécution de la décision}

L'administration dispose de plusieurs moyens pour contraindre un administré à respecter ses décisions. Le premier biais qui permet l'exécution, c'est l'existence de sanctions pénales prévues en cas de non respect d'un acte réglementaire, et sont prévus dans le code pénal lui même: R610-5 CP. \\
Deuxième moyen: les sanctions administratives, car certaines administrations possèdent un pouvoir de sanctions qui peuvent concerner spécifiquement certains actes. \\
Troisième moyen: l'exécution forcée qui permet à l'administration de se faire autoriser par un juge à faire exécuter sa décision. \\
Quatrième moyen: l'exécution d'office par laquelle l'administration agit d'office, elle pourra ici faire exécuter un acte sans aucune autorisation d'un juge. Cela a été posé par TC, 1902, Société immobilière Saint-Just, qui permet l'exécution d'office dans trois cas: si la loi le prévoit, en cas d'urgence, lorsqu'aucune voie de droit permet de sanctionner le comportement d'un administré ou ne permet de lui imposer l'application de la solution litigieuse. 

\subsection{Le caractère non suspensif des recours contentieux}

Un recours contentieux n'est pas suspensif. Si un recours était suspensif, ils permettraient de bloquer l'action de l'administration. \\
Dérogation importante: référé d'urgence, prévu par le CJA, le référé suspension prévu à l'article L521-1 du CJA, qui permet de demander au juge la suspension d'une décision dans l'attente d'une demande au fond. Deux conditions au référé-suspension: urgence et un doute sérieux sur la légalité de la décision litigieuse. 

\chapter{Les régimes des actes administratifs unilatéraux}

Loi du 12 Avril 2000: loi DCRA (Droits des citoyens dans leur relation avec l'administration), vient fixer un régime commun d'édiction des actes administratifs. En 2015, l'ensemble des règles de cette procédure non contentieuse a été codifié dans le CRPA, par une ordonnance du 23 Octobre 2015, entrée en vigueur le 1er Janvier 2016. Le code régit, comme son nom l'indique, les relations entre le public et l'administration mais parfois aussi entre l'administration et ses agents. Ce code synthétise l'ensemble des règles non contentieuses. 

\section{L'élaboration de l'acte administratif}

\subsection{La procédure administrative non contentieuse}

C'est la procédure de droit commun de l'élaboration de ces actes. Des procédures particulières peuvent être prévues. 

\subsubsection{Le principe du contradictoire}

C'est un principe qui a vocation à s'appliquer en matière juridictionnelle lors des procédures contentieuses. \\
Le respect du contradictoire est un PGD dégagé le CE, 4 Mai 1944, Dame Veuve Tromper-Gravier. Ce principe est aujourd'hui codifié dans le CRPA: L120-1. Le principe du contradictoire ne s'applique pas partout, il s'applique surtout à deux séries d'actes de l'administration: les décisions qui doivent être motivées (ce sont essentiellement les décisions défavorables) ; les actes qui sont pris en considération de la personne (essentiellement les sanctions de l'administration). \\
Limites à  l'art. L121-1 et suivants du CRPA, le principe ne s'applique pas lorsque l'administration statue à la demande d'un administré. Il ne s'applique pas non plus en cas d'urgence ou de circonstances exceptionnelles ou si l'application du principe compromet l'ordre public ou si il compromet les relations internationales de la France. Il ne s'applique pas à certains organismes, comme ceux de la sécurité sociale ou à pôle emploi. Ni non plus entre l'administration et ses agents sauf en cas de sanctions. Et enfin il ne s'applique pas si une procédure contradictoire spéciale est prévue par les textes.


Concrètement, le respect de ce principe a une double portée: il implique tout d'abord de pouvoir présenter ses observations à l'administration, de manière écrite ou orale. Cela doit être accompagné de la possibilité de se faire assister par un conseil: L122-1 du CRPA. \\
Ensuite, le principe a une portée significative lors d'une sanction, la personne sanctionnée doit pouvoir connaître les griefs retenu à son encontre ainsi que de se faire communiquer le dossier: L122-2 CRPA.

\subsubsection{La transparence de l'action administrative}

Il n'existe pas de principe général de transparence. Mais de nombreuses dispositions permettent une telle transparence, et celle-ci a été renforcée par le législateur depuis quelques années. 


Traitement des demandes. \\
Le CRPA a défini ce que sont ces demandes. Il entend par demande: "Les demandes et réclamations ainsi qu'un recours administratif" L110-1 CRPA. Il existe deux catégories de recours administratifs: les recours gracieux et hiérarchique. \\
Le CRPA donne des règles sur ces échanges, ils doivent être en langue française, ils peuvent être fait via voie électronique (et même uniquement dans certains cas): L112-8 CRPA. Le code encadre ces échanges, et pour qu'ils aient lieu, l'administré doit s'identifier pour communiquer avec l'administration, et ces échanges sont impossibles dans certains cas: hypothèse du secret défense, si l'ordre public s'y oppose, si la présence du demandeur est nécessaire, pour des raisons de bonne administration. L'administration doit donner un accusé de réception et peut elle même répondre par voie électronique sauf si le demandeur s'y oppose. \\
Contenu des échanges: des formalités minimales s'imposent à l'administration (L111-2 CRPA), les correspondances doivent indiquer la qualité de l'agent qui traite le dossier, son nom et son prénom ainsi que son adresse administrative. Limite: des motifs de sécurité publique permettent parfois d'anonymiser certaines correspondances. Dans le cas de la lutte contre le terrorisme, certains services ont étés dispensés d'indiquer leurs coordonnés. Le CRPA impose aussi la signature de l'acte par son auteur: L212-1, la signature peut être électronique, certains actes en sont dispensés: L212-2 CRPA. \\
Accusé de réception: dès lors qu'une demande est formulé, l'administration doit donner un accusé de réception: L112-3 CRPA. Cela concerne aussi les demandes électroniques. Il doit contenir toutes les mentions évoqués ci dessus. Cet accusé permet de connaître la date de naissance de certaines décisions, notamment les décisions implicites de l'administration. En l'absence d'accusé de réception, le demandeur pourra contester la décision à tout moment sans qu'on puisse vraiment opposer un délai de recours: L112-6 CRPA. Accusé de réception non nécessaire en cas de demandes abusives, en cas d'hypothèse d'urgence, ou si l'administration dispose d'un délai très bref pour répondre à la demande ou dans certains cas en matière électronique: comme si cela porterai atteinte aux systèmes de l'administration. \\


Les décisions implicites. \\
Ce sont les décisions nées du silence de l'administration. En principe, c'était des décisions de rejet après un silence de deux mois, c'était la règle posée par la loi d'Avril 2000. Cette règle a été renversée: L231-1 CRPA, qui trouve son origine dans une loi de 2013. Désormais, le silence de l'administration pendant deux mois équivaut à une décision implicite d'acceptation. Cette règle est rentrée en vigueur en 2014 pour l'État et en 2015 pour les collectivités locales. \\
Il existe de très nombreuses dérogations: L231-4 et suivants du CRPA. \\
Elle ne s'applique pas aux décisions non individuelles, en cas de recours administratif, aux demandes qui ont un caractère financier, quand elle porterait atteinte soit à l'OP soit à la protection des libertés fondamentales soit au respect des conventions internationales ou à un principe à valeur constitutionnel, dans les relations entre l'administration et ses agents. Des décrets en CE peuvent rajouter des dérogations. Le Gouvernement en a adopté 40. La règle ne concerne donc que 1200 procédure sur 3600. \\
Cela a donc créé une grande critique de la part de la doctrine, la règle précédente ne posait aucunes difficultés. Le principe nouveau n'est en fait qu'une exception. \\
Cela complique aussi les choses pour l'administration, qui doit répondre à tous les administrés, ce qui peut poser certaines difficultés matérielles. \\
Cette règle est entourée de garanties procédurales. L232-3 CRPA, il doit y avoir une attestation de la décision. Comme pour toute demande, l'administration a obligation d'accusé de réception, et à l'occasion de cela, l'administration doit trouver quelle règle s'applique: si le silence vaut acceptation ou si il vaut rejet. \\
Dans le cas où la demande a été adressée à une autorité incompétente: elle est obligée de transmettre à l'autorité incompétente. L'administré sera prévenu de cette transmission: L114-2 CRPA. \\
Le point de départ du délai de naissance de la décision implicite d'acceptation est la date de la réception de la demande: L114-3 du CRPA.

\subsubsection{La communication des documents administratifs}

Dans certains cas, c'est une obligation pour l'administration. Depuis 1978, la loi prévoit un droit d'accès aux documents administratifs. \\
Outre ce droit d'accès, il existe une liberté de réutilisation d'informations publiques, qui a été complété en 2016 par l'obligation de communication de ces données par voie numérique. C'est l'Open Data. Cela ne concerne que les documents de certains organismes. Les établissements publics ne sont pas concernés. Le tout est codifié au L300-2 CRPA. \\
L'administration a obligation de mettre à disposition ces documents. Cette obligation ne concerne pas la défense, le secret (de la vie privée, médical, secret défense). \\
Pour obtenir communication de ces documents, il faut une demande de l'administré, que l'administration peut refuser. Ce refus ne peut pas être directement contesté devant le juge, l'administré doit saisir une AAI, la CADA (Commission d'accès aux documents administratifs). Si la CADA refuse, l'administré pourra contester la décision devant le juge. \\
Certains documents de la présidence devaient tomber sous le joug de la loi de 1978. S'est posé la question si cela ne heurtait pas l'article 67 de la Constitution, car certains documents montraient des commandes de sondages. Le juge a conclu que ces documents devaient être communiqués: TA Paris, Avrillier, 2010. Le juge pose quelques limites: il ne peut pas concerner les documents du Chef de l'État en tant que personne privée, les autres documents sans rapport avec les missions dévolus à l'État dans l'exercice de missions de service public. 


Liberté de réutilisation des informations publiques. \\
Cela a été fait sous l'influence de la directive ISP du 17 Novembre 2003, transposé par l'ordonnance de Juin 2005. Cela est désormais codifié dans le CRPA, depuis fin 2016. \\
Cette liberté, c'est la liberté pour toutes personnes de réutiliser les informations publiques à toutes fins, différentes desquelles elles ont étés récoltées. Ce sont des données publiques. \\
Deux limites: les données publiques produites dans le cadre d'un SPIC ne sont pas réutilisables, car il les utilises lui même à des fins commerciales. La deuxième limite est la propriété intellectuelle, certaines données en font l'objet et ne sont pas réutilisable.


Automaticité de la communication des données publics. \\
C'est ce qu'on appelle l'Open Data. C'est l'idée de mettre librement les données publiques en ligne, sans que les administrés n'aient à en faire la demande. \\
En 2016, loi pour une République numérique, le législateur met en place un service public de la donnée, qui vise à mettre à disposition des données à disposition en ligne et gratuitement. Cela implique nécessairement une obligation pour les personnes publiques: communiquer leurs données publiques: L312-1-1 CRPA. Cela concerne autant l'État que les collectivités. \\
Cela s'est doublé d'un principe de gratuité des données publiques, qui a été affirmée par une loi de 2015 et ne prévoit que quelques dérogations, notamment les données produites par les SPIC. \\
Deux visions de l'Open Data: considérer que les données publiques sont un bien commun, servant à tous, produite par la personne publique dans l'intérêt général. L'idée de gratuité vient du fait que la production a déjà été financée. \\
Une autre conception consiste à dire que l'Open Data est souvent utilisé à des fins commerciales, et donc certains considèrent qu'il ne serait pas illégitime de faire payer ceux qui réutilisent à des fins commerciales. 

\subsubsection{La négociation}

L'acte administratif est un acte de puissance publique. La puissance publique est que l'administration puisse imposer sa volonté aux administrés. L'acte est donc par principe, élaboré par l'administration seule. Or, cela évolue depuis une vingtaine d'années, qui tend à associer le public à l'élaboration de l'acte administratif. \\
L'administration va recueillir l'avis des administrés, c'est parfois une obligation procédurale. L'administration pourra ou non prendre en compte ces avis. C'est l'idée de faire participer les administrés, qui a l'air d'aller à l'encontre de la notion de puissance publique, cependant, cela présente des avantages. Dans certains domaines techniques, l'administration peut recueillir les avis de certains opérateurs spécialisés. Un deuxième avantage serait que par cette idée de consultation, l'administration va indirectement recueillir le consentement des administrés, donnant une légitimité à la nouvelle norme. Un troisième avantage est que cela conserve la sécurité juridique, car la consultation les prévient de normes juridiques à venir. \\
On distingue classiquement deux techniques. On a d'abord la consultation et d'autre part, la concertation. La consultation, c'est l'idée d'uniquement recueillir les avis. La concertation implique une certaine idée de négociation, ou du moins, un échange, un dialogue. \\
Certains auteurs ont parlés de contractualisation de l'action administrative, qui prendrait une forme contractuelle. En cas de concertation, cela aboutit à une certaine négociation qui ressemble au fait de négocier un contrat. Les actes administratifs négociés ne sont pas des contrats: ils restent des actes unilatéraux. \\
Ces procédures de négociation concerne aujourd'hui deux domaines: l'environnement et l'urbanisme. Dans le domaine de l'environnement, la participation du public est constitutionnelle, dans la charte de l'environnement. \\
Ces procédures ont étés médiatisés dans une affaire, évoqué notamment par le CE, 20 Juin 2016, NDDL. 

\subsection{La motivation des actes administratifs}

C'est l'idée selon laquelle les administrateurs doivent pouvoir rendre compte de leur action et rendre compte de pourquoi ils ont pris telle ou telle décision. C'est une obligation fondamentale qui n'a pas existé pendant longtemps. Il a fallu attendre la loi du 11 Juillet 1979, elle est codifiée au L211-1 du CRPA. 

\subsubsection{Le champ d'application de la motivation}

Champ d'application organique. \\
Qui sont les auteurs soumis à cette obligation ? L211-1, l'obligation de motivation s'impose à toutes les personnes publiques, aux personnes privées chargées de service public, peu importe sa nature selon le CRPA. L'obligation de motivation concerne aussi aujourd'hui les relations entre les administrations. 


Champ d'application matériel. \\
Quelles sont les décisions concernées ? Deux d'après le CRPA, d'abord, la catégorie évoqué à l'article L211-3 du CRPA qui soumet à l'obligation de motivation les décisions individuelles qui dérogent aux règles générales fixées par les lois et règlements. L211-2 CRPA, sont soumis à obligation de motivation les décisions individuelles défavorables: les mesures restrictives de libertés (mesure de police administrative), les sanctions de l'administration, les autorisations conditionnées, les décisions qui mettent fin à des situations créatrices de droits, et enfin les décisions qui refusent un avantage qui est un droit pour l'administré. \\
Cette liste peut être complétée par des décrets en Conseil d'État. 

\subsubsection{La portée de la motivation}

"Indiquer les considérations de droit ou de fait qui constitue le fondement de la décision", L211-5 du CRPA. Cela vient préciser deux éléments, les indications doivent être effectuées par écrit dans le corps de la décision concernée. La motivation doit être effectuée sans délai. Le juge censure les motivations standards, qui ne sont pas convenable. Les décisions non motivée ou mal motivée sont des décisions illégales: CE, 24 Juillet 1981, Belasri. \\
L'obligation disparaît dans trois cas prévus: en cas d'urgence absolue (l'administré peut faire la demande des motifs dans le délai du contentieux), dans les décisions implicites (idem qu'avant sur la demande), enfin, les faits couverts par le secret (médical, défense). Le secret médical ne dispense pas de motiver, les faits couverts par le secret doivent être cachés. 

\section{L'entrée en vigueur de l'acte administratif}

L'acte administratif ne devient obligatoire qu'à l'entrée en vigueur de celui-ci. Cette entrée fait l'objet de règles de plus en plus techniques. 

\subsection{Les conditions de rentrée en vigueur}

Ces conditions sont variables suivant la nature de l'acte, si il est individuel ou réglementaire. 

\subsubsection{Les décisions réglementaires}

L221-2 du CRPA: cette disposition s'applique aux actes réglementaires mais aussi aux décisions d'espèces. \\
L'entrée en vigueur de ces actes est subordonné à la publication ou à l'affichage de ces actes. L'administration a l'obligation de publier les actes réglementaires dans un délai raisonnable, c'est un principe général du DA consacré par CE, 12 Décembre 2003, Syndicat des commissaires et des hauts fonctionnaires de police. \\
En principe, l'entrée en vigueur a lieu le lendemain de l'affichage. Il arrive parfois qu'un acte nécessite des mesures d'applications, et quand elles sont nécessaires, l'entrée en vigueur de l'acte est déplacé au jour de l'édiction de ces mesures. Il arrive assez souvent qu'un texte législatif ou réglementaire prévoit une règle différente, c'est le cas notamment de certains règlements locaux, qui doivent être transmis au préfet avant de pouvoir rentrer en vigueur: L222-1 du CRPA. \\
Concernant les actes réglementaires nationaux, L221-3 du CRPA, quand l'acte est publié au JO, il rentre en vigueur le lendemain ou à la date qu'il fixe. Il y a l'hypothèse de l'urgence, où l'acte peut rentrer en vigueur le jour même. 


\subsubsection{Les décisions non réglementaires}

L221-8 CRPA, elles rentrent en vigueur dès la notification, qu'elle soit favorable ou défavorable. \\
Le CRPA ne parle plus d'entrée en vigueur mais d'opposabilité, ce qui est la même chose. Des textes peuvent prévoir des règles particulières.

\subsection{Sécurité juridique}

C'est un principe qui n'a pas toujours existé en droit interne. Le principe est issu du droit allemand, qui a été transposé en droit de l'Union comme un principe général. Sa portée est assez mal défini, le principe est en constante évolution et concerne de plus en plus des pans entiers de l'administration. \\
Le JA a, pendant longtemps, refusé de consacrer un tel principe, partant de l'idée que la mutabilité du SP ne permet pas de garantir un tel principe. Le JA n'a pas toujours été insensible à la sécurité juridique, il a donc finalement consacré son propre principe en 2006: CE, Ass., KPFG. \\
En droit positif, ce principe semble imposer l'édiction de mesures transitoires. La sécurité juridique a une portée protéiforme, il a plusieurs conséquences. 

\subsubsection{La non rétro-activité des actes administratifs}

CE, 26 Décembre 1925, Rodieres, une décision administrative ne peut statuer que pour l'avenir. Il en a fait un PGD: CE, Sect., 1948, Société du Journal l'Aurore. \\
La date d'entrée en vigueur ne peut être antérieur à la date d'adoption de l'acte.\\
Ce principe a ses racines dans l'article 2 du C.Civ. \\
L221-4, le CRPA a codifié lui même le principe, avec des dérogations: si l'acte est lui même pris en application d'une loi rétroactive ; un acte pris en application d'une décision d'annulation du JA. 

\subsubsection{L'obligation d'édicter des mesures transitoires}

En principe, l'acte réglementaire est d'application immédiate. C'est l'intérêt général qui commande une telle application. Cela pose une difficulté concrète: un acte peut modifier le droit positif de manière brutale. \\
En droit de l'UE, le juge a dégagé le principe de la protection de la confiance légitime, qui interdit de changer brutalement le droit positif. Le JA a toujours refusé de consacrer un tel principe explicitement: CE, 2001, Entreprise transport Friybuth. \\
Le juge évite de se sentir lié par un principe étranger, il a donc consacré son propre principe dans l'affaire KPMG. \\
L'application de dispositions nouvelles doit être accompagné de mesures transitoires pour permettre de s'adapter. Il faut édicter des mesures transitoires lorsque les nouvelles dispositions portent une atteinte excessive aux situations contractuelles en cours et lorsque l'application immédiate porte une atteinte excessive aux intérêts en présence. \\
Quand l'intérêt général le commande, ces principes peuvent être contournés. \\
L221-5 et L221-6 du CRPA, le code reprend presque mot pour mot l'arrêt KPMG. Le code vient préciser ce que peuvent être ces mesures transitoires: une date d'entrée en vigueur différé ; en des règles particulières qui vont s'appliquer temporairement ; la précision des conditions d'application de la nouvelle réglementation, qui pourraient varier avec le temps. 

\subsubsection{La question des revirements de jurisprudence}








\section{La sortie de vigueur de l'acte administratif}










\end{document}
