\documentclass[10pt, a4paper, openany]{book}

\usepackage[utf8x]{inputenc}
\usepackage[T1]{fontenc}
\usepackage[francais]{babel}
\usepackage{bookman}
\usepackage{fullpage}
\setlength{\parskip}{5px}
\date{}
\title{Cours de Droit Administratif (UFR Amiens)}
\pagestyle{plain}

\begin{document}
\maketitle
\tableofcontents


\part{Les actes de l'administration}

\chapter{L'acte administratif unilatéral}

Cette question de l'acte unilatéral a évolué sous l'influence du législateur. Le CRPA est entré en vigueur en 2016, un Code des Relations entre le Public et l'Administration. Celui-ci vient codifier toute la jurisprudence administrative, qui n'était pas très claire alors.

\section{Définitions}

La première innovation de ce code est de donner en droit positif une définition de l'acte unilatéral de l'administration. L200-1 CRPA, "Les actes administratifs unilatéraux contiennent les actes décisoires ainsi que les actes non décisoires". Ce n'est donc pas vraiment une définition mais une énumération. \\
Un acte décisoire est un acte qui vient sanctionner l'utilisation d'une prérogative de puissance publique, puisque via ces actes, la puissance publique peut imposer sa volonté, agir contre la volonté des intéressés. Jusqu'en 2016, tous les actes décisoires étaient des actes unilatéraux, aujourd'hui, ce sont aussi des actes non décisoires, qui n'ont aucun effet juridique, ne produit aucun droit et aucune obligation pour les administrés. \\
Dans le CRPA, il est rappelé qu'il existe trois catégories d'actes décisoires, précisés à l'article L200-1, alinéa 2, on y compte les actes réglementaires (celui qui a pour destinataire des personnes désignés abstraitement). Ensuite, on y compte les actes individuels, qui, à l'inverse, a pour destinataire des personnes nominativement désignées ; ces actes peuvent être des actes individuels collectifs comme des décrets de nomination qui visent plusieurs personnes. La troisième catégorie sont les autres actes décisoires non réglementaires ni individuel, qu'on nomme en fait des décisions d'espèce qui se rapporte à une situation donné comme la dissolution d'un conseil municipal, une déclaration d'utilité publique, etc. \\
Une autre distinction à faire est celle des décisions qui viennent créer des droits pour les administrés. Ces décisions ont un régime plus protecteur pour les administrés. \\
Enfin, on distingue les décisions explicites de l'administration, formalisé dans un acte des décisions implicites, qui naissent du silence de l'administration suite à une demande d'un administré. Au bout d'un certain délai suite à la demande, une réponse est supposée ; alors que la réponse supposée était un rejet, la réponse supposée est aujourd'hui l'acceptation de la demande. 

\subsection{Les catégories d'actes administratifs unilatéraux}

Pendant longtemps, il y avait un lien direct entre le caractère décisoire d'un acte et la possibilité de faire un recours contre cet acte. On considérait en effet que seul les actes décisoires pouvaient faire grief au requérant. Cette solution était certaine: CE, 1950, Dame Lamotte, qui fait du REP un principe général du DA, et le juge indique que le REP est possible contre "toutes décisions administratives". \\
Aujourd'hui, il faut dissocier le fait qu'un acte soit décisoire pour qu'il soit attaqué. On constate depuis quelques années que certains actes, pourtant décisoires, ne sont jamais susceptible de recours contentieux, alors qu'à l'inverse, certains actes non décisoires, peuvent faire l'objet depuis quelques années de recours juridictionnel. \\
Soft Law: le droit souple n'était pas susceptible de recours, aujourd'hui, c'est possible. \\
Le JA n'a jamais tenté de définir ce qu'était l'acte administratif unilatéral. 

\subsubsection{Les actes décisoires de l'administration}

En principe, l'acte décisoire fait grief et est susceptible de recours. Pourtant, certains actes décisoires ne font jamais grief et ne sont jamais susceptibles de recours contentieux. Ce sont les mesures d'ordre intérieur (MOI). 


Les actes décisoires sont susceptible de recours contentieux. On doit donc se demander quel est le critère de l'acte décisoire pour le JA. Il l'a dégagé à l'examen de certains actes de l'administration comme des circulaires.


Concernant les circulaires, elles s'inscrivent dans le pouvoir d'instruction appartenant au chef de service qui dispose d'un pouvoir réglementaire pour gérer leur service. Les chefs de services disposent aussi d'un pouvoir d'instruction, leur permettant de donner des ordres aux subordonnés. Ce pouvoir d'instruction appartient au chef de service mais aussi au PM qui peut donner des instruction aux membres du gouvernement, leur demandant d'agir dans un sens donné (CE, Sect., 26 Décembre 2012, Association Libérez les Mlle). \\
Ce pouvoir d'instruction permet au chef de service d'édicter des instructions ou des circulaires. Ces dernières sont souvent longues, techniques, denses. C'est la communication d'un supérieur hiérarchique à l'attention d'un subordonné, où le chef de service va à la fois donner des ordres, rappeler le droit positif, la JP, viennent interpréter le droit positif. Par principe, ces circulaires sont donc interne aux services, et c'est pour cette raison que les administrés ne pouvaient pas contester ces circulaires. Cela posait deux séries de difficultés: une circulaire peut s'avérer illégale, car une instruction ou une interprétation pouvait être contraire à une norme supérieure ; la deuxième concerne le contenu même car pouvait concerner directement les administrés, car les circulaires pouvaient contenir de nouvelles règles. C'est pour ces deux raisons que le JA a accepté de contrôler certaines circulaires.


Trois temps de la JP: CE, Ass., 29 Janvier 1954, Notre Dame du Kreisker, dans laquelle le juge accepte la recevabilité des recours dirigés contre les circulaires réglementaires de l'administration. Le JA distingue deux circulaires: la réglementaire et l'interprétative, la première ajoute quelque chose au droit positif, modifie l'ordonnancement juridique, posait une nouvelle règle de droit, etc. Le JA a considéré cette première comme décisoire. Concernant les circulaires interprétatives, le JA refusait de les contrôler. Deux difficultés de cette JP: la distinction entre les circulaires réglementaires et les interprétatives: la JA nous dit que la réglementaire ajoute quelque chose au droit positif alors que l'interprétative, non. Cela est curieux car l'interprétation est un acte de volonté de celui qui interprète, où une méthode est choisie, il y a donc le choix d'une interprétation parmi d'autres. Une circulaire interprétative ajoute donc aussi quelque chose au droit positif, indirectement. \\
La seconde difficulté concernait les circulaires réglementaires. Cependant, en France, le pouvoir réglementaire est cantonné à certaines autorités: le PM, les autorités dont la loi donne un pouvoir réglementaire. Des autorités incompétentes ont tout de même édictés ce genre de circulaires, et le juge, quand il jugeait au fond, annulait ce genre de circulaires. Avec les circulaires interprétatives, que le juge refuse de contrôler, celle là pouvait être illégale.


Deuxième temps: le JA va accepter de contrôler certaines circulaires interprétatives, notamment celles qui sont illégales. CE, 15 Mai 1987, Ordre des avocats à la Cour de Paris ; CE, Sect., 1993, IFOP. Dans ces deux cas, le JA accepte de contrôler des circulaires interprétatives illégales. \\
Difficulté: suivant cette jurisprudence, le juge confond la recevabilité du recours et le contrôle au fond de l'acte. 


Troisième temps: CE, Sect., 18 Décembre 2002, Mme Duvigneres, cet arrêt étend les possibilités de recours contre les circulaires. Le JA nous indiques que les circulaires impératives à caractère général sont désormais susceptible de recours. \\
Une circulaire impérative est une circulaire qui tend à imposer quelque chose à ses destinataires, comme les circulaires réglementaires qui sont toujours impératives. \\
En reconnaissant ce genre de recours, le JA admet implicitement que toute autorité dispose d'un pouvoir d'édicter de telles circulaires. \\
Le JA n'est pas du tout regardant quant à la forme de l'acte qu'il examine. Le juge évoque notamment les circulaires ou les instructions. En pratique, ces actes peuvent trouver des dénominations très variées, comme des notes de services, des recommandations, ce qui importe, c'est le caractère impératif de l'acte. \\
Pour savoir si l'acte est impératif ou non, le juge va s'attacher à la rédaction de l'acte. Si la circulaire se contente de préconiser, de recommander un comportement, la circulaire n'est pas impérative. À l'inverse, si la circulaire indique des conditions, alors l'acte est considéré comme impératif et susceptible d'un recours contentieux. \\
Il existe des actes qui en apparence ne sont pas impérative, mais qui le sont en fait. Certains actes intitulés "recommandation", comme celle du CSA par exemple, sont impératives malgré leur apparence. CE, Ass, 2009, Mr Hollande, était en cause des règles de propagandes électorales, les recommandations donnés par le CSA étaient bien impératives et susceptible de recours. \\
Dans le cas d'un communiqué de presse d'une autorité, il peut être considéré comme un acte et susceptible d'un recours. CE, 16 Mars 2009, EMM, le CE admet un recours dirigé contre un communiqué de presse qui contenait une interprétation du droit positif. \\
On peut prendre en troisième exemple un tweet du ministère de l'intérieur, qui imposait aux supporters de "bien se comporter aux abords des stades" et disait aussi qu'ils ne devaient pas tenir de propos politiques, idéologique, raciste, ou xénophobe. Ce tweet est impératif car impose un comportement aux supporters, et il est anticonstitutionnel car il est en contradiction avec la liberté d'expression. Si ce tweet avait été contesté en justice, le recours aurait été certainement accepté.


Depuis un décret du 8 Décembre 2008, il existe des règles de publication des circulaires, celles-ci, quand elles sont ministérielles, doivent être publiées sur le site internet du PM. Si cela n'a pas été fait, la circulaire n'est pas opposable à l'administré. Le décret dispose que si les circulaires antérieures n'ont pas été publiées, elles sont considérés comme implicitement abrogées. CE, 23 Février 2011, Association La Cimade. \\
Le CE a étendu le champ d'application de ce texte, CE, 16 Avril 2010, Mr A, le CE considère que les obligations de publications concernent aussi les instructions non écrites. Cette JP supprime la catégorie des instructions non écrites puisqu'elles doivent toutes être écrites. \\
Un décret du 6 Septembre 2012 admet que les circulaires puissent être publiées sur un autre site internet que celui du PM, notamment les sites des autres ministères. \\
Depuis 2016, dans le CRPA, il est codifié l'obligation de publication: L312-2, qui dispose que doivent être publiées les "instructions, circulaires, notes et réponses ministérielles, seulement celle qui contiennent une interprétation du droit positif ou une description des procédures administratives". On constate deux lacunes: le champ d'obligation de publication qui ne concerne que les circulaires interprétatives, les circulaires réglementaires ayant étés totalement étés écartés. En 2016, dans le CRPA, il n'y a plus de sanctions au manque de publications, on peut donc penser que le JA va tenter de ré-appliquer sa JP.


Les actes décisoires qui ne font jamais, ou presque jamais grief, ce sont les MOI. Ces mesures sont internes aux services, ce sont souvent des sanctions prononcés par un chef de service à l'encontre d'un agent, d'un usager. \\
Ces MOI sont décisoires. Le JA refuse de contrôler ces actes au contentieux, il considère qu'une MOI ne fait jamais grief au requérant. Cela s'explique pour deux raisons, déjà par l'adage que le juge ne s'occupe pas des affaires mineures par crainte d'être encombré. La seconde raison est que le contrôle de ces mesures auraient une incidence sur la discipline du service. \\
Ce refus du JA a posé des difficultés. La première tient à ce que ces MOI ne sont pas toujours mineur pour l'administré comme le renvoi d'un élève d'un collège, la mise en isolement d'un détenu dans une prison, etc. La seconde difficulté est celle de l'influence de la CEDH (art. 13, droit au recours effectif devant un juge), la CEDH a considérée pendant les années 90 que l'absence de recours pouvait être vu comme une atteinte à ce droit (Ramirez-Sanchez c/ France). \\
Le CE a progressivement réduit la catégorie des MOI. Le CE considère aujourd'hui que certaines mesures qui étaient alors considérés comme des MOI ne le sont plus et peuvent donc faire l'objet d'un recours contentieux.


On peut noter quatre temps d'évolutions de ces MOI. \\
D'abord, pendant les années 90 où le JA va accepter de contrôler des mesures qui étaient auparavant des MOI. CE, Kherouaa, 1992, le JA accepte de contrôler une mesure d'exclusion d'une élève, mesure qui n'est donc pas considéré comme un MOI. \\
CE, Ass, 17 Février 1995, arrêt Hardouin et deuxième arrêt Marie (du même jour), le JA accepte de contrôler des sanctions infligées à Mr Hardouin, un militaire et des sanctions infligés à Mr Marie, un détenu. \\
La nature de la mesure mais aussi la gravité, et les effets de la mesure sur l'administré peuvent faire qu'une MOI n'en est pas une. Le contrôle se fait donc nécessairement au cas par cas. \\
En 2007, trois décisions rendues le même jour, le 14 décembre: CE, Ass, Boussouar, concernant les mesures de transfert d'un détenu, posant un caractère subsidiaire d'identification des MOI. Même si les critère de nature et de gravité ne sont pas remplis, la mesure n'est pas une MOI selon le critère qu'il pose: l'atteinte aux droits et libertés de la personne concernée. Le juge distingue le transfert d'un établissement pour peine à une maison d'arrêt, qui n'est pas une MOI, eu égard à la nature et la gravité de la mesure. Lorsque c'est l'inverse, puisqu'il n'aggrave pas la situation du détenu, cela semble être une MOI eu égard à la nature et à la gravité, sauf si le critère subsidiaire est rempli (le transfert peut porter atteinte au droit à une vie familiale normale, à l'accès aux soins, etc). \\
Cette évolution s'est poursuivie au sein de la prison car le JA a accepté progressivement de contrôler les refus d'emploi, les rétentions de correspondance, etc. \\
CE, 21 Mai 2014, Garde des Sceaux, qui concernait un avertissement infligé à un détenu, avertissement qui est la sanction la plus légère et le CE a dit que celui-ci n'était pas une MOI. Il a émergé aujourd'hui un contentieux pénitentiaire, avec des spécialistes de ce droit. \\
Dans un quatrième temps, CE, Sect., 25 Septembre 2015, Mme Bourjolly, qui concernant une mesure de réaffectation imposée à un agent. Deux apports à retenir: le JA donne une position de principe, pour lui une mesure de ré-affectation est une MOI, par principe. Cependant, si les critères sont remplis, une action peut être menée contre cet acte. Si la ré-affectation est une mesure de sanction, alors une procédure doit être respectée, et la mesure sera généralement annulé, l'administration ne respectant pas cette procédure. Cette décision évoque des MOI discriminatoire et le juge semble nous dire qu'une MOI discriminatoire est susceptible de recours contentieux. Il affirme donc qu'une MOI peut être susceptible de recours. \\
En plus du développement du contentieux pénitentiaire, on peut noter le développement du contentieux disciplinaire (collège/université, etc.). Cela a tendance à encombré le JA. 

\subsubsection{Les actes non décisoires}

En principe, un acte non décisoire ne fait pas grief et n'est donc pas susceptible d'un recours au contentieux. Trois catégories d'actes sont considérés traditionnellement comme non décisoires: les actes préparatoires, CE, Ass., 15 Avril 1996, Syndicat CGT des hospitaliers de Bédarieux, le JA pose une limite, le préfet admet que si l'acte n'est pas décisoire, le préfet est le seul à pouvoir former un recours ; les actes confirmatifs, des actes dont le contenu est identique à un acte précédent ; les déclarations d'intention qui ne révèlent l'existence d'aucune décision, CE, 5 Octobre 2015, Comité d'entreprise du siège de l'IFREMER, était en cause un discours du PM et non une décision. \\
Il existe aujourd'hui deux catégories d'actes non décisoires mais qui sont pourtant susceptible d'être l'objet de recours au contentieux. On a d'abord les lignes directrices de l'administration et le droit souple de l'administration. 


Concernant le droit souple, il a été pendant longtemps une notion doctrinale, assez confuse. On désignait parfois les normes informelles de l'administration, les normes concertés de l'administration (qui sont édictés après consultation ou concertation avec les administrés), ou encore ce qu'on entendait être le droit mou ou le droit doux, celui qui n'en est pas un, c'est à dire une norme non contraignante. Cette troisième conception a été prise par le CE. \\
On peut définir très simplement le droit souple comme une norme n'étant pas juridique, une norme de comportement qui ne produit pas d'effets juridique. Le droit souple, ce sont toutes ces normes qui ne sont pas contraignantes, qui n'impose aucun comportement mais qui en inspire. Par exemple, une recommandation du ministre de l'intérieur d'allumer des feux de route en journée n'est pas contraignante, elle tend seulement à influencer le comportement des administrés. \\
À priori, le droit souple n'est pas du droit. C'est une norme, certes, mais elle n'a pas d'effets juridiques. Pourtant, cette norme est très souvent respecté, produit des effets pratiques certains. Une théorie sociologique dit que ces actes ne sont pas respectés en fonction de leur nature mais en fonction de l'autorité qui les a édités. C'est parce que telle autorité l'a édicté que tel administré va la respecter. \\
Le droit souple sont des normes apparus dans l'ordre juridique internationale. Dans le droit international, le droit souple est très utile pour inviter les États à adopter tel ou tel comportement, ainsi qu'en témoigne les recommandations de l'ONU. On peut aussi penser aux recommandations de la commission européenne. \\
Ce droit souple a toujours existé dans l'ordre interne, on peut par exemple assimilé les plans de développement avant les années 80 comme du droit souple. Cependant, il a évolué quantitativement dans le domaine de la régulation économique notamment, géré par les AAI, AAI qui font un usage important de ce genre de normes. En effet, pour les AAI, cela leur permet de faire évoluer rapidement les normes applicables, très utile dans des normes très techniques, le tout sans aucun formalisme. Parfois, il permet de se substituer au droit dur dans le sens où les AAI n'ont pas la compétence d'édicter des normes de droit dur. Un dernier avantage, qui est aussi un inconvénient, c'est que n'étant pas impératif, il n'est pas contrôlé.  C'est pour cette raison que le JA a évolué et a accepté, en 2016, de contrôler le droit souple.


CE, Ass., 21 Mars 2016, deux arrêts du même jour: Société Fairvesta | Numéricable. Dans les deux étaient en cause une prise de position de l'autorité de la concurrence et un communiqué de l'AMF. Le JA va ouvrir un nouveau REP contre ces actes de droit souple. \\


Régime contentieux du droit souple. \\
Condition de recevabilité: le juge va distinguer deux séries d'actes de droit souple, dégagé par CE, 11 octobre 2012, Casino Guichard Perrachon, dans laquelle le juge parle des "actes qui énoncent des prescriptions individuelles dont l'autorité pourrait censurer ultérieurement la méconnaissance". Le critère de recevabilité du recours est donc le critère de sanction. \\
Deuxième catégorie d'acte de droit souple: ceux qui produisent des effets concrets, qui ne sont pas des effets juridiques. Dans les deux décisions de 2016, il était question d'une prise de position d'une autorité, dans l'autre affaire, il était en cause un communique de l'AMF qui mettait en garde les investisseurs contre certains investissements. Si recours contentieux il peut y avoir, c'est en raison de ce qu'implique ces actes. Le juge considère que le recours est recevable quand l'acte "produit des effets notables, notamment de nature économique", il prend en compte le fait que l'acte puisse influencer les destinataires de l'acte. \\
Première remarque: le juge est indifférent quant à la dénomination de l'acte, le juge prend de la même manière les avis, les communiqués, les recommandations, les prises de position, etc. Le juge n'est pas regardant quant à la forme de l'acte, ce qui compte, ce sont les effets de l'acte. \\
Deuxième remarque: en prenant en compte les effets de l'acte, il analyse ses conséquences, qui détermine si le recours est recevable ou non. Les effets concrets sont souvent des effets économiques, mais pas que. Dans les deux arrêts ci-dessus de 2016, les effets économiques sont évidents. Le juge prend aussi en compte que l'acte influe en pratique sur les comportements. Les critères sont donc alternatifs mais peuvent se cumuler. \\
Troisième: cette méthode, d'analyser les effets, n'est pas révolutionnaire, c'est ce que fait le juge aussi lorsqu'il contrôle les MOI, où il contrôle les effets des mesures. CE, 30 Janvier 2015, Région PACA, était en cause une prise de position de l'ARAFER, le juge a considéré que cet acte de droit souple pouvait faire l'objet d'un contentieux (cet arrêt annonce ceux de 2016). \\
Quatrième: ce critère des effets sera difficile à manier pour plusieurs raisons. D'abord à cause des termes employés par le juge qui parle de "l'influence significative des actes de droit souple" ou encore les "effets notables", où on peut se demander ce que cela veut dire quand on sait que les actes de droit souple sont globalement respectés, et que donc le champ de recours peut être large. Dans un REP, le juge apprécie l'acte au jour de son édiction, il n'est pas censé prendre en compte les changements postérieurs à l'acte. Le droit souple fait partie du non droit, en principe ; or, depuis 2016, un recours est possible, donc, depuis 2016, ces actes produisent de nouveaux effets juridiques puisqu'on l'associe à un régime juridique, ces actes ne sont donc plus du non droit ; on pourrai donc parler de "semi-droit". \\
Dernière remarque: la solution rendue en 2016 est restée cantonnée aux actes de droit souple rendu par les autorités de régulation. À noter qu'il n'existe pas de définition de la régulation en droit positif, ce qui pose une difficulté. En pratique, cela renvoie à des AAI, mais ce ne sont pas les seules autorités de régulation. Rien ne justifie de cantonner cette décision à la régulation, même si en pratique, ces autorités édictent beaucoup d'actes de droit souple. 


La construction d'un contentieux du droit souple. \\
Concernant les autres conditions de recevabilité. À l'instar de tout acte susceptible de recours, d'autres conditions classiques: l'intérêt à agir du requérant et le délai d'action. Tout administré n'a pas forcément intérêt à agir, qui doit être direct et certain. Les destinataires de l'acte de droit souple ont un intérêt direct et certain, ces destinataires peuvent être des opérateurs ou des consommateurs. D'autres requérants sont possible: les concurrents, les co-contractants des opérateurs concernés. Le juge apprécie très souplement l'intérêt à agir: CE, Novembre 2016, concernant un communiqué du CSA qui portait sur une publicité particulière, concernant les enfants trisomiques, la publicité a été jugée comme une entrave à l'avortement par le CSA, et la personne physique, trisomique qui avait tourné dans le publicité, son intérêt à agir est difficile à cerner, mais a été admis. \\
Le délai de recours: CE, Sect., 13 Juillet 2016, Société GDF-Suez, le juge considère que le délai de recours est un délai classique de deux mois, qui court à compter de la publication de l'acte sur le site internet de l'autorité. Ce délai s'applique sauf réglementation contraire comme un texte législatif pourrait le prévoir. Si un requérant laisse passer le délai, il peut quand même contester indirectement l'acte de droit souple en demandant dans un premier temps l'abrogation de l'acte de droit souple à l'autorité, et contester ensuite le refus de l'autorité, qui permettra au juge de contrôler l'acte. Le requérant pourra aussi contester le refus d'édicter un acte de droit souple (CE, Sect., 2007, Tiniez). \\
L'office du juge du droit souple: on se place dans l'hypothèse où le recours est recevable. On se pose donc la question de l'étendue du contrôle du juge de droit souple. Contrôle adapté est un terme qui ressort des deux arrêts de 2016, "en prenant compte de la nature et des caractéristiques de l'acte contesté, en prenant en compte le pouvoir d'appréciation dont dispose l'autorité". On a donc l'impression qu'il y a un recours particulier alors que non, c'est bien un REP. Ce REP semble très classique, le juge va contrôler d'une part la légalité externe et la légalité interne, comme il fait pour tous les actes administratifs. Externe: compétence de celui qui édicté, si il y a un vice de procédure ou un vice de forme. Interne: l'erreur de fait, l'erreur de droit, l'erreur dans la qualification juridique. Deux spécificités de ce recours: concernant le contrôle de la qualification juridique des faits, cela est variable suivant les domaines, même si le contrôle normal est annoncé, le juge opère en réalité un contrôle assez superficiel (ce qui est difficile de faire bien dans un acte qui ne relève normalement pas du domaine du droit) ; le recours en annulation peut être cumulé avec un recours en responsabilité de la puissance publique (un acte illégal constitue nécessairement une faute de la puissance publique), les arrêts de 2016 confirment cela. 


Les lignes directrices de l'administration. \\
Identification des lignes directrices. Elles avaient un autre nom jusqu'en 2014: directives de l'administration, un terme qui était assez ambigu avec les directives de l'UE. Ce nom de ligne directrice, on le trouve dès CE, 19 Septembre 2014, Jousselin. \\
Ces lignes directrices émanent d'autorités disposant d'un pouvoir discrétionnaire. Cette autorité va précisé les critères ou conditions sur lesquelles elle va exercer son ou ses pouvoirs discrétionnaires. Ces lignes directrices permettent à une autorité de s'auto-encadré. Au demeurant, ces lignes ne sont pas impératives dès lors que les conditions posées ne doivent pas être strictement respectés. En d'autres termes, l'autorité qui édicte ces conditions doit toujours pouvoir s'en écarter pour apprécier les demandes individuelles. \\
CE, Sect., 4 Février 2015, Mme Cortes-Ortiz, le JA poste quatre éléments de définition des lignes directrices. Le juge rappelle le pouvoir discrétionnaire de l'autorité qui les édicte, quand un texte prévoit l'attribution d'un avantage sans avoir défini l'ensemble des conditions permettant son attribution. Par ses lignes directrices, l'autorité vient s'auto-encadré, ou encadré des services qui sont sous son autorité. Le juge précise que dans ses lignes directrices, l'autorité va donner des critères pour mettre en oeuvre le texte, le juge précise que l'autorité n'édicte aucune conditions nouvelle. Le juge nous rappelle qu'il est toujours possible de déroger à ces lignes, pour tenir compte des situations particulières, mais aussi pour des motifs d'intérêt général. 


Le régime contentieux des lignes directrices. \\
N'étant pas impératives, elles ne sont normalement pas susceptible de recours contentieux: CE, Sect., 1970, Crédit Foncier de France. On peut cependant les contester de manière détourné, par la voie de l'exception d'illégalité. \\
À l'occasion d'un recours, un requérant va contester la décision d'application de ces lignes directrices. À l'occasion du recours, le juge peut contrôler ces lignes, au regard des normes supérieures, et si elles se révèlent illégales, le juge va écarter ces lignes du règlement du litige. \\
Depuis 2015, ces lignes directrices deviennent opposable (arrêt Cortes-Ortiz). Il y a la possibilité de se prévaloir de ces normes devant le juge, qui considère qu'elles sont invocables dès lors qu'elles ont étés publiées par l'administration. Le juge fait désormais la distinction entre les lignes directrices et les simples orientations générales de l'administration. Le juge considère qu'il y a lignes directrices lorsque l'administré peut prétendre un avantage, et que les lignes ont étés publiées. Quand l'administré n'a pas de droit à prétendre à un avantage, les lignes sont en fait des orientations générales, ne pouvait pas être invoqués. 


Les chartes de l'Administration. \\
Ce sont des documents qui se développent de plus en plus en matière administrative (par exemple, les chartes de la laïcité dans le service public, une charte de doctorat). Une charte est un document conventionnel, dans lequel il y a eu un consentement des parties. Elles sont non impératives, elles sont destinées à orienter l'action des pouvoirs publics. Les chartes produisent quand même parfois certains effets juridiques, qu'à l'égard des signataires de cette charte. Elle ne produira donc jamais des effets à l'égard des tiers et donc des usagers du service public: CE, Sect., 8 Février 2012, UICMARA. Certaines chartes n'en sont pas vraiment et ne contiennent pas vraiment d'obligation réciproque comme les chartes de la laïcité qui sont présentes dans les collèges, hôpitaux, etc. Ces chartes sont susceptibles de recours, si elles sont impératives ou non.

\section{Le caractère administratif de l'acte unilatéral}

\subsection{La détermination des actes administratif unilatéraux}

Principal critère: un critère organique, l'auteur de l'acte. 


Les actes des personnes publiques. \\
Il existe un principe et deux dérogations. \\
Principe: les actes édictés par une personne publique sont présumées être administratif. La présomption est simple et peut être renversée dans deux cas. Les actes relatifs à la propriété privé de l'administration sont des actes de droit privé, cette propriété privée ne sont pas rattachés au service public, la personne publique gère donc ces propriétés comme un propriétaire ordinaire. \\
Seront aussi de droit privé les actes non réglementaires relatif à la gestion d'un SPIC. 


Les actes des personnes privées. \\
C'est le principe inverse, il y a là une présomption du caractère privée des actes des personnes privées. Deux dérogations. \\
D'abord, les décisions d'une personne privée gérant un SPA et mettant en oeuvre des prérogatives de puissance publique. \\
Avant la seconde guerre mondiale, le CE a posé seulement le critère du SP en considérant que les actes d'une personne privée gérant un service public sont administratif (CE, 1942, Montbeurt). Deuxième temps: CE, Sect., 1961, Magnier, qui a un triple apport, il précise le critère du SP, il cantonne l'arrêt Montbeurt au SPA, seule les décisions individuelles d'une personne privée gérant un SPA sont administratifs. Il faut enfin que l'acte mette en oeuvre des prérogatives de puissance publique. Troisième temps: l'arrêt Magnier est étendu: CE, Sect., 15 Mai 1991, Association Girondins de Bordeaux Football Club, le JA va considéré que le règlement de la ligue est bien un acte administratif, la ligue étant un SPA organisant des compétitions officielles. \\
CE, 17 Octobre 2012, Mr Singa, concernant une décision prise par un évêque pour l'organisation du culte catholique, il voulait utiliser des biens du service public. Le JA va considéré que cet acte est un acte de droit privé en application des trois précédents arrêts. Cela est étonnant dans la mesure où il existe bien un service public concernant ce culte, car l'affaire se déroulait en Alsace-Moselle, où il y a un service public du culte. \\
Actes réglementaires de personnes privées rattachées à un SPIC: TC, 15 Janvier 1968, Air France c/ époux Barbier. Concernant le règlement d'un SPIC, comme il touche à l'organisation d'un SP, c'est un acte administratif alors même qu'il est édicté par une personne privée. 

\subsection{Les actes de gouvernement}

Ces actes sont édictés par des personnes publiques, des autorités administratives. Ils ne relèvent pas l'exercice d'une fonction administrative, ils relèvent d'une fonction gouvernementale. Ces actes sont insusceptible de recours. \\
Il n'existe pas aujourd'hui, de définition d'actes de gouvernement alors qu'il existait au XIXe siècle, il se définissait par son mobile politique: c'est parce qu'il était pris pour un motif politique que c'était un acte de gouvernement. Cela a été abandonné suite à CE, 18 Février 1875, Prince Napoléon. \\
Le juge considère qu'il existe aujourd'hui deux catégories d'actes de gouvernement: les actes de relations entre les pouvoirs publics constitutionnels ; et les actes relatifs aux relations diplomatiques de la France. \\
La tendance est à la réduction des actes de gouvernement, notamment la deuxième catégorie. Le JA le fait sous l'influence de l'art. 13 de la CEDH (le droit au recours effectif). La doctrine propose depuis longtemps la disparition de cette catégorie, qui décline depuis une vingtaine d'années. 


Les actes relatifs aux relations entre les pouvoirs publics constitutionnels. \\
Parmi ces actes, on trouve un certain nombre de décision concernant la procédure législative, comme la décision du Gouvernement refusant le dépôt d'un PJL (CE, 29 Novembre 1968, Tallagrand). \\
L'acte de promulgation de la Loi n'est pas détachable de la procédure législative: CE, 29 Octobre 2015, Fédération démocratique Alsacienne. \\
On peut inclure dans cette catégorie l'ensemble des pouvoirs propres du PR: nommer un membre du CC (CE, Ass., 9 Avril 1999, Mme Ba), faire un référendum (CE, 1962, Brocas), recourir à l'article 16 (CE, 2 Mars 1962, Rubin de Cervens). \\
Ces actes peuvent concerner les rapports au sein d'un même pouvoir comme le décret de composition du Gouvernement: CE, 29 Décembre 1999, Lemaire.


Les actes relatifs aux relations diplomatiques de la France. \\
C'est dans cette deuxième catégorie que le juge réduit le plus les actes de gouvernement. Si l'acte est détachable des relations internationales, il sera considéré comme un acte de gouvernement. \\
Acte ayant un lien direct avec les RI: la décision de suspendre l'application d'un traité (TC, 2 Février 1950, Radio Andorre), à l'inverse, la décision de ratifier un traité peut être contrôlée au regard de l'article 53 de la Constitution. \\
Est considéré comme un acte de gouvernement la décision de reprendre les essais nucléaires, car, selon le juge, la décision est préalable à la négociation d'un traité: CE, 29 Décembre 1995, Greenpeace. \\
CE, Sect., 28 Mars 2014, Mr de Baynast, était en cause la décision d'un groupe français qui a refusée une candidature à la cour permanente d'arbitrage. Pour le juge, cet acte n'est pas détachable de la procédure conduite devant la CPI donc est un acte de gouvernement, insusceptible de recours. \\
À l'inverse, ne sont pas des actes de gouvernement les actes qui sont détachables des RI. Deux illustrations: n'est pas un acte de Gouvernement la décision d'octroi d'un permis de construire à une ambassade étrangère (CE, 22 Décembre 1978, Vothanhnghia). La décision par laquelle la France autorise ou refuse une extradition est susceptible de recours, et n'est donc pas un acte de gouvernement: CE, Ass., 17 Octobre 1993, Royaume-Uni de Grande Bretagne et d'Irlande du Nord.


\section{Les effets de l'acte administratif unilatéral}

\subsection{Le caractère exécutoire des décisions administratifs}

Les actes de l'administration font l'objet d'une présomption de légalité. C'est ce qu'on appelle le "privilège du préalable", ou encore "autorité de chose décidée". C'est de ce privilège que va découler le caractère exécutoire de la décision. L'administré est tenu d'exécuter l'acte de l'administration, et elle peut même le contraindre à s'exécuter sans nécessairement passer par un juge. Le JA a considéré en 1982 que ce caractère exécutoire est selon le juge, la règle fondamentale du droit public: CE, Ass., 2 Juillet 1982, Mr Huglo. \\
La formule est sans doute excessive mais elle permet de marquer l'importance de cette règle. 

\subsection{L'exécution de la décision}

L'administration dispose de plusieurs moyens pour contraindre un administré à respecter ses décisions. Le premier biais qui permet l'exécution, c'est l'existence de sanctions pénales prévues en cas de non respect d'un acte réglementaire, et sont prévus dans le code pénal lui même: R610-5 CP. \\
Deuxième moyen: les sanctions administratives, car certaines administrations possèdent un pouvoir de sanctions qui peuvent concerner spécifiquement certains actes. \\
Troisième moyen: l'exécution forcée qui permet à l'administration de se faire autoriser par un juge à faire exécuter sa décision. \\
Quatrième moyen: l'exécution d'office par laquelle l'administration agit d'office, elle pourra ici faire exécuter un acte sans aucune autorisation d'un juge. Cela a été posé par TC, 1902, Société immobilière Saint-Just, qui permet l'exécution d'office dans trois cas: si la loi le prévoit, en cas d'urgence, lorsqu'aucune voie de droit permet de sanctionner le comportement d'un administré ou ne permet de lui imposer l'application de la solution litigieuse. 

\subsection{Le caractère non suspensif des recours contentieux}

Un recours contentieux n'est pas suspensif. Si un recours était suspensif, ils permettraient de bloquer l'action de l'administration. \\
Dérogation importante: référé d'urgence, prévu par le CJA, le référé suspension prévu à l'article L521-1 du CJA, qui permet de demander au juge la suspension d'une décision dans l'attente d'une demande au fond. Deux conditions au référé-suspension: urgence et un doute sérieux sur la légalité de la décision litigieuse. 

\chapter{Les régimes des actes administratifs unilatéraux}

Loi du 12 Avril 2000: loi DCRA (Droits des citoyens dans leur relation avec l'administration), vient fixer un régime commun d'édiction des actes administratifs. En 2015, l'ensemble des règles de cette procédure non contentieuse a été codifié dans le CRPA, par une ordonnance du 23 Octobre 2015, entrée en vigueur le 1er Janvier 2016. Le code régit, comme son nom l'indique, les relations entre le public et l'administration mais parfois aussi entre l'administration et ses agents. Ce code synthétise l'ensemble des règles non contentieuses. 

\section{L'élaboration de l'acte administratif}

\subsection{La procédure administrative non contentieuse}

C'est la procédure de droit commun de l'élaboration de ces actes. Des procédures particulières peuvent être prévues. 

\subsubsection{Le principe du contradictoire}

C'est un principe qui a vocation à s'appliquer en matière juridictionnelle lors des procédures contentieuses. \\
Le respect du contradictoire est un PGD dégagé le CE, 4 Mai 1944, Dame Veuve Tromper-Gravier. Ce principe est aujourd'hui codifié dans le CRPA: L120-1. Le principe du contradictoire ne s'applique pas partout, il s'applique surtout à deux séries d'actes de l'administration: les décisions qui doivent être motivées (ce sont essentiellement les décisions défavorables) ; les actes qui sont pris en considération de la personne (essentiellement les sanctions de l'administration). \\
Limites à  l'art. L121-1 et suivants du CRPA, le principe ne s'applique pas lorsque l'administration statue à la demande d'un administré. Il ne s'applique pas non plus en cas d'urgence ou de circonstances exceptionnelles ou si l'application du principe compromet l'ordre public ou si il compromet les relations internationales de la France. Il ne s'applique pas à certains organismes, comme ceux de la sécurité sociale ou à pôle emploi. Ni non plus entre l'administration et ses agents sauf en cas de sanctions. Et enfin il ne s'applique pas si une procédure contradictoire spéciale est prévue par les textes.


Concrètement, le respect de ce principe a une double portée: il implique tout d'abord de pouvoir présenter ses observations à l'administration, de manière écrite ou orale. Cela doit être accompagné de la possibilité de se faire assister par un conseil: L122-1 du CRPA. \\
Ensuite, le principe a une portée significative lors d'une sanction, la personne sanctionnée doit pouvoir connaître les griefs retenu à son encontre ainsi que de se faire communiquer le dossier: L122-2 CRPA.

\subsubsection{La transparence de l'action administrative}

Il n'existe pas de principe général de transparence. Mais de nombreuses dispositions permettent une telle transparence, et celle-ci a été renforcée par le législateur depuis quelques années. 


Traitement des demandes. \\
Le CRPA a défini ce que sont ces demandes. Il entend par demande: "Les demandes et réclamations ainsi qu'un recours administratif" L110-1 CRPA. Il existe deux catégories de recours administratifs: les recours gracieux et hiérarchique. \\
Le CRPA donne des règles sur ces échanges, ils doivent être en langue française, ils peuvent être fait via voie électronique (et même uniquement dans certains cas): L112-8 CRPA. Le code encadre ces échanges, et pour qu'ils aient lieu, l'administré doit s'identifier pour communiquer avec l'administration, et ces échanges sont impossibles dans certains cas: hypothèse du secret défense, si l'ordre public s'y oppose, si la présence du demandeur est nécessaire, pour des raisons de bonne administration. L'administration doit donner un accusé de réception et peut elle même répondre par voie électronique sauf si le demandeur s'y oppose. \\
Contenu des échanges: des formalités minimales s'imposent à l'administration (L111-2 CRPA), les correspondances doivent indiquer la qualité de l'agent qui traite le dossier, son nom et son prénom ainsi que son adresse administrative. Limite: des motifs de sécurité publique permettent parfois d'anonymiser certaines correspondances. Dans le cas de la lutte contre le terrorisme, certains services ont étés dispensés d'indiquer leurs coordonnés. Le CRPA impose aussi la signature de l'acte par son auteur: L212-1, la signature peut être électronique, certains actes en sont dispensés: L212-2 CRPA. \\
Accusé de réception: dès lors qu'une demande est formulé, l'administration doit donner un accusé de réception: L112-3 CRPA. Cela concerne aussi les demandes électroniques. Il doit contenir toutes les mentions évoqués ci dessus. Cet accusé permet de connaître la date de naissance de certaines décisions, notamment les décisions implicites de l'administration. En l'absence d'accusé de réception, le demandeur pourra contester la décision à tout moment sans qu'on puisse vraiment opposer un délai de recours: L112-6 CRPA. Accusé de réception non nécessaire en cas de demandes abusives, en cas d'hypothèse d'urgence, ou si l'administration dispose d'un délai très bref pour répondre à la demande ou dans certains cas en matière électronique: comme si cela porterai atteinte aux systèmes de l'administration. \\


Les décisions implicites. \\
Ce sont les décisions nées du silence de l'administration. En principe, c'était des décisions de rejet après un silence de deux mois, c'était la règle posée par la loi d'Avril 2000. Cette règle a été renversée: L231-1 CRPA, qui trouve son origine dans une loi de 2013. Désormais, le silence de l'administration pendant deux mois équivaut à une décision implicite d'acceptation. Cette règle est rentrée en vigueur en 2014 pour l'État et en 2015 pour les collectivités locales. \\
Il existe de très nombreuses dérogations: L231-4 et suivants du CRPA. \\
Elle ne s'applique pas aux décisions non individuelles, en cas de recours administratif, aux demandes qui ont un caractère financier, quand elle porterait atteinte soit à l'OP soit à la protection des libertés fondamentales soit au respect des conventions internationales ou à un principe à valeur constitutionnel, dans les relations entre l'administration et ses agents. Des décrets en CE peuvent rajouter des dérogations. Le Gouvernement en a adopté 40. La règle ne concerne donc que 1200 procédure sur 3600. \\
Cela a donc créé une grande critique de la part de la doctrine, la règle précédente ne posait aucunes difficultés. Le principe nouveau n'est en fait qu'une exception. \\
Cela complique aussi les choses pour l'administration, qui doit répondre à tous les administrés, ce qui peut poser certaines difficultés matérielles. \\
Cette règle est entourée de garanties procédurales. L232-3 CRPA, il doit y avoir une attestation de la décision. Comme pour toute demande, l'administration a obligation d'accusé de réception, et à l'occasion de cela, l'administration doit trouver quelle règle s'applique: si le silence vaut acceptation ou si il vaut rejet. \\
Dans le cas où la demande a été adressée à une autorité incompétente: elle est obligée de transmettre à l'autorité incompétente. L'administré sera prévenu de cette transmission: L114-2 CRPA. \\
Le point de départ du délai de naissance de la décision implicite d'acceptation est la date de la réception de la demande: L114-3 du CRPA.

\subsubsection{La communication des documents administratifs}

Dans certains cas, c'est une obligation pour l'administration. Depuis 1978, la loi prévoit un droit d'accès aux documents administratifs. \\
Outre ce droit d'accès, il existe une liberté de réutilisation d'informations publiques, qui a été complété en 2016 par l'obligation de communication de ces données par voie numérique. C'est l'Open Data. Cela ne concerne que les documents de certains organismes. Les établissements publics ne sont pas concernés. Le tout est codifié au L300-2 CRPA. \\
L'administration a obligation de mettre à disposition ces documents. Cette obligation ne concerne pas la défense, le secret (de la vie privée, médical, secret défense). \\
Pour obtenir communication de ces documents, il faut une demande de l'administré, que l'administration peut refuser. Ce refus ne peut pas être directement contesté devant le juge, l'administré doit saisir une AAI, la CADA (Commission d'accès aux documents administratifs). Si la CADA refuse, l'administré pourra contester la décision devant le juge. \\
Certains documents de la présidence devaient tomber sous le joug de la loi de 1978. S'est posé la question si cela ne heurtait pas l'article 67 de la Constitution, car certains documents montraient des commandes de sondages. Le juge a conclu que ces documents devaient être communiqués: TA Paris, Avrillier, 2010. Le juge pose quelques limites: il ne peut pas concerner les documents du Chef de l'État en tant que personne privée, les autres documents sans rapport avec les missions dévolus à l'État dans l'exercice de missions de service public. 


Liberté de réutilisation des informations publiques. \\
Cela a été fait sous l'influence de la directive ISP du 17 Novembre 2003, transposé par l'ordonnance de Juin 2005. Cela est désormais codifié dans le CRPA, depuis fin 2016. \\
Cette liberté, c'est la liberté pour toutes personnes de réutiliser les informations publiques à toutes fins, différentes desquelles elles ont étés récoltées. Ce sont des données publiques. \\
Deux limites: les données publiques produites dans le cadre d'un SPIC ne sont pas réutilisables, car il les utilises lui même à des fins commerciales. La deuxième limite est la propriété intellectuelle, certaines données en font l'objet et ne sont pas réutilisable.


Automaticité de la communication des données publics. \\
C'est ce qu'on appelle l'Open Data. C'est l'idée de mettre librement les données publiques en ligne, sans que les administrés n'aient à en faire la demande. \\
En 2016, loi pour une République numérique, le législateur met en place un service public de la donnée, qui vise à mettre à disposition des données à disposition en ligne et gratuitement. Cela implique nécessairement une obligation pour les personnes publiques: communiquer leurs données publiques: L312-1-1 CRPA. Cela concerne autant l'État que les collectivités. \\
Cela s'est doublé d'un principe de gratuité des données publiques, qui a été affirmée par une loi de 2015 et ne prévoit que quelques dérogations, notamment les données produites par les SPIC. \\
Deux visions de l'Open Data: considérer que les données publiques sont un bien commun, servant à tous, produite par la personne publique dans l'intérêt général. L'idée de gratuité vient du fait que la production a déjà été financée. \\
Une autre conception consiste à dire que l'Open Data est souvent utilisé à des fins commerciales, et donc certains considèrent qu'il ne serait pas illégitime de faire payer ceux qui réutilisent à des fins commerciales. 

\subsubsection{La négociation}

L'acte administratif est un acte de puissance publique. La puissance publique est que l'administration puisse imposer sa volonté aux administrés. L'acte est donc par principe, élaboré par l'administration seule. Or, cela évolue depuis une vingtaine d'années, qui tend à associer le public à l'élaboration de l'acte administratif. \\
L'administration va recueillir l'avis des administrés, c'est parfois une obligation procédurale. L'administration pourra ou non prendre en compte ces avis. C'est l'idée de faire participer les administrés, qui a l'air d'aller à l'encontre de la notion de puissance publique, cependant, cela présente des avantages. Dans certains domaines techniques, l'administration peut recueillir les avis de certains opérateurs spécialisés. Un deuxième avantage serait que par cette idée de consultation, l'administration va indirectement recueillir le consentement des administrés, donnant une légitimité à la nouvelle norme. Un troisième avantage est que cela conserve la sécurité juridique, car la consultation les prévient de normes juridiques à venir. \\
On distingue classiquement deux techniques. On a d'abord la consultation et d'autre part, la concertation. La consultation, c'est l'idée d'uniquement recueillir les avis. La concertation implique une certaine idée de négociation, ou du moins, un échange, un dialogue. \\
Certains auteurs ont parlés de contractualisation de l'action administrative, qui prendrait une forme contractuelle. En cas de concertation, cela aboutit à une certaine négociation qui ressemble au fait de négocier un contrat. Les actes administratifs négociés ne sont pas des contrats: ils restent des actes unilatéraux. \\
Ces procédures de négociation concerne aujourd'hui deux domaines: l'environnement et l'urbanisme. Dans le domaine de l'environnement, la participation du public est constitutionnelle, dans la charte de l'environnement. \\
Ces procédures ont étés médiatisés dans une affaire, évoqué notamment par le CE, 20 Juin 2016, NDDL. 

\subsection{La motivation des actes administratifs}

C'est l'idée selon laquelle les administrateurs doivent pouvoir rendre compte de leur action et rendre compte de pourquoi ils ont pris telle ou telle décision. C'est une obligation fondamentale qui n'a pas existé pendant longtemps. Il a fallu attendre la loi du 11 Juillet 1979, elle est codifiée au L211-1 du CRPA. 

\subsubsection{Le champ d'application de la motivation}

Champ d'application organique. \\
Qui sont les auteurs soumis à cette obligation ? L211-1, l'obligation de motivation s'impose à toutes les personnes publiques, aux personnes privées chargées de service public, peu importe sa nature selon le CRPA. L'obligation de motivation concerne aussi aujourd'hui les relations entre les administrations. 


Champ d'application matériel. \\
Quelles sont les décisions concernées ? Deux d'après le CRPA, d'abord, la catégorie évoqué à l'article L211-3 du CRPA qui soumet à l'obligation de motivation les décisions individuelles qui dérogent aux règles générales fixées par les lois et règlements. L211-2 CRPA, sont soumis à obligation de motivation les décisions individuelles défavorables: les mesures restrictives de libertés (mesure de police administrative), les sanctions de l'administration, les autorisations conditionnées, les décisions qui mettent fin à des situations créatrices de droits, et enfin les décisions qui refusent un avantage qui est un droit pour l'administré. \\
Cette liste peut être complétée par des décrets en Conseil d'État. 

\subsubsection{La portée de la motivation}

"Indiquer les considérations de droit ou de fait qui constitue le fondement de la décision", L211-5 du CRPA. Cela vient préciser deux éléments, les indications doivent être effectuées par écrit dans le corps de la décision concernée. La motivation doit être effectuée sans délai. Le juge censure les motivations standards, qui ne sont pas convenable. Les décisions non motivée ou mal motivée sont des décisions illégales: CE, 24 Juillet 1981, Belasri. \\
L'obligation disparaît dans trois cas prévus: en cas d'urgence absolue (l'administré peut faire la demande des motifs dans le délai du contentieux), dans les décisions implicites (idem qu'avant sur la demande), enfin, les faits couverts par le secret (médical, défense). Le secret médical ne dispense pas de motiver, les faits couverts par le secret doivent être cachés. 

\section{L'entrée en vigueur de l'acte administratif}

L'acte administratif ne devient obligatoire qu'à l'entrée en vigueur de celui-ci. Cette entrée fait l'objet de règles de plus en plus techniques. 

\subsection{Les conditions de rentrée en vigueur}

Ces conditions sont variables suivant la nature de l'acte, si il est individuel ou réglementaire. 

\subsubsection{Les décisions réglementaires}

L221-2 du CRPA: cette disposition s'applique aux actes réglementaires mais aussi aux décisions d'espèces. \\
L'entrée en vigueur de ces actes est subordonné à la publication ou à l'affichage de ces actes. L'administration a l'obligation de publier les actes réglementaires dans un délai raisonnable, c'est un principe général du DA consacré par CE, 12 Décembre 2003, Syndicat des commissaires et des hauts fonctionnaires de police. \\
En principe, l'entrée en vigueur a lieu le lendemain de l'affichage. Il arrive parfois qu'un acte nécessite des mesures d'applications, et quand elles sont nécessaires, l'entrée en vigueur de l'acte est déplacé au jour de l'édiction de ces mesures. Il arrive assez souvent qu'un texte législatif ou réglementaire prévoit une règle différente, c'est le cas notamment de certains règlements locaux, qui doivent être transmis au préfet avant de pouvoir rentrer en vigueur: L222-1 du CRPA. \\
Concernant les actes réglementaires nationaux, L221-3 du CRPA, quand l'acte est publié au JO, il rentre en vigueur le lendemain ou à la date qu'il fixe. Il y a l'hypothèse de l'urgence, où l'acte peut rentrer en vigueur le jour même. 


\subsubsection{Les décisions non réglementaires}

L221-8 CRPA, elles rentrent en vigueur dès la notification, qu'elle soit favorable ou défavorable. \\
Le CRPA ne parle plus d'entrée en vigueur mais d'opposabilité, ce qui est la même chose. Des textes peuvent prévoir des règles particulières.

\subsection{Sécurité juridique}

C'est un principe qui n'a pas toujours existé en droit interne. Le principe est issu du droit allemand, qui a été transposé en droit de l'Union comme un principe général. Sa portée est assez mal défini, le principe est en constante évolution et concerne de plus en plus des pans entiers de l'administration. \\
Le JA a, pendant longtemps, refusé de consacrer un tel principe, partant de l'idée que la mutabilité du SP ne permet pas de garantir un tel principe. Le JA n'a pas toujours été insensible à la sécurité juridique, il a donc finalement consacré son propre principe en 2006: CE, Ass., KPFG. \\
En droit positif, ce principe semble imposer l'édiction de mesures transitoires. La sécurité juridique a une portée protéiforme, il a plusieurs conséquences. 

\subsubsection{La non rétro-activité des actes administratifs}

CE, 26 Décembre 1925, Rodieres, une décision administrative ne peut statuer que pour l'avenir. Il en a fait un PGD: CE, Sect., 1948, Société du Journal l'Aurore. \\
La date d'entrée en vigueur ne peut être antérieur à la date d'adoption de l'acte.\\
Ce principe a ses racines dans l'article 2 du C.Civ. \\
L221-4, le CRPA a codifié lui même le principe, avec des dérogations: si l'acte est lui même pris en application d'une loi rétroactive ; un acte pris en application d'une décision d'annulation du JA. 

\subsubsection{L'obligation d'édicter des mesures transitoires}

En principe, l'acte réglementaire est d'application immédiate. C'est l'intérêt général qui commande une telle application. Cela pose une difficulté concrète: un acte peut modifier le droit positif de manière brutale. \\
En droit de l'UE, le juge a dégagé le principe de la protection de la confiance légitime, qui interdit de changer brutalement le droit positif. Le JA a toujours refusé de consacrer un tel principe explicitement: CE, 2001, Entreprise transport Friybuth. \\
Le juge évite de se sentir lié par un principe étranger, il a donc consacré son propre principe dans l'affaire KPMG. \\
L'application de dispositions nouvelles doit être accompagné de mesures transitoires pour permettre de s'adapter. Il faut édicter des mesures transitoires lorsque les nouvelles dispositions portent une atteinte excessive aux situations contractuelles en cours et lorsque l'application immédiate porte une atteinte excessive aux intérêts en présence. \\
Quand l'intérêt général le commande, ces principes peuvent être contournés. \\
L221-5 et L221-6 du CRPA, le code reprend presque mot pour mot l'arrêt KPMG. Le code vient préciser ce que peuvent être ces mesures transitoires: une date d'entrée en vigueur différé ; en des règles particulières qui vont s'appliquer temporairement ; la précision des conditions d'application de la nouvelle réglementation, qui pourraient varier avec le temps. 

\subsubsection{La question des revirements de jurisprudence}

En principe, un revirement de jurisprudence est rétro-active. Ceci peut être très attentatoire à la sécurité juridique des administrés. C'est aujourd'hui encore le principe. Cependant, pour palier à la difficulté, je JA a développé une solution, CE, Ass., 16 Juillet 2007, Tropique travaux signalisation, le JA déroge à la rétroactivité du revirement de jurisprudence pour préserver justement la sécurité juridique. Cela s'appelle la modulation dans le temps du revirement de JP. Le JA peut dire que le revirement prendra effet que pour l'avenir ou à partir d'une certaine date. 

\section{La sortie de vigueur de l'acte administratif}

La sortie de vigueur, c'est la question de la fin de l'AAU. En principe, l'acte prend fin selon ce qu'il dispose. Il peut disposer que son effet cesse à une certaine date. En général, les actes ne prévoient rien et se voient à être appliqué sans limitation de durée. L'administration peut mettre fin à une décision via deux techniques, elle peut abroger un AAU (annulation pour l'avenir), ou le retirer (disparition rétro-active). Cf L240-1 du CRPA. \\
Ces possibilités de retirer un acte sont souvent utilisés pour faire évoluer l'action publique, donc dans l'intérêt général. Cependant, l'abrogation ou le retrait peut avoir des effets sur l'administré, notamment lorsqu'il crée des droits. Les retirer pourrait donc poser des problèmes de sécurité juridique. \\
Il faut donc concilier d'une part la sécurité juridique des administrés et d'autre part l'intérêt général, qui nécessite que les actes puissent évoluer. Cette conciliation fait que les règles de retrait sont assez complexe. Le CRPA a codifié ces règles, et les a simplifiés dans les articles L240-1 et suivants. \\
Le CRPA distingue les Actes qui créent des droits et ceux qui ne créent pas de droit.  

\subsection{La sortie de vigueur des actes créateur de droit}

Elle peut s'effectuer à l'initiative de l'Administration ou à la demande du bénéficiaire. 

\subsubsection{L'abrogation et le retrait à l'initiative de l'administration}

Quand un acte est créateur de droit, la sortie de vigueur est très encadré. Le CRPA pose un principe et deux dérogations. \\
Le principe est que l'administration peut abroger ou retirer un acte créateur que si deux conditions sont réunis, il faut d'abord que l'acte soit illégal (l'acte légal ne peut donc pas être sorti de vigueur par l'administration), et que la sortie de vigueur se fasse dans les quatre mois à partir duquel l'acte a été pris (L242-1 CRPA). CE, Ass., 2001, Ternon, posait ces conditions. CE, 2009, Coulibaly, de même pour l'abrogation. \\
Deux dérogations sont posées à l'article L242-2 du CRPA. L'administration peut abroger ou retirer un acte sans condition de délai. D'abord, si le bénéficiaire ne remplit plus les conditions du droit qu'il a acquis. Ensuite, il est possible de retirer une décision qui accorde une subvention si les conditions de la subvention n'ont pas étés remplis. Il est à noter qu'un tiers peut demander à ce qu'un tel acte peut être retiré dans ces deux situations. 

\subsubsection{L'abrogation et le retrait à l'initiative du bénéficiaire}

Dans une tel hypothèse, l'administration aura l'obligation de retirer l'acte, ou au moins la possibilité de le faire.
L242-3 CRPA, hypothèse dans laquelle l'administration à l'obligation: deux conditions doivent être remplis, si l'acte est illégal, et si l'abrogation ou le retrait peut intervenir dans les quatre mois suivant la prise de décision. \\
Deuxième hypothèse dans laquelle l'administration a la possibilité de retirer l'acte, sans conditions de délais, même si la décision est légale. Deux conditions: remplacer par une décision plus favorable ; ne porte pas atteinte aux droits des tiers (L242-4 CRPA). 

\subsection{La sortie de vigueur des actes non créateurs de droit}

Ce sont des actes réglementaires, ou non réglementaire non créateur de droit comme les décisions d'espèce ou décisions individuelles. 

\subsubsection{Le régime de l'abrogation des actes non créateurs de droit}

Principe important: l'acte non créateur de droit peut être modifié ou abrogé à tout moment sans motif et sans condition de délai (L243-1 CRPA). Ce principe s'explique par le principe de mutabilité du service public. \\
Au delà du principe, dans certains cas, l'administration peut avoir l'obligation d'abroger un acte réglementaire. L'abrogation des actes réglementaires est un PGDA: CE, Ass., 1989, Alitalia, posant le principe d'obligation d'abrogation des actes réglementaires illégaux. En 1989, l'administré devait faire une demande pour qu'un acte illégal soit retiré. En 2007, le législateur a considéré que l'administration se devait de retirer les actes, ce qui a été codifié en L243-2 CRPA. \\
Peu importe que l'acte ait été illégal dès l'origine ou qu'il soit devenu illégal, cela concerne tous les actes illégaux. Si l'illégalité a cessé, l'obligation disparaît. \\
L'administration aura aussi l'obligation d'abroger les actes non réglementaires non créateur de droit dans l'hypothèse dans laquelle ces actes seraient devenus illégaux dans un changement de droit ou de fait: CE, Sect., 1990, Association Les Verts. Solution codifié en L243-2 CRPA. Si l'acte était illégal depuis le début, l'administration n'a pas l'obligation de le retirer.

\subsubsection{Le régime de retrait des actes non créateurs de droit}

Ce retrait est possible sous deux conditions uniquement. Si l'acte en question est illégal et si le retrait intervient dans les quatre mois de la prise de décision: L243-3 CRPA. Le code ne prévoit qu'une dérogation, concernant une sanction, ces sanctions pouvant être retirés à tout moment:: L243-4 CRPA. 

\subsection{La notion d'acte créateur de droit}

Le juge n'a jamais de définition, donc le CRPA non plus. \\
La certitude est qu'un règlement administratif n'est jamais créateur de droit (CE, Vannier, 1961, nul n'a d'acquis à l'égard d'un règlement). \\
On doit donc se satisfaire d'une énumération du JA. Une décision créatrice de droits n'est pas systématiquement une décision favorable. \\
Concernant les décisions obtenus par fraude, ils ne créent jamais de droit pour les administrés: CE, 2002, APHM, codifié en L242-1 CRPA. \\
Les décisions qui accordent un avantage financier sont en principe toujours créatrices de droit, même si elles sont illégales: CE, Sect., 2002, Soulier. Ces décisions ne peuvent être abrogés que dans un certain délai, dans les quatre mois suivant la prise de décision. \\
Concernant les contrats de recrutement d'agents publics, ils sont là aussi créateurs de droit: CE, 2008, Mr C. \\
Certaines décisions sont trompeuses, CE, Mr P, qui concernait une radiation d'un demandeur d'emploi, le juge a considéré que c'était une décision créatrice de droits car même si cette décision met fin à certaines allocations, cette décision permet l'obtention de nouvelles aides publiques. 


\chapter{Le contrat administratif}

Jusqu'au milieu du XIXe siècle, quand l'administration passait des contrats, ils étaient de droit privé. Le juge a ensuite reconnu des contrats publics, répondant à des définitions particulières, et obéissant à un régime juridique original. Il y a un profond déséquilibre dans ces contrats, en faveur de la puissance publique, au nom de l'intérêt général. \\
C'est au XXe siècle que le juge a élaboré ce régime des contrats administratifs. On appelle ce domaine le contractualisme, qui évoque cette tendance à la multiplication des contrats, car l'État use et abuse de ces contrats dans des domaines de plus en plus nombreux, étant devenus comme un outil d'administration comme peut l'être l'acte administratif. \\
Le contrat dans l'administration, c'est l'administration contractuelle, où les personnes publiques passent des contrats entre eux, ce qui est inédit car les pers. publiques avaient avant des relations par acte unilatéral. \\
Il existe des contrats d'affaire, des contrats par lesquels l'administration demande à un opérateur privé de satisfaire ses besoin en contrepartie d'une redevance économique. Ce sont les marchés publics. \\
Il y a aussi les concessions, ce sont des contrats à objet et montant très importants. Ceux-ci conduisent à des contentieux très importants. Cela a créé le domaine du droit public des affaires. 

\section{La nature contractuelle de l'acte administratif}

Aujourd'hui, il existe un critère pour distinguer le contrat de l'acte unilatéral. Mais parfois, ces deux formes tendent à converger. 

\subsection{La distinction du contrat et l'acte unilatéral}

Trois intérêts à cette distinction: le régime juridique qui s'applique à ces actes, le régime du contrat est bien distinct de l'AAU, notamment la disparition de ceux-ci, qui peuvent être plus facile pour les contrats. Dans certains domaines, l'administration ne peut pas passer des contrats, notamment celui de la police administrative, du pouvoir réglementaire (résulte de CE, 1985, Association Les amis de la Terre). Le régime contentieux des contrats est différent des AAU, les contrats faisant l'objet d'un recours de plein contentieux. \\
Au début du XXe siècle, les auteurs de la doctrine proposaient des critères quantitatifs liés à l'auteur de l'acte: l'acte unilatéral aurait un seul auteur alors que l'acte plurilatéral aurait plusieurs auteurs. Cette distinction n'est pas vraiment acceptable dans la mesure où certains actes unilatéraux ont eux même plusieurs auteurs comme les arrêtés interministériels ou les arrêtés inter-préfectoraux. \\
Le critère qualitatif est donc préférable. Ce qui caractérise le contrat pour un auteur de doctrine est qu'un contrat crée des normes qui s'imposent à elle même, c'est donc le contenu de l'acte qui permet de définir si c'est un AU ou un contrat. Si l'acte régit des relations tiers, c'est un acte unilatéral. \\
Aujourd'hui, le JA prend en compte ce critère qualitatif, notamment les obligations réciproques des partis auxquels toutes deux ont consentis: CE, 29 Juin 2012, PRO2C. On note que le juge a requalifié un acte unilatéral en contrat, car cet AU contenait des obligations réciproques. Le contrat doit d'ailleurs respecter les règles de mises en concurrence. 

\subsection{Le rapprochement de l'acte unilatéral et du contrat}

On trouve parfois des AU dans des contrats administratif.

\subsubsection{L'inclusion de l'acte unilatéral dans le contrat}

Il y a dans des contrats des stipulations unilatérales. C'est principalement l'hypothèse des clauses réglementaires, une clause incluse dans le contrat, destiné à s'appliquer à des tiers au contrat. Ce sont principalement des clauses relatives au fonctionnement du service public, notamment les clauses tarifaires, le contrat pouvant prévoir les tarifs que paieront les usagers du service. \\
Les contrats passées entre l'État et les entreprises publiques peuvent avoir pour objet d'organiser des services publics, on appelle cela des contrats programmes, des contrats d'entreprise, des contrats de régulation. Ces contrats d'entreprise contiennent exclusivement des mesures réglementaires. Ils ont une forme contractuelle, car voulu ainsi par le législateur mais ont un contenu réglementaire. \\
Certains auteurs ont considérés qu'il s'agissait là d'actes mixes: forme contractuelle mais contenu réglementaire. Cette théorie était proposée Madiot, elle n'a jamais été reprise par le JA pour qui les choses sont binaire (contrat ou AU). 

\subsubsection{L'association du contrat et de l'acte unilatéral}

On peut constater l'association de la technique contractuelle à l'acte unilatéral. Il y a des AU dont l'édiction est précédé du consentement d'un tiers: hypothèse de concertation préalable. \\
On constate une association de la forme contractuelle à l'acte unilatéral, ce sont les hypothèse d'une combinaison de l'AU et du contrat. On les retrouve dans le cadre d'autorisation administrative. Cette dernière est en principe un AU mais parfois, ces autorisation prennent une forme contractuelle. En cas d'aide publique, elle doit être attribué par contrat si la somme dépasse 23 000 euros, donnant des responsabilités d'intérêt général au bénéficiaire. \\
Pour les opérateurs radios et hertziens en général, ceux-ci ont besoin d'une autorisation donné sous forme de contrat. Le contrat autorise l'utilisation de fréquence données  et charge le bénéficiaire d'obligations. \\
Le juge traite ces actes comme des actes unilatéraux: CE, 1998, Compagnie Luxembourgeoise de radiodiffusion. Si le juge traite ces actes comme des AU, c'est parce qu'ils considèrent que les effets de la convention dépendent avant tout des effets de l'AU. Le contrat est considéré comme étant incorporé à l'AU. C'est Guy Brémant, un commissaire du gouvernement, qui a créé cette théorie de l'incorporation, et le fait d'y considéré comme un AAU fait que les tiers peuvent émettre un recours. 

\section{Le caractère administratif du contrat}

Le régime contentieux du contrat administratif est particulier. Comment distinguer le contrat administratif du contrat de droit privé. \\
Dans certains cas, c'est la loi qui détermine la qualification, et ce peut donc être la loi qui qualifie un contrat d'administratif. Hors, si il existe une absence de qualification législative, il existe des critères jurisprudentiels. \\
Il se peut qu'il y ait une évolution dans le contrat, par exemple qu'une personne publique devienne une personne privée ou qu'une loi requalifie un contrat. Le TC a posé un principe important en 2006, un principe de stabilité de la nature des conventions: TC, 16 Octobre 2006, Caisse centrale de ré-assurance, la nature d'un contrat s'apprécie au jour de la conclusion de ce contrat, l'évolution n'a donc aucune influence. \\
Cela a trois conséquences: si une personne publique devient une personne privée, cela ne change en rien la nature du contrat, le changement du droit applicable ne change en rien la nature du contrat, la nature du contrat ne change pas même en cas de cession du contrat à une personne privée, même si cette cession se fait de manière rétro-active: TC, 11 Avril 2016, Société Fosmax. 

\subsection{Les contrats administratifs par détermination de la Loi}

La qualification législative peut prendre plusieurs formes. Le législateur peut décider que des contrats sont administratifs, ou privés. Il peut aussi prévoir qu'un contrat soit sous l'égide du JA, ce qui en fait un contrat administratif. \\
Par volonté du législateur, sont des contrats administratifs les contrats de travaux public (Loi du 28 Pluviose, an VIII), les contrats liés au domaine public et d'occupation du domaine public (art. L2331-1 CG3P), les marchés publics dès lors qu'ils sont passés par une personne publique (art. 3 de l'ordonnance du 23 Juillet 2015). Les marchés de partenariat sont aussi des contrats administratifs lorsque passés par des personne publiques, ce n'était auparavant pas le cas (ordonnance du 23 Juillet 2015). Les contrats de concession sont les nouvelles délégations de service publics, ils sont administratifs lorsque passés par des personnes publiques (art. 3 de l'ordonnance du 29 janvier 2016). Les contrats d'achat obligatoire d'électricité (entre EDF et les producteurs autonomes d'énergie), ces contrats sont administratifs (art. 88 de la loi du 12 Juillet 2010). 

\subsection{Les critères jurisprudentiel du contrat administratif}

Ces critères sont dégagés par le juge. Deux viennent se cumuler, un critère organique des partis au contrat, et un critère matériel, concernant l'objet du contrat et plus largement son contenu.

\subsubsection{Le critère organique}

Il est simple. Pour que le contrat soit administratif, il faut qu'au moins une personne publique soit partie au contrat. Cependant, ce principe comprend de nombreuses dérogations. 


Entre une personne publique et une personne privée. \\
C'est en principe un contrat administratif. Deux dérogations: lorsque la personne publique agit au nom et pour le compte d'une personne privée, sous mandat de celle-ci. Dans cette hypothèse, le critère organique n'est pas rempli: CE, 3 Juin 2009, OPAC du Rhône. La deuxième dérogation concerne les contrats relatif aux domaines privés.


Entre deux personnes privées. \\
Ils sont en principe de droit privé, sauf dans quatre hypothèse. \\
D'abord, la théorie du mandat. Hypothèse dans laquelle une des personnes privées parti au contrat agirait pour le compte d'une personne publique: CE, 30 Janvier 1931, Brossette. On évoque ici l'idée d'un mandat explicite, expressément prévu par une convention. Le JA a admis l'hypothèse d'un mandat tacite, que le JA relève grâce à un faisceau d'indices: CE, Sect., 3 Mai 1975, Société d'équipement de la région Montpellieraine, confirmé par TC, Commune d'Agde. La théorie de ce mandat tacite semble être abandonnée. TC, 9 Juillet 2012, Compagnie générale des eaux, une personne privée délégataire d'un service public avait passé un second contrat pour faire exécuter le service public et avait été présumé agir pour son compte par le juge. \\
Ensuite, la deuxième dérogation concernait certains contrats de travaux publics. TC, 8 Juillet 1963, Perot. Le contrat passé par un entrepreneur privé avec une société privé chargée de la construction d'une autoroute sont des contrats administratifs. Le juge adopte cette solution en disant que la construction d'une route est un travail public et appartient par nature à l'État. Le juge considère que même si le contrat est privé, le concessionnaire a agi pour le compte de l'État. On peut se demander pourquoi il y a ici la théorie du mandat implicite alors que la loi pourrait qualifier un tel contrat d'administratif. Le JA a étendu cette solution non seulement à la construction de routes ou d'autoroutes mais aussi d'accessoires à ces voiries, comme des tunnels, etc. CE, 24 Avril 1968, Tunnel routier du Mont-Blanc. CE, 9 Février 1994, Société des autoroutes Paris-Rhin-Rhône. Aujourd'hui, l'arrêt de principe est TC, 9 mars 2015, Mme Rispail, dans laquelle le juge considère que des contrats d'entretien, de construction, ou d'exploitation d'autoroute sont des contrats de droit privé. Il justifie cette nouvelle solution et considère explicitement que le concessionnaire ne peut pas être regardé comme avoir agi pour le compte de l'État. L'exploitation des autoroutes a été cédé en 2005 à des opérateurs privés, ce qui a conduit à l'abandon de la JP Perot en 2015. \\
La troisième dérogation concerne les personnes privées transparentes, une personne privée créée à l'initiative d'une personne publique, contrôlée par elle, et financé par elle. CE, 21 Mars 2007, Boulogne-Billancourt. \\
La quatrième dérogation concerne l'accessoire d'un contrat administratif. Selon le JA, un contrat entre deux personnes privées peut être un contrat administratif lorsqu'il est l'accessoire d'un contrat public. TC, 8 juillet 2013, S2EP c. EDF et ERDF, était en cause les contrats d'achat obligatoire d'énergie par EDF, qui sont administratif selon la loi. Les contrats de raccordement au réseau sont-ils administratifs ? D'après le juge, non, car le contrat n'est pas accessoire au contrat obligatoire d'achat. 


Entre deux personnes publiques. \\
Il y a une présomption d'administrativité. TC, 1983, UAP. Le juge nous dit que les contrats entre personnes publiques sont en principe des contrats de droit public "sauf dans les cas où ils font naître que des rapports de droit privé entre les parties". La présomption est donc simple, elle peut être renversée. Cette présomption, si elle existe n'a pas d'intérêt car le juge vérifiera quand même le critère organique et le critère matériel. TC, 15 Novembre 1999, Commune Bourisp. 

\subsubsection{Le critère matériel}

Il se dédouble en deux sous-critères alternatifs. Le JA a d'abord dégagé le critère de clauses exorbitantes du contrat puis le critère de rattachement du contrat au service public. 


Le rattachement du contrat au service public. \\
Le contrat qui se rattache au service public est l'hypothèse de l'arrêt de principe CE, Ass., 20 Avril 1956, époux Bertin, dans cette affaire, les époux Bertin avaient hébergé et nourri des ressortissant Russes. Le JA avait considéré que les époux, en vertu de la convention passé entre les époux et l'administration, s'occupait d'un service public et que le contrat était donc administratif. Arrêts Terrier et Terron, de 1903 et 1910, là encore, on a une hypothèse dans laquelle il y a délégation de service public par contrat. \\
Le contrat qui se rattache au service public est aussi celui qui fait participer le co-contractant au service public. À partir de quand commence la participation au SP ? CE, 1994, Codiam, il s'agissait de fournir des YV à des malades, le JA considère que faire livrer des TV fait participer le co-contractant au SP hospitalier. TC, 1998, Bergas, consistant à la fourniture de TV à des détenus, et là, le TC, à l'inverse, va dire que les co-contractants ne participent pas à l'exécution du SP. Le TC a confirmé cette position en TC, 2007, Codiam. \\
Le CE étant têtu, CE, 2014, CHU Rouen, concernant la fourniture d'une connexion Internet à des malades, le CE va inventer le SP de fourniture d'accès à Internet à des malades, et va donc le rendre co-contractant et participant au SP, faisant du contrat un contrat administratif. \\
CE, 2 Mai 2016, CHRU Montpellier, une convention prévoyait la mise en relation des patients avec des entreprises de transport. Le CE considère que ce contrat n'est pas conclu pour les besoins du SP, et n'est pas administratif. \\
Concernant les contrats de louage de service. Ce sont des contrats de recrutement d'agents publics. TC, 25 mars 1996, Berkani, le juge distingue selon que l'agent travaille pour un SPA ou pour un SPIC. Dans le premier cas, c'est un contrat administratif, dans le deuxième cas, c'est un contrat de droit privé. \\
Concernant les contrats conclus par les SPIC avec leurs usagers sont toujours des contrats de droit privé, même si ils contiennent des clauses exorbitantes de droit commun. TC, 1962, Dame Bertrand. CE, 1923, De Robert Lafreygère ; CE, Sect., 1957, Jalonque de lagos. \\
Concernant la question des contrats conclus par un SPIC pour les besoins de son activité. Ils sont en principe de droit privé sauf dans trois hypothèses. Ils sont administratifs lorsque l'activité en cause relève de prérogative de puissance public (TC, 29 Décembre 2004, époux Blankeman). Ils sont administratifs lorsqu'ils contiennent une clause exorbitante de droit commun (TC, 2006, Caisse centrale de ré-assurance). 


Le contrat peut être administratif en vertu de l'exorbitance du contrat ou de son régime. \\
L'exorbitance est quand le contrat contient des clauses particulières. \\
Concernant les clauses exorbitantes de droit commun. Dégagée en CE, 1912, Granites porphayoites des Vosges, dans laquelle le CE définit le contrat d'administratif en fonction de ses clauses même si il ne parle pas des clauses exorbitantes de droit commun. Cette notion est très floue, elle a fait l'objet de deux définitions bien distinctes (aujourd'hui le droit positif parle de clauses administratives). Première définition: CE, 1950, Stein, une clause exorbitante, c'est une clause impossible en droit privée, prohibée dans un tel contrat. TC, 1999, Commune Bourisp, une clause impossible en droit privé est par exemple l'octroi d'une exonération, qui attribue des pouvoirs particuliers comme une résiliation unilatérale, ou dans l'affaire commune de Bourisp, un contrat ayant l'objet d'une cession d'une parcelle privée d'une commune à l'autre où une commune accordait des avantages aux habitants de l'autre commune. Difficulté de cette définition: le droit privé évolue, ce qui peut être exorbitant à un moment peut ne plus l'être par la suite, d'où la deuxième définition. \\
La deuxième définition parle d'une clause inhabituelle en droit privé. C'est la clause qui vient créer une forte inégalité entre les partis comme par exemple conférant des pouvoirs particuliers à l'administration, permettant par exemple un pouvoir de direction, de surveillance. Difficulté: si la clause est simplement inhabituelle en droit privée, elle n'est pas impossible. Il y a donc un vrai flou dans les définitions. Une clause n'est jamais exorbitante en elle même, pour déterminer si il y a inégalité entre les parties, il faut apprécier l'ensemble du contrat et l'ensemble des obligations contractuelles. Une clause exorbitante dans un contrat donné peut ne pas l'être dans un autre. \\
Aujourd'hui, arrêt de principe: TC, 2014, Axa IARD. La clause exorbitante est un pouvoir donné à l'administration, qui crée donc une inégalité au profit de la personne publique. Le juge pose aussi un deuxième critère, cette clause doit être stipulé dans l'intérêt général. Le juge ne parle plus de clauses exorbitantes mais parle d'un contrat qui relève d'un régime exorbitant. On parle aujourd'hui de clauses administratives. La clause administrative n'est donc plus celle qui est nécessairement exorbitant au droit commun, c'est une clause, apprécié au regard du contrat entier, qui crée une inégalité en faveur de l'administration au nom de l'intérêt général. Un exemple est le contrôle que peut effectuer la puissance publique sur son co-contractant. \\
Cette nouvelle définition ne facilite pas pour autant l'office du juge. Dans l'affaire Generim, un contrat de vente de parcelles privées de la ville de Marseille, pour la construction d'un hôtel de luxe. Il y avait des clauses particulières notamment l'obligation que l'hôtel soit tenu pendant 10 ans avant d'être vendu à un sous acquéreur. Ces contrats de ventes sont en principe de droit privé: le CE, le 10 Février 2016, n'a pas réussi à appliquer la définition donné par le juge des conflits et a donc renvoyé l'affaire devant le TC. TC, 4 Juillet 2016, le juge des conflits nous indique qu'il n'y a pas de clauses administratives dans ce contrat sans vraiment nous dire pourquoi. 


Le régime exorbitant de droit commun. \\
Ce n'est pas une clause, c'est un régime qui découle d'un texte, donc d'un élément extérieur au contrat. CE, Sect., 19 Janvier 1973, Société d'exploitation de la rivière du sant. Cette affaire concernait l'achat par EDF d'énergie, complété par un décret de 1955 qui n'existe plus aujourd'hui. EDF avait l'obligation de passer un contrat pour racheter l'énergie des producteurs autonomes. \\
On peut se demander si ce critère a encore sa place, car la loi détermine déjà ces contrats comme administratif. Que reste-t-il donc de cette jurisprudence ? Elle n'est plus du tout appliquée. Le juge se réfère encore à ce critère dans certaines décisions, notamment les contrats passés par les SPIC pour leurs besoins. Le juge ne l'applique cependant jamais en pratique. 

\chapter{Le régime du contrat administratif}

Les personnes publiques disposent comme les personnes privées de la liberté contractuelle. D'après le CC, cette liberté contractuelle de la personne publique est constitutionnel. CC, Loi relative au secteur de l'énergie. \\
Cette liberté a une portée très différente de celle des personnes privées dans la mesure où la personne publique est portée par l'objectif de l'intérêt général. Dans certains cas, la liberté contractuelle de la personne publique est plus restreinte, elles n'ont pas forcément le choix de la personne co-contractante, notamment avec la procédure de mise en concurrence. \\
À d'autres égards, notamment l'exécution du contrat administratif, la personne publique dispose d'une plus grande liberté. Ces prérogatives sont exercés dans l'intérêt général. 

\section{Le régime de la formation/passation du contrat administratif}

Ce régime est encadré par le droit de la concurrence que l'on appelle aussi le droit de la commande publique.

\subsection{Le droit de la commande publique}

C'est une matière à part entière, lourde et technique. Ce droit vient limiter le choix du co-contractant par la personne publique. Ce choix est limité par des règles procédurales: de publicité puis de mise en concurrence. \\
La publicité, c'est tout simplement l'information aux acteurs économiques qu'un contrat de commande publique va être passé. En pratique, des avis sont publiés, de manières locale ou nationale. \\
La mise en concurrence va imposer à la personne publique de choisir son co-contractant selon des critères posés, des critères de prix, des critères qualitatifs. La personne publique devra choisir le co-contractant qui remplit le mieux les critères. 


Ces procédures de mise en concurrence visent un triple objectif. \\
Le premier est un objectif financier pour la personne publique: garantir la meilleure utilisation des deniers publiques, garantir que le contrat est passé au meilleur coût et pour la meilleure offre. \\
Le deuxième tient à éviter la corruption. \\
Le troisième objectif résulte essentiellement du droit de l'UE. Toutes les procédures viennent de directives, et visent à établir une égal concurrence entre les opérateurs qui souhaiteraient contracté avec la personne publique. 


Les contrats de la commande publique sont des contrats par lesquels une personne publique vient satisfaire ses besoins. Il existe deux types de contrats: ceux de marché public et ceux de concession (qui englobe les anciennes délégations de service publique). \\
Ces deux catégories sont encadrés par des procédures plus ou moins contraignantes. Très contraignantes en matière de marché public car la personne publique doit fixer par avance des critères de sélection mais aussi pondérer et hiérarchiser ces critères. Il n'y pas, en principe, de négociation possible et de choix de co-contractant. \\
À l'inverse, les contrats de concession sont plus souples à passer, il existe un principe de libre choix du co-contractant (CE, 23 Mai 2008, Musée Rodin), ce qui signifie que la personne publique dispose d'une marge de manoeuvre plus grande. Reste que la passation de ces concessions sont de plus en plus encadrée: il y a là aussi des critères à dégager et à pondérer. \\
La détermination du contrat est donc déterminante. 


Il y a existé depuis longtemps un droit de la commande publique en droit interne. Depuis les années 80, cette matière est entièrement régie par le droit de l'UE via des directives. Les directives actuellement applicables sont des directives de 2014. Une ordonnance du 23 Juillet 2015 vient transposer ces directives, qui abroge d'ailleurs l'ancien code des marchés publics. \\
Ces textes seront sûrement bientôt codifiés dans un nouveau code des marchés publics. 


Toutes ces règles viennent à s'appliquer au dessus de certains seuils, posés par l'UE. Ces procédures s'appliquent aux plus gros contrat de la commande publique: plus de 100 000 euros pour l'État en fournitures et 5 millions d'euros en travaux. \\
Cependant, des principes ont étés dégagés par le juge: libre accès à la commande publique, égalité de traitement des candidats, transparence de la procédure: CE, Ass., 23 Février, ATNMP. Ces principes s'appliquent à tout contrats de la commande publique et demande des procédures minimales de mises en concurrence. 

\subsection{La question des contrats de la commande publique}

Il faut distinguer ces contrats, des marchés des concessions mais aussi les autres contrats administratifs.

\subsubsection{La définition des contrats de la commande publique}

Résulte de l'article 3 de l'ordonnance de 2015 qui fixe des critères pour reconnaître le contrat de la commande publique: le premier est que le contrat est un contrat écrit ; le deuxième est que le contrat est passé entre une personne publique et un opérateur économique (celui qui exerce une activité économique), qui peut être une personne publique (EPIC) ou privée ; le troisième est que le contrat porte soit sur la fourniture de bien soit sur la prestation de services soit sur des travaux, dont conclus pour les besoins de la personne publique ; le quatrième est que le contrat est conclu selon le versement d'un prix à l'opérateur économique, il est fixe et déterminé à l'avance. \\
Le contrat de la concession a pour variante l'objet du contrat, il peut avoir pour objet soit la prestation de service soit la réalisation de travaux. Il ne peut donc pas y avoir de fournitures de biens. La dernière différence tient au fait que en matière de concession le co-contractant supporte un risque économique, sa rémunération est donc liée à l'exploitation. \\
Il a longtemps existé une troisième catégorie de la commande publique, qui étaient des contrats de partenariat, mis en place par une ordonnance de Juin 2004. C'était des contrats globaux par lesquels on confiait aux co-contractants la réalisation de travaux ou d'ouvrage public, à qui on confiait aussi l'entretien. La personne publique versait des loyers au co-contractants, permettant d'étaler la dépense. Ces contrats ont étés requalifiés en marché public. 


Les règles de passation ont un coût pour la personne publique, mais aussi un coût contentieux. Si une procédure est mal réalisé, c'est le double que risque de payer une personne publique.

\subsubsection{La distinction des contrats de la commande publique}

Il faut distinguer les contrats de la commande publique entre eux: marché public ; concession. Puis avec les autres contrats administratifs. \\
La qualification du contrat va déterminer la nature de la passation. Les personnes publiques ont une tendance à vouloir se soustraire aux procédures. Le juge fait donc un travail de requalification des contrats. 


1. Distinction entre les contrats de la commande publique (marché/concession). \\
Ces contrats peuvent avoir les mêmes parties ou les mêmes objets, et peuvent porter les deux sur les services publics. C'est donc le critère financier qui permet de distinguer les contrats. Ce critère a largement évolué depuis les années 60. Il était alors envisagé de matière organique, en prenant en compte l'origine de la rémunération. Soit le co-contractant est rémunéré par un prix versé la personne publique, ce qui en fait nécessairement un contrat de marché public, soit par les usagers du service, alors le contrat était qualifié à l'époque de délégation de service public. Ce critère n'est plus pris en compte. \\
Aujourd'hui, on se concentre sur la nature de la rémunération: CE, 15 Avril 1996, Préfet des bouches du Rhône, définit le contrat de concession de service public: le co-contractant est rémunéré substantiellement aux résultats d'exploitation. Critère repris par le législateur par la loi Mursef en 2001. CE, 30 Juin 1999, SMITOM, le CE explique que la rémunération peut être substantiellement lié aux résultats sans être majoritairement le cas comme les rémunérations mixtes. \\
CE, 7 Novembre 2008, Département de la Vendée, arrêt de principe aujourd'hui: le CE nous indique ce qu'est une rémunération substantiellement lié aux résultats d'exploitation: une part significative du risque d'exploitation supporté par le co-contractant. \\
CE, 5 Juin 2009, Avenance, concernant un contrat de restauration scolaire. Ces contrats étaient considérés depuis longtemps comme des contrats de marché public, c'était les usagers qui payent l'exploitant. Le CE a considéré que le co-contractant ne supportait aucun risque d'exploitation dès lors que le nombre d'usagers ne varient pas d'une année sur l'autre. Les usagers sont "captifs" et seront toujours le même nombre. La solution a été la même concernant des contrats de fourniture d'eau potable. \\
Concernant les contrats de mobilier urbain, ils sont souvent fournis gratuitement par le co-contractant à la personne publique et ont l'autorisation d'exploiter la publicité. CE, 2005, Decaux, la personne publique ne payant pas, ils sont quand même qualifiés de contrat de marché public. 


2. Distinction des contrats de la commande publique avec les autres contrats administratifs. \\
D'une part les contrats de subvention, d'autre part les contrats d'occupation. \\
Les contrats de subvention sont réalisés par contrat lorsque la subvention dépasse 23 000 euros. On peut se demander si c'est un contrat de subvention ou un contrat de marché public. Le juge a posé un faisceau d'indices pour le déterminer. Il prend en compte trois indices: l'initiative du projet, si c'est la personne publique, c'est un contrat de commande publique ; le juge peut prendre en compte le fait que l'opérateur exécute une prestation de services qui peut répondre au besoin de la personne publique, ce qui qualifierait le contrat de commande publique ; le lien qui existe entre la subvention et la prestation réalisée, si il y a un lien direct, c'est un contrat de commande publique.  \\
CE, 23 Mai 2011, Commune de Six-Fours les plages, concernant l'organisation d'un festival de musique. Pour l'organiser, la commune avait confiée l'organisation à un opérateur privée, conclue sans aucune publicité ni mise en concurrence. Ce contrat va être requalifier en contrat de marché public de services. Pour le CE, ce n'est pas une DSP pour une raison simple: pour qu'il y ait délégation, il faut qu'il y ait service public. En l'espèce, il n'y avait pas de contrôle de la personne publique sur le festival. Pour requalifier, le CE note que c'est la commune qui a eu l'initiative d'organiser le festival, pour les besoins culturels de la commune et le juge va enfin noter le lien réel entre la subvention et le coût réel du festival. \\
Définition législative du contrat de subvention depuis 2014, Loi relative à l'économie sociale et solidaire, venant modifier une loi d'Avril 2000. Cette définition reprend la jurisprudence. \\
Distinction entre commande publique et occupation domaniale. Typiquement, ce sont des contrats pour des terrasses de café. CE, 15 Mai 2013, Ville de Paris: un contrat qualifié d'occupation du domaine public, pour l'installation de colonnes Maurice. Dans le contrat litigieux, la Ville de Paris autorisait l'occupation à titre gracieux et en contrepartie, la société Decaux prenait en charge l'installation des colonnes. En 2005, le CE avait qualifié ce genre de contrat de marché public. Le TA de Paris considère qu'il s'agit d'une DSP, la CAA considère que c'est un marché public, le CE vient dire que c'est une simple convention domaniale. Le CE rejette la qualification de SP, et rejette la qualification de DSP. Pour le CE, ce n'est pas un marché public car le contrat n'est pas conclu pour les besoins de la personne publique. Ensuite, le juge considère qu'il n'y a aucun prix versé par la personne publique. Ensuite, le CE considère que le contrat ne porte pas sur un SP car il n'existe pas, donc ce n'est pas une DSP. 

\section{Le régime de l'exécution du contrat administratif}

La personne publique a des prérogatives très exorbitantes lui permettant d'imposer ses vues au co-contractant. Elles ne sont pas sans contreparties, car le co-contractant dispose d'un droit particulier, notamment celui d'un droit à l'équilibre financier. \\
Le régime de l'exécution a été forgé par le JA au début du XXe siècle. Il a dégagé des règles générales applicables aux contrats administratifs. Ce sont donc des règles jurisprudentielles, elles n'ont pas la valeurs de PGD. C'est pour cette raison qu'il est toujours possible de déroger par des clauses à ces exorbitances. Aujourd'hui, le régime du contrat est déterminé surtout par le contrat lui même. 

\subsection{Les prérogatives exorbitantes de l'administration}

\subsubsection{Un pouvoir de contrôle et de modification unilatérale}

1. L'étendu du pouvoir \\
L'Administration peut surveiller mais aussi contrôler et diriger l'exécution du contrat. Selon ce pouvoir, l'administration pourra donner des ordres de service que son co-contractant devra exécuter. CE, 22 Février 1952, Société pour l'exploitation des procédés ingrand. \\
L'administration peut modifier unilatéralement le contrat, pendant son exécution. Il le fait dans l'intérêt général, et dans le but de la mutabilité et de la continuité du SP. CE, 10 Janvier 1902, Gaz Deville-les-rouens, qui concernait des éclairages publics qui utilisaient le gaz. L'administration a pu imposé à ses concessionnaires d'utiliser l'électrique en modifiant unilatéralement le contrat. \\
CE, 21 Mars 1910, Compagnie générale française des tramways, qui concernait une exploitation de tramway. Le juge a dit que la ville pouvait imposer de créer de nouvelles lignes de tramways alors que ce n'était pas prévu à la base. \\
Trois conditions pour cette modification: qu'en vue de l'intérêt général d'abord, qu'en vue d'adapter le service à une condition nouvelle ensuite, et enfin, un avantage financier pour le co-contractant (théorie du fait du prince). La modification ne peut d'ailleurs jamais résulter en une modification des avantages financiers.


2. La limitation par le droit de la commande publique \\
Depuis quelques années, il est considéré qu'un changement trop important du contrat fait qu'il est un nouveau contrat qui doit respecter de nouveau la procédure. CJUE, 2008, Pressetext, il faut passer une nouvelle convention en cas de modification substantielle. Le juge dégage quatre hypothèses. \\
La première est l'introduction de nouvelles conditions. \\
La deuxième est une hypothèse de l'extension du contrat à des services non prévus initialement. \\
La troisième est la modification de l'équilibre économique du contrat, e faveur du co-contractant de l'administration. \\
Enfin, la quatrième est la substitution d'un nouveau co-contractant au co-contractant initial. C'est une hypothèse de cession du contrat. \\
Ces hypothèses ont étés codifiés via les directives de 2015 et les ordonnances qui ont suivies. \\
Le CE a tendance à appliquer très souplement ces hypothèses. CE, Sect., 11 juillet 2008, Ville de Paris, concernant le service de Vélib', qui a été étendu de la ville à la couronne. Le CE a déclaré que cette extension n'a pas besoin d'un nouveau contrat. 

\subsubsection{Le pouvoir de sanction unilatéral}

Dégagé par CE, 31 Mai 1907, Delplanque. Ce pouvoir permet à l'administration de sanctionner son co-contractant si il ne remplit pas ses devoirs contractuels, il n'y a pas besoin d'un juge. \\
Divers types de sanctions peuvent être prise pour garantir la continuité du SP. La première sanction qu'elle peut prendre est la sanction financière. Elles peuvent être aussi coercitives pour permettre la réalisation du contrat par la contrainte, l'administration peut se substituer elle même à son co-contractant ou un tiers et ce, aux frais du co-contractant. Ces solutions coercitives nécessitent une faute grave du co-contractant et ne met pas fin au contrat. \\
Un troisième type de sanctions est la résiliation unilatérale du contrat par l'administration qui peut le faire en cas de faute grave du co-contractant, sauf dans une concession de SP, où seul le juge peut le faire. \\
L'administration, avant de prendre des sanctions, doit mettre en demeure son co-contractant fautif. Le juge exerce un contrôle maximum sur ces sanctions et elle doit être proportionné par rapport à la faute. 

\subsubsection{Le pouvoir de résiliation unilatéral dans l'intérêt général}

Il existe d'autres hypothèses de résiliation du contrat, comme celle ci-dessus. Il existe une résiliation en cas de force majeure. \\
En cas d'intérêt général, il y a des conditions et des conséquences à cette résiliation. \\
D'abord il faut remarquer que ce pouvoir peut être réaffirmé par une stipulation du contrat. Ces clauses sont souvent qualifiées de clauses administratives. Ensuite, ce pouvoir a fortement évolué sous l'influence de la jurisprudence administrative qui est venu l'encadrer et le préciser. Ensuite, cette évolution tient à ce que ce pouvoir peut être attribué au co-contractant. 


1. Les conditions de la résiliation dans l'intérêt général. \\
Admis par CE, Ass., 2 Mai 1958, Distillerie Magnac-Laval, qui précise quels sont les motifs d'intérêt général qui peuvent justifier une telle résiliation. Deux motifs sont possibles: l'abandon du service, l'exploitation de ce service sur de nouvelles bases. L'administration doit dédommager son co-contractant. \\
CE, 19 Janvier 2011, Commune de Limoges, où était en cause un contrat d'occupation du domaine public. L'opérateur exploitait un hôtel et un restaurant que la commune voulait exploiter sous un service public, donc sur de nouvelles bases, et a donc pu résilier la convention. \\
En principe, l'administration ne peut pas renoncer à un tel pouvoir (CE, 6 Mai 1985, Association Eurolat). Cependant, en pratique, de plus en plus de conventions ont une clause où les pouvoirs publics renoncent à ce pouvoir. \\
Si le contrat est formé entre deux personnes publiques, les deux peuvent profiter du pouvoir: CE, 4 Juin 2014, Commune d'Aubigny-les-potheres. Le juge semble poser des conditions plus rigoureuses dans ce cas là: CE, 27 Février 2015, Commune de Béziers 3. Le juge nous explique que dans de tels contrats, la résiliation n'est possible que dans l'intérêt général, donc ne serait possible ni pour fautes ni pour cas de fautes majeures. La résiliation pour intérêt général ne peut être fait que dans deux hypothèses: qu'en cas de disparition de la cause du contrat, ou en cas de bouleversement de l'équilibre (donc modification substantielle du contrat ou imprévision) de la convention. 


2. Les conséquences de la résiliation dans l'intérêt général. \\
Trois conséquences.


a. La première est l'indemnisation du co-contractant. En cas de résiliation unilatéral du contrat, l'administration doit indemniser son co-contractant. Le juge a admit que le contrat peut prévoir cette indemnisation: CE, 7 Mai 1952, Eloy. L'indemnisation doit prévoir les dépenses exposés, mais aussi de la privation de gain du co-contractant. Ceci étant, en 2011, le CE a encadré cela: CE, 4 Mai 2011, CCI de Nîmes, le juge distingue deux situations: suivant que le co-contractant soit privé ou public ; si privé, le CE considère que le contrat peut prévoir une indemnisation inférieur au préjudice subit (en 2012, il disait même que le contrait pouvait prévoir aucune indemnisation, CE, 19 Décembre 2012, AB Trans) ; si personne publique, l'indemnisation ne pourra jamais être inférieur au préjudice, c'est l'application d'un PGD selon lequel une personne publique ne peut jamais céder à des libéralités (CE, Sect., 19 Mars 1971, Mergui). \\
Cela n'est probablement plus conforme au droit de la commande publique, qui impose le dédommagement en cas de résiliation pour intérêt général.


b. La deuxième est la reprise de contrats passés avec des tiers. Le juge a considéré qu'en cas de résiliation du contrat principal, la puissance publique a l'obligation de reprendre les contrats conclus avec les tiers. La personne publique se substitue donc à son co-contractant. \\
Une limite: les engagement anormaux, la personne publique ne reprendra pas des contrats qui contiennent de telles clauses. 


c. Concernant le contentieux. En cas de résiliation, si elle était illégale, le co-contractant ne peut prétendre qu'à être indemnisé. Cela a fortement évolué. CE, Sect., 2011, Commune de Béziers 2, en cas de résiliation illégal, le co-contractant peut non seulement demander à être indemnisé mais il peut aussi demander la reprise des relations contractuelles. 

\subsection{Les droits et obligations des co-contractants}

\subsubsection{Les obligations du co-contractant}

Il a bien sûr l'obligation d'exécuter le contrat, mais aussi de l'exécuter de manière personnelle. Cela limite le fait de céder le contrat à un tiers. Il ne peut le faire qu'avec l'autorisation de l'administration: CE, 1905, compagnie départementale des eaux. Cela limite les sous traitances, ceux-ci doivent être agréer par l'administration. \\
Il a l'obligation d'exécuter le contrat alors même que l'administration ne s'est pas exécuté: CE, 7 juin 1929, Compagnie française des câbles télégraphiques. Cela prohibe l'exception d'inexécution. Cela est fondé aussi sur la continuité du SP. Une seule limite: la force majeure. Il peut simplement demander à l'administration de résilier le contrat, le co-contractant pourra saisir le JA pour demander une telle résiliation. \\
Jean de Soto, dans une note, mettait en avant que cette solution n'est pas nécessairement justifié. Rien n'interdirai d'admette l'exception d'inexécution. \\
Terneyre a plaidoyer pour cette exception dans les contrats administratifs. Le CE a entendu ses critiques, 75 ans après de Soto. CE, 8 octobre 2014, Gremke location. Le CE va, pour la première fois, admettre les clauses de résiliation du co-contractant si deux conditions sont réunies: le contrat ne doit pas porter sur l'exécution d'un SP, afin de garantir la continuité du SP ; le co-contractant ne pourra pas directement résilier le contrat en application de la clause, il doit informer la personne publique de sa volonté, la personne publique peut s'y opposer pour un motif d'intérêt général. Le juge ne précise pas le degré de la faute pour que le co-contractant résilient, et, de toute façon, si le co-contractant se voit opposer un refus, il peut aller devant le JA. Cependant, ce n'est pas une si grande innovation. Deux critiques: en pratique, l'administration s'opposera systématiquement à cette résiliation, donc quoi qu'il arrive, il faudra aller devant le juge ; la deuxième critique est de se demander pourquoi le juge se limite aux contrats non relatifs au SP, si l'administration peut s'opposer pour un motif d'intérêt général, alors l'administration peut s'opposer à une résiliation de SP. 

\subsubsection{Le droit du co-contractant à l'équilibre financier de la convention}

Si l'administration commet une faute ou ne s'exécute pas, l'administration engage sa responsabilité contractuelle. 


1. La théorie de la force majeure. \\
Se définit par trois éléments: indépendant de la volonté des parties, imprévues ou imprévisibles, rend impossible l'exécution du contrat. La force majeure libère le co-contractant de sa responsabilité, tant le co-contractant privée, que la personne publique.


2. La théorie du fait du prince. \\
C'est ce qu'on appelle aussi l'aléa administratif. C'est l'hypothèse dans laquelle la personne publique vient aggraver les conditions d'exécution du contrat en modifiant unilatéralement le contrat, le co-contractant a droit à une indemnisation. CE, 20 Octobre 1971, Compagnie des chemins de fer de Bayonne à Biarritz. \\
Cette théorie ne fonctionne pas systématiquement. Marche seulement si c'est la personne publique co-contractante qui aggrave le contrat. À l'inverse, si c'est une autre personne publique qui aggrave le contrat, la théorie du fait du prince ne trouve pas à s'appliquer, mais l'imprévision le pourrait: CE, Sect., 5 Novembre 1962, Produpétrol. \\
L'aggravation peut aussi résulter d'une mesure générale comme un règlement de police, donc une mesure générale étrangère au contrat. Dans ce cas, cette théorie peut s'appliquer. 


3. La théorie de l'imprévision ou l'aléa économique. \\
Cela implique un bouleversement économique du contrat. Il est toujours possible mais devient économiquement désastreuse. CE, 24 Mars 1916, compagnie du Gaz de Bordeaux, le juge a admit que la montée des prix dues à la guerre était une imprévision et a demandé à la personne publique d'indemniser son co-contractant. \\
L'imprévision est un élément imprévu, on peut penser à une catastrophe climatique, ensuite c'est un élément extérieur et indépendant aux volontés des parties, enfin il faut qu'il y ait un bouleversement de l'économie du contrat. \\
Si une imprévision apparaît, une période extra-contractuelle apparaît, pendant laquelle le contrat est toujours exécuté et où l'administration doit subvenir à l'équilibre économique du contrat en indemnisant le co-contractant. \\
Quatre remarques: les fondements n'ont jamais été explicités par le JA. L'indemnité d'imprévision ne sera pas, la plupart du temps, entièrement supporté par l'administration, le JA laissant au co-contractant une charge de l'imprévision. L'imprévision est toujours temporaire, mais si le bouleversement économique devient définitif, le JA considère qu'il y a force majeur, chacune des parties pouvant résilier: CE, 9 Décembre 1932, Compagnie des tramways de Cherbourg ; CE, 2014, Commune de Staffelfeldel. Le juge n'applique quasiment plus cette théorie car les contrats eux même envisagent l'imprévision, en prévoyant des clauses de variation de prix, et cette théorie ne peut marcher qu'en cas d'absence de clauses. Cette théorie marque l'influence du JA sur le juge judiciaire car cette solution a été reprise progressivement en droit privé, par la C.Cas d'abord en 2000 et prévu ensuite en droit des contrats. 

\chapter{Le contentieux des contrats administratifs}

Le contentieux peut avoir plusieurs objets. Le recours peut demander une annulation du contrat, une interprétation, ou encore l'exécution du contrat. On va s'intéresser surtout à l'annulation. Ce contentieux relève de la compétence du JA, et, en principe, le contentieux contractuel ne peut pas faire l'objet d'un arbitrage en vertu d'un principe qui interdit l'arbitrage pour les personnes publiques, il peut y avoir quelques exceptions. \\
La problématique va tourner autour de la recevabilité des recours. Cette problématique ne pose pas de problème pour les parties au contrat. Il s'agit d'un recours de plein contentieux. En revanche, les tiers au contrat ne disposaient pendant longtemps, d'aucuns recours. Concernant des contrats de la commande publique, ces dernières années, un problème s'est posé: agir contre un contrat de commande publique qui nous a évincés, si c'est fait illégalement. \\
Depuis une dizaine d'années, de nouveaux recours ont étés mis en place pour les tiers aux contrats, qui disposent aujourd'hui de plusieurs voies de recours.

\section{Contentieux des tiers au contrat}

Ce peut être dans certains cas un recours pour excès de pouvoir. Depuis quelques années se développent aussi un plein contentieux des tiers aux contrats. Un REP est possible contre un acte détachable du contrat. Un REP est aussi possible contre certains contrats.

\subsection{REP contre les actes détachables}

C'est une solution traditionnelle pour contester un acte détachable d'un contrat et non le contrat lui même. Cet acte est un acte extérieur au contrat mais qui a un lien direct avec le contrat. Celui-ci a été ouvert par CE, 11 Novembre 1903, Commune de Gorre, pour les parties du contrat. Ouvert ensuite aux tiers: CE, 11 Août 1905, Martin. \\
Le recours est admis contre un acte détachable du contrat, peu importe la nature du contrat. Le contrat peut aussi être de droit privé, l'acte étant administratif: CE, 22 Juillet 1977, Centieri Navali Santa Maria. 

\subsubsection{Les conditions de recevabilité}

Ces conditions propres à ce recours, on peut en dénombrer quatre. \\
La première concerne les moyens invoqués qui concernent soit l'acte détachable soit le contrat lui même. \\
La deuxième est de savoir quels sont les actes contestables, ce sont souvent des actes qui précèdent la conclusion du contrat. Ces actes antérieurs peuvent être contestés par tous, les parties comme les tiers. Pour les actes postérieurs, on entend les mesures d'exécution du contrat: mesure de résiliation unilatéral, etc, ils peuvent aussi être attaqués par des tiers mais pas par les parties: CE, Ass., 2 Février 1987, Société TV6. \\
La troisième condition est que le tiers doit montrer que l'acte en question l'a lésée de manière directe et certaine: CE, 11 Mai 2011, Société Lyonnaise des eaux. \\
La quatrième condition (CE, Département Tarn-et-Garonne) donne une condition de non recours en pleine juridiction: si il y a un tel recours en cours, le REP est fermé. Le contentieux des actes détachables peut être engagé par le préfet, qui peut toujours faire un REP contre un tel acte. Il est toujours possible de contester en REP les actes de contrat de droit privé. 

\subsubsection{Les effets d'annulation des actes détachable}

Quand les tiers font un tel recours, ils cherchent à faire tomber la convention et donc à la faire annuler. Cependant, l'annulation d'un acte détachable n'a aucun effet direct sur le contrat. Seul le juge du contrat dispose du pouvoir d'annulation, le juge du REP ne peut donc pas annuler un contrat. Si l'acte détachable est annulé, le juge du REP peut enjoindre à la pers. publique soit de mettre fin au contrat soit de saisir le juge du contrat afin qu'il prononce effectivement l'annulation de la convention. Pour cela, il faut que le tiers en fasse la demande, solution permise depuis CE, Sect., 7 Octobre 1994, époux Lopez. \\
Le juge qui prononce l'injonction est le juge de l'exécution. Le juge d'exécution dispose depuis 2011 d'une pluralité de possibilités. Depuis 2011, l'annulation de l'acte détachable ne remettra pas forcément en cause le contrat. CE, 24 Février 2011, Ophrys, offre au juge de l'exécution une palette de pouvoirs qu'il pourra moduler selon les circonstances de l'affaire. \\
Le juge de l'exécution peut décider la poursuite de l'exécution du contrat alors même qu'il est irrégulier, sous réserve de régularisation prise par la personne publique. Il peut décider de la résiliation du contrat, son annulation non rétroactive. Il peut décider de la résolution du contrat (une annulation rétroactive), qui ne peut se prononcer qu'en cas d'irrégularité très grave. Le juge ne peut décider de cette résolution que si l'intérêt général ne s'y oppose pas. CE, 29 Décembre 2014, Commune D'Uchaux. \\
C'était un contentieux souvent très long et souvent aléatoire pour le requérant. 

\subsection{Le REP contre le contrat}

Le juge a dégagé deux REP contre certains contrats: la possibilité de contester les clauses réglementaires ; contre certains contrats de recrutement d'agents publics non titulaire. 

\subsubsection{REP contre les clauses réglementaires}

CE, Ass., 10 Juillet 1996, Cayzeele, ouvre ce REP. \\
Une clause réglementaire est une clause qui se rapporte à l'organisation du service public, qui a donc vocation à s'appliquer à des tiers au contrat. Par exemple, une clause tarifaire du service public. \\
S'est posée la question si ce REP pouvait être formé contre toutes les clauses réglementaires ou uniquement celles qui peuvent être considérés comme divisible du contrat. En effet, certaines clauses réglementaires concernent l'objet même du contrat. Si le juge annulerai une telle clause, il devrai annuler l'ensemble du contrat. CE, Sect., 8 Avril 2009, Association Alcaly, dans laquelle le juge considère que les clauses réglementaires sont par nature divisible du reste de la convention. Cela signifie que tous tiers peut contester de telles clauses. Certains contrats sont entièrement réglementaires comme ceux conclus entre l'État et les entreprises publiques.  

\subsubsection{REP contre les contrats de recrutement d'agents publics non titulaires}

CE, Sect., 30 Octobre 1998, Ville de Lisieux, ouvre aux tiers la possibilité de contester en REP de tels contrats. Un agent public non titulaire est celui qui n'est pas fonctionnaire. \\
Le juge fonde sa solution, "eu égard à la nature particulière des liens entre la collectivité et l'agent", on peut se demander ce que le juge veut dire par nature particulière. On peut penser à  la participation au service public du contractuel. \\
La question s'est posée il y a deux ans de savoir si ce REP était toujours possible, car le plein contentieux s'est ouvert aux tiers. Le juge a répondu que le REP est toujours ouvert: CE, 2 Février 2015, Commune d'Aix en Provence.


\subsection{Les recours de pleine juridiction contre le contrat}

La loi et le juge ont développés beaucoup de nouveaux recours. On est passé d'un état de pénurie à un état d'abondance en matière de contentieux contractuel ouvert aux tiers au contrat. \\
Le premier a été créé dans les années 90 par le législateur sous l'influence de l'UE: le référé pré-contractuel, ouvert avant la conclusion du contrat. \\
En 2007, le CE a créé une action en contestation de la validité du contrat et a été étendu en 2014. \\
En 2014, le législateur a mis en place un référé contractuel, ouvert après la conclusion du contrat. \\
Pendant longtemps, le déféré préfectoral était spécifique, aujourd'hui, il est de droit commun.

\subsubsection{Le référé pré-contractuel}

Il concerne les manquements au droit de la commande publique. Il permet de contester la publicité et les procédures de mise en concurrence. C'est une directive de l'UE en 1989 qui l'a mis en place. Cette directive demandait la possibilité de mettre en oeuvre une procédure rapide avant que le contrat n'ait été conclu. \\
Ce référé, on le trouve aujourd'hui dans le CJA aux articles L551-1 et suivants. Ils ont étés modifiés à plusieurs reprises.


1. Les conditions de recevabilité \\
La première de ces conditions tient au moment du recours: avant la signature du contrat. Le législateur a posé d'autres conditions spécifiques, notamment le requérant et les moyens invocables.


Les personnes ayant qualité pour agir, dans l'article L551-1, le CJA évoque deux catégories de requérants, l'État et son représentant (le préfet), ou par toutes personnes qui avait un intérêt à conclure le contrat et qui a été lésé par un manquement dans la procédure de passation. \\
Concernant l'intérêt à conclure le contrat, on s'est demandé si le requérant devait avoir participé à la procédure de passation, le juge ne l'admet pas: CE, 3 Novembre 1995, District de l'agglomération Nancéennes. Cette solution s'explique que parfois, il n'y a aucune publicité et donc l'opérateur lésé peut être empêché de participer à la procédure. Après 1995, le CE a précisé comme un opérateur ayant été évincé: CE, avis, 2012, Société Gouelle. Une personne qui n'a pas participé à la procédure malgré qu'elle ait pu le faire, a un intérêt à agir, mais n'a pas été lésé, son recours serait donc irrecevable: CE, 23 Juillet 2012, Commune de Villefranche sur mer. \\
Concernant le requérant qui doit avoir été lésé. Cette sous condition n'était pas contrôlée par le juge pendant très longtemps. C'est ce qui résultait de CE, 16 Octobre 2000, société Stéreau. Cela a eu deux conséquences: le juge ouvrait le recours en cause et a été rapidement encombré par ces procédures ; la deuxième difficulté portait à ce que certains vices plutôt mineurs, qui n'avaient pas affecté le requérant, pouvait conduire à l'annulation de la passation. Arrêt de principe: CE, Sect., 3 Octobre 2008, SMIRGEOMES, le juge du pré-contractuel contrôle la lésion du requérant. Le juge va donc contrôler la portée du manquement, un simple vice de forme mineur ne peut donc pas être considéré comme ayant lésé un opérateur. \\
Il n'est plus possible d'invoquer les manquements à un stade de la procédure où ils n'ont plus d'effets, comme un manque de publicité alors que le candidat est effectivement candidat. Le juge admet que la lésion ne puisse être que potentiel. Une telle lésion peut être soit directe, soit indirecte. Suite à cet arrêt, les référés pré-contractuels ont largement diminués, mais il a permis aussi de sécuriser les procédures de passation. 


Concernant les moyens invocables. \\
En principe, selon la loi, ce sont seulement les manquements à la publicité et à la mise en concurrence. Cela a pour conséquence que seuls les contrats devant respecter cela sont attaquables. Pendant longtemps, c'était les seuls moyens invocables. Le JA ne contrôlait pas la compétence de la personne publique (soit qui passait, soit qui candidatait). Cela résultait de CE, 21 Juin 2000, Syndicat intercommunal de la côte d'amour. \\
L'incompétence devrait être un moyen relevé d'office, hors, le JA refusait d'admettre un tel moyen. La première évolution apparaît en 1999, quand le JA accepte de contrôler la mise en concurrence mais aussi la violation des règles de la concurrence (qui prohibe les ententes et les abus de position dominantes), sous conditions que ce manquement soit imputable à la personne publique responsable de la passation du contrat. CE, 2 Juillet 1999, Société Bouygues. \\
CE, 18 Septembre 2015, CNAM des Pays de Loire. Dans cette décision, le juge du référé pré-contractuel va contrôler la compétence de la personne publique candidate. Il s'agissait en l'espèce d'un établissement public, limité par le principe de spécialité. Si le juge doit contrôler une collectivité dans un cas semblable, il contrôlera l'intérêt local à agir: CE, 30 Décembre 2014, Armor SNC. 


Concernant les délais. \\
La procédure n'est ouverte qu'avant la conclusion du contrat. Dès lors que le contrat est signé, le juge perd tous ses pouvoirs. Les personnes publiques se sont parfois lancés dans des courses à la conclusion du contrat, dans le but de fermé la voie du recours pré-contractuel. Il était fréquent que les contrats soient signés le jour même des attributions. \\
Le législateur a mis en place deux limites: il y a une suspension de la signature lorsqu'il y a un recours. Cette suspension est codifié dans l'article L551-4 du CJA. \\
Le législateur a mis en place une deuxième solution: les clauses de stand still. C'est un délai imposé entre la notification d'attribution du contrat et la signature du contrat. Cela permet de former un recours pré-contractuel. C'est en principe un délai de onze jours, qui s'étend parfois à seize jours. \\
Concernant les MAPA, les Marchés Aux Procédures Adaptés, ils sont formés via une procédure très légère et ne prévoient pas de notification d'attribution du contrat. Le concurrent évincé ne pourra pas généralement bénéficier du stand still, et donc le référé pré-contractuel est souvent fermé. 


2. Les pouvoirs du juge \\
C'est un juge qui doit statuer rapidement. C'est en principe un juge unique. Il statue en premier et dernier ressort (seul un pourvoi est possible). Concernant ces pouvoirs, le juge de référé pré-contractuel doit saisir dans les 20 jours suivant sa saisine. Il dispose de pouvoirs très important, qui dépassent très largement le pouvoir d'un juge des référés. Un tel juge statue généralement de manière provisoire, mais pas le référé du pré-contractuel qui dispose d'ailleurs des mêmes pouvoirs qu'un juge du plein contentieux. On dit qu'il statue en la forme des référés, cela veut dire qu'il n'a que la forme, il statue comme un juge du fond. \\
Si il constate un manquement, il peut ordonner à la personne publique de se conformer à ses obligations, suspendre la passation du contrat, ou suspendre encore un acte relatif à cette passation. Il peut aussi annuler un acte concernant la procédure de passation, de manière définitive. Il pourra encore, si il le souhaite, supprimer certaines clauses du contrat litigieux. \\
Il est à noter que concernant la suspension de la procédure. Depuis 2009, pour suspendre la procédure, le JA doit mettre en balance les conséquences négatives de la suspension avec les positives. Il doit effectuer un bilan, un contrôle de proportionnalité. Cela signifie qu'un manquement ne conduira pas systématiquement à une suspension. \\
Le juge peut prononcer des mesures d'office, alors que le requérant ne les a pas soutenu, codifié dans CJA, L551-12. 


\subsubsection{L'action en contestation de la validité du contrat}

C'est l'action qui va permettre à un tiers de demander l'annulation d'un contrat administratif. Ils ne disposaient pas d'une telle voie de droit depuis très longtemps. Avant 2007, il n'y avait que le référé pré-contractuel. \\
Cette voie a d'abord été ouverte par CE, Ass., 16 Juillet 2007, Tropic Travaux Signalisation. À ce moment, le recours n'est ouvert qu'aux concurrents évincés de la conclusion d'un contrat. Cela signifie que cela concernait que les contrats soumis à des règles de mises en concurrence. \\
Le recours a été étendu dans CE, Ass., 4 Avril 2014, Département du Tarn-et-Garonne. Le CE ouvre cette voie à tous les tiers au contrat, cependant, cette décision vient limiter la recevabilité du recours, car il est désormais plus encadré. \\
Ces deux décisions ont des conséquences sur les contentieux des actes détachables car dès lors qu'un tiers dispose d'un recours de pleine juridiction, ce tiers ne peut plus contester un acte détachable en REP. L'ouverture d'une pleine juridiction ferme le REP. 


1. La recevabilité du recours. \\
S'est posé pendant quelques années la question de quelle recevabilité mettre en place pour les tiers. L'arrêt tropic mets des conditions de recevabilité assez souple. Dans l'arrêt Tarn-et-Garonne, les conditions sont plus strictes: celles-ci doivent-elles s'appliquer à tous les tiers, y compris les concurrents évincés ? Ou ceux-ci disposent-ils des conditions plus souples ? CE, Sect., 5 Février 2016, Société voyage Guirette, vient harmoniser les conditions de recevabilité: le JA va considérer que tous les tiers, y compris les concurrents évincés sont soumis aux conditions de validité de l'arrêt Tarn-et-Garonne.


Concernant les requérants ayant intérêt à agir. \\
Sous l'empire de l'arrêt tropic, seuls les concurrents évincés, qui sont les mêmes personnes qui ont un intérêt à conclure le contrat en référé pré-contractuel. Par conséquent, le recours est possible même si il n'a pas participé à la procédure de passation: CE, 11 Avril 2012, Société Gouelle. \\
Le juge distingue deux catégories de tiers dans l'arrêt Tarn-et-Garonne, ceux qui seraient privilégiés, et ceux qui seraient intéressés (termes de la doctrine). Les privilégiés sont le préfet, les membres des assemblées locales (de l'opposition surtout), des organes délibérants. Les intéressés, ce sont les tiers qui ont étés lésés par un tel contrat, de façon directe et certaine par la conclusion du contrat. Ces tiers intéressés, ce sont d'abord les concurrents évincés. Outre le concurrent évincé, on peut penser au contribuable local, notamment un usager du service public qui pourrait vouloir contester un contrat relatif à l'exécution du SP. CAA Nancy, 12 Mai 2014, Communauté urbaine du grand Nancy. Autre tiers: les sous traitants du co-contractant: CE, 14 Octobre 2015, Région Réunion. 


Concernant les moyens invocables. \\
La jurisprudence a évoluée là aussi. Sous l'empire de l'arrêt tropic, tous les moyens étaient invocables. Le JA a encadré les moyens invocables. Les tiers privilégiés peuvent toujours invoqués tous moyens. À l'inverse, les tiers intéressés ne peuvent invoquer que certains moyens: ceux en rapport à la lésion dont ils se prévalent, qui concernent donc soit la procédure de passation du contrat soit le contrat lui même. Les moyens doivent démontrer la lésion. 


Concernant les contrats attaquables. \\
Sous l'empire de l'arrêt tropic, il s'agissait uniquement des contrats de la commande publique. Depuis Tarn-et-Garonne, le recours est ouvert à tous les contrats administratifs, y compris les conventions d'occupation d'espace public: CE, 2 Décembre 2015, Orange. 


Concernant les délais de recours. \\
C'est un délai de deux mois à compter de la publicité du contrat, c'est à dire l'avis d'attribution du contrat. La plupart du temps, cet avis est publié dans un journal local ou officiel d'attribution. La demande indemnitaire n'est pas concerné par ce délai, le délai étant dans ce cas là de quatre ans: CE, Avis contentieux, 11 Mai 2011, Société Révillon Schmitt Prévot. 


Concernant la portée dans le temps. \\
Les arrêts tropic et Tarn-et-Garonne opèrent un revirement de jurisprudence, qui sont donc en principe rétroactif. Pour la première fois, dans l'arrêt tropic, le juge module dans le temps l'effet de sa jurisprudence, pour des motifs de sécurité juridique. Il va considérer que ce nouveau recours ne portera que sur les contrats conclus après son arrêt, soit après le 4 Avril 2014. Il y a donc des difficultés d'application. \\
Concernant les contrats conclus avant le 16 Juillet 2007, ils ne peuvent pas faire l'objet d'un recours de pleine juridiction. \\
Concernant les contrats conclus entre 2007 et 2014 peuvent faire l'objet d'un recours tropic, soumis aux conditions de l'arrêt tropic. \\
Concernant les contrats conclus après le 4 Avril 2014, ils peuvent faire l'objet d'un recours Tarn-et-Garonne. 


2. Les pouvoirs du juge \\
Comme dans la plupart des contentieux contractuel, tout vice du contrat n'emportera pas nécessairement l'annulation du contrat. Le juge dispose de pouvoirs modulables, lui permettant d'adapter sa solution selon la situation. \\
Il pourra tout d'abord décider de poursuivre l'exécution du contrat alors même qu'il est vicié, sous des motifs d'intérêt général. \\
Il peut ensuite imposer des mesures de régularisation, qui seront imposés à la personne publique. \\
Il peut ensuite prononcer la résiliation de la convention, c'est à dire son annulation non rétro-active. \\
Il peut aussi prononcer la résolution du contrat (annulation rétroactive), qui ne peut être prononcé que dans les cas les plus graves, comme le vice du consentement, ou tout autre vice d'une particulière gravité, notamment relatif à l'objet du contrat. \\
La résiliation et la résolution ne sont possible qu'à condition que l'intérêt général ne s'y oppose pas, notamment à l'impératif de continuité du service public. Elles peuvent donc être prononcées avec un effet différé, le contrat pouvant être considérés comme annulé à partir d'une certaine date.


Le dernier pouvoir du juge est de pouvoir accorder des indemnités aux tiers, si ils sont subi un préjudice lié au contrat. Une action en responsabilité. \\
À noter qu'un tiers peut aller en référé-suspension ou le tiers peut demander la suspension du contrat en attente du jugement au fond. Il est soumis aux mêmes conditions de recevabilité. Il est à noter que le délai du juge est un délai long, alors que le contrat peut prendre effet immédiatement, d'où l'intérêt du référé suspension. 

\subsubsection{Le référé contractuel}

Il a été créé par l'ordonnance du 7 Mai 2009, codifié dans le CJA à l'article L551-13 et suivants. Procédure dont la mise en place était une obligation du droit de l'UE. Cette procédure vient aussi sanctionner les manquements aux obligations de publicité et de mise en concurrence. Une jurisprudence ne peut servir de transposition, c'est pourquoi cette procédure a été créée, même en présence du recours tropic. Le législateur a donc créé un recours supplémentaire post-contractuel, qui vient se cumuler avec les deux autres recours. Les requérants peuvent donc cumuler les recours. \\
Pour éviter des cumuls de recours, le JA n'a eu de cesse que de limiter la recevabilité du référé contractuel. De manière général, ce référé est complètement calqué sur le référé pré-contractuel. On a les mêmes requérants, les mêmes moyens invocables. Ce qui diffère, c'est le délai de recours mais aussi l'intérêt à agir.


Concernant le délai de recours. \\
On a une procédure post-contractuelle. Le délai varie selon que l'attribution du contrat a, ou non, été publié. Si il y a eu publication de l'attribution du contrat, le délai de recours est de 31 jours. À l'inverse, en l'absence de publication comme dans un MAPA, le délai de recours est de 6 mois à compter du lendemain de la signature. 


Le requérant doit avoir un intérêt particulier à agir. \\
La procédure n'est pas là pour palier les lacunes d'un requérant qui auraient oubliés de former un référé pré-contractuel. CE, Sect., 19 janvier 2011, Grand port maritime du Havre, cette décision vient limiter les requérants qui ont intérêt à agir en référé contractuel. \\
Le CJA nous explique qu'en principe, un requérant qui a formé un référé pré-contractuel n'est plus apte à former un référé contractuel par la suite. Il nous explique aussi que les requérants qui ont négligés de former un référé pré-contractuel, ne sont pas aptes à former un référé contractuel. \\
Deux catégories de requérants qui ont intérêt à agir: 1. ceux qui n'ont pas pu former de référé pré-contractuel soit parce que la personne publique n'a pas respecter la clause de stand still soit en l'absence de respect de la suspension automatique de la signature du contrat. 2. Les requérants qui ont formés un référé pré-contractuel et obtenu une décision non respectés par la personne publique, peuvent aller devant un juge du référé contractuel. 

\subsubsection{Le déféré préfectoral}

C'est le recours en justice du préfet, par lequel il demande au JA de contrôler un acte d'une collectivité. Ce recours a été mis en place en 1982 suite à la fin de la tutelle de l'État sur les collectivités. \\
Le déféré préfectoral peut porter sur un contrat administratif. Cela fait poser la question de la nature juridique d'un tel recours. Pour certains, ce déféré était considéré comme un REP: CE, Sect., 28 Juillet 1991, commune de Sainte-Marie. Cela permettait au juge du REP d'annuler une convention illégale. 


Le juge a évolué à deux reprises: il a requalifié le déféré en recours de pleine juridiction: CE, 23 Décembre 2011, Ministre de l'intérieur. \\
La deuxième évolution est dans l'affaire Tarn-et-Garonne, le JA fait rejoindre le déféré dans les recours de droit commun, où le préfet demeure un tiers privilégié.


1. La recevabilité \\
Deux questions: le champ du déferrement préfectoral, tous les contrats ? seulement certains ? la question se pose car les collectivités doivent envoyer certains contrats au préfet. Le préfet peut envoyer tous les contrats devant le JA: CE, 14 Mars 1997, département des alpes maritimes. \\
Deuxième question, concernant le déféré provoqué. C'est un déféré qui est demandé par un tiers. Un administré va demander au préfet de déférer un contrat. Cela était très courant avant 2007, car les tiers ne disposaient pas de voie de recours. En pratique, ce déféré n'a plus d'intérêt, et peut même être dangereux pour les tiers au contrat, pour deux raisons: à la demande de déféré provoqué, le préfet peut refuser, et le JA a considéré que ce refus n'était pas susceptible de recours (CE, 25 Janvier 1991, Brasseur) ; si le préfet accepte un tel déferrement puis refuse en cours d'instance, le CE a considéré que ce désistement n'a pas pour effet de ré-ouvrir le délai contentieux du tiers (CE, 6 Décembre 1999, Aubettes). \\
Aujourd'hui, le déféré n'a donc plus aucun intérêt pour les tiers au contrat. 


2. Les pouvoirs du juge \\
Dans le cadre du déféré, depuis 2011, le juge a les mêmes pouvoirs que dans le plein contentieux contractuel. Il a des pouvoirs modulables, qu'il utilise selon la faute ou selon l'intérêt général.


\section{Les recours des parties au contrat}

On a soit un contentieux de l'annulation, soit de l'exécution.

\subsection{Le contentieux de l'annulation}

C'est un contentieux dans lequel une partie demande au JA d'annuler la convention qu'il a lui même conclu. Pendant longtemps, les solutions étaient binaires: légal, il n'était pas annulé, illégal, il l'était. \\
Depuis CE, Sect., 28 Décembre 2009, Commune de Béziers (I), il y a une rénovation de l'action en contestation de la validité du contrat ouverte aux parties du contrat. Cet arrêt a opéré deux évolutions.


La première évolution est que certains moyens ne peuvent plus être invoqués par les parties au contrat. Le JA nous indique que les moyens invoqués ne doivent pas aller à l'encontre de la loyauté des relations contractuelles. Cette loyauté, c'est par exemple la bonne foie. L'arrêt a dans ses visas directement les articles du code civil. \\
Il arrivait parfois que certains concluent un contrat avec la personne publique et que, se rendant compte qu'il n'était pas forcément viable, ils ont demandés au juges d'annuler le contrat, pour se soustraire à ses obligations, ce qui va à l'encontre de la loyauté. 


La deuxième évolution tient au pouvoir du juge. Depuis 2009, tout vice entachant le contrat ne conduira pas forcément à une annulation. Le JA dispose des mêmes pouvoirs modulables que l'on retrouve dans un recours tropic ou Tarn-et-Garonne.


Le CE a unifié les pouvoirs du juge du contrat. Que ce soit dans le contentieux tropic, Tarn-et-Garonne, ou Béziers I, le juge a les mêmes pouvoirs. \\
Concernant les vices qui pourraient conduire à une résolution, le contrat ne sera annulé que dans les cas les plus graves, vice de consentement surtout, ou soit le caractère illicite du contenu du contrat. \\
À noter qu'un vice de procédure de passation de la convention n'a ici aucun impact car le vice ne peut pas lésé une des parties, au contraire. 

\subsection{Le contentieux de l'exécution}

Si on met le contentieux de la section de côté, il faut savoir que pendant longtemps, les parties au contrat ne pouvaient pas demander l'annulation des mesures d'exécution prises par la personne publique. Ils ne pouvaient demander qu'une indemnisation si la mesure d'exécution étaient considérés comme illégal. CE, 25 Octobre 2013, Région Languedoc-Roussillon, le principe demeure. \\
Une dérogation a été posée dans CE, Sect., 21 Mars 2011, Commune de Béziers (II) dans laquelle le juge vient introduire une action en contestation de la validité d'une mesure de résiliation du contrat. Depuis cette décision, la partie au contrat ayant subie une résiliation unilatérale peut demander son annulation au JA. Cela n'exclut pas de demander des indemnités. Cela concerne toutes les mesures de résiliation, tant dans l'intérêt général que pour cause de force majeure ou à titre de sanction. \\
Le JA a la possibilité de prononcer la reprise des relations contractuelles si la résiliation a été considérée comme illégal. Cette reprise ne sera pas prononcée dans la plupart des cas, il faut que le requérant le demande. Cette reprise ne pourra être demandé qu'eu égard à la portée de la mesure d'exécution et doit être formée dans les deux mois. Pour prononcer une telle mesure, le JA devra apprécier la gravité des vices ou les éventuels manquements. Surtout, le juge devra se demander si la reprise des relations contractuelles ne portent pas une atteinte excessive à l'intérêt général. Le juge et le CE n'a jamais prononcé une telle reprise. \\
Pour que le juge autorise une telle reprise, il faut que le requérant le demande, or, en général, il ne souhaite pas de reprise, mais plutôt une indemnisation. \\
La partie au contrat concernée peut demandé au référé suspension pour suspendre la résiliation.

\part{La responsabilité administrative}

\section{Introduction}

Tout comme en droit civil, il se peut que l'administration cause un dommage à un administré, qui se doit de pouvoir obtenir réparation de ce préjudice. Ce sont les exigences de l'État de droit qui commande une telle réparation. Cette responsabilité administrative, on doit la qualifier de civile et non pas de pénale, elle est délictuelle, mais toutefois soumise à des règles particulières, spécifiques, forgés par le juge dans un premier temps, complétés par le législateur dans un second temps. 


Une responsabilité qui a évoluée au fil des années. Jusqu'à il y a un siècle, l'État était irresponsable et le juge n'a eu de cesse que de faire progresser cette responsabilité, au nom du libéralisme politique. Le juge cherche à indemniser le mieux possible les victimes de l'administration. 

\subsection{Une responsabilité spécifique}

Ce qui justifiait l'irresponsabilité, c'était la souveraineté. Une exception était notable: les dommages de travaux publics, prévu par une loi de Pluviose, an VIII. \\
TC, 8 Février 1873, Blanco, la responsabilité de l'État est admise par principe. Le juge des conflits nous précise que la responsabilité est régie par un droit autonome "ne peut être régie par les règles du code civil", car elles permettraient d'engager trop facilement la responsabilité de l'État et ne tiendrai pas compte de la spécificité de l'Administration. \\
Si il y a des règles autonomes, les conditions d'engagements de la responsabilité sont les mêmes qu'en civil: un préjudice, un fait générateur, un lien de causalité entre les deux. \\
L'État peut avoir recours au droit privé: SPIC, gestion privée du domaine public etc. Dans ces domaines, c'est le droit privée qui s'applique, avec le juge judiciaire. 

\subsubsection{Une responsabilité étendue}

Au départ, l'État était totalement irresponsable. Petit à petit, le JA a admis la responsabilité administrative dans quasiment tous les domaines de l'action administrative. Quelques domaines résistent toujours où on ne peut pas engager la responsabilité de l'État, et cela concerne essentiellement les domaine régaliens de l'État. \\
Pendant longtemps, le JA n'a pas admis la responsabilité de l'État sous le régime de Vichy.


Concernant les activités régaliennes. \\
Dans la plupart de ces activités, on peut engager la responsabilité de l'État, comme par exemple dans le domaine de la police. \\
CE, 13 Janvier 1899, Lepreux, l'État était alors irresponsable sur le domaine de la police. Évolution: CE, 10 Février 1905, Tomaso-Grecco, à l'époque le JA va demander une faute caractérisé pour engager la responsabilité de l'État. Ensuite, ce sera faute lourde, puis aujourd'hui, une faute simple suffit. \\
Concernant le service public de la justice, l'État était jusqu'à récemment irresponsable: CE, Ass., 12 Juillet 1969, Sieur L'Etang, qui rappelle cette irresponsabilité. Cette irresponsabilité a été remise en cause par la loi du 5 Février 1972 qui admet la responsabilité de l'État du fait des décisions du juge judiciaire, et étendu au JA par CE, 29 Décembre 1978, Darmont. Aujourd'hui, la responsabilité de l'État est engagé dans ce domaine là essentiellement en cas de lenteur. \\
Concernant la responsabilité du fait des lois, elle a été pendant très longtemps refusée, en raison de la souveraineté du législateur, l'idée selon laquelle le législateur ne peut pas mal faire. Le JA se déclare incompétent pour déceler une faute du législateur. Depuis, la responsabilité de la puissance publique peut être engagé dans une responsabilité sans fautes. Dans un deuxième temps, CE, Ass., 8 Février 2007 Gardedieu, le CE admet la responsabilité de l'État si une loi est non conventionnelle. \\
Concernant le domaine des relations diplomatiques, elles appellent à l'édiction d'actes de gouvernement. Ils sont par principe non susceptible de recours contentieux, et ne peuvent donc pas, à fortiori, permettre d'engager la responsabilité de l'État. Le juge l'a cependant nuancé car le CE a admis une responsabilité sans faute du fait de conventions internationales. \\
Concernant les opérations militaires, ce sont les dernières irresponsabilités de l'État. "Les zones de guerre sont hors du droit", c'est une solution constante réaffirmé dans CE, 23 Juillet 2010, Société Touax, qui concernait les opération en ex-Yougoslavie. 


Il est à noter qu'une loi peut prévoir une indemnisation dans un domaine où l'irresponsabilité de l'État est admise. De même pour une convention internationale. \\
L'irresponsabilité dans le domaine militaire est curieux car abandonné dans toutes les autres mais logique car le CE considère que la décision d'envoyer les troupes constituent un acte de Gouvernement: CE, 5 Juillet 2000, Mégret. Question: solution compatible avec la CEDH et le droit à un recours effectif ? La réponse est positive, la CEDH elle même admet certains cas d'irresponsabilité de la puissance publique: CEDH, 14 Décembre 2006, Markovic c/ Italie. 

\subsection{La responsabilité de l'État français du fait des actes commis pendant le régime de Vichy}

Pendant très longtemps, le JA a considéré l'État comme irresponsable les actes commis pendant le régime de Vichy. Il est revenu dessus récemment.

\subsubsection{Irresponsabilité initiale}

Immédiatement après la seconde guerre mondiale: CE, Ass., 4 janvier 1952, époux Giraud, le CE déclare irresponsable l'État. Le juge considère que les actes pris sous le régime de Vichy doit être assimilé à des dommages de guerre pour lesquelles l'État est en principe irresponsable. L'ordonnance du 9 Août 1944 constate la nullité de tous les actes qui établissent une discrimination basé sur le fait d'être juif. En raison de cette disparition rétroactive, le CE a considéré l'État comme irresponsable de ceux-ci. \\
Un décret du 13 Juillet 2000 a donné des réparations aux orphelins de parents juifs, que le CE a considéré comme donnant des aides: CE, 6 Avril 2001, Pelletier. \\
Une raison sociologique peut expliquer ces solutions, le CE n'a pas eu une attitude irréprochable pendant la guerre. Il a appliqué à la lettre les actes pris pendant le régime de Vichy et certains conseillers d'État étaient encore en poste après la guerre. 


\subsubsection{La reconnaissance solennelle de la responsabilité de l'État dans la déportation des juifs}

Cette reconnaissance a été admise en raison de la faille dans le raisonnement des années 50. C'est pas parce que ces actes sont nuls qu'ils ne permettent pas d'engager la responsabilité, mais c'est parce qu'ils sont nuls que cela témoigne de la gravité de ces actes: CE, Ass., 12 Avril 2002, Papon. Pour la première fois, le CE admet la responsabilité de l'État pour les agissements administratifs qui ont facilité ou permis la déportation. \\
Le CE va considéré que Mr Papon a commis une faute qui d'abord celle du service administratif, à savoir l'application de règles antisémites, qui est aussi une faute personnelle de Mr Papon, qui a agi avec beaucoup de zèle. 


CE, Ass., Avis contentieux, 16 Février 2009, Hoffman Glemane, qui reconnaît solennellement la responsabilité de l'État du fait de l'agissement qui ont permis les déportations. Cela correspond aux agissements qui ne résultent pas de la contrainte directe de l'occupant: c'est à dire les arrestations, les internements, etc. \\
On aurai pu considéré que le juge judiciaire était compétent, par voie de faits. Cependant, le CE a refusé cette qualification car ce sont des qualifications juridiques qui ont conduit à des préjudices. \\
Le préjudice est considéré comme un préjudice particulier par le juge. Il parle de dommages exceptionnels d'une gravité extrême, qui ne sont pas qu'individuel, il parle de préjudice collectif. \\
Il existe un principe de réparation par équivalence, en général en argent. Dans cette affaire de déportation, le juge va y déroger, il va prononcer une indemnisation morale du préjudice subi: cela correspond donc à une reconnaissance solennelle du préjudice. Le CE va considéré que cette reconnaissance a été accompli par l'État, car plusieurs chef d'État ont reconnu cette responsabilité (Chirac, en 1995 notamment). Pour le reste, le JA va refusé l'indemnisation financière aux victimes. 

\subsection{L'engagement de la responsabilité administrative}

Pour se faire indemnisé d'un préjudice, il n'est pas possible de saisir le juge directement, il existe en contentieux administratif la règle de la décision préalable. La victime doit faire une demande d'indemnisation afin d'obtenir une décision préalable (en général un refus), que le requérant pourra ensuite attaquer devant le JA et obtenir un dédommagement. La décision préalable doit être nécessairement chiffré.


Le juge n'a eu de cesse d'étendre cette responsabilité. Le JA a eu à utiliser trois techniques, pour indemniser le plus possible les victimes de l'administration. \\
Une première technique est de substituer l'État à la faute de son agent. \\
Le JA exige de moins en moins des fautes lourdes afin de faciliter les démarches d'indemnisation. \\
Une troisième technique est l'idée de la responsabilité sans fautes, qui permet de dédommager les victimes même lorsqu'il n'y a aucunes fautes. Cela est basé sur une idée de solidarité.


\chapter{Les conditions générales d'évaluation du préjudice}


























\end{document}
