\documentclass[10pt, a4paper, openany]{book}

\usepackage[utf8x]{inputenc}
\usepackage[T1]{fontenc}
\usepackage[francais]{babel}
\usepackage{bookman}
\usepackage{fullpage}
\setlength{\parskip}{5px}
\date{}
\title{Cours de Sociologie de l'action collective (UFR Amiens)}
\pagestyle{plain}


\begin{document}
\maketitle
\tableofcontents

% Bibliographie
% L. Mathieu. Comment lutter ? Ed. Textuel

\chapter{Introduction}

La sociologie de l'action collective est un domaine ultra-classique de la science politique. C'est un domaine dont un tiers de l'activité des chercheurs concerne. \\
La sociologie de l'action collective vise à répondre à une énigme sociologique: pourquoi les individus se mobilisent collectivement pour défendre leurs intérêts ? On peut aussi se demander pourquoi il n'y a pas plus de mobilisation alors que les inégalités de distribution des chances sociales et économiques sont si grandes ? Ces inégalités se sont d'ailleurs accrus. Piketty montre que dans les années 80, la redistribution s'est inversé. Avant les années 80, la part qui revenait au travail était plus grande que la part qui revenait aux possédant. Ce rapport s'est inversé dans les années 80. \\
On note aussi que pour 9 enfants sur 10 nés depuis la fin des années 80, la chance de mobilité sociale est égale à 0. Ces chances sont nulles en dépit de la prolongation des études et de l'investissement massif des parents pour la réussite de leurs enfants. Cet investissement sert donc à limiter le déclassement. Le tout a été démontré par C. Pevgny.  \\
Dans un second temps, le cours s'intéressera à la forme. Comment se mobilise-t-on ? \\
On peut observer des carrières militantes se faire pour cause biographique: parents militants, etc. ; par "accident", sans se soucier de l'idéologie, en suivant des amis, en allant à une réunion, etc. \\
On notera d'ailleurs que ceux qui se mobilisent ne sont jamais compétent à fond (exemple du Petit Journal et de la manif pour tous), la sociologie sait cela depuis un moment. 


Ce cours se base sur deux prémisses, que l'on pourrait démontrer. \\
La première est que la défense des causes est la seule solution permettant de réduire les inégalités et de parvenir à une société démocratique. En clair, aucune cause ne peut l'emporter si elle n'est pas soutenue par une mobilisation. Il n'y a pas de cause plus légitime qu'une autre, on doit partir de l'idée que toutes les causes se valent. Ce point de vue, en tant que citoyen, pose problème. Mais en sociologie, on doit se dire qu'elles se valent toutes, et celle qui l'emportera sera celle qui sera le plus défendu. \\
Les inégalités qui parsèment le monde ne sont pas naturelles, cela signifie donc que des acteurs se sont mobilisés pour faire advenir le monde que nous connaissons. Toutes les mobilisations ne sont pas visible. \\
Comme c'est par les mobilisations que triomphent les causes, c'est par elles que la démocratie est possible. Les mobilisations sont donc nécessaires pour la démocratie. Les mobilisations sont nécessaires pour équilibrer les rapports de force. Buffet, un des hommes les plus riches du monde, dit qu'il y a bien une lutte des classes dans le monde, et se vante de l'avoir gagné. \\
La démocratie n'est pas la seule voie de la démocratie, et il semble même que le schéma d'un homme, une voix, soit défavorable à la démocratie. \\
La deuxième prémisse est qu'aucune cause triomphe sans être défendue. Peu importe sa légitimité. Certaines causes sont mal défendues, peu défendues, ou n'émergent pas. 


Le fait d'être collectif, de mener une action collective, pose un certain nombre de problèmes que ne posent pas les actions individuelles. Quand on monte une action collective, deux problèmes majeurs se posent: ne pas être seul, même si quelques personnes peuvent suffire ; ensuite, le problème de l'action: être et resté d'accord suffisamment longtemps sur l'objet et les formes de la mobilisation. \\
Il faut que les collectifs partagent le même objectif, ce qui n'est pas souvent le cas. \\
Les collectifs peuvent prendre des formes très diverses. De manière très large, un collectif peut être un mouvement social, ou peut être plus étroit, comme un parti, un syndicat, un groupe d'intérêt, etc. \\
Enfin, on a l'impression que les mouvements sociaux sont des actions d'exclus, d'acteurs qui sont incapable d'accéder au jeu politique, aux moyens traditionnel de ce jeu: le vote, l'assemblée nationale, etc. Cela est un à priori, car les actions collectives, sont, au contraire, menés en général par des gens qui ont un capital culturel.

\section{Devenir collectif: formes et raisons}

Nous allons laisser de côté les actions individuelles, tout ce qui est de la révolte individuelle. Ce qui nous intéresse, en science politique, c'est de savoir comment et pourquoi des individus décident d'agir ensemble au nom d'un principe ou d'un objectif, qu'ils déclarent être commune. \\
L. Boltanski a écrit en 1984 un article, "La dénonciation", dans la revue fondée par P. Bourdieu, il s'intéresse à des lettres envoyés par des lecteurs du journal "Le Monde" pour dénoncer des injustices. Le Monde avait l'habitude de publier régulièrement quelques une de ces lettres. Boltanski lit ces lettres et apprend que Le Monde a conservé toutes les lettres, y compris les non publiés, et va s'intéresser aux conditions qui font qu'elles sont publiées ("conditions de félicité", Gauffman). \\
Avec Boltanski, on s'intéresse à des actions individuelles, mais qu'est-ce qui fait qu'une dénonciation individuelle va être jugée suffisamment intéressante par le responsable de la rubrique pour faire l'objet d'une publication ? Qu'est-ce qui fait que cette dénonciation est susceptible d'intéresser le journal, et donc, un plus large public ? \\
Boltanski va traiter l'intégralité des lettres envoyés au journal Le Monde. La rupture avec Bourdieu se fait dans la méthode car Boltanski va considéré que celui qui juge n'a que les informations de la lettre, et va donc se cantonner à ses informations alors que pour Bourdieu, les mécanismes que doit étudier la sociologie sont cachés, aux acteurs eux mêmes. Pour Boltanski, les logiques de l'action sont là, cachés dans l'action elle même, à contrario de Bourdieu.


Boltanski montre que toutes dénonciations va faire intervenir quatre types de protagonistes: le dénonciateur ; une victime, en faveur de laquelle la dénonciation est effectuée ; un persécuteur, un coupable, qui réalise l'injustice ; un juge, celui qui sélectionne les lettres. \\
Boltanski va montrer que pour que la plainte soit jugée valide, il faut que les quatre protagonistes soient de taille équivalente. Boltanski explique que le dénonciateur est soumis à une contrainte de désingularisation, le dénonciateur doit se grandir. Il a deux tactiques: la première est de revendiquer pour lui un acteur collectif (une association par exemple) ; la deuxième est de rapprocher son cas d'une injustice reconnu par tous comme tel. Ces deux tactiques permettent de se généraliser, on peut noter d'ailleurs que parfois, les tentatives d'agrandissement peuvent rater. \\
Ce qui compte dans une action collective, ce n'est pas la justice de la cause mais la justesse de l'agrandissement. Paradoxalement, plus l'on trouve sa cause légitime, plus on doit veiller à l'agrandir de façon correct, donc à chercher parfois à en limiter la portée (sinon, on risque d'être non crédible). \\
Pour convaincre, la cause doit donc devenir public. Pour devenir public, un énoncé doit respecter un certain nombre de règles. Pour monter la défense d'une cause, et la publicisé, il faut que la cause passe "l'épreuve du public", D. Cefai: la publicité, c'est trois choses, ce qui est à tout le monde et qui n'est à personne. Un discours public peut donc potentiellement être reçu par tout le monde. Deuxième caractéristique de la publicité: la visibilité, ce qui est public est ce qui peut se voir de tous et par tous, il y a une réciprocité dans l'espace public entre celui qui voit et celui qui est vu. Troisième caractéristique: l'accessibilité, un espace public est un espace qui est accessible à tous. \\
L'épreuve de la publicité contraint les porteurs de revendications à structurer leurs discours d'une certaine façon. \\
On distingue trois types de publics: un public passif, de spectateurs, dont l'attention varie, auxquels on risque un jour ou l'autre de demander leur avis (référendum) ; le public agissant, qu'on voit aujourd'hui beaucoup grâce aux réseaux sociaux, ce que Weber appelle les agents politiques actifs ; enfin, on a les acteurs publics institués, ceux auxquelles une mobilisation s'adresse. 


On s'aperçoit que l'action collective a deux types de contraintes: une d'agrandissement, et une de publicité. Ces contraintes pèsent fondamentalement sur la façon dont les causes sont présentées, sur les personnes qui vont suivre la cause. Ces contraintes détermines le succès ou l'échec de la cause. \\
Devenir collectif est donc, en soi, un problème. 

\section{Mouvements sociaux, groupes d'intérêts, partis, syndicats, associations}

Ces différentes formes partagent un certain nombre de traits communs, un certain nombre de différence. \\
Une première distinction est celle entre mouvements sociaux et groupes d'intérêts. Pendant très longtemps, ce n'était pas les mêmes personnes qui travaillaient sur l'un et l'autre, il n'était pas légitime de travaille sur les groupes d'intérêts. O. Nay a publié un lexique de science politique qui définit les deux termes. Dans son lexique, on voit clairement un parti pris où il y a des gentils et des méchants. On lui a fait remarquer que les mouvements sociaux n'étaient pas forcément progressiste. 


Groupe catégoriel: situation elle même défini par l'organisation.\\
Groupe de conviction: intérêt inclusif, logique de l'adhésion volontaire. \\
Groupe territorialisé: défense d'un territoire donné. 


Les syndicats sont différents des partis.


\part{Pourquoi se mobilise-t-on ?}

\chapter{Les approches psychosociales des comportements collectifs}

\section{La grande peur des foules: sociologie et politique des temps démocratiques}

\subsection{La crainte des foules dans un temps démocratique}

À la fin du XIXe siècle, marqué par le triomphe de la République, il y a le coup du 16 Mai où McMahon tente un coup contre le Parlement. Il ne réussira pas, et en 1879, Jules Grévy sera le premier président républicain d'une nouvelle République qui repose sur le suffrage universel. En confiant le pouvoir au plus grand nombre, il se crée une crainte que les décisions prises ne soient pas rationnelles. Depuis Aristote, il y a une croyance que le nombre est réticent à la raison, il serait plutôt sensible à la démagogie. \\
À cette crainte s'ajoute une autre crainte, celle de l'individualisme, avec Drumont par exemple, qui dénonce un soit disant complot juif. \\
1792: Loi Le Chapelier, qui supprime les corporations. Cela laisse l'État et l'individu en face à face. L'État reconnaît des individus neutres, donc seulement en fonction de leur propriété politique. \\
Nisbet: La tradition sociologique. Il va relire un certain nombre de classique comme Durkheim et Comte, et va montrer que la sociologie va se présenter comme une sorte de remède à l'individualisme. Même une société d'individu, où l'individualisme est très présent, produit des normes collectives. Il existe des processus macro-sociologiques, morphologiques, de l'interdépendance, des solidarités, etc, même dans une société individualiste. 


Une troisième crainte est la peur des foules. Au XIXe siècle, ceux qui écrivent des ouvrages sont essentiellement issus de la bourgeoisie. Quand certains s'intéressent aux classes laborieuses, ils les décrivent avec dégoût, les voit comme une classe dangereuse, violente dans les relations. Dangereuse pour l'hygiène, dangereuse pour l'ordre public, dangereuse parce que révolutionnaire. Cela va être un enjeu d'aménagement urbain: c'est pourquoi Haussman troue de grandes avenues à Paris. \\
Aujourd'hui encore, les aménageurs urbains aménagent les places pour qu'elles soient sécurisés, facilement évacuable, facilement prenable d'assaut. \\
Les foules de classes populaires sont dangereuses aussi car présentées aussi comme irrationnelles. Le simple fait de réunir des individus les ferait arrêter d'être des individus rationnels. Trois auteurs vont créer des théories des foules. 

\subsection{Les théories du comportement des masses}

Hyppolite Taine, Gabriel Tarde, Gustave Le Bon créent les théories des foules. Leur premier point commun c'est que quand des individus sont réunis dans une foule, ils cessent de penser comme des individus. Le fait d'être un groupe transforme le comportement d'un individu. \\
Les trois auteurs vont parler d'hypnose, de contagion et de suggestion. Les membres de ces foules seraient comme hypnotisée, ce que Tarde propose comme la loi de l'imitation. L'hypnotiseur est le meneur et va donc émerger la figure du leader. Y. Cohen parlera de figure du chef qui émerge au XIXe siècle, pour qu'il y ait foule, il faut qu'il y ait leader. \\
Taine est un historien qui fait autorité. Il s'intéresse à la Révolution Française, il est traumatisé par la commune de Paris. Il cherche à comprendre les origines de la commune de Paris. Pour lui, les foules seraient tombées sous l'emprise des passions, entraînées par des criminels, des repris de justice. Selon Taine, l'homme a une nature animal, il craint sans cesse un retour à l'état de nature. Pour Taine, lorsque des foules se réunissent, l'homme retombe à l'état de nature. Ce retour à l'état sauvage est du à une contagion des émotions. 


Gabriel Tarde va "scientificiser" cette approche. Il propose une loi sociologique qui est la loi de l'imitation. Pour lui, toute collectivité humaine repose sur une loi de l'imitation: "la société pourrait se définir comme une collection d'êtres en tant qu'ils sont en train de s'imiter entre eux". Pour lui, l'homme en société est comme un individu somnambule, il croit avoir des idées propres alors qu'elles sont en fait suggérés par d'autres. \\
Tarde va se poser la question de ce qui fait qu'un rassemblement devient une foule. Pour commencer, il faut que le groupe dispose d'une communauté qui les prédispose à s'assembler. La foule dispose d'une communauté d'action à contrario du rassemblement. Ensuite, il faut un "atmosphère moral du moment", ce serai un faisceau d'idées à fortes charges émotionnelles d'après Tarde. Enfin, il faut qu'il y ait une étincelle. \\
Ces trois caractéristiques réunies, la foule se crée et ne fait que suivre ce qui vient d'émerger: une conscience collective commune. Tarde explique que cette foule s'en portera mieux si elle a été correctement travaillé par celui qui saura saisir cette foule. C'est là qu'un mécanisme d'imitation se crée, chaque individu imitant son semblable, son voisin, pour faire de la foule un espace compact. 


Gustave Le Bon va écrire "psychologie des foules" en 1895, qui sera un réel succès éditorial. Dans ce livre, Le Bon pioche allègrement ses théories chez Taine et Tarde. Le Bon emprunte tout un état d'esprit, et ce qui est flagrant chez lui, c'est la peur des foules, de la démocratie. Pour lui, les foules sont destructrices de la civilisation elle même. Le Bon applique la notion de foule un peu partout par exemple comme les membres d'un jury d'assises, les arènes parlementaires. \\
Pour Le Bon, ce n'est pas seulement la foule qui est dangereuse, c'est la démocratie aussi. Pour lui, tout ce qui est moderne est dangereux. Il va populariser un rejet de la démocratie. \\
Il va proposer l'idée que les foules sont hypnotisées par le leader. Ce qui l'intéresse c'est comment un leader hypnotise une foule. Pour lui, il y aurai trois techniques: une idée simple, qui s'applique à tout (affirmation d'un dogme), la répétition sans cesse de ce dogme sous une forme imagée et frappante, et enfin, favoriser la contagion des propos pour créer un courant d'opinion. Le terme qui sera inventé plus tard pour caractériser ça sera propagande, par S. Tchakotine dans son livre "Le viol des foules". Depuis, le modèle a été affiné. \\
Le Bon va scientificiser cela en se basant sur le travail de Charcot qui travaille sur l'hypnose. Ce que Charcot démontre de manière individuelle, Le Bon va essayer de l'adapter auprès du grand nombre. Pour Le Bon, la foule est le sujet hypnotisé par le leader, créant une dépendance entre "le maître et son troupeau". \\
Le Bon prétend avoir trouvé le moyen de gouverner les foules. Son dernier chapitre s'achève "Connaître l'art d'impressionner l'imagination des foules, c'est connaître l'art de les gouverner". \\
Ses idées vont inspirer des théoriciens totalitaires. 


Park publie en 1921 un traité de sociologie général, où l'on trouve un chapitre sur de la sociologie collective. On va trouver chez Park une référence au travail de Le Bon, mais en gardant seulement l'idée que c'est par le collectif que se produit le changement social. Il garde l'idée qu'il y a un maître de la foule, que les individus dans la foule n'ont pas tout à fait le même comportement. Cependant, Park a une vision plus optimiste de la foule en disant que pour un changement social, il faut qu'il y ait un passage collectif. \\
Park a comme étudiant Herbet Blumer, un des premiers à faire de la sociologie des médias et de la communication. Il va commencer par introduire une distinction entre ce que la théorie des foules appelait les foules, la masse, le public, qu'il distingue avec les mouvements sociaux. Pour lui, les mouvement sociaux ont une intention, qui cherchent à établir un nouvel ordre de vie alors que les foules, les masses, les publics, ne sont pas forcément réunis autour d'un agir ensemble intentionnel. \\
Il distingue les mouvements généraux (ouvriers, pacifistes etc), et des mouvements spécifiques qui portent sur des revendications particulières et qui reposent sur une appartenance et une conscience commune. Pour Blumer, participer à un mouvement procure un certain nombre de gratification: un frisson, un enthousiasme, la sensation d'être membre d'un collectif. \\
Blumer va proposer la notion de réaction circulaire, qui en fait une loi de l'auto-renforcement, chaque individu reproduit la stimulation d'un autre individu. Cela peut expliquer les phénomènes de radicalisation: en se reflétant réciproquement, ils s'intensifient. \\
Idée de carrière: agitation sociale, exaltation populaire, formalisation (organisation), l'institutionnalisation passant par sa reconnaissance par les pouvoirs publics. \\
Blumer va mener son enquête sur un cortège de célébration d'un général revenu victorieux. Il va passer un questionnaire dans la foule. Il va essayer de montrer que les individus partagent quelque chose. Il tire deux conclusions: les raisons de la présence des individus est loin d'être liée à l'événement lui même. Les individus sont capables de dire pourquoi l'événement a eu lieu néanmoins. En revanche, les individus ont beaucoup plus de mal à dire pourquoi le grand jour est grand pour eux. Le message produit reste extérieur. \\
Selon Blumer, il y a deux éléments en un seul: l'événement en lui même, et ceux qui le regarde.


Kornhauser publie The Politics of Mass Society en 1951. C'est un Allemand qui a quitté l'Allemagne. Selon lui, sous l'effet des médias de masse (radio, cinéma), nous serions entrés dans l'ère des sociétés de masse. Pour lui, les images du parti Nazi de 1930, sont des images traumatisantes, et va chercher à montrer que le modèle nazi se reproduit dans d'autres sociétés qui ne sont pas des sociétés totalitaires. \\
Ces sociétés de masse sont des sociétés qui ne sont ni démocratiques ni totalitaires, où les individus sont manipulés par les moyens de communication. Ce qui caractérise les sociétés de masse, c'est l'absence d'organisation sociale: il n'y a pas de corps intermédiaire structurant les relations entre l'État et les classes populaires, laissant ces derniers en pâture face à la puissance de l'État. Isolé, les individus isolés, n'ont pas accès au pouvoir et donc ne peuvent pas résister à la propagande étatique. \\
Cette thèse a été partiellement démonté par une analyse plus fine du régime Hitlérien, qui n'était pas en fait une société de masse, le régime se basant en fait sur des classes intermédiaires. 

\subsection{Ted Gurr et le modèle de la frustration collective}

Gurr a écrit Why Men Rebel? \\
Cet ouvrage est la conséquence d'une étude de presque une décennie et financé par le gouvernement américain qui est inquiet de la montée de la conflictualité sociale (contre la guerre au Vietnam, les mouvements des droits civiques, etc). Jusque là, la Science Politique américaine avait insisté sur une notion appelé l'apathie, qui est l'idée que les régimes démocratiques ne sont soutenables que dans la mesure où les individus ne participent pas trop: c'est le courant behavioraliste. Si les individus commencent à trop participer, ils risquent de trop exiger auquel le gouvernement ne va pas pouvoir répondre, augmentant la frustration vis à vis du régime. \\
Son étude montre que les américains s'intéressent très peu à la politique, ont peu de compétences politiques et peu d'américains sont impliqués dans un parti. \\
L'ouvrage de Gurr cherche donc à répondre à un problème: le passage d'un état apathique à un état conflictuel. 


Il s'intéresse essentiellement à la question de la violence. Ce qui l'intéresse, c'est le surgissement de la violence. Il va poser trois types de questions: quelles sont les sources de la violence ? Comment passe-t-on de la violence collective à la violence politique ? Comment expliquer l'intensité de la violence politique ? \\
Gurr va distinguer trois types de violences: le turmoil, une violence spontanée ; la conspiration, une violence hautement organisé avec une faible participation populaire ; la guerre interne, une violence politique hautement organisée avec une forte participation, de grande intensité. \\
Pour Gurr, la violence naît d'un mécontentement qui va se politiser et conduire à un passage à l'acte dirigé contre le acteurs politiques. \\
Ce mécontentement vient selon Gurr de la frustration relative: la différence perçue entre les attentes des individus et ce qu'ils pensent être en mesure d'obtenir. Tant que les individus ne perçoivent pas cette différence, ils ne perçoivent pas de frustration relative. \\
La première arme que donne Gurr au gouvernement, c'est qu'il leur explique que la population doit ne pas percevoir une telle frustration. \\
Ce mécontentement peut avoir des intensités différentes et donc pareil pour la violence. Cela dépend des facteurs sociologiques et psychologique. Les facteurs psychologiques reposent sur la solidité des croyances. Les facteurs sociologiques sont les sanctions qui peuvent être imposer (sociales, morale, juridique), la facilité d'utiliser la violence, et enfin, la légitimité du pouvoir en place. \\
Gurr pose une hypothèse majeure: le potentiel de violence collective varie fortement selon l'étendu et l'intensité de la frustration parmi les membres d'une collectivité.


Gurr identifie trois types de valeurs: les welfare values (les valeurs relatives au bien être), les power values (les valeurs relatives au pouvoir, l'influence que l'on a sur la marche des choses mais aussi l'autonomie que l'on dispose), les interpersonal values (la capacité pour un individu d'avoir des interactions non autoritaire avec d'autres individus). \\
Gurr met ensuite en évidence que les individus perçoivent leur capacité à atteindre ses objectifs, ces valeurs. Ces capacités sont dépendantes de la position de l'individu, et de ce qu'il peut potentiellement espérer obtenir. Ces capacités peuvent être mesurés à partir de ce que l'on peut appeler des opportunités, qui dépendent de trois facteurs: le niveau d'opportunité personnelle, les opportunités sociales, les opportunités politiques. À partir de ces trois éléments, ils vont percevoir leurs chances sociales. \\
Il distingue donc trois types de privations relatives: decremental deprivation, la demande reste stable alors que la situation sociale se dégrade. Aspirational deprivation, le niveau d'aspiration évolue alors que la satisfaction des besoins ne bougent pas. La progressive deprivation, où on a pendant un temps, une augmentation conjointe de la satisfaction des besoins et de l'aspiration, ensuite, l'aspiration va continuer d'augmenter alors que la satisfaction des besoins va chuter. Ce dernier est le plus à même de créer de la violence collective. \\
Ces trois types de privations relatives n'ont pas les mêmes effets, Gurr explique que cela dépend de l'intensité de sa motivation. Plus la motivation est élevé, plus la privation sera importante. L'intensité de la violence collective va aussi dépendre du type de valeurs que recherche les individus, pour Gurr, la privation relative est plus grande et donc la violence plus forte, d'abord pour les questions d'égalité économique (welfare), ensuite sur le power, et enfin sur les relations individuelles. 

\chapter{Le paradoxe de l'action collective}

Mancur Olson: The logic of collective action. Olson se repose sur une logique de calcul économique, avec la rationalité de n'importe quel individu qui ferait un calcul coût/bénéfice (homo oeconomicus). Les individus, selon lui, s'engagent après avoir fait ce calcul: que va coûter l'engagement ? Que va-t-il rapporter ? \\
Cela suppose que tous les individus quelque soit leurs origines sociales, leur politisation, leur occupation professionnelle, etc., Olson pense que n'importe quel individu peut faire ce calcul. Il a une vision de l'homme très particulière et très réductrice. Cela veut donc dire que le mécontentement ne suffit pas pour provoquer un engagement. On pourrait presque dire que le mécontentement est une variable accessoire chez Olson. Cette posture a tendance à délégitimé les mouvements sociaux. \\
Olson part avec ces deux à-priori: individus calculateur et pas forcément besoin de mécontentement pour protester. Olson souhaite remettre en cause un postulat: les individus qui ont des intérêts communs tendent à défendre ces intérêts communs. Il veut démontrer qu'au contraire, le fait que les individus sachent qu'en se groupant, ils pourront obtenir bénéfice commun, de un, ne suffit pas à susciter leur engagement, et de deux, il est même probable que cette mobilisation ne verra pas le jour. \\
Pour Olson, il faut d'autres types d'incitations que le bénéfice collectif pour se mobiliser. Donc, pour lui, personne ne va se mobiliser, et donc l'objectif commun ne sera jamais atteint. C'est le paradoxe de l'action collective selon Olson: alors même que les individus ont des intérêts en commun, ils ne vont pas les défendre. \\
Pour Olson, la passivité des acteurs sociaux tient à un élément fondamental, c'est la notion de bien collectif. Si les mouvements sociaux ont très peu de chance de voir le jour, c'est parce qu'ils visent un type de bien particulier, un bien collectif. C'est un bien qui bénéficie à l'ensemble du groupe et qui ne peut être refusé à aucun de ses membres. En cas de victoire d'une action collective, ceux qui ne se sont pas engagé dedans en bénéficient aussi, ce sont, selon Olson, des passagers clandestins, des "free riders". \\
Le problème est que tous les individus vont faire le même calcul coût/bénéfice et tout le monde va calculer qu'il faut rester en marge et laisser les autres se mobiliser. Ce qui finit à aucune mobilisation. Les intérêts individuels entrent donc en conflit avec les intérêts collectifs et peuvent entraver le développement d'une mobilisation qui avait toutes les chances de réussir. \\


Ce qu'il y a d'étonnant avec cette logique, c'est qu'il y a des mobilisations. Dans sa logique, toutes les mobilisations sont improbables. Dans ce cas, on peut se poser la question de comment se fait-il qu'il y ait des mouvements sociaux ? \\
Marx disait que l'apathie était entretenu par une fausse conscience propagé par la bourgeoisie sur la mode "votre patron vous veut du bien". C'est une logique paternaliste. Cette fausse conscience est dans une logique du patron bien sympa, contre qui une action collective n'est pas légitime. \\
Olson est au contraire de Marx. Il dit que l'apathie est le résultat du calcul rationnel que l'individu effectue. Cela signifie que si un jour une Révolution survient, elle est forcément la résultante d'une élite révolutionnaire. Olson est plus Léniniste que Marxiste. \\
Olson conçoit ces individus comme étant en lien avec d'autres individus. L'individu tel que pensé en économie est un individu seul. Hors, Olson les imagine en interaction avec autrui. Olson va dire que ces interactions à autrui sont toujours basés sur l'intérêt. Cet élément est totalement déterminant, puisque cela permet de déterminer chez Olson comment l'action collective survient. \\
Le premier facteur explicatif tient à la taille des groupes dans lesquels les individus sont plongés. Plus le groupe est petit, plus ils vont se mobiliser car il est facile d'y exercer un contrôle social (on sait qui est là ou n'est pas là), de plus, si on est dans un petit groupe, il est facile de visualiser l'utilité que l'on a. Une étude mené sur le PSI a montré que plus la taille de la section était réduite, plus la participation était grande. La communication dans les petits groupes est aussi plus simple. \\
Paradoxalement, on pourrait penser que dans nos sociétés actuelles, c'est la force du nombre qui l'emporte. Or, ce que nous montre Olson, c'est l'inverse, car plus il est petit plus il aura du mal à se faire entendre et à montrer une représentation. Les petits groupes sont plus faciles à organiser, cependant, ils risquent d'être privés des ressources qui permettent de créer un rapport de force favorable. 


Olson va ensuite se poser une deuxième question: comment des groupes peuvent-ils croître en taille alors qu'au fur et à mesure de leur agrandissement, la logique de free rider l'emporte ? \\
Au fur et à mesure que le groupe monte en taille, il va falloir organiser ce groupe pour qu'il offre des incitations sélectives. \\
Une organisation permet d'offrir à ceux qui décident d'y appartenir un certain nombre de rétributions. Elles peuvent être de différents ordres, des rétributions d'ordre matériels par exemple (poste permanent dans un parti, un emploi public). C. Mattina a publié un livre qui porte sur le système clientélaire Marseillais et comment ce système permet les rétributions. Un mouvement social ne se tient que si il fabrique une organisation capable de donner des rétributions. \\
Il existe d'autres types de rétributions, comme des rétributions symboliques. Cela peut être des choses de l'ordre de l'affectif ou de l'émotionnel voir du sentimental. V. Jerome montre comment ELLV est structuré autour de relations affectives et sexuelles, le parti étant une cristallisation de relations affectives, de couples, etc. Chez les Black Panthers aux US, les rétributions sont là aussi très clairement sexuelles (O. Fillieule), où les femmes étaient conçus comme des rétributions pour les hommes prenant des risques pour le mouvement. \\
Le côté rétribution symbolique renvoie à une rétribution "frisson dans le dos", le fait simple de monter un truc collectif peut créer un plaisir qui suffit pour s'engager dans une action collective. On perçoit cet engagement surtout quand on s'intéresse aux logiques de désengagements. F. Joshua nous parle de cycles militants avec une fièvre militante qui entraîne vers l'action collective et qui se traduit ensuite par une redescente, soit parce que la fièvre est retombée soit par un désalignement du militant avec le mouvement. Quand on arrive à travailler sur le "malheur militant", on se rend compte qu'il est très présent et que donc le militant milite aussi beaucoup par plaisir.


Il existe peu de mobilisations qui sont spontanée. Que ce soit des rassemblements après les attentats de Charlie, ou Nuit Debout, il y a toujours une organisation qui se mobilise. Une organisation est donc nécessaire à la mobilisation. \\
Ce point de vue peut poser problème notamment pour ceux qui se définissent comme autonome et qui refusent d'appartenir aux organisations. Il existe dans toute organisation, selon R. Michels, une loi d'airain de l'oligarchie. Selon cette loi, même l'organisation qui se dit le plus démocratique possible et qui fonctionne selon cela, le pouvoir revient toujours à ceux qui se trouvent au centre de l'organisation. Donc même si il existe des procédures de désignation, elles vont tourner autour d'une oligarchie. Tout organisation sécrète une oligarchie. Cela tient à deux éléments: ceux placés au centre ont le privilège de l'information, ils savent ce qu'il se passe ; ils ont aussi le privilège de position, en disant ce que l'organisation défend. Ceux qui sont au centre ont donc plus de chances d'être élus car sont au centre (le capital revient au capital pour faire une analogie). \\
À partir des années 60-70, dans un certain espace des mouvements sociaux, on va trouver un certain nombre d'acteurs qui vont se revendiquer autonomes car l'organisation détourne les moyens de la cause au profit de ses portes paroles. Face à ce constat, il peut y avoir deux types de comportement. Le premier est celui des autonomes. Le deuxième est de se dire qu'il faut faire avec, en cherchant à limiter les effets, comme en limitant les mandats, le nombre de reconduite des mandats, etc, comme le fait EELV. \\
Politiquement, on peut défendre une cause sans forcément faire une organisation. Mais cela pose forcément d'autres engagements, qui sont complets, avec des individus entièrement dédiés à la cause (les Zadistes en sont le parfait exemple). 


Olson parvient à expliquer l'émergence des mouvements collectifs par l'existence d'une organisation et des rétributions qui tiennent le mouvement collectif. Ces organisations vont proposer des rétributions individuelles en surplus du bien collectif visé. \\
Le fait d'adhérer à une organisation va pouvoir attiré des bénéfices. Par exemple, les syndicats américains offrent une protection par une mutuelle santé. Ce ne sont des choses qui sont possibles que parce que l'organisation en question a atteint une taille suffisamment importante. Le fait d'être très largement collectif va permettre d'avoir accès à des biens auxquels les individus n'ont pas accès seul. \\
Il existe aussi des incitations négatives: contraintes, pressions, stigmatisations, qui vont lutter contre la stratégie du passager clandestin en faisant en sorte d'obliger à l'action collective: c'est la stratégie du piquet de grève. Il peut y avoir entre gréviste et non gréviste des affrontements violents. Aux US, on trouve dans l'histoire des mouvements sociaux des répressions très violentes comme dans des usines d'armement où des "jaunes", des individus recrutés par les patrons pour casser littéralement les grèves (tant d'un point de vue travail que violenter physiquement les grévistes). D. Eribon dans "Retour à Reims", montre très bien comment un comportement peut être mis au ban d'un collectif via ces contraintes. \\
Beaud et Pialoux dans "La misère du monde", décrivent une grève dans une usine. Ils font une observation d'un cortège de gréviste qui remonte la chaîne de production pour inciter les ouvriers à faire grève sauf les intérimaires. Ils vont montrer qu'il y a une remise en cause du "jaune" car, avec la transformation du mode de recrutement, l'intérimaire ne peut pas faire grève, et donc il sera moins mis au ban du collectif. Cela montre aussi une certaine logique à casser l'action collective. \\
Cette théorie d'Olson nous permet de comprendre pourquoi certaines formes de mécontentement ne débouche pas sur des mouvements contestataires. 


Daniel Gaxie, en 1977, va travailler à partir du schéma d'Olson et va l'appliquer aux partis politiques. Olson et Gaxie n'ont pas du tout le même passif, Gaxie est très inspiré de la sociologie de Bourdieu dans laquelle les agents ne sont pas rationnels mais agissent selon des déterminismes sociaux. La posture de Gaxie et Olson est donc très différente d'Olson mais vont aller piocher dans la rétribution. \\
Là où Olson voit un calcul des acteurs, Bourdieu et Gaxie vont eux, voir des stratégies d'entrepreneur. Ils partent à partir des organisations et vont voir comment ces organisations construisent des ressources pour lutter contre le paradoxe de l'action collective, contre les logiques de free rider. \\
Gaxie lutte contre l'idée que la classe ouvrière ou du moins que les militants vont le devenir lorsqu'ils prennent conscience du problème. "C'est en fournissant à certains de ses membres des avantages personnels qu'un groupe pourra en fait créer une organisation", D. Gaxie. Une première tendance serait de dire qu'une organisation sert les intérêts de ceux qui la dirige. On peut penser que le prix à payer de toute organisation collective, c'est que des acteurs soit rémunérés pour s'occuper à temps pleins ou presque de la cause défendue. Les bénéfices individuels sont donc vus comme un moyen de faire marcher l'organisation. Gaxie retourne ce que dit Olson. L'intérêt individuel pourrait donc être le moyen de réaliser l'intérêt collectif (alors que Olson disait l'inverse). \\
On pourrait dire qu'une organisation militante est une structure de redistribution des ressources collectives chez Gaxie. Gaxie nous montre qu'on peut avoir une lecture simpliste d'Olson qui délégitime les mouvements collectifs ou qu'on peut avoir une lecture bien plus complète, qui nous montre qu'il faut accepter le reniement de ses idées pour pouvoir les défendre. C'est un fort paradoxe que de voir que pour militer, la première chose à faire est de savoir mettre entre parenthèses ses idées. \\
La lecture simpliste d'Olson mène à un fort individualisme alors que si on comprend que l'individuel et le collectif se complète, on a une lecture plus complexe d'Olson. 

\chapter{L'école de la mobilisation des ressources}

Cette école théorique (qu'on abrège PMR, Paradigme de la Mobilisation des Ressources) est une école importante. Elle est en partie le fait d'auteurs qui sont les élèves d'Olson, notamment Mayer Zald et J. McCarthy. Ils ont écrit "Social movement in an organisational society". \\
Ce paradigme émerge dans un contexte un peu particulier, celui de la publication du libre d'Olson, mais aussi marqué par les contextes des années 60-70, un contexte dans lequel la participation politique a complètement changée de sens. Dans la théorie sociologique, jusqu'aux années 60, les formes de mobilisation étaient des formes qui allaient contre le régime démocratique (révoltes ouvrières, mouvements anarchistes, révolution Russe). La contestation était vue comme un coup porté à la démocratie. À partir des années 70, la sociologie va voir ces mouvements comme se situant à l'intérieur du cadre démocratique. Ils réclament une inclusion de population marginalisé dans le système démocratique. \\
Ils ont en tête le mouvement des droits civiques américains. Comme son nom l'indique, ce mouvement se situe à l'intérieur de la démocratie et ses cibles sont des cibles institutionnelles, ce sont les institutions qui font obstacles à l'intégration de populations marginalisés dans le système démocratique. Ces mouvements sociaux ne visent pas à renverser le système mais au contraire à lui permettre de s'institutionnaliser.


Cette époque est aussi marquée par une transformation des formes conventionnelles de participation politique. Lester Milbrath, en 1965, avait publié "Political participation", dans lequel il différenciait les participations conventionnelles des non conventionnelles. La première est une participation qui permet de maintenir l'ordre politique, à l'inverse donc, les participations non conventionnelles visent au désordre. Au premier lieu de la participation conventionnelle, on a le vote, l'adhésion à un parti politique. Manifestation, sitting, grève, etc, sont des protestations non conventionnelles car déstabilise l'ordre politique tel qu'il est en menaçant la stabilité du pouvoir. \\
Il légitime donc les participations conventionnelles. Or, ces participations, sont les participation des groupes institués. Quand ces groupes institués manifestent contre la guerre du Vietnam, la dichotomie de Milbrath en prend un coup. La question qui va se poser: que faire de cette nouvelle morphologie ?


Cela va amener à appeler participation politique toutes les formes par lesquelles tout acteurs individuels ou collectifs s'efforcent d'influencer ou de soutenir des décisions politiques. Ces acteurs nomment-ils ce qu'ils font de politique ? \\
La contestation politique des années 60 est venu du centre politique, ce sont les individus les plus intégrés politiquement qui ont usés des techniques les plus radicales. \\
Lors du freedom summer, en 1964, un milliers d'individus, issus d'une bourgeoisie et diplômé en droit, vont passer un été dans le Mississippi (un État américain qui est très raciste et limite esclavagiste) pour inscrire les noirs sur les listes électorales. Cela fera quatre morts. La quasi-totalité de ces individus ont étés tabassés, etc. Doug McAdam s'interroge sur le devenir de ces gens là, mais plus globalement, leurs trajectoire, et l'avant freedom summer est ce qui nous intéresse. \\
Il va réussir à avoir la liste de tous les CV de ceux qui ont postulés et ceux qui ont été sélectionnés. Il montre que le groupe de ceux qui postulent et ceux qui sont partis sont très proche: même appartenance sociale, mêmes études prestigieuse, et souvent des origines géographiques semblables. Les différences sont subtiles, trois éléments font la différence entre les sélectionnés et les postulants: la disponibilité géographique (avoir le temps, être sorti de la tutelle parentale, ne pas avoir une charge de famille) ; sur le plan des convictions, le deuxième élément, ce sont les affinités idéologiques: ceux qui sont partis partagent les valeurs du mouvement ; le troisième élément, c'est l'intégration sociale: ceux qui partent disposaient de liens entre eux plus forts que ceux qui sont restés. \\
Doug McAdam montre que contrairement à ce qu'on voyait dans le passé, ce sont les individus les moins intégrés qui se mobilisent avec les moyens les plus radicaux. 


Le PMR a l'avantage de remettre la question de la participation politique au centre de la question. Cette école, fondée par McCarthy et Zald (élève de Olson), ce qui les intéresse c'est de contredire leur maître. Olson montrait que pour produire des gratifications, il fallait des organisations. C'est l'organisation qui explique l'émergence des mouvements. À l'inverse, l'impuissance de celle-ci à mobiliser des ressources va expliquer l'échec du mouvement. \\
La conclusion logique, c'est qu'une organisation est nécessaire pour faire une mobilisation, ce qui contredit Olson. Le travail va donc être sur la question de savoir comment une organisation s'y prend pour mobiliser des ressources qui seront investis dans la mobilisation collective dans la logique de lutter contre les logique de free rider. \\
À partir du moment où ils disent que l'important est de voir comment les organisations mobilisent des ressources, ils écartent la question du pourquoi on se mobilise. La vraie question de science politique, c'est comment réussit ou échoue une mobilisation. \\
McCarthy et Zald laissent de côté entièrement l'idée que c'était le mécontentement qui faisait les mobilisations. "Il y a toujours dans n'importe quelle société assez de mécontentement pour engendrer des mobilisations mais ces mécontentements peuvent être créés, défini et manipulés par des entrepreneurs de cause et des organisations". La question posée sera: comment sont créées les causes autours desquelles il y aura une mobilisation ? \\
Ils nous disent qu'un bon entrepreneur peut faire émerger une mobilisation sans mécontentement. \\
Ils vont nous décrire ces causes comme des créations sociales et vont nous expliquer ce qui fait qu'une cause marche ou non. Pour ce faire, ils vont utiliser un vocabulaire économique et vont décrire les mouvements sociaux comme des entreprises, où il existe un marché, des parts de marché. 


Les organisations cherchent à mobiliser des ressources. Gamson parle du "ventre mou", car cette notion de ressource est tellement étendue qu'on ne sait pas ce qui est ressource ou non. Potentiellement, tout est ressource. Elles vont être défini par leur interaction. À priori, le fait d'avoir occupé des postes publics n'est pas une ressource pour obtenir des gains électoraux, alors que ça peut en être une dans le sens où on peut se prévaloir de compétences. De même, se présenter comme en dehors de ces emplois peut être une ressource ou non. \\
Historiquement, ce qui a été présenté comme la première ressource, c'était la masse, la taille du groupe. Le nombre, c'est quelque chose d'identitaire mais aussi une force d'agir. Ce nombre donne aussi une puissance économique, car si le nombre donne ne serait-ce qu'un peu, l'organisation va pouvoir lancer des campagnes, affréter des bus, etc. \\
L'autre grand type de ressources est de donner une conscience identitaire, c'est à dire que les individus se disent comme membre d'un tout, comme membre du PC, de la CGT, de Sens Commun, etc. \\
Un autre grand type de ressource, c'est la capacité à agir. C'est à dire des argumentaires, des lieux, des tenus, des couleurs, etc. On donne des capacités d'agir à des individus. C'est l'idée que l'organisation puisse dire à ses adhérents: "Regardez, on obtient des résultats", cela est très déterminant dans le succès ou non d'une mobilisation. \\
Oberschall dit que "la mobilisation désigne le processus par lequel un groupe mécontent assemble et définit des ressources dans la poursuite de buts propres". Dans cette lecture, les ressources peuvent être vus comme un investissement au sens économique du terme. On espère donc un retour sur investissement. \\
Ces organisations, on peut les voir comme réfléchissant comme des entreprises, se disant "là on y va", ou là non, car certains qu'ils vont se planter. Il y a donc un calcul qui va être déterminé par les moyens qu'a l'organisation ou non, y-a-t-il la possibilité d'organiser une manifestation ? Cela va-il floper ou non ? Ce sont les questions que se posent ces organisations. Il est à noter que les organisations ne peuvent pas se mobiliser sur tout: pour investir, il faut qu'il y ait un risque, sinon il n'y a pas de retour sur investissement. Et elles ne peuvent pas investir sur tout donc il y a un calcul coût/bénéfice, dont l'objectif est de maintenir l'organisation vivante et de continuer la mobilisation. \\
On peut donc dire que les différentes organisations se font concurrence. Certaines travaillent sur le même marché donc se font concurrence. Elles cherchent donc à croître leurs parts de marchés. Ce faisant, en prenant un risque avec leur capital, elles vont chercher à sécuriser les investissements. Les organisations vont donc se reposer sur des professionnels, notamment de l'événementiel qui vont essayer de garantir le succès d'une mobilisation à moindre frais. \\
S. Lefebvre, dans ONG et Cie, nous parle de la professionnalisation des campagnes et du militantisme. Ils nous raconte comment Greenpeace, à la suite d'un changement de direction, cherche à créer des dons répétés, par virement automatique. Ce nouveau directeur (Bode), réussit à porter le financement de Greenpeace à plus de 800 millions d'euros. Bode va demander à ses militants d'aller dans la rue pour trouver cet argent. Créant un sentiment de dévalorisation chez les membres, il va extériorisé cette recherche, ce qui va conduire à la création d'entreprise spécialisée dans le "street fund rising".


McCarthy et Zald vont caractériser un certain nombre de termes. \\
Un mouvement social est un "ensemble d'opinions, de croyances, de préférences favorables à des changements de la structure sociale et/ou à une autre répartition des récompenses dans la société". \\
Pour eux, une organisation du mouvement est "une organisation qui identifie ses objectifs aux buts du mouvement social et tente de satisfaire ses objectifs". \\
L'ensemble des organisations d'un mouvement social s'appelle selon eux une industrie du mouvement social. \\
On peut avoir différentes organisations sur un même secteur. Ils définissent ce secteur par l'ensemble des industries d'un seul mouvement social. \\
Pour McCarthy et Zald, le développement des organisations est un produit de luxe car c'est parce que les individus sortent de la satisfaction de leurs besoins primaires qu'ils peuvent engager leurs ressources dans des mouvements sociaux. \\
Paradoxalement, pour expliquer un certain militantisme, McCarthy et Zald sont obligé d'identifier un acteur à l'écart des logiques économiques: les militants par conscience, qui ne tirent aucun profit de la mobilisation. 


Oberschall va distinguer deux axes de développement à l'intérieur d'une organisation. Il y a des groupes très structurés, d'autres beaucoup plus atomisés. \\
Il distingue dans un premier axe, horizontal, plusieurs modèles de formation de groupe. Il y a d'abord un modèle communautaire, très restreint, très lié. Il y a, à l'opposé, un modèle peu organisé. Entre les deux, on trouve le modèle associatif, où il est facile de rentrer et de se retirer. \\
Dans un axe vertical, il distingue la position du groupe dans la société. Il donne un critère qui celui de la qualité des relations entretenus par le groupe avec le centre de pouvoir. Oberschall appelle cela le niveau d'intégration. Des groupes peuvent accéder facilement au centre du pouvoir décisionnaire, pour une raison ou une autre (relais, liens familiaux, territoriaux etc), d'autres y accèdent difficilement voire pas du tout, ce sont des groupes segmentés. \\
Oberschall nous dit qu'en fonction du type de connexion entre ces deux axes fait varier la forme et la réussite de la mobilisation. L'intégration fait freiner les risques de mobilisation. Dans des groupes peu intégrés, la forme d'organisation fait varier la forme et la réussite des mobilisations: les groupes organisés ont tendance à mieux se mobiliser, alors que les moins organisés connaîtront des "poussées de fièvre". 


Klandermans et MacAdam vont critique l'école de mobilisation des ressources. Leur critique repose surtout sur la critique de la vision que défendent McCarthy et Zald défendent, la vision économiste, voire utilitariste de cette école. \\
Klandermans veut montrer le poids des dimensions affectives, dimensionnelles, de l'investissement dans les organisations. Il va utiliser la formule de "choc moral". Tibery montre qu'il y a un pic de participation, y compris encore aujourd'hui et dans les élections perçus de second ordre, ce pic est formé par une génération qui a subi un choc moral le 21 Avril 2002 avec le passage au second tour de JM Lepen. \\
Le choc moral peut être vécu isolément, ou des organisations peuvent mobiliser ces consensus. L'idée est de transformer un capital de sympathie en la canalisant pour faire une manifestation effective. 

\chapter{La frame analysis, l'analyse de cas}

C'est une théorie qui permet de s'intéresser à des choses beaucoup plus fines, aux relations entre les individus pour expliquer leur engagement. L'hypothèse centrale de cette analyse est que les mobilisations sont davantage dues à des interactions micro-sociologique qu'à des déterminants macro-sociologique. \\
La théorie de la mobilisation des ressources explique les mobilisations par de la macro-sociologie, alors que le frame analysis va dire qu'un individu participe à une mobilisation car une personne de leur entourage lui ont demandé d'y aller. \\
Snow parle d'un ajustement de cadre. C'est cet ajustement qui explique la mobilisation. Braconnier et Dormagen montrent très bien cette logique lorsqu'ils observent qu'un immeuble vote alors que l'autre, non. \\
La logique de cette théorie est de dire que les individus qui rejoignent un mouvement, partagent les positions et les revendications, la même condamnation d'une situation jugée injuste, ils imputent les responsabilités aux mêmes types d'acteurs, ils envisagent le même type de réponse. Chacun de ces quatre éléments est très difficile à obtenir. Snow va montrer qu'il est très compliqué de faire en sorte que tout un groupe partage la conviction qu'une situation est jugée injuste. Pour ce faire, ils vont prendre Goffman, pour qui l'ordre social est fondé à partir des interaction. Il y a deux règles fondamentale dans ceux-ci: il faut sauver la face de la personne avec qui l'on est en interaction ; il faut prévenir toutes les situations où l'on pourrait manquer de respect. Deux types de techniques pour cela: l'évitement (les gens passent leurs temps à éviter les sujets qui fâchent) ; la réparation (à chaque fois qu'on a commis une erreur, on cherche à la réparer). 


\section{Le cadre d'injustice}





\sectionn{Les alignements de cadre}










\part{Comment se mobilise-t-on ?}

\end{document}
