\documentclass[10pt, a4paper, openany]{book}

\usepackage[utf8x]{inputenc}
\usepackage[T1]{fontenc}
\usepackage[francais]{babel}
\usepackage{bookman}
\usepackage{fullpage}
\setlength{\parskip}{5px}
\date{}
\title{Cours de Sociologie de l'action collective (UFR Amiens)}
\pagestyle{plain}


\begin{document}
\maketitle
\tableofcontents

% Bibliographie
% L. Mathieu. Comment lutter ? Ed. Textuel

\chapter{Introduction}

La sociologie de l'action collective est un domaine ultra-classique de la science politique. C'est un domaine dont un tiers de l'activité des chercheurs concerne. \\
La sociologie de l'action collective vise à répondre à une énigme sociologique: pourquoi les individus se mobilisent collectivement pour défendre leurs intérêts ? On peut aussi se demander pourquoi il n'y a pas plus de mobilisation alors que les inégalités de distribution des chances sociales et économiques sont si grandes ? Ces inégalités se sont d'ailleurs accrus. Piketty montre que dans les années 80, la redistribution s'est inversé. Avant les années 80, la part qui revenait au travail était plus grande que la part qui revenait aux possédant. Ce rapport s'est inversé dans les années 80. \\
On note aussi que pour 9 enfants sur 10 nés depuis la fin des années 80, la chance de mobilité sociale est égale à 0. Ces chances sont nulles en dépit de la prolongation des études et de l'investissement massif des parents pour la réussite de leurs enfants. Cet investissement sert donc à limiter le déclassement. Le tout a été démontré par C. Pevgny.  \\
Dans un second temps, le cours s'intéressera à la forme. Comment se mobilise-t-on ? \\
On peut observer des carrières militantes se faire pour cause biographique: parents militants, etc. ; par "accident", sans se soucier de l'idéologie, en suivant des amis, en allant à une réunion, etc. \\
On notera d'ailleurs que ceux qui se mobilisent ne sont jamais compétent à fond (exemple du Petit Journal et de la manif pour tous), la sociologie sait cela depuis un moment. 


Ce cours se base sur deux prémisses, que l'on pourrait démontrer. \\
La première est que la défense des causes est la seule solution permettant de réduire les inégalités et de parvenir à une société démocratique. En clair, aucune cause ne peut l'emporter si elle n'est pas soutenue par une mobilisation. Il n'y a pas de cause plus légitime qu'une autre, on doit partir de l'idée que toutes les causes se valent. Ce point de vue, en tant que citoyen, pose problème. Mais en sociologie, on doit se dire qu'elles se valent toutes, et celle qui l'emportera sera celle qui sera le plus défendu. \\
Les inégalités qui parsèment le monde ne sont pas naturelles, cela signifie donc que des acteurs se sont mobilisés pour faire advenir le monde que nous connaissons. Toutes les mobilisations ne sont pas visible. \\
Comme c'est par les mobilisations que triomphent les causes, c'est par elles que la démocratie est possible. Les mobilisations sont donc nécessaires pour la démocratie. Les mobilisations sont nécessaires pour équilibrer les rapports de force. Buffet, un des hommes les plus riches du monde, dit qu'il y a bien une lutte des classes dans le monde, et se vante de l'avoir gagné. \\
La démocratie n'est pas la seule voie de la démocratie, et il semble même que le schéma d'un homme, une voix, soit défavorable à la démocratie. \\
La deuxième prémisse est qu'aucune cause triomphe sans être défendue. Peu importe sa légitimité. Certaines causes sont mal défendues, peu défendues, ou n'émergent pas. 


Le fait d'être collectif, de mener une action collective, pose un certain nombre de problèmes que ne posent pas les actions individuelles. Quand on monte une action collective, deux problèmes majeurs se posent: ne pas être seul, même si quelques personnes peuvent suffire ; ensuite, le problème de l'action: être et resté d'accord suffisamment longtemps sur l'objet et les formes de la mobilisation. \\
Il faut que les collectifs partagent le même objectif, ce qui n'est pas souvent le cas. \\
Les collectifs peuvent prendre des formes très diverses. De manière très large, un collectif peut être un mouvement social, ou peut être plus étroit, comme un parti, un syndicat, un groupe d'intérêt, etc. \\
Enfin, on a l'impression que les mouvements sociaux sont des actions d'exclus, d'acteurs qui sont incapable d'accéder au jeu politique, aux moyens traditionnel de ce jeu: le vote, l'assemblée nationale, etc. Cela est un à priori, car les actions collectives, sont, au contraire, menés en général par des gens qui ont un capital culturel.

\section{Devenir collectif: formes et raisons}

Nous allons laisser de côté les actions individuelles, tout ce qui est de la révolte individuelle. Ce qui nous intéresse, en science politique, c'est de savoir comment et pourquoi des individus décident d'agir ensemble au nom d'un principe ou d'un objectif, qu'ils déclarent être commune. \\
L. Boltanski a écrit en 1984 un article, "La dénonciation", dans la revue fondée par P. Bourdieu, il s'intéresse à des lettres envoyés par des lecteurs du journal "Le Monde" pour dénoncer des injustices. Le Monde avait l'habitude de publier régulièrement quelques une de ces lettres. Boltanski lit ces lettres et apprend que Le Monde a conservé toutes les lettres, y compris les non publiés, et va s'intéresser aux conditions qui font qu'elles sont publiées ("conditions de félicité", Gauffman). \\
Avec Boltanski, on s'intéresse à des actions individuelles, mais qu'est-ce qui fait qu'une dénonciation individuelle va être jugée suffisamment intéressante par le responsable de la rubrique pour faire l'objet d'une publication ? Qu'est-ce qui fait que cette dénonciation est susceptible d'intéresser le journal, et donc, un plus large public ? \\
Boltanski va traiter l'intégralité des lettres envoyés au journal Le Monde. La rupture avec Bourdieu se fait dans la méthode car Boltanski va considéré que celui qui juge n'a que les informations de la lettre, et va donc se cantonner à ses informations alors que pour Bourdieu, les mécanismes que doit étudier la sociologie sont cachés, aux acteurs eux mêmes. Pour Boltanski, les logiques de l'action sont là, cachés dans l'action elle même, à contrario de Bourdieu.


Boltanski montre que toutes dénonciations va faire intervenir quatre types de protagonistes: le dénonciateur ; une victime, en faveur de laquelle la dénonciation est effectuée ; un persécuteur, un coupable, qui réalise l'injustice ; un juge, celui qui sélectionne les lettres. \\
Boltanski va montrer que pour que la plainte soit jugée valide, il faut que les quatre protagonistes soient de taille équivalente. Boltanski explique que le dénonciateur est soumis à une contrainte de désingularisation, le dénonciateur doit se grandir. Il a deux tactiques: la première est de revendiquer pour lui un acteur collectif (une association par exemple) ; la deuxième est de rapprocher son cas d'une injustice reconnu par tous comme tel. Ces deux tactiques permettent de se généraliser, on peut noter d'ailleurs que parfois, les tentatives d'agrandissement peuvent rater. \\
Ce qui compte dans une action collective, ce n'est pas la justice de la cause mais la justesse de l'agrandissement. Paradoxalement, plus l'on trouve sa cause légitime, plus on doit veiller à l'agrandir de façon correct, donc à chercher parfois à en limiter la portée (sinon, on risque d'être non crédible). \\
Pour convaincre, la cause doit donc devenir public. Pour devenir public, un énoncé doit respecter un certain nombre de règles. Pour monter la défense d'une cause, et la publicisé, il faut que la cause passe "l'épreuve du public", D. Cefai: la publicité, c'est trois choses, ce qui est à tout le monde et qui n'est à personne. Un discours public peut donc potentiellement être reçu par tout le monde. Deuxième caractéristique de la publicité: la visibilité, ce qui est public est ce qui peut se voir de tous et par tous, il y a une réciprocité dans l'espace public entre celui qui voit et celui qui est vu. Troisième caractéristique: l'accessibilité, un espace public est un espace qui est accessible à tous. \\
L'épreuve de la publicité contraint les porteurs de revendications à structurer leurs discours d'une certaine façon. \\
On distingue trois types de publics: un public passif, de spectateurs, dont l'attention varie, auxquels on risque un jour ou l'autre de demander leur avis (référendum) ; le public agissant, qu'on voit aujourd'hui beaucoup grâce aux réseaux sociaux, ce que Weber appelle les agents politiques actifs ; enfin, on a les acteurs publics institués, ceux auxquelles une mobilisation s'adresse. 


On s'aperçoit que l'action collective a deux types de contraintes: une d'agrandissement, et une de publicité. Ces contraintes pèsent fondamentalement sur la façon dont les causes sont présentées, sur les personnes qui vont suivre la cause. Ces contraintes détermines le succès ou l'échec de la cause. \\
Devenir collectif est donc, en soi, un problème. 

\section{Mouvements sociaux, groupes d'intérêts, partis, syndicats, associations}

Ces différentes formes partagent un certain nombre de traits communs, un certain nombre de différence. \\
Une première distinction est celle entre mouvements sociaux et groupes d'intérêts. Pendant très longtemps, ce n'était pas les mêmes personnes qui travaillaient sur l'un et l'autre, il n'était pas légitime de travaille sur les groupes d'intérêts. O. Nay a publié un lexique de science politique qui définit les deux termes. Dans son lexique, on voit clairement un parti pris où il y a des gentils et des méchants. On lui a fait remarquer que les mouvements sociaux n'étaient pas forcément progressiste. 


Groupe catégoriel: situation elle même défini par l'organisation.\\
Groupe de conviction: intérêt inclusif, logique de l'adhésion volontaire. \\
Groupe territorialisé: défense d'un territoire donné. 


Les syndicats sont différents des partis.


\part{Pourquoi se mobilise-t-on ?}

\chapter{Les approches psychosociales des comportements collectifs}

\section{La grande peur des foules: sociologie et politique des temps démocratiques}

\subsection{La crainte des foules dans un temps démocratique}

À la fin du XIXe siècle, marqué par le triomphe de la République, il y a le coup du 16 Mai où McMahon tente un coup contre le Parlement. Il ne réussira pas, et en 1879, Jules Grévy sera le premier président républicain d'une nouvelle République qui repose sur le suffrage universel. En confiant le pouvoir au plus grand nombre, il se crée une crainte que les décisions prises ne soient pas rationnelles. Depuis Aristote, il y a une croyance que le nombre est réticent à la raison, il serait plutôt sensible à la démagogie. \\
À cette crainte s'ajoute une autre crainte, celle de l'individualisme, avec Drumont par exemple, qui dénonce un soit disant complot juif. \\
1792: Loi Le Chapelier, qui supprime les corporations. Cela laisse l'État et l'individu en face à face. L'État reconnaît des individus neutres, donc seulement en fonction de leur propriété politique. \\
Nisbet: La tradition sociologique. Il va relire un certain nombre de classique comme Durkheim et Comte, et va montrer que la sociologie va se présenter comme une sorte de remède à l'individualisme. Même une société d'individu, où l'individualisme est très présent, produit des normes collectives. Il existe des processus macro-sociologiques, morphologiques, de l'interdépendance, des solidarités, etc, même dans une société individualiste. 


Une troisième crainte est la peur des foules. Au XIXe siècle, ceux qui écrivent des ouvrages sont essentiellement issus de la bourgeoisie. Quand certains s'intéressent aux classes laborieuses, ils les décrivent avec dégoût, les voit comme une classe dangereuse, violente dans les relations. Dangereuse pour l'hygiène, dangereuse pour l'ordre public, dangereuse parce que révolutionnaire. Cela va être un enjeu d'aménagement urbain: c'est pourquoi Haussman troue de grandes avenues à Paris. \\
Aujourd'hui encore, les aménageurs urbains aménagent les places pour qu'elles soient sécurisés, facilement évacuable, facilement prenable d'assaut. \\
Les foules de classes populaires sont dangereuses aussi car présentées aussi comme irrationnelles. Le simple fait de réunir des individus les ferait arrêter d'être des individus rationnels. Trois auteurs vont créer des théories des foules. 

\subsection{Les théories du comportement des masses}

Hyppolite Taine, Gabriel Tarde, Gustave Le Bon créent les théories des foules. Leur premier point commun c'est que quand des individus sont réunis dans une foule, ils cessent de penser comme des individus. Le fait d'être un groupe transforme le comportement d'un individu. \\
Les trois auteurs vont parler d'hypnose, de contagion et de suggestion. Les membres de ces foules seraient comme hypnotisée, ce que Tarde propose comme la loi de l'imitation. L'hypnotiseur est le meneur et va donc émerger la figure du leader. Y. Cohen parlera de figure du chef qui émerge au XIXe siècle, pour qu'il y ait foule, il faut qu'il y ait leader. \\
Taine est un historien qui fait autorité. Il s'intéresse à la Révolution Française, il est traumatisé par la commune de Paris. Il cherche à comprendre les origines de la commune de Paris. Pour lui, les foules seraient tombées sous l'emprise des passions, entraînées par des criminels, des repris de justice. Selon Taine, l'homme a une nature animal, il craint sans cesse un retour à l'état de nature. Pour Taine, lorsque des foules se réunissent, l'homme retombe à l'état de nature. Ce retour à l'état sauvage est du à une contagion des émotions. 


Gabriel Tarde va "scientificiser" cette approche. Il propose une loi sociologique qui est la loi de l'imitation. Pour lui, toute collectivité humaine repose sur une loi de l'imitation: "la société pourrait se définir comme une collection d'êtres en tant qu'ils sont en train de s'imiter entre eux". Pour lui, l'homme en société est comme un individu somnambule, il croit avoir des idées propres alors qu'elles sont en fait suggérés par d'autres. \\
Tarde va se poser la question de ce qui fait qu'un rassemblement devient une foule. Pour commencer, il faut que le groupe dispose d'une communauté qui les prédispose à s'assembler. La foule dispose d'une communauté d'action à contrario du rassemblement. Ensuite, il faut un "atmosphère moral du moment", ce serai un faisceau d'idées à fortes charges émotionnelles d'après Tarde. Enfin, il faut qu'il y ait une étincelle. \\
Ces trois caractéristiques réunies, la foule se crée et ne fait que suivre ce qui vient d'émerger: une conscience collective commune. Tarde explique que cette foule s'en portera mieux si elle a été correctement travaillé par celui qui saura saisir cette foule. C'est là qu'un mécanisme d'imitation se crée, chaque individu imitant son semblable, son voisin, pour faire de la foule un espace compact. 


Gustave Le Bon va écrire "psychologie des foules" en 1895, qui sera un réel succès éditorial. Dans ce livre, Le Bon pioche allègrement ses théories chez Taine et Tarde. Le Bon emprunte tout un état d'esprit, et ce qui est flagrant chez lui, c'est la peur des foules, de la démocratie. Pour lui, les foules sont destructrices de la civilisation elle même. Le Bon applique la notion de foule un peu partout par exemple comme les membres d'un jury d'assises, les arènes parlementaires. \\
Pour Le Bon, ce n'est pas seulement la foule qui est dangereuse, c'est la démocratie aussi. Pour lui, tout ce qui est moderne est dangereux. Il va populariser un rejet de la démocratie. \\
Il va proposer l'idée que les foules sont hypnotisées par le leader. Ce qui l'intéresse c'est comment un leader hypnotise une foule. Pour lui, il y aurai trois techniques: une idée simple, qui s'applique à tout (affirmation d'un dogme), la répétition sans cesse de ce dogme sous une forme imagée et frappante, et enfin, favoriser la contagion des propos pour créer un courant d'opinion. Le terme qui sera inventé plus tard pour caractériser ça sera propagande, par S. Tchakotine dans son livre "Le viol des foules". Depuis, le modèle a été affiné. \\
Le Bon va scientificiser cela en se basant sur le travail de Charcot qui travaille sur l'hypnose. Ce que Charcot démontre de manière individuelle, Le Bon va essayer de l'adapter auprès du grand nombre. Pour Le Bon, la foule est le sujet hypnotisé par le leader, créant une dépendance entre "le maître et son troupeau". \\
Le Bon prétend avoir trouvé le moyen de gouverner les foules. Son dernier chapitre s'achève "Connaître l'art d'impressionner l'imagination des foules, c'est connaître l'art de les gouverner". \\
Ses idées vont inspirer des théoriciens totalitaires. 


Park publie en 1921 un traité de sociologie général, où l'on trouve un chapitre sur de la sociologie collective. On va trouver chez Park une référence au travail de Le Bon, mais en gardant seulement l'idée que c'est par le collectif que se produit le changement social. Il garde l'idée qu'il y a un maître de la foule, que les individus dans la foule n'ont pas tout à fait le même comportement. Cependant, Park a une vision plus optimiste de la foule en disant que pour un changement social, il faut qu'il y ait un passage collectif. \\
Park a comme étudiant Herbet Blumer, un des premiers à faire de la sociologie des médias et de la communication. Il va commencer par introduire une distinction entre ce que la théorie des foules appelait les foules, la masse, le public, qu'il distingue avec les mouvements sociaux. Pour lui, les mouvement sociaux ont une intention, qui cherchent à établir un nouvel ordre de vie alors que les foules, les masses, les publics, ne sont pas forcément réunis autour d'un agir ensemble intentionnel. \\
Il distingue les mouvements généraux (ouvriers, pacifistes etc), et des mouvements spécifiques qui portent sur des revendications particulières et qui reposent sur une appartenance et une conscience commune. Pour Blumer, participer à un mouvement procure un certain nombre de gratification: un frisson, un enthousiasme, la sensation d'être membre d'un collectif. \\
Blumer va proposer la notion de réaction circulaire, qui en fait une loi de l'auto-renforcement, chaque individu reproduit la stimulation d'un autre individu. Cela peut expliquer les phénomènes de radicalisation: en se reflétant réciproquement, ils s'intensifient. \\
Idée de carrière: agitation sociale, exaltation populaire, formalisation (organisation), l'institutionnalisation passant par sa reconnaissance par les pouvoirs publics. \\
Blumer va mener son enquête sur un cortège de célébration d'un général revenu victorieux. Il va passer un questionnaire dans la foule. Il va essayer de montrer que les individus partagent quelque chose. Il tire deux conclusions: les raisons de la présence des individus est loin d'être liée à l'événement lui même. Les individus sont capables de dire pourquoi l'événement a eu lieu néanmoins. En revanche, les individus ont beaucoup plus de mal à dire pourquoi le grand jour est grand pour eux. Le message produit reste extérieur. \\
Selon Blumer, il y a deux éléments en un seul: l'événement en lui même, et ceux qui le regarde.


Kornhauser publie The Politics of Mass Society en 1951. C'est un Allemand qui a quitté l'Allemagne. Selon lui, sous l'effet des médias de masse (radio, cinéma), nous serions entrés dans l'ère des sociétés de masse. Pour lui, les images du parti Nazi de 1930, sont des images traumatisantes, et va chercher à montrer que le modèle nazi se reproduit dans d'autres sociétés qui ne sont pas des sociétés totalitaires. \\
Ces sociétés de masse sont des sociétés qui ne sont ni démocratiques ni totalitaires, où les individus sont manipulés par les moyens de communication. Ce qui caractérise les sociétés de masse, c'est l'absence d'organisation sociale: il n'y a pas de corps intermédiaire structurant les relations entre l'État et les classes populaires, laissant ces derniers en pâture face à la puissance de l'État. Isolé, les individus isolés, n'ont pas accès au pouvoir et donc ne peuvent pas résister à la propagande étatique. \\
Cette thèse a été partiellement démonté par une analyse plus fine du régime Hitlérien, qui n'était pas en fait une société de masse, le régime se basant en fait sur des classes intermédiaires. 

\subsection{Ted Gurr et le modèle de la frustration collective}

Gurr a écrit Why Men Rebel? \\
Cet ouvrage est la conséquence d'une étude de presque une décennie et financé par le gouvernement américain qui est inquiet de la montée de la conflictualité sociale (contre la guerre au Vietnam, les mouvements des droits civiques, etc). Jusque là, la Science Politique américaine avait insisté sur une notion appelé l'apathie, qui est l'idée que les régimes démocratiques ne sont soutenables que dans la mesure où les individus ne participent pas trop: c'est le courant behavioraliste. Si les individus commencent à trop participer, ils risquent de trop exiger auquel le gouvernement ne va pas pouvoir répondre, augmentant la frustration vis à vis du régime. \\
Son étude montre que les américains s'intéressent très peu à la politique, ont peu de compétences politiques et peu d'américains sont impliqués dans un parti. \\
L'ouvrage de Gurr cherche donc à répondre à un problème: le passage d'un état apathique à un état conflictuel. 


Il s'intéresse essentiellement à la question de la violence. Ce qui l'intéresse, c'est le surgissement de la violence. Il va poser trois types de questions: quelles sont les sources de la violence ? Comment passe-t-on de la violence collective à la violence politique ? Comment expliquer l'intensité de la violence politique ? \\
Gurr va distinguer trois types de violences: le turmoil, une violence spontanée ; la conspiration, une violence hautement organisé avec une faible participation populaire ; la guerre interne, une violence politique hautement organisée avec une forte participation, de grande intensité. \\
Pour Gurr, la violence naît d'un mécontentement qui va se politiser et conduire à un passage à l'acte dirigé contre le acteurs politiques. \\
Ce mécontentement vient selon Gurr de la frustration relative: la différence perçue entre les attentes des individus et ce qu'ils pensent être en mesure d'obtenir. Tant que les individus ne perçoivent pas cette différence, ils ne perçoivent pas de frustration relative. \\
La première arme que donne Gurr au gouvernement, c'est qu'il leur explique que la population doit ne pas percevoir une telle frustration. \\
Ce mécontentement peut avoir des intensités différentes et donc pareil pour la violence. Cela dépend des facteurs sociologiques et psychologique. Les facteurs psychologiques reposent sur la solidité des croyances. Les facteurs sociologiques sont les sanctions qui peuvent être imposer (sociales, morale, juridique), la facilité d'utiliser la violence, et enfin, la légitimité du pouvoir en place. \\
Gurr pose une hypothèse majeure: le potentiel de violence collective varie fortement selon l'étendu et l'intensité de la frustration parmi les membres d'une collectivité.


Gurr identifie trois types de valeurs: les welfare values (les valeurs relatives au bien être), les power values (les valeurs relatives au pouvoir, l'influence que l'on a sur la marche des choses mais aussi l'autonomie que l'on dispose), les interpersonal values (la capacité pour un individu d'avoir des interactions non autoritaire avec d'autres individus). \\
Gurr met ensuite en évidence que les individus perçoivent leur capacité à atteindre ses objectifs, ces valeurs. Ces capacités sont dépendantes de la position de l'individu, et de ce qu'il peut potentiellement espérer obtenir. Ces capacités peuvent être mesurés à partir de ce que l'on peut appeler des opportunités, qui dépendent de trois facteurs: le niveau d'opportunité personnelle, les opportunités sociales, les opportunités politiques. À partir de ces trois éléments, ils vont percevoir leurs chances sociales. \\
Il distingue donc trois types de privations relatives: decremental deprivation, la demande reste stable alors que la situation sociale se dégrade. Aspirational deprivation, le niveau d'aspiration évolue alors que la satisfaction des besoins ne bougent pas. La progressive deprivation, où on a pendant un temps, une augmentation conjointe de la satisfaction des besoins et de l'aspiration, ensuite, l'aspiration va continuer d'augmenter alors que la satisfaction des besoins va chuter. Ce dernier est le plus à même de créer de la violence collective. \\
Ces trois types de privations relatives n'ont pas les mêmes effets, Gurr explique que cela dépend de l'intensité de sa motivation. Plus la motivation est élevé, plus la privation sera importante. L'intensité de la violence collective va aussi dépendre du type de valeurs que recherche les individus, pour Gurr, la privation relative est plus grande et donc la violence plus forte, d'abord pour les questions d'égalité économique (welfare), ensuite sur le power, et enfin sur les relations individuelles. 























\part{Comment se mobilise-t-on ?}

\end{document}
