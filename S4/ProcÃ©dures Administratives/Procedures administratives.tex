\documentclass[10pt, a4paper, openany]{book}

\usepackage[utf8x]{inputenc}
\usepackage[T1]{fontenc}
\usepackage[francais]{babel}
\usepackage{bookman}
\usepackage{fullpage}
\setlength{\parskip}{5px}
\date{}
\title{Cours de Procédures Administratives (UFR Amiens)}
\pagestyle{plain}

\begin{document}
\maketitle
\tableofcontents

\section{La diversité des domaines du recours de plein contentieux}

Il existe une certaine variétés de recours au plein contentieux, certains distingue le recours objectif et subjectif, ce qui peut être périlleux. On distinguera plutôt le recours contractuel et extracontractuel. \\
Le premier étant un recours contre un acte lié à un contrat de la personne publique ou au contrat lui même.. 

\subsection{Les recours de plein contentieux en matière extra-contractuelle}

Soit on fait un REP soit on fait un plein contentieux. \\
Un recours qui tend à faire condamner l'Administration par le versement dommages-intérêt est un recours de plein contentieux, que l'on peut aussi classer dans les contentieux subjectifs. \\
L'Administration a donc une responsabilité, elle doit réparer les dommages si elle en cause. \\
Un recours qui tend à donner une somme d'argent à un administré en application d'un texte légal (législatif ou réglementaire), est un recours de plein contentieux: pensions, rémunérations, allocations, indemnité, etc. \\
En plein contentieux, le juge a une grande marge de manoeuvre, le juge peut donc fixer une certaine somme, et ordonner son paiement à l'administration. \\
Les recours dirigés contre les actes qui réclament une somme à l'administration sont aussi de plein contentieux. Cela recouvre le contentieux fiscal. Le juge pourra annuler les actes ou bien moduler la somme due à l'Administration.


Le contentieux électoral est un plein contentieux. Le juge peut annuler une élection, rectifier le résultat, l'ordre d'arrivée des candidats, voire même demander des dommages et intérêts si il y a eu un préjudice. C'est là un contentieux objectif. \\
Récemment, le CE a fait basculer un domaine qui relevait du REP: les actes qui donnaient une sanction à un administré étaient du domaine du REP mais fait désormais partie du plein contentieux: CE, Ass., 16 Février 2009, Société Atom. Le juge souhaitait utiliser les prérogatives du juge de plein contentieux et non du juge du REP (qui doit juger au jour de l'édiction de l'acte). Cela ne concerne pas toutes les sanctions prises, comme les MOI. \\
Le CE a décidé dans plusieurs affaires consécutives de scinder en deux le contentieux des sanctions. Les sanctions portant aux administrés relèvent du plein contentieux. En revanche, les mesures disciplinaires relèvent du REP: agents de l'État (CE, Ass., 13 Novembre 2013, Dahan), des détenus (CE, 1er Juin 2015, M. Boromée), ou des membres des fédérations sportives (CE, 2 Mars 2010, Fédération française d'athlétisme). 


\subsection{Les recours de plein contentieux en matière contractuelle}

Traditionnellement, les recours de plein contentieux en matière contractuelles étaient réservés aux tiers. Cependant, depuis 2007, l'accès des tiers au juge du contrat (qui est de plein contentieux), s'est progressivement ouvert. \\
Depuis 2009, cette matière a beaucoup évolué.

\subsubsection{Les recours des parties}

On compte deux recours, un recours en déclaration d'invalidité, qui est un recours d'un plein contentieux. Ce recours permet, en principe, la disparition rétroactive du contrat, mais on peut demander aussi des intérêts. \\
La nullité d'un contrat est en principe demandé par une partie. Elle le fait car elle souhaite s'affranchir d'une obligation que lui impose le contrat, ces stipulations n'étant pas dans ses intérêts. \\
Soit le juge déclarait la nullité du contrat, soit il le déclarait légal. Il peut aussi constater des fautes et imposer des dommages-intérêts.


Désormais, le juge a plus d'options, depuis CE, Ass., 28 Décembre 2009, Commune de Béziers. Le juge a une certaine marge de manoeuvre. Le CE a réformé le droit contractuel à cause de l'exigence de loyauté qu'implique un contrat. Le juge a pris plusieurs choses en considération: il y a des vices graves (comme un vice de consentement), et des vices mineurs (de procédure). \\
Le juge s'est aperçu que des parties cachaient des clauses illégales dans le contrat pour les faire annuler, ce qui est un comportement déloyal. Cela renvoie à l'exigence de bonne foie en droit privé. \\
La logique est la même que la jurisprudence de l'association "assez", le juge souhaite plus d'alternatives pour s'adapter. \\
Le juge du contrat a plusieurs pouvoirs: d'abord, il peut décider que la poursuite de l'exécution est possible sous réserve de régularisation (prises par la personne publique ou conclues entre les partis), permettant la poursuite et la stabilité de la relation contractuelle. Le juge du contrat n'a pas le droit de modifier les clauses du contrat car le coeur du contrat est que deux volontés se rencontrent, si le juge s'en mêle, la volonté originelle est bouleversée. \\
Ensuite, il peut prononcer la résiliation ou l'annulation du contrat avec un effet différé. 


Les mesures d'exécution du contrat sont des mesures prises par la personne publique en lien avec le contrat. Pendant très longtemps, le juge de plein contentieux estimait qu'il disposait juste du pouvoir de condamner l'administration à des dommages-intérêts si ces mesures étaient dommageables. \\
Cependant, le contractant préfère parfois que le contrat continue plutôt que d'avoir des dommages-intérêts. \\
Exceptionnellement, certaines mesures d'exécution pouvaient faire l'objet de conclusions en annulation. Le contractant pouvait donc contester des mesures. \\
Depuis 2011, le recours de plein contentieux des contrats a été rénovée. Désormais, le juge du contrat dispose, de façon générale, à la demande de la partie lésée, des pouvoirs d'annuler et de suspendre la mesure d'exécution du contrat, en vue de la reprise des relations contractuelles: CE, Sect., 21 mars 2011, Commune de Béziers II. 

\subsubsection{Les recours des tiers}

Pendant longtemps, il n'y avait aucun recours de plein contentieux ouvert aux tiers. Les possibilités pour un tiers de contester un contrat étaient très limitées, mais a été élargies. Ceci s'explique notamment par la concurrence qui se fait dans l'objectif de décrocher les contrats publics, et donc de laisser la possibilité aux concurrents de remettre en cause ces contrats si ils se sentent lésés dans la concurrence. \\
Dès 1905, les REP sont possibles contre les actes détachables du contrat, c'est à dire un acte non contractuel, de l'administration et lié au contrat: CE, 4 Août 1905, Martin. Cependant, on ne pouvait pas obtenir l'annulation du contrat. Cette voie avait donc assez peu de conséquences. \\
Le déféré préfectoral contre un contrat est apparu en 1982, où les préfets pouvaient contester les actes contractuel et les actes unilatéraux des collectivités locales. \\
On a ensuite créé les référés précontractuels et contractuels, dont le but est de protéger le principe de non discrimination, en application du droit de l'Union. Ces référés ont été créés dans le but de permettre à des entreprises européennes de contester les contrats de ses concurrents qui gagnaient ces contrats. Ces référés ont pour but de vérifier que la mise en concurrence n'a pas été biaisée par de la discrimination, des pots de vin, etc. Cette procédure concerne des contrats qui engagent des sommes conséquentes. \\
CE, Ass., 10 Juillet 1996, Cayzeele, va ouvrir la voir au REP contre les clauses réglementaires des contrats. \\
CE, Sect., 30 Octobre 1998, Ville Lisieux, ouvre la voie au REP contre le contrat d'engagement d'un agent public. \\
CE, Ass., 16 Juillet 2007, Société Tropic Travaux signalisation, permet d'ouvrir le RPC aux concurrents évincés d'un contrat pour qu'ils en contestent la validité ou certaines de ses clauses. Cela donne un pouvoir large au juge de plein contentieux. 


Le juge du contrat peut: \\
Prononcer la résiliation du contrat ou modifier certaines de ses clauses. C'est la première et dernière fois que la jurisprudence l'admet. \\
Prononcer la poursuite du contrat, éventuellement sous réserve de mesures de régularisation par la collectivité contractante. \\
Accorder des indemnisations en réparation des droits lésés. \\
Si l'annulation du contrat ne porte pas atteinte de manière excessive à l'intérêt général ou aux droits des cocontractants, il peut annuler totalement ou partiellement avec un effet différé ou non, le contrat. 


Récemment, la JP CE, Ass., 4 Avril 2014, Département de Tarn-et-Garonne, permet à "tout tiers à un contrat administratif susceptible d'être lésé dans ses intérêts de façon suffisamment directe et certaine par sa passation ou ses clauses" le RPC en contestation de validité du contrat, lequel peut être assorti d'une demande de suspension et de conclusions indemnitaires. \\
En fonction des vices qui affectent la validité du contrat et des conséquences de sa décision, le juge peut: inviter les parties à prendre des mesures de régularisation, prononcer la résiliation du contrat, prononcer l'annulation totale ou partielle du contrat, condamner à indemniser le préjudice découlant de l'atteinte à des droits lésés. 

\section{Les demandes accessoires}

Une demande accessoire est une demande que l'on fait en complément d'une autre demande. Un référé-suspension est par exemple une demande accessoire d'un jugement au fond qui est la demande principale. 

\subsection{Les référés}

Le CJA ne renseigne pas grand chose sur les procédures principales. Cependant, les référés sont bien définis dans le CJA. \\
Deux articles sont communs à tous les référés. Le premier est le L511-1 du CJA qui dispose que les mesures du juge des référés est provisoire, sont données dans les meilleurs délais, et n'est jamais saisi du principal. Le juge des référés délivre donc des ordonnances et peut revenir sur sa décision. \\
Un référé hybride va être un référé qui ne respecte que certaine de ces caractéristiques. Le référé contractuel est un juge qui rend sa décision dans les meilleurs délais, mais est saisi du principal. Le référé liberté rend souvent des mesures définitives. \\
L'art. L511-2 du CJA désigne les juges des référés. Le juge des référés est un juge unique et un juge expérimenté. Il existe parfois des formations collégiales, mais cela est rare. Le juge unique permet d'être rapide. 

\subsubsection{Les référés d'urgence}

Ils ont été profondément réformés par la loi du 30 Juin 2000. Trois référés sont prévues en urgence, le juge doit statuer en urgence. \\
D'abord, on a le référé suspension, puis le référé libertés, et enfin le référé mesures utiles. 


Le référé suspension est le premier évoqué par le CJA dans l'art. L521-2, al. 1er. Il y est disposé que c'est un référé accessoire à une demande principale. Ce référé est soumis à deux conditions: il faut qu'une urgence justifie la suspension et il faut qu'il y ait un doute sérieux sur la légalité de la décision. La condition d'urgence a été explicité par CE, Sect., 29 janvier 2001, Confédération nationale des radios libre, pour le CE, la décision contestée doit préjudicier "de manière suffisamment grave et immédiate à un intérêt public, à la situation du requérant ou aux intérêts qu'il entend défendre". \\
Si les deux conditions sont réunis, la suspension se fait jusqu'à ce que la demande principale soit traitée. Une fois traitée, la suspension disparaît, soit parce que l'acte litigieux a été annulé, soit parce qu'il a été validé.


Le référé liberté prévoit dans l'art. L521-2, al. 2, que toutes mesures peuvent être prises pour la sauvegarde d'une liberté fondamentale qui aurait subi une atteinte grave et manifestement illégale, le tout doit être justifié par l'urgence. \\
L'atteinte doit, en principe, être manifestement illégale et porter atteinte sur une liberté ou sur un droit dont la protection est garanti par une norme constitutionnelle, un engagement international, ou un principe général du droit (CE, 12 Novembre 2001, Commune de Montreuil-Bellay). \\
Le juge a là une marge de manoeuvre importante ("toutes mesures nécessaires"). \\
Le référé liberté a beaucoup de succès et a été une grande réussite du contentieux administratif. L'objectif était de créer une voie rapide et efficace pour protéger les droits et libertés dans le cadre de l'utilisation régulière des prérogatives de puissance publique. Si l'utilisation n'est plus régulière, que ce n'est plus dans l'exercice de ses pouvoirs, il existe la voie de fait devant le juge judiciaire.


Jusqu'en 2013, application de TC, 8 Février 1935, Action française, le juge judiciaire était compétent en cas de: exécution forcée d'une décision décision de l'Administration, même régulière, dans des conditions irrégulières portant atteinte au droit de propriété ou à une liberté fondamentale ; décision elle même manifestement insusceptible d'être rattachée à un pouvoir de l'administration portant une atteinte grave au droit de propriété ou à une liberté fondamentale. \\
En 2013, le JA ayant fait ses preuves, le TC a réduit le champ de la voie de fait ci dessus. TC, 17 Juin 2013, Bergoend c/ Société ERDF, le juge judiciaire est compétent en cas: d'exécution forcée d'une décision, même régulière, dans des conditions irrégulières, portant atteinte à la liberté individuelle ou aboutissant à l'extinction d'un droit de propriété. Liberté individuelle=ne pas être détenu. Extinction de propriété=expropriation. Et en cas d'une décision manifestement insusceptible d'être rattachée à un pouvoir de l'Administration et portant atteinte à la liberté individuelle ou aboutissant à l'extinction d'un droit de propriété. 


Le référé liberté est doublement original car ce n'est pas un recours accessoire. On peut former un référé liberté sans demande principal. \\
Les mesures ordonnées doivent en principe être provisoire mais elles peuvent être définitives, dès lors que des mesures provisoires ne permettent pas de faire cesser l'atteinte constatée (CE, 30 mars 2007, Ville de Lyon). 


Le dernier référé d'urgence est le référé-mesures utiles, il est défini à l'art. L521-3 CJA. Il ne peut pas ordonner la suspension d'une décision de justice. Il a le pouvoir d'ordonner à l'administration des mesures utiles sans faire obstacle à la mise en oeuvre de la décision de l'administration. Elle est utilisée essentiellement pour faire communiquer un document en urgence à l'administré ou dans les travaux publics lorsqu'un bâti menace de s'effondrer quelque part. \\
Le juge des référé peut alors imposer à l'administration de prendre des mesures utiles comme la mise en oeuvre de sécurisation. 

\subsubsection{Les référés de droit commun, les référés ordinaires}

Ils sont ordinaires car il n'y a pas d'urgences à prouver pour obtenir une ordonnance de la part du juge. 


Le premier référé est le référé constat, qui permet au juge de désigner un expert pour constater des faits susceptibles de donner lieu à un litige devant la juridiction administrative. \\
Le référé-instruction permet de demander au juge de prescrire une mesure d'instruction ou une mesure d'expertise. Son domaine d'action est plus large que le référé précédent. \\
Le référé provision permet au juge d'accorder une avance au créancier sur une somme qu'il estime lui être due. L'avance peut lui être donné si la créance qu'il détient n'est pas sérieusement contestable. 

\subsubsection{Les référés spéciaux}

À la différence des deux précédents, ils concernent certaines matières spécifiques. Ils sont proches des autres référés mais s'adaptent à la matière. \\
On a par exemple le référé contractuel et le référé pré-contractuel, le référé en matière fiscale, en matière de communication audiovisuelle, suspension du préfet, suspension en matière d'urbanisme et de protection de la nature ou de l'environnement. \\
Le référé pré-contractuel est une transposition de la directive du 21 Décembre 1989, transposé en 1992. En 2000, il a été codifié dans le CJA. Il vise à permettre un respect des principes de l'UE en matière de marché public, de non discrimination, de transparence. Ces référés permettent une réponse rapide. \\
Le pré-contractuel est une saisie avant le contrat et concerne que certains contrats (des montants importants), en vue de prévenir sa passation en cas de manquement au droit applicable. Le juge peut ordonner à l'auteur du manquement (l'administration, presque toujours), de se conformer à ses obligations, suspendre et annuler l'exécution de toutes décisions qui se rapporte à la passation du contrat, et supprimer les clauses et prescriptions destinés à figurer dans le contrat. \\
Le référé contractuel a été introduit en 2009, transposant une directive de 2007. Là aussi, il ne concerne que les gros contrats. On ne peut pas former un référé contractuel après un référé pré-contractuel. Ce référé contractuel est réservé aux tiers, seuls les manquements les plus graves peuvent être invoqués, comme l'absence totale de publicité ou de mise en concurrence. Le juge peut annuler le contrat, le résilier, réduire la durée et même condamner l'auteur de l'irrégularité à une pénalité financière. 

\subsection{Les voies d'exécution}

De façon générale, une voie d'exécution est une procédure par laquelle un créancier peut obtenir du débiteur son exécution. \\
En matière administrative, une voie d'exécution est surtout quand un administré demande au juge de garantir l'exécution d'une décision juridictionnelle. L'administration ne fera pas appel à ce juge car dispose de l'avantage du préalable (qui est une PPP). \\
L'injonction est un ordre émanant d'une autorité habilitée. L'astreinte est une condamnation pécuniaire éventuelle par jour de retard dans l'exécution d'une décision de justice ou d'une injonction juridictionnelle, c'est donc une menace. \\
Le pouvoir de prononcer des astreintes est un principe général du droit régissant l'organisation et le fonctionnement des juridictions: CE, Ass., 10 Mai 1974, Barre et Honnet, en vertu duquel le JA peut condamner une personne privée à une obligation de faire sous astreinte. Pendant longtemps, le juge ne voulait pas prononcer d'astreintes à l'encontre de l'administration. Aujourd'hui, le livre IX du CJA s'applique aux personnes morales de droit public et distingue plusieurs catégories de procédures. \\
En premier lieu, il existe des procédures tendant à obtenir une injonction avec ou sans astreinte, profondément réformées en 1995. Ces procédures sont soit préventives, soit curatives. Les procédures préventives ont étés créées en 1995. Pour qu'une telle mesure préventive soit prise, elle doit être demandée par l'administré qui forme le recours. \\
L'astreinte accompagne une injonction, car l'injonction sans astreinte ne vaut pas grand chose.



Les procédures curatives ont étés créées en 1980. Avant cette loi, il n'y avait pas de voie juridique pour demander le paiement de son dû à l'administration. \\
Depuis 1995, elles diffèrent selon qu'elles visent l'inexécution d'une décision du CE ou bien d'une décision d'un TA ou d'une CAA. Les TA et les CAA sont donc désormais compétentes pour mettre en oeuvre des procédures curatives. \\
Il existe des procédures tendant à obtenir un mandatement d'office ou un paiement sur présentation du jugement. Lorsqu'une décision juridictionnelle passée en force de chose jugée condamne une personne publique à payer de l'argent, cette somme doit être payée dans les deux mois. À défaut, l'article 1er de la loi du 16 Juillet 1980 prévoit deux procédés. \\
S'agissant de l'État, le comptable assignataire doit procéder au paiement à la demande du créancier sur simple présentation de la décision de justice. Quand il s'agit de collectivités locales et d'établissements publics, c'est le représentant de l'État dans le département ou l'autorité de tutelle qui procède au mandatement d'office. En cas d'insuffisance de crédit, le représentant de l'État ou l'autorité de tutelle met en demeure la collectivité ou l'établissement public de créer les fonds. Si les crédits ne sont pas créés, il y pourvoit directement et procède au mandatement d'office. 

\section{Les voies de recours}

C'est une expression passée dans le langage courant mais trompeuse. Les voies de recours concernent un recours sur une décision de justice. On distingue deux grandes catégories de recours.

\subsection{Les recours devant une juridiction supérieure}

\subsubsection{L'appel}

Toute partie présente dans une instance ou qui y a été appelé peut interjeter appel (Art. R. 811-1, al 1er, CJA). Cela consiste à contester la décision rendue par un TA. Il doit être formé dans un délai de deux mois à compter du rendu de jugement, et en référé, le délai est de quinze jours. \\
Pendant longtemps, l'appel ne comptait aucune limitation. Depuis un décret de 2003, certains litiges (comme le permis de conduire ou la pension), ne sont pas susceptibles d'appel. Question: est-ce conforme à un recours effectif ? CC, CEDH, CE, ont répondu que oui. L'existence du recours en cassation a lui, une valeur constitutionnelle. \\
Les CAA ont étés créées en 1987 et mis en oeuvre en 1989. C'est donc petit à petit que le CE a donné compétence aux CAA. Depuis le 1er Avril 2015, les appels devant le CE concernent seulement deux hypothèses: sur les litiges relatifs aux élections municipales et départementales ; sur les ordonnances de référé-liberté. 


L'appel vise: \\
En premier lieu, à demander à la juridiction d'appel de vérifier la régularité de la procédure suivie et de la décision rendue en premier ressort. Si il y a irrégularité, il peut annuler, sinon, il passe à la question suivante. \\
En second lieu, l'appel vise à statuer sur le recours présenté devant le juge de premier ressort. L'appel a un effet dévolutif: le juge d'appel est saisi de l'ensemble du litige qu'il va juger une seconde fois avec les mêmes pouvoirs que le premier juge. Il rejuge donc entièrement l'affaire. \\
Si il y a eu un problème de forme, la première question, la CAA renvoie à un TA ou juge lui même directement au fond. Si il n'y a pas de problème, il passe à la deuxième question et statue au fond.

\subsubsection{Le recours en cassation}

De manière générale, toutes les décisions rendues en dernier ressort par les juridictions administratives peuvent être déférés au CE par la voie du recours en cassation (art. L821-1 CJA). \\
Le recours en cassation est ouvert de plein droit contre toutes les décisions des juridictions administratives (CE, 2 Février 1947, d'Aillières), sauf pour ce qui concerne les décisions du CE et des décisions qui peuvent faire l'objet d'un appel. \\
La cassation ne rejuge pas le litige, mais juge la régularité de la décision. Le juge de cassation veille donc à la régularité externe et interne de la décision de justice. \\
Les questions de pur faits "relèvent de l'appréciation souveraine des juges du fond". Ce principe n'existe pas vraiment dans un REP, ce sera surtout dans les litiges où des responsabilités sont en causes. Le juge de cassation contrôle, notamment l'exactitude matérielle des faits sur lesquels s'est fondé le juge du fond ainsi que la qualification juridique des faits opérée par lui. \\
Légalité externe: incompétence, vice de procédure, etc. Légalité externe: détournement de pouvoir, contrôle des motifs (contrôle de la qualification des faits). \\
Comme le dispose l'art. L821-2, al. 2 du CJA, le CE, en cassation, si il annule, le CE peut renvoyer, devant la même juridiction ou une autre, ou bien il peut régler l'affaire lui même au fond, ce qu'il fait le plus souvent. Il le fait pour deux raisons: d'abord, gagner du temps ; ensuite, le CE, contrairement à la C.Cas, est habitué à régler les affaires au fond. 

\subsubsection{Le recours dans l'intérêt de la Loi}

Ce recours consiste à demander au CE la remise en cause d'une décision de justice afin qu'elle soit à l'origine d'une jurisprudence. Il ne s'agit pas de remettre en cause le fond de la décision, c'est la portée de la décision que l'on souhaite limiter. Seul un ministre peut faire ce recours. \\
Ce recours est apparu avec le ministre-juge. Elle est donc probablement tombé en désuétude, surtout qu'il existe depuis 1987 l'avis de contentieux. 

\subsubsection{Le règlement de juge}

C'est une voie de recours visant à éviter la contrariété de deux jugements, l'un étant définitif, et l'autre susceptible de recours. \\
C'est dans l'hypothèse où deux décisions sont contradictoires, et la première décision n'est plus susceptible d'appel car le délai est passé. C'est à la partie dans le jugement non définitif de faire un recours devant la juridiction compétente pour statuer en appel sur la première décision rendue, qui était définitive. \\
La juridiction saisie devra déclarer nulle et non avenue la décision de justice incriminée et renvoyer devant la juridiction compétente pour connaître de la demande. 

\subsection{Les recours en rétractation}

Ce sont des recours formés contre la juridiction qui a pris une décision. En principe, une décision de justice a autorité de chose jugée, ce qui interdit au juge de se re-saisir d'une affaire. Cette procédure est donc très encadré et concerne très peu d'hypothèses. 

\subsubsection{L'opposition}

C'est l'hypothèse où une décision a été rendue contre une personne qui n'a pas pu produire de défense. Ce recours vise à respecter le principe du contradictoire. Il n'est pas possible de former ce recours si il y a eu une partie qui s'est défendue et qui avait les mêmes intérêts que la personne qui n'a pas produit de défense. (art. R831-1 CJA). Il n'est pas possible non plus de faire ce genre de recours devant un TA. \\
Si l'opposition est fondée, le juge déclare sa précédente décision comme nulle et non avenue, et il faudra ressaisir le juge pour que l'affaire soit juger. 

\subsubsection{La tierce opposition}

Cela est ouvert à une personne qui a été mise en cause dans une affaire, elle n'est pas été appelé ou même cité dans l'instance, mais la décision préjudicie à ses droits. Dans ce cas, elle peut former tierce opposition. L'objectif est là aussi le respect du contradictoire et possible devant toutes les juridictions administratives. \\
Si le recours est retenu, là aussi la précédente décision est nulle et non avenue. 

\subsubsection{Le recours en révision}

Ce recours vise à demander au CE de déclarer nulle et non avenue une de ses précédentes décisions. Le recours n'est possible que devant le CE. \\
Ce recours est différent des deux premiers car ne vise pas exclusivement à protéger le principe du contradictoire. \\
Il peut être former dans des cas très précis: si la décision du CE a été rendue sur des pièces fausses ; si la partie a été condamné faute d'avoir produit une pièce décisive qui était retenue par son adversaire ; si la décision est intervenue sans qu'aient été respecté les règles relatives à la composition de la formation de jugement, à la tenue de l'audience, ainsi qu'à la forme du prononcé de la décision (fautes formelles). 

\subsubsection{Le recours en rectification d'erreur matérielle}

Il vise à faire corriger une erreur matérielle. Ce recours ne peut se faire que devant une CAA ou le CE. \\
Ce recours peut se faire si un nouvel élément matériel pourrai inverser la décision. \\
En 1997, un décret a créé une forme d'auto-saisine du juge en vue de de rectifier une erreur matérielle non susceptible d'avoir exercé une influence sur le jugement de l'affaire. En 2010, cette auto-saisine a été étendue aux TA car cette auto-saisine est fortement pratique pour corriger une minute d'un jugement en cas d'erreurs. 

\chapter{La recevabilité du recours}

Souvent utilisé en droit, la recevabilité est rarement défini. \\
Le juge est toujours saisi d'une demande, c'est à dire d'une ou plusieurs conclusions (prétentions), que l'on demande au juge de confirmer en affirmer son ou leur bien fondé. \\
Avant de juger le bien fondé, le juge doit juger de la recevabilité: la question peut être être posée au juge ? Le requérant a-t-il respecté les règles de procédure ? \\
La recevabilité au sens strict du terme, porte sur tout élément subordonnant l'examen du bien fondé d'une prétention par le juge compétent. \\
Le juge doit répondre à trois questions: suis-je compétent ? La demande est-elle recevable ? La demande est-elle bien fondée ? Pour certains la compétence est un élément de la recevabilité. 

\section{Les principales conditions de recevabilité}

On peut les classer en trois catégories: celles qui sont liées à la personne qui présente le recours ; celles qui sont dépendantes des prétentions de l'auteur du recours ; celles qui sont tributaires des règles d'introduction et de rédaction du recours. 

\subsection{Les conditions liées à la personne présentant le recours}

\subsubsection{La capacité à agir}

Pour les personnes physiques, à la faculté d'agir en justice sans représentant légal en application des règles du Code civil. En contentieux administratif, comme en privé, les mineurs ou les majeurs sous tutelle ne peuvent pas agir en justice. \\
Pour les groupements, la capacité à agir est conditionné au fait qu'ils doivent posséder la personnalité morale de droit privé ou public. Exception: un association dissoute peut prononcer un recours contre cette dissolution au nom de cette association qui n'a plus la personnalité morale.

\subsubsection{La qualité pour agir}

Cela se rapporte à la faculté de représenter un sujet de droit devant une juridiction. La représentation résulte de la loi ou bien d'un contrat de mandat. Des parents peuvent par exemple représenter leurs enfants en justice parce que le code civil le prévoit. \\
Une personne qui dit en représenter une autre doit justifier de sa qualité pour le faire. \\
Les avocats peuvent représenter les parties sans avoir à justifier de leurs mandats, ils sont dit "crus sur robe". 

\subsection{Les conclusions liées aux conclusions de l'auteur du recours}

\subsubsection{La représentation obligatoire par un avocat}

En fonction de l'objet de ses conclusions, le requérant doit se faire assister par un mandataire de justice. On dit alors que le ministère d'avocat est obligatoire. \\
Cela permet d'augmenter la qualité de la justice mais aussi de réguler le nombre de demandes devant les tribunaux. \\
Un décret du 2 Novembre 2016 modifie la recevabilité en matière administrative, il a eu tendance à augmenter les cas de représentations obligatoire. \\
L'État est toujours dispensé de représentation par un mandataire de justice. 


Devant les juridictions administratives, une partie peut, voire doit, se faire représenter par un avocat à la cour ou un avocat aux conseils. Il n'y a pas d'avocat de droit privé ou de droit public, un avocat est un avocat. \\
Avant 2012 existait les avoués.


Devant les TA, le ministère d'avocat est en principe facultatif. Exceptionnellement, il est obligatoire lorsque les conclusions tendent à demander des sommes pécuniaires ou lorsque c'est contractuel (art. R. 431-2 CJA). \\
Des exceptions aux exceptions sont prévues, il y en six de prévues: les litiges en matière de contravention de grande voirie, lorsque le défendeur est une collectivité, un établissement public ou lorsque c'est une question d'aide. \\
Devant les CAA, l'assistance d'un avocat est la règle. Jusqu'en 2016, il y avait deux exceptions seulement. Dorénavant, il n'y a qu'une exception, pour les contraventions de grande voirie. \\
Devant le CE, le principe est l'assistance d'un avocat au conseil pour les recours contre les décisions administratives. Les exceptions sont nombreuses et comportent notamment: les REP contre les actes administratifs, les recours en appréciation de légalité, les litiges en matière électorale. \\
En cassation, le principe de l'aide d'avocat demeure, et les exceptions sont plus réduites. En cas d'appel formé devant le CE, le ministère d'avocat est facultatif. 

\subsubsection{L'intérêt pour agir}

C'est une condition classique, elle vise à limiter la possibilité pour les individus à agir en justice. Il faut être intéressé par l'issue du recours. C'est l'objet de la demande qui permet d'apprécier l'intérêt pour agir de l'auteur du recours et non le raisonnement qui la soutient. \\
L'intérêt invoqué doit, en outre, être réel et légitime. On dit aussi qu'il doit être direct. \\
Le degré d'exigence d'intérêt pour agir diffère du type de contentieux. Certain, l'intérêt ne doit pas être hypothétique. Direct, cela signifie que l'intérêt doit directement nous impacter. Légitime, c'est à dire que l'intérêt doit respecter les lois et les règlements d'ordre public. \\
Par exemple, un contribuable d'une commune a intérêt à agir contre toutes mesures ayant une incidence sur les finances de la commune (CE, 29 Mars 1901, Casanova). Cela n'est pas valable pour le contribuable de l'État, l'intérêt n'étant pas suffisant direct. Autre exemple, une association a intérêt à agir contre un acte réglementaire ou une décision individuelle positive qui porte atteinte aux intérêts dont elle a la charge (CE, 28 déc. 1906, Syndicats des patrons coiffeurs de Limoges). \\
En principe, les agents publics n'ont pas intérêt à contester les mesures relatives à l'organisation du SP au sein duquel ils sont affectés, sauf si ces mesures portent atteinte à leurs droits ou dans la mesure où la disposition attaquée affecterait leurs conditions d'emploi et de travail.  

\subsubsection{La décision administrative préalable}

C'est une règle qui remonte au ministre-juge, qui consiste à dire qu'on peut saisir le JA que d'une décision. C'est pourquoi on parle de recours alors qu'on parle d'action ou d'assignation en droit privé. \\
Jusqu'au décret 2016-1480 du 2 novembre 2016, l'article R421-1 du CJA disposait que "sauf en matière de travaux public, la juridiction ne peut être saisie que par voie de recours formé contre une décision". Désormais, l'exception en matière de travaux public a disparu. \\
Le JA ne peut donc être saisi d'un recours que contre une décision administrative. On dit que cette décision a pour effet de "lier le contentieux". De nos jours, on estime que l'obligation de lier le contentieux permet de formaliser l'existence d'un litige. \\
La décision préalable peut être explicite (écrite ou verbale) ou bien implicite. La décision implicite résulte du silence gardé par l'administration et de l'expiration d'un délai. \\
Jusqu'à récemment, un principe était posé comme quoi le silence valait rejet. Aujourd'hui, il est inversé, avec de nombreuses exceptions.


En matière de REP et pour certains RPC, la décision préalable est la décision qui fait l'objet du recours. \\
La décision doit faire grief à l'auteur du recours, sans quoi le recours est irrecevable. \\
Faire grief, cela signifie concrètement que l'on ne peut pas former de recours contre les mesures préparatoires. Cela parce que ces mesures visent, à terme, à une décision qui elle fera l'objet du REP. Ne font pas grief non plus les avis et les recommandations. \\
Toutefois, les avis et recommandations des AAI, qui ont une incidence sur les opérateurs peuvent faire l'objet d'un recours car peuvent faire grief: CE, Ass., 21 Mars 2016, Société NC Numericable.\\
Les actes par lesquels les chefs de service indiquent à leurs subordonnés la façon d'interpréter ou d'appliquer un texte juridique sont parfois susceptible de REP. \\
Les MOI ne peuvent pas faire l'objet d'un REP. \\
En matière de contentieux indemnitaire, l'exigence de décision préalable conduit à soumettre une réclamation préalable à l'administration avant de former le recours. \\
On s'est rendu compte qu'il pouvait être intéressant pour les administrés de demander à l'administration de revenir sur sa décision, donc on a créé des recours administratifs préalable obligatoire (RAPO), c'est si l'administration refuse que l'on peut aller devant le juge. 

\subsection{Les conditions de recevabilité liées à la présentation du recours}

\subsubsection{La rédaction de la requête}

Ce document doit être rédigé d'une certaine manière et contenir des informations précises. La requête est un acte motivé adressé par écrit à une juridiction pour introduire la procédure devant elle. Elle doit être écrit en français. Elle doit indiquer les noms et domiciles des partis, contient l'exposé des faits et moyens, ainsi que l'énoncé des conclusions soumises au juge. Depuis 2016, ceux qui contestent un contrat ou son exécution doivent reproduire le contrat. \\
Les conclusions sont les prétentions soumises au juge. Elles doivent identifier la décision administrative contestée et préciser ce qui est demandé au juge. \\
Les moyens constituent le fondement juridique au soutien des conclusions. Ce sont donc les motifs juridiques qui, selon le requérant, justifient le bien fondé de sa demande.

\subsubsection{Le délai de recours}

C'est une condition spécifique au contentieux administratif, car il y a la règle de la décision administrative préalable. Le délai et cette décision préalable sont liés. \\
En principe, le recours n'est recevable que dans un délai de deux mois suivant la publicité exigée pour les décisions. Contre une décision implicite, il y a un délai de deux mois après l'avis de réception, et dans les deux prochains mois, le recours est possible. \\
On compte deux publicités possible, la notification et la publication. La notification est quand on prévient une personne intéressée par l'acte de la décision administrative. La publication est la formalité par laquelle un acte réglementaire est porté à la connaissance du public par voie de diffusion dans un recueil officiel ou dans la presse, soit par voie d'affichage. \\
Concernant les décisions individuelle, une obligation existe et qui dit que ces décisions doivent comporter l'information pour faire le recours et dans quel délai. Si ces informations ne sont pas présentes, le CE dit qu'il y a tout de même une limite, CE, Ass., 13 Juillet 2016, Czabaj, cette limite est d'un an, à compter de la notification de la décision.


Le délai de recours est en principe de deux mois, mais il y a des délais spéciaux. \\
Il existe des délais abrégés comme le recours contre les élections départementales qui doit être déposé devant le greffe du TA compétent au plus tard à 18h le cinquième jour qui suit l'élection. On a 10 jours pour les municipales ou les élections européennes. \\
Il existe des délais allongés comme dans certains DOM-COM, où le délai passe de deux à trois mois, en raison des moyens de communication moins bien développé ou à cause de leur éclatement (ce sont des îles et îlots). Depuis qu'il y a le numérique, on se demande si cette justification est toujours vraiment pertinente même si l'internet n'est probablement pas sur tous les îlots. \\
Il existe des délais en cas d'éloignement de l'auteur du recours. Par exemple, si le recours se fait devant le CE qui statue en premier et dernier ressort, le délai est augmenté d'un mois pour les requérants venant des DOM-TOM. 


Enfin, le délai est franc. Cela veut dire qu'on ne prend pas en compte le jour où le délai commence à courir ni le jour où le délai finit de courir. \\
Toute décision administrative peut faire l'objet, dans le délai imparti pour l'introduction d'un recours contentieux, d'un recours gracieux ou hiérarchique, qui interrompt le cours dudit délai.

\section{Le régime des conditions de recevabilité}

Toutes les conditions de recevabilité n'entraînent pas automatiquement l'irrecevabilité du recours. Les causes d'irrégularité peuvent être corrigées à la demande du juge dans un certain délai. Toutes les causes ne sont pas régularisables. \\

\subsection{Les causes d'irrecevabilité non régularisables}

Ce sont des causes d'irrecevabilité dites absolues. Il s'agit par exemple du délai de recours, du défaut d'intérêt à agir, contre une décision insusceptible de faire grief, etc. 

\subsection{Les causes d'irrecevabilité régularisables}

À l'invitation du juge, le requérant peut régulariser ses causes d'irrecevabilité. Si cela est corrigé avant ou après le délai de recours, il y a des cas différends. \\
On peut régulariser avant l'expiration du délai de recours si la requête ne comporte aucun moyen ou aucunes conclusions. On ne peut pas régulariser après le délai d'expiration. \\
Certaines peuvent se régulariser après, comme un manque de signature, si une personne n'a pas qualité à agir, ou bien si il manque un avocat. 


Lorsque des conclusions sont entachées d'une irrecevabilité, le juge a l'obligation de demander la régularisation. Il n'a pas le droit de rejeter d'office. \\
Si l'irrecevabilité n'a pas été couverte par le requérant, le juge va rejeter la requête par ordonnance. 

\part{Le procès administratif}

Deux phases: l'instruction et le jugement. 

\chapter{L'instruction}

\section{Procédure d'instruction}

C'est le moment où le juge étudie le dossier. L'instruction possède trois caractéristiques générales: \\
\begin{itemize}
\item Procédure inquisitoire car elle est dirigée par le juge au lieu de l'être par les mandataires des parties. \\
Cela car les parties sont inégalitaire, l'administré d'un côté, l'administration, de l'autre, le juge peut d'ailleurs ordonner à l'administration de produire des pièces
\item Procédure écrite, les moyens et les conclusions des parties doivent faire l'objet de mémoires rédigés.
\item Procédure contradictoire (L. 5 CJA ; CE, Sect., 12 Mai 1961, Société La Huta), c'est à dire que le juge doit veiller à ce que chaque partie puisse discuter les éléments avancés par la partie adverse ou par le juge. 
\end{itemize}


La conduite de l'instruction est supervisée par un juge particulier au sein de la formation de jugement, celui en charge des fonctions de rapporteur. Il ne doit pas être confondu avec le rapporteur public, qui ne fait pas partie de la formation de jugement. 

\subsection{Le déroulement de l'instruction}

\subsubsection{Le déclenchement de l'instruction}

Une juridiction administrative est divisée en chambre, qui représente chacune une formation de jugement. À Amiens, on compte quatre chambres. Chaque chambre a ses spécialités. \\
Chaque chambre a son président, son ou ses rapporteurs, ainsi qu'un rapporteur public. Traditionnellement, une chambre a quatre juges. \\
Chaque chambre possède son greffier ainsi qu'un ou des assistants de greffes. 


En pratique, la procédure débute avec l'enregistrement de la requête au greffe de la juridiction compétente. Chaque juridiction a en effet son greffe général qui réparti selon les chambres les affaires reçus. \\
Le greffier est un auxiliaire de justice chargé tout au long du procès de garantir le respect et l'authenticité de la procédure. \\
Immédiatement après le déclenchement de la requête, le président de la juridiction désigne un rapporteur chargé de son instruction, sous l'autorité du président de sa formation de jugement. La fonction de rapporteur est souvent donné aux juges les moins expérimentés. \\
Toutefois, si la solution de l'affaire est d'ores et déjà certaines, une dispense d'instruction peut être décidé pour accélérer le traitement de l'affaire (art. R. 611-8 CJA). En pratique, cela signifie que l'on passe directement à la phase de jugement. Cette hypothèse est souvent mise en oeuvre lorsque les requêtes sont certaines d'être rejetés, notamment en cas d'irrecevabilité. 


À la suite de sa désignation, le rapporteur va procéder à l'établissement d'un plan d'instruction. Cela consiste à organiser un calendrier de communication des requêtes et des mémoires entre les parties en demande et en défense. \\
L'exécution matérielle des mesures décidées par lui est assurée par le greffe. 

\subsubsection{La conduite de l'instruction}

L'essentiel de  l'instruction est constitué par des échanges de mémoires se répondant dans les délais fixés par le rapporteur et au respect desquels veille le greffier. \\
Le rapporteur peut demander toutes pièces aux parties ou tous documents utiles à la solution du litige. Il peut ordonner d'office, ou bien à la demande d'une partie, des mesures d'investigation (expertise, visite des lieux, enquête, etc). \\
Ces décisions prises par le juge doivent toujours être soumises au respect du principe de contradictoire. C'est à dire que quand il ordonne la communication d'une pièce, il doit envoyer cette pièce à la partie adverse. \\
Si une partie ne respecte pas les délais impartis ou ne répond pas aux sollicitations du rapporteur, le juge peut adresser à cette partie défaillante une mise en demeure avec un délai, à l'expiration duquel elle sera réputée avoir acquiescé au fait exposé par son adversaire.


Pour que le respect du contradictoire soit assuré, les parties doivent toujours avoir communication des mémoires et pièces du dossier et disposer d'un délai suffisant pour présenter des observations en réponse. \\
Il existe des dérogations à l'échange contradictoire. Par exemple, il n'est pas nécessaire pour le greffe de communiquer un mémoire ne contenant aucun moyen ou aucun élément nouveau. \\
Le juge est, en principe, obligé de soumettre à une discussion contradictoire les éléments qu'il va apporter au débat. Ainsi, il doit communiquer aux parties le moyen d'ordre public qu'il envisage de soulever d'office. 

\subsubsection{Les incidents de l'instruction}

Parfois, un événement survient  au cours de l'instruction, perturbant le cours de celui-ci. Ces incidents conduisent le rapporteur à en tirer les conséquences. En voici quelques un, qui sont fréquent devant les juridictions administratives. 


Par la demande incidente, le défendeur ne se contente pas de demander le rejet du recours, mais il va présenter des conclusions supplémentaires qui vont élargir l'objet du litige (on appelle cela des conclusions reconventionnelles). Dans ce cas, le rapporteur va devoir voir si elle est recevable, et si oui, l'intégrer à l'instruction. 


Par l'intervention, un tiers demande à participer au procès. \\
La demande d'intervention est soumise à des conditions de recevabilité propres ; elle doit s'associer aux conclusions de l'une ou l'autre des parties. \\
Après avoir statué sur la demande en intervention, le cours de l'instruction reprend. Si la demande est acceptée, le tiers est considéré comme partie au procès et se verra communiquer les pièces du dossier. 


Par le désistement volontaire, l'auteur du recours manifeste sa volonté d'abandonner sa conclusion. \\
Le juge va statuer sur cette demande, et, si il l'accepte, il va en donner acte. Dans ce cas, la procédure s'arrête, mettant fin au litige. \\
Le juge doit vérifier notamment que la demande est réel et que le demandeur n'a pas subi de pressions.


Lorsque l'objet du litige disparaît, comme dans le cas où une décision faisant l'objet d'un REP est retiré, la procédure n'a alors plus lieu d'être. La juridiction va alors constater le non-lieu à statuer. 

\subsection{La clôture de l'instruction}

La clôture peut résulter d'une décision du président de la formation de jugement. \\
Devant les TA et les CAA, le président de la formation de jugement peut fixer par ordonnance, préventivement, la date à partir de laquelle l'instruction sera close. Cela oblige les parties mais aussi le rapporteur à étudier les éléments du dossier jusqu'à la date butoir. \\
Si une telle volonté n'est pas exprimée, la clôture peut être automatique: devant les TA et les CAA, l'instruction est close 3 jours francs avant la date de l'audience. Devant le CE, l'instruction est close soit après que les avocats ont formulés leurs observations orales, soit, en l'absence d'avocat, après appel de l'affaire à l'audience. \\
Les mémoires produits après clôture de l'instruction ne donnent pas lieu, en principe, à communication. Ils doivent néanmoins être examinés par la juridiction, et si il l'estime nécessaire, le président de la formation de jugement peut rouvrir l'instruction (art. R. 613-3 CJA). \\
Depuis le décret du 2 Novembre 2016, le président de la formation de jugement a la faculté, après clôture de l'instruction, de prendre une nouvelle mesure d'instruction sans rouvrir l'instruction: il peut ainsi inviter une partie à produire des éléments ou des pièces en vue de compléter l'instruction, mais sans avoir à la ré-ouvrir (art. 613-1-1 CJA). 

\section{La matière litigieuse}

\subsection{La détermination de la matière litigieuse}

La matière litigieuse, c'est l'ensemble des conclusions et des moyens soumis au juge et discutés par les parties. 

\subsubsection{Les conclusions}

Les conclusions des parties déterminent le contenu et dessinent les contours de l'objet du litige. \\
En application du principe dispositif, l'objet du litige est déterminé exclusivement par les conclusions des parties (celle de l'auteur du recours et, si cela est possible, celles ajoutées par le défendeur). Cela oblige et limite le juge, qui doit répondre aux conclusions et ne peut pas les modifier. Le juge ne peut pas statuer "ultra petita": il ne peut pas étendre l'objet du litige. Les intervenants ne peuvent pas non plus étendre l'objet du litige, ils ne peuvent que s'associer aux conclusions des parties.

\subsubsection{Les moyens}

Les moyens, ce sont les arguments qui soutiennent les conclusions. On dit que ce sont le fondement juridique au soutien des conclusions. \\
Ce sont donc les motifs de droit ou de fait qui, selon une partie, justifient le bien-fondé de sa demande. 


Il appartient à l'auteur du recours d'alléguer au moins un moyen au soutien de sa prétention. Le défendeur n'est pas obligé de répondre, mais il a évidemment intérêt à développer les raisons, qui, selon lui, doivent conduire au rejet de la demande. \\
Le juge peut interpréter les écritures des parties mais il doit, en principe de s'en tenir à ces écrits. Il n'a pas la faculté de relever d'office des moyens au soutien de la demande ou de la défense. Il existe néanmoins des exceptions.


Ces exceptions sont au nombre de deux. \\
La première est celle des moyens d'ordre public, où le juge a l'obligation de le soulever d'office si il n'a pas été invoqué par les parties et qu'il ressort des pièces du dossier. L'incompétence de l'auteur d'un acte attaqué est par exemple un moyen d'ordre public. C'est la jurisprudence qui dresse la liste des moyens qui sont des moyens d'ordre public, on en compte une vingtaine, l'incompétence de l'auteur étant le plus célèbre. Les moyens d'ordre public intéressent soit la procédure contentieuse (compétence du juge, délai de recours, etc), soit le fond du litige (compétence, nullité, etc). Si le juge oublie de soulever le moyen d'ordre public, sa décision peut être cassé. \\
La deuxième est que le juge s'autorise à soulever d'office certains moyens qui, bien que dépourvus de caractère d'ordre public, font échapper la décision contestée à l'annulation. Par exemple, le juge peut soulever d'office le moyen tiré d'une substitution de base légale (le juge corrige la base légale sur laquelle a été prise la décision). 

\subsection{L'évolution de la matière litigieuse}

L'évolution de la matière litigieuse, autrement dit, la possibilité pour les parties de présenter des conclusions et des moyens nouveaux est limitée par l'expiration du délai de recours. \\
L'expiration du délai de recours n'a pas qu'une incidence sur la recevabilité de la requête. L'expiration du délai de recours entraîne la "cristallisation" de la matière litigieuse. \\
D'une part, l'expiration du délai de recours a pour effet de figer les conclusions présentées par l'auteur du recours avant ce délai. Le requérant ne peut donc plus étendre l'objet de son recours en présentant de nouvelles conclusions. \\
D'autre part, l'expiration du délai de recours limite la faculté de l'auteur du recours d'invoquer des moyens nouveaux. À l'expiration du délai de recours, le requérant n'est autorisé qu'à soulever des moyens que si ils se rattachent à la même cause juridique que celles dont relèvent les moyens invoqués avant l'expiration du délai. \\
La cause juridique de la demande est ici entendue comme un regroupement de plusieurs moyens en fonction de leur caractéristique commune. Par exemple, dans un REP, on distingue deux causes juridiques: les moyens de légalité externe (incompétence, vices de formes et de procédure) et les moyens de légalité interne (erreurs sur les motifs, détournement de pouvoir, violation directe de la règle de droit). En matière indemnitaire, il existe plus d'une dizaine de causes juridiques. \\
À n'importe quel moment, un moyen d'ordre public peut être invoqué par les parties. \\
Depuis le décret du 2 Novembre 2016, le président de la formation de jugement, ou au CE, le président de la chambre chargée de l'instruction peut, lorsque l'affaire est en état d'être jugée, et sans clore l'instruction, fixer par ordonnance la date à compter de laquelle les parties ne peuvent plus invoquer de moyens nouveaux (art. R. 611-7-1 CJA). 

\chapter{Le jugement}

\section{La procédure de jugement}

Il incombe au juge chargé des fonctions de rapporteur d'étudier l'ensemble des éléments du litige afin de se faire une opinion sur le sort de la requête. Le rapporteur doit préparer un projet de jugement à l'issue de l'instruction. Ce projet est protégé par le secret du délibéré. \\
Le projet de jugement est ensuite transmis à un juge chargé de la fonction de "réviseur", il va porter un second regard sur le dossier et, le cas échéant, apporter des modifications. Le juge rapporteur étant un juge souvent inexpérimenté, c'est un juge plus expérimenté qui va opérer cette révision, qui n'est d'ailleurs pas prévu par les textes, c'est une pratique. \\
Dans les TA et dans les CAA, le travail d'instruction est réalisé par le seul rapporteur, sauf si le président de la formation de jugement juge utile une séance collective d'instruction entre les membres de la formation de jugement. \\
Au CE, l'instruction est confiée à une chambre (avant sous section), au sein de laquelle est désigné un rapporteur. Les membres de la chambre entendent le rapport du rapporteur et les observations du réviseur au cours d'une séance d'instruction. \\
À l'issue du travail de révision, le dossier et le projet de jugement sont transmis au rapporteur public: "Un membre de la juridiction, chargé des fonctions de rapporteur public, expose publiquement, et en toute indépendance, son opinion sur les questions que présentent à juger les requêtes et sur les solutions qu'elles appellent" (art. L. 7 du CJA). 


Une fois que l'affaire est en état d'être jugée, elle est inscrite au "rôle" (liste de toutes les affaires) d'une formation de jugement et la date d'une audience publique est fixée. Les parties ou leurs mandataires sont informés de la tenue de cette audience publique qui précédera le délibéré secret des juges. 

\subsection{L'audience publique}

C'est une étape qui existe devant toutes les juridictions. C'est une étape intermédiaire entre l'instruction et le jugement. \\
Tous les membres de la formation de jugement assistent à l'audience, celui qui n'y assiste pas ne peut pas délibérer. Le projet de jugement préparé par le rapporteur leur est communiqué avant l'audience. \\
L'audience publique est toujours ouverte aux parties et leurs mandataires et, plus généralement, à toutes personnes souhaitant y assister sauf si le président de la formation de jugement décide le huis-clos. \\
L'audience publique débute par l'appel de l'affaire (c'est le greffier qui appelle l'affaire), ce qui permet aux avocats de se manifester. Ensuite, le rapporteur procède à la lecture des visas, tant la loi applicable que les mémoires déposés. Ensuite, les parties présentent leurs observations et le rapporteur public prononce ses conclusions. Les avocats ne font pas de plaidoiries devant le juge administratif. Le rapporteur n'a pas forcément à prendre la parole, il peut présenter ses conclusions à l'écrit.


Pendant longtemps, l'ordre de parole était: rapporteur ; parties ; commissaire du gouvernement. Pour la CEDH, le commissaire du gouvernement était une partie, à qui il faut donc pouvoir répondre pour respecter le principe du contradictoire, ce qui n'était pas le cas vu qu'il parlait en dernier. \\
Sous l'influence de la CEDH (CEDH, 7 Juin 2001, Kress c. France ; CEDH, 12 Avril 2006, Martinie c. France), l'organisation de l'audience publique a évolué, en particulier la place du rapporteur public. Deux problèmes: on ne peut pas parler après le rapporteur public, et, deuxième problème, il participe au délibéré. \\
La CEDH n'a pas condamné la France car certes, on ne pouvait pas prendre la parole après le commissaire du Gouvernement, mais on pouvait cependant adresser une note en délibéré, ce qui permettait de répondre par écrit au commissaire du gouvernement après l'audience (CE, 12 Juillet 2002, Leniau ; décret du 19 Décembre 2005). \\
On a tout de même décidé de moderniser le procès administratif. Le décret du 7 janvier 2009 a consacré la possibilité offerte aux parties de prendre connaissance avant l'audience du sens des conclusions et prévu le droit de répliquer oralement aux conclusions pendant l'audience après la prise de parole du rapporteur public. Donner le sens des conclusions signifie que "les parties doivent être mise en mesure de connaître, dans un délai raisonnable avant l'audience, l'ensemble des éléments du dispositif de la décision que le rapporteur public compte proposer à la formation de jugement d'adopter, à l'exception de la réponse aux conclusions qui revêtent un caractère accessoire". \\
Devant les TA et les CAA, le décret du 23 Décembre 2011 a modifié l'ordre de passage: les parties ou leurs mandataires présentent leurs observations orales après le prononcé des conclusions du rapporteur public. Devant le CE, c'est l'ordre prévu par le décret 7 Janvier 2009 qui demeure, avec la possibilité d'une double prise de parole. 


Les conclusions du rapporteur public revêtent une double dimension: le rapporteur public dresse un ultime état des lieux du litige, dans son étendue ; il propose également la solution qu'il estime devoir être retenue, après avoir développé les raisons qui la fondent. 


\subsection{Le délibéré}

\subsubsection{Les formations de jugement}

Le délibéré se tient entre les membres de la formation de jugement, c'est à dire des juges qui vont participer à la prise de la décision de justice, à condition qu'ils aient assisté à l'audience. \\
"Les jugements sont rendus en formation collégiale, sauf s'il en est autrement disposé par la Loi", art. L. 3 du CJA. Cela s'oppose au juge unique, que l'on voit en général dans les pays anglo-saxons. \\
Par principe, la formation collégiale est impair pour éviter le partage de voix. Le seul tribunal qui est pair est le TC. \\
On considère qu'avoir plusieurs juges permet d'avoir un risque moindre de pression, d'avoir une justice plus éclairé, plus impartial. \\
Cela implique bien sûr que plus de juges sont mobilisés par affaire, augmentant donc le temps de traitement des affaires. 


Les exceptions au principe de collégialité ont fortement augmenté depuis les années 1990. \\
La loi du 10 Janvier 1990 signe le début du juge unique, elle organise le contentieux des arrêtés préfectoraux de reconduite à la frontière des étrangers en situation irrégulière. C'est un juge unique qui apprécie la légalité d'un arrêté préfectoral. Cette loi prévoit aussi la possibilité d'un juge unique en appel. L'objectif est d'accélérer le traitement d'un contentieux de masse. \\
La loi du 4 Janvier 1992 instaure un juge unique au référé pré-contractuel. La loi du 30 Juin 2000 prévoit aussi des juges uniques dans des référés d'urgence. 


En matière de référé, le principe est le juge unique. Très récemment, depuis la loi du 20 Avril 2016, on a modifié l'article L. 511-2 du CJA qui dispose désormais que lorsque la nature de l'affaire se justifie, le président de la juridiction peut décider qu'elle sera directement jugée par une formation collégiale de référé, composée de trois juges des référés. 


Dans les TA et les CAA, les jugements sont rendus soit par une chambre, soit par les chambres réunies, soit par l'assemblée plénière de la juridiction. La difficulté et la nature de l'affaire font que tel ou tel formation sera utilisée. C'est le président de la juridiction qui décide cela.


Au CE, il existe quatre formations de jugement possible, en fonction là encore de la difficulté et de l'importance de l'affaire. \\
Les deux formations les plus importantes sont la Section du contentieux (15 juges), et l'Assemblée du contentieux (17 juges). L'assemblée du contentieux a la particularité de réunir des membres de tout le CE. \\
Les deux formations les moins importantes sont les chambres (ex sous-sections) réunis (9 juges) et la chambre (ex sous-section) jugeant seule (3 juges). Ce sont en général des arrêts d'espèce.


La formation de la "chambre jugeant seule" correspond à l'une des dix chambres de la Section du contentieux. La formation de jugement est composée du président de la chambre chargée de l'affaire, de l'un de ses deux assesseurs et du rapporteur. \\
La formation de "chambres réunies" correspond à deux chambres qui vont statuer ensemble. Une des chambres a instruit l'affaire et la juge avec l'autre chambre. La formation de jugement (9 juges) est composée du président de la Section du contentieux ou de l'un des trois présidents adjoints ; de la chambre d'instruction viennent le président, deux assesseurs et le rapporteur ; pour l'autre chambre, le président et deux assesseurs ; le dernier juge est un conseiller d'État ne siégeant dans aucune des deux chambres qui jugent l'affaire. 


Concernant la formation du jugement de la Section du contentieux, on a 15 juges. \\
Elle est composée par le président de la Section (Bernard Stirn aujourd'hui), ses trois présidents adjoints, les dix présidents des autres chambres. On ajoute à cela le rapporteur de l'affaire. 


Concernant l'Assemblée du contentieux, on a 17 juges. \\
Il y a le vice-président du CE (Jean-Marc Sauvé aujourd'hui) ; les sept présidents de section (contentieux, intérieur, travaux publics, administration, finances, sociale, rapport et études), les trois présidents adjoints de la Section du contentieux, le président de la chambre sur le rapport de laquelle l'affaire est jugée, les quatre présidents de chambre les plus anciens dans leurs fonctions en dehors du précédent, et enfin, le rapporteur. 

\subsubsection{Les caractéristiques du délibéré}

"Le délibéré des juges est secret", Art. L. 8 du CJA. Cela concerne surtout les formations collégiales. Cela signifie que les parties ou le public plus généralement, n'ont pas le droit d'y assister. Autre conséquence, les opinions et votes exprimés au cours du délibéré ne peuvent pas être diffusés à posteriori. Ce principe est absolu. \\
Ce secret vise à préserver l'autorité des décisions et à éviter de voir des discordances des membres d'une juridiction. \\
Jusqu'en 2006, le rapporteur public pouvait assister au délibéré. Il était présent mais ne prenait pas la parole et ne participait pas à la prise de décision. Sous l'influence de la CEDH, la participation du rapporteur public a été remise en cause (CEDH, 7 Juin 2001, Kress c. France ; CEDH, 12 avril 2006, Martinie c. France). \\
Le décret du 1er Août 2006 a modifié la pratique. Devant les TA ou les CAA, la présence du rapporteur public est exclue. Devant le CE, la présence du rapporteur public est autorisée sauf si une partie s'y oppose avant le délibéré. CEDH, 15 Septembre 2009, Yvonne Etienne c. France: les modifications sont conforme à l'article 6§1 de la Convention. À l'heure actuelle, il est très rare qu'un rapporteur public assiste au délibéré. 


Le délibéré débute par la lecture du projet de jugement par le rapporteur et continue par une discussion entre les membres de la formation de jugement. \\
S'il apparaît au cours du délibéré qu'une nouvelle mesure d'instruction est nécessaire ou qu'un moyen d'ordre public doit être versé au débat contradictoire ou encore qu'une note en délibéré est produite contenant des éléments que le juge doit prendre en compte, l'affaire est radiée du rôle et l'instruction est rouverte. \\
Dans les autres cas, les débats entre les juges s'achèvent par l'adoption d'un jugement auquel il est procédé par vote à main levé si des désaccords existent au sein de la formation de jugement. Le vote des juges, qui est obligatoire, concerne le sens général du jugement et les motifs qui le fondent. \\
À l'issue du délibéré, s'ouvre la phase ultime de la rédaction du jugement qui est confiée au rapporteur. Le jugement est ensuite transmis au président de la formation de jugement qui va en contrôler le fond et la forme. \\
Le jugement est rendu le jour où il est "lu" publiquement, avant d'être notifié aux parties. Les décisions ne sont jamais lues intégralement. Le président d'une formation de jugement prononce, après avoir ouvert la séance d'audience publique, les mots "les décisions sont lues", qui sont réputés valoir pour toutes les décisions "lues" et donc rendues ce même jour. 

\section{La chose jugée}

La chose jugée est contenue dans un jugement. Les jugements, au sens large du terme, désignent toutes les décisions rendues par les juridictions administratives. \\
Typologie des jugements: jugement (stricto sensu) d'un TA ; arrêt d'une CAA ; décision du CE (certains parlent d'arrêts) ; ordonnance des juridictions administratives (rendues en matière de référé). \\
Les jugements doivent comporter certaines mentions obligatoires pour rappeler au juge que les formes sont des garanties que les justiciables sont en droit d'exiger et permettre au juge d'appel ou de cassation de contrôler leur régularité. \\
Le jugement contient: le nom des juges qui ont siégé à l'audience et au délibéré ; la formation de jugement (qui a rendu le jugement) ; le nom des parties, l'analyse des conclusions et des mémoires ; le visa des textes dont il est fait application ; la date de la décision. D'autres mentions sont obligatoires, celles-ci sont les plus importantes. Le jugement mentionne, le cas échéant, les mesures d'instruction ordonnées, la tenue de l'audience publique, et les prises de paroles successives, et la production de notes en délibéré.

\subsection{Le contenu de la chose jugée}

La chose jugée, ce sont les motifs et le dispositif du jugement. Ce sont véritablement le contenu de la chose jugée. 

\subsubsection{La motivation}

"Les jugements sont motivés" art. L. 9 du CJA. Tous les jugements doivent être motivés, c'est à dire que tous les jugements doivent comporter des motifs justifiant la solution. Cela a pour but d'éviter l'arbitraire du juge, et pour permettre un contrôle des juridictions supérieures. \\
Les motifs étaient traditionnellement contenu dans ce qu'on a appelés "considérant". Depuis 2015, le CE utilise moins cette formule. C'est donc entre le visa et le dispositif que l'on trouve la motivation. 


L'exigence de motivation se matérialise par l'obligation du juge de répondre aux moyens de droit ou de faits invoqués par les parties au soutien de leur demande. Si un juge rend une décision non motivée, elle peut être censurée. \\
Plus précisément, l'obligation de motivation signifie que le juge doit, en principe, répondre à tous les moyens invoqués par la partie qui perd son procès. En revanche, le juge n'est jamais obligé de répondre à tous les moyens invoqués par la partie qui obtient gain de cause. On dit que le JA pratique l'"économie de moyens". Cette règle de l'économie de moyens donne l'opportunité au juge de se limiter à statuer au regard d'un seul moyen fondé lorsqu'il accueille la demande du requérant. Le moyen utilisé peut être un de ceux invoqué par la partie gagnant ou un moyen d'ordre public. \\
On note cette économie de moyens dans la traditionnelle phrase "sans qu'il soit besoin de statuer sur les (autres) moyens du recours". \\
La réponse du juge peut porter soit sur l'irrecevabilité du moyen, soit sur son inopérance, soit sur l'absence de bien fondé du moyen. 


L'obligation de répondre à tous les moyens invoqués par la partie perdante trouve une limite tenant à l'ordre d'examen des questions devant le juge. \\
Le JA doit examiné dans l'ordre: le désistement d'une partie ; sa compétence pour statuer ; la survenance d'un non lieu ; la recevabilité du recours ; le fond du litige (c'est une fois arrivé ici que le juge doit répondre aux moyens). 

\subsubsection{Le dispositif}

Le dispositif est ce que décide le juge. Le contenu du jugement va être la réponse donnée à la demande des parties. Cette réponse porte sur la question principale, c'est à dire l'objet même du litige. \\
La réponse apportée à la question principale ne va pas nécessairement trancher le litige qui lui est soumis. La décision de justice peut: constater le désistement, constater un non lieu, constater l'incompétence de la juridiction, déclarer le recours irrecevable. Dans ces cas là, le litige n'est pas tranchée. La décision peut donc aussi: rejeter au fond la demande mal fondée ; ou bien accueillir la demande qui est bien fondée.


À l'occasion du règlement de la question principale, le juge est fréquemment conduit à prendre diverses mesures qui portent sur des questions accessoires. \\
Par exemple, le juge peut, sur demande ou d'office, supprimer des passages injurieux, outrageants ou diffamatoires qui figurent dans les mémoires et autres pièces produites par les parties et condamner le responsable à des dommages-intérêts (art. L741-2 du CJA). \\
Par exemple, le juge peut condamner une partie aux dépens, c'est à dire aux frais occasionnés par les mesures d'instruction éventuellement prescrites dont le coûts n'est pas supporté par l'État (art. R. 761-1 du CJA). Dès lors qu'ils existent, les dépens doivent être mis à la charge d'une ou plusieurs parties. Ils sont, en principe, supportés par la partie perdante, sauf si les circonstances particulières de l'affaire justifient qu'ils soient mis à la charge d'une autre partie ou partagés entre les parties. \\
Autre exemple, le juge peut, à la demande d'une partie, condamner l'autre partie aux frais non compris dans les dépens (appelés aussi "frais irrépétibles"), c'est à dire les honoraires d'avocats et les autres dépenses exposées par les parties en cours d'instance (frais d'huissiers, de déplacement, etc.). \\
En principe, le juge condamne la partie tenue aux dépens ou, à défaut, la partie perdante, à payer à l'autre partie la somme qu'il détermine, au titre des frais exposés et non compris dans les dépens. \\
Le juge "tient compte de l'équité ou de la situation économique de la partie condamné. Il peut, même d'office, pour des raisons tirées des mêmes considérations, dire qu'il n'y a pas lieu à cette condamnation" art. L. 761-1 du CJA. \\
Le juge peut infliger à l'auteur d'une requête qu'il estime abusive une amende dont le montant ne peut excéder 3000 euros (art. R. 741-12 du CJA). 

\subsection{L'autorité de la chose jugée}

Dès qu'il est lu, et avant même sa notification, le jugement est doté de l'autorité de la chose jugée: ce qui a été jugé ne peut être remis en cause ni par la juridiction qui a statué ni par une autre, sauf dans le cadre d'une voie de recours. \\
Lorsqu'un jugement n'est plus susceptible d'une voie de recours devant une juridiction supérieure, il devient définitif. \\
L'autorité de chose jugée ne doit pas être confondue avec la force obligatoire du jugement: le jugement s'impose à ses destinataires parce qu'il est doté d'une force obligatoire. S'ils forment un nouveau recours visant à rejuger l'affaire, ils méconnaissent l'autorité de chose jugée. \\
Dans la mesure où l'autorité de chose jugée rend intangible ce qui a été déjà jugé, elle ne concerne que les décisions de justice par lesquelles le juge a statué sur la demande. Les ordonnances provisoires du juge des référés ne sont pas revêtues de l'autorité de chose jugée. En revanche, elles sont dotées d'une force obligatoire. 

\subsubsection{Les fonctions de l'autorité de chose jugée}

Deux fonctions: une fonction "négative" et une "positive". \\
L'autorité de chose jugée se présente sous une forme positive lorsqu'elle s'oppose à ce qu'un juge remette en question les constatations effectuées par un autre tribunal à l'occasion d'un litige dont l'objet recoupe partiellement celui dévolu au second juge saisi. \\
L'autorité se présente sous une forme négative lorsqu'elle s'oppose à ce qu'un juge statue, en dehors d'une voie de recours, sur une demande qui a déjà fait l'objet d'une précédente décision de justice. \\
En vertu de l'autorité négative de chose jugée, ce qui a été jugée doit être considéré comme définitivement réglé. Cela produit donc deux conséquences: quand le jugement est rendu, le juge est dessaisi de l'affaire et ne peut pas revenir dessus ; ensuite, saisi d'un recours tendant au re-jugement du litige, le juge est tenu de le rejeter pour cause d'irrecevabilité dont le nouveau recours est entaché. 

\subsubsection{L'identification de l'autorité de chose jugée}

L'identification de l'autorité de chose jugée vise à répondre à deux questions: quels éléments du jugement sont revêtus de l'autorité de chose jugée ? À quelles conditions peut-on faire jouer l'autorité de chose jugée ? 


L'autorité de chose jugée s'attache au dispositif du jugement et aux motifs qui en constituent le support nécessaire. Lorsque le recours est rejeté pour irrecevabilité, l'autorité de chose jugée du jugement ne s'oppose pas à ce qu'on forme un nouveau recours en invoquant les mêmes moyens. En effet, seuls les motifs qui sont le support nécessaires au soutien du dispositif sont revêtus de l'autorité de chose jugée. 


En principe, les jugements sont dotés d'une autorité relative de chose jugée. Cela veut dire que pour certains jugements, on va appliquer les dispositions du code civil, à l'article 1351. C'est un principe classique selon lequel "L'autorité de chose jugée n'a lieu qu'à l'égard de ce qui a fait l'objet du jugement. Il faut que la chose demandée soit la même ; que la demande soit fondée sur la même cause etc." \\
On note donc trois conditions cumulatives: l'identité d'objet ; l'identité de cause juridique ; l'identité de parties. Par conséquent, avant d'opposer à une requête l'irrecevabilité tirée de de l'autorité de chose jugée, le juge doit vérifier s'il existe une triple identité d'objet, de cause et de parties, entre la requête initiale et la nouvelle requête. \\
L'identité d'objet renvoie aux conclusions de l'auteur du recours. Si elles diffèrent d'un recours à l'autre, l'autorité de chose jugée ne joue pas. \\
L'identité de cause juridique renvoie aux moyens invoqués au soutien du recours. Si les moyens invoqués au soutien du nouveau recours relèvent d'une cause juridique différente de celle dont relèvent les moyens invoqués au soutien du présent recours, l'autorité de chose jugée ne joue pas. \\
L'identité des parties renvoie à l'auteur du recours. Si l'auteur du recours est différent ou si le défendeur est différent, l'autorité de chose jugée ne joue pas. 


Exceptionnellement, un jugement peut être doté d'une autorité absolue de chose jugée, ce qui signifie qu'on n'applique pas les trois conditions cumulatives. \\
L'autorité de chose jugée signifie que ce qui a été jugé vaut à l'égard de tous (effet erga omnes). On s'intéresse seulement à l'objet du jugement, la cause et les parties sont indifférents. \\
Le jugement qui annule un acte administratif est revêtu de l'autorité absolue. L'acte annulé ne pourra plus être appliqué et aucun recours n'est possible à son encontre. Les jugements rendus en matière électorale sont revêtus de l'autorité absolue de chose jugée. 











\end{document}
