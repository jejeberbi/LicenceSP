\documentclass[10pt, a4paper, openany]{book}

\usepackage[utf8x]{inputenc}
\usepackage[T1]{fontenc}
\usepackage[francais]{babel}
\usepackage{bookman}
\usepackage{fullpage}
\setlength{\parskip}{5px}
\date{\today}
\title{Cours d'Organisations Européennes (UFR Amiens)}
\pagestyle{plain}

\begin{document}
\maketitle
\tableofcontents


\chapter{La notion d'organisation européenne}

\section{La notion d'organisation}

Il faut distinguer deux sortes d'organisations internationales. \\
La première catégorie ne nous intéressera pas dans ce cours, il s'agit des OI privées, donc des ONG. Ce sont, d'un point de vue quantitatif, les OI les plus nombreuses. Il est difficile d'en établir le nombre, mais avec le statut ECOSOC (assemblée sociale de l'ONU), on peut compter plus de 4600 ONG (auxquelles il faut rajouter les ONG n'ayant pas ce statut). Les ONG sont créées par une initiative privée ou mixte, qui ne sont jamais créées par des accords inter-gouvernementaux et qui regroupe des personnes privées et/ou public physique ou morale, de différentes nationalités. \\
La deuxième catégorie est le sujet de ce cours, ce sont des organisations interétatiques. Ces organisations sont forcément constituées par des États, ainsi que par des traités, qui constituent son acte constitutif. On compte environ 400 OI de ce type. Elles se sont développées au XXe siècle. 

\subsection{L'objet de l'OI}

De ce point de vue, on peut distinguer les OI à objet général, et les OI à objet spécial. Les premières ont pour ambition de régler les relations entre les États qui sont membre de l'OI dans des domaines variés, on peut citer l'ONU qui est l'OI à vocation généraliste (voir même universelle), par excellence. L'UE, progressivement, a acquis une certaine vocation généraliste. \\
Les deuxièmes ont vocation à servir dans un domaine précis: relations économiques, militaire (OTAN). La CECA était une OI extrêmement spécialisée. \\
Il ne faut pas confondre les OI à objet spécial avec ce qu'on appelle des institutions spécialisées. Ce dernier terme a une signification juridique bien précise, ce sont des institutions visées à l'article 57 de la charte des Nations Unis. \\
Il ne faut pas non plus confondre l'idée de spécialisation et le principe de spécialité. Toutes les OI sont gouvernés par le principe de spécialité. C'est même une différence juridique principale entre les États et les OI. Une OI n'exerce que les compétences qui lui ont étés donnés par le traité constitutif. \\
Enfin, ce n'est pas parce qu'une OI a de nombreuses compétences que cette OI a de nombreux pouvoirs. 

\subsection{Les pouvoirs d'une OI}

Ces pouvoirs, on va les caractériser en fonction des limitations qu'ils apportent à la souveraineté des États qui sont membres de l'OI. Selon que la souveraineté des États subit des restrictions plus ou moins importantes, on distinguera les OI à base de coopération et les OI à base d'intégration. \\
Les OI de coopération sont des OI qui reposent essentiellement sur des mécanismes de concertation intergouvernementale, dont la principale caractéristique est de préserver la souveraineté des États qui sont membre de l'OI. Un premier indice pour repérer ce genre d'OI est d'aller regarder ses organes. Dans une OI de coopération, les organes seront généralement constitués de représentants des États membres. \\
Le plus souvent, les actes adoptés dans un organe d'une OI de coopération (actes unilatéraux), n'ont aucun effet obligatoire sur les États membre de l'OI. Il peut arriver que sur certains sujets, dans certaines OI, les organes aient reçu un véritable pouvoir décisionnel. Dans ce cas, les décisions doivent souvent être prise à l'unanimité et respecte donc la souveraineté au maximum. \\
Dans ces OI, le recours aux traités est assez fréquent. Les États coopèrent entre eux, au sein de l'OI, en élaborant et en adoptant des conventions internationales, qui vont porter sur des sujets divers. \\
La quasi-totalité des OI fonctionne sur le schéma des OI de coopération. 


Contrairement aux OI de coopération, les OI d'intégration procède d'une vision fédéraliste de l'unification régionale. Autrement dit, les États membres ont acceptés de déléguer une partie de leur compétence à une autorité supra-nationale. Cela peut se manifester de plusieurs manières. \\
Dans une OI d'intégration, les organes intergouvernementaux existent aussi Cependant, à côté de ces organes, on va en trouver d'autres, indépendants des gouvernements des États membres. L'indépendance de ces organes peut être liée au mode de désignation de cet organe, comme le parlement européen qui est indépendant car il est élu directement. La commission est considéré comme indépendante, même si c'est moins évident, car prévu surtout par les statuts des commissaires dans les traités. \\
Non seulement certains organes sont indépendants, mais en plus, ces organes peuvent édicter des actes et disposent donc d'un véritable pouvoir décisionnel. Ce pouvoir décisionnel va pouvoir s'exercer selon le principe de la majorité et non selon l'unanimité comme dans une OI de coopération. On note aussi que ces décisions, peuvent, pour certaines, s'appliquer directement dans les ordres juridiques internes et peuvent donc être invocables directement (effet direct). 


Cette distinction comporte des limites, sur le plan juridique notamment. \\
La première est celle de toutes classifications: il n'y a pas de catégorie d'OI clairement défini. Une OI peut mettre en oeuvre des principes inhérent aux deux catégories. \\
L'UE par exemple, si elle a poussée des mécanismes d'intégration là où personne n'a jamais poussé aussi loin ces mécanismes, utilise aussi des mécanismes venant des OI de coopération classique. 

\section{La notion d'organisation européenne}

On distingue classiquement les OI à vocation mondiale/universelle et les OI à vocation régionale. \\
S'agissant de ces dernières, ce sont des OI qui sont destinés à un groupe limités d'États, même si ce groupe peut être important. Ceux-ci vont être liés par une solidarité géographique, ou encore économique, politique, culturelle. L'idée est que ces OI régionales visent à rassembler des États dont les intérêts sur certains sujets peuvent être concordants. \\
Historiquement, la mise en place de ces OI est un phénomène généralisé. Les premières OI régionales sont apparus à la fin du XIXe siècle, d'abord sur le continent européen, mais on en voit apparaître très rapidement sur le continent américain. Dans la seconde moitié du XXe siècle, les OI apparaissent en Afrique puis en Asie. 


Le terme d'européen dans les OI européenne n'est pas un terme univoque: des OI rassemblent des États extra-européens, d'autres, non. \\
On peut s'interroger sur ce qu'est un État européen, surtout quand l'UE pose comme condition d'être un pays européen pour intégrer l'Union. \\
Par exemple, la question de la Turquie montre que la notion d'européanité n'est pas univoque. \\
L'Islande aussi est un bon exemple: c'est une île qui est plus proche du continent américain que du continent européen. Pourtant, l'Islande fait partie du conseil de l'Europe, de l'AELE, etc. La candidature de l'Islande pour rentrer dans l'UE avait été acceptée sans aucun problème (aujourd'hui, elle a été stoppée par la volonté du peuple Islandais). Dans le cas de l'Islande, c'est l'histoire qui a décidé de sa géographie puisque colonisé par des colon norvégiens. L'Islande a toujours été considéré comme une terre européenne. Jusqu'à la fin de la seconde guerre mondiale, c'était d'ailleurs le Danemark qui administrait l'Islande. \\
On pourrait aussi parler de Chypre, qui, géographiquement, se situe plutôt en Asie mineure qu'en Europe. Chypre fait partie, depuis longtemps, du conseil de l'Europe. Une partie de l'île de Chypre fait également partie de l'UE, pour des raisons politiques. La question de l'européanité de Chypre ne s'est jamais posée lors de sa candidature à l'UE. \\
La question de l'européanité de la Turquie ne s'est pas posée pour son intégration dans le Conseil de l'Europe. Elle est également membre de l'OTAN. La candidature de la Turquie à l'UE a été acceptée, malgré sa position géographique. Pour comprendre pourquoi cette candidature a été accepté, il faut regarder l'histoire où l'on voit que les relations entre la Turquie et l'Europe sont des composantes importantes de l'histoire politique: l'empire Ottoman a été en Europe pendant 6 siècles dans les Balkans. 

\chapter{L'OSCE}

Il s'agit de la plus grande organisation régionale de sécurité au monde, en terme de nombre d'États adhérents. Cette OI tente de maintenir la stabilité, la paix et la démocratie pour plus d'un milliard de personnes. Elle tente cela à travers essentiellement le dialogue politique. \\
Cette OI est assez surprenant, notamment du point de vue de son évolution historique. La création de la CSCE (Conférence sur le Sécurité et la Coopération en Europe) aux débuts des années 70 intervient dans un contexte très précis: la guerre froide. C'est une institution de la guerre froide, chargée d'essayer de maintenir un certain dialogue à un moment où les tensions peuvent être importantes. On pouvait donc penser que cette OI prendrait fin avec la guerre froide. Elle va cependant rester et prendre une nouvelle forme en développant ses missions, va continuer à s'institutionnaliser, au point où on va transformer la CSCE à l'OSCE. \\



\section{Les origines de la CSCE}



\chapter{Le Conseil de l'Europe}

C'est une organisation intergouvernementale européenne spécialement chargée de la protection des droits de l'Homme, de la démocratie, et de l'État de droit. Le Conseil de l'Europe est essentiellement une organisation de coopération politique: elle repose sur le principe du fonctionnement intergouvernemental.  \\
Elle a été créée le 5 Mai 1949 par dix États fondateurs: la France, la Belgique, le Luxembourg, les Pays-Bas, le Danemark, l'Irlande, l'Italie, la Norvège, le Royaume-Uni, la Suède. Elle compte aujourd'hui 47 États membre pour une population de 820 millions d'habitants. \\
Il ne faut pas confondre le Conseil de l'Europe avec l'UE, même si tous les États membres de l'UE sont membres du Conseil de l'Europe. Il ne faut pas non plus le confondre avec le conseil européen, étant une institution de l'UE qui réunit à l'occasion de sommets les chefs d'État membre de l'UE. \\
Le Conseil de l'Europe est une OI dont le siège se trouve à Strasbourg. \\
Les activités du Conseil de l'Europe sont principalement connus sous l'angle de la Convention de Sauvegarde des Droits de l'Homme et des Libertés fondamentales du 4 Novembre 1950 dite CEDH. La CEDH comporte le mot européen alors que juridiquement, peu de textes dans cette convention possède le mot européen, et si elle est proposé par le Conseil de l'Europe, tous pays peut la signer. \\
Le Conseil de l'Europe évoque la grande Europe à contrario de l'UE qui est la petite Europe, d'intégration. Les symboles sont commun avec l'UE: le drapeau, l'hymne. 

\section{L'organisation du Conseil de l'Europe}

\subsection{Les origines du Conseil de l'Europe}

C'est une OI dont l'objet principal est la coopération politique entre ses membres. Dans cette perspective, créé le 5 Mai 1949, le Conseil de l'Europe constitue une innovation radicale dans le contexte de l'après guerre mondiale, surtout dans un continent où les premiers États-Nations se sont déclarés souverain les uns contre les autres. Du XIIIe siècle au XXe siècle, l'Europe connaissait un conflit majeur par génération. \\
Le bouleversement qui suit la seconde guerre mondiale et en particulier avec la découverte des camps d'extermination a ainsi la portée suivante: un conflit armée entre l'Allemagne, l'Angleterre et la France n'est plus imaginable, et dès lors, les États devront coopérer. La souveraineté ne peut plus être absolue, elle doit être partagée, en coopérant. \\
Cette coopération doit être réalisé dans le cadre d'un Conseil de l'Europe, dont l'initiative revient essentiellement aux Britanniques, et en particulier à Winston Churchill dès 1946. L'instauration du Conseil de l'Europe a été fortement soutenu par les USA. Il s'agissait pour eux d'inscrire durablement l'Europe dans un processus de libéralisme (démocratie libérale et libéralisme économique). \\
La question Allemande est aussi au coeur du processus de construction du Conseil de l'Europe. Le 5 Mai 1949 est instauré le Conseil de l'Europe alors que la RFA se dote d'une Constitution démocratique et fédérale le 8 Mai 1949. La volonté est d'ancrer la coopération démocratique dans le processus de construction européenne. Le Conseil de l'Europe, sous cet angle est une innovation du XXe siècle, mais pourtant, ses origines intellectuelles sont plus anciennes.

\subsubsection{Les origines intellectuelles du Conseil de l'Europe}

L'idée d'une Europe unie figure dans de nombreuses oeuvres intellectuelles qui ont toutes en commun une volonté de limiter la toute puissance de l'État Nation, essentiellement à partir de la Renaissance. La volonté de ces intellectuels est d'instaurer un territoire de paix, de pacification des relations inter-étatique à partir d'institutions qui sont des institutions de coopération politique. Dans des oeuvre très anciennes, on retrouve des mécanismes repris par les diplomates. \\
On citera d'abord Emeric Crucé, un moine, un intellectuel ecclésiastique, professeur de mathématique, il écrit en 1623 "Le nouveau cynée", dans lequel on trouve un projet d'organisation européenne. Il estime nécessaire une assemblée permanente des États dont le siège serait à Venise en Italie. Cette assemblée serait composée d'ambassadeurs des États, et l'objet de cette assemblée serait de régler les différends survenant entre les États. L'assemblée en question n'a pas vocation à prendre des décisions qui s'imposent aux États, ceux-ci demeurant souverain. C'est donc bien un mécanisme de coopération, mais elle a le mérite d'introduire des mécanismes de coopération permanente. 


Ensuite, on citera Sully, qui, en 1635 propose le plan Sully à Henry IV. Sully envisage de remodeler politiquement le territoire Européen, à partir surtout d'un critère linguistique: il envisage de créer une Europe à 15 États d'importance égale afin d'éviter des conflits d'hégémonie. La coopération de ces 15 États seraient placés sous un Conseil de l'Europe composé des représentants des États. \\
Le Conseil aurait compétence pour traiter toutes questions d'intérêts communs aux membres mais aussi de régler les différends entre les États. \\
Pour Sully, le conseil devrait pouvoir prendre des décisions exécutoires qui s'imposeraient aux États. En ce sens, Sully promeut le principe d'une primauté du droit de l'organisation sur la politique et le droit des États. Il est formulé ici, l'idée d'un pouvoir supra-national, ou bien l'idée d'une confédération. 


Ensuite, William Penne, un quaker (un chrétien conservateur de valeurs pacifiques) anglais, qui va fonder la Pennsylvanie. En 1694, il écrit "Essai pour le présent et la paix future en Europe". Comme Sully, Penne propose la constitution d'une assemblée représentant les États européens. La particularité chez lui, c'est que l'assemblée ne repose pas sur le principe de l'égalité souveraine des États. Pour lui, les États devraient disposer d'un nombre de délégués en proportion de leur importance politique et démographique. Il invente le concept d'assemblée politique. Il devrait pas prendre ses décisions à l'unanimité mais à la majorité qualifiée de trois quarts. \\
C'est un projet très ouvertement supra-national. En cas de conflit, les décisions adoptés peuvent être imposés aux États membres dès lors que leurs délégués constituent la minorité politique. 


Ensuite, on a Emmanuel Kant, 1795, et son projet pour une "Paix perpétuelle". On trouve chez Kant un esprit fédéraliste. Il préconise que les États renoncent à la guerre et surtout se soumettent au droit, non étatique, et international. Pour Kant, la soumission au droit est une question de raison pratique. \\
Il reprend l'idée d'une primauté du droit sur l'action politique. Il milite pour une fédération des peuples qui ne formeraient cependant pas un seul même État mais une fédération ou une confédération d'États-Nations. 


Enfin, on finira par citer Victor Hugo (même si il y en a d'autres), il est l'auteur d'une formule importante le 28 août 1876, lors d'un discours devant l'AN. Il prononce l'idée d'États-Unis d'Europe: c'est à dire le passage d'une confédération à une fédération basé sur la solidarité de fait. 

\subsubsection{La jeunesse politique du Conseil de l'Europe}

Toutes ces idées sont repris avec vigueur par des politiques après la fin de la seconde guerre mondiale. Cela est présenté comme le moyen de sauvegarder la paix en conservant les nationalismes. \\
Le 19 Septembre 1946, Churchill prononce un discours important à Zurich en Suisse, lors duquel il invite les États Européens et tout d'abord la France et la future Allemagne à se diriger vers la construction des "États-unis d'Europe". C'est le premier politique à reprendre cette expression à son compte. \\
Il propose pour ce faire une approche pragmatique et progressive. Il s'agira d'installer un conseil européen. En 1947, un comité international de coordination est institué pour mettre en pratique cette idée de faire un conseil européen. Ce comité est placé sous présidence britannique. Ce comité convoque à La Haye du 7 au 10 Mai 1948 un congrès de l'Europe qui rassemble 800 personnalités, des Hommes politiques, des ministres en exercice, des diplomates, on y trouve aussi des parlementaires nationaux, des universitaires, des représentants des secteurs culturels, des philosophes, des secteurs industriels, des patrons, des syndicaux, des représentants des confessions religieuses. Ce n'est pas seulement une question politique mais une question sociale aussi. \\
À l'issue de ce congrès, un message aux européens est adopté, essentiellement destiné aux États. Le congrès exhorte ces États à la coopération politique et en particulier à la réalisation par cette coopération d'actions communes qui doivent toucher à deux domaines au coeur du maintien de la paix: la coopération militaire, l'union économique. \\
Le texte s'achève par quatre revendications: une Europe unie qui assure sur son espace la libre circulation des Hommes, des idées et des biens ; une déclaration européenne des droits de l'Homme qui proclame des droits fondamentaux communs aux peuples européens avec des effets contraignants ; l'instauration d'une cour de justice supranationale qui participe de la garantie du droit européen ; une assemblée européenne qui a pour base politique les peuples des États membres.


Les gouvernements français et belge vont décider de donner suite à ce message. Ils vont utiliser le cadre que constitue une autre organisation européenne existante: l'Union Occidentale (UO), une organisation militaire créée en 1948. \\
La France et la Belgique saisisse le conseil de l'UO d'une proposition en vue de la création d'une assemblée parlementaire européenne, en vue de le remplacer. \\
Un comité pour l'étude et le développement de l'unité européenne est créé et composé des cinq membres de l'UO, la France, la Belgique, les Pays Bas, Royaume-Uni, Luxembourg. \\
Deux tendances s'opposent à ce comité. La première est intégrationniste et est incarné par la France et les trois États du Benelux. Ces trois États sont membres d'une OI d'intégration, ils ont une union douanière entre eux (la dénomination Benelux vient de là). Les intégrationniste souhaitent l'instauration d'une organisation ouvertement supranationale, doté d'une assemblée parlementaire ayant la capacité de prendre des décisions juridiquement contraignantes. \\
La tendance intergouvernementale est incarné par le RU qui ne veut pas entendre parler d'une assemblée autre que la leur ni opérer de transfert de compétences étatiques vers une organisation. \\
Un compromis entre les deux tendances est rapidement atteint pour un Conseil de l'Europe, dotée de deux organes, une assemblée parlementaire composée par les parlementaires des États membres avec une compétence purement consultative. Le deuxième organe est un comité des ministres qui réunit en principe les ministres des affaires étrangères des États membres. C'est un organe intergouvernemental qui préserve la souveraineté des États, c'est le seul compétent pour prendre des décisions contraignantes.


Le statut du Conseil de l'Europe est définitivement adopté le 5 Mai 1949. Ce statut est imprégné d'une idéologie bien précise, qui permet de résoudre deux questions: la question allemande, où il est établi dans le préambule du statut que l'Allemagne et que les États européens en général ont vocation à adhérer à ce statut, en perspective d'une réconciliation. \\
Ensuite, cela permet de traiter la question du communisme politique car l'idéologie qui est celle du statut de Londres est celle du libéralisme. La construction du Conseil de l'Europe se fait sous la surveillance étroite des USA, qui étaient très présent à l'époque (via le plan Marshall). Par ailleurs, le statut illustre une opposition frontale à l'Europe communiste, centrale et orientale, qui est en cours de construction. L'idéologie libérale doit donc être un marqueur de l'Europe occidentale par opposition à l'Europe de l'Est, c'est pourquoi elle est explicitement formulée. \\
Il est formulé qu'une démocratie véritable a un libéralisme économique. Il est sous entendu que la démocratie populaire, telle que pratiqué par les communistes est fausse. \\
Le Conseil de l'Europe incarne donc en quelques sortes une internationale libérale, essentiellement promu par les USA, qui va très profondément orienté la construction des partis politiques des Gouvernements, notamment des partis de gauche. Pour pouvoir agir dans le cadre des institutions européennes, les partis de Gauche vont progressivement se convertir au libéralisme. 

\subsection{La composition du Conseil de l'Europe}

Le Conseil de l'Europe compte aujourd'hui 47 États membres. 

\subsubsection{Les membres du Conseil de l'Europe}

On distingue des membres originaires de membres adhérents à l'organisation. \\
Il y a 10 membres originaires, que l'on a déjà cité. On compte les cinq membres de l'UO:  France, RU, Benelux. D'autres États se sont joints pour signer le statut de Londres: Irlande, Italie, Danemark, Norvège, Suède. \\
Ensuite, on a 37 membres adhérents. Ce sont ceux qui ont suivi la procédure d'adhésion. Les deux premiers, le 9 août 1949, ce sont la Grèce et la Turquie. Intègre ensuite, en 1950, en troisième et quatrième adhérent, la RFA, et l'Islande. En 1956, l'Autriche. 1961, Chypre. 1963, Suisse. 1965, Malte. 1976, Portugal. 1977, Espagne. 1978, Liechtenstein. 1988, Saint-Marin. 1989, Finlande. \\
Avant la chute du mur de Berlin, le Conseil de l'Europe compte 23 États. Une ouverture à l'est et au centre européen permet de conduire de nombreux États dans une dynamique libérale. \\
1990, quelques mois après la chute du mur, le premier État de l'Europe centrale à adhérer est la Hongrie. Suit en 1991, la Pologne. \\
En 1991 a lieu la dissolution de l'URSS, ce qui va permettre l'entrée massive d'État au Conseil de l'Europe. 1992, Bulgarie. 1993, Estonie, Lituanie, République Tchèque, Slovaquie, Roumanie, Slovénie. 1994, Andorre. 1995, Lettonie, Albanie, République de Macédoine, Moldavie, Ukraine. 1996, Croatie, Fédération de Russie. 1999, Géorgie. 2001, Arménie, Azerbaïdjan. 2002, Bosnie-Herzégovine. 2003, Serbie. 2004, Monaco. 2007, Monténégro.  \\
On s'aperçoit qu'il y a peu d'États européens qui ne sont pas membre du Conseil de l'Europe comme la Biélorussie.

\subsubsection{L'adhésion au Conseil de l'Europe}

Les conditions de l'adhésion. \\
Elles sont fixées dans le statut de Londres du 5 Mai 1949, en son article 4: "Tout État européen jugée capable de se conformer aux dispositions de l'article 3 et comme ayant la volonté, peut être invité par le comité des ministres à devenir membre du Conseil de l'Europe". \\
Article 3: "Tous membres du Conseil de l'Europe reconnaît le principe de la prééminence du Droit et le principe en vertu duquel toutes personnes placées sous sa juridiction doit jouir des Droits de l'Homme et des Libertés fondamentales".


L'article 4 détermine trois conditions pour être membre adhérent, une première est une condition formelle: le membre adhérent doit avoir été invité par le Conseil des Ministres. Le Conseil de l'Europe se présente donc comme un "club" des démocraties européennes, tous les États ne peuvent pas poser une candidature. Ils doivent être invités par le principal organe de l'organisation: le comité des ministres. On est davantage dans une logique d'intégration que dans une logique de coopération avec ce mode d'adhésion. \\
Après avoir recueilli l'avis de l'assemblée consultative, le comité des ministres adresse donc une invitation à l'État de devenir membre de l'organisation. La décision d'adhésion est prise par le comité des ministres à la majorité des deux tiers sous la forme d'une résolution (art. 20 du statut). L'adhésion peut donc se faire contre la volonté d'un membre.


Condition géo-politique: être européen. Cette question est une question géopolitique et non géographique. La géographie est bien sûr prise en compte, mais les critères sont aussi culturels mais aussi politique. L'adoption d'un régime politique de type occidental (démocratie, laïcité...), d'une économie de marché, sont des éléments qui sont pris en compte. C'est une raison pour laquelle la Turquie a été considéré dès 1949 comme européen: une partie de son territoire est européen, son organisation politique est occidental (laïque et démocratique), l'histoire de la Turquie depuis le Moyen-âge est européenne. \\
Il en résulte que le Conseil de l'Europe dispose d'un espace territorial particulièrement étendu avec des frontières communes avec des États telle que la Corée ou la Chine. Cette condition de l'État européen permet de dire que l'organisation n'est pas rigoureusement fermé aux États clairement non européen. \\
Des États non européens peuvent avoir un statut observateur, tout en n'en étant pas membre, ils participent aux travaux. Les USA par exemple le sont, le Canada aussi, ou encore le Mexique, le Japon, et même le Saint-Siège. Cette qualité d'observateur n'est pas prévu par les statuts du Conseil de l'Europe mais permet aux États tiers de participer à la coopération politique à l'oeuvre, sans pour autant exercer le pouvoir décisoire. Cette participation est loin d'être marginale. \\
La présence d'un représentant du pape peut poser des questions sur la laïcité de l'organisation. \\
Les USA ont été associés aux travaux du Conseil de l'Europe à partir du 10 Janvier 1996. Cette entrée confirme à nouveau l'intérêt des américains pour tout ce qui concerne la construction européenne. Il est fort probable que l'intérêt américain ait été attisé par l'adhésion de la Russie. 


La troisième condition est matérielle, il faut que l'État soit démocratique. C'est peut être le sujet qui peut être le plus sujet à controverse. C'est le comité des ministres qui apprécie du caractère démocratique des États qu'il invite après consultation de l'assemblée consultative. Elle envoie sur place des missions d'information, composé d'experts et de délégués parlementaires qui vont examiner le fonctionnement des institutions de l'État au regard des exigences de l'article 3. \\
Ces missions examinent en particulier le respect du principe de la démocratie pluraliste et parlementaire. C'est ainsi par exemple que l'Espagne n'a pas été admise immédiatement au Conseil de l'Europe après la chute du régime franquiste en 1975. L'Espagne n'a été admise qu'en 1977, les organes du Conseil de l'Europe ayant décidé d'attendre l'organisation d'élections libre en Espagne. \\
Cette condition a conduit au rejet de l'adhésion de la Biélorussie, qui est un État autoritaire, à la frontière occidentale de la Russie. Il est placé sous la dictature d'un Président élu au suffrage universel direct, mais sur la base d'un parti unique. Il y existe pas ou peu de libertés publiques. C'est un État où est pratiqué l'oppression. \\
Le critère démocratique a été progressivement affiné. Tout d'abord, les États candidats doivent s'engager à adhérer à la CEDH. Ensuite, depuis le sommet européen de Vienne de 1993, les États doivent s'engager à ratifier les différents instruments du Conseil de l'Europe qui protège les minorités nationales et linguistiques. \\
Cette condition est sujet à controverse, car ont été admis dans le Conseil de l'Europe des États dont on ne peut pas fermement admettre qu'ils sont démocratiques et qui pratiquent ou ont pratiqués la violation des droits fondamentaux. Le premier exemple est celui de l'Ukraine dont l'adhésion est de 1995, qui s'était engagé à interdire la peine de mort en application du protocole 6. Cependant, en 1996, on dénombrait, après son adhésion 179 exécutions capitale, faisant de l'Ukraine le deuxième pays pratiquant la peine de mort. \\
Le deuxième exemple est la Russie dont l'adhésion date de 1996. L'assemblée consultative était majoritairement pour cette adhésion. Pourtant, au même moment, la Russie était engagé dans un conflit interne en Tchétchénie, où le Président promettait d'écraser cette population. L'adhésion a tout de même eu lieu, dans l'objectif d'éviter un retour au communisme. Il faut noter la pression très forte exercé par les 15 membres de l'UE de l'époque pour cette adhésion, qui ont fait du chantage de ne pas les intégrer dans l'UE. Il y a aussi des raisons stratégiques: instaurer des relations plus étroites entre la Russie et l'occident, en particulier en économie. 


La participation de l'UE au Conseil de l'Europe. \\
Les 28 États membres de l'UE sont membres du Conseil de l'Europe. Pourtant, l'UE ne peut pas être membre à part entière du Conseil de l'Europe car sur le plan de sa personnalité juridique, l'UE n'est pas un État mais une OI. \\
En revanche, rien n'interdit une participation étroite de l'UE aux travaux du Conseil de l'Europe. Cette question de la participation se pose pour plusieurs raisons: tous les États membres de l'UE sont membre du CdE. L'adhésion au CdE est considéré comme un point de passage obligé pour adhérer à l'UE. Les symboles des deux organisations sont les mêmes: l'hymne européen, le drapeau à douze étoiles sur fond bleu, sont les symboles du CdE que la CEE a adopté en 1984. L'Europe des 28, l'UE, étant une organisation d'intégration, entend toujours parler d'une seule voix dans le cadre des OI au sein desquels tous ses États membres sont parti, par la voie du commissaire européen compétent, le commissaire aux droits de l'homme.


Il existe des mécanismes de coopération juridictionnelle informelle entre la CEDH et la CJUE. Par exemple, on peut noter que la CEDH a eu l'occasion à plusieurs reprises, de contrôler le respect par le droit de l'UE de la CEDH. Cela en raison de la participation des 28 États membres de l'UE au Conseil de l'Europe: CEDH, 18 Février 1999, Matthews c. Royaume-Uni. \\
Dans une perspective analogue, la CJUE applique régulièrement, très fréquemment, la CEDH en tant que source matérielle du droit de l'UE. La CJUE tire de la CEDH des principes généraux du droit. CJUE, 7 Janvier 2004, Cadet, était en question une affaire de transsexualisme et la possibilité pour un partenaire transsexuel de bénéficier d'une pension dans le cadre d'un mariage ; pour appliquer le principe d'égalité des rémunérations, la CJUE va faire référence à la CEDH et plus particulièrement, elle va faire référence à l'arrêt de principe rendu par CEDH, 2002, Goodwin c. Royaume-Uni. \\
Une troisième illustration est fourni par le nouvel article 6 alinéa 2 du TUE du 13 Décembre 2007 qui stipule "L'Union adhère à la CEDH". L'UE n'adhère pas au Conseil de l'Europe car il ne le peut pas, mais elle adhère aux outils de ce conseil. L'UE ne pourra être parti à la CEDH que quand un traité d'adhésion aura été signé, celui est prévu dans l'article 17, protocole 14 de la CEDH. Ce protocole de 2004 est entré en vigueur en 2010. On attend donc une adhésion de l'UE à la CEDH, ce qui n'est pas le cas pour le moment car une proposition d'adhésion a été adopté en Avril 2013, mais ce protocole d'adhésion a été estimé contraire au principe de primauté du droit de l'UE par la CJUE: CJUE, 18 Décembre 2014. 


\subsubsection{La perte de la qualité de membre}

Plusieurs hypothèses: la suspension du membre, un retrait temporaire des droits de représentation ; l'exclusion qui est définitive ; le retrait volontaire.


Concernant la suspension et l'exclusion, il s'agit de procédure politiquement délicate à mettre en oeuvre, elle pose des problèmes diplomatiques. Ce sont des cas rares, voire hypothétiques. Deux hypothèses peuvent se rencontrer. \\
La première est celle où un État membre n'exécute pas ses obligations financières. Dans cette hypothèse, le membre peut être suspendu de son droit de représentation, tant au comité des ministres qu'à l'assemblée parlementaire. \\
La deuxième est l'hypothèse dans laquelle il s'agit de sanctionner un État pour infraction à l'article 3 du statut (principe de la primauté du droit, la protection des droits de l'Homme). Cette deuxième hypothèse est réglée d'un point de vue procédural par l'article 8 du statut qui prévoit une procédure d'expulsion graduelle: une procédure qui va de la suspension jusqu'à un retrait forcé. Il y a et il y a eu des cas de suspension mais il n'y a jamais eu de retrait forcé. En cas de retrait forcé, l'État en infraction est invité par le comité des ministres à se retirer du Conseil de l'Europe. Si l'État concerné ne tient pas compte de cette invitation, le conseil des ministres peut décider à l'unanimité que l'État en question a cessé d'appartenir au Conseil de l'Europe. \\
La Grèce, en 1967, tombe dans la dictature des colonels. L'assemblée parlementaire demande au comité des ministres la suspension puis l'exclusion de la Grèce jusqu'au retour de la Démocratie. Si la Grèce n'a plus participé aux travaux des Conseils de l'Europe, ce n'est pas en raison de son retrait forcé mais d'un retrait volontaire: pour éviter l'humiliation diplomatique, la Grèce a décidé de sortir d'elle même. \\
Concernant la Turquie et l'invasion de Chypre en 1974, le coup d'État militaire de 1980, et la répression turque à l'égard des kurdes, on constate qu'aucune procédure n'a été engagée à son encontre. \\
Un exemple de procédure de suspension actuel concerne la Russie. Du fait de l'invasion par la Russie de la Crimée, l'assemblée consultative a décidée de suspendre les droits de représentation des 18 parlementaires Russes par une décision du 10 Avril 2014. En revanche, la Russie a toujours place au comité des ministres.


Concernant le retrait volontaire, tout État peut décider souverainement de se retirer de l'organisation. Il y a deux conditions: l'État doit notifier sa décision au secrétaire général du Conseil de l'Europe, l'État doit s'acquitter de ses obligations à l'égard du Conseil de l'Europe. \\
Il existe un cas de retrait volontaire, celui de la Grèce, pour échapper à un retrait forcé en Décembre 1969. 

\subsection{Les organes du Conseil de l'Europe}

Le siège des organes du Conseil de l'Europe se trouve à Strasbourg. Le choix de cette localisation présente un caractère hautement symbolique, elle marque dès 1949 la volonté affiché du rapprochement entre la France et l'Allemagne. Les langues officielles de l'organisation sont peu nombreuses: l'anglais en tant que langue international, le français en tant que langue diplomatique. \\
Plusieurs tentatives ont été effectuées pour rajouter des langues de travail, ce qui pose des problème pour la Russie, l'allemand (Autriche, Suisse, Belgique, Allemagne etc.). Les raisons de cet échec pour introduire de nouvelles langues sont essentiellement financières et tiennent au coût de traduction des documents. \\
Le statut de Londres détermine deux organes politiques principaux: le comité des ministres en tant qu'organe décisoire, l'assemblée parlementaire en tant qu'organe essentiellement consultatif. À ces deux organes politiques s'ajoutent un organe un administratif: le secrétaire général qui représente la permanence de l'organisation. Ils sont dit statutaire car sont prévus par les statuts. D'autres ont ensuite étés rajoutés.

\subsubsection{Les organes statutaires}

L'article 10 du statut stipule "les organes du Conseil de l'Europe sont le comité des ministres, l'assemblée consultative. Ces deux organes sont assistés par le secrétariat du Conseil de l'Europe"


1. Le Comité des Ministres. \\
C'est un organe intergouvernemental. Il est celui qui incarne la souveraineté des États membres, la logique de coopération des États membres. Le comité des ministres est le seul organe à pouvoir agir et décider au nom de l'organisation intergouvernemental. L'article 13 du statut précise: "Le comité des ministres est l'organe compétent pour agir au nom du Conseil de l'Europe". \\
Le comité des ministres est en principe composé de représentants ayant la capacité d'engager la responsabilité de l'État. Il s'agit donc d'un ministre et plus particulièrement, de ceux des affaires étrangères (article 14 du statut). En pratique, les ministres des affaires étrangères se font souvent remplacer par un suppléant qui est un autre membre du gouvernement, en France il s'agit du ministre (ou secrétaire d'État) aux affaires européennes.


Le comité des ministres est présidé à tour de rôle par les représentants des États membres. Le rythme de rotation est rapide: les présidences ne durent que 6 mois. Le rôle fondamental de la présidence est de déterminer l'ordre du jour, et donc de sélectionner les sujets traités par l'OI. Depuis Novembre 2016, la présidence est assurée par le ministre des affaires étrangères de Chypre. \\
Dans sa formation la plus élevé, donc que des ministres des affaires étrangères, le comité ne se réunit qu'une fois par présidence. À cette occasion de cette réunion solennelle, les ministres des affaires étrangères font le point sur les dossiers en cours. Ils donnent les impulsions nécessaires aux activités de l'organisation. À la dernière réunion solennelle, ont étés traités les sujets de la Syrie, de l'état d'urgence en Turquie, les décrets pro-corruption en Roumanie, la lutte contre le terrorisme. \\
Le comité des ministres, par ailleurs, fait le point sur l'exécution par les États des arrêts de condamnation prononcés par la CEDH. \\
Le comité tient en principe ses réunions à huis-clos sauf décision contraire. Ce qui explique cela, c'est que le secret facilite le dialogue et la coopération entre les gouvernements. 


Très rapidement, il a été créé un organe permanent chargé de préparer les travaux de la réunion solennelle, et capable d'agir à la place du comité. Chaque État possède donc un représentant à Strasbourg, un ambassadeur. \\
En pratique, cet organe permanent est l'organe de travail réel puisqu'il se réunit une fois par semaine. Les décisions prises par les ambassadeurs ont la même force juridique que les décisions adoptés par les ministres des affaires étrangères. 


Les attributions du comité. \\
Selon les statuts, le comité est capable d'agir au nom du Conseil de l'Europe. Le Comité peut donc prendre des décisions dans tous les domaines de coopération de l'organisation. Différents modes de votations sont utilisés: l'unanimité tout d'abord des suffrages exprimés et au moins la majorité des membres est requise pour les questions les plus importantes. Ces questions les plus importantes sont la réalisation des buts de l'organisation, l'approbation du rapport d'activité annuel, les amendements aux stipulations du statut, toutes questions qu'un membre considère comme étant importante. \\
La majorité des deux tiers des membres est également requise pour l'invitation d'un État à devenir membre du conseil. Majorité des deux tiers des suffrages exprimées et majorité des membres pour toute autre résolution. Le mode de votation de droit commun est donc aux deux tiers. \\
Le comité reflète donc la logique de coopération intergouvernementale. Ce sont les représentants des États et non l'assemblée consultative qui agissent au nom du Conseil de l'Europe, ce sont ensuite les États, qui adoptent les décisions les plus importantes à l'unanimité, mais les modalités de vote des deux tiers signalent également qu'il y a un certain mécanisme d'intégration. 


2. L'assemblée consultative \\
C'est la première assemblée parlementaire internationale dans l'histoire des OI. C'est la première fois qu'un organe plénier est conçu de sorte à ce que les membres expriment librement leurs idées personnelles et non les idées de leurs gouvernements d'appartenance. En ce sens, l'assemblée consultative a créé un précédent utile pour la création d'autres institutions analogues, notamment pour le parlement européen. \\
L'article 22 du statut fixe la fonction de l'assemblée en stipulant que c'est l'organe délibérant du Conseil de l'Europe. C'est à dire qu'il discute publiquement, contrairement au comité. \\
Il y a une incertitude: l'assemblée a été créé sous le titre d'assemblée consultative. Pourtant, on parle d'assemblée parlementaire pour cet organe. Cette assemblée parlementaire a néanmoins aucun pouvoir normatif. 


Concernant la composition l'assemblée consultative. \\
Conformément à la volonté des fondateurs de l'OI, l'assemblée consultative illustre les aspects supra-nationaux ou intégrateur du Conseil de l'Europe. Les membres de cette assemblée ne sont pas des représentants des gouvernements des États comme c'est le cas au comité des ministres. Les membres de l'assemblée consultative sont des parlementaires, il s'agit donc de personnalités politiques élus au suffrage universel, et ont donc vocation à représenter les peuples des États membres. \\
Ils ne sont pas élus comme le sont les députés au Parlement européen. Les délégués parlementaires à l'assemblée consultatives sont désignés par les parlementaires nationaux. Ils sont désignés via un suffrage universel indirect. En pratique, les modes de désignation varie en fonction des États membres. Pour la France, l'AN et le Sénat désignent chacun des délégués, 12 députés sont délégués à l'AC du Conseil de l'Europe, 6 du côté des sénateurs. \\
La durée du mandat du délégué varie là encore en fonction de l'État membre. Le mandat du délégué est de 5 ans en France, 4 ans pour les délégués Allemands, 1 an seulement pour les délégués Belges. \\
La composition de l'AC est calculée sur la base d'une pondération par État comme c'est le cas au Parlement Européen. Cette pondération tient compte d'abord du poids démographique de l'État (plus il est peuplé, plus il a de délégués), ensuite de l'importance politique de l'État (plus il a un leadership, plus il aura un nombre important), enfin, la pondération tient compte de la participation financière. \\
L'AC comporte 318 délégués. Cette composition reflète un principe d'équilibre entre les grands États et les plus petits. Il existe 5 grands États qui comptent le nombre maximum de délégués, soit 18: la France, le Royaume-Uni, l'Italie, l'Allemagne, la Russie. L'équilibre est très relatif, on comprend mal que la Russie puisse n'avoir que 18 délégués alors qu'ils ont plus de 300 millions d'habitants. De même, on comprend mal que la Turquie n'ait que 15 délégués alors que l'État Turque a presque 80 millions d'habitants. Cela contraste avec les micro-États qui disposent tous de deux délégués au minimum. 


Concernant le fonctionnement. \\
On constate que l'assemblée consultative s'est progressivement parlementarisé. \\
Le premier aspect de la parlementarisation de l'AC est la constitution parmi les délégués de groupes d'appartenance politique. Pour être reconnu et avoir une existence officielle au sein de cette AC, un groupe politique doit avoir 20 délégués d'au moins 6 nationalités différentes. Il existe actuellement 5 groupes politiques. Le plus important est le PPE, qui a son équivalent au parlement européen, c'est un parti de droite parlementaire plutôt conservateur. Le deuxième est le groupe socialiste, la gauche parlementaire classique. Le troisième est l'alliance des démocrates et des libéraux pour l'Europe, c'est un groupe relevant du centre-droit. Le quatrième est celui des conservateurs européens. Le cinquième est celui de la gauche unitaire européenne. Le Président de l'AC est logiquement un membre du PPE, désigné pour deux ans, il s'agit d'un espagnol, Pedro Agramunt. \\
Deuxième aspect de la parlementarisation, le fonctionnement de l'AC s'est progressivement organisé en sessions parlementaires. Selon les statuts, l'assemblée doit tenir une session ordinaire annuel qui ne doit pas excéder un mois. Dans la pratique, l'AC a pris l'habitude de diviser la session ordinaire en quatre parties, chacune durant une semaine. Première session en Janvier, puis deuxième en Avril, troisième en Juin, et enfin, quatrième en Septembre. Les débats sont publics, ce qui permet à l'AC de constituer une tribune politique comme c'est le cas au Parlement européen. \\
Troisième aspect, l'AC a créé pour palier à la limitation de la session plénière, des commissions permanentes de travail et qui sont chargés de préparer les travaux de la session ordinaire plénière. \\
Enfin, quatrième aspect, la pratique est venu introduire un mécanisme de questions écrites et orales à destination du comité des ministres. Ce mécanisme des questions confinent à l'instauration d'un contrôle parlementaire de l'organe intergouvernemental, qui est mis sous pression par l'AC. 


Concernant les attributions de l'assemblée consultative. \\
Selon l'article 23 du statut de Londres, l'Assemblée peut délibérer sur toutes questions entrant dans les compétences de coopération du conseil de l'Europe. Selon leur objet et leurs destinataires, les actes adoptés par l'AC ont une dénomination différente. Lorsque l'AC propose au comité des ministres d'agir dans un sens déterminé, l'acte prend la forme d'une recommandation. Lorsque l'AC fait connaître son point de vue aux États membre sur leur fonctionnement démocratique, sur le respect des droits des libertés fondamentales, l'acte prend la forme d'une résolution. Enfin, lorsque l'AC adopte un acte sur demande du comité des ministres, l'acte prend la forme d'un avis. \\
Quel que soit ces actes, l'AC ne dispose d'aucun pouvoir d'adopter des décisions juridiquement contraignantes. Elle n'a pas de pouvoir décisoire. Il faut cependant noter que l'AC dispose d'attributions électives fondamentales: c'est l'AC qui procède à l'élection de trois fonctions essentiels, elle élit le secrétaire général de l'organisation ; elle élit le commissaire au droit de l'Homme ; elle élit les juges à la CEDH. \\
L'AC assume essentiellement un rôle de veille politique. Un premier sujet sous la surveillance de l'AC concerne l'instauration d'une procédure de médiation en cas de violence faites aux femmes en Roumanie, ou encore, dans ce même pays, les mécanismes pour favoriser la corruption. L'AC surveille aussi l'Azerbaïdjan qui incarcère systématiquement les opposants politiques.

3. Le secrétariat général. \\
Le secrétariat du Conseil de l'Europe est composé d'un secrétaire général, d'un secrétaire général adjoint et d'un personnel administratif d'environ 2000 agents. Ils bénéficient d'un statut de fonctionnaire international, ils sont rémunérés par l'organisation et sont soumis à des obligations d'indépendance à l'égard des États dont ils sont ressortissants, ils ont des égards de loyauté à l'égard de l'organisation. Ils bénéficient de privilèges et immunités. Immunité juridique d'abord, ils ne peuvent pas être poursuivi de manière ordinaire, et ont des privilèges sociaux et fiscaux, ils n'acquittent pas ni d'impôt sur le revenu ni la TVA, de plus ils ne sont pas associés au régime de sécurité sociale de leur pays. Cela leur permet l'indépendance. \\
Le secrétaire général et son adjoint sont élus par l'AC sur la base d'une recommandation du comité des ministres. Le secrétaire général est élu pour 5 ans, comme son adjoint. Il s'agit actuellement d'un Norvégien. Du point de vue de ses fonctions, selon le statut, le Secrétaire général n'a qu'une fonction administrative: il assure au quotidien la gestion de l'OI. Et pourtant, en fonction de la personnalité du secrétaire général, la fonction du secrétariat peut revêtir des aspects hautement politique puisque dans l'intervalle des réunions, les questions sont traités par le secrétariat général. On note également que le secrétaire général a accès au comité des ministres, il participe au débat, et depuis 1955, il peut inscrire à l'ODJ toute question relative aux objectifs et aux compétences du Conseil de l'Europe. \\
Dans la pratique, le Secrétaire général joue un rôle essentiel, il en assume la permanence. De plus, c'est le secrétaire général qui représente le Conseil de l'Europe à l'extérieur, au sein des États membres ou tiers ou au sein des autres OI. 

\subsubsection{Les organes non statutaires}

Ils sont plus de 70. Il s'agit d'organes créés dans la pratique, créé par des actes de droit dérivé. \\
On peut citer tout d'abord la réunion des Chefs d'État et de Gouvernement. Le Conseil de l'Europe a vu progressivement l'organisation de sommets du Conseil de l'Europe. Le premier s'est tenu à Vienne en 1993, réunissant les Chefs d'État et de Gouvernement. Le contexte était propice puisqu'il s'agissait de célébrer la première vague d'adhésion des pays d'Europe orientale. Ce sommet a permis au Conseil de l'Europe de préciser les critères de l'adhésion à l'organisation, notamment obliger les États candidats à ratifier la CEDH avant leur adhésion mais également l'obligation faites aux candidats de respecter les droits des minorités. \\
Deux autres sommets seront organisés, le deuxième à Strasbourg en 1997 et le troisième à Varsovie en 2005. 


Un autre organe non statutaire intéressant est le congrès des pouvoirs locaux et régionaux du Conseil de l'Europe. Il s'agit d'une assemblée regroupant les représentants des collectivités décentralisés ou fédérés des États membres. Dans le cadre de ce congrès sont débattus toutes les questions qui relèvent de la démocratie locale. Le Conseil de l'Europe ne s'intéresse pas uniquement à des questions de liberté, il s'intéresse aussi à des questions de l'ordre des organisations politiques des États membres.


Un autre organe auquel on peut faire référence est le commissaire aux droits de l'Homme. C'est une institution récente, créé en 1999, du fait de l'élargissement à l'Europe de l'ouest pour assiste la CEDH dans la fonction de protection des droits fondamentaux. Le but est de désigner une personnalité qui fait office de médiateur. \\
Le commissaire aux droits de l'Homme est donc un organe qui veille au respect des droits de l'Homme et des Libertés fondamentales en bénéficiant d'un statut d'indépendance. Il agit de sa propre initiative. Le commissaire actuel est un ressortissant de la Lettonie. Mujnignicks est le commissaire actuel. \\
Pour agir en qualité de commissaire, il a des attributions complémentaires de la CEDH. Il peut notamment intervenir devant la CEDH en qualité de tiers intervenant car il connaît très bien le climat politique des États dans lequel les affaires surgissent. Il peut donc à ce titre présenter des observations écrites sur toutes les affaires traités par la Cour. Il peut assister aux audiences et exerce l'anicus curiae (l'ami de la cour). \\
Il rend aussi des rapports publics sur la situation des Droits de l'Homme dans les différends États. Il dispose pour ce faire d'un pouvoir de visite sur place, il doit pouvoir tout visiter: les locaux administratifs, les prisons, les hôpitaux, bref, tout lieu public où sont en oeuvre les libertés fondamentales. Le dernier rapport sur la France date de 2014 et le commissaire relève un certain nombre de difficultés: d'abord, l'insalubrité des prisons qui rend leur fonctionnement incompatible avec les stipulations de la CEDH et en particulier celle qui protège la dignité de la personne humaine ; ensuite, l'insalubrité des centres de rétentions administratifs où sont placés les étrangers en situation irrégulière et en particulier que le centre de rétention en sous sol du palais de justice de Paris est l'un des pires d'Europe et qu'il est comparable à ceux que l'on trouve en Moldavie ; l'opacité des dépenses des élus de l'AN et du Sénat. 

\subsection{Les activités du Conseil de l'Europe}

D'après le statut du Conseil de l'Europe, le but de l'OI est de réaliser "une union plus étroite entre ses membres". Le Conseil de l'Europe a vocation à examiner toutes questions d'intérêts communs, par la conclusion d'accords ou par l'adoption de coopération visant tous les domaines de la diplomatie. \\
Deux types d'activités doivent être envisagés: l'activité politique et l'activité normative.

\subsubsection{La coopération politique}

La volonté originale est celle d'une coopération politique tout azimut entre ses membres. C'est notamment ce que vise Churchill lors de son discours de Zurich. La coopération n'a pas d'objet pré-déterminé, le Conseil de l'Europe devait être un forum, un lieu de discussion permanent en terme de politique économique, sociale, culturelle, de politique de sécurité et de défense et enfin de protection des droits de l'Homme et de la démocratie. \\
Pourtant, la plupart des initiatives en terme de débat vont échouer dès lors que le Conseil de l'Europe va se concentrer sur les deux derniers aspects. Cela car les objectifs de coopération sont trop vastes pour une OI de cette nature, ils couvrent des domaines bien trop fondamentaux, notamment en matière de défense ou de politique sociale, les États ne vont jamais reconnaître au Conseil de l'Europe la compétence pour traiter de questions au coeur de leur souveraineté nationale. \\
La plupart des objets de coopération relevant du conseil de l'Europe vont être traitées plus efficacement par d'autres organisations européennes, comme l'UE. Les questions de défense sont réglés plutôt par l'OTAN que par le Conseil de l'Europe. Les questions de démocratie et en particulier la surveillance des élections ont étés traités par l'OSCE. La coopération politique s'est réduite donc à des questions de sécurité et de droits de l'homme. \\
L'activité du Conseil de l'Europe constitue donc essentiellement une activité normative.


\subsubsection{L'activité normative du Conseil de l'Europe}

L'activité juridique du Conseil est notable. Elle se caractérise par l'adoption d'engagements internationaux, initiés par le comité des ministres, préparés avec l'AC. Ces conventions, comme tout traités, ont vocation à être signées puis ratifiées. \\
Au mois de Février 2017, on compte 220 conventions internationales conclues dans le cadre du Conseil de l'Europe et qui une fois ratifiées constitue du Droit. Ces conventions n'interviennent pas exclusivement en matière de protection des droits fondamentaux. Elles interviennent dans d'autres domaines tel que la santé publique, la lutte contre la corruption, l'organisation administrative, etc. \\
La principale des conventions du Conseil de l'Europe, c'est la convention européenne de sauvegarde des droits de l'Homme, signée à Rome. \\
En matière de droits fondamentaux, on peut citer la charte sociale européenne, adoptée à Turin en 1961. La charte sociale Européenne est l'une des seules à énoncés ce qu'on appelle des droits sociaux. Y figurent par exemple la reconnaissance du droit au logement, du droit à la santé, du droit à l'éducation, du droit à une assistance sociale et médicale, du droit à un revenu minimum. Ce sont ce qu'on appelle des droits-créances, car ils impliquent de la part des États une action, une obligation positive d'agir, qui oblige à la création de services publics. La différence est notable avec la CEDH qui contient surtout des droits de première génération. \\
La charte sociale européenne introduit ensuite un contrôle du respect de ces droits par les États. Il s'agit d'un système de réclamation collective. Il permet à des associations représentatives (en général à des syndicats nationaux), de saisir un comité européen des droits sociaux, le CEDS. Le CEDS est saisi de requêtes où les requérants font état de la violation de la charte. Le comité européen des droits sociaux a pour mission de déterminer si les législations et si les pratiques nationales sont ou non conforme à la charte sociale européenne. Pour ce faire, il rend des rapports, statuant sur la recevabilité, mais également sur le fond. \\
Ce comité est composé de 15 membres indépendants mais qui n'ont pas la qualité de magistrat. Ce sont des experts en matière de droits sociaux. Le rapport établi par le comité n'est pas juridiquement obligatoire, il s'agit d'un avis, il est rédigé dans la forme d'une décision de justice, mais ce n'en est pas une. La décision revient au comité des ministres qui dispose d'une option: soir il l'entérine, soit il le rejette. En pratique, le comité des ministres a toujours validé les rapports du CEDS, ce qui en fait un organe quasi-juridictionnel. 







\section{La Convention de Sauvegarde des Droits de l'Homme et des Libertés fondamentales}


\part{La garantie juridictionnelle du droit européen}























\end{document}
