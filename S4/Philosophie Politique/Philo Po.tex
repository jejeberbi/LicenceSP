\documentclass[10pt, a4paper, openany]{book}

\usepackage[utf8x]{inputenc}
\usepackage[T1]{fontenc}
\usepackage[francais]{babel}
\usepackage{bookman}
\usepackage{fullpage}
\setlength{\parskip}{5px}
\date{\today}
\title{Cours de Philosophie Politique (UFR Amiens)}
\pagestyle{plain}



\begin{document}
\maketitle
\tableofcontents

\chapter{Introduction générale: la notion d'autorité politique}

\section{Qu'est-ce que l'autorité ?}

\subsection{Précisions sémantiques}

Pour bien comprendre une notion, il faut les distinguer de notions voisines, avec lesquelles elle peut se confondre. On peut partir de la notion de pouvoir, qui peut prendre deux formes distinctes, la forme de la puissance ou la forme de l'autorité. \\
La puissance renvoie à un rapport de force "brutal", c'est donc l'exercice d'une coercition qui prend la forme d'une violence. Max Weber dans "Le savant et le politique", définit la puissance "Toute chance de faire triompher au sein d'une relation sociale sa propre volonté, même contre les résistances". Thomas Hobes, dans le Léviathan, part de l'hypothèse de l'état de nature, état infra-politique où c'est le règne de la puissance, la guerre de chacun contre chacun. \\
L'autorité ne peut pas être confondu avec la puissance, c'est bel et bien une forme de pouvoir, mais limité par un cadre juridique. L'autorité est aussi un pouvoir qui rencontre une docilité, car il est reconnu comme légitime. L'autorité est conforme à l'intérêt général, car il permet d'éviter cet état de nature où chacun craint une mort violente. Max Weber sur l'autorité: "La chance de trouver des personnes déterminées, prêtes à obéir à un ordre", il fait donc bien ressortir l'idée de la domination consentie.


Hannah Arendt a écrit l'oeuvre "La crise de la culture", dans laquelle elle distingue l'autorité de la coercition, mais aussi de la persuasion. Elle dit "Là où la force est employée, l'autorité proprement dite a échouée". Avoir de l'autorité, c'est donc le moyen de faire l'économie de la puissance et de la coercition. L'usage de la violence est un aveu de faiblesse inavoué. \\
L'autorité exige une relation verticale, hiérarchique, asymétrique. La persuasion présuppose l'égalité et elle opère par argumentation, elle est donc incompatible avec l'autorité. L'ordre autoritaire, toujours hiérarchique s'oppose donc à l'ordre égalitaire. 


Aujourd'hui, l'autorité ne va plus de soi, c'est pourquoi on parle d'une crise de l'autorité qui est multiforme. On connaît une crise de l'autorité pédagogique, de l'autorité publique en général, de l'autorité politique bien évidemment avec une crise de confiance envers les hommes politiques. 

\subsection{Ordre politique et ordre moral}

Dans les deux, que l'on parle de la politique ou de la morale, on renvoie à une autorité. \\
Quand on parle de la morale, on renvoie à la conscience, c'est une autorité interne à l'individu. La morale renvoie au registre de l'autonomie. Si il y a sanction, elle est interne aussi: la culpabilité, le regret, le remord, ce sont des sanctions que l'individu s'inflige à lui même. \\
En revanche quand on est dans la sphère politique, on a à faire à une autorité externe, capable d'exercer sur les individus une contrainte. Si, dans la morale, on exerce l'autonomie, dans la sphère politique, on exerce l'hétéronomie.


Pourquoi l'institution d'une telle autorité externe est-elle nécessaire ? \\
Kant disait "L'Homme se caractérise par son insociable sociabilité". Cette formule est intéressante car les deux mots sont antinomiques, pour Kant, l'homme est à la fois l'un et l'autre. L'être humain est traversé par des tendances qui sont contradictoires. Il prend le contre-pied de Hobes et Rousseau, le premier disant que l'homme est fondamentalement méchant, le second qu'il est fondamentalement bon. Pour Kant, c'est bien plus compliqué, pour lui, l'homme est monolithe. \\
Par sociabilité, les hommes veulent vivre en société parce qu'ils ont conscience que cela leur est nécessaire. Aristote disait déjà que l'homme était un animal politique. Cependant, chacun tend à ne pas s'imposer à lui même les exigences liées à cette existence commune. Pour Kant, nous sommes des êtres de raison et avons conscience que pour vivre en société, nous devons réduire notre liberté pour ne pas perturber celle des autres. \\
Cependant, nous sommes aussi des êtres de désir, et nous sommes tous enclins à nous affirmer aux dépens des autres, par égoïsme ou égocentrisme. Sans ordre politique, les relations humaines seraient conflictuelles, passionnelles. On appelle cette situation l'anomie, où il n'existe pas de règles. \\
"L'homme est un animal qui a besoin d'un maître" Kant. Cela veut dire que si il n'existe pas d'autorité supérieure qui nous pousse à nous limiter, nous serons enclins à faire un mauvais usage de notre liberté. Spinoza nous dit que si les hommes étaient gouvernés spontanément par la raison, il n'y aurai pas besoin de justice. \\
Il y a donc une nécessité d'une autorité politique et du droit. Le droit vient d'un mot latin, "directus", qui signifie ce qui est en ligne droite, donc ce qui n'est pas courbe, pas tordu, dans un sens géométrique. En géométrie, pour tracer des lignes droites, rectiligne, il faut une règle, et dans la société, pour que les hommes soient droits, il leur faut des règles communes. 


Les Grecs opposaient la pléonexie et l'isonomie. La pléonnexie est la tendance à vouloir plus que sa part et à s'affirmer aux dépens d'autrui. Spontanément, les rapports humains ne sont pas droits, mais tordu, déséquilibré, asymétrique. \\
Pour contrer cette pléonexie, il faut une isonomie (iso, la même, nomie, la loi), un principe selon lequel on doit établir l'égalité de tous devant la loi commune. \\
Pour Kant, la Loi ne fait pas obstacle à la liberté, mais fait obstacle à ceux qui voudraient faire un mauvais usage de la liberté. \\
L'Anarchisme, au contraire, dira que là où est l'État, la liberté n'est pas. L'anarchie défend l'idée que l'État est anti-liberté. Pour un anarchiste, toute obéissance à une autorité extérieure est une abdication. \\
Le Marxisme dit que l'État est le représentant des intérêts de la classe dominante. C'est donc ici une critique de l'institution étatique. 

\section{Axes de problématiques}

L'autorité politique est un phénomène qui ne va pas de soi. La puissance renvoie à une violence immédiate, manifeste, relativement facile à comprendre. En revanche, l'autorité est un phénomène qui est à la fois plus complexe et plus énigmatique. \\
La Boétie, au XVIe siècle, écrit "Discours sur la servitude volontaire". C'est un oxymore, car la servitude est, par habitude, subi. La Boétie, lui, imagine une servitude volontaire, dans le sens où elle serait acceptée, consenti. Il va analyser cette anomalie qu'est la tyrannie. Comment se fait-il qu'un seul puisse dominer toute une multitude ? Comment se fait-il que des hommes dont la nature est d'être libre, puisse abdiquer leur volonté pour se soumettre à un tyran ? \\
La Boétie démontre que le pouvoir du tyran n'est pas intrinsèque (car l'homme seul est faible, face à la multitude), ce n'est qu'un pouvoir d'emprunt. Le tyran va fonder son pouvoir sur la construction d'une image. Ce n'est donc pas le pouvoir en tant que tel qui gouverne mais l'image. \\
Le tyran va dominer les uns aux moyens des autres, via une hiérarchie pyramidale. \\
L'éducation ou la coutume est aussi une explication de ce pouvoir. Avec la coutume, l'individu intériorise son statut subalterne. \\
Melville, en 1853, écrit "Bartleby le scribe", qui met en scène un narrateur homme de loi, qui engage un scribe, apparaissant comme un travailleur très consciencieux jusqu'à ce qu'il y ait une rupture, et qu'il refuse la relation verticale. 


\chapter{Les mécanismes de l'obéissance à l'autorité}

\section{Une passion: la crainte}

\subsection{L'anneau de Gygès}

C'est un récit que l'on trouve dans République, de Platon, lors d'un dialogue avec des sophistes. Les thèses des sophistes sont que les lois ne sont que de simples conventions dépourvu de valeur intrinsèque. Un sophiste donne le récit de l'anneau de Gygès, celui-ci est un berger, et un jour de tremblement de terre, il découvre un chevalier qui a un anneau au doigt. Il le prend et se rend compte que celui-ci a un pouvoir, celui de rendre invisible son possesseur. La question qui se pose est de savoir si il existe un homme assez bon, assez droit, pour rester juste alors qu'il peut persévérer dans l'injustice sans se faire prendre. \\
Dans le récit, Gygès va se faire passer pour un berger du Roi, va séduire la Reine, et tuer le Roi. \\
Ce récit nous montre que l'obéissance n'est pas quelque chose d'interne, l'individu ne s'impose pas, par lui même, d'être obéissant. L'obéissance se voit alors donc comme une contrainte externe. On a donc la crainte de la sanction, du châtiment, mais plus encore, du déshonneur. 

\subsection{Les sophistes: l'opposition entre nature et convention}

On trouve chez les sophistes, une opposition entre les "phusis" et les "nomos" (nature et convention/loi). \\
Les sophistes affirment et soutiennent la primauté de la nature. La nature c'est l'ensemble des désirs, des inclinations individuelles et égoïstes. À contrario, les règles ne sont que des conventions que nous respectons que dans le but d'éviter les sanctions et le déshonneur. \\
Hobbes, dans le Léviathan, dit que pour dissiper les passions humaines, l'État léviathan doit faire peur, inspirer la crainte. Hobbes est donc quelqu'un d'étatiste. Il dit que sans l'épée (without sword), les lois ne sont que des mots (words), dépourvus du pouvoir de se faire respecter. C'est une lecture pessimiste car selon Hobbes, on obéit aux lois que par crainte, et jamais par sens de l'intérêt général. 

\section{Le rôle de l'image}

Il n'y a pas de pouvoir politique pérenne sans ostentation. L'ostentation est l'action de se montrer. Ce qui fait autorité, ce n'est pas le pouvoir lui même, mais c'est l'image du pouvoir. L'image du pouvoir doit frapper nos sens et notre imagination. \\
Blaise Pascal prend, dans "Les pensées", l'image des magistrats et des médecins. Il dit que l'un comme l'autre sont des illusionnistes. Les médecins n'étant pas très compétent, ils soignent plutôt leur image que leur patient, en se distinguant du commun par des signes distinctifs, ce qui fait qu'on les respectes et qu'on les admire. On a à faire à une véritable mystification de la profession de médecin. \\
Pascal met les magistrats sur le même plan que les médecins, qui doivent faire croire, autant que les médecins car ils n'ont pas de vrai sens de la justice. Ils sont incapable d'établir, de connaître, des normes de justice qui soient valable universellement. \\
Le pouvoir fabrique de toute pièce un langage imaginaire capable de suppléer au vide. \\
Selon Pascal, il existe chez l'Homme deux facultés qui sont concurrentes: la raison et l'imagination. En droit (au sens de ce qui devrait être), la raison devrait dominer, elle devrait être souveraine. Dans les faits, c'est l'imagination qui domine. C'est une puissance dominatrice, elle subordonne la raison. \\
Avec la construction de cette image, le pouvoir va prendre une forme imagé, faisant appel à l'imagination, et c'est par la fascination qu'elle suscite que l'on s'y soumet. \\
Ce qui fait autorité, ce n'est pas le pouvoir, mais l'image du pouvoir.


On note trois notions qui sont liées entre eux. \\
D'abord, le pouvoir, qui s'applique sur des signes extérieurs. Il y a donc une communication, d'où les campagnes et les stratégies de communication. \\
Tout cela a pour but d'être capable d'exercer une influence: être capable d'agir, sur les consciences des uns et des autres ou même de l'inconscient. La domination est toujours médiatisé: elle passe par des intermédiaires comme des signes, des symboles. Il s'agit donc de maîtriser les représentations. \\
Il y a donc une dimension performative du langage (Austin: How to do things with words). On note deux types d'énoncé, un premier qui est un énoncé constatif, et un autre qui est un énoncé performatif. Le premier se contente de constater un état de fait qui préexistait. Le deuxième est celui qui accomplit une action, et qui engendre une situation qui n'existait pas préalablement. Quand on réfléchit à la politique, on s'aperçoit qu'elle met au premier plan cette dimension performative ou pragmatique du langage. Les hommes politiques ont donc la volonté de faire des choses, d'agir, d'influencer le cours des choses via la parole. \\
Notre monde politique et social est un monde de signes et de symboles. Pascal: "Les cordes qui attachent donc le respect à tel ou tel en particulier sont des cordes d'imagination". 


Peoplisation de la vie politique: cette notion renvoie d'abord à une inflation de la vie privée, c'est la propension des médias à appréhender les personnalités du monde politique à travers le prisme de leurs vies privées. La frontière entre la vie privée et la vie publique devient de plus en plus floue. \\
Quand on lit les Mémoire de De Gaulle, on se rend compte qu'il y avait avant une distinction bien claire entre l'individu et la fonction. Il y avait donc une séparation entre sphère publique et privée. Cette séparation, cette frontière, est remise en question par cette peoplisation de la vie politique. \\
Cette propension a tendance à normaliser les hommes politiques, remettant donc en question leur autorité. 

\section{Le rôle de la coutume}

\subsection{Une conception relativiste de la justice}

"Plaisante justice qu'une rivière borne, vérité au deçà des Pyrénées, erreur au delà", Pascal. Ce n'est pas parce qu'une loi est juste que nous lui obéissons, c'est parce que nous sommes accoutumés à lui obéir que nous lui obéissons et que nous la trouvons justes. \\
Pour Pascal, la coutume devient comme une seconde nature. \\
Dimension négative de la coutume: Pascal est un philosophe sceptique, il pense que la raison humaine ne peut atteindre aucun savoir universel. Il y a donc selon Pascal une infirmité de l'Homme qui ignore ce qu'est la Justice au sens universel du terme. Pour Pascal, il n'y aucun droit naturel, il n'y a que du droit positif. \\
La coutume est donc le fondement de notre ordre social, plus que la raison.


"La coutume de voir les rois accompagnés de gardes, de tambours, d'officiers, et de toutes les choses qui ploient la machine vers le respect et la terreur fait que leurs visages, quand il est quelque fois seul et sans ses accompagnement, imprime dans leurs sujets le respect et la terreur". \\
Pascal utilise le vocabulaire de la physique (machine, imprime, ploient), il insiste sur le fait que l'homme a un comportement mécanique. \\
Pascal met en évidence un mécanisme d'association car le Roi, on ne le voit pas tout seul. Son image est donc associé ici aux gardes, aux officiers, et donc, à fortiori, à une démonstration de force, qui impressionne donc les sujets. Il y a un mécanisme de répétition pour que l'association se fasse. \\
La puissance et le caractère impressionnant de l'environnement est transféré sur la personne même du Roi. \\
Pascal en conclut donc que lorsqu'on voit le Roi tout seul, on ne peut pas le regarder comme un homme normal ou ordinaire. C'est quelque chose qu'on ne contrôle pas.


Avec l'imagination puis la coutume, nous finissons par confondre l'image et la réalité. Nous finissons par confondre le simulacre et la réalité, le mot et la chose. \\
On a beau être lucide, dans les faits, on ne peut pas s'empêcher de considérer un magistrat, un Roi ou autre comme un être hors du commun. 

\subsection{La distinction entre trois types d'esprit}

Pascal, dans Les Pensées va établir une typologie des esprits. Il en distingue trois sortes: les naïfs, les demi-habiles et les habiles. \\
Les naïfs sont caractérisés par leur crédulité. Ils sont les proies de leur imagination, au moins à deux niveaux. D'abord, ils s'imaginent que les lois sont des normes justes en elle même. Ensuite, ils s'imaginent que entre eux et ceux qui possèdent, il y a un fossé, voire même une différence de nature. Ceux-là n'auront jamais tendance à remettre en question l'autorité. \\
Les demi-habiles prennent conscience qu'à la source de l'ordre social et politique, il y a une mystification, une tromperie, un leurre. Concrètement, ils savent que les lois sont réductibles à de simples conventions relatives et particulières. À un deuxième niveau, ils prennent conscience que les hommes qui ont l'autorité sont des hommes comme eux. Ceux là dénoncent cet ordre politique et sociale, qu'ils combattent pour créer un ordre rationnel. Pour Pascal, cela est impossible, et ceux-là sapent le fondement même du pouvoir. \\
Les habiles savent que les lois ne sont que des conventions arbitraires, mais ils savent également qu'elles sont nécessaire à la mise en ordre de la société. Ces lois ne sont peut être pas rationnels, mais elles sont raisonnables dans la mesure où elles garantissent un ordre social et politique. Pour l'habile, toutes les vérités ne sont pas bonnes à dire. Aucune société ne peut fonctionner sans une certaine d'opacité selon Pascal. Une transparence absolue conduirait à une déstabilisation de l'ordre politique et social. La mystification est un moindre mal, un mal nécessaire.  

\chapter{L'autorité à la recherche d'un fondement rationnel}

\section{Le légalisme de Socrate et le débat avec les sophistes}

\subsection{La justice selon les sophistes: une approche nihiliste}

Il faut se référer à un texte de Platon: "Le Gorgias", qui a existé, c'était l'un des plus grands orateurs de l'antiquité. Dans cette oeuvre, un personnage fictif apparaît, Callicles. Il va défendre sa propre conception de la justice, une conception nihiliste. C'est un sophiste, et, en tant que tel, il réfléchit par une opposition entre la nature et la convention. \\
Ici, Callicles parle de la justice naturelle, qui est présente dans la nature, et qui est fortement inégalitaire, en plus d'être spontanée, immédiate. Au fond, la nature veut que le plus fort ait plus que le plus faible. \\
À contrario, il existe la justice conventionnelle, la justice humaine, qui prétend établir entre chacun une égalité. Mais, selon Calliclès, cette justice humaine n'a aucune valeur car elle n'est pas naturel, mais est aussi contre-nature. C'est un artifice contre nature. Cette justice humaine n'est que l'expression et le produit du ressentiment de la jalousie des plus faibles. Pour Callicles, la vrai justice est la justice naturelle, qui favorise le plus fort. C'est un nihiliste car dit que la justice humaine n'est basée sur rien. 

\subsection{Le légalisme de Socrate}

Socrate va défendre l'idée d'un respect intransigeant, inconditionnel et absolu des lois. Son légalisme est décrit dans "Le Criton". \\
Socrate va faire une prosopopée, c'est une figure de rhétorique qui consiste à donner la parole à des êtres qui ne la possède pas. La loi, pour Socrate, lui dirait qu'il n'est pas juste qu'il échappe à la Loi. C'est grâce à la loi qu'est né Socrate, qu'a été éduqué Socrate, tout ce qu'il est, il l'est devenu grâce aux lois. Si les lois ne lui plaisait pas, il était libre de quitter la cité Athénienne. En restant, de manière tacite, Socrate s'est engagé envers les Lois. \\
Ce qui détermine Socrate à obéir aux lois, ce n'est pas la crainte mais un sens de la justice ou un sens civique. L'existence et la pérennisation de l'ordre public implique la soumission absolue de tout citoyen à l'ordre légal. 

\section{La notion d'État de droit}

\subsection{Position du problème: la justice humaine apparaît comme une impasse (comme une aporie)}

Kant: "L'Homme est un animal qui a besoin d'un maître", il a donc besoin d'être dominé. Si il ne l'est pas, il fera un mauvais usage de sa liberté. Sauf qu'il y aura forcément quelqu'un au sommet de la hiérarchie, se pose la question alors de la domination de celui-ci. La réponse de Kant est: "nul part ailleurs que dans l'espèce humaine". \\
"Le chef suprême doit être juste en lui même et pourtant être un Homme. Mais dans un bois aussi courbe que celui dont est fait l'Homme, on ne peut rien tailler de tout à fait droit". 

\subsection{La solution: la mise en place d'un État de droit}

L'État de droit, dans une première approche, c'est le gouvernement de la Loi (the rule of law, J. Locke). Il s'oppose à un gouvernement d'Homme, gouverné par une volonté particulière. L'État de droit, c'est donc la situation dans laquelle nul n'est au dessus de la loi, nul ne peut se soustraire à la loi. \\
La liberté politique consiste à ne jamais être soumis à la volonté arbitraire d'un autre individu. \\
Dimension juridique: prééminence du droit ; dimension morale: la garanti des libertés individuelles. \\
Pour que l'ordre légal puisse être effectif, la loi doit posséder certains attributs: être clair, car si elle est confuse, elle est incertaine et produit de l'insécurité juridique ; elle doit être générale, elle ne vise personne en particulier, elle ne fait pas de cas par cas, et pose des normes valables pour tout un chacun ; elle doit être public, car nul n'est censé ignorer la Loi ; la loi ne doit pas être rétroactive, ce serai une forme d'arbitraire ; elle doit être relativement stable, elle doit évoluer avec lenteur, sinon elle crée de l'insécurité juridique ; enfin, la Loi doit instaurer une forme d'égalité, ce que les Grecs appelaient dans l'antiquité l'isonomie, l'égalité de tous devant la Loi. 


Montesquieu, "De l'esprit des Lois". Dans cet ouvrage, Montesquieu crée sa théorie de l'équilibre des pouvoirs, de peur de tomber dans l'État monolithique, c'est à dire un État qui concentrerai et monopoliserai dans ses mains l'ensemble de tous les pouvoirs. \\
Il part d'une idée très simple: celui qui détient un pouvoir tend tout naturellement à en abuser. Il faut donc que les pouvoirs s'équilibrent. "Pour qu'on ne puisse abuser du pouvoir, il faut que, par la disposition des choses, le pouvoir arrête le pouvoir" Montesquieu. À tout pouvoir il faut opposer un autre pouvoir. Il faut donc que le pouvoir soit toujours limité, modéré, pour les libéraux politiques du XVIIIe siècle. \\
L'État de droit est un concept qui permet d'articuler obéissance à l'autorité et garanti des libertés individuelles. 


Les concepts de Démocratie et d'État de droit son souvent confondus. Tocqueville démontrait que dans une démocratie, il pouvait y avoir une tyrannie de la majorité. Au contraire, une monarchie peut être constitutionnelle et donc être un État de droit. \\
Le concept de la Démocratie est un concept politique, alors que l'État de droit est un concept juridique et moral. La démocratie désigne un régime politique qui accorde la souveraineté au peuple. 

\section{Les théories du contrat social}

Cette notion de contrat social est associée à certains auteurs, notamment Thomas Hobbes, qui soutient l'idée de l'absolutisme politique. Ensuite, on a John Locke, qui théorise le libéralisme politique. Enfin, on a Rousseau et son pacte démocratique. 

\subsection{Les principes fondamentaux du contractualisme}

\subsubsection{La dimension polémique}

Les théories du contrat social permettent de s'interroger sur la légitimité de l'ordre politique. De manière générale, les théories du contrat social reposent sur le principe selon lequel la souveraineté politique a son origine dans le peuple. Ces théories sont donc en opposition totale avec la théorie du droit divin selon laquelle "toute autorité vient de Dieu, quiconque résiste à la puissance s'oppose à la volonté de Dieu". Le prince est donc assimilé à un ministre de Dieu sur Terre. \\
La légitimité est ici descendante. Le contractualisme vise à faire l'inverse, puisque l'autorité viendrait des individus. Dans la théorie de droit divin, tout droit de résistance à l'autorité est par principe exclu. Bossuet, dans "La politique tiré de l'écriture sainte", écrit "Les Hommes naissent tous sujets". "Le trône royal n'est pas le trône d'un homme, mais le trône de Dieu lui même", ce serai donc un sacrilège de s'opposer au Roi. \\
Spinoza, dans "Le traité théologico-politique" dit "La théorie du droit divin est réductible à une pure et simple superstition". Tout cela conduit à une forme d'aliénation, car nous abdiquons notre raison, notre esprit critique, notre sens critique. "Faire que les Hommes combattent pour leurs servitudes comme si il s'agissait de leur liberté", définition de l'aliénation selon Spinoza. \\
La liberté qui existe chez les individus leur permet de passer un contrat. Les théoriciens du contrat social désacralisent le pouvoir politique. 

\subsubsection{Des théories du droit naturel}

Les théoriciens du contrat social partent du principe que les hommes naissent libre et non sujet. Pufendorf écrit "Le droit de la nature et des gens", il dit "Les jurisconsultes romains ont très bien reconnu que selon le droit naturel, tous les hommes naissent libre". \\
Le droit naturel est un droit par principe, antérieur et plus fondamental que le droit positif établi par les Gouvernements. Le droit naturel, c'est le droit de tout Homme de disposer et de se gouverner lui même afin de survivre, vivre. Les Hommes font usage de leur droit naturel pour passer entre eux un contrat social. \\
Le droit naturel a un caractère imprescriptible. En 1776, dans la déclaration d'indépendance des US: "Nous tenons pour évidente par elle même les vérités suivantes: tous les hommes sont créés égaux ; ils sont doués par le créateur de certains droits inaliénable ; parmi ces droits se trouvent la vie, la liberté, et la recherche du bonheur. Les Gouvernements sont établis parmi les Hommes pour garantir ces droits et leur pouvoir émane du consentement". Ce droit naturel qui est posé ici est un axiome de la conscience politique moderne. \\
1789, DDHC, l'esprit est le même. 


D'un point de vue politique, il existe deux paradigmes: le premier est celui de la théorie du droit divin, conduisant à une logique théocratique, où l'homme est un sujet corrompu, incapable de se gouverner lui même, donc naturellement destiné à être gouverné par un autre. \\
Le deuxième est celui du contrat social qui nous dit que les hommes sont des sujets voués naturellement à se gouverner eux même, qui ont la libre disposition d'eux même. Sur ce fondement, va être amené librement à passer entre eux un contrat social. 

\subsubsection{La méthode}

Trois éléments: le premier, c'est l'État de nature. Le deuxième, c'est le contrat social. Enfin, c'est l'État civil, social, ou politique. \\
Cette méthode a aussi une dimension polémique. Aristote soutenait que l'Homme est un animal politique car selon lui, l'Homme est le seul à posséder le "logos", la parole. La nature ne fait rien en vain, au hasard ou selon une nécessité aveugle. Aristote a une conception téléologique. Pour Aristote, si l'Homme a le don de parole, c'est qu'il est destiné à vivre dans la cité: le lien politique a donc un caractère politique. Pour lui, celui qui vit en autarcie est soit un animal, soit un Dieu. \\
Pour Hobbes, le lien politique est artificiel, à contrario d'Aristote. Pour lui, le pouvoir politique est fabriqué par les hommes, c'est un artifice. Il défend donc une conception artificialiste du pouvoir politique. \\
L'état de nature est une fiction, et on peut lui donner essentiellement une définition négative: la situation dans laquelle ne règne aucune autorité politique. C'est une situation que l'on peut qualifier d'infra-politique. Le passage à l'état politique ou à l'état civil va présupposer un acte volontaire des hommes vivant à l'état de nature. \\
On accorde deux dimensions au contrat social. D'abord, il prend la forme d'un pacte d'association. À travers ce pacte, les hommes consentent à s'unir en un peuple, qui a une dimension donc politique. Nous passons d'une multitude à un peuple doué d'une unité et d'une cohésion. Les Hommes abandonnent leurs libertés naturelles en échange d'une liberté artificielle et politique. Cette dernière est à la fois conditionné et limité par des lois communes qui rendent possible un ordre public. Les philosophes ne posent pas la question des faits (quid factis ?) mais la question du droit (quid juris?), ils posent donc la question de la légitimité de l'autorité politique. 


Pour Hobbes, la fin de l'État est la sécurité. Pour Locke, la sécurité n'est pas une fin, mais un moyen, et la fin est la liberté. Pour Rousseau, il va essayer de réfléchir aux conditions de l'autonomie politique. 


\subsection{Hobbes: le Léviathan ou la théorie de la souveraineté absolue}

\subsubsection{L'état de nature: "jus in omnia", droit sur toutes choses}

À l'état de nature, il n'existe aucune norme de justice, aucune norme morale. Le droit de chaque individu s'étend aussi loin que sa force: ce que l'individu peut faire, a la capacité de faire, il a le droit de le faire. \\
Les individus sont des atomes condamnés à se heurter en permanence: "L'Homme est un loup pour l'Homme" ; "Les Hommes sont naturellement ennemis" ; "L'état de nature est un état de misère". À l'état de nature, les hommes ont une liberté entière mais stérile car elle est en proie à la liberté égale de toutes les autres. L'état de nature est pour Hobbes est un état de guerre, qui ne correspondent pas à des batailles ou à des combats effectifs ici. L'état de guerre c'est: "un espace de temps où la volonté de s'affronter en bataille est suffisamment avéré". Il règne dans l'état de guerre une insécurité constante. \\
Quelque part, l'état de nature est invivable. 

\subsubsection{L'institution du pouvoir politique souverain}

Selon Hobbes, il y a une double motivation. La motivation est à la fois passionnelle et rationnelle. \\
Passionnelle car la peur et la crainte de la mort violente règne dans l'état de nature. Cette crainte est universelle dans l'état de nature. \\
Rationnelle car l'Homme, en tant qu'être de raison, est capable de se représenter un certain nombre de préceptes rationnels que Hobbes appellent des lois de nature. "La Loi naturelle est ce que nous dicte la droite raison touchant les choses que nous avons à faire ou à omettre pour la conservation de notre vie". La Loi naturelle interdit donc à un Homme ce qui détruit sa propre vie. L'être humain doué de raison doit donc tendre à assurer paix et sécurité. C'est la Loi de nature qui va le conduire à renoncer à notre droit de nature, donc notre droit sur toutes choses. \\
C'est sous l'influence de la raison que chacun va abandonner sa souveraineté absolue à un autre. "L'Homme est plus libre dans la cité où il obéit aux lois que dans la nature où il obéit à lui même". 

\subsubsection{La nature, les termes et la fin du contrat social}

"La convention de chacun avec chacun passée de telle sorte que c'est comme si chacun disait à chacun: j'autorise cet homme ou cette assemblée et je lui abandonne mon droit de me gouverner moi même, à cette condition que tu abandonne ton droit et que tu autorise toutes ces actions de la même manière". Le contrat n'est pas passé entre les citoyens et le souverain mais juste entre les citoyens. Il y a une réciprocité. Cela met le souverain à part. \\
L'État léviathan, c'est une sorte de personne morale qui incarne l'unité de la société. La souveraineté est absolu pour Hobbes ou elle n'est pas. Hobbes s'oppose au libéralisme pour qui le pouvoir doit toujours être limité. Limiter la souveraineté serait l'affaiblir pour Hobbes et donc le ramener à l'état de nature. \\
"L'abandon d'un droit absolu ne peut être lui même qu'absolu" Hobbes. Pour lui, la loi ne peut pas être injuste, car, dans l'état social, seul la loi peut nous permettre de distinguer le juste de l'injuste. 


La justice n'est pas l'obéissance mécanique à la Loi. C'est une vertu qui consiste à parfois nous amener à remettre en question telle ou telle loi (Alain). \\
Dans un contrat, il devrait y avoir réciprocité. Avec Hobbes, on échange notre liberté naturelle contre la sécurité. Cependant, on a à faire à un marché de dupes car on va payer la sécurité avec ce qui n'a pas de prix (la liberté). 

\subsection{Locke et la théorie libérale}

"Traité sur le Gouvernement civil" et "Les lettres sur la tolérance" et "Traité sur la tolérance", sont des ouvrages majeurs de Locke sur la théorie libérale du contrat social. 

\subsubsection{Une conception originale de l'état de nature}

Locke va se démarquer de Hobbes pour dire que l'état de nature n'est pas mortel comme l'imaginait Hobbes, ce n'est pas un État de guerre. Ce qui le caractérise, c'est plutôt une forme d'instabilité, d'incertitude. Selon Locke, l'état de nature est déjà un embryon de vie sociale. Les hommes sont déjà des êtres de raison et il y a déjà une sorte de droit naturel qui s'impose à tous. Il y a déjà des normes de justice qui s'imposent aux hommes. \\
Par exemple, la raison enseigne à chacun que tous étant égaux et indépendant, personne ne doit nuire à l'autre en ce qui concerne sa vie, sa liberté ou ses biens. Il existe déjà ce qu'il appelle une propriété. Il a une définition large de la propriété: ce qui appartient en propre à tel ou tel sujet. La vie, la liberté et les biens sont une propriété. \\
Pour Locke, ce qui permet la propriété, c'est le travail. À l'état de nature, il existe des droits pré-politiques. On trouve à l'état de nature, des ordres de justice autonomes par rapport aux souverains. 

\subsubsection{L'institution d'une autorité civile}

Le pouvoir civil ne va pas créer des droits mais il va se contenter de garantir et de protéger les droits qu'il existait déjà à l'état de nature. L'autorité civile va donner une force exécutoire au droit naturel. \\
L'état social nécessite des lois communes qui sont établis et connus. Mais aussi un arbitre qui applique les lois de manière impartiale. Et enfin, un pouvoir capable de donner force exécutoire à la décision du juge. 

\subsubsection{Le droit de résistance}

Ce que Locke va légitimer, c'est le droit de résistance. Pour Locke, le Gouvernement civil ne dispose que d'un "trust". C'est à dire que la souveraineté n'est pas illimité, le pouvoir n'est pas inconditionnel, les citoyens sont tenus d'obéir que dans la mesure où le Gouvernement civil garantit leurs libertés. \\
L'entrée dans une association politique n'est pas irréversible. Pour le citoyen, il est possible de dénouer les liens politiques et de retrouver la liberté de ce qui était l'état de nature. \\
Le peuple dispose naturellement d'un droit de résistance à l'oppression et si le souverain ne se conforme pas, il devient arbitraire, oppressif, violent. \\
La légalité vient d'en haut dans cette théorie, par le pouvoir souverain. En revanche, la légitimité vient d'en bas, c'est un pouvoir auquel le peuple consent. 


Idéalisme juridique: Locke soutient que les droits positifs peuvent être jugés, évalués, en se référant à un droit naturel. 

\subsection{J.J Rousseau: le pacte démocratique}

\subsubsection{Un paradoxe initial}

"L'Homme est né libre et partout il est dans les fers", Rousseau. Pour lui, la liberté est inscrite dans la nature même de l'Homme. Le paradoxe, c'est que l'expérience nous montre que le plus souvent, les Hommes ont abdiqués leur liberté. Rousseau essaye de rendre raison de l'aliénation de l'homme que l'on peut constater. \\
"Renoncer à sa liberté, c'est renoncer à sa qualité d'Homme" Rousseau. Rousseau va chercher à comprendre le paradoxe. \\
Il va poser la question au niveau des faits (quid factis ?) mais aussi au niveau de ce qui devrait être (quid juris?), la question de la légitimité donc. \\
"Discours sur l'origine et les fondements de l'inégalité parmi les Hommes", dans cette oeuvre, il s'intéresse à la formation du pouvoir. D'un point de vue historique, le pouvoir réel établi une forme de sujétion, d'assujettissement des individus. Le pouvoir réel se fonde sur ce que Rousseau appelle un pacte Léonin (une sorte de pacte qui est déséquilibré, asymétrique car est entièrement à la faveur de l'un et à la défaveur de l'autre). \\
Dans "Le contrat social", Rousseau va examiner le vrai contrat, un contrat qui permettrai d'articuler obéissance et liberté mais aussi obéissance et égalité. 

\subsubsection{La description du pacte Léonin}

Comme tous les philosophes du contrat, Rousseau part d'une description originale de l'état de nature. Il va se démarquer à la fois de Hobbes puis de Locke. Rousseau et Hobbes n'ont pas la même conception de la nature humaine, Hobbes soutient un pessimisme anthropologique. Rousseau va opposer à ce pessimisme un optimisme anthropologique. \\
Pour Rousseau, l'Homme est naturellement innocent, et c'est la société qui va corrompre de l'extérieur la nature humaine. Pour Rousseau, Hobbes a projeté sur l'Homme naturel des propriétés qui relèvent de l'Homme social. \\
Selon Rousseau, l'homme est gouverné par deux sentiments: l'amour de soi et la pitié. La pitié serait une répugnance naturelle à voir souffrir son semblable. Pour Rousseau, ce sentiment est propre à tout être vivant. Concernant l'amour de soi, pour Rousseau, il doit être opposé à l'amour propre. Le premier est un sentiment immédiat alors que  le second est un sentiment social. L'amour de soi est équivalent à une sorte d'instinct de conservation. En revanche, l'amour propre est un sentiment social qui implique la raison, la réflexion, c'est à dire la comparaison avec les autres. Il y a dans l'amour propre un projet totalement fou, totalement absurde, car dedans, chacun désire qu'autrui l'estime plus qu'il ne s'estime lui même.\\
Pour Rousseau, l'état de nature est un état relativement paisible. Les Hommes y sont égaux dans la mesure où si il existe des inégalités, celles-ci ne sont pas sensible. Dans l'état de nature, il n'y aucune tendance à la sociabilité. Ce sont des circonstances extérieurs qui vont amener les Hommes à se rapprocher et à s'associer pour former une société. \\
Premier sens de  commerce: échange. 


Pour Locke, le droit de propriété était un droit naturel et était justifié par le travail. Rousseau va, lui, en produire une analyse critique. La propriété privé est une sorte de péché originel au niveau social et politique. C'est l'institution de la propriété privé qui va faire perdre à l'homme son innocence originel. C'est au travers de la propriété que l'inégalité et la violence vont s'insinuer dans notre société. \\
"Le premier qui ayant enclos un terrain s'avisa de dire, ceci est à moi, et trouva des gens assez simple pour le croire fut le vrai fondateur de la société civile" c'est ici la genèse de la propriété privée où l'on s'arroge le droit d'en priver les autres, avec une dimension d'exclusion. \\
"Que de crimes, que de guerres, que de meurtres, que de misère et d'horreur n'eut point épargner au genre humain celui qui, arrachant les pieux ou comblant le fossé, eut crier à ses semblables: gardez vous d'écouter cet imposteur, vous êtes perdus si vous oubliez que les fruits sont à tous et les arbres à personne". \\
Dans les prémisses de l'État social, les plus forts vont s'approprier les richesses, vont les accaparer, les confisquer. Cette appropriation reste précaire et fragile, c'est pourquoi les plus forts vont vouloir transformer un état de fait en Droit. Ce sont donc les plus forts qui ont l'initiative de la parole politique. Leurs buts, c'est d'institutionnaliser, c'est à dire protéger par la loi la propriété, c'est à dire institutionnaliser la propriété. \\
"Les lois donnèrent de nouvelles entraves aux faibles et de nouvelles forces aux riches". C'est pourquoi on parle de Pacte Léonin.


NB: Il faut mettre ce texte en relation avec un texte de Marx: La question Juive où Marx fait une critique véhémente du droit de propriété. Marx va désacraliser, démystifier les droits de l'Homme. Marx est ce qu'on appelle un "philosophe du soupçon" d'après un de ses commentateurs car recherche toujours le latent des choses. \\
Marx cherche d'ailleurs le contenu latent des droits de l'Homme. Ces droits ont manifestement une prétention universelle et un caractère naturel. Ce que Marx démontre, c'est qu'au delà de ce contenu manifeste, il y a un contenu latent, qui est que les droits de l'Homme sont les droits de l'homme égoïste et bourgeois, et représente donc l'idéologie d'une classe déterminée. Il dit que les droits de l'Homme posent une égalité de droit pour masquer une inégalité de fait. 

\subsubsection{La construction du contrat social}

À quel condition un pouvoir peut-il être légitime ? C'est la question à laquelle va essayer de répondre Rousseau. \\
Le problème est d'associer obéissance et liberté. "Trouvez une forme d'association qui défende et protège de toute la force commune la personne et les biens de chaque associé, et par laquelle, chacun s'unissant à tous, n'obéisse pourtant qu'à lui même et reste aussi libre qu'auparavant", Rousseau. Il essaie de penser un contrat social qui garantisse aux citoyens une forme d'autonomie. \\
L'individu à l'état de nature possède une indépendance. Le citoyen, à l'état social, civil, doit conquérir une autonomie politique. \\
Selon Rousseau, il faut une aliénation absolue, il faut que chaque associé transfère à la communauté l'ensemble de ses droits "l'aliénation totale de chaque associé avec tout ses droits à toute la communauté". On a un paradoxe d'une souveraineté absolu établi au nom de la Liberté. \\
À la source de l'autorité et des lois, on trouve pour Rousseau le concept de la volonté générale. Chacun s'oblige à obéir inconditionnellement à la volonté générale. Chacun doit être à la fois sujet et citoyen. Être sujet, c'est être membre de l'État et, en tant que tel, être tenu d'obéir ; mais en tant que citoyen, je suis membre du souverain. \\
"L'obéissance à la Loi qu'on s'est soi même prescrite est liberté" Rousseau, c'est l'autonomie. La Loi est légitime dans la mesure où elle est l'expression de la volonté générale. Cette volonté est générale à deux niveaux, dans sa source, dans son origine (ce n'est pas la volonté d'un seul ni de quelques uns) ; dans son objet, dans son but, car la Loi vise l'intérêt général. \\
Rousseau parlera de l'enceinte sacrée des Lois. Celui qui obéit aux lois obéit au fond à personne. \\
La volonté générale a quatre caractéristiques: inaliénable, indivisible, infaillible, absolue. Inaliénable car la volonté générale ne peut pas se déléguer, si elle le fait, le peuple renonce à sa souveraineté. Indivisible car la volonté générale n'est pas la somme des intérêts particuliers. Infaillible car elle ne peut pas se tromper, elle est toujours droite et tend toujours à l'utilité publique.  

\chapter{L'exercice du pouvoir}

Pour Platon il faut unir savoir et pouvoir, c'est la théorie du philosophe-roi. \\
Pour Machiavel, l'exercice du pouvoir est fondamentalement amoral et il répond avant tout à un impératif pragmatique. \\
Relation entre exercice du pouvoir et exigence de transparence ? Distinction entre exigence et tyrannie de la transparence ?

\section{Autorité, sagesse et science}

\subsection{La théorie du philosophe-roi (Platon)}

Pour Platon, le meilleur des gouvernements se confond avec le gouvernement des meilleurs. On trouve chez Platon une critique de la Démocratie. Il développe donc une philosophie aristocratique. 

\subsubsection{L'opposition entre philosophie et rhétorique}

Au moment où vivent Platon et Socrate, Athènes est une démocratie. Dans une démocratie, la parole est reine. À l'intérieur de cette société démocratique vont naître deux conceptions antinomiques de la parole. Il y a d'abord la parole rhétorique et il y a la parole philosophique.


Rhétorique peut être associé à Gorgias. La rhétorique est l'art de parler avec éloquence, de manière persuasive. Gorgias disait "Le discours est un grand tyran", le discours est omnipotent, il a tous les pouvoirs. "Le discours agit sur l'âme comme la drogue agit sur le corps" Gorgias. Le discours a un pouvoir exorbitant, démesuré, disproportionné car au fond, les discours sont du vent. \\
La rhétorique consiste à parler devant les autres, en s'adressant à leurs passions de manière à être persuasif. Pour un orateur comme Gorgias, la valeur d'un discours ne se mesure pas à sa vérité mais aux effets et à l'impact qui se fait sur l'auditoire ; la valeur première n'est donc pas le vrai mais le vraisemblable. 


Philosophique peut être associé à Socrate, qui appelle l'art du dialogue la dialectique, en vue de découvrir la vérité. Pour Socrate, la valeur d'un discours ne se mesure pas à ses effets mais à sa vérité. L'objectif ultime n'est pas de persuadé en s'adressant aux émotions mais de convaincre en s'adressant à la raison. 

\subsubsection{Justification de la théorie du philosophe-roi}

Pour Platon, il s'agit d'unir autorité et sagesse, pouvoir et savoir, domination et science. "La cité sera juste le jour où les philosophes seront Roi ou les Rois, philosophe" Platon. Il va établir un lien nécessaire entre justice et Ordre. La justice consiste à se conformer à un ordre donné alors que l'injustice se confond avec la subversion de l'ordre. \\
Pour un philosophe de l'antiquité, l'Ordre est d'abord observable dans la nature. Cela s'appelle le kosmos (harmonie, beauté en grec). Cet Ordre, l'Homme va devoir s'en inspiré pour le reproduire à l'intérieur de la cité. \\
Platon envisage deux choses: il y a d'abord un ordre interne à l'être humain. C'est à dire qu'il est constitué de trois parties qui sont distinctes: la raison, le courage, et enfin les désirs. L'idée est qu'un individu, pourra être jugé vertueux si et seulement si la raison domine le courage qui domine lui même les désirs. Il y a injustice quand cet ordre est subverti. Un ordre analogue doit être réalisé à l'intérieur de la cité, les philosophes incarnant la raison, doivent être placés au sommet de la hiérarchie sociale. En dessous, on a les gardiens ou les soldats. Enfin, on a les artisans, qui créent les richesses matérielles. \\
Pour Platon, la politique est une science et c'est pourquoi il dit que le meilleur des gouvernements et le gouvernement des meilleurs. Il dit donc que la démocratie, c'est le règne de l'amateurisme, le règne de l'opinion. 


Platon, quand il pense à l'ordre de la cité, il appose une forme de communisme aux gardiens. Platon a déjà cette idée que cette classe intermédiaire possède tout en commun, c'est à dire les biens et les personnes comme les enfants et les femmes. Cela pour éviter les querelles et que les gardiens soient dévoués corps et âmes à l'intérêt général. Cela permet d'éviter le népotisme. \\
Platon imagine aussi une forme d'eugénisme (favoriser le meilleur gêne), les unions sexuelles ne doivent donc pas être laissées au hasard: il faut se faire reproduire entre eux les sujets les plus doués. 


La théorie politique de Platon est profondément anti-démocratique. Pour lui, la politique est une science. Par conséquent, la politique est réservée par nature à une élite. \\
À contrario, ce qu'on appelle la démocratie, c'est le règne de l'amateurisme, c'est le règne des opinions (doxa). Pour Platon, l'opinion est le plus bas degré de la connaissance, les plus incertaines et les plus fragiles. \\
Il va établir une hiérarchie entre les différents régimes. Au sommet, on trouve l'aristocratie, le gouvernement des meilleurs. On a ensuite la timocratie, le gouvernement de ceux qui cherchent les honneurs. On a ensuite l'oligarchie, le gouvernement de quelques uns, des riches. Après vient la démocratie, qui est le règne du désir, du caprice, d'une liberté excluant toutes limites (la licence), la démesure (obris), qui est l'opposé de la sagesse, le tout conduisant au chaos. Le pire des régimes qui succèdent tout naturellement est la tyrannie, à la licence extrême, va succéder une extrême servitude, où c'est le démagogue qui conduit le peuple.

\subsection{Les critiques de la philosophie politique de Platon}

\subsubsection{Une conception erronée du pouvoir}

Kant, dans son projet de paix perpétuel, dit "Que les philosophes deviennent Roi, cela n'est pas souhaitable", parce que "détenir le pouvoir corrompt inévitablement le jugement libre de la raison". Le pouvoir a intrinsèquement une puissance corruptrice. Kant dira que "le pouvoir corrompt, le pouvoir absolu corrompt absolument". \\
Pour Kant, il ne faut pas faire confiance au pouvoir et à ceux qui le détiennent pour s'auto-limiter. \\
G. Orwell, dans "La ferme des animaux", écrit toute une métaphore du système soviétique et décrit la révolte des animaux contre le fermier accusé de les exploiter. Leur projet va être de faire table rase de la société ancienne, pour créer une société neuve et vraiment égalitaire. Les cochons qui deviendront l'élite vont pactiser avec l'ennemi des animaux pour obtenir plus de pouvoir. La formule finale "Tous les animaux sont égaux, mais certains le sont plus que d'autres". 

\subsubsection{Une conception erronée de la science politique}

On trouve cette critique dans le traité politique de Spinoza. Il dit "Les philosophes ne doivent pas exercer le pouvoir, parce que leur défaut, c'est l'abstraction". Ces philosophes ne décrivent pas la nature humaine tel qu'elle est mais ils décrivent la nature humaine tel qu'ils l'imaginent. Ils construisent donc une sorte de chimère et la doctrine politique qu'ils en déduisent est toujours inapplicable alors que la politique est une science appliquée liée à l'expérience et pratique. \\
Spinoza conclut: "Personne ne serait moins qualifié pour être à la tête d'une communauté publique que les théoriciens ou philosophes".

\subsubsection{Une conception liberticide de l'ordre social}

La première critique va venir de l'élève même de Platon, Aristote. Il va critiquer le communisme de Platon. Aristote distinguait ce qui relevait de l'oïkos (foyer, maison, sphère privée), de la polis (la cité, la sphère publique). Aristote dit donc qu'il n'est pas possible de tout mettre en commun. Aristote se méfie d'une volonté d'unité ou d'unification qui irait jusqu'à nié les différences et la multiplicité. \\
Il développe trois arguments contre Platon. D'abord un argument théorique et philosophique, puis un argument plus pratique, et enfin, un argument psychologique. 


"Le processus d'unification se poursuivant avec trop de rigueur, il n'y aura plus de cité: car la cité est par nature une pluralité et son unification étant trop poussée, la cité deviendra famille et de famille, individu". Cela est le premier argument d'Aristote, dans le livre II de La Politique. \\
En pratique, on a très peu de soin de ce qui est collectif, ce qui est mis en commun. "Chacun se soucie au plus haut point de ce qui lui appartient en propre, mais quand il s'agit de ce qui appartient à tout le monde, on s'y intéresse bien moins". Platon soutenait que les enfants communs seraient regardés avec un commun amour, Aristote lui répond que ces enfants communs seraient regardés avec une égal indifférence. \\
Selon Locke, l'Homme aime posséder. Ce que remarque d'ailleurs Aristote, c'est que posséder en propre certaines choses, c'est la condition de la générosité. 


Une deuxième critique vient des philosophes libéraux. K. Popper, en 1942, écrit "La société ouverte et ses ennemis". Dans son oeuvre, c'est une opposition entre la société ouverte et la société fermée. Pour Popper, Platon est un partisan de la société fermée. On trouve chez Platon l'idée d'un pouvoir qui englobe toute la société et qui encadre de manière totalitaire l'ensemble de la vie individuelle. \\
La société close est une société holiste, c'est à dire que le tout a plus d'importance que les individus. La société ouverte, "c'est la plus grande révolution que la société humaine est connue" selon Popper. C'est l'avènement d'une société non plus holiste mais individualiste. Individualisme renvoyant ici à une émancipation de l'individu qui acquiert une liberté, une dignité en lui même. \\
Cette émancipation a d'ailleurs lieu à Athènes où le Pyrée, un port, permet à Athènes de pratiquer le commerce. Ce commerce va rendre possible les échanges et va mettre en relation avec des groupes qui ont d'autres moeurs et va permettre l'expérience de la relativité des institutions. Cela va conduire à une désacralisation du pouvoir et des institutions que l'on a. \\
Popper parle ensuite de l'émergence d'une "grande génération" en la personne de Périclès et Socrate. Périclès est l'inventeur de l'isonomie (l'égalité de tous les individus devant la Loi). Socrate, lui, soutiendra que l'individu a une valeur en soi, et va mettre au premier plan les capacité rationnelles et critiques de l'individu. \\
Platon apparaît donc comme un ennemi de la société ouverte et apparaît comme nostalgique de la société close. 


NB: Il existe un texte intéressant, écrit par B. Constant, "De la liberté des Anciens comparée à celle des modernes". La liberté des anciens, c'est ce que Constant appelle une liberté-participation, consistant à exercer collectivement et directement plusieurs parties de la souveraineté. En échange de cette participation politique, les citoyens étaient assujettis à une discipline implacable. \\
En revanche, la liberté des modernes est une liberté-indépendance, une liberté essentiellement individuelle. Cette liberté individuelle prend la forme de l'existence d'une sphère privée qui échappe à la puissance publique. Il voit dans le libéralisme deux choses bien distinctes, l'État et la société civile. Le premier dispose d'un pouvoir de coercition. En revanche, la société civile est une sphère à l'intérieur de laquelle les individus recherchent l'accomplissement de leurs buts particuliers, tout en respectant la loi. La liberté-indépendance est donc la sécurité de ces individus poursuivant leurs buts. 

\section{L'idéal technocratique et ses dérives}

\subsection{L'administration scientifique et rationnelle de notre société moderne et complexe}

\subsubsection{Qu'est-ce que la technocratie ?}

La technocratie, c'est au sens premier du terme, la prise en main de l'organisation de la société par des techniciens et plus précisément par des experts, ceux qui sont censés posséder des compétences. Ces experts prétendent avoir la compétence pour gérer la vie collective, la vie de la cité. Sur la base de leur savoir, ils sont supposés promouvoir l'intérêt général, ceci, sur une base scientifique. \\
Un philosophe du XIXe siècle, Auguste Comte, qui est le fondateur d'une doctrine appelé le "positivisme". Au XIXe, cette doctrine accorde une valeur suprême à la science. Il a créé la loi des trois états, le premier étant l'état théologique. Le deuxième est l'état métaphysique qui consiste à se repérer avec des réalités abstraites. Enfin, on a l'état positif ou scientifique ; c'est à dire que l'esprit s'en tient à ce qui est observable ; et de cette observation, il en déduit des lois scientifiques. \\
Ce dernier état implique un reflux de la religion et de la tradition.


Comte dit que grâce à la science, on va pouvoir se libérer de la foie et de la tradition. Saint-Simon "Il faut remplacer le gouvernement des Hommes par l'administration des choses". En 1819-1820, Saint-Simon publie l'organisateur. C'est un ouvrage qui remet en cause radicalement l'organisation de la société fondée sur la différenciation gouvernant/gouverné. Il fait une hypothèse: "Si la France conserve tous ses hommes de génies (dans le domaine des science, des beaux arts, des arts et métiers), et perde le même jour Monsieur le frère du Roi, Monseigneur le Duc d'Angoulême, et tous les grands officiers de la couronne auxquels on ajoute tous les ministres d'États ; cette perte de 30 000 individus réputés les plus importants de l'État ne leur causerait de chagrin que sous un rapport purement sentimental car il n'en résulterait aucun mal politique." C'est une métaphore entre frelons et abeilles. \\
"Que font les gouvernants si ce n'est se poser en élite et détourner le travail des gouvernés". Saint-Simon remarque que l'acte de gouverner consiste à donner aux frelons la plus forte portion de miel prélevé sur les abeilles. Il est temps d'après Saint-Simon de faire coïncider pouvoir économique et pouvoir politique. 

\subsection{Les dérives de la technocratie}

Référence: F. Kafka, Le château ; Le procès. 

\subsubsection{La technocratie: "un État dans l'État"}

La technocratie, en apparence, semble rationnelle. Cependant, accorder le pouvoir à une élite savante et experte, n'est pas sans conséquences et sans dangers. Le pouvoir politique réel risque d'être accaparé par une élite séparée et éloignée de la société civile et chercherai un intérêt particulier en le montrant comme un intérêt général. Les technocrates useraient d'alibi pour justifier sa main mise sur le pouvoir politique. \\
Le danger, c'est que le pouvoir des experts conduisent à un déficit démocratique, au paradoxe d'une démocratie sans le peuple. Le peuple serait dépossédé au profit des experts de tout pouvoir réel.


Dès le début du XXe siècle, des philosophes et sociologues ont constatés que nos démocraties modernes se caractérisaient par certains traits oligarchiques. \\
La montée des "populismes" montrent que nos sociétés sont traversés par des tendances oligarchiques. \\
La classe politique se professionnalise et se sépare de la société civile. Elle s'érige en élite, qui va cherché à protégé ses privilèges et à se reproduire. Cette élite va cultiver l'entre-soi. Paréto, un sociologue, développe la loi d'airain de l'oligarchie. L'airain était un métal dur et incorruptible, une loi d'airain est donc une loi à laquelle on ne peut pas échapper. Cette loi est la tendance naturelle dans toute organisation à produire une élite oligarchique qui se donne ses buts propres, sa vie propre et qui a ses intérêts propres.


La sociologie contemporaine tend à valider cette analyse, notamment avec P. Bourdieu. Il écrit en 1989 un ouvrage "Noblesse d'État: grandes écoles et esprit de corps". \\
On a à faire d'après Bourdieu à une aristocratie d'adolescence. C'est à dire qu'à une aristocratie liée à la naissance comme dans l'ancien régime, a succédé la sélection d'une élite entre 20 et 25 ans par le biais de l'ENA. Cette grande école va développer une sorte d'esprit de corps. \\
L'ENA va recruter des experts, des techniciens, qui vont venir constituer la haute fonction publique. L'ENS va elle, sélectionner une sorte d'élite culturelle. 

\subsubsection{La technocratie: monstre et labyrinthe}

Quand la technocratie a un fonctionnement labyrinthique et monstrueux, on va l'appeler la bureaucratie. La bureaucratie se présente comme un mode d'organisation rationnelle de la société mais cette bureaucratie va croître, elle va s'autonomiser et devenir en elle même un pouvoir. Kafka (1883-1924) met en scène cette logique bureaucratique qui aliène les individus dans "Le château". Nietzsche disait "L'État est le plus froid des monstres froids". \\
Kafka montre toutes les confusions, les lenteurs et les malentendus qui règnent dans l'administration bureaucratique. Une métaphore montre la bureaucratie comme un monstre tentaculaire et une autre comme un labyrinthe. 

\section{L'approche technique de l'exercice du pouvoir}

"Le prince" de Machiavel, c'est dans ce livre qu'il développe l'approche technique du pouvoir. C'est lui qui dit "La fin justifie les moyens". \\
Pour lui, la pratique du pouvoir a trois exigences fondamentales, la première est le pragmatisme, qui est l'exigence d'efficacité. La deuxième exigence est le réalisme politique: il faut faire de la politique en envisageant les Hommes tel qu'ils sont et non en les envisageant tels qu'ils devraient l'être. La troisième exigence est celle de l'amoralité du jeu politique, c'est à dire que la sphère du jeu politique est contraire à toutes considérations morales ; c'est ainsi qu'on parle des Hommes politiques comme des bêtes politiques. \\
Le texte de Machiavel est révolutionnaire car il donne à la politique devenu un art, une autonomie par rapport à la religion et à la morale. On a souvent dit que Machiavel était celui qui avait fait descendre la politique du ciel sur la Terre. 

\subsection{Une anthropologie pessimiste}

La nature humaine, l'essence de l'Homme, est mauvaise selon Machiavel. Il a un pessimisme radical. Dans Le Prince, chapitre 17, il dit que les Hommes "sont ingrats, changeants, dissimulés, ennemis du danger et avides de gagner". Si on est réaliste, il faut admettre que tous les Hommes sont des tyrans en puissance, ils sont fondamentalement égoïste. \\
Machiavel dit qu'il ne faut pas aller contre la nature humaine, on ne peut pas la réformer. Il dit  donc que pour asseoir son pouvoir, il faut diviser pour régner: dresser l'égoïsme des uns contre l'égoïsme des autres. Le but de Machiavel n'est pas de viser un but ou un idéal mais d'éviter le mal absolu, qui pour lui, est la guerre absolue ; c'est l'idée du moindre mal. 


Ce qu'on appelle couramment les vices, apparaissent politiquement comme des maux nécessaires. C'est là tout le sens de "la fin justifie les moyens". "Car, à bien examiner les choses, on trouve que, comme il y a certaines qualités qui semblent être des vertus et qui forment la maxime du prince, de même il en est d'autres qui paraissent être des vices et dont peuvent résulter néanmoins sa conservation et son bien être". \\
La politique c'est donc l'art de dissimuler, mais aussi la capacité à faire preuve d'une certaine duplicité (faire une chose et son inverse), voire une forme d'hypocrisie. \\
"Le prince doit-il s'employer à se faire aimer ou à se faire craindre ?" Si il peut faire les deux, tant mieux. Selon Machiavel, ce serait le second: les Hommes appréhendent moins d'offenser celui qui se fait aimer que celui qui se fait craindre. L'intérêt dissout immédiatement le lien constitué par l'amour ou l'estime. En revanche, la crainte, elle, est plus durable et crée des liens plus solides, plus stables. "Les hommes aiment ou estiment à leur gré", à contrario, il dépend du prince d'être craint: "les Hommes craignent au gré du prince". \\
Machiavel parle de "Virtu" qu'on peut traduire par force ou vertu mais qui n'est pas la vertu au sens moral ou traditionnel du terme. Cette force, c'est une aptitude, une faculté active d'utiliser la fortune, le sort à ses propres fins. Cela implique d'avoir la capacité à saisir le moment opportun. 


On peut soutenir que Machiavel est l'inventeur de ce qu'on appelle la "Raison d'État". C'est un mode d'intervention politique qui suspend le droit commun et la morale commune au nom d'une nécessité supérieure. 


\subsection{La stratégie du "Bouc émissaire"}

Émissaire vient d'un mot latin (emittere) qui signifie "envoyer dehors". L'expression "bouc émissaire" au premier sens du terme vient de la religion hébraïque. On trouve des traces de l'expression dans l'ancien testament. \\
Dans la religion juive, le jour de la fête des expiations, il était de coutume de choisir un bouc que l'officier religieux chargeait de tous les pêchés d'Israël, puis on le chassait dans le désert. On est dans une logique sacrificielle. \\
Ensuite, cette expression va prendre un sens figuré en étant appliqué au champ social et politique. Du point du vue moral, cette logique sacrificielle est immorale. D'un point de vue politique, cette pratique peut se révéler utile, efficace, voire indispensable.


Au sens figuré, le bouc émissaire va être un groupe ou une personne chargé de toutes les fautes. La collectivité et ce qu'il incarne se délivre de la culpabilité en la projetant et en la concentrant sur la victime. L'individu ou le groupe visé sont instrumentalisé de sorte à ce qu'ils assurent la cohésion et l'unanimité. \\
La collectivité concentre le mal sur une entité, la sacrifie et expulse par là même sa violence interne. C'est manière pour la collectivité de se réconcilier avec elle même. Le bouc émissaire a une fonction cathartique. La violence désordonné des individus qui pourraient menacer l'ordre social va être canalisé et orienté vers un ennemi commun. \\
De son côté, Machiavel va donner une illustration de ce mécanisme. Il va faire référence au Duc Cesar Borgia. Il occupe la Romagne, mais cette province est gouvernée par une multitude de seigneurs qui se font la guerre en permanence et qui dépouillent les sujets. Borgia va avoir un but: ramener la paix. Il va nommer un ministre qui est un homme particulièrement cruel et expéditif et va lui donner tous les pouvoirs. Ce ministre va ramener la tranquillité, cependant il va concentrer sur sa personne le ressentiment du peuple. Borgia, parvenu à ses fins, va exécuter son propre ministre. Cela va purger les esprits et Borgia va apparaître comme un libérateur.


L'ordre politique est sous-tendu par la violence. La violence est inhérente à la sphère politique. Par conséquent, la politique n'est pas la réalisation d'un idéal moral. De manière plus réaliste, la politique, c'est l'art du moindre mal. \\
La politique implique donc que l'on s'affranchisse des exigences ou des injonctions de la moral. 


L'idée majeure de Machiavel c'est que l'espace politique n'est pas soutenu par des principes qui s'imposent à lui de l'extérieur, comme la religion ou la morale. Au contraire, l'espace politique est structuré à partir de principes internes ou immanents. \\
Machiavel rabaisse la politique du ciel vers la terre. \\
On assiste à une sécularisation du politique. Cela signifie qu'il y a un reflux du religieux. Le pouvoir politique obtient une certains indépendance vis à vis de la religion. \\
Machiavel va faire une histoire politique de la religion. Ce qui est nouveau car avant, c'était l'inverse, avec la religion qui subordonnait le politique. La religion est un instrument parmi d'autres au service du politique. Machiavel voit en Moïse un chef politique, voire un chef de guerre. Pour unifier son peuple, il va utiliser les armes de la foie. 


\section{La Transparence: nécessité démocratique ou impasse}

Quand on applique le concept de transparence à la société, on utilise une métaphore, c'est une image. La transparence est un concept qui trouve son origine dans le domaine de la minéralogie. La transparence, c'est un phénomène par lequel les rayons lumineux visibles sont aperçus au travers de certaines substances. \\
Par extension, ce terme va être associé à l'ordre socio-politique. C'est alors l'exigence de ne rien cacher, de ne rien laisser dans l'ombre. Les citoyens exigent donc d'être informés et corrélativement, de mettre fin à l'arbitraire et à la corruption. Souvent, les pouvoirs ont étés opaques dans leurs origines, dans leurs modes de fonctionnement et dans leurs décisions. L'opaque est une porte ouverte à l'arbitraire. \\
L'exigence de transparence semble donc intrinsèquement liée à celle d'un approfondissement de la démocratie. \\
Néanmoins, cette exigence est plus complexe et plus ambivalente qu'elle n'y paraît. Insensiblement, la transparence devient une injonction absolue. Cette injonction tend à coloniser l'ensemble des strates de la société. Nous passons d'une exigence légitime de la transparence à la tyrannie de la transparence: c'est l'idée que rien ne doit demeurer cacher. \\
Au fond, une société totalement transparente conduirait sur une pente totalitaire. La vie individuelle et la vie sociale voire la vie politique, ont besoin de zones d'ombres, d'opacité, pour se développer harmonieusement. "Ce qui fait de l'État un enfer, c'est que l'Homme a essayé d'en faire un paradis" Orline. 

\subsection{L'opacité apparaît consubstantiel à l'exercice du pouvoir politique}

L'autorité est fondée sur deux principes distincts. Il faut d'abord qu'il y ait un clivage gouvernant/gouverné. Il faut ensuite que le pouvoir possède une certaine opacité, une certaine distance. Il doit paraître comme quelque chose de mystérieux, d'insaisissable. \\
Cette distance va prendre trois formes: il doit y avoir une ségrégation spatiale du pouvoir ; une ségrégation fonctionnelle ; et enfin une ségrégation ontologique. 

\subsubsection{Ségrégation spatiale}

En 1908, Zimmel écrit "La clandestinité a toujours fait partie des accessoires du pouvoir aristocratique. Elle exploite d'abord un fait psychologique, l'inconnu semble toujours effrayant, puissant, menaçant et elle tente de dissimuler la faiblesse numérique de la classe dirigeante". Il fait référence aux empereurs chinois qui régnaient depuis la Cité Interdite. Ces attributs participent de la pérennité du pouvoir. \\
On a à faire à un artifice qui rend possible la construction d'une image. 

\subsubsection{Ségrégation fonctionnelle}

L'Homme politique ne doit pas et ne peut pas obéir aux mêmes règles de conduite que l'Homme ordinaire. Par exemple, la raison d'État justifie le recours à des procédés que la morale ordinaire condamne ou réprouve. \\
Au fond, l'opacité, l'hypocrisie, sont des outils indispensables. "Qui ne sait pas dissimuler ne sait pas régner", Machiavel. 

\subsubsection{Ségrégation ontologique}

C'est une référence à Pascal dans "Les pensées", quand il parle de l'image du pouvoir qui fait le pouvoir avec les magistrats. \\
L'idée générale, c'est qu'une société ne peut pas survivre au régime d'une pleine et entière transparence. 

\subsection{Les lumières ou l'avènement d'une exigence de publicité}

Au XVIIIe siècle, en se fondant sur la notion de progrès, les philosophes vont entreprendre de lutter contre l'obscurantisme. Les lumières s'opposent à l'obscurité, à l'opacité. Leur projet global, c'était de dissoudre les zones d'opacité, de s'opposer à tout ce qui fait obstacle à la visibilité des choses. Les lumières prétendent établir une forme de transparence en s'appuyant sur la raison, l'esprit critique et le libre examen. Ils s'opposent au règne des superstitions, qu'elles soient religieuses ou politique. \\
Kant écrit "Qu'est-ce que les lumières ?", où il écrit sa propre définition des lumières. Pour lui, cela désigne pour l'humanité la sortie de la minorité et par voie de conséquences, l'accès à la majorité. La minorité s'oppose à  l'autonomie ou si l'on veut, à la faculté de l'autodétermination. \\
Kant utilise la formule qui est la devise des lumières "Aie le courage de te servir de ton propre entendement". \\
"Je ne pourrai jamais me faire à la manière de parler de ceux qui disent que tel individu ou tel peuple n'est pas encore mûr pour la liberté", Kant. Kant dit que le raisonnement où la maturité est nécessaire pour avoir la liberté est un sophisme, car selon lui la liberté n'arriverait jamais. Il dit "On ne peut mûrir pour la liberté que dans la liberté elle même". 


La philosophie du contrat social, ce sont des armes de guerre tournés contre la théorie du droit divin. La souveraineté politique est le produit d'un contrat et ce contrat est un acte bilatéral et le pouvoir politique se fonde de manière explicite sur le consentement de ceux qui contracte. \\
D'un point de vue historique, la Révolution va se saisir de ce principe. Au fond, les révolutionnaires vont établir une équation entre le secret et la monarchie. La démocratie implique le triomphe de la transparence. Par exemple, la souveraineté du peuple implique la publicité des débats. Voir l'article 15 de la DDHC. 

\subsection{De la transparence au risque de la tyrannie de la transparence}

Benddam, en 1780, écrit Panopticon (pan, tout en grec, et opticon, voir). C'est un système qui est pleinement et entièrement transparent. C'est une utopie carcérale. Il imagine la prison idéale qui est bâtie sous une forme d'amphithéâtre, au centre duquel il y a une tour. La subtilité, c'est que le gardien situé dans la tour centrale voit sans être vu lui même. Cela fait que les prisonniers ne savent pas quand ils sont vus et ni si ils sont vus. La surveillance n'a pas besoin d'être effective pour être efficace. \\
Foucault dira "l'effet du panoptique est d'induire chez le détenu un état conscient et permanent de visibilité qui assure le fonctionnement automatique du pouvoir. La surveillance est permanente dans ses effets même si elle est discontinue dans son action". \\
Des trois fonctions du cachot, on ne retient que la première, et on supprime les deux autres. La première est d'enfermer, la deuxième est de priver de lumière, et la dernière est de cacher. Dans une telle situation, les deux dernières fonctions sont supprimées. \\
On a à faire à une sorte de prison sans murs, à une hyper-surveillance. Tout cela ne correspond plus à un schéma qui sera pyramidal. \\
Nous finissons par occulter une distinction fondamentale de nos sociétés libérales: la distinction entre l'État d'un côté et la société de l'autre. 


\chapter{Qu'est-ce qu'une société juste ?}

\section{Introduction}

\subsection{Justice}

C'est une notion complexe, qui peut être défini au moins à trois niveaux. La Justice renvoie à une valeur, un idéal et a donc plutôt ici une signification morale. Alain disait de la Justice: "La Justice est ce doute sur le droit qui sauve le droit", au nom de la Justice, un sujet peut donc désobéir à la Loi.


La deuxième signification de la justice renvoie à une vertu individuelle ou sociale. Vertu au sens de aptitude. Elle consiste à rendre simplement à chacun ce qu'il lui est du. \\
Un exemple est celui de la sagesse et la justice légendaire du Roi Salomon. 


La Justice peut également désigner la faculté qu'a une société, à travers ses institutions, à rendre à chacun ce qui lui revient légitimement. Cela implique donc de donner l'égalité de tous devant la Loi. \\
Dans cette logique, on peut dire qu'il faut donner plus à ceux qui ont moins, c'est la discrimination positive, ce que les américains appelleraient "affirmative actions". \\
La stricte égalité de tous devant la Loi est injuste car elle est incapable de résorber les inégalités. Par soucis d'équité, il faut compenser les inégalités en donnant plus à ceux qui ont moins. 


Enfin, la justice est une institution, ce qui lui donne une signification judiciaire. 

\subsection{Une société juste ?}

La question de la société juste va varier dans le temps et dans l'histoire. Par exemple, pour Platon, la question est la suivante: à qui est-il juste que le pouvoir revienne ? \\
Pour nous, la question de la justice sociale se formule autrement: comment est-il juste que l'État se rapporte à la société ? C'est donc la question des relations de l'État avec la société. La question fondamentale est celle de l'intervention de l'État, de la nature de l'intervention, du champ d'intervention, et sa légitimité. \\
Historiquement, on aura un État libéral, qui se veut minimal dans ses interventions puis ensuite un État social, un État providence. 

\section{La bipolarité de notre vie politique}

\subsection{Les deux paradigmes fondamentaux}

Ce sont des idéaux-types que l'on va présenter, dans les faits, ils peuvent être mélanger. 

\subsubsection{Libéralisme}

Son acte de naissance, c'est au XVIIIe siècle avec les Lumières écossaises, et notamment de Adam Smith, David Hume, et Adam Ferguson. En France, on a Montesquieu, Voltaire. \\
Dans la perspective libérale, pour qu'une société puisse être jugée juste, il est nécessaire et suffisant que deux conditions soient remplis: le premier est que le pouvoir reviennent à une entité légitime. Entre le libéralisme et la souveraineté populaire, il existe un lien qui est intrinsèque. Les libéraux ont toujours combattus pour l'avènement de la démocratie. Le libéralisme va faire l'apologie du système représentatif. \\
C'est l'idée selon laquelle les Hommes naissent et demeurent libres et égaux en droits, par conséquent, tous les citoyens doivent également "concourir à la formation de la Loi, personnellement ou par l'intermédiaire des représentants". L'idée est donc que toute autorité légitime doit provenir du peuple. \\
Un État, même démocratique, n'est pas nécessairement juste en lui même. Tocqueville évoquera la possibilité d'un despotisme démocratique. La solution serait de poser des contre-pouvoirs. Chez les libéraux, il y a cette intuition que si on ne limite pas le pouvoir, ce n'est pas le pouvoir lui même qui va se limiter. Le but est de garantir l'autonomie de la société civile et des individus qui la compose ainsi que de défendre les droits et libertés individuelles. \\
Hayek, un économiste néo-libéral, dit "Ce n'est pas la source mais la limitation du pouvoir qui l'empêche d'être arbitraire". \\
Il naît clairement une séparation entre l'État, qui possède la puissance coercitive et la société civile où doit être garanti une certaine liberté. La démocratie n'est donc pas à priori immunisé de tout risque d'arbitraire. \\
Dans le libéralisme, les inégalités induites ne peuvent pas être jugées comme injuste et illégitime. 

\subsubsection{Socialisme}

On peut situer l'acte de naissance du socialisme à Karl Marx, donc au XIXe siècle. Il y a une opposition fondamentale entre les droits formels, et de l'autre, les droits réels ou matériels. \\
Accorder à tous, des droits-libertés comme le font les libéraux, n'est pas suffisant. À partir d'un certain seuil d'inégalités réelles, matérielles, économiques, concrètes, l'égalité civile et politique n'est plus qu'un leurre, une mystification. Cette égalité fait abstraction des inégalités économiques et concrète qui séparent l'individu. \\
De ce point de vue, Marx cherche l'égalité matérielle. Dans "Les douze thèses sur Feuerbach", Marx dit "les philosophes n'ont fait juste à présent qu'interpréter le monde de différentes manières", il va donc chercher à le changer. L'idée est que l'exigence d'égalité doit investir les sphères sociales et économiques. Lacordaire disait "La liberté c'est le renard libre dans le poulailler libre". \\
Les niveaux d'inégalité obéissent à des effets de seuil. C'est à dire qu'à partir d'un certain niveau d'inégalité, nous ne sommes plus tous sur le même bateau, ce qui menace la cohésion sociale. Pour Marx, il faut intervenir dans le champ économique lui même, pour réduire les inégalités, il en va de la cohésion sociale. Il faut donc éviter la paupérisation croissante de la société. \\


\subsection{Quels droits pour les individus ?}

Dans la perspective libérale, les droits-libertés jouent un rôle fondamental et le libéralisme se fonde sur une approche individualiste. Cela veut dire qu'il n'y a pas de référence à un groupe, et demande une abstention de la part de l'État. Cela va être la liberté d'opinion, la liberté religieuse, etc. \\
Dans la perspective socialiste, on évoque plutôt des droits-créances. Ceux-ci ont une dimension non pas individuelle mais sociale, ce sont des droits collectifs. Par exemple, dans le préambule de la Constitution du 27 Octobre 1946, "La Nation assure à la famille les conditions nécessaires à son développement". On note encore le droit de grève, le droit à l'instruction, etc. La conséquence est que ces droits créances impliquent cette fois-ci, non pas une abstention mais une action positive et effective de la part de la puissance publique. \\
Avec l'arrivée de l'État providence, on va arriver à une institutionnalisation de la solidarité. Les anglophones appelleront cela le "welfare state".


On passe à un nouveau paradigme. On dira que la redistribution étatique prend la place de la réciprocité qui est la règle sur le marché économique. \\
La solidarité ne relève plus d'actes de charité individuel et isolé, mais va au contraire s'objectiver, s'inscrire dans les institutions. Cette montée en puissance de l'État providence s'explique au moins par une situation de crise des solidarités primaires (famille, corporation, Église), car avec les révolutions industrielles, l'exode rural, ces phénomènes vont déstructurer ces solidarités primaires. 

\section{Les vertus créatrices des échanges encadrées par le droit}

\subsection{Les vertus sociales et politiques du marché}

\subsubsection{B. Mandville: "la fable des abeilles", XVIIIe siècle}

"Les vices privées font les vertus publiques", c'est le sous-titre de la "fable des abeilles". Ce sous titre a une apparence paradoxal. Ce récit a un caractère fabuleux, imaginaire, mais Mandville cherche bien à parler de notre société. \\
Le récit évoque "une vaste ruche bien fourni d'abeilles". Il parle d'une ruche prospère d'un point de vue économique, c'est à dire une ruche qui produit un grand nombre de richesse. Cependant, cette prospérité, loin d'être fondée sur les vertus des individus, est au contraire, fondée sur les vices. \\
"La vanité élève des palais, les vices créent du travail" ; "le luxe donnait du travail à un million de pauvres gens et l'odieux orgueil à un million d'autres". "Les vices stimulent l'esprit d'invention, le travail des producteurs, et induisent un progrès social". "Les pauvres eux même vivent mieux que les riches avant".  


Les abeilles ont voulus moralisés leur société et par fiction, ils ont demandés à Jupiter de réaliser leur souhait. La conséquence est qu'il n'y aura plus de démesure, mais ce sont au contraire la modération, la sobriété, la frugalité, l'humilité, qui vont dominer. L'économie va sombrer dans une économie de subsistance: statique, atone. \\
Les valeurs morales traditionnelles sont inadaptées au monde économique tel qu'il est réellement. Car si l'économie fait preuve de mesure, il n'y a pas de création de richesses possible. \\
L'intérêt personnel, l'intérêt égoïste, sont les moteurs essentiels de l'activité économique. L'intérêt égoïste dynamise l'économie, mais au delà, il produit des effets sociaux-politiques. Paradoxalement, l'intérêt individuel travaille non seulement pour lui même mais également pour la collectivité. Il y a un écart entre les motivations, les intentions d'un individu et ses effets. 


\subsubsection{La division du travail}

"La richesse des Nations" s'ouvrent sur une analyse de la division du travail. La division du travail avait déjà été abordée par Platon, mais la perspective était différente. \\
Pour Platon, la division du travail s'expliquait par l'indigence humaine, par l'incapacité de chaque individu à se suffire à lui même, l'absence d'autosuffisance. Ce que Platon montre, c'est qu'il est rationnel à chaque individu de se spécialiser dans une tâche en fonction de ses talents, de ses aptitudes, et de ses goûts. Le besoin de quelque chose devient alors le besoin de quelqu'un d'autre, il y a là une création d'interdépendance. \\
À contrario, Smith va donner un fondement plus positif à la division du travail. Quand il parle de la division du travail, il va évoquer les thèmes de l'abondance, de la productivité. C'est ce qui permettra de passer à une simple économie de subsistance à une économie capable de produire un surcroît de richesse et d'accumuler ces richesses. Dans son exemple de la manufacture d'épingle, Smith démontre qu'un ouvrier produira en moyenne 20 épingles par jour ; en revanche, si les ouvriers partagent les opérations nécessaire à la production d'une épingle, ils peuvent en produire plus de 40 000 par jour. \\
Trois raisons au surcroît de productivité: l'augmentation de la dextérité de l'ouvrier qui fait toujours la même tâche ; le gain de temps à ne pas passer d'une tâche à une autre ; le développement de l'emploi des machines que la division du travail rend possible. \\
"La division du travail est ce qui, dans une société bien gouvernée, donne lieu à cette opulence qui se répand jusque dans les dernières classes du peuple".


NB: Chaplin montre la déshumanisation et l'aliénation que Marx avait déjà mis en évidence. L'ouvrier fuit donc le travail comme la peste. 

\subsubsection{"La main invisible"}

Smith va opposer l'Homme à l'animal. Il va montrer que chez l'Homme et chez lui seul, la différenciation, la diversité, sont porteurs d'effets bénéfiques. Chez l'animal, les différences naturelles entre sous espèces restent sans effets positifs pour l'espèce. Il prend l'exemple du chien: "Le mâtin ne peut ajouter aux avantages de sa force en s'aidant de la légèreté du lévrier, de la sagacité de l'épagnol, ou de la docilité du chien de berger". En revanche, chez l'Homme, il existe un instinct de "trafic, de troc et d'échange". Cet instinct de trafic rapproche les Hommes, les rend indispensables les uns aux autres. \\
Smith se démarque d'Aristote qui établissait un lien entre la main et l'intelligence (la main étant un outil à fabriquer des outils). Smith dit que l'Homme est devenu ce qu'il est non pas parce qu'il a une main, mais parce qu'il en a deux, une pour donner et l'autre pour recevoir. 


Le terme commerce a vu sa signification considérablement évoluée, notamment au XVIIIe siècle. Il y a eu une variation sémantique qui va accompagner une évolution sociologique et idéologique. À l'origine, le terme commerce a une signification qui est négative, en latin: "neg-otium" (l'absence de loisir). Les affaires commerciales étaient extérieures à toutes les activités reconnues et valorisées. \\
Au XVIIIe siècle, le commerce va désigner de manière beaucoup plus large toute relation paisible et équilibré entre les Hommes. Il va donc désigner les échanges au sens large (tant de biens que d'idées). \\
Hirschman, en 1977, publie "Les passions et les intérêts", il décrit les révolution du XVIIIe siècle. Il nous dit que l'égoïsme intelligent, paradoxalement, nous affranchis de la logique du court terme et de l'intérêt immédiat. Le calcul d'intérêt, l'égoïsme intelligent, neutralise les passions les plus dangereuses. Auparavant, l'activité lucrative était condamné, d'un point de vue philosophique, moral, et religieux. Au XVIIIe, tout cela va être considéré comme vertueux. La liberté économique permet, non pas d'annihiler l'égoïsme des Hommes, mais de l'orienter dans une direction socialement favorable. \\
L'échange a une vertu sociale et politique car elle a une vertu pacificatrice: Montesquieu "là où on trouve la paix, on trouve le commerce". "Il est heureux pour les Hommes d'être dans une situation où pendant que leurs passions leur inspire la pensée d'être méchant, ils ont pourtant intérêt à ne pas l'être". Cette paix dû au commerce n'est pas lié à la morale ou à la philanthropie, elle est dû à un simple calcul d'intérêt. Cet intérêt rend l'individu prévisible et constant et stabilise ainsi l'ordre social. \\
Les marxistes diront que le commerce est la continuation de la guerre par d'autres moyens. 


L'expression de la "main invisible", apparaît qu'une seule fois dans "La richesse des Nations" et une deuxième fois dans un texte bien moins connu du grand public. Elle a cependant eu d'énormes répercussions. La main invisible, c'est une métaphore pour désigner la capacité d'autorégulation du marché, de la sphère économique. \\
Pour bien comprendre cette métaphore, il faut l'opposer à une main visible et organisatrice (l'État) qui intervient à l'intérieur de la sphère économique. \\
On s'aperçoit que Adam Smith va s'opposer à ceux qu'il appelle les "Hommes de système", aux "penseurs systématiques". Ceux là sont ceux qui croient aux vertus d'une main organisatrice. "L'Homme de système semble imaginer qu'il peut disposer les différents membres d'une grande société aussi aisément que sa main dispose des différentes pièces d'un échiquier. Mais il ne lui vient pas à l'esprit que les pièces sur l'échiquier ont d'autres principes de mouvement que celui que la main leur confère. Alors que dans le grand échiquier de la société humaine, chaque pièce singulière a son propre principe de mouvement, entièrement différent de ce que la législation pourrait choisir de lui imprimer". \\
C'est pourquoi la société doit être dirigé par une main invisible. "Ce n'est pas de la bienveillance du boucher, du marchand de bière ou du boulanger que nous attendons notre dîner mais bien plutôt de soin qu'ils apportent à leur intérêt". \\
Pour Smith, les prérogatives de l'État doivent être réduits à l'essentiel, c'est à dire l'État minimal/limité. Les fonctions régaliennes sont l'armée, la police et la justice. Selon Smith, il y a une quatrième fonction régalienne qui consiste à ériger et entretenir certains ouvrages publics et institutions. "Tout Homme, tant qu'il n'enfreint pas les lois de la justice, demeure en pleine liberté de suivre la route que lui montre son intérêt". "Défendre la société de tout acte de violence ou d'invasion de la part des autres sociétés indépendantes" ; "Protéger autant qu'il est possible chaque membre de la société contre l'injustice et l'oppression de tout autre membre ou bien de devoir établir une administration exacte de la justice". Concernant la quatrième fonction, c'est l'idée d'entretenir des biens que l'intérêt privé ne serait jamais à même d'entretenir, car les profits ne seraient pas assez intéressant à l'échelle du temps des particuliers (comme l'éducation par exemple). 


\subsection{L'échange est un jeu à somme positive}

\subsubsection{La critique du "dogme de Montaigne"}

Ce dogme a été critiqué au XXe siècle par Von Mises dans "L'action humaine", publiée en 1949. Il évoque et critique le "dogme de Montaigne". Un dogme est une croyance erronée. \\
Montaigne publie ses essais au XVIe siècle. Montaigne écrit le texte suivant: "le marchand ne fait bien ses affaires qu'à la débauche de la jeunesse ; le laboureur, à la chèreté des blés ; l'architecte, à la ruine des maisons". Chez Montaigne, l'échange est envisagé comme un jeu à somme nul, ce que l'un gagne, l'autre le perd. L'avantage de l'un a pour condition nécessaire le désavantage de l'autre. Les libéraux vont montrer eux que l'échange est à somme positive. \\
Il y a quelque chose de mystérieux et de paradoxal dans l'échange. Objectivement, la quantité de biens reste exactement la même avant et après l'échange. Mais les acteurs trouvent un avantage dans le seul fait d'avoir échangé. Dans l'échange, chacun cède ce qui, à ses yeux a moins de valeur, pour obtenir ce qui à ses yeux, en a plus. De ce point de vue, l'échange doit être distingué du vol ou du pillage. 

\subsubsection{La critique du mercantilisme}

On trouve cette critique chez Smith, dans "La richesse des Nations". Mercantilisme est un terme inventé par Smith lui même, quand il évoque ce qu'il appelle le "système mercantile". On classe derrière ce terme un certain nombre d'économistes qui adhèrent à des principes communs, par exemple, Montchrestien. \\
Pour les mercantilistes, le but du commerce, c'est de développer la puissance militaire du royaume en développant sa puissance économique. Cela signifie que les affaires des Nations se gèrent comme les affaires des marchands. Les mercantilistes vont développer une politique économique protectionniste et un nationalisme économique agressif. Les mercantilistes sont obsédés par la balance commerciale. \\
Les mercantiles ont l'idée que la puissance des Nations vient d'une large disposition de métaux précieux (de larges fonds donc aujourd'hui) afin d'entretenir une armée. Le commerce est assimilable à la guerre. 


Smith nous dit que le mercantilisme conduit à une forme de jalousie commerciale. "L'amour de notre propre Nation nous dispose souvent à regarder avec la jalousie et l'envie les plus maligne, la prospérité et l'accroissement de toute autre Nations voisines". Selon Smith, cette pensée est le produit d'une pensée superficielle, qui ignore la dynamique créatrice des échanges. \\
Pour les libéraux, dans la société internationale, le développement des forces productives d'une Nation, ne va pas nuire aux économies voisines. Les libéraux font l'apologie du libre échange et de l'ouverture, par opposition au protectionnisme et à la fermeture sur soi. Il existe deux types de géographie, d'abord une géographie politique et au contraire, une géographie économique. 


\section{Origine, sens et montée de l'État providence}

\subsection{Origine et signification}

S'oppose à l'État gendarme qui est l'État minimal. L'État providence cherche le bien-être de ses citoyens en octroyant des droits économiques et sociaux. Il a une vocation sociale. 


\subsection{Une réalité plurielle}


\subsection{Les trois mondes de l'État providence}

Andersen écrit "Les trois mondes de l'État providence: essai sur le capitalisme moderne". Il fait le constat que le capitalisme a pour effet de transformer l'Homme en marchandise (par exemple, le travail est un bien comme les autres dont la valeur est fixée par la loi de l'offre et la demande). Dans cette situation, l'État providence va se présenter comme une volonté de démarchandisation de l'Homme au sein des sociétés capitalistes. En particulier, il va falloir démarchandiser les revenus, pour permettre un capitalisme à visage humain, pour cela, il faut que les individus puissent avoir des revenus de substitution hors du marché du travail.  \\
Avec l'essor de l'État providence, le citoyen attend qu'une partie de plus en plus grande de son bien être et de ses ressources dérivent de prestations indépendantes de son travail. Ces prestations sociales sont accordés en raison de leur seule appartenance à la communauté nationale. 


Selon Andersen, il y a trois régimes de l'État providence.


Au plus haut, le régime le moins protecteur, le régime libéral: couverture résiduelle (ciblée sur les plus faibles), dualisme social renforcé ; les destinataires des aides sont les pauvres ; les responsables de l'aide sont ici les autorités publiques ; ce régime a un fonctionnement circonstanciel ; la redistribution est ici faible. 


On a ensuite le régime corporatiste (Bismarck): couverture professionnelle, assurance sociale obligatoire adossée au travail salarié ; les destinataires sont ici les travailleurs ; les responsables sont ici les caisses d'assurances, il y a une pluralité de régime d'assurances ; le fonctionnement est ici dans une logique de contribution et de rétribution ; la redistribution est ici moyenne, dans une logique horizontale.


On a enfin le modèle universaliste, l'État social (Beveridge). C'est dans ce modèle que les revenus sont le moins marchandisés. Ici, la couverture est universelle ; ici, les destinataires sont tous les citoyens ; le responsable est ici l'État ; la logique de fonctionnement est ici l'impôt ; la redistribution est ici forte, dans une logique verticale. 

\chapter{Le droit de punir}

\section{Introduction: droit de punir}

Le terme peine renvoie en premier lieu à la souffrance de nature morale ou qui est lié à un effort difficile, c'est son sens premier. \\
Par extension, la peine désigne également une souffrance infligée par une autorité instituée selon un droit établi. Cette souffrance est infligée à un individu qui a transgressé une norme, une règle, ou une loi. \\
Il faut mettre en évidence la dimension coercitive de la peine. C'est une violence publique à la différence de la violence privée. \\
Dans l'expression "droit de punir", le terme droit à un principe de légitimité. Le droit doit être distingué du fait. Le droit est ce qui devrait être, alors que le fait est ce qui est. La violence inhérente à la peine se présente comme légitime. \\
Il y a un paradoxe dans le "droit de punir". Le droit se présente d'une manière générale comme un désaveu de la violence. Cependant, le droit utilise cette même violence que par ailleurs il condamne. On peut donc se demander ce qui rend la violence étatique originale et légitime. 

\section{La pulsion vindicative comme forme archaïque de la Justice (et son nécessaire dépassement)}

\subsection{Pas de justice sans rétribution}

Eschyle a écrit une tragédie Grecque, appelée "l'Orestie", qui met en scène ce qu'on appelle la malédiction des Atrides. Les Atrides, ce sont les descendants d'Atrée. L'idée est qu'Atrée va commettre un crime qui va être contagieux et va affecter tous ses descendants. \\
Il y a au départ deux frères jumeaux: Atrée et Thyeste. Atrée, gagnant, va devenir Roi de Mycène. Son frère va entreprendre de lui voler le pouvoir et de séduire sa femme. Atrée va chercher à se réconcilier avec son frère, et se réconcilier autour d'un festin, composé en fait des enfants de Thyeste. \\
Atrée aura un fils qui s'appelle Agamemnon, Thyeste aura un fils qui s'appelle Egisthe. Agamemnon va accepter de sacrifier sa propre fille pour se rendre à Troie. La femme d'Agamemnon va s'unir à Egisthe et va tuer Agamemnon. Celle-ci a un fils qui va tuer sa propre mère pour se venger de la mort de son père. \\
C'est une illustration de la loi du Tallion présente dans l'ancien testament. C'est l'idée que la peine paye le crime comme l'argent paye la marchandise. On rétablit un équilibre que le crime a rompu. \\
Dans la vengeance, il y a l'idée que le passé n'est jamais mort car il pèse toujours sur le présent. 


\subsection{Les apories de la justice privée}

Se faire justice soi même renvoie à une violence privée et archaïque. Cette attitude plonge ses racines dans un instinct de pulsion animal, c'est un réflexe. Cette justice privée nous enferme dans une triple impasse: morale, logique et enfin sociale. \\
Impasse morale car pour compenser un crime, que l'on juge abominable, on lui oppose un autre crime. Il y a ici contradiction car en faisant la justice, on se rend coupable de l'injustice que l'on prétend combattre. La réparation n'est qu'apparente. \\
Impasse logique car la vengeance enferme dans un cercle vicieux, une sorte de fuite en avant, une escalade de la violence. À une justice vengeresse, on pourra toujours opposer une autre justice vengeresse. \\
Impasse sociale car cette justice privée ne prend pas la forme d'une Loi commune qui rendrai possible la concorde sociale. Au contraire, avec la vengeance, on a à faire à une puissance de dissolution de l'ordre social. \\
Il faut donc endiguer cette expansion du chaos. 

\subsection{Le juge doit se substituer au justicier}

La sanction doit se substituer à la vengeance. Punir n'est pas se venger. \\
Le justicier se venge, la vengeance est passionnelle, elle est souvent démesuré, disproportionnée. Le justicier est évidemment partial, sont but n'est pas de restaurer l'ordre ou l'équilibre mais au contraire, il risque de le dissoudre. Le pire est que la vengeance est sans fin. \\
Le juge sanctionne, elle est rationnelle, est censée être mesurée, elle est encadrée par une procédure de justice déterminée préalablement et la sanction a un terme, car celui qui a exécuté sa peine a payée sa dette. La justice légale institue des intermédiaires, des médiations, entre celui qui subit la peine et celui qui l'inflige. \\
P. Rickeer dans "Le juste" disait "C'est dans la structure du procès, tel qu'il devrait se dérouler dans un État de droit, qu'il faut chercher le principe de la coupure entre vengeance et justice". Il y a donc une opposition entre la Loi du Tallion et une justice instituée qui va peser le pour et le contre. \\
Dans "l'Orestie", l'oeuvre se conclut sur le procès d'Oreste, et Athéna va instituer un tribunal: "le tribunal de l'Aréopage". Selon la légende, ce tribunal était situé sur une colline où Ares a été jugé pour le meurtre du fils de Poséidon. 


La Justice est une institution qui se présente comme la clé de voûte de l'édifice sociale. 

\section{Les principes régulateurs du droit de punir}

\subsection{L'Ancien Régime et l'éclat des supplices}

M. Foucault: "Surveiller et punir". Le texte de Foucault s'ouvre sur le description du supplice de R.F Damiens. Ce personnage avait tenté d'assassiner le Roi Louis XV et ça sera la dernière personne écartelé légalement en France. Le régicide avait la main droite coupé, et du plomb fondu lui était versé dans les plaies. \\
"Le supplice permet qu'on retourne le crime sur le corps visible du criminel". "Il fait du corps du condamné le lieu d'application de la vindicte souveraine, le point d'ancrage pour une manifestation du pouvoir, l'occasion d'affirmer la dissymétrie des forces". Le corps du condamné devient la chose du Roi. Le Roi va y imprimé de manière manifeste et visible, sa marque. \\
Il faut que "la justice criminelle, au lieu de se venger, enfin punisse". Il va falloir instaurer des châtiments sans supplice avec l'idée que le châtiment doit se suffire à lui même. Le droit de punir va désormais prendre la forme d'une défense de la société. Beccaria aussi critique la torture au nom d'une philosophie utilitariste. \\
Ce qui est déterminant, c'est la représentation qui devient plus importante que l'inscription sur le corps du criminel. Ce qui est essentiel, ce n'est pas le corps mais l'âme qu'il faut amender. 

\subsection{Le droit de punir obéit à des principes régulateurs}

\subsubsection{Principe de légalité}

"Pas de crimes, pas de peines sans lois". Sa formulation, on va la trouver d'abord  au XVIIIe siècle chez Montesquieu, mais aussi chez Beccaria. Beccaria: "Seul les Lois peuvent fixer les peines qui correspondent aux délits, ce pouvoir ne pouvant être détenu que par le législateur qui réunit toute la société par un contrat social". C'est un rempart fondamental contre l'arbitraire que l'on voyait dans l'ancien régime. \\
Le principe de légalité rend possible une certaine sécurité juridique. Ce principe a aussi une fonction cognitive, cela veut dire que le principe de légalité met à notre disposition un certain nombre de choses qui permet d'apprendre quels sont les actes considérés comme portant atteinte à la société. 

\subsubsection{Le principe de nécessité}

L'article 8 de la DDHC indique que la loi ne doit établir que des peines évidemment et strictement nécessaires. Par rapport à l'ancien régime, les supplices et les tortures n'ont pas d'utilité sociale ou individuelle manifestes. Dans la mesure où ils ne sont pas utiles, ils sont illégitimes et injustes. \\
Ce principe a été développé par des philosophes utilitaristes, notamment Bentham. L'idée est que notre but doit être d'augmenter autant que possible la somme de plaisir et de bien être dans le monde et diminuer la quantité de souffrance et de douleurs. La torture apparaît donc comme superflue, injuste et illégitime. 

\subsubsection{La peine porte sur les actes et non sur les intentions}

Cela permet d'éviter l'arbitraire. Il faut bien distinguer le droit et la morale. La morale s'intéresse tant aux actions qu'aux intentions intérieures. La conscience permet d'être à soi même son propre bourreau (conscience, culpabilité, etc). \\
En revanche, le droit, lui, renvoie à une autorité extérieure et reconnu. Le droit juge les actes en fonction de normes instituées, communes, et objectives. \\
Hume, dans "Le traité de la nature humaine", évoque une situation "Si la loi l'exige, un juge peut priver un citoyen généreux, vertueux, de sa fortune, pour transférer celle-ci à un avare". Les anomalies locales sont cependant équilibrés par un bénéfice global: "le mal local est largement compensé par l'obéissance constante à la règle et par la paix et l'ordre qu'elle institue dans la société". 


Le droit de punir se distingue de la pulsion vindicative. Plus une société se civilise, plus le droit de punir est monopolisé par les institutions et par ceux qui les représentes. 

\section{Les fondements et les fonctions du droit de punir}

\subsection{L'expiation}


\subsection{La rétribution}


\subsection{L'utilité sociale}
















\end{document}
