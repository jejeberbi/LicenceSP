\documentclass[10pt, a4paper, openany]{book}

\usepackage[utf8x]{inputenc}
\usepackage[T1]{fontenc}
\usepackage[francais]{babel}
\usepackage{bookman}
\usepackage{fullpage}
\setlength{\parskip}{5px}
\date{}
\title{Cours de Sociologie de l'État et des politiques publiques (UFR Amiens)}
\pagestyle{plain}

% Rattrapage 13h-16h - 20 septembre

\begin{document}
\maketitle
\tableofcontents

\part{Sociologie de l'État}

\chapter{Introduction}

État nation ; État providence ; État producteur. Depuis peu, il y a un recul de l'État alors que dans les années 60, il y a une recherche de développement équitable des territoires (DATAR). \\
Pierre Rosanvallon crée le concept d'État régulateur qui est la forme actuelle de l'État. C'est un État minimisé par rapport à l'État providence ou producteur. Il est désormais un arbitre et non plus un acteur central comme c'était le cas auparavant. \\
L'État est un fait social au sens de Durkheim car l'État remplit les propriétés du fait social (extériorité, antériorité, coercition). De plus, l'État s'est forgé sans que l'on s'en rende compte. Il n'a pas été pensé à priori mais à posteriori. \\
L'État peut être pensé comme une abstraction philosophique mais aussi comme un ensemble d'acteurs producteur d'actions publiques. Il faut éviter l'anthropomorphisme vis à vis de l'État. On va chercher à savoir précisément qui décide, car l'État ne décide pas. \\
Dans ce cours, on va penser l'État comme un composé d'acteurs en interaction, qui peuvent être en conflit ou dans des alliances. L'État n'est pas un ensemble homogène, il subit des divisions. On étudiera l'État comme dans la démarche de Jean-Gustave Padioleau "L'État au concret". 


Le pouvoir est une notion importante. Pour certains, la science politique est la science du pouvoir. Une première conception du pouvoir est essentialiste ou naturalisante qui postule que le pouvoir fait partie de la nature du porteur de pouvoir. On ne peut pas se satisfaire de cette définition. \\
On oppose à la conception essentialiste la conception relationnelle du pouvoir qui considère, à la suite de Robert Dahl, que le pouvoir s'inscrit dans une relation entre au moins deux personnes: un détenteur du pouvoir et celui qui accepte de se soumettre. Le pouvoir est la capacité qu'un individu A fasse faire une action à B qu'il n'aurait pas fait sans A. \\
Le pouvoir politique est un pouvoir qui s'exerce sur un groupe social, un collectif. Ce pouvoir s'exerce dans un but collectif, celui de la cohésion du groupe. Pour perdurer, il est obligé de s'appuyer sur un discours d'accompagnement, des systèmes de valeurs: le pouvoir doit être reconnu par ceux qui s'y soumettent comme étant légitime (la légitimité est une affaire de croyance, on croit en la pertinence ou au bien fondé de quelque chose). Goffman dans "Stigmate" et "Mise en scène de la vie quotidienne", explique que l'État, la société, la démocratie, etc. ne sont pas des institutions figées. Comme la croyance est la source de la légitimité, les institutions se doivent de se renouveler, de changer ses discours. \\
La légitimation ne s'arrête pas aux idéaux-types de Max Weber. La science par exemple est un pourvoyeur de légitimité. La démocratie participative peut aussi être un puissant vecteur de légitimation des politiques publiques.


Il a existé des formes de pouvoirs politiques spécifiques, notamment dans des sociétés non étatiques. En effet, l'État n'a pas toujours existé. Il faut se méfier d'un double préjugé très répandu qui consisterait à croire que l'État est une nécessité naturelle ou croire que c'est une nécessité historique. C'est une forme d'ethnocentrisme qui place l'État comme au dessus des autres. \\
On distingue schématiquement quatre types de pouvoirs qui précèdent l'État, le premier est le pouvoir politique indifférencié. Dans celui-ci, le pouvoir politique se confond avec les autres pouvoirs. \\
Le second est le pouvoir politique personnalisé, incarné dans une personne particulière (chef de clan, chef de tribu), il tient sa position à la faveur des dieux ou de ses ancêtres. Ce type de pouvoir n'est pas propre aux sociétés sans États. \\
Le troisième est le pouvoir politique patrimonial. Il se caractérise par une certaine indifférenciation et une forte personnalisation dans le sens où le pouvoir est le patrimoine des ou du détenteur. C'est Max Weber qui met en exergue ce concept. Le pouvoir est une propriété. Dans ce type de régime, on ne distingue pas le budget public et le budget privé. Les empires, les sultanats sont des pouvoirs patrimoniaux. On distingue deux types de pouvoir politique patrimoniaux: il se manifeste par une forte centralisation dans certains cas et peut être, à contrario, très éclaté. Empire, Sultanat, sont des pouvoirs patrimoniaux centralisés. Le pouvoir patrimonial fragmenté, paradoxalement, est celui qui préexiste directement à la création de l'État, comme dans la société féodale. En occident, les seigneuries se sont multipliées avec l'éclatement des carolingiens. 


Dans l'État, le pouvoir politique est centralisé ; différencié ; spécialisé ; institutionnalisé. 


\chapter{La formation de l'État en occident}

Ce que nous appelons État correspond à une forme d'organisation politique qui s'est imposé en occident. Pour saisir le processus historique de formation de ces États, on peut essayer de constater les caractéristiques les plus repérables. \\
Bernard Lacroix dans "Le traité de Science Politique", 1985, volume 1, dit "le mot État est apparu et a été généralisé en Europe à la fin du XVe siècle". Cette apparition tardive montre que l'État a été créé récemment. \\
L'État exerce un pouvoir sur un territoire élargi. \\
L'État se distingue des détenteurs du pouvoir. \\
Il y a une nette séparation entre les règles de fonctionnement de l'État et de la société civile. \\
L'État suppose une laïcité minimale. \\
Les fonctions de Gouvernement sont remplis par des organes spécialisés, qui se distinguent des autres pouvoirs (notamment économiques et religieux). \\
L'État s'appuie sur une administration institutionnalisée. L'État est donc légal rationnel. 

\section{La centralisation du pouvoir politique et la crise de la féodalité}

C'est la crise de la féodalité qui va permettre un début de formation de l'État. \\
La crise féodale a deux dimensions, une dimension socio-économique et une dimension politique.


La dimension socio-économique renvoie à l'expansion économique en occident à partir de la fin du XIe siècle, qui va s'étendre jusqu'au début du XIIe. Cette expansion va avoir deux conséquences importantes: l'essor des villes et avec ceci, le développement d'une classe sociale: la bourgeoisie urbaine. Cet essor s'effectue en dehors du système féodal car les villes possèdent une relative autonomie. \\
Le développement du commerce entraînent donc la création de la bourgeoisie urbaine qui n'est pas contrôlée par la noblesse, ce qui en fait une entorse au système féodal. \\
Une partie de la paysannerie va s'intégrer aux circuits commerciaux en relation avec les villes et échappent donc en partie au contrôle de la noblesse. Cela provoque aussi un premier exode rural. \\
L'agriculture devient une matière commerciale. On assiste à une monétarisation de l'économie et donc à une perte de contrôle économique et sociale des paysans par les seigneurs. 


La dimension politique est le fait que le pouvoir féodal est affaibli. Cet affaiblissement est lié aux concurrences de plus en plus fortes entre les seigneurs dû à la croissance démographique. \\
Le morcellement de l'autorité en seigneurie entraîne des concurrences et donc des conflits entre seigneurs pour le contrôle d'un territoire. \\
Ce processus a été analysé par Norbert Elias dans "La dynamique de l'occident". Elias souligne le rôle prédominant et décisif joué par le Roi qui a cherché à étendre son autorité par la voie de la conquête militaire. Cette stratégie n'est pas propre au Roi. \\
Tout au long du XIIe siècle, des luttes incessantes entre seigneur vont se produire et déboucher sur une concentration progressive du pouvoir aux mains notamment des seigneurs les plus puissants. \\
Quatre phases conduisant au pouvoir politique centralisé: 
\begin{itemize}
\item Phase de concurrence libres et unification du domaine dynastique (XIIe siècle).
\item Lutte des maisons princières (XIIIe et XIVe siècle).
\item Phase des apanages (XIVe et XVe siècle): les affrontements vont être interne aux grandes familles. Ce phénomène va réduire la dynamique monopolistique, le Roi devant morceller son pouvoir. 
\item Phase de victoire du monopole Royal (XVe et XVIIe siècle). 
\end{itemize}


Ce que Norbert Elias appelle "Les chances de puissance sociale" qu'on définira comme les possibilités de domination déterminées par le nombre de dépendants que le monopoliste peut s'attacher, sont dans les mains du Roi. \\
Le Roi détient le monopole de la fiscalité et de la contrainte physique. L'État a un degré élevé de monopolisation. Ces deux monopoles entraînent une centralisation. \\
Ces deux monopoles vont générer la formation d'autres monopoles fondamentaux et à la base de la création de l'État. Grâce aux monopoles fiscale et militaire, vont se constituer d'autres monopoles et notamment le monopole juridique et également le monopole d'exercice de la justice. Pour ce dernier, le Roi va tout au long du XVIIe siècle, lutter contre les tribunaux ecclésiastiques. \\
Le monopole de la production juridique va disparaître de l'Église et va apparaître chez des légistes royaux. \\
Cette dépossession des pouvoirs de l'Église vers l'État est ce qui permet de dire que le processus est un minimum laïque. 


À partir du moment où le monopole s'agrandit, il va s'exiger la mise en place d'une administration aux fonctions multiple et une division poussée du travail. \\
La complexification rend la marge de manoeuvre plus faible. À partir d'un certain seuil, le monopole échappe à la main d'un seul individu. Le Roi a une dépendance fonctionnelle vis à vis de sa cour et de son administration. On dira que le monopole se socialise: il s'ouvre à la gestion et au contrôle de couches entières de la population: administration, gestionnaire. À partir du moment où le monopole se socialise, le monopole privé devient public. 

\section{Les facteurs explicatifs de la différenciation du pouvoir politique}

Ce sont surtout des facteurs culturels. Il y a le rôle de la religion ainsi que des doctrines philosophiques et juridiques.


L'État se construit sur la base d'une dissociation et d'une automatisation du politique par rapport au social. Le politique se désencastre du social. \\
Le christianisme a joué un rôle majeur voire accélérateur de différenciation du pouvoir politique. Cette différenciation est inscrite dans la théologie qui prône une séparation entre la spiritualité et le temporel. \\
Cette différenciation se comprend très bien dans la formule "Rendre à César ce qui est à César, rendre à Dieu ce qui est à Dieu". \\
En proclamant l'autonomie du pouvoir spirituel par rapport au pouvoir temporel, l'Église a dessiné en négatif les contours d'un domaine politique spécifique. \\
Comme le montre Max Weber, l'organisation interne de l'Église a contribué a posé le cadre du futur modèle étatique et lui a servi de modèle d'organisation. L'Église va influencer les fondements de la légitimité de l'État. On sait que l'Église va très loin dans une théorie de la souveraineté qui avait pour fonction d'asseoir la légitimité et l'autorité du pape. Cette théorie de la souveraineté va, en quelque sorte, être laïcisé et va servir à l'élaboration d'une théorie de la souveraineté du monarque, et donc, de l'État. C'est là qu'interviennent les légistes royaux qui vont favoriser la différenciation.  


La redécouverte du droit romain entre le XIIe et le XIIIe siècle participent à la différenciation. Ce droit est fondé sur un système juridique qui a la capacité à penser un système politique autonome et détenteur de sa propre légitimité. \\
Il y a une distinction entres institutions privés et institutions publiques. Le domaine politique est entendu dans le cadre du droit romain comme fonctionnant selon les exigences de l'intérêt public. L'État y est pensé comme producteur privilégié de droit, ce n'est plus la société qui impose ses normes et ses valeurs au pouvoir politique, mais au contraire, c'est l'État qui fixe les normes. Ce qui signifie le droit, pour le Roi, de ne pas être soumis à un pouvoir extérieur. \\


La doctrine du double corps du Roi est l'idée que la souveraineté d'un État n'est pas exercée par le Roi en tant que personne physique mais par le Roi en tant que fonction. C'est la fonction qui confère la souveraineté. C'est l'idée que développe Jean Bodin au XVIe siècle. Selon lui, la souveraineté est absolue, n'appartient à personne, mais la souveraineté appartient à l'État ; les droits régaliens seraient donc la propriété de l'État et donc non pas du Roi. \\
Il y là, dépatrimonialisation du pouvoir politique. \\
Le Roi a donc un corps physique et un corps symbolique détenteur de la souveraineté, d'où l'idée de double corps du Roi. \\
Dans cette idée, la souveraineté est perpétuelle. Quelque soit le monarque, la souveraineté existe toujours, d'où la formule "Le Roi est mort, vive la Roi!". 


"La supériorité des lois fondamentales du royaume" signifie que le pouvoir souverain ne peut être arbitraire, il ne peut pas utiliser le droit ou qu'il ne peut pas utiliser le pouvoir politique à des fins personnels. C'est l'idée de l'existence de lois supérieures immuables et inviolables, et autant que ses sujets, le Roi doit s'y soumettre, au risque de perdre sa légitimité. Le droit qu'il édicte doit se conformer à ce droit. \\
Ceci signe le début d'un État de droit ainsi que de la hiérarchie des normes. \\
C'est par cette mutation progressive en État de droit que la différenciation s'opère puisque le pouvoir politique détient ses propres règles de fonctionnement, respectant les lois fondamentales. 

\section{L'institutionnalisation et la spécialisation}

Ce long processus a nécessité la mise en place d'une administration. Weber a distingué cinq traits, dans "Économie et société", caractéristiques de l'appareil d'administration étatique. \\
La spécialisation des fonctions est le premier trait. Les tâches, au sein de l'administration, sont clairement subdivisés et correspondent à des compétences particulières, à des droits et des obligations spécifiques. L'administration publique étatique repose sur une division du travail. \\
La hiérarchisation des fonctions est le deuxième trait. Les différentes tâches sont appliqués dans le cadre d'une division hiérarchique du travail. Chaque poste est sous le contrôle et la subordination hiérarchique d'un poste qui lui est supérieur. Ce contrôle s'exerce selon des règles juridiques, des règlements abstraits et subjectifs ("modèle bureaucratique"), ces règles évitent l'arbitraire. \\ 
La réglementation des fonctions est le troisième trait. Des règles juridiques définissent les conduites à tenir dans le règlement des tâches administratives. \\
La séparation de la fonction d'administration et des moyens d'administration est le quatrième trait. Les postes n'appartiennent pas au fonctionnaire. Le poste est exercé, il n'est pas possédé (vénalité des offices). \\
La professionnalisation est le cinquième trait. L'exercice des fonctions spécialisées nécessite une formation, une compétence, un recrutement des fonctionnaires qui se fait donc sur la compétence et non sur l'hérédité. Les recrutements vont donc se basés sur des concours. C'est l'aptitude qui est décisive dans l'obtention des postes. \\
Weber considère tout cela à une rationalisation du pouvoir politique qui devient légal-rationnel. 


Une telle administration commence à se mettre en place à la fin du XIVe siècle, sous le Roi Philippe le Bel. Tout au long du XVe et du XVIe est créée un impôt unique, la taille. \\
Au cours du XVIe et du XVIIe s'opère une spécialisation et une institutionnalisation de l'État en France. Elle se repère à trois niveaux, d'abord au niveau central (spécialisation dans le conseil du roi), la mise en place de secrétaires d'État. Un gouvernement spécialisé se met en place auprès du Roi. \\
Au niveau local, des commissaires royaux sont mis en place en 1435. Au début du XVIIe siècle est créé l'intendant, c'est l'ancêtre du préfet. Il détient des pouvoirs important au niveau local. \\
Comme l'a étudié Thocqueville, la révolution française ne marque pas une rupture avec ce processus, mais au contraire, favorise la centralisation. En 1790, sont créés les départements. La centralisation administrative est perpétué par Napoléon par la création des préfets nommés par l'État et donc représentant de l'État. Le préfet contrôle le département. \\
Ce n'est qu'en 1982 que la décentralisation va apparaître, avec les lois Deferre. Ces lois vont affaiblir les corps préfectoraux au profit des collectivités territoriales. \\
L'administration se spécialise de façon croissante au début du XVIIIe siècle, et on voit apparaître les postes de commis, recrutés par concours, également des ingénieurs chargés de fonction technique, qui obtiennent des diplômes d'écoles qui se créent dans la lignée du processus (Ponts et chaussé, École des mines, Polytechnique). Ces ingénieurs se forment dans des grands corps techniques. \\
La compétence, le savoir, deviennent des sources de légitimité. Les grands corps techniques existent encore et sont une spécificité française. Ces grands corps deviennent des hauts fonctionnaires.


Le "pantouflage" renvoie à une pratique dans la haute fonction publique. La pantoufle est, historiquement, la somme que devait verser un grand commis de l'État lorsqu'il quittait le service de l'État. Aujourd'hui, cela désigne des hauts fonctionnaires qui font leur carrière en dehors de leur corps d'appartenance. Les hauts fonctionnaires vont dans le privé ou rentrent en politique: Pierre Birnbaum, "Les sommets de l'État". \\
Un membre de haut corps reste à vie titulaire de son poste et peut retourner à n'importe quel moment à son poste. Cela pose un problème de démocratie car l'administration est censé être au service du politique. Mais aussi, cela leur offre une grande sécurité politique, ils peuvent prendre bien plus de risques que les citoyens lambda. De plus, leur ancienneté continue de courir. 


Il existe un certains nombre de formes de carrières politiques. La majorité, historiquement, était la carrière ascendante, ou carrière par le bas. Cette forme est basée sur le militant, et donc aussi les partis. L'individu commence par être un sympathisant, puis devient adhérents, puis militant, et gravit petit à petit les échelons du parti. Cet individu va finir par se présenter à des élections locales pour aller vers les élections nationales. Il pourra finir nommé au Gouvernement. \\
Il existe une carrière par le milieu qui est lié au parti politique. Ces partis ont besoin de permanents et ceux-ci vont profiter de leur place pour briguer des mandats électifs. \\
La carrière dominante aujourd'hui est la carrière descendante (par le haut), où les individus qui sortent d'une grande école se font nominer dans un cabinet ministériel comme conseiller. L'individu finit nommé secrétaire d'État puis ministre etc. Il manque à ces individus des suffrages, donc ceux-ci vont investir le parti qui les a nommés. Ceux-ci vont ensuite briguer des mandats électoraux et vont se faire "parachuter" afin de se greffer une légitimité électorale et ainsi continuer leurs carrières politiques. 


Le pouvoir exécutif possède l'administration. Ce sont juridiquement deux entités distinctes, cependant, il existe une entité entre les deux, ce sont les cabinets. Ceux-ci sont très opaques, certains sont nommés avec publicité au JORF, d'autres, non. Les nominations se font selon une récompense des siens, des fidèles, la compétence, l'interconnaissance. 


\chapter{Les trajectoires multiples de l'État}

Il serait faux de croire que dans le cadre de l'Europe, les trajectoires soient les mêmes.


\section{L'État en Europe}

"Comparative research on cultures and nations", 1968, est une oeuvre de Stein Rokkan, qui utilise une méthode diachronique (comparaison dans le temps). Il constate des différences dans le mode de développement des États en Europe, notamment en terme de centralisation, de différenciation, d'institutionnalisation, spécialisation. Ils remontent dans le temps pour construire les variables succeptibles d'expliquer ces différences. \\
Ces variables se situent dans les temps modernes. Il distingue principalement trois variables, la variable économique, la variable territoriale, la variable culturelle. 


Pour la variable économique, elle explique les différences en terme de différenciation. Il prend en compte des facteurs qui lui permet d'opposer l'Ouest à l'Est de l'Europe. À l'ouest de l'Europe, on constate une construction étatique précoce, une économie marchande sur la base du commerce atlantique. C'est le cas en France, en Angleterre et en Espagne en particulier. Ce dynamisme économique procure les ressources financières nécessaires au développement de l'État. En effet, l'accroissement des richesses favorisent l'accroissement des ressources militaires et des ressources administratives. Tout cela est soutenu par la bourgeoisie urbaine qui a intérêt à avoir un État fort qui peut garantir la sécurité et l'ordre dans le cadre des circuits commerciaux. Cela n'est pas l'intérêt de l'aristocratie foncière. \\
La bourgeoisie urbaine constitue la classe montante, qui va conduire, par l'achat de charges administratives notamment, à l'anoblissement de certains de ces nobles. \\
À l'est, l'économie demeure très agraire, c'est le cas de la Russie, de la Pologne, de la Prusse. Rokkan parle en ce sens d'ordre impérial agrarien. Le poids des seigneurs féodaux demeurent très importants, ce qui freine la différenciation de l'État. De plus, le pouvoir politique est accaparé par l'aristocratie foncière et conserve donc des traits patrimoniaux. C'est ce qui explique des rythmes de constructions étatiques différents à l'est et à l'ouest. \\
Il distingue aussi le centre de l'Europe, qu'il appelle l'épine dorsale, qui va du monde germanique au monde Italien. Dans ce centre de l'Europe, le modèle de construction étatique est tardif car il a été freiné par la résistance d'un système de cité-État. 


Pour la variable territoriale, elle permet de comprendre les différences en terme de centralisation entre les différents pays européens. Le centre de l'Europe, qui va exactement des Pays-Bas à l'Italie du Nord, en passant par la vallée du Rhin, est au centre des échanges économiques depuis le Moyen-âge. Les routes commerciales sont nombreuses, et les échanges économiques se développent dans de nombreux centres urbains. La concurrence entre ces villes, empêche la formation d'un centre politique. En effet, le nombre de villes pouvant prétendre au poste de capitale va freiner la constitution d'un centre politique. C'est pourquoi la centralisation y est plus tardive. XIXe siècle pour l'Allemagne et l'Italie. \\
Si on prend le cas de Londre et Paris, on voit qu'ils acquièrent vite le statut de centre s'imposant sur un vaste territoire, ce qui accélère la formation d'un centre politique. 


Pour la variable culturelle, principalement la variable religieuse, joue surtout sur un axe Nord-Sud. Le protestantisme se développe dans le nord de l'Europe alors que le catholicisme se développe plutôt au Sud. \\
La religion joue un rôle dans la formation des identités nationales mais aussi dans la différenciation du spirituel et du temporel. \\
En Angleterre, l'Église Anglicane, est le fruit d'un conflit entre la hiérarchie catholique, le Roi est déclaré chef de l'Église d'Angleterre. Progressivement, les ordres réguliers sont supprimés, le mariage des prêtres est autorisé et le culte se dirige progressivement vers du Calvinisme, puis du Luthérien. \\
Le rôle de la papauté sera différent selon les situations. Le pape va freiner la construction de l'État en Allemagne, dans sa lutte contre l'Empereur et le développera en France dans le cadre de sa lutte contre les protestants.


La France et l'Angleterre sont deux États qui se distinguent par leur degré de centralisation et de différenciation. À la suite de deux auteurs, qui ont écrit "Sociologie de l'État", ils considèrent la France comme étant un État fort, et l'Angleterre comme un État faible, au sens de peu de différenciation, de spécialisation, de centralisation et d'institutionnalisation. Ces deux pays sont ceux qui prennent le plus tôt la forme d'État, mais ceux ci se distinguent des traits caractéristiques de l'État.


Une différence dans le degré de centralisation s'observe dans des différences du système féodal. C'est en France que le système féodal a été le plus poussé, mais paradoxalement, c'est en France que la centralisation est le plus poussé. La principale explication réside dans le fait qu'en France, la lutte entre le Roi et les seigneurs a été beaucoup plus forte. Cela a requis une forte monopolisation du pouvoir par la Monarchie et a donnée lieu en France à une répression pour donner le pouvoir au domaine royal. En Grande-Bretagne, cette lutte a été beaucoup moins intense, pour au moins deux raisons, un moindre morcellement du territoire et donc du pouvoir politique car l'unité territoriale est réalisé entre le IXe et le Xe siècle car c'est une île. ; la seconde raison est la préservation par le Roi des pouvoirs seigneuriaux et des coutumes. Le sujets du Roi acceptent donc de se soumettre au Roi et ce, dès le XVIe siècle, en dehors de tout lien vassalique. À noter que de manière très tôt, la noblesse peut se faire entendre du Roi via une représentation parlementaire: magna carta, 1215. Cette charte énonce l'indépendance de la justice, la propriété foncière individuelle. Dans le cas britannique, l'aristocratie cherche plus à contrôler le Roi qu'à détruire les institutions royales.


En Grande Bretagne, par rapport à la France, on constate une moindre différenciation de l'État par rapport à la société. Une des explications principales vient du droit, notamment public. En Grande-Bretagne, le corps politique ne peut se différencier car le droit administratif n'existe pas contrairement à la France où s'élabore un droit public, distinct du droit privé, un droit propre à l'État. Ceci s'explique car la Grande-Bretagne a été beaucoup moins influencée par Rome mais reste aussi attachée à la common-law. Ce droit n'accorde aucune prérogative de puissance publique à l'État, celui-ci est donc soumis au droit commun. \\
On constate aussi qu'en France, surtout depuis la Révolution française, l'État se donne un nouveau rôle, une nouvelle mission, qui est de produire du social. Société à solidarité organique: les individus n'ont plus de liens direct qui les unissent, c'est l'État qui réalise la liaison. \\
En Grande-Bretagne, il n'y a pas de construction d'identité nationale. Il y a plusieurs raisons à cela. La première est qu'on a très tôt une forte unité territoriale, ensuite, une très forte unité linguistique, puis le rôle de l'Église Anglicane qui construit un sentiment et une appartenance commune.


Concernant la spécialisation et l'institutionnalisation, la distinction s'observe tant au niveau du monopole administratif, du monopole militaire et de l'interventionnisme économique. \\
Du point de vue du monopole administratif, alors que se développe en France comme en Espagne la monarchie absolue, il y a, en Angleterre, très peu d'agents professionnels. Ce sont des amateurs, des fidèles, qui entourent le Roi et le conseille et non des fonctionnaires rétribués. Il n'y a pas de fonction publique en Grande-Bretagne avant le XIXe siècle. Les gouvernements britanniques font appels à des commissions ad hoc, spécialisés mais temporaire, pour traiter de manière ponctuelle des problèmes et partir une fois le problème réglé. Cela évite la constitution d'une élite administrative. On trouve plutôt en Grande-Bretagne ce qu'on appelle "l'establishment" composé à la fois de l'aristocratie, de la bourgeoisie et des classes moyennes. L'aristocratie n'a pas de privilèges fiscaux contrairement à la France. \\
Concernant le monopole militaire, le pouvoir central, en Grande-Bretagne, n'a pas de monopole militaire car il n'y a plus de guerre à mener. Très tôt, les rapports sociaux intérieurs sont pacifiés. La police n'est jamais devenu centralisée et est resté très peu professionnalisée. \\
Enfin, au niveau de l'interventionnisme économique, il est resté très faible jusqu'à aujourd'hui: on y trouve bien moins d'entreprises nationalisées. L'individualisme, le libre jeu du marché, est très promu en Grande-Bretagne, et a réduit donc son interventionnisme. Il y a un développement très précoce du capitalisme en Grande-Bretagne. C'est donc le marché qui est premier et non l'État, alors, qu'en France, l'État organise le marché, historiquement. 

\section{L'État post-colonial: l'État néo-patrimonial}

Schmuel Eisenstadt parle du concept de "néo-patrimonialisme". \\
Nous allons nous intéresser aux effets de la décolonisation en Afrique noire et dans le monde musulman. Dans les deux cas, le processus de colonisation vient se produire pendant la seconde moitié du XIXe siècle et la première moitié du XXe. Elle est le fait des États Européens, constitué de longue date, parmi lesquels la France ou encore l'Angleterre au Proche-Orient et en Afrique Noire. Les États colonisateurs sont fortement constitués au moment où ils colonisent d'autres aires géographiques. \\
Ces États vont exporter le modèle occidental de l'État dans les aires géographiques qu'ils colonisent. Aux indépendances, les nouveaux États qui vont se constituer vont poursuivre ce phénomène de reproduction du modèle de l'ancien colonisateur. Cependant, et notamment du fait des structures sociales et culturelles pré-existantes, cela aboutit à une forme d'hybridation du modèle occidental, c'est cette hybridation qu'on appelle l'État "néo-patrimonial". 


\subsection{L'effet de diffusion et reproduction du modèle occidental}

La colonisation européenne a principalement deux effets sur les formes d'organisation du pouvoir politique: le premier est ce qu'on appelle la territorialisation du pouvoir politique (le fait d'associer le pouvoir politique à un territoire). Une telle association favorise un phénomène de centralisation du pouvoir politique. \\
Le tracé de ces frontières a été faite de manière très arbitraire, "à la règle", au congrès de Berlin en 1875. Cet arbitraire a été tel, cette totale négation des ethnies, qu'il a morcelé des ethnies co-existante et a regroupé des ethnies concurrentes. Avant la colonisation, dans ces aires géographiques, la territorialisation n'était pas achevée. \\
Le pouvoir se fondait sur les ethnies et les communautés. Il se définit par rapport à la société, à sa structuration sociale et aux ethnies. 


Au lendemain des indépendances, en 1963, la conférence Addis-Abbedha va créer l'OUA (l'Organisation de l'Unité Africaine), qui va prononcer l'intangibilité des frontières issus de la colonisation. Cela a été fait dans le but d'éviter le bain de sang, ce qui a repoussé le problème. \\
La colonisation se traduit par l'imposition des structures administratives sur le modèle de l'État occidental. Cependant, l'imposition de ces structures administratives du centre n'a pas totalement supprimée les structures pré-existantes. Les colonisateurs ont par exemple installé leurs propres fonctionnaires afin de faire appliquer le droit positif européen. \\
Soucieuse d'asseoir leur autorité, les métropoles ont fait appel aux chefs locaux ayant un ascendant sur les populations pour les utiliser comme courroie de transmission entre la métropole et les territoires colonisés. Ils ont aussi été utilisés pour réduire les éventuels poches de résistance en utilisant les ressorts de légitimité admis et partagé par les populations locales. Ces chefs locaux assurent l'emprise du colonisateur en même temps qu'ils illustrent la persistance de principes traditionnels d'organisation de la société locale. \\
Il y a deux niveaux administratifs imbriqués, juxtaposés. À l'indépendance, les nouveaux États se trouvent devant des structures administratives hétérogènes reposant sur deux héritages bien différents. De manière surprenante, la décolonisation ne va pas interrompre ce mouvement de diffusion du modèle occidental. Il y a un puissant vecteur de mimétisme, l'imitation du modèle: drapeaux, hymnes, constitutions. Le mimétisme n'est pas seulement engendré par la colonisation, il est aussi renforcé par les coopérations qui font que les individus africains viennent faire leurs études en occident et vont donc transmettre un système de valeur qu'ils  ont acquis dans un pays anciennement colonisateur. La coopération international ne reconnaît que des États souverains qui ont des frontières claires et délimités: la coopération internationale est un puissant facteur de reproduction de l'État. La forme de l'État permet la reconnaissance de l'ONU, les prêts internationaux, etc. 


L'auteur part de la situation de l'État patrimonial. L'État néo-patrimonial est donc dans la continuité de celui-ci. La principale différence entre l'État occidental et l'État néo-patrimonial, se situe de la différenciation. Elle est très faible dans le cadre de l'État néo-patrimonial. \\
La première caractéristique de l'État patrimonial est l'appropriation privée du pouvoir politique. En effet, aux indépendances, se mettent souvent en place, des régimes de parti unique qui va permettre la constitution d'une élite qui se fait sur une base soit professionnel (comme les militaires), une élite technique (dictature technocratique), sur une base religieuse, ethnique. Cette élite restreinte s'accapare le pouvoir politique et cherche à limiter l'accès des autres groupes aux pouvoirs politiques et notamment aux ressources qu'impliquent la détention du pouvoir, essentiellement économiques, sous la forme de matières premières. Le détenteur du pouvoir est aussi le détenteur de l'économie. \\
L'appropriation des richesses se fait sur le plan internet et externe. Sur le plan interne, tout le processus de développement économique du pays est contrôlé par l'État et sont accaparés par les détenteurs du pouvoir politique. Sur le plan externe, l'accès des pays occidentaux aux ressources naturelles des États en développement se réalise en échange de versement financier aux détenteurs du pouvoir politique. On constate aussi un détournement important des aides internationales à des fins personnelles. \\
Cette confusion montre la faible différenciation. L'appropriation du pouvoir politique est l'appropriation du pouvoir économique: non différenciation du privé/public.


Il y a aussi une faible différenciation de l'appareil administratif. La bureaucratisation atteint des niveaux records dans les États issus de la colonisation. Le degré d'institutionnalisation et spécialisation est élevé mais l'administration est très peu différenciée. L'administration s'est très largement développée mais ne s'est pas autonomisée du pouvoir politique, au contraire, l'administration est un moyen pour l'élite au pouvoir de faire perdurer son pouvoir. Le pouvoir politique cherche à éviter l'élaboration d'une nouvelle élite. Le gouvernement pléthorique de l'administration permet d'absorber les jeunes diplômés dans les fonctions de l'État. \\
La faiblesse de la différenciation, on peut l'observer aussi du point de vue des rapports entre l'administration et la société. Le nombre élevé de fonctionnaires a pour effet la faiblesse de leurs revenus. Cela génère un effet de corruption et de clientélisme important entre la fonction publique et la société privé. 


Il y a des modes de légitimation spécifiques. Il y a une très forte présence de la légitimation traditionnelle, notamment via la religion, l'appartenance à un dynastie. Fondée sur le charisme avec les héros des révolutions nationales. Il y a une forte personnalisation du pouvoir politique avec des cultes de la personnalité. 


La catégorie de l'État néo-patrimonial est très général, trop large. 


\chapter{Les analyses des États contemporains}

Il y a eu trois formes de l'État dans son long processus de création. Il y a l'État providence, l'État producteur, l'État régulateur. 

\section{L'État providence et les développements de l'interventionnisme étatique}

Au tournant du XIXe siècle, tous les États européens ont été obligés de reconnaître les droit sociaux des citoyens. Ce que T.H.Marchal appelle une citoyenneté sociale. Il distingue trois étapes dans la construction de l'État moderne, correspondant à trois dimensions de la citoyenneté. Chacune de ces étapes a permis la reconnaissance de trois ensembles de droits. \\
Les premiers droits sont les libertés individuelles: au XVIIe siècle en Grande-Bretagne, au XVIIIe en France. C'est la dimension civile de la citoyenneté. C'est la liberté de la personne, d'expression, de foie, de la justice. \\
Les seconds droits sont les droits politiques, qui correspondent à la citoyenneté politique. Cela s'est fait au cours du XIXe siècle. Elle implique un citoyen à l'exercice du pouvoir via le biais de l'élection. \\
L'étape la plus tardive correspond aux droits sociaux qui correspondent à la citoyenneté sociale qui passerait par un minimum de bien être économique, de la reconnaissance du droit à la sécurité face aux aléas et aux risques de l'existence (la mort, la maladie, le chômage, le veuvage). La mise en place des États providence en Europe a permis la reconnaissance de cette citoyenneté sociale. 


L'extension de la sphère d'intervention étatique est le produit de l'État providence depuis la seconde moitié du XIXe siècle. \\
L'avènement de l'État providence amène à une révolution de la place de l'État. Jusqu'à la révolution industrielle, l'État se cantonnait aux fonctions régaliennes. La mise en place de cet État s'explique par deux causes principales: des causes sociales d'abord, et politiques ensuite. \\
Les causes sociales sont les effets de la révolution industrielle sur le prolétariat urbain. Cela bouleverse les structures sociales comme le traduise des phénomènes comme l'exode rural ou l'urbanisation massive ; l'approfondissement de la division du travail ; la dislocation de la structure familiale qui n'est plus le lieu de la création économique. Ce prolétariat urbain a des conditions de vie très dures. \\
Les causes politiques sont l'accroissement du prolétariat qui entraînent la multiplication de conflits sociaux qui traduit des antagonismes de classes. Il y a une montée en puissance du mouvement socialiste ou des mouvements ouvriers dans les usines dont le patronat ou encore l'État a peur. L'État va rechercher à recréer de la cohésion sociale. La réponse principale qui va être donnée est la prise en charge des effets du développement de l'économie capitalisme via l'intermédiaire des assurances sociales. On passe de l'assistance à l'assurance collective. La responsabilité de l'accident du travail n'est pas individuel, il est le produit de la société industrielle qui est donc responsable.


L'Allemagne de Bismarck joue un rôle de précurseur. Il va faire voter des lois sociales dans les années 1880 et apparaît donc comme un précurseur. Dans la seconde moitié du XIXe siècle, l'Allemagne connaît une accélération de l'industrialisation accompagné par le succès du mouvement ouvrier et socialiste. Bismarck décide d'accompagner sa politique de répression du mouvement ouvrier par une politique de réforme sociale. \\
1871: première loi sur les accidents du travail, approfondi par la loi de 1884, avec une assurance maladie, assurance invalidité et vieillesse en 1889. Ainsi naissait le premier système d'assurance sociale obligatoire, qui garantit un revenu faible de compensation en cas de perte de salaire liés à un risque social. \\
Les caractéristiques du système Bismarckien va avoir beaucoup d'influence dans le reste de l'Europe. \\
En France, c'est sous la IIIe République qu'apparaît la question de l'interventionnisme de l'État dans l'économie. Mais les libéraux sont très farouches face aux républicains. Il faut attendre 1945 pour qu'un vrai système de sécurité sociale s'impose en France. \\
En 1942 Beveridge publie un rapport dont l'objectif était de penser le système social britannique. Il a eu beaucoup d'influence en Europe. Il met en avant le principe d'universalité, il rejette le principe d'une assurance relevé aux seuls salariés. Le but de sont rapport est d'éliminer la pauvreté donc l'ensemble des citoyens sont couvert par le système, qui verserait tous des prestations d'un même montant. Ce système, pour lui, doit être financé par l'impôt. \\
Bismarck: objectif de compensation de revenu pour les salariés alors que l'ensemble de la population est concernée chez Beveridge. Pour Bismarck, l'ouverture aux droits est conditionnés par le versement d'une cotisation alors que chez Beveridge c'est par l'impôt. \\
Le Plan français de Sécurité Sociale, de 1945 se situe entre Bismarck et Biveridge. Il veut atteindre les objectifs de Biveridge avec les moyens de Bismarck. Le plan est à l'époque fondé sur les 3U, Unité, Universalité, Uniformité. Les missions données à ce plan relèvent pour une part de Biveridge, notamment via le ciblage des prestations depuis 1970. En revanche, la sécu mise en place en 1945, jusqu'aux trente glorieuses relève de Bismarck. 

\section{L'État producteur}

C'est en 1944 que l'État producteur se met en place. C'est l'engagement de l'État en tant qu'agent économique. C'est l'État investisseur, dans le domaine industriel, financier, économique. À partir de 1945, le rôle de l'État s'accroît dans l'économie. Cela montre une véritable rupture dans la conception de l'État en France. \\
Les nationalisations entre 1944 et 1946 sont un levier (sanction, stratégique), le deuxième est la planification. Il y a une étatisation de l'économie française. 



\part{Les politiques publiques}

\chapter{Introduction}

Le mot politique est polysémique: le politique, la politique, les politiques publiques. Trois grandes questions: pourquoi les autorités publiques agissent-elles ? Comment agissent-elles ? Quels sont les effets de leur action ? \\
À la suite Jean-Claude Thuenig dans "Le traité de Science Politique", dit des politiques publiques "Une politique publique se présente sous la forme d'un programme d'action propre à une ou plusieurs autorités publiques ou gouvernementales". Il considère qu'une politique publique se repère à cinq traits.
\begin{itemize}
\item Le contenu de la politique publique: droit, budgets et financements (il peut y avoir des énoncés performatifs)\footnote{D. Austin a écrit: un énoncé performatif auto-réalise la chose nommée, il faut que le pouvoir soit assez légitime pour que l'énoncée se réalise.}, des institutions ;
\item Vise un public ciblé, plus ou moins large selon la nature de la politique publique ; 
\item Recours à la coercition: la possibilité de recours à la sanction pour imposer le contenu de la politique publique ;
\item L'inscription du contenu de la politique publique dans un programme, la politique publique s'inscrit dans une durée et une succession d'actes ;
\item Une politique publique correspond à une orientation plus ou moins explicitée, plus ou moins nette. C'est à dire que la politique publique s'inscrit dans une finalité (logique de politique publique). 
\end{itemize}

On distingue quatre grands types de politiques publiques.\footnote{T. Lowi et R. Salisbury}
\begin{itemize}
\item Les politiques réglementaires par lesquelles les autorités publiques édictent des règles s'appliquant à tous ;
\item Les politiques distributives qui reposent sur la distribution d'autorisation, sont personnalisées, à condition de remplir certains critères ;
\item Les politiques redistributives, où la puissance publique redistribue en fonction de certains critères des richesses ;
\item Les politiques constitutives, les autorités édictent des règles sur le pouvoir, fixe les règles du jeu politique. 
\end{itemize}


Cette typologie ne porte que sur l'action normative, que sur le droit. On peut aussi définir les politiques publiques par leurs secteurs d'intervention (les politiques régaliennes par exemple, les politiques économiques, de bien être, d'éducation, territoriale, d'environnements, de loisirs). \\
L'expression politique publique a une extension de domaines assez grands. 


Il y a des phases distinctes et successives dans une politique publique. \\
Le "pourquoi" de la politique fait référence à une question: qu'est-ce qui fait qu'à un moment, un problème vécu à l'état individuel devient une politique publique. C'est ce que l'on appelle un processus de "mise sur agenda". \\
Comment s'opère la décision, comment la mise en oeuvre se fait ? \\
Charles O. Jones dans une étude nommée "An introduction to the study of public policy", est considéré comme étant à la base des analyses des politiques publiques, à la base des travaux de Thoenig, Mény, Muller, Jobert. \\
Jones a créé une grille séquentielle de 5 séquences des politiques publiques.


La première séquence appartient au processus par lequel un problème émerge et est pris en charge publiquement: mis sur agenda. \\
Deuxième séquence, cela débouche sur la formulation d'une solution, d'une réponse, et donc sur une décision. \\
La troisième séquence est la légitimation de la politique publique, de la décision. \\
La quatrième séquence est la mise en oeuvre. \\
La cinquième phase est l'évaluation de la politique publique: mesure de la valeur de la politique publique, c'est un moyen de contrôler l'action des gouvernants. Cela peut donc être un formidable outil démocratique. Il y a au moins quatre critères d'évaluation des politiques publiques: l'efficacité (résultat obtenu, par rapport aux objectifs), l'efficience (les résultats obtenus par rapport aux coûts), la pertinence (la PP posait-elle la bonne question ? Public ciblé était-il le bon ?), la cohérence (entre les différentes mesures). 

\chapter{L'émergence des problèmes et leur mise sur agenda}

La première question qui se pose est de savoir comment une question fait l'objet d'une action d'une autorité publique. C'est donc l'émergence d'un problème vécu individuellement sur la scène publique. On dit qu'un problème public est publicisé. 

\section{Les conditions de l'émergence des problèmes}

Ces conditions sont au nombre de trois. 


La première condition est que le problème soit perçu comme collectif. Toujours dans cette condition, il faut que la cause du problème soit imputable à un tiers. \\
Dans la seconde condition il faut que le ou les individus concernés par le problème ait la possibilité de s'organiser pour faire connaître leurs problèmes. Il faut qu'il y ait une entreprise de publicisation du problème. Cette capacité, c'est ce qu'on appelle les ressources qui permettent de publicisé des problèmes, ces ressources sont principalement intellectuelles, scolaires, organisationnelle, etc. \\
Il faut, en troisième condition, qu'il y ait une population qui fasse écho à la problématique exprimée. La taille et les caractéristiques de la population directement concernée comptent énormément. Il faut aussi prendre en compte l'existence potentiel d'un public, non confronté à un problème mais se sentant concerné et pouvant se mobilisé. Une condition essentielle de la publicisation d'un problème est sa formulation: il faut qu'il soit formulé de telle sorte qu'un plus large public possible se sente concerné par le problème (l'audience du problème). L'audience s'appuie sur différents facteurs, Padioleau dans "L'État au sommet", donne plusieurs variables favorables à l'audience: l'importance de la population directement concernée, le degré de complexité du problème, le degré de nouveauté dans la publicisation, le degré de formulation abstraite ou concrète, l'utilisation de symboles, la traduction du problème dans des termes politiques. Les relais aussi sont importants, surtout si ils ont accès à un public important: il y a des relais généraliste et des relais spécialisés. \\
Les relais généralistes peuvent se définir comme les acteurs qui se donnent pour vocation de traiter tout problème social quel qu'il soit, ce sont par exemple les partis politiques, les médias, les églises, etc. Plus les relais sont nombreux, plus la publicisation sera importantes. Une controverse publique peut exister entre ces différents relais. \\


\section{Les modèles de mise sur agenda}

Cob et Edler parlent de "The agenda setting". \\
En France, c'est Philippe Garraud qui définit l'agenda comme: "l'agenda politique est l'ensemble des problèmes faisant l'objet d'un traitement sous quelques formes que ce soit de la part des autorités publiques et qui est donc susceptibles de faire l'objet d'une ou plusieurs décisions". \\
On distingue cinq modèles de mise sur agenda.


Le premier de ces modèles est celui de la participation, le second est celui de l'offre politique, troisième: la médiatisation, quatrième: l'anticipation, cinquième: action corporatiste silencieuse. 

\subsection{Le modèle de la participation}

C'est le modèle le plus visible, le plus évident, il correspond au modèle de l'offre et la demande. C'est le modèle de la mobilisation collective. C'est un groupe social (syndicat, association, ONG) qui soutient par la mobilisation une revendication à laquelle les autorités publiques répondent. \\
Ici, l'initiative est celui des groupes organisés: parti, syndicat, groupe de pressions. Cette intervention s'appuie sur une forte mobilisation de l'opinion sous la forme des répertoires de l'action collective (manifestation, pétition, grève, etc.). Il y a en général un degré de conflit avec les autorités publiques. L'opinion est impliquée de sorte à faire pression sur les pouvoirs publics. \\
Ce modèle passe l'utilisation massive des médias afin de faire croître l'audience du problème. \\
Cette mobilisation s'appuie sur l'organisation d'événements, joue sur des symboles. \\
Dans ce cadre là, les acteurs politique sont secondaire.

\subsection{Le modèle de l'offre politique}

À la différence du premier modèle, ce sont les acteurs politique qui jouent le rôle moteur. Un thème est politisé, mis en avant par les acteurs politiques comme une réforme. \\
La médiatisation est là aussi forte. \\
Cependant, on a au départ une absence de mobilisation, mais peut en entraîner. \\
On a une absence d'événement et le rôle des groupes organisés est très faible. 

\subsection{Le modèle de la médiatisation}

Ce sont là les médias qui jouent le rôle moteur. Ils sont autonome car les médias sont un vecteur de publicisation. Les groupes organisés n'ont pas un rôle moteur. \\
L'exemple le plus typique est celui de la presse d'investigation: Panama Papers, affaires de financement public, etc. \\
En général, la controverse publique est fortement présente, le monde politique réagissant en lançant des perquisitions ou le monde journalistique essayant de marginaliser ce genre de presse. \\
L'événement joue ici un rôle moteur, car c'est ici tout l'enjeu que d'avoir le scoop pour faire le buzz. 


Ces trois modèles ont comme point commun d'être fortement médiatisés. Les deux derniers modèles s'opposent à ces trois-ci car sont beaucoup plus silencieux.


\subsection{Le modèle de l'anticipation}

Le rôle moteur appartient aux autorités publiques, mais plutôt aux techniciens, aux experts, aux hauts-fonctionnaires, qui travaillent dans l'ombre et qui, en rendant des rapports, va conduire le Gouvernement à en publier certains, et donc à le publiciser. \\
La médiatisation est généralement faible ainsi que sa politisation car le sujet est souvent très technique, on peut prendre comme exemple les réformes de l'assurance maladie, les politiques de luttes contre la toxicomanie, de santé publique, du SIDA, etc. 

\subsection{Le modèle de l'action corporatiste silencieuse}

C'est ici un groupe disposant d'un accès privilégié aux autorités publiques. Le problème est ici plus ciblé, concerne un public restreint. Là aussi la politisation, la médiatisation et la mobilisation sont faible. Une forme de confinement, de huis clos se fait autour de cette action. C'est l'oeuvre typiquement des lobbies. 


Ces modèles sont mêlés dans la réalité, voir peuvent se succéder. Ce sont des idéaux-types. \\
Aparté, "policy windows", fenêtre d'opportunités politiques. Homologie structurale (Bourdieu): similitude forte entre des acteurs appartenant à des champs sociaux différents favorisant la prise en charge publique d'un problème. 


\chapter{La décision et la mise en oeuvre}

Correspond à la formulation et légitimation dans la grille séquentielle de Jons. La mise en oeuvre correspond à l'application. 

\section{La rationalité décisionnelle et ses limites}

La décision est la phase la plus mise en avant dans une politique publique. Dans cette conception courante de la politique publique: décider c'est agir, comme si le moment de la décision représentait l'apogée du travail politique. Cela renvoie à la valorisation de l'acte individuel ("j'ai décidé de...", "l'État veut"). \\
Sfez a mis en avant les postulats sur lesquels se font les décisions: elle est supposée être libre comme si les institutions fonctionnaient comme un seul homme, comme si il n'y avait aucune contraintes, ce qui est faux: contraintes juridiques, budgétaire, etc. ; elle est supposée être individuel ; elle est supposée être rationnel, or, le politique n'est pas un scientifique, il ne fait pas des politiques publiques objectives ; la linéarité du processus. 


La conception spontanée est fondée sur la croyance du choix optimal par un décideur clairement identifiable et rationnel. Cette conception repose sur quatre postulats. \\
Premièrement, c'est l'idée qu'en toutes circonstances, il existe un critère ou des critères objectifs, accessible au décideur et partagé par ceux qui l'entoure (ministre, conseiller...). \\
Deuxièmement, les préférences sont stables et explicites. \\
Troisièmement, c'est que toutes les alternatives possibles sont connus avant de choisir la solution. L'information serait donc transparente et le décideur décide en parfaites connaissances de causes. \\
Quatrièmement, le décideur a pour seule finalité de trouver la solution la plus optimale au problème et son choix ne tient compte que des données propres au problème.


Ces postulats ont étés vivement critiqués notamment à l'aide des travaux de March et Simon, qui fondent le concept de rationalité limitée des politiques publiques. À l'aide de ce concept, ils essayent de démontrer que la rationalité du décideur n'est pas tenable. Ils reprennent donc ces quatre postulats et les critiques. \\
Sur le premier postulat, ils démontrent que les critères de choix sont difficiles à déterminer et sont la plupart du temps contradictoires et ont un degré de subjectivité élevé et ne sont donc pas partagés. \\
Sur le deuxièmement, les préférences sont instables et beaucoup liés au contexte. La conjoncture ferait donc plus la décision que le décideur. \\
Troisièmement, l'information est imparfaite et incomplète et souvent très aléatoire sur les effets de la décision à mettre en oeuvre. Ils montrent comment la principale incertitude de la prise de décision provient de l'ignorance quant aux conséquences futures de la décision. \\
Enfin, les finalités du décideur sont multiple, ses intérêts sont variés et le décideur ne choisira pas la meilleure solution au problème mais sans doute celle qu'il estime la plus rentable politiquement. \\
Dans le domaine des politiques publiques, la décision est le résultat d'une série d'interactions et d'une série de contraintes. Il faut donc concevoir la prise de décisions comme la résultante d'un système de contrainte et non comme le produit d'une action individuelle. On distingue trois grands types de contraintes: socio-économiques, de pouvoir, institutionnelles. 

\section{Les facteurs de distorsion entre décision et mise en oeuvre}

La phase d'application est très peu valorisée. À partir du moment où l'acteur politique prend une décision, cela parait automatique qu'elle soit mise en oeuvre. Dans les faits, il y a une autonomie relative des échelons subalternes au Gouvernement. \\
L'administration est subordonée à la loi. Selon le principe démocratique, les fonctionnaires doivent être placés sous l'autorité des élus. L'administration a un troisième type de subordination, sociale, avec le principe de la primauté de l'intérêt général qui conduit au principe de neutralité qui devrait normalement faciliter l'exécution de la règle édicté. \\
La vision des politiques publiques qui découlent de tout ça repose donc sur une forte séparation entre la décision et l'application. C'est cette vision que l'analyste des politiques publiques remet en cause lors de l'étude de la mise en oeuvre. \\
Cette vision "top-down", très linéaire ne correspond pas à la réalité et on peut chercher à montrer qu'il y a des facteurs de distorsions. On peut donc mettre en évidence une non automaticité de l'application de la décision, ou encore les interprétations possible par les fonctionnaires de mises en oeuvres, mettant en oeuvre une politique publique non semblable à la décision qui avait été prise.


Facteur principal de distorsion entre la décision et la mise en oeuvre: autonomie d'action des échelons administratifs d'exécution (Lipsky, qui parle de "Street level bureaucrat"). Il montre que la base administrative échappe en partie au contrôle du sommet et montre que la mise en oeuvre de l'action administrative est loin de se résumer à la simple application de la règle. Tout un ensemble de facteurs a pu mettre en évidence que souvent, les décisions ne sont que partiellement voire pas du tout appliquée par l'administration. \\
Le premier facteur c'est la distance du niveau d'autorité et les niveaux d'exécution, plus cette distance est importante et également, plus les niveaux d'exécution sont nombreux, plus l'application de la décision est difficile. \\
Un second facteur important est le mode de formulation de la décision. Moins une décision est formulée clairement, précisément, plus l'application en est difficile. Si les termes ne sont pas clair, les risques de ré-interprétation, de formulation, etc, sont accrus. Plus les objectifs sont flous et non formulés, plus les distorsions seront importantes. \\
Un troisième facteur concerne les instruments dont possèdent l'administration pour mettre en oeuvre la politique publique. Se pose ici le problème des outils techniques ou de la maîtrise des outils techniques. \\
Un quatrième facteur de distorsion est l'existence de stimulants ou de sanctions. Plus les incitations à la réussite sont importante, plus celle-ci a des chances d'être exécuter, conformément aux objectifs de départ. La question est de savoir si les acteurs administratifs ont intérêt ou non à mettre en oeuvre une décision, donc l'existence ou non de sanction en cas de non application d'une sanction. \\
Le cinquième facteur renvoie à l'existence ou non de délais pour la mise en oeuvre. La présence d'un délai favorise la distorsion. \\
Les contextes électoraux et pré-électoraux sont favorables aux distorsions, les élus étant moins regardant pendant ces périodes, étant plutôt intéressé par la politique que les politiques. \\
Au niveau territorial, il peut y avoir parfois davantage de proximité entre l'agent administratif et les usager qu'entre l'agent administratif et sa hiérarchie. Dans les services sociaux, il peut y avoir des rapports d'interconnaissance qui peuvent conduire à une adaptation de la règle en fonction des cas par les agents. \\
Pierre et Catherine Grémion, dans "Le préfet et ses notables", montrent qu'il y a des liens, des relations de confiance et de réciprocité entre le préfet (représentant de l'État) et les élus locaux, contre même la hiérarchie du préfet, qui va chercher à adapter la règle pour favoriser le gouvernement local. \\
Dernier facteur, le principe de résistance au changement. C'est un phénomène classique connu depuis longtemps par la sociologie politique. Il est moins couteux en terme d'investissement, de connaissance, et de détention de pouvoir, de ne pas appliquer la règle qui remettrai en cause ces positions ou qui amènerait à remettre en cause les habitudes. \\
L'automaticité de l'application de la décision est donc loin d'être effective, et c'est bien la mise en oeuvre qui fait la Politique. 

\section{Les limites du modèle séquentiel}

Le schéma cybernétique se fonde sur plusieurs présupposés, dont par exemple l'existence d'un champ politique autonome, ou encore qui oublie que le champ politique ne prend pas en charge certaines demandes, en anticipe d'autres, de lui même, etc. \\
Le schéma cybernétique est une représentation de la logique démocratique dans sa plus pure forme, le modèle ne va que dans un sens, sur le postulat que le système politique ne marche que sur l'offre et la demande. \\
Le schéma cybernétique, étant systémique, ne prend pas en compte les crises, les blocages, les grèves, et les prends pour un dysfonctionnement.


Le modèle séquentiel, lui, est plus complet, et a des vertus heuristiques. Il permet de simplifier la recherche. Cependant, ce modèle donne tout de même une vision linéaire qu'il ne faudrait pas confondre avec le réel. Dans cette grille, la réalité est simplifiée pour en faciliter l'intelligibilité. La grille séquentielle est non conforme à la "texture empirique". 

\subsection{La non linéarité des politiques publiques}

Les politiques publiques ne sont pas concrètement, une succession linéaire de séquences. L'idée centrale du modèle séquentiel est la linéarité, l'idée que les politiques publiques forment un enchaînement linéaire qui va de l'émergence du problème, de l'identification d'un problème à sa mise en oeuvre et à son évaluation. \\
Cela renvoie à une "conception balistique de l'action publique", ce qui signifie que l'autorité étatique poursuit une cible, un but, au travers d'une série de décisions mise en oeuvre. La vision linéaire et balistique ne correspond pas à la phase empirique de l'étude. On peut faire cinq constats. \\
Le premier constat est que la cible de l'action publique est souvent équivoque. Les objectifs sont flous, par conséquent, c'est souvent à travers l'application, que se définissent plus clairement les objectifs et les finalités. L'objectif dépend donc de la mise en oeuvre et non pas de la décision. C'est ce qui constitue la critique majeure de la conception balistique. M. Delmas-Marty dans "Le flou du droit" dit que le flou est volontaire car il n'est pas contraignant, permet la souplesse, des aménagements. D'une certaine manière, on peut appliquer cela aux politiques publiques qui ne se voient pas fixer des objectifs clairs. \\
Le développement de l'évaluation est devenu la règle dans le cadre européen et les objectifs sont donc de plus en plus nécessaires. La fixation d'objectifs est beaucoup plus prégnant au niveau local qu'au niveau étatique. 


Le moment de la décision est souvent bien difficile à identifier. Elle ne précède pas forcément la mise en oeuvre, car, au cours de la mise en oeuvre, des décisions importantes peuvent infléchir le sens de l'action publique. La formulation des problème peuvent aussi mettre en lumière d'autres problèmes qui seront donc identifié et formulé dans le même temps. \\
La métaphore balistique nous montre que la cible est mouvante et l'orientation du tir est donc modifié. \\
Il n'y a pas forcément de décisions dans le cadre d'une politique publique. Une non décision peut correspondre à une politique publique dans la mesure où elle produit des effets et correspond à un choix politique. La cible peut être atteinte sans tirer. \\
Les décalages, les distorsions, peuvent être très important, à tel point que dans certains cas, la mise en oeuvre est quasiment autonome par rapport à la décision. Les publics cibles font parfois la politique (réforme des transports routiers qu'ils peuvent ne pas appliquer). Dans ce cas, on peut dire que la balle tire sans tireur et rate sa cible. \\
Enfin, l'évaluation peut se produire à la fois en amont et en aval. L'évaluation est lui même un processus d'apprentissage correspondant à un enchevêtrement de décisions, de mises en oeuvre et d'évaluations.


C'est donc à partir de ce constat d'inadéquation de la vision séquentielle et balistique des politiques publiques que March, Simon, Olsen et Cohen, le concept de rationalité limité pour les deux premiers, et tous les quatre le concept de politiques publiques comme des "anarchies organisées" et construisent "le modèle garbage cam". \\
Le modèle du garbage cam repose sur trois idées principales: d'abord, les auteurs parlent de "préférence incertaine", c'est à dire que les acteurs politiques ne savent pas très bien ce qu'ils veulent, ou alors une conflictualité entre eux fait qu'il y a des objectifs difficiles de concilier entre eux. Donc il y a une action avant une définition des préférences (on a ici une inversion des séquences), et les problèmes et les objectifs se formulent dans l'action. \\
Une "technologie floue", une faible maîtrise de la technologie favorise ce genre d'actions, comme le fait aussi une phase de mise en oeuvre non contrôlée. L'action se fait par tâtonnement successif. La mise en oeuvre est donc toujours une phase relativement autonome. C'est l'idée d'une politique publique qui serait incrémentaliste. L'incrémentalisme va donner une action faite de pleins de petites touches. \\
"Participation fluctuante", les auteurs du modèle de la poubelle nous dise qu'il y a une très grande ouverture dans la prise de la décision: les acteurs sont nombreux, ne sont pas du tout en accord. Le contenu même de la décision est donc difficile à cerner. Les conséquences de ce mode de fonctionnement sont donc multiples: 1. l'évaluation devient extrêmement difficile car les objectifs sont flous et fluctuants, 2. le lien problème/solution est inversé, ce sont le plus souvent les solutions qui permettent de formuler les problèmes, 3. la décision par résolution du problème n'est pas le cas le plus fréquent, 4. Le mode d'action s'apparente alors à une poubelle dans laquelle se décharge au fur et à mesure et en allant, des problèmes et des solutions des participants: "Chaque occasion de choix comme une poubelle dans laquelle les différentes sortes de problèmes et de solutions sont jetés par les participants au fur et à mesure de leur apparition. La décision et l'action sont le produit d'un processus aléatoire lié à des conjonctures particulières, un processus que personne ne maîtrise véritablement, les décisions et l'action sont plus le fruit du hasard que d'une volonté maîtrisée." \\
Les auteurs remettent donc radicalement en cause l'idée d'un enchaînement linéaire de séquence mais aussi de remettre en cause la rationalité de l'action publique. 


Critiques au modèle de la poubelle. \\
Il semble que les acteurs politiques sont tout à fait passif dans le cadre de ce modèle, comme si ils étaient juste porteur de solutions et de problèmes mais ne seraient pas dotées d'une capacité d'action stratégique. \\
Les tenants de cette conception ont une préférence systématique pour les éléments aléatoires, les éléments structurants sont donc systématiquement sous-évalué. \\
Ce modèle a du point de vue empirique, a surtout été testé à partir de l'étude d'organisations et les modèles de décisions dans les universités et peu testé sur des analyses plus globales de politiques publiques. \\
Pour toutes ces raisons, le modèle de l'anarchie organisée ne peut pas, comme le modèle séquentiel, valoir comme représentation fidèle de la réalité. Il met en exergue le caractère ambigu dans les choix, dans les préférences et le caractère aléatoire dans tout le processus de toutes politiques publiques.


Dans la continuité, Pierre Lascoumes parle de bricolage pour définir les politiques publiques. Effectivement, bricolage va lui servir à caractériser l'action publique sur la base de trois constats. Il fait une analogie avec les machines sculptures de Tinguely car avance l'idée que les politiques publiques traduisent l'idée du recyclage: "Tant leur matériau que leur dynamique nous semble adéquat pour exprimer la récupération et le détournement de composants usagés, leur assemblage hétéroclyte, leur agencement en machines hasardeuse, leur animation par des efforts dispersés, réguliers, aussi bien qu'aléatoire, leur impulsion sans source clairement lisible, leur déplacement erratique ou obsessionnel, leurs élans inattendus, enfin les alternances d'agitation et de surplace qui caractérisent aussi les mouvements de la plupart des politiques publiques". \\
Premier constat: une politique publique résulte d'un processus d'ajustement progressif d'intérêts et de projets qui s'inscrit dans une série de contraintes. \\
Deuxième constat: une politique publique ne prend forme que dans sa mise en oeuvre, du fait du flou qui caractérise les décisions publiques et plus largement tout projet d'actions publiques. \\
Troisième constat: une politique publique est une reconstruction à posteriori, elle résulte de la mise en cohérence d'une série d'ajustement, de tâtonnement, de réponses au coup par coup. L'idée de bricolage ramène à l'idée que c'est aux analystes de politiques publiques qui redonnent de la cohérence aux politiques publiques. 

\end{document}
