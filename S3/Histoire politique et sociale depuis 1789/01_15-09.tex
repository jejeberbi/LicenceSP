\documentclass[10pt, a4paper, openany]{book}

\usepackage[utf8x]{inputenc}
\usepackage[T1]{fontenc}
\usepackage[francais]{babel}
\usepackage{bookman}
\usepackage{fullpage}
\setlength{\parskip}{5px}
\date{}
\title{Cours d'histoire politique et sociale de la France depuis 1789 (UFR Amiens)}
\pagestyle{plain}



\begin{document}
\maketitle
\tableofcontents

% Antonetti - Manuel


\chapter{Introduction}

On va chercher à comprendre ce qu'est la société Française en 1789, juste avant la révolution Française qui balayera la monarchie, construite depuis 1000 ans. On cherche à comprendre pourquoi il y a eu une Révolution. \\
La France est coutumière de ce genre d'événements, il se passe des événements avant 1789, des révoltes, des soulèvements, liés aux impôts trop élevés ou à la faim. La Révolution est quelque chose de bien plus profond, il suppose que ceux qui la font souhaitent changer le système. Quand, au moyen-âge, les paysans se soulèvent, ils font une fronde car ils ne souhaitent pas changer le système, ils ne veulent et ne peuvent pas prendre la place du seigneur. \\
La révolution, pour être faite, a donc besoin d'une idée intellectuelle suffisamment mature pour substituer le système à un autre.

\section{L'état de la France en 1789}

Le premier élément saillant est la population, 30 millions de français en 1789 font de la France le Royaume le plus peuplé d'Europe. Cette population permet à la France d'être un État riche, il a une forte capacité fiscale, mais aussi une forte capacité militaire en cas de guerre. \\
C'est donc un royaume très puissant, d'autant plus que l'économie est florissante car le commerce est important, notamment le commerce colonial et triangulaire. L'économie française repose néanmoins essentiellement sur l'agriculture, qui est aussi florissant. Les français du XVIIIe investissent en grande partie dans la terre par opposition aux anglais qui investissent dans l'industrie. \\
Quand un pays est riche grâce à son agriculture, au moindre événement climatique, les récoltes s'effondrent et donc l'économie aussi. L'économie agricole est donc très fragile. 


Au niveau politique, il y a la monarchie absolue de droit divin. Quand on dit que le Roi gouverne seul car Dieu lui a donné ce pouvoir, cela suppose de croire en Dieu pour les français. Cela se passe très bien durant tout le Moyen-âge, mais, à la fin du XVIIe siècle, on commence à remettre en question la religion et même l'existence de Dieu. Ce qui conduit donc à douter de la nature divine du pouvoir.


La société est composée d'ordres, les trois ordres que sont le Clergé (Auratores, ceux qui prient), la Noblesse (Bellatores, ceux qui se battent) et le Tiers État (Laboratores, labor, le travail). Cette société d'ordres est apparu au Moyen-âge où elle se justifiait totalement. Au Moyen-âge, les seigneurs prélèvent l'impôt, et n'en payent pas car ils payent "le prix du sang", car ils se battent constamment. Quant au Clergé, sa position se justifie car la société est convaincue de la religion chrétienne et le clergé prie donc pour le salut de tous les hommes, et est considéré donc comme rendant un service à la société et donc, ne paye pas d'impôts. Tous les autres, le Tiers État, payent des impôts car ils ne prient pas et ne se battent pas.


La société d'ordres commence à ne plus se justifier car la noblesse ne fait plus la guerre et le clergé est remis en question. D'autant que cette société d'ordres est une société profondément inégalitaire, pas seulement entre les ordres mais à l'intérieur des ordres aussi. \\
Dans le clergé, on va distinguer le haut clergé et le bas clergé. Le haut clergé, ce sont les évêques et les abbés. Ceux-ci sont nommés par le Roi depuis 1516 et va, au XVIIIe, désigné des nobles qui n'ont pas forcément la foie et ne respectent pas le voeu de célibat, voire même, ne vivent pas dans leurs abbayes. Ce haut clergé vit dans un luxe très important. Le bas clergé, lui, est constitué de prêtres et de moines et vivent "chichement" et presque misérablement, ils vivent de l'aumône et des maigres terres qu'ils exploitent. \\
Ces inégalités se sont longtemps justifiés par l'éducation car on suppose que la noblesse est plus éduqué. Cependant, au XVIIIe, le tiers état est de plus en plus instruit. 


Concernant la noblesse, tous les nobles ne sont pas égaux et on distingue la haute noblesse de la petite noblesse. La haute noblesse, c'est une poignée de personnes, ce sont les princes, les ducs, certains marquis, certains comtes. Ce sont des aristocrates (vivant à proximité du Roi) très riches, en terre comme en argent. Cette noblesse forme la Cour, principalement basé à Versailles. \\
La petite noblesse est une noblesse de terre, de campagne. Cette noblesse est la noblesse qui a peu d'argent et survit uniquement grâce à ses privilèges fiscaux. C'est une noblesse qui ne peut pas voir le Roi.


Pour le tiers État, c'est l'ordre inégalitaire par excellence, et sont criantes à l'intérieur de cet ordre. Sur 30 millions de Français, très peu sont privilégiés. \\
Tout en haut du tiers état, on a la haute bourgeoisie qui s'est enrichi par le commerce, qui vit dans des hôtels particuliers, qui fait construire des châteaux, qui essayent de vivre comme des nobles. C'est une bourgeoisie tellement riche qu'elle prête de l'argent au Roi. C'est une bourgeoisie très éduquée, qui fréquente les meilleurs écoles. \\
Plus bas, il y a la moyenne bourgeoisie des villes. Ce sont des commerçants aisés ou des agriculteurs aisés, des gens qui ont des manufactures. C'est la bourgeoisie des juristes, les avocats, les magistrats etc. Ils ont de l'argent et en prête aussi au Roi, moins, certes. Cette moyenne bourgeoisie est aussi très éduquée. \\
Plus bas, il y a une basse bourgeoisie qui constitue la majorité des artisans, cordonniers, bouchers, agriculteurs etc. \\
Tout en bas, c'est le monde des domestiques, des ouvriers agricoles, des ouvriers de manufactures, des employés en général. Ils gagnent peu et payent en plus de l'impôt. 


Nous avons à faire à une société de corps. C'est à dire qu'un individu, seul, n'est rien. Un individu c'est avant tout le membre d'un corps, d'un groupe. Le premier corps auquel appartient un individu est la famille. Famille qui est sacrée et dirigée par le père, qui est le chef de famille et qui a la capacité de décidé de beaucoup de chose (mariage arrangé entre autres), le père règne en maître dans sa famille. \\
Le deuxième corps auquel on appartient est la paroisse. L'individu appartient à la paroisse car il n'y a aucune diversité religieuse. \\
Ensuite, les individus intègres des corps de métiers. Corporation de métiers qui ont des maîtres. \\
Il y a aussi la communauté d'habitant, le village ou la ville dirigé par un maire qu'on appelle mayor. \\
On appartient aussi à une province. Cette province est dirigée par un gouverneur. Au dessus de la province, il y a un plus grand corps, celui de son ordre. \\
L'individu appartient aussi au Royaume mais quand on pousse encore la réflexion, l'individu appartient à la société des chrétiens. \\
Avec ce découpage, on se rend compte que tout tourne autour des chefs et de Dieu. L'individu n'existe donc pas, il a le nom de son père. Cette société sera donc ébranlée quand on commencera à penser que l'individu existe, voire même qu'il a des droits en tant qu'individu en tant que tel.

\subsection{La place du Royaume de France en 1789}

L'Europe, en 1789, c'est en grande partie des monarchies absolus. Le régime français n'est donc pas isolé. On considère que l'absolutisme est le régime qui est le plus adapté aux vastes territoires. L'Espagne, le Portugal, les deux Sicile, l'Autriche, la Russie sont les grandes puissances d'Europe monarchistes et absolus. \\
Il y a un ensemble de petites Républiques, ce sont généralement des villes comme Venise, Genève etc. et sont plutôt des oligarchies malgré leur nom. On pense que les Républiques sont valable que pour les petits territoires, notamment les villes. \\
Il y a deux cas à part, le Royaume-Uni qui, depuis la fin du Moyen-âge, est dans une monarchie parlementaire ; la Pologne, vaste monarchie d'Europe centrale, a une particularité: c'est une monarchie élective. Les aristocrates désignent un Roi, et ont déjà élus un prince étranger: Henri III. \\
La France a toujours eu une relation particulière avec la Pologne. Depuis Henri III, le France protège la Pologne. \\
Le Roi de France, est, en 1789, en position d'arbitre au sein de l'Europe car il est un des plus puissants, ce qui s'explique par la monarchie absolu, contrairement au Royaume-Uni, et l'idée nationale est très forte. La France a aussi des alliances clés, des alliés traditionnels: les cousins du Roi de France, les Bourbons, dont l'un est Roi d'Espagne, et l'autre Roi des deux Sicile. Autre allié traditionnel de la France: la Russie. Depuis 1750-1760, un nouvel allié, qui a longtemps été l'ennemi, est l'Autriche. Ce réseau d'alliance est fait pour fonctionner contre l'Angleterre. \\
La France fait jouer ces alliances pour protéger la Pologne car les voisins polonais sont menacés par la Prusse. 


Dans ce réseau d'alliance, on comprend fondamentalement que deux pays ont tout intérêt à ce que le géant français soit affaibli, l'Angleterre et la Prusse. \\
L'Europe des monarchies n'a donc pas marchés, les autres Rois n'ont pas aidés Louis XVI car il était dans l'intérêt des autres monarchies que la France s'affaiblisse. \\
Les États-Unis sont une quelque sorte de modèle, car ils ont obtenu leur indépendance contre un Roi, à l'aide d'un roi, Louis XVI. Le modèle américain va donc beaucoup jouer en France. 

\subsection{Les institution de la France en 1789}

La France est une monarchie absolue de droit divin, avec cette idée que le Roi représente Dieu sur Terre avec un pouvoir qui est théoriquement illimité. En pratique, il y a des lois fondamentales qui encadrent le pouvoir du Roi et surtout des limites pratiques à l'autorité du Roi. \\
La France est un pays immense, très vaste, et l'information circule lentement, comme les ordres. \\
Le Gouvernement qui existe en 1789 est un gouvernement bureaucrate. C'est quatre secrétaires d'États, un contrôleur général des finances et un chancelier pour la justice. Ces personnes sont censés donnés tous les ordres en France, sous ordre du Roi. Tout part du Roi et tout aboutit au Roi. En pratique, le Roi ne peut pas décider de tout, donc il va se faire aider par des fonctionnaires, des bureaucrates, qui vont l'aider à gouverner. Le Roi ne décide que des grandes orientations alors que dans la fiction juridique, il est censé décidé de tout. \\
Ce gouvernement va être de plus en plus critiqué et notamment un de ses instruments, les lettres de cachets. Ce sont des lettres écrites par le Roi qui ordonne à telle ou telle personne de faire une action. La lettre de cachet est un instrument juridique. Il en faut donc des centaines par jour pour faire fonctionner un royaume de 30 millions d'habitants. C'est le secrétaire de la main qui était le seul à pouvoir imiter la signature du Roi et qui faisait des lettres pour les ministres. \\
Quand on va découvrir que le Roi ne décide plus de grand chose, on va critiquer le système quand on va comprendre qu'il est devenu entièrement bureaucratique. 


Le Gouvernement est centralisateur. Le Royaume est divisé en 33 intendances, chaque intendance est géré par un intendant qui est un représentant du Roi. Les intendances sont elle même divisées en subdélégation à la tête desquelles on trouve un subdélégué. Ceux-ci aidaient aussi à faire remonter l'information. \\
La centralisation est la grande marque de fabrique de la royauté Française. Les Rois font tout pour centraliser leur Royaume. Ils font cela car la France est une collection de provinces et donc le roi essaye d'empêcher que ces provinces se détachent. En 1789, les cultures sont certes différentes, mais tout le monde se voit comme sujet du Roi de France. Cette cohésion nationale va jouer un rôle important pendant la Révolution, car supprimé le Roi, c'est prendre le risque que des provinces décident de réclamer leur indépendance.


\section{Les faiblesses et crises de la monarchie}

\subsection{Les faiblesses}

La première faiblesse du Royaume de France est Louis XVI. Louis XIV a construit la monarchie absolue française autour de lui, il a construit un pouvoir très fort concentré autour de lui. Tout le régime est basé autour de cette figure du Roi. Louis XIV adore être Roi, et c'est pour ça que ça marche si bien. Cependant, ses successeurs détesteront cela, Louis XV comme Louis XVI. \\
Louis XVI est le petit fils de Louis XV, c'est même le cadet. Il n'était pas destiné à être Roi, il avait un frère aîné qui selon les dire, était fait pour être Roi, mais il est mort. \\
Louis XVI est quelqu'un d'extrêmement intelligent et cultivé car il a reçu l'éducation d'un prince. C'est un personnage qui a été élevé dans une grande piété, c'est un prince chrétien. Il est élevé que tout roi qu'il est il sera jugé par Dieu. \\
Quand Louis XVI devient Roi, il est présenté comme un nouvel Henri IV mais Louis XVI a un gros défaut, il manque de volonté, il est indécis en permanence, il est hésitant et il incarne en lui même tout les pouvoirs, ce qui est un problème, c'est une faiblesse. 









\end{document}
