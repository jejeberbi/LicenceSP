\documentclass[10pt, a4paper, openany]{book}

\usepackage[utf8x]{inputenc}
\usepackage[T1]{fontenc}
\usepackage[francais]{babel}
\usepackage{bookman}
\usepackage{fullpage}
\setlength{\parskip}{5px}
\date{}
\title{Cours d'histoire politique et sociale de la France depuis 1789 (UFR Amiens)}
\pagestyle{plain}



\begin{document}
\maketitle
\tableofcontents

% Antonetti - Manuel


\chapter{Introduction}

On va chercher à comprendre ce qu'est la société Française en 1789, juste avant la révolution Française qui balayera la monarchie, construite depuis 1000 ans. On cherche à comprendre pourquoi il y a eu une Révolution. \\
La France est coutumière de ce genre d'événements, il se passe des événements avant 1789, des révoltes, des soulèvements, liés aux impôts trop élevés ou à la faim. La Révolution est quelque chose de bien plus profond, il suppose que ceux qui la font souhaitent changer le système. Quand, au moyen-âge, les paysans se soulèvent, ils font une fronde car ils ne souhaitent pas changer le système, ils ne veulent et ne peuvent pas prendre la place du seigneur. \\
La révolution, pour être faite, a donc besoin d'une idée intellectuelle suffisamment mature pour substituer le système à un autre.

\section{L'état de la France en 1789}

Le premier élément saillant est la population, 30 millions de français en 1789 font de la France le Royaume le plus peuplé d'Europe. Cette population permet à la France d'être un État riche, il a une forte capacité fiscale, mais aussi une forte capacité militaire en cas de guerre. \\
C'est donc un royaume très puissant, d'autant plus que l'économie est florissante car le commerce est important, notamment le commerce colonial et triangulaire. L'économie française repose néanmoins essentiellement sur l'agriculture, qui est aussi florissant. Les français du XVIIIe investissent en grande partie dans la terre par opposition aux anglais qui investissent dans l'industrie. \\
Quand un pays est riche grâce à son agriculture, au moindre événement climatique, les récoltes s'effondrent et donc l'économie aussi. L'économie agricole est donc très fragile. 


Au niveau politique, il y a la monarchie absolue de droit divin. Quand on dit que le Roi gouverne seul car Dieu lui a donné ce pouvoir, cela suppose de croire en Dieu pour les français. Cela se passe très bien durant tout le Moyen-âge, mais, à la fin du XVIIe siècle, on commence à remettre en question la religion et même l'existence de Dieu. Ce qui conduit donc à douter de la nature divine du pouvoir.


La société est composée d'ordres, les trois ordres que sont le Clergé (Auratores, ceux qui prient), la Noblesse (Bellatores, ceux qui se battent) et le Tiers État (Laboratores, labor, le travail). Cette société d'ordres est apparu au Moyen-âge où elle se justifiait totalement. Au Moyen-âge, les seigneurs prélèvent l'impôt, et n'en payent pas car ils payent "le prix du sang", car ils se battent constamment. Quant au Clergé, sa position se justifie car la société est convaincue de la religion chrétienne et le clergé prie donc pour le salut de tous les hommes, et est considéré donc comme rendant un service à la société et donc, ne paye pas d'impôts. Tous les autres, le Tiers État, payent des impôts car ils ne prient pas et ne se battent pas.


La société d'ordres commence à ne plus se justifier car la noblesse ne fait plus la guerre et le clergé est remis en question. D'autant que cette société d'ordres est une société profondément inégalitaire, pas seulement entre les ordres mais à l'intérieur des ordres aussi. \\
Dans le clergé, on va distinguer le haut clergé et le bas clergé. Le haut clergé, ce sont les évêques et les abbés. Ceux-ci sont nommés par le Roi depuis 1516 et va, au XVIIIe, désigné des nobles qui n'ont pas forcément la foie et ne respectent pas le voeu de célibat, voire même, ne vivent pas dans leurs abbayes. Ce haut clergé vit dans un luxe très important. Le bas clergé, lui, est constitué de prêtres et de moines et vivent "chichement" et presque misérablement, ils vivent de l'aumône et des maigres terres qu'ils exploitent. \\
Ces inégalités se sont longtemps justifiés par l'éducation car on suppose que la noblesse est plus éduqué. Cependant, au XVIIIe, le tiers état est de plus en plus instruit. 


Concernant la noblesse, tous les nobles ne sont pas égaux et on distingue la haute noblesse de la petite noblesse. La haute noblesse, c'est une poignée de personnes, ce sont les princes, les ducs, certains marquis, certains comtes. Ce sont des aristocrates (vivant à proximité du Roi) très riches, en terre comme en argent. Cette noblesse forme la Cour, principalement basé à Versailles. \\
La petite noblesse est une noblesse de terre, de campagne. Cette noblesse est la noblesse qui a peu d'argent et survit uniquement grâce à ses privilèges fiscaux. C'est une noblesse qui ne peut pas voir le Roi.


Pour le tiers État, c'est l'ordre inégalitaire par excellence, et sont criantes à l'intérieur de cet ordre. Sur 30 millions de Français, très peu sont privilégiés. \\
Tout en haut du tiers état, on a la haute bourgeoisie qui s'est enrichi par le commerce, qui vit dans des hôtels particuliers, qui fait construire des châteaux, qui essayent de vivre comme des nobles. C'est une bourgeoisie tellement riche qu'elle prête de l'argent au Roi. C'est une bourgeoisie très éduquée, qui fréquente les meilleurs écoles. \\
Plus bas, il y a la moyenne bourgeoisie des villes. Ce sont des commerçants aisés ou des agriculteurs aisés, des gens qui ont des manufactures. C'est la bourgeoisie des juristes, les avocats, les magistrats etc. Ils ont de l'argent et en prête aussi au Roi, moins, certes. Cette moyenne bourgeoisie est aussi très éduquée. \\
Plus bas, il y a une basse bourgeoisie qui constitue la majorité des artisans, cordonniers, bouchers, agriculteurs etc. \\
Tout en bas, c'est le monde des domestiques, des ouvriers agricoles, des ouvriers de manufactures, des employés en général. Ils gagnent peu et payent en plus de l'impôt. 


Nous avons à faire à une société de corps. C'est à dire qu'un individu, seul, n'est rien. Un individu c'est avant tout le membre d'un corps, d'un groupe. Le premier corps auquel appartient un individu est la famille. Famille qui est sacrée et dirigée par le père, qui est le chef de famille et qui a la capacité de décidé de beaucoup de chose (mariage arrangé entre autres), le père règne en maître dans sa famille. \\
Le deuxième corps auquel on appartient est la paroisse. L'individu appartient à la paroisse car il n'y a aucune diversité religieuse. \\
Ensuite, les individus intègres des corps de métiers. Corporation de métiers qui ont des maîtres. \\
Il y a aussi la communauté d'habitant, le village ou la ville dirigé par un maire qu'on appelle mayor. \\
On appartient aussi à une province. Cette province est dirigée par un gouverneur. Au dessus de la province, il y a un plus grand corps, celui de son ordre. \\
L'individu appartient aussi au Royaume mais quand on pousse encore la réflexion, l'individu appartient à la société des chrétiens. \\
Avec ce découpage, on se rend compte que tout tourne autour des chefs et de Dieu. L'individu n'existe donc pas, il a le nom de son père. Cette société sera donc ébranlée quand on commencera à penser que l'individu existe, voire même qu'il a des droits en tant qu'individu en tant que tel.

\subsection{La place du Royaume de France en 1789}

L'Europe, en 1789, c'est en grande partie des monarchies absolus. Le régime français n'est donc pas isolé. On considère que l'absolutisme est le régime qui est le plus adapté aux vastes territoires. L'Espagne, le Portugal, les deux Sicile, l'Autriche, la Russie sont les grandes puissances d'Europe monarchistes et absolus. \\
Il y a un ensemble de petites Républiques, ce sont généralement des villes comme Venise, Genève etc. et sont plutôt des oligarchies malgré leur nom. On pense que les Républiques sont valable que pour les petits territoires, notamment les villes. \\
Il y a deux cas à part, le Royaume-Uni qui, depuis la fin du Moyen-âge, est dans une monarchie parlementaire ; la Pologne, vaste monarchie d'Europe centrale, a une particularité: c'est une monarchie élective. Les aristocrates désignent un Roi, et ont déjà élus un prince étranger: Henri III. \\
La France a toujours eu une relation particulière avec la Pologne. Depuis Henri III, le France protège la Pologne. \\
Le Roi de France, est, en 1789, en position d'arbitre au sein de l'Europe car il est un des plus puissants, ce qui s'explique par la monarchie absolu, contrairement au Royaume-Uni, et l'idée nationale est très forte. La France a aussi des alliances clés, des alliés traditionnels: les cousins du Roi de France, les Bourbons, dont l'un est Roi d'Espagne, et l'autre Roi des deux Sicile. Autre allié traditionnel de la France: la Russie. Depuis 1750-1760, un nouvel allié, qui a longtemps été l'ennemi, est l'Autriche. Ce réseau d'alliance est fait pour fonctionner contre l'Angleterre. \\
La France fait jouer ces alliances pour protéger la Pologne car les voisins polonais sont menacés par la Prusse. 


Dans ce réseau d'alliance, on comprend fondamentalement que deux pays ont tout intérêt à ce que le géant français soit affaibli, l'Angleterre et la Prusse. \\
L'Europe des monarchies n'a donc pas marchés, les autres Rois n'ont pas aidés Louis XVI car il était dans l'intérêt des autres monarchies que la France s'affaiblisse. \\
Les États-Unis sont une quelque sorte de modèle, car ils ont obtenu leur indépendance contre un Roi, à l'aide d'un roi, Louis XVI. Le modèle américain va donc beaucoup jouer en France. 

\subsection{Les institution de la France en 1789}

La France est une monarchie absolue de droit divin, avec cette idée que le Roi représente Dieu sur Terre avec un pouvoir qui est théoriquement illimité. En pratique, il y a des lois fondamentales qui encadrent le pouvoir du Roi et surtout des limites pratiques à l'autorité du Roi. \\
La France est un pays immense, très vaste, et l'information circule lentement, comme les ordres. \\
Le Gouvernement qui existe en 1789 est un gouvernement bureaucrate. C'est quatre secrétaires d'États, un contrôleur général des finances et un chancelier pour la justice. Ces personnes sont censés donnés tous les ordres en France, sous ordre du Roi. Tout part du Roi et tout aboutit au Roi. En pratique, le Roi ne peut pas décider de tout, donc il va se faire aider par des fonctionnaires, des bureaucrates, qui vont l'aider à gouverner. Le Roi ne décide que des grandes orientations alors que dans la fiction juridique, il est censé décidé de tout. \\
Ce gouvernement va être de plus en plus critiqué et notamment un de ses instruments, les lettres de cachets. Ce sont des lettres écrites par le Roi qui ordonne à telle ou telle personne de faire une action. La lettre de cachet est un instrument juridique. Il en faut donc des centaines par jour pour faire fonctionner un royaume de 30 millions d'habitants. C'est le secrétaire de la main qui était le seul à pouvoir imiter la signature du Roi et qui faisait des lettres pour les ministres. \\
Quand on va découvrir que le Roi ne décide plus de grand chose, on va critiquer le système quand on va comprendre qu'il est devenu entièrement bureaucratique. 


Le Gouvernement est centralisateur. Le Royaume est divisé en 33 intendances, chaque intendance est géré par un intendant qui est un représentant du Roi. Les intendances sont elle même divisées en subdélégation à la tête desquelles on trouve un subdélégué. Ceux-ci aidaient aussi à faire remonter l'information. \\
La centralisation est la grande marque de fabrique de la royauté Française. Les Rois font tout pour centraliser leur Royaume. Ils font cela car la France est une collection de provinces et donc le roi essaye d'empêcher que ces provinces se détachent. En 1789, les cultures sont certes différentes, mais tout le monde se voit comme sujet du Roi de France. Cette cohésion nationale va jouer un rôle important pendant la Révolution, car supprimé le Roi, c'est prendre le risque que des provinces décident de réclamer leur indépendance.


\section{Les faiblesses et crises de la monarchie}

\subsection{Les faiblesses}

La première faiblesse du Royaume de France est Louis XVI. Louis XIV a construit la monarchie absolue française autour de lui, il a construit un pouvoir très fort concentré autour de lui. Tout le régime est basé autour de cette figure du Roi. Louis XIV adore être Roi, et c'est pour ça que ça marche si bien. Cependant, ses successeurs détesteront cela, Louis XV comme Louis XVI. \\
Louis XVI est le petit fils de Louis XV, c'est même le cadet. Il n'était pas destiné à être Roi, il avait un frère aîné qui selon les dire, était fait pour être Roi, mais il est mort. \\
Louis XVI est quelqu'un d'extrêmement intelligent et cultivé car il a reçu l'éducation d'un prince. C'est un personnage qui a été élevé dans une grande piété, c'est un prince chrétien. Il est élevé que tout roi qu'il est il sera jugé par Dieu. \\
Quand Louis XVI devient Roi, il est présenté comme un nouvel Henri IV mais Louis XVI a un gros défaut, il manque de volonté, il est indécis en permanence, il est hésitant et il incarne en lui même tout les pouvoirs, ce qui est un problème, c'est une faiblesse. \\
Il est à noté que Louis XVI est une personne qui n'a jamais voyagé, il est certes cultivé, mais il n'a jamais visité le pays. Il ira seulement une fois en Normandie. \\
Louis XVI est un personnage qui est très attaché à la France, probablement plus que Louis XV qui était détaché. Louis XVI veut plaire. Il est extrêmement généreux, et remplace même son épouse quand il s'agit de faire l'aumône. C'est pour ça que son surnom "Le nouvel Henry IV" lui est donné. \\
C'est aussi quelqu'un qui s'intéresse à la modernité, il lit l'encyclopédie, demande que le premier vol en montgolfière parte du château de Versailles, est passionné de serrurerie (qui est une technologie très pointue à l'époque). \\
C'est donc un Roi capable et pleins de volontés. Cependant, la France est un pays à réformé entièrement et il se sent seul car il est souverain absolu et est mal conseillé. 


La France de 1789 est un pays qui n'a plus de capitaine. C'est sa première faiblesse. \\
Sa deuxième faiblesse est Marie Antoinette, qui jouera un rôle politique en 1789. C'est une écervelée qui ne pense qu'à jouir de sa vie de Reine. C'est la dernière fille d'une fratrie d'une dizaine d'enfants, donc jamais elle ne s'imaginait être reine, elle n'a donc reçu aucune éducation. Lorsqu'elle arrive en France, elle a tout pour plaire, elle est jeune, elle est belle, elle est extrêmement populaire. \\
Louis XVI et Marie Antoinette sont un couple qui forme une faiblesse, ils n'auront pas d'enfants pendant 7 ans. Cela lui donnera l'air ridicule dans cette société du XVIIIe siècle. Ils auront d'abord une fille, un fils, et deux autres enfants dont la dernière mourra à la naissance. Il est à noter que le Dauphin est malade, il va mourir pendant la tenue des États-Généraux. \\
De nombreux scandales vont jalonner le couple Royal. L'attitude de la Reine en est le premier, elle est sensé donner un héritier et passer sa vie à aider les pauvres, elle fait tout, sauf ça. Elle cherche quelque chose de très nouveau, de l'intimité. Elle va même aller vivre autre part qu'à Versailles, et une autorisation est requise pour la voir, ce qui n'avait jamais été vu avant. Elle va même se faire construire un faux village normand dans laquelle elle va vivre habillée en bergère. \\
Comme rien ne peut être reproché à Louis XVI, on va voir de nombreuses accusations se porté contre la reine, dont une grande partie sont mensongers. On dira d'elle qu'elle avait des amants, qu'elle se moquait du peuple, qu'elle vivait une vie de débauche. Or, critiquer la reine, c'est aussi critiqué la monarchie. Il faut rajouter le fait qu'ils sont mal conseillés. Il n'y a pas un Colbert pour le règne de Louis XVI. Il n'y a pas de génie du Gouvernement.


Autre faiblesse, il y a un manque d'uniformité. La France est un agglomérat de provinces rattachés au Royaume, c'est un ensemble territorial assez artificiel. Ces provinces ont été généralement rattachés par des voies de droit, il n'y a pas eu de reconquête militaire. Pour intégrer ces provinces, les Rois se sont engagés à respecter les privilèges de chaque province, comme par exemple, la coutume de la province, son propre système juridique. Ce manque d'uniformité rend le royaume difficile à gouverner. \\
On ne parle pas la même langue, les impôts ne sont pas les mêmes, on ne mesure pas les choses de la même manière. Il y a même des douanes intérieures, entre les provinces. 


Le statut de la fonction publique est une autre faiblesse. À l'époque, on vend les charges publiques. Elles sont héréditaires. Ce sont naturellement les plus riches qui achètent ces fonctions et notamment la bourgeoisie. Quand quelqu'un est propriétaire d'une fonction, la monarchie ne peut pas reprendre cette fonction, sauf à la racheter. Or, la monarchie n'a pas d'argent. La royauté a donc de moins en moins de prise sur ceux qui achètent des offices, donc sur la bourgeoisie. \\
Cette bourgeoisie, c'est une bourgeoisie qui est en permanence humiliée par la noblesse, qui accapare les plus hautes charges de l'État parce qu'elle est noble. L'aristocratie a le monopole des plus hautes charges de l'État alors qu'en réalité, la bourgeoisie est mieux préparé pour ces charges car elle est mieux instruite que ces nobles. \\
Les bourgeois détenant des offices sont anoblis au bout de quelques générations, ceux ci sont dénigrés en étant appelés "noblesse de robe", faisant opposition à la "noblesse d'épée".


Les parlements sont des tribunaux supérieurs, ils sont une douzaine dont le plus prestigieux est à Paris. Ils sont composés de hauts magistrats qui achètent leur charge et sont donc de la très haute bourgeoisie, provinciale ou parisienne. \\
Ils ont aussi une fonction bien particulière, ils enregistrent les lois du Roi pour les appliquer. À l'occasion de cet enregistrement, ils peuvent montrer au Roi que la Loi est mauvaise, ce sont des remontrances. Ces remontrances seront systématique à partir du milieu du XIXe siècle. Ces remontrances vont paralyser le pouvoir législatif du Roi. \\
Ces remontrances, à l'origine, sont secrètes. À partir du milieu du XVIIIe siècle, les parlementaires vont prendre l'habitude de publier leurs remontrances. \\
Dans les remontrances, les parlements vont développer des théories nouvelles. Ils vont construire des théories politiques autour de leurs remontrances. Ils vont notamment avancer l'idée que les parlements seraient antérieures à la royauté et vont prétendre être les représentants de la Nation. 


\subsection{Les crises de la monarchie}

\subsubsection{Crise intellectuelle et morale}

Les philosophes des lumières ont évidemment préparés la révolution. Le plus important est "Montesquieu" pour avoir écrit "L'esprit des lois". Il est important car il offre un modèle institutionnel et politique, c'est le modèle de la séparation des pouvoirs. Pour lui, le pouvoir législatif appartient aux représentants du peuple. \\
Voltaire est aussi important car il va radicaliser la remise en cause de l'absolutisme. Pour lui, l'absolutisme fait craindre les caprices du souverain, il voudrait une monarchie, certes, mais tempéré par le conseil des philosophes. Voltaire dénonce aussi le fanatisme religieux à l'occasion de grandes affaires du droit pénal: l'affaire Khalas notamment, qui met en cause un protestant, accusé du meurtre de son fils par une opinion publique catholique. Voltaire dénonce les excès de l'Église et du clergé. Il aura une large audience, on le lit beaucoup, on le commente beaucoup. Quand il dénonce ces excès là, Voltaire s'attaque à un des piliers du trône. \\
Rousseau a aussi été un chantre de la philosophie. Il propose de refondre entièrement la société. Il part de l'État de nature pour arrivé à la société politisé. Pour lui, la société politique idéale, c'est une société où triomphe la liberté et l'égalité civile. Il alimente la pensée des révolutionnaires les plus excessifs. \\
Diderot est l'un des créateurs de l'encyclopédie. Il permet de faire un synthèse des connaissances techniques de l'humanité. Ceux qui écrivent dedans sont des philosophes, et qui sont convaincus que l'humanité est en constant progrès, qu'il faut être tolérant, qu'il faut s'ouvrir au maximum pour pouvoir progresser. Un mot qui revient tout le temps est la "raison", qui revient tout le temps. Celle-ci s'oppose à la foie, à la croyance. Il sera notamment dit dans l'encyclopédie que le pouvoir est donné par les hommes. Cette encyclopédie sera interdite de publication en France mais sera vendu en contrebande en France. Il est à noté que même le Roi et ses censeurs la lisait. \\
Ce sont essentiellement les bourgeois qui lisent ces oeuvres. Ils organisent d'ailleurs des salons, des réunions mondaines au cours desquelles on invite un philosophe ou l'élève d'un philosophe, ou encore des écrivains qui ont des messages à faire passer. C'est ce qui fait progresser ces idées. \\
Cette philosophie va proposer un véritable programme politique et va donner des arguments politiques aux futurs révolutionnaires. 

\subsubsection{La crise économique et financière}

Le règne de Louis XV a été un règne relativement prospère, les récoltes sont globalement bonnes. Celui de Louis XVI, au contraire, est marqué par plusieurs crises économiques. Le Royaume est frappé par une inflation très forte à partir de 1776. Celle-ci va toucher toute la population et va impacter surtout les plus riches, et notamment la noblesse et va la mettre dans une situation très périlleuse. Un noble ne peut pas travailler, il ne peut pas pratiquer le commerce, ça lui est interdit. La seule chose qu'il peut faire, c'est se battre. Si il déroge à ces règles, ils perdent leur noblesse. \\
Quand l'inflation les frappes, ils vont augmenter les impôts seigneuriaux. À partir des années 1776-1777, on voit une augmentation des taxes seigneuriales. Ils augmentent aussi les loyers. \\
En 1775 et en 1778, il y a un climat particulière chaud l'été et l'automne. Il y aura donc une production très importante de raisin et de vin, dû aux automnes chauds. Mais comme la France est en guerre contre les anglais, le vin ne se vend pas et les viticulteurs sont ruinés. \\
En 1785, c'est un peu le contraire qui se passe. On a ce coup ci, une sécheresse pendant l'été. Les prairies seront donc roussis, le fourrage sera donc impossible et les éleveurs auront du mal à nourrir leurs bêtes. \\
En 1788 et en 1789, pendant les deux étés, il y a énormément plût. Les blés ont pourri sur pied, et cela a provoqué une certaine famine. \\
Quand les campagnes sont en crise, les ruraux non propriétaires vont en ville. Il va donc y avoir une augmentation du nombre d'habitants dans les villes qui reçoivent la misère agricole. Les villes étant fortifiés, les populations "misérables" vont s'installer à la périphérie des villes: les faubourgs. La conséquence est l'augmentation du chômage dans les villes. Ces populations vont rester solidaires avec celles des campagnes. 


Ceux qui vont s'en sortir sont les propriétaires de terre assez conséquentes. \\
Le plus gros propriétaire terrien est l'Église. Elle est aussi riche car existe depuis le Moyen-âge, et les propriétaires ont fait des dons à l'Église pour qu'ils prient pour l'éternité pour ceux-ci. Les abbayes seront donc très riches, vu qu'elles ne servent qu'à ça. \\
Conjugué au fait que l'Église est une institution qui se veut éternelle, elle devient le plus gros propriétaire terrien. \\
Les aristocrates sont les deuxièmes propriétaires. Seuls les aînés héritent: des titres et des terres. \\
À noter que ce sont les deux groupes privilégiés, les crises vont donc creuser encore plus le fossé entre ces deux groupes et le tiers État. 


De cette crise économique va découler une crise financière, liée au déficit des finances royales. Les finances royales sont en déficit depuis la fin du règne de Louis XIV. Louis XIV et Louis XV ont fait des guerres modernes où l'armement et les soldats coûtent cher. Or le Roi n'a pas augmenté ses ressources . Il a bien créé de nouveaux impôts mais l'État n'a pas les moyens humains et financiers de lever ses impôts. \\
Quand Louis XVI prend les reines du Royaume, il est déjà en déficit. Cela va déclencher une inflation. En bref, l'État est en détresse financière qui va encore s'accentuer d'une manière dramatique avec la guerre d'indépendance américaine. 


\subsubsection{La crise politique}

La crise politique est lié aux parlements. Ils constituent une élite économique, intellectuelle, corporatiste, et parfaitement capable d'exercer les responsabilités politiques. Cette élite pourrait gouverner le pays. Ils ont en plus, le pouvoir de bloquer l'action du pouvoir royal par l'opposition parlementaire avec l'exercice du droit de remontrance. C'est aussi une élite qui a construit un discours politique cohérent et élaboré même si les bases historiques sont fausses. \\
Leur discours va se résumer à dire qu'il n'y a plus d'états généraux réunis, de fait, les parlements vont représenter le peuple à la place des états généraux. Ils le prétendent et ils vont l'affirmer. Ils vont faire passer ce discours au sein de la population, d'abord de la moyenne bourgeoisie puis la petite bourgeoisie puis par répercussion, la petite bourgeoisie va transmettre le discours aux gens les moins aisés. \\
En 1789, il y a beaucoup de parisiens qui adhèrent à ce discours. La royauté laisse faire par respect du système traditionnel. Louis XV va très longtemps laissé faire jusqu'à ce qu'il veuille réformer le royaume dans le sens d'une plus grande égalité fiscale et qu'il en est empêché par les parlements. Il a voulu taxé le clergé et la noblesse. En 1770, Louis XV supprime les parlements et lancera une série de réformes. Mais il sera terrassé par une maladie en 1774. Louis XVI rétablira les parlements en 1774 sous les mauvais conseils d'un de ses ministres. Pendant tout son règne, Louis XVI sera toujours contesté par les parlements. \\
À la mort de Louis XV, le ministre des finances est en place est un génie de la finance, l'abbé Terray. Quand Louis XV meurt, il va voir Louis XVI et lui dit qu'il faut déclarer la banqueroute (renoncer à payer les créanciers). En 1776, le déficit est de 22 Millions de libre et 74 millions de livre dépensé par anticipation. Cependant, Louis XVI, qui adore la popularité va rejeter la proposition par peur de mécontenter les créanciers et va donc nommé Turgo nouveau ministre des finances. Il va avoir une politique de tentative de réforme structurelle. La réforme structurelle signifie à cette époque, donner plus de libertés économiques. Il va par exemple déclaré la liberté du commerce des grains. Cela va déclarer toute une suite d'émeutes: la guerre des farines. \\
Il préconise aussi de libéraliser les corporations de métiers, jusque là aux mains des maîtres, mais échoue. Louis XVI qu'on présente comme un souverain absolu n'a pas assez de pouvoir absolu pour imposer ces réformes. \\
Turgo a deux remèdes pour sauver les finances de l'État: réduire les dépenses, il réduira notamment les dépenses de Versailles. Deuxième chose, il proposera d'augmenter le rendement de l'impôt en essayant d'imposer petit à petit une sorte d'égalité fiscale. Il supprimera par exemple la corvée royale et la remplace par un impôt: cet impôt est du par tous les propriétaires, y compris la noblesse et le clergé. "Tout système qui tendrait à établir entre les hommes une égalité de devoir et à détruire les distinctions nécessaires amènerait le désordre, suite inévitable de l'égalité absolue et produirait le renversement de la société", c'est ce qu'écrivent les parlements. \\
Louis XVI va vouloir imposer les édits par voie de justice donc par voie d'autorité mais Turgo va se montrer terriblement maladroit et dira en public "N'oubliez pas Sire, que c'est la faiblesse qui a mis la tête de Charles 1er sur le billot". Louis XVI, vexé, renvoie le 12 Mai 1776 Turgo. Il engagera donc Necker en ministre des finances. 


Necker est l'homme à la mode à l'époque. Il est connu des milieux bourgeois parisiens, il a la cote. Il est à noter qu'il est Genevois. Il est éclairé, a des connaissance en économie et finance. Il est nommé en pleine guerre d'indépendance américaine. Il est donc confronté à des dépenses considérables auxquelles l'État doit faire face. Il emprunte donc de l'argent, énormément. Le total des emprunts en 1781 s'élève à 450 millions de livre d'emprunts. Le bruit court que les caisses du Roi sont vides et donc Necker essaiera de ramener la confiance des créanciers. Necker publiera donc les comptes de la Monarchie, ce qui n'avait jamais été fait avant. \\
Dans ce qu'il a publié, on voit que les recettes s'élèvent à 264 Millions de livre, et les dépenses s'élèvent à 254 millions de livre. Necker a oublié, sciemment, de donner les dépenses extra ordinaires: les emprunts et les dépenses de la guerre d'indépendance. Si on les rajoute, on sait que le déficit est en réalité de 90 millions de livre. Il est à noter que les dépenses de la Cour sont dans ce document, jusqu'aux détails des dépenses de la Reine. Louis XVI ne supportera pas le détail du compte rendu et renverra Necker le 19 Mai 1781. 


Louis XVI nommera Calonne, le dernier atout de Louis XVI. C'est un officier de justice, originaire d'Arras. C'est quelqu'un de très intelligent, et sa grande qualité est qu'il connaît parfaitement la France car il a été intendant, et surtout, il a un esprit de réforme. Avant les réformes structurels qu'il voudrait lancer, il va emprunter 487 millions de livre pour financer la guerre américaine et lancer ses réformes structurelles. \\
Il propose un plan de réforme au droit. Un plan extrêmement ambitieux, où on réforme tout, c'est une refonte totale, qui passera obligatoire par une réduction du rôle politique du clergé et de la noblesse. Il souhaite l'égalité fiscale. Calonne étant un officier de justice, il sait que les parlements s'opposeront. Il demande donc au Roi de passer outre les parlements. Il utilisera, pour faire la réforme, une veille institution qui est l'assemblée des notables et qui a le pouvoir d'enregistrer les lois du Roi. Cette assemblée est réuni le 22 Février 1787 à Versailles. \\
Cette assemblée est une assemblée des notables les plus connus. Tout en haut, on trouve les 7 princes du sang, ensuite, 14 prélats, 36 nobles d'épée de la plus fine aristocratie, 33 parlementaires, 12 députés des états provinciaux, 25 maires des plus grandes villes de France. Il n'y a que des privilégiés dans cette assemblée. Ce sont 144 privilégiés devant lesquels Calonne va faire le procès des privilèges. \\
Cette assemblée acceptera quelques réformes. Mais quand vient la question de l'égalité fiscale, l'assemblée des notables la refuse, à un tel point que Calonne démissionnera. L'assemblée des notables dira au Roi que la seule solution pour réformé en profondeur, c'est de convoquer les états généraux. \\
Dans un premier temps, Louis XVI refuse les états généraux car il sait qu'il peut se retrouver avec une assemblée qui prétendra légiférer à sa place. Il va donc tenter d'imposer de force les nouveaux impôts. Le parlement refuse, Louis XVI exile des parlementaires, le Roi est bien obligé de les rappeler et le parlement accepte un nouvel emprunt mais l'opposition reprend de plus belle en 1788. Les parlements inondent de remontrance le pouvoir royal, parfois très violente. Le mouvement s'accentue lorsque la noblesse et le clergé s'agitent à leur tour. Tout le monde conteste l'autorité du Roi contre l'égalité fiscale. Ceci est la pré révolution aristocratique, un moment très bref où dans l'histoire de France, le clergé, la noblesse, et les parlements s'insurgent contre l'autorité royale. \\
Louis XVI est complètement perdu, et se décide à supprimer les parlements grâce à l'amoignon et crée une cour plénière pour l'enregistrement des textes en 1788. Quand Louis XVI fait cela, c'est la levée de bouclier général, tout le monde va s'insurger. L'assemblée du clergé de France s'insurge et réclame la suppression des édits de 1788, la noblesse rentre dans ses provinces où elle commence à s'agiter. \\
Le peuple soutient les parlements qui prétendent protéger le peuple parce que les parlements passent pour l'ultime rempart contre le despotisme. C'est là le grand paradoxe de cette pré révolution. \\
C'est dans ce contexte que Louis XVI convoque les états généraux et les convoque pour le 1er Mai 1789. Louis XVI va rappeler Necker et va rétablir les Parlements. 


\part{La Révolution de 1789}

Louis XVI compte sur l'appuie du tiers État contre le clergé et la noblesse. Mais il ne veut pas pour autant remettre en question la monarchie hérité de ses pères. Il ne veut surtout pas revenir sur l'héritage de Louis XIV, l'absolutisme. Louis XVI a promis de respecter la monarchie traditionnelle lors de son sacre. \\
Il n'est pas question non plus de remettre en question les ordres. On peut avoir l'impression parfois, qu'il change d'avis, mais on ne peut pas savoir ce qu'il pensait vraiment. \\
Les États généraux s'ouvrent le 5 Mai 1789 et il faudra un été pour que le trône s'effondre. Le Roi va perdre sa souveraineté et la révolution va se faire sans vraiment trop de débordements, jusqu'en 1792. Cette période, jusqu'en 1792, s'appelle la première Révolution, qui sera suivi d'un durcissement, causé par le comportement de Louis XVI notamment qui n'accepte pas de devenir un Roi citoyen contrairement à ce que certains pensent.

\chapter{La première Révolution}

Au printemps 1789, la première force politique du Royaume, c'est le Roi, son Gouvernement, ses officiers et l'armée bien sûr. Il a tous les pouvoirs. Il veut réformer l'appareil administratif du royaume en ouvrant les États généraux pour le rendre plus cohérent et plus efficace. Il veut moderniser l'État et veut pour cela, trouver de nouvelles ressources financières, en particulier dans l'égalité fiscale. \\
La deuxième force politique sont les ordres privilégiés: le clergé et la noblesse qui veulent préserver leurs privilèges, le maintien des ordres. Mais ils veulent encore plus: participer au pouvoir politique. \\
La troisième force politique sont les "patriotes". Ce sont en grande partie des gens du tiers État, ils veulent détruire l'ordre socio-politique ancien, ainsi que l'absolutisme, au nom de la raison et veulent construire un nouvel ordre socio-politique. Ils veulent construire une société basé sur trois principes: souveraineté nationale, liberté individuelle, l'égalité civile. Ces principes circulent lors des élections pour les états généraux. 

\section{La chute de l'Ancien Régime}

\subsection{Les événements}

\subsubsection{La Révolution juridique}

En Mai 1789, on va inaugurer les états généraux qu'on avait pas réuni depuis le 17e siècle. Les députés des états généraux sont des personnes élues et sont aussi assez représentatifs de la société de l'époque. Ils sont élus par ordre et par bailliage. Le bailliage est une circonscription judiciaire et dans ce bailliage, chaque ordre va désigner ses propres députés. Les nobles élisent leurs députés, pareil pour les prêtres, et pour les tiers état aussi. Les élections font intervenir tout ceux qui payent l'impôt. L'assise électorale est donc déjà assez étendu. \\
Les élections se font dans un climat assez libéral, elles sont plus ou moins libre. On peut faire campagne. Tout ça augmente encore la légitimité des députés. Il y a des pamphlets qui sont édités pendant les campagnes, ce sont des écrits politiques dont le plus célèbre est "Qu'est-ce que le tiers État", écrit par Sieyès. Il y aura aussi la rédaction de cahiers de doléances, il y en aura un par ordre et par bailliage. Ces cahiers permettent aujourd'hui aux historiens de jauger de l'opinion publique de l'époque. \\
Dans l'ensemble, les opinions exprimées sont assez modérées. Les cahiers sont très respectueux du Roi et de la Monarchie. Ils sont modérées car ont fait l'objet d'une synthèse, les opinions les plus extrêmes ayant dû être sans doute évacuées. 


Il y avait environ 1200 députés pour représenter 30 millions de Français. 290 députés pour le clergé, ce sont des curés essentiellement, et beaucoup sont issus du bas clergé et sont libéraux. On trouve aussi quelques grands prélats, libéraux aussi, dont un en particulier: le diable boiteux, c'est l'évêque d'Autain, Talleyrand-Périgord, c'est un aristocrate qui a été forcé à intégrer les ordres. Il y a aussi des prêtres anti-libéraux, notamment l'abbé Maury. \\
La noblesse compte environ 270 députés où on trouve là aussi de tout. Il y a des représentants de la petite noblesse, ce sont les plus nombreux. C'est cette petite noblesse qui est attaché à ses privilèges. Il y a aussi des aristocrates plus fortunés, la haute noblesse, la noblesse de cour, c'est une noblesse plutôt libéral, ouverte. Elle l'est parce que c'est la mode. Un des grands aristocrates de ce bord est Lafayette. L'aristocratie est, dans son ensemble, conservatrice. \\
Il y a 600 députés du tiers État. Il y en a autant car Louis XVI a accepté de les doublé pour mieux représenter le nombre du tiers état mais aussi pour avoir une majorité contre les deux autres ordres qui résisteraient pour garder leurs privilèges. On trouve parmi ces députés des transfuges des autres ordres, notamment l'abbé Siéyès, ou encore le comte de Mirabeau. Il y a une immense majorité de juristes dans ces députés, il y a des avocats comme Robespierre, Barnave, Mounier, Le Chapelier. Parmi ces 600 députés, ce sont majoritairement des bourgeois, des gens fortunés. \\
Il n'y a pas de révolutionnaires dans cette assemblée. Ce sont des gens posés qui veulent mettre en place des réformes doucement. Ils attendent par dessus tout de pouvoir voter l'impôt. Ils veulent aussi une réunion périodique des états généraux, voire permanente, que ces états généraux puissent voter les impôts, ils voudraient plus de liberté. Les députés du tiers état voudraient donc que les états généraux deviennent un équivalent du parliament à l'anglaise.


Ils arrivent à Versailles fin Avril. Ils arrivent dans une ville qu'ils ne connaissent pas. Ils vont découvrir pour la première fois ce qu'est le château de Versailles. Il y a là un effet psychologique important car Versailles est un lieu de pouvoir impressionnant. Les députés vont pouvoir entrer dans Versailles, tout le monde peut y entrer, sauf les moines. \\
La première séance, inaugurale, sera le 5 Mai 1789. Elle se déroule dans la salle des menus-plaisirs qui est en fait la salle où on stockait les décors de théâtre du château. Celle salle sera décorée pour l'occasion, dans un style néo-classique. \\
La séance inaugurale est très procédurale. Elle est précédée d'une messe solennelle dans la cathédrale saint-louis de Versailles. Tous les députés sont invités à cette messe. Cette messe sonne un peu comme la dernière grande pompe royale. Le cortège du Roi comprend le Roi et la reine, la famille royale, les princes de sang, les prélats, le bas clergé, suivi de la noblesse. Il est imposé aux gens du tiers état de porter l'habit traditionnel noir.


Après la messe, les députés se rendent dans la salle des menus plaisirs où chacun prendra sa place. Le tiers état est expulsé au fond. Les nobles sont aux premiers rangs. En face, le trône de France, précédé des marches pour les princes de sang. \\
L'ambiance est tendue. C'est la première fois que Louis XVI parlera à son "peuple". Il déclarera juste la séance ouverte et laissera parler son garde des sceaux, Barentin, qui est "la bouche du roi". Le discours de celui-ci est plat, décevant, il ne fera qu'une chose bien: rappeler les principes de l'absolutisme. Il dira que la seule tâche des états généraux est de voter de nouveaux impôts. \\
Le discours est aussi dur car le Roi ne s'adresse pas qu'au tiers État, il est difficile de faire un discours qui touche tout le monde. \\
Cependant, ce n'est pas pour le Roi ou son garde des sceaux que les députés sont venus. Ils sont venus pour entendre le discours de Necker de qui ont attend un discours flamboyant, mais sera finalement d'un ennui mortel. Tous les députés seront déçus. \\
La monarchie reste enfermée dans ses cadres traditionnels là où on pensait qu'elle en sortirai. 


Le climat est assez tendu lors des premières séances. Le Roi n'était présent que lors de la première séance inaugurale. Tout est fait pour mettre les députés du tiers État mal à l'aise, pour les remettre à leur place. \\
Les députés du tiers état mais aussi ceux du petit clergé se sentent exclus, car ils ont tous le même costume. \\
Tout semble annoncer des débats houleux, compliqués entre les privilégiés et les non privilégiés. Il y a une réelle opposition entre les deux. C'est un climat qui semble électrique. 


Les premières séances permettent de vérifier les pouvoirs des députés, à la synthèse des cahiers de doléance. Vérifier les pouvoirs car les mandats sont impératifs. Cette vérification se fait ordre par ordre dans leurs propres quartiers. Le tiers État propose de faire cela en commun pour commencer à gommer la division en ordres. \\
La noblesse refuse, car elle est dominée par la petite noblesse. Du côté du clergé, on hésite, car c'est là la tendance inverse. \\
Le 12 Juin, le tiers commence, seul, la vérification de ses pouvoirs. Mais quelques membres du clergé viennent se joindre à eux. Le flot gonfle, quelques aristocrates aussi se joignent à eux. Pendant 5 jours, on voit monter le flot de ceux qui rejoigne le tiers. \\
Cela mène au 17 Juin 1789 où Siéyès propose, avec le soutien des députés du clergé et quelques aristocrates, de se déclaré Assemblée Nationale. Ils estiment représenté la Nation. \\
Cette Assemblée Nationale va commencer par accepter que soient levés les impôts traditionnels et précise que tout nouvel impôt devra obtenir son autorisation. Au plan juridique, l'ancien régime est mort. 

\subsubsection{La Révolution politique}

Louis XVI réagit très mal quand il apprend que l'AN a été déclaré. Quand il apprend que l'AN a osé l'autorisé, lui, à percevoir les impôts, il ne peut pas accepter cela. En effet, cette décision remet en cause l'absolutisme. \\
Face aux événements, Louis XVI ferme la salle des menus plaisirs. Cela est efficace à court terme car les députés se retrouvent à le rue le 20 Juin 1789. Ils iront dans la salle du jeu de paume de Versailles pour continuer à se réunir. Ils vont prêter le serment du jeu de paume. Bailly est président de l'assemblée à ce moment. Ils prêtent le serment de ne jamais se séparé jusqu'à ce que la Constitution du Royaume fut établi. \\
Le 23 Juin, nouvelle assemblée commune dans la salle des menus plaisirs, en présence du Roi, qui est complètement dépassé, meurtri par le deuil et qui a le sentiment de ne plus rien maîtriser. Louis XVI annule toutes les décisions prises par les états généraux, il ordonne à tous les députés de siéger que par ordres et non de manière commune. Il propose un programme de réforme avec trois grandes mesures: vote des impôts, des emprunts et du budget par les états généraux. Il proclame la liberté individuelle et la liberté de la presse et l'abolition de la corvée. Mais cela est trop limité: on abolit pas la féodalité, ni la dîme.


Le tiers état n'est pas content et quand le Roi est parti, et que les députés suivent, les députés du tiers état restent dans la salle. Bailly refusera, la nation ne pouvant recevoir d'ordres. Mirabeau dira "Allez dire à votre maître que nous sommes ici par la volonté du peuple et que l'on nous en arrachera que par la force des baïonnettes". \\
Louis XVI se refusera à les chasser. Le 24 Juin, la majorité du clergé rejoint l'assemblée, le lendemain, une grande partie des députés de la noblesse les rejoignent aussi. L'absolutisme est mort. \\
Le 9 Juillet, l'AN se déclare AN constituante. 

\subsubsection{La Révolution populaire}

Louis XVI n'accepte pas aussi facilement de perdre sa couronne, d'autant qu'on sait dans l'entourage du Roi, que plusieurs personnes ont incités le Roi de réagir militairement. On parle d'un coup d'État militaire du Roi. C'est à ce moment que Marie-Antoinette rentre dans la scène politique pour compenser un Louis XVI complètement dépassé. \\
Comme souvent, Louis XVI va hésiter. Il va ordonner d'installer à la périphérie de la capitale et de Versailles, qu'on fasse venir des régiments de soldats étrangers. Certains historiens pensent qu'il fait ça pour récupérer le pouvoir par la force, d'autres pensent au contraire qu'il fait ça pour garantir l'ordre à Paris qui commence à s'agiter. \\
Le 11 Juillet, Louis XVI commet la plus grande faute politique de son règne. Il va renvoyer Necker en disant qu'il n'a pas su gérer les États généraux. C'est le seul qui a la confiance du tiers État. \\
Aparté: Louis XVI et Marie-Antoinette sont en deuil depuis Juin, le dauphin étant mort. \\
La nouvelle du renvoi de Necker se répand le 12 Juillet, et dans le reste de la France. À ce moment là, un climat de peur va s'installer, selon des raisons diverses. Les bourgeois ont peur de la banqueroute car l'État est endetté auprès d'eux, elle a aussi peur de la réaction des aristocrates, notamment si le prochain ministre en est un, il risquerait de vouloir taxer les bourgeois. Du côté du peuple, il connaît la faim, il a peur de la disette mais aussi de la répression militaire. Ces peurs existent dans les campagnes où on sent un pouvoir qui chancelle. Les militaires ont peur aussi, on les masse autour de Paris, sont commandés par des aristocrates Français mais sont étrangers et se demandent ce qu'on va leur demander. Les aristocrates, la cour, la famille royal ont peur aussi.


Dans ce contexte, il va survenir des premiers affrontements le 13 Juillet entre les soldats étrangers et les parisiens. Les soldats sont rentrés dans Paris dans l'espoir de remettre l'ordre. La vieille ville de Paris est une ville Moyenâgeuse, les rues sont étroites, mal pavées, mal éclairées, il y a quelques grandes places et avenues qu'on essaye d'aménager ; la ville est en plus encerclé d'un immense mur qui ne sont plus des fortifications mais pour percevoir des impôts à l'entré et à la sortie de la ville. \\
Le lendemain, 14 Juillet, les émeutiers grondent et, sans doute manipulés par des hommes politiques divers, ils vont cherchés à s'armés. Ils vont donc se diriger vers un monument emblématique militaire: les invalides. Ils vont en trouver, des fusils notamment, et même un canon, mais il n'y a pas de poudre. Celle-ci est à la Bastille, qui est aussi le symbole de l'absolutisme monarchique à Paris. C'était une ancienne forteresse de défense, datant du moyen-âge, qui s'est transformée en prison lorsqu'elle a perdu son intérêt stratégique. Même si les plus folles rumeurs circulent sur cette bastille, le pouvoir ne pense qu'à la détruire, la bâtiment étant devenu insalubre, ce que les parisiens ne savent pas. \\
Les parisiens se rendent devant la Bastille, en font le siège. Elle tombe assez rapidement. Quand il la prenne, ils vont chercher des centaines de prisonniers, mais il n'y en a que 7 dont un fou, un faussaire, et un libertin. Les émeutiers sont déçus mais vont trouver la poudre pour s'armer. Ils vont arrêter le gouverneur de la Bastille, ils vont l'emmener de force à l'hôtel de ville où il est lynché, c'est un garçon de boucher qui lui coupe la tête avec son couteau de cuisine. La tête sera mise au bout d'une pique qu'on va promener. \\
Le second qui va tomber, c'est le symbole du pouvoir royal, le prévôt des marchands. Les émeutiers vont aller trouver l'intendant de Paris, qui va essayer de s'enfuir, Berthier de Sauvigny, qui est lui aussi massacré. \\
Ces événements vont priver, en très peu de temps, Paris des représentants du pouvoir royal.


Ces événements vont se répandre comme une traînée de poudre. Le 15 Juillet, dans la nuit, le Roi est informé. Le 16 Juillet, Louis XVI recule et va rappeler Necker devant la pression des émeutiers. Un vent de panique souffle à Versailles, les aristocrates quittent le château et vont même quitter la France. Au premier rang d'entre eux, le frère du Roi. Louis XVI et Marie-Antoinette sont seuls face à leur destin. \\
Marie-Antoinette dès ce moment dit qu'il faut s'enfuir, quitter le Royaume pour le reconquérir par les armes. Louis XVI refuse et refuse aussi d'utiliser la force, il ne veut pas tirer sur la foule.


Pendant ce temps, une municipalité est élue à Paris. Bailly est élu Maire de Paris. Cette élection va calmer la capitale car le pouvoir royal n'utilise pas la force et laisse les élections se dérouler. Le Roi, à l'invitation de Bailly, va se rendre à Paris. Il y va le 17 Juillet, il y est reçu en grandes pompes. Il se rend à l'hôtel de ville où il est reçu par Bailly. \\
Bailly va accueillir le Roi sur le perron de l'hôtel de ville et va remettre à Louis XVI une cocarde bleu et rouge. Louis XVI va la lier à sa cocarde blanche. Il est présenté à la nouvelle garde municipale de Paris, c'est la garde nationale, l'équivalent de l'ancienne milice bourgeoise. C'est Lafayette qui dirige cette milice. \\
Même si l'apparence est à la réconciliation, une lettre de Louis XVI est envoyé à son cousin, le Roi d'Espagne où il dit qu'il est prisonnier des émeutiers et signale que tout ce qu'il signera devra être considéré comme étant fais sous la contrainte. 

\subsubsection{La Révolution sociale}

C'est cette Révolution sociale qui va faire que la Révolution parisienne va devenir nationale. Les événements parisiens déclenchent la stupeur dans les provinces. Deux mouvements vont se dérouler en parallèle pendant l'été. Les villes sont les premières à apprendre les nouvelles. \\
Le mouvement municipal débute alors. Dans les plus grandes villes de France, on fait tomber les anciennes municipalités pour en élire de nouvelles. On va former dans toutes ces villes des gardes nationales locales. Ce sont des nouveaux pouvoirs municipaux qui se mettent en place, ils sont élus. Le mouvement est à peu près le même dans toutes les villes. \\
À ce moment là, vers le 15-16 Juillet que les administrations royales vont s'effondrer, en tête de cet effondrement: les intendants. \\
L'incertitude règne, c'est pour cela que les villes se replient sur elles mêmes, surtout celles qui sont fortifiés. Mais ce n'est pas dans les villes le plus dramatique.


Le second mouvement qui se met en place est ce qu'on appelle la Grande Peur. Dans les campagnes, la situation est différentes des villes. Les nouvelles sont déformés, le Roi est presque mort pour les paysans. L'autorité royale n'existe plus dans les campagnes pour faire régner l'ordre, les paysans ont donc peur. C'est l'anarchie, la débandade, et en plus, c'est la disette. \\
Ce mouvement de peur va entraîner des mouvements de brigandage: des fermes vont être pillés, des paysans tués. Les paysans vont donc se révolter, faire leurs propres émeutes. Ils vont se ruer dans les châteaux, les piller, en brûler quelques un, s'en prendre aux nobles locaux. Ils vont chercher les archives seigneuriaux qui fondent les droits seigneuriaux, et vont les détruire car c'est le symbole de la dureté. Ils vont détruire les pigeonniers, les piloris, etc.


L'appareil institutionnel monarchique s'effondre. On y annonce aux députés de l'Assemblée Constituante, qui viennent en majorité des provinces et qui sont les premiers à être directement impactés par la Grande Peur: ils pourraient être dépouillés de leurs bien et perdre le contrôle. \\
Les députés vont donc déclarés, la nuit du 4 au 5 août l'abolition du régime féodal, l'abolition des privilèges, l'abolition de la dîme. Ils le font pour sauver la situation, sous l'effet de la peur. \\
Le décret qui met en oeuvre cette abolition date du 11 août. À la publicité de ces décrets, les paysans sont content mais vont déchanter car le texte fait une distinction entre deux catégories de droits entre ceux qui sont abolis purement et simplement comme le servage ou le droit de chasse et les droits qui sont abolis moyennant rachat: ce sont tous les droits qui pèsent financièrement sur la terre. Les paysans ont le sentiment de s'être fait trahir et vont se désolidariser de la Révolution. \\
Les privilèges sont tout de même abolis, comme la dîme aussi. Mais fondamentalement, il y aura toujours cette épine du régime féodal dans le pied de la constituante. L'assemblée comprend qu'elle doit aller plus loin et proclamera la DDHC le 26 Août 1789. Elle a été écrite dans la précipitation. C'est censé être à la base le préambule de la Constitution. C'est l'archevêque Champion de Cicé qui a écrit la DDHC ; le préambule de la DDHC est écrit par Mounier et Mirabeau. \\
Tous ces grands textes seront décevants car ils n'ont aucune valeur en eux même si le Roi ne les signe pas. Le Roi refuse et rappelle de nouvelles troupes étrangères qu'il va installer autour de Paris. Dans ce contexte, il va organiser un banquet avec les officiers de ces régiments. Lors de ce banquet, la rumeur dit que la famille royale aurait assisté à ce banquet et lors de ce banquet, il paraîtrait que la cocarde tricolore aurait été piétiné. Cela ne peut être que faux mais ces rumeurs, mais surtout la tenue du banquet va galvaniser les populations parisiennes.


Il va y avoir des émeutes de la faim dans Paris, qui sont de plus en plus important et qui va être récupéré là encore. Ces émeutes vont se développer surtout au niveau des halles de Paris. Les poissardes vont se rassembler et vont décider de marcher sur Versailles dans une cohorte exceptionnelle. Ce sont surtout des femmes. Elles réussissent à quitter la ville car les soldats ne tirent pas sur les femmes. Elles sont accompagnés d'un détachement de la garde nationale et notamment Lafayette. \\
Elles arrivent le 5 Octobre devant le Château, qui est fermé, sachant que les poissardes arrivaient. Découvrant la splendeur du château, elles sont à la fois subjuguée et en colère. Lafayette demande à être reçu par le Roi. Le Roi recevra une délégation de femmes qui réclamera du pain, une des femmes s'évanouira. Une fois ranimé, il lui donne du pain, et elle demande de signer les grands textes. \\
Elles ne partiront pas de devant le château. \\
Le 6 Octobre, tôt le mâtin, une grille s'ouvre et les femmes arrivent à pénétrer dans les cours du château. Elles vont monter dans l'escalier de la Reine, dans une colère noire. Cet escalier est protégé par des gardes du corps, celui qui essaye de les bloquer va être massacré. La Reine a eu le temps de s'enfuir, en prenant le passage secret qui mène dans la chambre du Roi. \\
Lafayette et la garde nationale contiendra les femmes dans la cour de marbre. Elles demandent à voir la Reine. Elle ira au balcon, elle a le réflexe de faire une révérence et une femme criera "vive la reine", ce que la foule reprendra. Les poissardes exigent et Lafayette aussi, que le Roi aille à Paris. \\
Ils quittent donc Versailles, ils rentrent à Paris. Pendant le cortège, le Roi est appelé "boulanger", c'est donc bien une révolution de la faim. Ces journées du 5 et 6 Octobre sont donc emblématiques de la Révolution. \\
Ils seront installés aux tuileries qui restera le siège du pouvoir exécutif jusqu'en 1870.


À partir de ce moment, le sens de la Révolution va changé: le Roi n'existe plus, l'assemblée est censée gouverner. Celle-ci va essayer de doter la France d'une Constitution. 

\subsection{Les nouveaux principes}

Ces principes sont issus de deux textes fondamentaux: les décrets des 5 août et 11 août 1789 et dans la DDHC du 26 août 1789.

\subsubsection{La liberté}

Il y a trois libertés consacrés dans les deux textes. La sûreté, qui garantit contre les arrestations et les détentions arbitraires. La liberté d'opinion, elle regroupe la liberté de penser, de parler, d'écrire, d'imprimer. La liberté de conscience (libre de croire ou pas en Dieu ; ce n'est pas la liberté de culte).  \\
Il y a aussi des libertés collectives, les privilèges des provinces sont d'ailleurs aussi aboli avec le décret de 1789 abolissant les privilèges. Les privilèges municipaux sont aussi supprimés. Il y a en remplacement, trois grandes libertés collectives nationales: la première d'entre elle, c'est la liberté suprême de la Nation en en faisant le souverain (souveraineté nationale), c'est l'abandon donc de la souveraineté royale (Rousseau) ; la seconde est la liberté de chaque pouvoir (la séparation des pouvoirs), par rapport aux autres (Montesquieu) ; la troisième liberté collective est le concept de décentralisation administrative (chaque pouvoir local devrait être libre), concrète, cela va amener à la création de 83 départements qui sont gérés par des assemblées élus, la création des districts gérés par des citoyens élus, les municipalités, gérés par des citoyens eux aussi élus. C'est une nouveauté car pendant l'ancien régime, ce sont les intendants du Roi qui dirigeaient ou des aristocraties municipales installés depuis des siècles. \\
Les juges seront désormais élus. Les évêques aussi seront élus, les gradés de l'armée. Tout cela sera une catastrophe.  

\subsubsection{L'Égalité}

L'article 1er de la DDHC donne une grande place à l'égalité. Ce n'est nullement l'égalité sociale, mais l'égalité civile, c'est la négation des privilèges de l'ancien régime. C'est une Révolution juridique. Le droit d'aînesse est supprimé, la noblesse aussi. On adopte pour tout le monde, le même type de mise à mort (décapité), sous l'ancien régime, le roturier était pendu et le noble décapité. C'est ainsi que la Guillotine est inventée. \\
L'égalité de tous est accordée pour l'accès aux emplois publics. Ce principe aura beaucoup de mal à s'appliquer. \\
L'égalité fiscale est contenue dans la DDHC. \\
L'égalité de tous sera aussi déclaré pour l'accès à la citoyenneté. Tout le monde n'était pas régnicole avant la Révolution, ils deviendront tous citoyens après. \\
Pas d'égalité sociale, pas l'égalité politique dans le sens où la participation à la vie politique suppose pour les révolutionnaires une aisance matérielle et une instruction (d'où le suffrage censitaire masculin). Pour être élu, il faut non seulement payer un impôt mais aussi être propriétaire foncier. 

\section{Le nouveau régime}

Ce n'est pas la monarchie qui est supprimé en 1789, mais c'est l'absolutisme. Désormais la monarchie sera encadrée, par une Constitution. Celle-ci renie entièrement l'absolutisme en décrétant la séparation des pouvoirs. C'est la Monarchie Constitutionnelle (1789-1792). C'est un régime qui, dès le départ était voué à l'échec, et qui en plus, sera condamné encore par l'attitude de Louis XVI. 

\subsection{Une Monarchie Constitutionnelle}

Il existait une Constitution coutumière, qui comprenait notamment deux règles: le Roi ne désignait pas son successeur et ne pouvait pas aliéner les territoires de son Royaume.

\subsubsection{La constituante}

9 Juillet 1789: l'assemblée constituante commence son travail. Il faudra du temps pour qu'elle soit écrite. \\
Dans les faits, les textes sont préparés par des comités composés de quelques députés, qui mettent ensuite leurs travaux en discussion dans l'assemblée. Elle vote sur les articles séparément, ou par groupes d'articles. Ce qui est particulier, c'est qu'une fois un article voté, on la met en application de suite. \\
Il y a beaucoup de débats car les députés n'ont pas la même vision, on y trouve deux grands courants qui vont marquer pendant longtemps l'histoire politique. Le premier courant était favorable à une monarchie limitée par les électeurs. Le deuxième était favorable à un pouvoir exécutif royal fort. Les premiers s'installaient à la gauche du Président de l'Assemblée, les seconds à sa droite. \\
À l'intérieur des courants, il y avait des sous courants. On trouve trois grandes tendances politiques: celle des aristocrates qui veulent un retour à l'ordre ancien, le retour de l'absolutisme, des privilèges (Vicomte de Mirabeau, Cazalès, Abbé Maury) ; les monarchiens sont des modérés, ils sont des monarchistes qui souhaiteraient que la révolution s'arrête à l'été 1789, au socle fondamental ; la troisième tendance sont les constitutionnels qui veulent établir une constitution, c'est la tendance majoritaire qui compte beaucoup de juristes (Tronchet, Talérand, Siéyès). Barnave et d'autres sont les moins modérés, Robespierre est encore moins modéré et joueront des rôles par la suite. 


Après les journées d'Octobre, l'Assemblée Constituante sera ramenée à Paris dans la salle du manège, qui se trouvaient près du Palais des Tuileries. Ils vont prendre l'habitude de se réunir en dehors des séances par tendance dans ce que l'on appelle des clubs, en références aux clubs des MPs anglais. Le premier club est celui des bretons, qui va s'élargir pour réunir les députés qui pensent comme les bretons, ça va devenir le club des jacobins. Ils sont assez libéraux, veulent limiter le pouvoir royal. \\
Un club plus modéré, les feuillants, vont s'installé à Paris. \\
À cette époque, comme on déclare la liberté de la presse, énormément de journaux apparaissent. Certains seront très éphémères et d'autres vont s'installer. L'un des plus célèbre sera celui de Marat, "L'ami du peuple". "L'ami du Roi" est le journal opposant au précédent, tenu par Royet. \\
La vie politique, à Paris surtout, dans les villes de provinces aussi, la vie politique est effervescente. 

\subsubsection{La Constitution}

Tous les députés sont d'accords pour séparer les pouvoirs, avec une idée: l'assemblée vote la Loi et le Roi exécute. Sur ce principe très simple vont se poser deux difficultés. La première est la question des chambres, bicaméral ou monocaméral. La deuxième est la question de l'obligation du Roi d'exécuter une loi votée: le Roi dispose-t-il d'un droit de veto ? \\
Il y aura une chambre unique car les trois ordres viennent d'être réunis et il n'est pas question de les désunir en créant une chambre aristocratique. Cette chambre est composée de députés élus pour deux ans. Elle a le pouvoir de discuter de la Loi et de la voter. \\
Il faut qu'il y ait la signature du Roi pour qu'il y ait application. Il sera décidé qu'on ne peut pas contraindre le détenteur du pouvoir exécutif à signer, le Roi aura donc un veto, mais il a seulement un pouvoir suspensif pendant 2 législatures. \\
Le Roi dispose d'autres pouvoirs, c'est notamment lui qui nomme les ministres et les révoque comme il l'entend. Ce sont des simples exécutants: ils ne sont responsables que devant le Roi et ne peuvent pas rentrer dans l'assemblée sans y être invité. Le Roi ne peut même pas être convoqué au nom de la séparation des pouvoirs. L'assemblée se réunit aussi d'elle même. \\
Ce sont donc deux pouvoirs strictement séparés, qui n'ont pas de contacts. Ils manquent donc de souplesse, et ne pourront pas marcher. La raison officielle d'une telle séparation est qu'on veut mettre en oeuvre la pensée de Montesquieu, mais officieusement, une telle séparation des pouvoirs est faite parce qu'on se méfie de Louis XVI. \\
Elle est officiellement promulguée le 3 Septembre 1791 mais on commençait déjà à l'appliquer au fur et à mesure des votes. 

\subsection{Le fonctionnement du régime}

\subsubsection{Les succès du régime}

Le premier succès de la monarchie constitutionnelle est la DDHC et tous les grands principes qui en découlent. En soi, c'est un énorme succès politique. \\
Ensuite, c'est la monarchie constitutionnelle qui a mis en place des réformes administratives parmi lesquelles on connaît encore, comme les départements, les municipalités. \\
On peut aussi parler de la naissance, en France, de l'esprit national qui est né officiellement sous la monarchie constitutionnelle lors de la fête des fédérés le 14 Juillet 1790. Dans chaque ville, il y avait des gardes nationales locales pour remplacer les anciennes milices bourgeoises et fonctionnaient de manière isolé. Elles ont eu l'idée de se fédérer, et de jurer de défendre le royaume et ses lois. Cela s'est fait de manière assez spontanée, partant du Dauphiné. L'AN propose de recevoir à Paris des délégués de ces gardes nationales pour un jour symbolique: le 14 Juillet 1790. Cela, étonnamment, se fait. Ils se réunissent sur le champ de Mars, situé à l'époque à l'extérieur de Paris. C'est là qu'ils vont prêter serment à la Constitution, à la Loi et prête surtout serment de fidélité à la Nation. Le Roi va lui aussi prêter serment. \\
Il est très important de parler de cette naissance de la Nation car la Révolution aurait pu provoquer un éclatement territorial du Royaume. 

\subsubsection{L'échec du régime}

Les États Généraux avaient étés réunis pour voter de nouveaux impôts et régler le problème du financement de l'État. Ils ont complètement déviés de leur fonction première. La situation est gravissime et on la mesure en 1790, les comités financiers se sont faits communiquer les chiffres, c'est la première fois qu'un citoyen rentre véritablement dans les finances royales. En Août 1790, la dette atteint la somme de 1902 millions de livre. Le service de la dette s'élève à 257 millions de livre par an. À ces 257, il faut ajouter 360 millions de livre de dépense de l'État. Pour payer tout ça, les recettes s'élèvent à 641 millions de livre. \\
Pour contrer ce déficit énorme, on engage les réformes de fond. Les départements et les district avaient étés fais pour un prélèvement de l'impôt plus efficace. L'égalité fiscale est censé aussi permettre de remplir les caisses. \\


Necker va lancer des emprunts, mais l'État est tellement endetté que les emprunts seront refusés. Un nouvel impôt sur le revenu sera donc proposé. Cela ne marche pas car c'est un système où les contribuables doivent déclarer leurs revenus et pleins de fausses déclarations sont faites. \\
Un homme en particulier réfléchit à cette situation, et va se dire que la France est très riche, notamment l'Église, mais qui possède de la richesse foncière. Un projet naît dans l'esprit d'un homme qui consiste à récupérer les terres en nationalisant la terre, disant que la terre est une richesse nationale, qui doit appartenir à la Nation. C'est Talérand qui propose cela le 10 Octobre 1789. \\
Necker proposait au mieux d'emprunter 80 millions, alors que l'Église possédait plus de deux milliards de livre. La nationalisation va donc être ordonné. Le 2 Novembre 1789, les biens du clergé sont nationalisés et sont mis à la disposition de la Nation. \\
Concrètement, les Églises, les calices, etc. sont conservés, mais tout ce qui est foncier sera vendu. Talérand pointe le fait que ce sont des français qui accéderont à la propriété, ce serait donc un contingent de plus qui serait éligible aux élections et donc plus d'adhérents à la Révolution. \\
Concrètement, les biens sont vendus sur plusieurs années afin que le prix de la terre soit intéressante. Immédiatement, on va créer une nouvelle monnaie qui seront assise sur les bien du clergé nationalisés. Ce sont les assignats qui sont émis par l'État et gagé sur les biens de l'Église. \\
Les ventes commencent en 1790. Les terres cultivables seront d'abord vendus puis les forêts et aussi certaines abbayes qui finiront par être démantelé pour vendre la pierre. \\
Le système fiscal sera réformé lui aussi. Tous les droits indirects seront supprimés, notamment les octrois municipaux. On supprime aussi les anciens monopoles de l'État, notamment celui de la vente sur le tabac. Cela prive l'État de millions de livre de recettes. Les droits de péage sont supprimés aussi, que ce soit ceux de l'État ou ceux des aristocrates. \\
Une contribution foncière, calculée en fonction des revenus de chacun, est mise en place. Le souhait du législateur est d'être juste. Mais c'est tellement compliqué que les officiers municipaux qui doivent percevoir l'impôt n'y comprennent rien (pour certains, ils ne savent même pas lire). Même les spécialistes du droit de l'époque ont du mal à y comprendre, et les français feront de fausses déclarations. \\
L'État fera tourner la "planche à billet", provoquant des inflations. 

\subsubsection{Les erreurs religieuses}
















\end{document}
