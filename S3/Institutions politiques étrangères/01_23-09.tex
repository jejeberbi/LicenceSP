\documentclass[10pt, a4paper, openany]{book}

\usepackage[utf8x]{inputenc}
\usepackage[T1]{fontenc}
\usepackage[francais]{babel}
\usepackage{bookman}
\usepackage{fullpage}
\setlength{\parskip}{5px}
\date{}
\title{Cours d'Institutions politiques étrangères (UFR Amiens)}
\pagestyle{plain}

\begin{document}
\maketitle
\tableofcontents

\chapter{Introduction générale: les démocraties libérales}

La définition classique de la démocratie (gouvernement du peuple par le peuple), peut être assez idéaliste. La démocratie antique et la démocratie moderne sont complètement différentes. Le régime "démocratique" qui domine est le modèle moderne occidental et libéral. \\
La principale caractéristique de cette démocratie, c'est qu'elle est représentative. Dans ce modèle, ce n'est pas le peuple qui exerce le pouvoir, mais une élite qui est sensé représenté le peuple et qui va gouverner à sa place. Le mandat n'est pas impératif. \\
Des auteurs ont cherchés à définir la démocratie autrement. Pour Raymond Aron: "Si on disait que la démocratie est la souveraineté du peuple, il y aurait deux mots obscurs: la souveraineté et le peuple", comme ces mots sont obscurs ils pourraient être détournés, il dit donc que la démocratie est "la compétition pacifique pour la conquête du pouvoir politique". La compétition passe donc dans l'élection qui doit être non faussé, libre, transparente, scrutin secret etc. Il doit y avoir un droit de l'opposition, un droit de minorité, il faut que le conflit soit accepté et institutionnalisé. Le pouvoir des gouvernants doit être provisoires et réversible: les élections doivent donc être régulières.


L'Amérique du Nord et l'Europe de l'Ouest sont les pionniers de ce type de régime, et son instauration ne s'est pas fais sans heurts. Pendant longtemps, des auteurs pensaient que la démocratie étaient ingouvernable car elle demanderait trop de vertus de la part des citoyens (Montesquieu) et que c'était inadapté à l'État moderne et son vaste territoire. \\
Cette réticence est liée au fait que le seul modèle démocratique jusque là était le modèle athénien. \\
Les démocraties libérales ont du surmonter d'autres obstacles plus concrets, pour transformer les gens en citoyens. \\
Mais la démocratie libérale a fini par s'enraciner et même par s'exporter. Elle est aujourd'hui une référence. Cela tient du fait que c'est un régime assez efficace, qui a le mérite de garder une certaine stabilité, limitant le potentiel de conflit. Il permet l'obéissance des gouvernés car permet ou donne l'illusion à ceux-ci de participer à l'exercice du pouvoir. Il permet la pacification de l'expression des clivages. 


Il y a eu des étapes successives. La première est la fin de la seconde guerre mondiale où la démocratie s'est imposée dans les pays vaincus. \\
La deuxième est dans les années 70 où elle s'est implantée dans les pays d'Europe du Sud. Dans les années 80, troisième étape, dans les pays d'Amérique latine. Dans les années 90, quatrième et dernière étape, la démocratie s'étend aux pays de l'ex URSS et certains pays d'Afrique. \\
Les Américains penseront que la démocratie s'est imposé partout et sont très optimiste, mais sont vite revenus sur cette pensée car le pays le plus peuplé du monde n'est pas une démocratie, et d'autres sont des démocraties de façade. \\
Des pays sont même retournés en arrière, notamment la Russie. \\
Le contexte de la crise économique ne favorise pas non plus la démocratie, certains gouvernements se faisant et se défaisant selon les marchés (Grèce notamment). 

\chapter{Les États-Unis d'Amérique}

Les USA ont inventés un certains nombre de principes et d'institutions qui sont devenus des références pour un certains nombre d'autres démocraties libérales. Ils ont d'abord inventé la Constitution, car c'est la première constitution écrite qu'il existe, elle l'a été en 1787 et est toujours en vigueur aujourd'hui et a été assez peu révisée. \\
Ils ont inventés le fédéralisme sur sa forme moderne, le régime présidentiel, ils ont inaugurés la décolonisation et inspiré en quelque sorte le droit des peuples à disposer d'eux mêmes. \\
Le fait que les US soit un modèle s'explique car des intellectuels Européens se sont fait les relais du système américains, Tocqueville notamment, qui participera à la rédaction de la Constitution de 1848. De plus, les US ont directement contribués à la transition démocratique de l'Allemagne ou du Japon à la fin de la seconde guerre mondiale. \\
Les Américains même contribuent à la critique de leur propre système car il y a des blocages institutionnels de plus en plus fréquent. Le système Américain est assez unique, car ceux ayant essayé de copier les US ont assez peu eu de succès. Le régime présidentiel par exemple n'a jamais été fonctionnel en dehors des US, sauf peut être au Brésil. 

\section{Un pays neuf}

\subsection{La naissance des États-Unis}

L'histoire des US contient des événements plutôt marquant, encore aujourd'hui. Les US sont nés dans la douleur, d'un compromis difficile, et d'une unité nationale qui a du être construite et consolidé pour que le modèle survive. 

\subsubsection{De la colonisation à l'indépendance}

La colonisation de l'Amérique du Nord a commencée au XVIIe siècle. Les colons sont des puritains qui ont fuis l'Europe pour trouver une forme de liberté religieuse. Les puritains sont un mouvement protestant fondamentaliste, qui voulait redécouvrir la pureté originelle de la religion. Ils ne se reconnaissaient pas dans la religion catholique ou anglicane du Royaume-Uni. Ils y sont stigmatisés, ce qui explique leur déplacement vers l'Amérique. \\
La première vague, du May Flower contenait une centaine de colons. Ceux-ci avaient passés un pacte: le Mayflower Compact qui disait notamment que toutes les décisions se prendraient à la majorité par les hommes. \\
Des vagues successives se sont suivis et ont commencés à prospérer. Treize colonies, assez libre ont été fondées sur la côte Est des États-Unis. Les relations avec la métropole ont commencés à se dégradé sous plusieurs facteurs. Au XVIIIe siècle, la discussion politique s'était développée dans ces colonies et ont attisés une idée d'indépendance. Les treize colonies d'Amérique étaient entrés dans une ère pré industrielle alors que le RU voyait ses colonies comme une mine de matière première. Le conflit a vraiment été lancée quand le Parlement Londonien a voté une série de taxes. \\
Ces taxes ont permis aux colons de s'unir et de défendre certains principes. Ils ont inventés la notion de légalité et de constitutionnalité. Le droit britannique ne reconnaît pas la hiérarchie des normes même si il y a quelques grands textes qui pose un principe de consentement à l'impôt, or, ceux-ci ne sont pas supérieurs à la Loi. Les américains ont donc inventés une hiérarchie à l'intérieur du droit britannique.


Les gouverneurs coloniaux se sont exilés, provoquant une vacance du pouvoir. Des colonies se sont dotés de constitutions relativement démocratiques. La déclaration d'indépendance sera déclarée le 4 Juillet 1776. Cette déclaration annonce une série de droits, notamment que les Hommes naissent égaux, donne le droit à la poursuite du bonheur, etc. \\
En 1783, le monarque Britannique reconnaît l'indépendance des colonies. Cependant, elles étaient toutes souveraines.

\subsubsection{Le compromis fédéral}

Les treize colonies ne pouvaient pas s'isoler. Une solution a été retenu en 1777, un an après la déclaration d'indépendance. Elles fondent une confédération. Les choses se compliquent car les colonies sont en désaccord sur la manière d'évoluer. Certains veulent une vraie fédération, les grands États notamment, alors que les petits États souhaitaient garder leur souveraineté. \\
En 1787, une convention est convoquée à Philadelphie pour améliorer la confédération. Il y a ici un coup de force car les délégués outrepasseront leurs mandats. George Washington en est le président, Hamilton est présent aussi. Thomas Jefferson, qui a rédigé la déclaration d'indépendance n'est pas présent car est ambassadeur à Paris. \\
La Constitution adoptée reconnaît que tout pouvoir émane du peuple car il fallait opposer quelque chose à la légitimité divine du monarque britannique. Les constituants ont cependant peur de la démocratie. Hamilton disait "toutes les collectivités se divisent d'elles même entre les élites et les autres". \\
Des concessions sont faites de parts et d'autres. Le Sénat avec un fort pouvoir est une concession des fédéralistes aux plus petits États. \\
La Constitution est adoptée en 1787 et entrée en vigueur en 1789. Ce temps s'explique par le fait que les américains n'étaient pas prêt à se fédérer. La Constitution prévoyait qu'elle rentrerait en vigueur si neuf États sur les treize l'acceptaient. Hamilton écrira les "Federalist papers" pour vanter les mérites de la Constitution. \\
Elle a été votée sans vraiment prendre en compte l'avis des peuples des colonies qui ont adopté la Constitution via des représentants. Elle a fini par être ratifiée. C'est donc en 1789 que sont organisées les premières élections fédérales américaines. George Washington sera élu.

\subsubsection{La consolidation de l'unité nationale et l'expansion américaine}

Les circonstances d'adoption de la Constitution expliquent que l'unité nationale américaine était très fragile. Les élites ont essayé de créer un certains nombre de symboles: la fête nationale le 4 Juillet, Thanksgiving, le drapeau (qui date de 1777), George Washington qui est devenu le héros national. \\
L'unité nationale n'a été remise en cause qu'une seule fois en 1860 suite à l'élection d'Abraham Lincoln qui souhaitait abroger l'esclavage, et qui a donc déclenché la guerre sécession, remportée par le clan nordiste et qui a donc renforcée le lien fédéral. À noter que la Constitution est muette sur le fait de pouvoir quitter ou non la fédération: avec la victoire des nordistes, il est clair que non. \\
Il est à noter que les Américains avaient besoin de se fixer de nouveaux défis. Cela s'est fait au XIXe siècle, où le territoire des US a été multiplié par 10. Les Américains ont colonisés les terres intérieurs, pensant qu'elles étaient vierges, de nouveaux États fédérés se sont donc créés. \\
Il n'y a pas eu que la conquête de l'ouest, le Texas et la Californie ont été conquis dans une guerre contre le Mexique, Hawaï a été conquis, d'autres territoires ont été achetés. En 1890, le bureau de recensement fédéral a officiellement annoncé qu'il n'y avait plus de terres à conquérir. C'est aussi cette année là qu'ont fini les guerres indiennes. \\
C'est dans les années 1890 que les US ont regardés vers l'extérieur, ont exercés une certaine tutelle sur Cuba, etc. 

\subsection{La société américaine et ses clivages}

\subsubsection{"Melting pot" et "rêve américain": entre mythes et réalités}

L'idée du "melting pot" renvoie à l'idée de réunir tous les migrants/colons venant aux US et de les fusionner pour en faire de nouvelles personnes, entièrement américaines. Cela est remis en cause aujourd'hui. \\
La caractéristique première des US est que c'est un pays d'immigrés. À chaque nouvelle vague d'immigration, les américains en place se sont sentis menacés. Cette peur des nouveaux immigrants existent depuis le départ, Washington donnait déjà l'idée d'une réticence. Jefferson disait que les migrants arriveraient avec une trop grande habitude de la monarchie. \\
Au XIXe siècle, l'entrée sur le territoire est interdite aux indigents, aux polygames, aux personnes atteintes de maladies contagieuses ou "répugnantes", etc etc. \\
Dans les années 1920, les US ont établis des quotas par nationalité. Certaines étaient très exclus, les Russes, les japonais essentiellement. En 1965, ceci est abrogé et pose le principe d'égalité entre tous les immigrants. Il existe plusieurs manières d'obtenir une "carte verte": regroupement familial (deux tiers du total), immigration d'emploi, loterie (pour favoriser la diversité), les réfugiés ne sont pas comptés dans le quota. Au bout de cinq ans de résidence aux US, les migrants peuvent demander la nationalité. Ils appliquent le droit du sol strict. \\
Cela fait 20 ans que les institutions cherchent à réformé en profondeur le système d'immigration, cela n'a jamais été abouti. Ce sujet est un sujet très polémique aux US. Les US restent donc une terre d'immigration. \\
Il est à noter que les non-WASP peuvent accéder aux postes de responsabilités mais cela reste assez minoritaire. 


L'idée du rêve américain rend compte du pouvoir d'attraction exercé par les US. Ce rêve est un idéal où n'importe qui peut réussir, ce ne serai pas la naissance qui fait que l'on réussit ou non. Cela est relativement vrai au XVIIIe siècle. Cependant, aujourd'hui, la mobilité sociale n'est pas nécessairement plus forte aux US qu'en Europe. \\
Les inégalités sociales, elles, sont par contre très fortes. Les écarts de richesse ont tendance à s'accroître malgré la reprise économique. Il existe malgré tout, aux US, des personnes qui ont plus de chances de réussir selon leur naissance. \\
Étonnamment pour un pays où il n'y a jamais eu d'aristocratie, on peut voir la formation de certaines dynasties familiales, les Bush, les Kennedy, etc. \\
Parmi tous les présidents US, 7 descendent directement des passagers du May Flower. 


Au sens américains, une minorité est un groupe de personnes partageant des caractéristiques physiques et culturelles et qui du fait de ces caractéristiques, sont considérés comme distincts du reste de la population à partir du moment où ces personnes font l'objet d'une discrimination et sont sous représenté politiquement. \\
Aux US, une des premières minorités sont les afro-américains. Il y a 13\% d'afro-américains dans la population américaine. Cependant, les élus afro-américains sont 8\%, ils sont donc sous représentant. 28\% des afro-américains vivent sous le seuil de pauvreté alors que la moyenne nationale est de 15\%. Ils sont une première minorité numérique, mais aussi la première minorité historique. \\
Les asiatiques sont à peu près 5\%, les hispaniques représenteraient plus de 16\% de la population américaine. Selon certaines études, d'ici 2050, les hispaniques représenteraient un tiers de la population américaine, ce qui fait peur aux WASP.
























\end{document}
