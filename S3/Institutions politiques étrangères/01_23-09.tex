\documentclass[10pt, a4paper, openany]{book}

\usepackage[utf8x]{inputenc}
\usepackage[T1]{fontenc}
\usepackage[francais]{babel}
\usepackage{bookman}
\usepackage{fullpage}
\setlength{\parskip}{5px}
\date{}
\title{Cours d'Institutions politiques étrangères (UFR Amiens)}
\pagestyle{plain}

\begin{document}
\maketitle
\tableofcontents

\chapter{Introduction générale: les démocraties libérales}

La définition classique de la démocratie (gouvernement du peuple par le peuple), peut être assez idéaliste. La démocratie antique et la démocratie moderne sont complètement différentes. Le régime "démocratique" qui domine est le modèle moderne occidental et libéral. \\
La principale caractéristique de cette démocratie, c'est qu'elle est représentative. Dans ce modèle, ce n'est pas le peuple qui exerce le pouvoir, mais une élite qui est sensé représenté le peuple et qui va gouverner à sa place. Le mandat n'est pas impératif. \\
Des auteurs ont cherchés à définir la démocratie autrement. Pour Raymond Aron: "Si on disait que la démocratie est la souveraineté du peuple, il y aurait deux mots obscurs: la souveraineté et le peuple", comme ces mots sont obscurs ils pourraient être détournés, il dit donc que la démocratie est "la compétition pacifique pour la conquête du pouvoir politique". La compétition passe donc dans l'élection qui doit être non faussé, libre, transparente, scrutin secret etc. Il doit y avoir un droit de l'opposition, un droit de minorité, il faut que le conflit soit accepté et institutionnalisé. Le pouvoir des gouvernants doit être provisoires et réversible: les élections doivent donc être régulières.


L'Amérique du Nord et l'Europe de l'Ouest sont les pionniers de ce type de régime, et son instauration ne s'est pas fais sans heurts. Pendant longtemps, des auteurs pensaient que la démocratie étaient ingouvernable car elle demanderait trop de vertus de la part des citoyens (Montesquieu) et que c'était inadapté à l'État moderne et son vaste territoire. \\
Cette réticence est liée au fait que le seul modèle démocratique jusque là était le modèle athénien. \\
Les démocraties libérales ont du surmonter d'autres obstacles plus concrets, pour transformer les gens en citoyens. \\
Mais la démocratie libérale a fini par s'enraciner et même par s'exporter. Elle est aujourd'hui une référence. Cela tient du fait que c'est un régime assez efficace, qui a le mérite de garder une certaine stabilité, limitant le potentiel de conflit. Il permet l'obéissance des gouvernés car permet ou donne l'illusion à ceux-ci de participer à l'exercice du pouvoir. Il permet la pacification de l'expression des clivages. 


Il y a eu des étapes successives. La première est la fin de la seconde guerre mondiale où la démocratie s'est imposée dans les pays vaincus. \\
La deuxième est dans les années 70 où elle s'est implantée dans les pays d'Europe du Sud. Dans les années 80, troisième étape, dans les pays d'Amérique latine. Dans les années 90, quatrième et dernière étape, la démocratie s'étend aux pays de l'ex URSS et certains pays d'Afrique. \\
Les Américains penseront que la démocratie s'est imposé partout et sont très optimiste, mais sont vite revenus sur cette pensée car le pays le plus peuplé du monde n'est pas une démocratie, et d'autres sont des démocraties de façade. \\
Des pays sont même retournés en arrière, notamment la Russie. \\
Le contexte de la crise économique ne favorise pas non plus la démocratie, certains gouvernements se faisant et se défaisant selon les marchés (Grèce notamment). 

\chapter{Les États-Unis d'Amérique}

Les USA ont inventés un certains nombre de principes et d'institutions qui sont devenus des références pour un certains nombre d'autres démocraties libérales. Ils ont d'abord inventé la Constitution, car c'est la première constitution écrite qu'il existe, elle l'a été en 1787 et est toujours en vigueur aujourd'hui et a été assez peu révisée. \\
Ils ont inventés le fédéralisme sur sa forme moderne, le régime présidentiel, ils ont inaugurés la décolonisation et inspiré en quelque sorte le droit des peuples à disposer d'eux mêmes. \\
Le fait que les US soit un modèle s'explique car des intellectuels Européens se sont fait les relais du système américains, Tocqueville notamment, qui participera à la rédaction de la Constitution de 1848. De plus, les US ont directement contribués à la transition démocratique de l'Allemagne ou du Japon à la fin de la seconde guerre mondiale. \\
Les Américains même contribuent à la critique de leur propre système car il y a des blocages institutionnels de plus en plus fréquent. Le système Américain est assez unique, car ceux ayant essayé de copier les US ont assez peu eu de succès. Le régime présidentiel par exemple n'a jamais été fonctionnel en dehors des US, sauf peut être au Brésil. 

\section{Un pays neuf}

\subsection{La naissance des États-Unis}

L'histoire des US contient des événements plutôt marquant, encore aujourd'hui. Les US sont nés dans la douleur, d'un compromis difficile, et d'une unité nationale qui a du être construite et consolidé pour que le modèle survive. 

\subsubsection{De la colonisation à l'indépendance}

La colonisation de l'Amérique du Nord a commencée au XVIIe siècle. Les colons sont des puritains qui ont fuis l'Europe pour trouver une forme de liberté religieuse. Les puritains sont un mouvement protestant fondamentaliste, qui voulait redécouvrir la pureté originelle de la religion. Ils ne se reconnaissaient pas dans la religion catholique ou anglicane du Royaume-Uni. Ils y sont stigmatisés, ce qui explique leur déplacement vers l'Amérique. \\
La première vague, du May Flower contenait une centaine de colons. Ceux-ci avaient passés un pacte: le Mayflower Compact qui disait notamment que toutes les décisions se prendraient à la majorité par les hommes. \\
Des vagues successives se sont suivis et ont commencés à prospérer. Treize colonies, assez libre ont été fondées sur la côte Est des États-Unis. Les relations avec la métropole ont commencés à se dégradé sous plusieurs facteurs. Au XVIIIe siècle, la discussion politique s'était développée dans ces colonies et ont attisés une idée d'indépendance. Les treize colonies d'Amérique étaient entrés dans une ère pré industrielle alors que le RU voyait ses colonies comme une mine de matière première. Le conflit a vraiment été lancée quand le Parlement Londonien a voté une série de taxes. \\
Ces taxes ont permis aux colons de s'unir et de défendre certains principes. Ils ont inventés la notion de légalité et de constitutionnalité. Le droit britannique ne reconnaît pas la hiérarchie des normes même si il y a quelques grands textes qui pose un principe de consentement à l'impôt, or, ceux-ci ne sont pas supérieurs à la Loi. Les américains ont donc inventés une hiérarchie à l'intérieur du droit britannique.


Les gouverneurs coloniaux se sont exilés, provoquant une vacance du pouvoir. Des colonies se sont dotés de constitutions relativement démocratiques. La déclaration d'indépendance sera déclarée le 4 Juillet 1776. Cette déclaration annonce une série de droits, notamment que les Hommes naissent égaux, donne le droit à la poursuite du bonheur, etc. \\
En 1783, le monarque Britannique reconnaît l'indépendance des colonies. Cependant, elles étaient toutes souveraines.

\subsubsection{Le compromis fédéral}

Les treize colonies ne pouvaient pas s'isoler. Une solution a été retenu en 1777, un an après la déclaration d'indépendance. Elles fondent une confédération. Les choses se compliquent car les colonies sont en désaccord sur la manière d'évoluer. Certains veulent une vraie fédération, les grands États notamment, alors que les petits États souhaitaient garder leur souveraineté. \\
En 1787, une convention est convoquée à Philadelphie pour améliorer la confédération. Il y a ici un coup de force car les délégués outrepasseront leurs mandats. George Washington en est le président, Hamilton est présent aussi. Thomas Jefferson, qui a rédigé la déclaration d'indépendance n'est pas présent car est ambassadeur à Paris. \\
La Constitution adoptée reconnaît que tout pouvoir émane du peuple car il fallait opposer quelque chose à la légitimité divine du monarque britannique. Les constituants ont cependant peur de la démocratie. Hamilton disait "toutes les collectivités se divisent d'elles même entre les élites et les autres". \\
Des concessions sont faites de parts et d'autres. Le Sénat avec un fort pouvoir est une concession des fédéralistes aux plus petits États. \\
La Constitution est adoptée en 1787 et entrée en vigueur en 1789. Ce temps s'explique par le fait que les américains n'étaient pas prêt à se fédérer. La Constitution prévoyait qu'elle rentrerait en vigueur si neuf États sur les treize l'acceptaient. Hamilton écrira les "Federalist papers" pour vanter les mérites de la Constitution. \\
Elle a été votée sans vraiment prendre en compte l'avis des peuples des colonies qui ont adopté la Constitution via des représentants. Elle a fini par être ratifiée. C'est donc en 1789 que sont organisées les premières élections fédérales américaines. George Washington sera élu.

\subsubsection{La consolidation de l'unité nationale et l'expansion américaine}

Les circonstances d'adoption de la Constitution expliquent que l'unité nationale américaine était très fragile. Les élites ont essayé de créer un certains nombre de symboles: la fête nationale le 4 Juillet, Thanksgiving, le drapeau (qui date de 1777), George Washington qui est devenu le héros national. \\
L'unité nationale n'a été remise en cause qu'une seule fois en 1860 suite à l'élection d'Abraham Lincoln qui souhaitait abroger l'esclavage, et qui a donc déclenché la guerre sécession, remportée par le clan nordiste et qui a donc renforcée le lien fédéral. À noter que la Constitution est muette sur le fait de pouvoir quitter ou non la fédération: avec la victoire des nordistes, il est clair que non. \\
Il est à noter que les Américains avaient besoin de se fixer de nouveaux défis. Cela s'est fait au XIXe siècle, où le territoire des US a été multiplié par 10. Les Américains ont colonisés les terres intérieurs, pensant qu'elles étaient vierges, de nouveaux États fédérés se sont donc créés. \\
Il n'y a pas eu que la conquête de l'ouest, le Texas et la Californie ont été conquis dans une guerre contre le Mexique, Hawaï a été conquis, d'autres territoires ont été achetés. En 1890, le bureau de recensement fédéral a officiellement annoncé qu'il n'y avait plus de terres à conquérir. C'est aussi cette année là qu'ont fini les guerres indiennes. \\
C'est dans les années 1890 que les US ont regardés vers l'extérieur, ont exercés une certaine tutelle sur Cuba, etc. 

\subsection{La société américaine et ses clivages}

\subsubsection{"Melting pot" et "rêve américain": entre mythes et réalités}

L'idée du "melting pot" renvoie à l'idée de réunir tous les migrants/colons venant aux US et de les fusionner pour en faire de nouvelles personnes, entièrement américaines. Cela est remis en cause aujourd'hui. \\
La caractéristique première des US est que c'est un pays d'immigrés. À chaque nouvelle vague d'immigration, les américains en place se sont sentis menacés. Cette peur des nouveaux immigrants existent depuis le départ, Washington donnait déjà l'idée d'une réticence. Jefferson disait que les migrants arriveraient avec une trop grande habitude de la monarchie. \\
Au XIXe siècle, l'entrée sur le territoire est interdite aux indigents, aux polygames, aux personnes atteintes de maladies contagieuses ou "répugnantes", etc etc. \\
Dans les années 1920, les US ont établis des quotas par nationalité. Certaines étaient très exclus, les Russes, les japonais essentiellement. En 1965, ceci est abrogé et pose le principe d'égalité entre tous les immigrants. Il existe plusieurs manières d'obtenir une "carte verte": regroupement familial (deux tiers du total), immigration d'emploi, loterie (pour favoriser la diversité), les réfugiés ne sont pas comptés dans le quota. Au bout de cinq ans de résidence aux US, les migrants peuvent demander la nationalité. Ils appliquent le droit du sol strict. \\
Cela fait 20 ans que les institutions cherchent à réformé en profondeur le système d'immigration, cela n'a jamais été abouti. Ce sujet est un sujet très polémique aux US. Les US restent donc une terre d'immigration. \\
Il est à noter que les non-WASP peuvent accéder aux postes de responsabilités mais cela reste assez minoritaire. 


L'idée du rêve américain rend compte du pouvoir d'attraction exercé par les US. Ce rêve est un idéal où n'importe qui peut réussir, ce ne serai pas la naissance qui fait que l'on réussit ou non. Cela est relativement vrai au XVIIIe siècle. Cependant, aujourd'hui, la mobilité sociale n'est pas nécessairement plus forte aux US qu'en Europe. \\
Les inégalités sociales, elles, sont par contre très fortes. Les écarts de richesse ont tendance à s'accroître malgré la reprise économique. Il existe malgré tout, aux US, des personnes qui ont plus de chances de réussir selon leur naissance. \\
Étonnamment pour un pays où il n'y a jamais eu d'aristocratie, on peut voir la formation de certaines dynasties familiales, les Bush, les Kennedy, etc. \\
Parmi tous les présidents US, 7 descendent directement des passagers du May Flower. 

\subsubsection{La question des minorités}
Au sens américains, une minorité est un groupe de personnes partageant des caractéristiques physiques et culturelles et qui du fait de ces caractéristiques, sont considérés comme distincts du reste de la population à partir du moment où ces personnes font l'objet d'une discrimination et sont sous représenté politiquement. \\
Aux US, une des premières minorités sont les afro-américains. Il y a 13\% d'afro-américains dans la population américaine. Cependant, les élus afro-américains sont 8\%, ils sont donc sous représentant. 28\% des afro-américains vivent sous le seuil de pauvreté alors que la moyenne nationale est de 15\%. Ils sont une première minorité numérique, mais aussi la première minorité historique. \\
Les asiatiques sont à peu près 5\%, les hispaniques représenteraient plus de 16\% de la population américaine. Selon certaines études, d'ici 2050, les hispaniques représenteraient un tiers de la population américaine, ce qui fait peur aux WASP.


Il y a eu des mouvements d'affirmation des minorités, le premier mouvement est le mouvement des afro américains qui s'est développé après la seconde guerre mondiale. Ce mouvement est marqué par l'expérience de la guerre et le discours ambiant de libération des peuples avec la décolonisation. Cependant le mouvement avait plusieurs voix: pacifiste/directe ou encore intégration/appuyer sur les différences. \\
Le mouvement est marqué par la ségrégation. Le mouvement a un répertoire d'action inspiré de la lutte pour l'indépendance indienne. \\
Le mouvement de l'action directe, le contexte est entièrement différent: c'est dans les États du Nord, là où il n'y a pas officiellement de ségrégation. Le parti des black panters est créé en 1966, en réaction à Martin Luther King, ce parti a mis en place des milices armés pour protéger les quartiers noirs, protéger contre les violences policières mais va essayer aussi d'aider au niveau médical, etc. Ce parti se désintégrera dans les années 80. \\
Nation of islam est un autre mouvement qui cherchait à faire leur ségrégation, à l'envers. Les membres de ce parti ne se considérait même pas comme citoyen américain, ils refusaient de voter, de servir dans l'armée, etc. Il a été fondé dans les années 50 et a pris de l'ampleur dans les années 50-60 avec Malcolm X comme leader. Il existe toujours aujourd'hui, ne refuse plus de voter et avait même soutenu Obama en 2008. Ce parti reste extrémiste aujourd'hui. Le leader actuel est Farrakhan.


Tous ces mouvements ont plus ou moins conduit à une progression de l'acceptation des minorités. Le premier progrès est l'application du principe d'égalité devant la loi. Important: l'adoption en 1964 du "Civil right act", interdisant toute forme de ségrégation. L'affirmative action est une forme de discrimination positive, mise en place pour compenser la discrimination négative qui a eu lieu, et aussi pour établir une diversité et une égalité que le fonctionnement normal de la société ne permet pas. \\
En 1965, les entreprises prestataires de travaux publics étaient obligés de mettre en place des mécanismes d'affirmative action. Nickson a cherché à rétablir la paix sociale et a donc étendu ces mesures. Certains États fédérés sont allés encore plus loin dans leur législation locale. \\
À la question de si ces lois sont constitutionnelles, la Cour Suprême a pu se prononcer plusieurs fois. Dans l'affaire de "l'université de Californie contre Bakke", cette dite université réservait 16\% de ses places pour les minorités, un étudiant a porté plainte et cela est remonté jusqu'à la Cour Suprême. Elle a considéré que ce n'était pas inconstitutionnel de prendre en compte des critères ethno-racial pour le recrutement. Cependant, cette discrimination ne peut se faire qu'à qualification égale, les quotas rigides étant interdit, l'université a donc perdu. \\
Il y a eu quelques effets pervers de cette politique. En effet, pour permettre le traitement de faveur, c'est déjà racialisé l'individu. Ça a aussi renforcé la tendance communautaire. Dans certains États, les écoles permettaient même aux hispaniques de faire leurs scolarité dans leur langue, ce qui a produit des écoles de seconde zone. \\
Au bout du compte, cela n'a pas empêché les minorités de cumuler un certains nombre d'handicap sociaux. \\
Les américains contestent de plus en plus leurs politiques d'afirmative action. En 1990, lors d'un référendum en Californie, les citoyens californiens ont répondu massivement qu'il fallait arrêter cette politique à l'université. La Cour Suprême a considéré que la discrimination positive à l'école était non constitutionnel. 

\subsubsection{Le poids de la religion}

Il y a une corrélation très nette entre l'appartenance d'un individu à une religion et sa place dans la société. Les différentes Églises constituent des groupes de pression et jouent donc un rôle politique. La relation entre les Églises et l'État est très ambigus. \\
47\% des américains sont protestants. 20\% sont catholiques. 70\% des américains sont chrétiens si on rajoute les orthodoxes, les mormons, etc. 6\% se déclarent comme étant non chrétiens. Presque un quart se disent sans religion. \\
Protestant ne veut pas dire grand chose en tant que tel aux US, il existe de nombreuses Églises protestantes.


Les protestants ont longtemps été en position dominantes alors que les fidèles des autres religions étaient dans des classes inférieures, c'est un peu moins vrai maintenant. Les WASP contrôlent encore une bonne partie des positions de pouvoir. Kennedy était le seul président non protestant. \\
En principe, la constitution américaine prévoit une liberté de culte et une stricte séparation de l'État et des Églises. Pourtant, cela n'empêche pas les affaires publiques de baigner dans une religiosité. La politique américaine est imprégnée de la religion. \\
Ces dernières années, les américains ont vu voir renaître un fondamentalisme protestant dans la "bible belt", du Texas à la Floride. Ce sont des mouvements qui recrutent dans toutes les classes sociales, ont des moyens financiers importants et des positions d'influence. Ces mouvements fondamentalistes recherchent à faire pression sur la politique pour, par exemple, introduire la prière à l'école publique, interdire l'avortement, le mariage homosexuel, etc. 

\section{Le cadre institutionnel: un système de check and balances}

Équilibre: interaction de forces égales qui s'annule en s'opposant. Les pères fondateurs ont donc cherchés à créer des institutions avec un pouvoir aussi égal que possible mais avec des fonctions distinctes et en permanence rivalité. C'est ce qu'on appelle le "check and balancer": "frein et contrepoids". À chaque pouvoir doit correspondre un contre-pouvoir. Tout le système américain est marqué par cet équilibre. 

\subsection{La Constitution comme norme suprême}

\subsubsection{Le compromis fondateur et les amendements constitutionnels}

Pour les Américains, la Constitution est une norme fondatrice. C'est cette constitution qui a construit les US. Elle a su s'adapter, résister au changement. C'est un texte court, qui laisse place à la pratique et à l'interprétation. Il y a 7 article suivi de 22 amendements. La constitution est tellement sacrée que le texte d'origine n'a jamais été modifié et les amendements sont mis après. \\
La constitution est rigide, difficile à réviser. Dès le début de la guerre d'indépendance, les 13 colonies avaient chacune commencés à adopter leurs constitutions. Celles-ci étaient souple. Sauf que les parlements locaux en ont profité pour atteindre aux droits fondamentaux. Jefferson a perçu le danger, ce qui explique que la constitution fédérale est rigide. À ceci s'ajoute le fait que la constitution fédérale était un pacte fondateur, et qu'il ne fallait donc surtout pas la modifier. \\
Pour réviser, il faut le consentement d'une large partie des États fédérés, mais aucun État fédéré peut bloquer seul une procédure de révision. \\
La révision se fait en trois étapes. L'initiative appartient forcément au pouvoir législatif, donc au congrès, il faut la majorité des 2/3 de chacune des chambres ou venir des parlements fédérés si 2/3 des parlements fédérés sont d'accord. Si l'initiative vient du congrès, c'est lui qui rédige l'amendement, si l'initiative vient des parlements fédérés, on élit une convention qui fera ce travail de rédaction. Enfin, il faut que l'amendement soit ratifié par trois quart des États. 


Les 10 premiers amendements ont étés adoptés en bloc en 1791. Ces 10 amendements ont servi à garantir les libertés individuelles. Seul le 10ème amendement garantit le droit des États fédérés, en posant la compétence législative de droit commun aux États fédérés. Le premier amendement consacre la liberté, religieuse, d'association, etc. \\
Le deuxième amendement consacre le droit de porter des armes et est considéré comme un droit inaliénable aux US. \\
Après 1791, il y a eu 17 amendements supplémentaires. Certains élargissent encore les libertés fondamentale. Le 14 ème amendement par exemple, de 1868, garantit le droit du sol en matière de citoyenneté. Ça a été adopté pour garantir la citoyenneté aux anciens esclaves. Il prévoit aussi qu'aucun fédéré ne pourrait faire et appliquer des lois restreignant les privilèges des citoyens américains. Tous les citoyens ont accès à l'égale protection des Lois. \\
Les 18e et 21e amendement, concernent la prohibition de l'alcool, entre 1918 et 1933. \\
Les règles en matière de droit de vote sont posés par les États fédérés. Au départ, le suffrage était restreint. Tout ça a été éliminé pour que le suffrage universel masculin soit établi entre 1820 et 1840. Le 15ème amendement, en 1870, essaye de régler le problème que les amérindiens ou les afro-américains ne peuvent pas voter. \\
En 1964, le 20 ème amendement a banni les pratique des États fédérés cherchant à contourner les amendements précédents. \\
Les femmes ont pu voter à la deuxième moitié du XIXe siècle, mais cela varie selon les États et c'est harmonisé en 1920. \\
Ces amendements n'ont pas modifiés l'équilibre institutionnel, et ont toujours été fait dans une vision très pragmatique. 


\subsubsection{Le pouvoir judiciaire et l'invention du contrôle de constitutionnalité}

Aux US, le pouvoir judiciaire est fort. C'est une justice à tradition "common law" où la jurisprudence a une place importante, les juges sont obligés de se conformer à la règle du précédent dans des cas similaires. Aux US, il n'y a qu'un seul ordre de juridiction. Par contre, il y a des juridictions fédérales et des juridictions fédérés. Les juridictions fédérales sont compétentes pour le droit fédéral et les normes internationales. Tout le reste relève des juridictions des États fédérés. \\
À l'échelon fédéral, on a les district court, des CA, et la Cour suprême, pareil pour l'échelon fédéré. Mais c'est la cour suprême fédérale qui domine le tout si elle a l'occasion de se prononcer sur un litige. \\
La cour suprême est composée de 9 juges, présidé ppar le Chief Justice. Ils sont nommés par le président des US avec l'accord du Sénat. Le Président nomme aussi tous les juges fédéraux, avec accord du Sénat. Dans les tribunaux fédérés, il y a beaucoup de juges élus mais pose des problèmes de corruption. \\
Pendant les premières années de l'entrée en vigueur de la constitution, il y avait des "nominations de minuit" qui consiste à nommer tout un tas de fonctionnaires avant de quitter le pouvoir. Il faut toujours l'accord du Sénat. \\
Les membres de la Cour Suprême sont nommés à vie. Tous les présidents n'ont donc pas forcément la possibilité de nommer. Cela garantit aussi leur indépendance, une fois nommé ils ne doivent plus rien à personne. Les différents tribunaux se sont donnés la possibilité du contrôle constitutionnel. \\


Dans le contrôle de constitutionnalité, on pensait au départ à un processus de nullité, où les parlements fédérés pouvaient déclarer nulle une loi fédérale contraire à la Constitution. Finalement, avec l'affaire Marbury v/ Madison, en 1803, la cour suprême a posé un grand principe: elle a invoqué le caractère suprême, rigide, écrit, son objectif, et a donc déduit que la Constitution faisait partie du droit positif et était supérieure à toutes les autres normes juridiques. De là, la Cour a déduit qu'une norme contraire à la Constitution ne pouvait pas s'appliquer. La Cour a estimée que les juges avaient le pouvoir de trancher. Depuis, toutes les juridictions ont la compétences du contrôle de constitutionnalité. \\
La cour suprême a le moyen de tenir tête au législatif et à l'exécutif. Les juges prennent les décisions mais n'a pas le pouvoir de les appliquer. On dit souvent que les juges n'ont que le pouvoir de persuader. 


La période de la présidence de John Marshall a marqué une période d'affirmation des pouvoirs de la cour suprême et défend le fédéralisme. \\
Au cours d'une période très longue, il y eut une cour très conservatrice entre 1835 et 1950. En 1857, Scott v. / Sandford, la cour suprême a refusé de déclaré inconstitutionnel une loi sur l'esclavage. \\
Ca Cour a supporté les États sudistes. 1896, Plessy v/ Ferguson, d'où vient la phrase "separate but equal" où la cour juge que la ségrégation ne met pas en péril l'égalité. La cour suprême a beaucoup mis en péril la politique de new deal. \\
À partir de 1953, la Cour est devenu plutôt garante des libertés fondamentales. 1954 Brown v. / B.E Topeka, la Cour a utilisé le 14ème amendement sur l'égale liberté des lois pour renversé la jurisprudence de 1896. \\
Ensuite, la Cour a consacré l'égalité des genres, l'avocat lors d'un interrogatoire, etc. \\
1973, Roe v. / Wade, la CS a donné le droit à l'IVG aux femmes. \\
En 1972, un arrêt suspendait la peine de mort partout aux US, jugeant la peine contraire au 8ème amendement et au 14ème amendement, du moins, tant qu'on pouvait avoir des doutes sur le respect des droits de la défense, des doutes de discrimination, ou si la peine provoquait des souffrances inutiles. Il y a polémique aujourd'hui sur l'injection létale car si les produits sont mal dosés, il peut y avoir des souffrances horribles. La cour suprême jugera que l'injection létale n'est pas contraire à la constitution car elle n'a pas été pensée pour faire souffrir. \\
Raegan a pu nommer 4 juges, particulièrement conservateur. Bush père et fils ont pu nommer trois juges. À partir de là, il était normal de voir une cour ultra libérale sur l'économie et conservatrice. Il y a donc eu de nouveaux revirements de jurisprudences. \\
Aujourd'hui, Obama a pu nommé deux juges et la Cour est assez divisé. En 2015, la Cour suprême a décidé que le mariage homosexuel ne pouvait pas être interdit. En 2016, la CS a réaffirmé le droit à l'IVG. Il y a, aujourd'hui, deux pôles opposés. Un pôle très conservateur et un autre progressiste. Un membre est mort et le Sénat ne souhaite pas voter sur son successeur nommé par Obama. La prochaine nomination est donc importante. 

\subsection{Le fédéralisme et les relations Centre/Périphérie}

Un État fédéral est un État souverain comme n'importe quel autre État, et les règles de droits prises à l'échelon fédéral s'impose directement à tous les citoyens. Sauf que l'État fédéral se compose d'États fédérés avec lesquels ils partagent des attributs de sa souveraineté notamment le pouvoir législatif. Trois grands principes: la superposition, l'autonomie, la participation. \\
Superposition: deux ordres juridiques superposés, le droit fédéral et le droit fédéré. C'est le droit fédéral qui prime sur le droit fédéré sous le contrôle de la cour suprême. \\
Participation: les États fédérés participent à l'élaboration de la politique fédérale. Ils sont représentés à l'échelon fédéral via le Sénat. \\
Autonomie: les États fédérés fixent leurs statuts, leurs modes de fonctionnement. Ils ont leur propre constitution, et donc leurs propres pouvoirs et juridictions. La répartition des compétences législatives est prévue dans la constitution fédérale. 


Ce modèle a connu un grand succès. Les États les plus imposants sont des États fédéraux: Russie, Canada, Brésil etc. \\
En Europe, le fédéralisme a très peu de succès. L'Allemagne, la Suisse, l'Autriche, la Belgique, sont des pays fédéraux. \\
Le fédéralisme fait partie des fondements du système américain. Il fait pleinement partie du check and balances. 

\subsubsection{L'équilibre initial}

Au départ, il y avait seulement 13 États fédérés. Il y a aujourd'hui, 50 États fédérés. Il pourrait y avoir un jour un 51e État fédéré: Porto-Roco qui est un État libre associé. \\
Ces 50 États fédérés ont certaines prérogatives. Ils participent via le Sénat à la politique fédérale. Chaque État fédéré envoie deux Sénateurs à DC ; ils sont élus au suffrage universel direct. \\
Chaque État fédéré possède sa propre constitution qu'il définit librement. Il est à noté qu'ils sont obligés de respecter une condition: la constitution doit être républicaine. Toutes les constitutions sont globalement une copie de la constitution fédérale. De nombreux présidents américains ont été Gouverneur avant d'être président. Pour briguer les plus hautes fonctions aux US, il vaut donc mieux avoir un ancrage local (Obama était Sénateur). \\
Tous les fédérés ont aussi leur parlement. Ils ont tous, sauf le Nébraska un parlement bicaméral, ils ont leur propre cour suprême, etc.


Petit élément d'innovation, des procédures de démocratie directe sont prévue dans certains États fédérés: référendum et initiative populaire. En même temps que des élections fédérés, beaucoup de référendums étaient tenus, notamment sur le Cannabis. Il existe aussi le recall, la révocation populaire, c'est le gouverneur de Californie qui a été le dernier révoqué, en 2003. \\
Cela permet des choses contrairement au système purement représentatif national. \\
Cependant, les procédures sont souvent détournés par des intérêts économiques et la participation est globalement faible (un tiers des électeurs).


La répartition des matières législatives est à priori très favorables aux États fédérés. L'État fédéré a des compétences d'attribution, l'État fédéral a donc des compétences énumérés: il est compétent pour émettre la monnaie, entretenir une armée et déclarer la guerre, le commerce, conduire la politique étrangère, réglementer les questions de nationalité. Tout le reste relève des États fédérés qui ont donc une compétence de droit commun. \\
La pratique a modifiée les choses.

\subsubsection{Une tendance à la centralisation des pouvoirs ?}

Le phénomène de centralisation a été favorisé par la jurisprudence de la cour suprême. Elle a notamment permis l'extension des pouvoirs fédéraux en créant la théorie des pouvoirs implicites dans l'arrêt McCulloch c/ Maryland en 1819. La cour a jugée qu'en donnant le pouvoir de la monnaie à l'État fédéral, il avait le pouvoir implicite de créer une banque. \\
Par la suite, la cour a interprété de façon très extensive la constitution. Comme l'État fédéral a compétence pour gérer le commerce, elle peut faire des transports sa compétence, de l'industrie et donc par ricochet du droit du travail.


Le facteur financier a aussi favorisé cette centralisation. En 1913, un amendement a permis à l'État fédéral de directement imposé les contribuables, ce qui lui a permis d'avoir des ressources importantes et de les utiliser à sa guise, sans passer par l'intermédiaire des États fédérés. \\
Cela a rendu les États fédérés dépendants de l'échelon fédéral.


Le dernier facteur qui a favorisé cette centralisation, c'est le rôle de superpuissance mondiale que joue les US depuis la seconde guerre mondiale. 


Les États fédérés continuent de légiférer dans les domaines les plus important: droit civil, pénal, peine de mort, etc. Ils ont beaucoup plus de pouvoir qu'un État fédéré allemand par exemple. \\
L'échelon local reste le cadre politique naturel. Les Américains ont tendance à se méfier du pouvoir de l'État fédéral. Elle est particulièrement présente depuis une vingtaine d'année. À l'échelon local, on a vu la création de toutes sortes de groupements qui ont en commun la résistance à l'État fédéral. \\
Le mouvement Tea Party est de ceux là. Il s'est créé à l'arrivé au pouvoir d'Obama. Il attire plutôt les classes moyennes blanches. Il n'est pas vraiment structuré, il y a des mouvances libertaires, religieux etc. Leur point commun est le rejet de la gouvernance de Washington. La membre la plus connu est Sarah Palin. Ils ont une soixantaine de siège au congrès fédéral. Depuis 2010, ce mouvement est en déclin, assez divisé, a moins d'influence, mais continue de peser, notamment le parti républicain qu'il tire vers la droite. 

\subsection{Le régime présidentiel et les relations exécutif/législatif}

Dans un régime présidentiel, il y a une indépendance stricte des organes.Chaque organe a sa propre légitimité. L'exécutif et le législatif ne dépendent pas l'un de l'autre. Les mandats sont d'ailleurs incompatible. Ils ne peuvent pas se révoquer mutuellement. \\
Il y a une spécialisation des fonctions. Chaque pouvoir a sa fonction. Cela dit, il n'y a pas de séparation totale, il faut qu'il y ait des moyens d'actions réciproques. Chaque pouvoir doit pouvoir neutraliser l'autre pour éviter les abus. Chaque pouvoir doit coopérer avec l'autre parce que sinon, c'est le blocage. \\
En 1796, ça s'est imposé aux constituants car ils se méfiaient de tous les pouvoirs. Ils ont cherchés à rompre avec le modèle britannique. Ils ont involontairement repris quelques éléments du régime britannique. 

\subsubsection{Le Président}

L'exécutif est monocéphal aux États-Unis. Le Président est, à lui tout seul, tout le pouvoir exécutif. Il exerce donc le pouvoir réglementaire, chargé d'exécuter les lois, les promulgues, dirige l'administration, est responsable des affaires étrangères, il est commandant en chef des armés. \\
Dans la pratique, il est entouré d'une administration qui dépend entièrement de lui. Il y a d'abord les secrétaires d'État, ce n'est pas un gouvernement comme on l'entend, ce n'est pas collégial ou solidaire, ce sont des collaborateurs personnels du Président. "7 contre, 1 pour, les pour l'emporte" Abraham Lincoln sur l'abolition de l'esclavage. \\
Chaque secrétaire est nommé par le Président avec avis conforme du Sénat. Seul le Président peut révoquer ses secrétaires. 


"White house office", le bureau de la maison blanche: ce sont les conseillers, les assistants du Président. Ils sont 300-400 personnes. Ils sont officieux, et concurrence parfois les secrétaires d'État. \\
Des organismes sont sous le contrôle direct du Président: le conseil de sécurité national, le bureau du budget, etc. \\
Toute l'administration est contrôlé par le président. Tous les fonctionnaires sont nommés par le Président, avec avis conforme du Sénat. 


Spoils system: nommer des fonctionnaires de son bord politique pour être sûr d'avoir des fonctionnaires loyaux. \\
Depuis la fin du XIXe siècle, la Loi prévoit des procédures de recrutements au mérite. Il y a depuis une liste d'emplois qui doivent être accessible par concours. La grande partie des fonctionnaires sont recrutés par concours aujourd'hui, mais les plus hauts fonctionnaires restent nommés par le Président. 


Autre atout du Président: il possède sa légitimité propre, le Président étant élu par le peuple. \\
Pour être président des US, il faut être américain de naissance, avoir au moins 35 ans, et résider sur le territoire des US depuis au moins 14 ans. Pour être élu, il faut être candidat et pour être candidat il faut avoir été investi par un des petits partis. Les petits partis peuvent aussi présenter des candidats, sauf que ceux-ci n'ont aucune chance d'être élus. Sauf que les petits candidats perturbent parfois le jeu traditionnel, comme en 1992 où Ross Perot a réussi à obtenir 20\% des votes populaires et du coup, il a contribué à faire gagner Bill Clinton. \\
Premier temps d'une élection: sélection des candidats. Deuxième temps: élection du Président. \\
Le premier temps se déroule de Janvier jusqu'en Juin, le but étant d'avoir un seul candidat Républicain et un seul candidat Démocrate. Quand le Président en place se représente, il est réinvesti presque automatiquement. Les procédures mises en oeuvre ne sont pas prévues par la Constitution, mais se sont imposés dans la pratique: primaire et caucus. C'est l'État fédéré qui choisit entre les deux et qui organisent. Le système de vote change selon les États, tout comme l'ouverture (fermé, seulement membre du parti, ouvert, tous les électeurs, semi-ouvertes, les électeurs s'inscrivent à l'avance comme sympathisant d'un ou l'autre parti). Cela rallonge la durée de la campagne électorale et c'est en plus très coûteux sans vraiment qu'il y ait d'enthousiasme et comme il y a peu de participation, ce sont les plus radicaux qui vont voter. \\
Le deuxième temps est tout aussi compliqué car l'élection du Président est indirect. Les citoyens sont appelés à élire des grands électeurs, qui sont égal au nombre de représentants et de sénateurs que comptent l'État. La Californie, le Texas, New York, la Floride représentent le quart des grands électeurs. L'élection des grands électeurs se fait par scrutin de liste majoritaire à un tour: on appelle cela le "winner takes all". Le Président est officiellement élu par les grands électeurs à la mi décembre. Il faut recueillir la majorité absolu des voix des grands électeurs. Si il n'y a pas de majorité absolu, le Président est élu par la chambre des représentants et le vice président par le Sénat parmi les candidats les mieux placés. La Constitution ne précise pas qu'un grand électeur doit voter pour son candidat. La plupart des États fédérés considèrent qu'un grand électeur qui trahit la confiance de ses électeurs est illicite mais ne prévoient pas de sanctions. 


\subsubsection{Le Congrès}

Le congrès est composé de deux chambres, la chambre des représentants et le Sénat. Le bicaméralisme est ici une conséquence du fédéralisme. Il est égalitaire dans le domaine législatif et budgétaire. Les deux chambres doivent voter les lois en termes identiques. Beaucoup de lois échouent car les deux chambres sont incapable de voter des lois en même terme. \\
Il y a 435 représentants et 100 Sénateurs. Les représentants et sénateurs doivent être investi par leurs partis lorsqu'ils se présentent. La durée d'un mandat de représentant est de 2 ans. Les représentants sont sans cesse en campagne. \\
Les sénateurs sont plus prestigieux, la durée du mandat est de 6 ans. Mais il est renouvelé par tiers tous les 2 ans. Il est beaucoup moins facile de se faire réélire. \\
La discipline partisane est très faible aux États-Unis. Dans les deux chambres, le temps de parole n'est pas limité. Cela permet la "flibuste" où l'on peut parler indéfiniment pour faire obstruction. Le temps record d'obstruction est de 24h et 18 min. Pour éviter cela, quand il n'y a pas 60 voix pour éviter de laisser parler un sénateur, le Sénat ne vote pas le texte. Cela a permis une profusion des obstructions. 

\subsubsection{Une indépendance relative}

Les fonctions sont bien spécialisées, les pouvoirs sont bien indépendant, c'est donc purement un régime présidentiel. Cependant, le système serait instable si il n'y avait pas un minimum de collaboration. Ces relations sont d'autant plus nécessaires qu'il n'y a aucun moyen constitutionnel de résoudre les conflits. Beaucoup de présidents ont été confrontés à une sorte de cohabitation, avec un congrès dominé par le parti adverse après les mid-term elections. Obama quand il a été réélu, dès le début, il a été confronté à un congrès à majorité républicaine. \\
Depuis 1950, les USA ont connu 40 années de "cohabitation". Le Président peut s'immiscer dans la procédure législative. Il n'a bien sûr pas l'initiative des lois. Dans les faits, la majorité des lois émanent de l'administration, en donnant les textes de lois aux parlementaires de son parti. D'autres part, le Président doit informer périodiquement le congrès sur l'état de l'Union et qu'il doit recommander à l'attention du congrès les mesures qu'il estime nécessaire ; la coutume veut donc que le Président aille prononcer son discours sur l'état de l'Union tous les ans, en début d'année, devant le congrès. À noter que le Président a l'initiative de proposer le budget. \\
Le congrès a pris l'habitude depuis 1920, d'autoriser le Président à adopter certaines normes (équivalent des ordonnances en France). \\
Enfin, le Président doit promulguer les lois par le congrès. Mais il peut refuser de les promulguer (il a 10 jours pour notifier son refus), le congrès peut surmonter le veto en re-votant la loi à la majorité des deux tiers. Francklin Roosevelt détient le record du nombre de veto. Obama en a usé beaucoup lors de son deuxième mandat. Il existe aussi le veto de poche.


Réciproquement, le congrès a énormément de prérogatives pour bloquer l'exécutif. Son arme principale est l'argent, car il est le seul compétent pour voter le budget. En 1973, par exemple, le congrès a refuser de voter les crédits militaires pour continuer la guerre du Vietnam. Aujourd'hui, le congrès refuse de voter les crédits pour fermer Gitmo. \\
Le shutdown est un blocage du congrès sur le budget pour bloquer la réforme de la santé. Cela a fait fermé un grand nombre d'institutions fédérales. \\
Le congrès peut vraiment contrôler ce que fait l'exécutif grâce à ses commissions. En dehors des commissions régulières, ils peuvent créer des commissions d'enquête qui sont, aux US, très puissantes. Ils peuvent faire comparaître n'importe qui, et ne pas répondre à leurs questions est un délit fédéral. \\
Le congrès est aussi le seul à pouvoir déclarer la guerre. La pratique fait qu'un Président peut engager des troupes dans déclarer la guerre. Le congrès a donc adopté le "War Powers Act" qui oblige le Président à informer le congrès dans les 48h si des troupes sont engagés et il doit demander l'accord du congrès pour poursuivre les opérations au delà de 60 jours. Le Président a, dans la pratique, déjà outrepassé cette loi, en Libye. \\
Le Sénat a un pouvoir de nomination puisqu'il doit donner son avis conforme aux nominations du Président. Il peut donc bloquer des nominations et aussi mener des enquêtes approfondis sur les personnes nominées. \\
Le Sénat doit donner son autorisation, aux deux tiers, pour ratifier les traités. Il avait empêché de ratifier, en 1919, le traité de Versailles instituant la SDN. Plus récemment, il a empêché la ratification d'un traité qui interdit les tests nucléaires. Pour contourner ça, le Président utilise des executive agreements. \\
Enfin, le congrès peut destitué le Président via la procédure d'impeachment. Tous les agents publics fédéraux civils peuvent faire l'objet d'une procédure d'impeachment. Cette procédure se fait en deux temps, la mise en accusations par la chambre des représentants ; deuxième temps, le Sénat juge et vote à la majorité des deux tiers (c'est le Président de la Cour suprême qui préside le Sénat à ce moment là). Aucun président n'a été destitué mais il y a eu des tentatives contre Tyler et Johnson au XIXe siècle, ces tentatives étaient politiques. Nixon a failli se faire destituer avec le scandale du Watergate. B. Clinton a aussi fait l'objet d'une procédure d'impeachment, quatre chefs d'accusation étaient retenus: parjure (mensonge sous serment), obstruction à la justice (destruction de preuve), obstruction de témoin, abus de pouvoir. La majorité requise n'a pas été atteint au Sénat. \\
Cette dernière procédure peut montrer une certaine crise du système américain, où la procédure a été détourné pour engager, en quelques sortes, la responsabilité politique du Président. 

\subsubsection{Qui dirige vraiment les États-Unis?}

Les constituants voulaient instaurer l'équilibre le plus parfait possible. Selon les périodes, un pouvoir avait pris le dessus. Jusqu'au XXe siècle, l'exécutif était plutôt faible sauf exceptionnellement avec quelques personnalités comme Washington ou Lincoln. De plus, le congrès essayait d'user au maximum de ses prérogatives. La Cour Suprême a aussi essayé de se formé comme un pouvoir. \\
La tendance s'est inversée avec les deux guerres mondiales. Ce n'est pas une caractéristique américaine, toutes les démocraties libérales ont connu ce phénomène. On peut évoquer de manière générale la complexité croissante des affaires publiques ou encore la personnalisation du pouvoir via les médias qui mettent le Président sur le devant de la scène. \\
Les US se sont imposés comme superpuissance et l'exécutif semble être la seule à pouvoir relever ce défi. Wilson a profité de la première guerre mondiale pour avoir une politique autoritaire. Roosevelt a concentré des pouvoirs grâce à la crise, et à la guerre par ensuite. \\
Trueman, Eisenhower, ont défendu eux aussi un exécutif fort. Cela a conduit à une critique de la présidence. Le Président le plus contesté à ce jour est sans doute Nixon. Il a passé son temps à essayer de museler les autres pouvoirs constitués. \\
Depuis la démission de Nixon, il y a eu un nouvel équilibre, avec des personnalités plus modestes. Ronald Raegan avait de nouveau une vision très autoritaire du pouvoir exécutif, mais un scandale l'a empêché d'avoir autant de pouvoir qu'il voulait. Georges W. Bush a profité du 11 Septembre pour concentrer des pouvoirs. \\
Obama est plutôt, de fait, un Président faible, constamment en guerre contre le Congrès. 


De manière générale, on peut dire que la Présidence des US, est la plus forte, car c'est lui qui personnifie l'Union, c'est le pouvoir le plus centralisé. Mais il a pas autant de pouvoirs qu'un président Français par exemple. Le Président US est toujours obligé de composé avec le congrès, il a en face de lui des pouvoirs qui peuvent l'encadrer. 

\section{Acteurs et vie politique: la démocratie américaine et ses contradictions}

\subsection{Les partis politiques}

Le bipartisme américain est particulier. On peut parler de bipartisme quand il existe deux partis puissants et structurés, qui totalisent à eux deux 80-90\% des votes. Depuis le milieu du XIXe siècle, tous les élus ou presque sont soit Républicains, soit démocrates. Ils ont un quasi monopole du mode de scrutin. \\
Le bipartisme s'explique essentiellement par le mode de scrutin: majoritaire à un seul tour. \\
Contrairement à ce que l'on croit, il existe de nombreux petits partis aux US. À gauche, le parti le plus important est le parti vert, ils présentent des candidats et ont une centaine d'élus à l'échelon local. À droite, un petit parti est celui de la réforme. Le parti libertarien est plutôt fort, il est ultra-libéral mais plutôt libertaire aussi, même si il est inclassable, il est très proche du Parti Républicain.  \\
Les petits partis ne peuvent pas espérer conquérir le pouvoir. 


Il y a très peu de discipline partisane, ce qui fait des Républicains et des Démocrates des partis très proches, mais qui restent inclassable sur un axe gauche/droite.

\subsubsection{Genèse et évolution des partis républicains et démocrates}

Au début de la démocratie US, il y avait deux courants d'idée qui s'opposaient, qui correspondaient à deux classes sociales. Il y avait les fédéralistes, qui étaient favorable à une centralisation du pouvoir, qui trouvait ses alliés auprès des classes dominantes. L'autre courant prônait au contraire la défense de l'échelon local face à l'échelon fédéral. Les anti-fédéralistes ont fondés un parti "Républicain-démocrate" qui comprenait des paysans et des planteurs du Sud des US. \\
Au début du XIXe siècle, les Républicains-démocrates sont devenus les "Démocrates", les fédéralistes ont fondés le parti "Républicain". À partir de là, le bipartisme n'a pas été remise en cause, mais les idées des deux parties ont beaucoup évolué, et se sont même inversé. \\
Au XIXe siècle, pendant la guerre de sécession, les Républicains étaient des progressistes, et étaient contre l'esclavage. Les démocrates ont connu des contradictions. \\
Au début du XXe siècle, le parti démocrate est devenu le défendeur de l'État providence, les Républicains ont donc commencé à réclamés plus d'autonomie pour les États fédérés. \\
Dans les années 60, avec Kennedy, le parti démocrate est devenu le défenseur des droits des minorités alors que les républicains sont devenu le parti des WASP. 

\subsubsection{Une "alternance sans alternative" ?}

Il y a quand même un consensus sur les grands principes entre les républicains et les démocrates. Les deux adhèrent aux institutions en place et aux droits en place. \\
Aujourd'hui, les républicains sont plus méfiants à l'interventionnisme internationale et prône baisse d'impôts et baisse des dépenses publiques. Ils sont beaucoup attachés à la défense des valeurs familiales ou à la répression pour la sécurité. Cette ligne là s'est affirmé avec Raegan qui voulait réaliser la "Révolution conservatrice", c'est pour ça qu'il est la référence. Ils sont assez divisé depuis la défaite de 2008 et sont divisés entre la ligne dur (Cruz, Ryan), et la ligne plus pragmatique, qui estime que le parti doit se recentrer (Jeff Bush, McCain). \\
Les démocrates ont une ligne centriste qui domine depuis des années. Depuis Roosevelt et Kennedy, il y a un discours qui s'attache à la défense des pauvres et des minorités. En 2003, près de la moitié des démocrates ont votés pour l'intervention en Irak. Les démocrates ne se sont jamais présentés comme représentant de la lutte des classes. La ligne centriste était très présente avec Clinton, qui avait aussi repris une grande partie des idées de ses adversaires pour qu'ils se replient sur des idées non crédibles. Obama avait l'air de l'aile gauche des démocrates, mais a plutôt gouverner au centre. \\
Sur le long terme, les deux grands partis ne s'opposent pas de manière radicale. Les clivages se situent plutôt à l'intérieur de chaque parti. Les clivages dans les partis correspondent aux clivages géographiques. Les structures des partis sont assez décentralisés, et bénéficient d'une forte autonomie, ils se retrouvent seulement à la convention tous les quatre ans. Cela explique la discipline partisane inexistante. \\
Au final, le choix est donc très limité. 


\subsection{Les groupes d'intérêts et le lobbying}

On appelle groupes d'intérêts toute organisation qui a pour but de représenter les intérêts de sections particulières de la société dans le but d'influencer le pouvoir politique. Dans les pays anglos-saxons, on appelle cela les lobbbys (couloir/antichambre car cherche à occuper les couloirs pour parler aux représenter). 

\subsubsection{Typologie et moyens d'action des groupes de pression}

Aux US, ces groupes ont une existence légale et doivent se déclarer. Des lois ont essayés d'encadré ce genre de pratique, mais ne sont pas arrivé au bout de la procédure à cause du premier amendement. Ils ont tout de même, après la 2nde guerre mondiale, adopté une loi qui contraint les lobbys à la transparence. Les lobbys doivent s'enregistrer auprès des services du congrès. Ces exigences ont été augmentées dans les années 90, les lobbys doivent respecter des règles éthiques. Obama a adopté un décret pour éviter les conflits d'intérêt dans l'administration: les anciens lobbyistes ne peuvent intégrer l'administration et les anciens fonctionnaires ne peuvent pas aller chez les lobbys avant un certain temps. \\
Les lobbys les plus visibles sont les syndicats qui représentent les salariés: l'AFLCIO. Depuis l'époque de Roosevelt, ces syndicats sont proches du parti démocrate. Les syndicats agricoles sont très puissants aussi. Il n'y a pas vraiment de syndicat du patronat, mais il y a la "Business Round Table", qui, informelle, est quand même puissante. L'AIPAC est un lobby très puissant qui défend la cause Israélienne. Il y a un ensemble de cabinet professionnels qui vend leur compétence de lobbyiste. 


Ces groupes interviennent dans la procédure législative. Leur principal canal d'action sont les commissions législatives, car ils s'efforcent d'être auditionnés par celles-ci. Parfois, ils aident à écrire les propositions de lois, voire même, les écrivent totalement. Le congrès est la cible privilégié. \\
Ils agissent en amont des élections aussi, en finançant les candidats susceptibles de défendre leurs intérêts ou en faisant des campagnes contre ceux qui ne défendent pas leurs intérêts. Les ressources financières sont très importantes pour une campagne, depuis 1950, le candidat qui gagnait était toujours celui qui dépensait le plus. Pour les élections fédérales, il y a un financement public de possible. Les candidats acceptant ce financement ont un plafond et ne peut plus recevoir d'argent de fonds privés. Alors que si il refuse ce financement, il n'a pas de plafond. \\
PAC, Polical Action Commitee sont des structures de droit privés par lesquels l'argent des campagnes transitent. Elles doivent s'enregistrer auprès de la commission de contrôle électoral. Ce sont seulement des individus qui peuvent donner à ces PAC, donc les dirigeants des groupes de pression donnent personnellement au PAC. Chaque individu peut donner 5000\$ au maximum à un PAC. Pour contourner, ils peuvent donner de l'argent à un parti pour défendre ou contrarier une cause. Ils peuvent financer des études, etc. C'est ce qu'on appelle la "soft money". Des publicités négatives peuvent aussi être financés. \\
En 2005, le congrès a adopté une loi pour réglementer ces financements indirect, mais elle a été en grande partie censurée par la cour suprême en 2010. Elle a décidée que les groupes divers étaient totalement libre de financer des publicités politique ou des événements liés aux élections tant qu'il ne soutenait pas ouvertement un candidat précis. \\
En 2012 sont nés les "super PAC" qui reçoivent de l'argent sans limite pour financer du militantisme politique sans soutenir un candidat précis. 

\subsubsection{Un problème de légitimité ?}

Tout cela semble montrer qu'il y aurait un problème, mais faut-il condamner le lobbying en général ? La conception française fait qu'on se méfie des groupes intermédiaires, entre le citoyen et le pouvoir politique. \\
Aux US, les groupes de pression sont considérés comme légitime et fonctionnelle, sont vus comme un gage de pluralisme, permettent une vrai participation de la société civile au débat politique. La vraie question est donc la légitimité des lobbying et si l'accès aux centres de pouvoir sont également ouverte à tous les groupes. \\
Pour certains chercheurs, les lobbys sont suffisamment nombreux pour que la sphère politique reste indépendante, où les lobbys, contradictoires, se neutralisent. Pour d'autres, même si il y a un grand nombre de lobbys, leur pouvoir différent fait qu'il ne s'annule pas du tout entre eux.

\subsection{"Opinion", participation politique et comportements électoraux}

L'opinion publique tel qu'on l'utilisera ici reflète les relations qu'ont les dirigeants avec les citoyens. Ensuite, la participation politique ne se résume pas à la participation électorale, même si c'est un marqueur majoritaire. 


\chapter{La Grande-Bretagne}

\section{Un État attaché à sa souveraineté et à son unité}

\subsection{Le "Brexit": une rupture inéluctable ?}

L'accord négocié avant le référendum du Brexit a permis à Cameron de se montrer comme grand vainqueur et pouvoir faire campagne pour le maintien dans l'UE. Officiellement, les travaillistes étaient pour, les conservateurs ont choisi la neutralité pour éviter l'éclatement. La majorité des conservateurs ont fait campagne pour le pour, mais d'autres ont fait campagne pour le contre, notamment Boris Johnson, chef de file pour le Brexit. \\
Ce référendum peut être interprété comme un vote anti-élite, la majorité de la classe politique étant pour le maintien, les élites économiques et syndicales ont aussi fait campagne pour un maintien. Le vote leave a gagné avec 52\% des voix. \\
Les conséquences sont aujourd'hui très incertaine. La première incertitude est la date pour quitter l'UE. Juridiquement, depuis le traité de Lisbonne, son article 50 permet un leave de l'Europe. Il laisse 2 ans à l'État membre quand il a notifié l'usage de cet article pour négocier le leave. Pour le moment, l'article 50 n'a pas encore été invoqué ce qui donne des incertitudes néfastes: sur l'économie ou encore qui va négocier (élections en France et en Allemagne). \\
Cameron refusait d'envisager un tel scénario, et l'administration n'a aucun expert pour travailler sur ce sujet sensible. \\
Les négociations vont buter sur le fait que le RU veut garder un accès au marché unique sans en subir les conséquences. Cela annoncerai un "hard brexit" avec des relations réduites au strict minimum. \\
Il semble tout de même que le Brexit aura des conséquences majeures: le RU va perdre une grande partie des aides européennes, il va y avoir des conséquences sur les échanges d'étudiants, pour certains, ce départ permettrait de faire avancer l'intégration, pour d'autres, cela accélérerait la dislocation de l'UE. \\
En principe, le RU devrait retrouver sa souveraineté, notamment celui de son Parlement. Or, cette souveraineté a été remis en cause par le référendum lui même, s'exprimant à la place du Parlement. En principe, le référendum sur le Brexit n'est pas contraignant juridiquement, la décision finale devrait appartenir au Parlement. Le Gouvernement a persévéré dans la mise en oeuvre du Brexit, sans l'accord du Parlement, et la haute cour a dénié au Gouvernement le fait de pouvoir lancer ce Brexit. \\
Les autres États constitutifs du RU ont votés contre le Brexit, risquant de poser quelques soucis. 

\section{Le Royaume-Uni et ses quatre nations}

Le RU est un État unitaire pluri-national, mais il y a des conflits qui ne sont pas encore résolus aujourd'hui. L'Irlande du Nord a par exemple connu pendant 30 ans une guerre civile assez sanglante pendant que le Pays de Galles et l'Écosse avait des relations plutôt pacifiés avec les Anglais. \\
L'État unitaire possède une seule organisation judiciaire, une seule hiérarchie de norme. Par contre, il existe une certaine décentralisation.

\subsection{Une tradition de gouvernement local}

Cette décentralisation consiste en un transfert de compétence envers des collectivités locales, disposant d'un pouvoir administratif, déterminé par la Loi nationale, subordonné à celle-ci. Cela découle du principe de souveraineté parlementaire, le Parlement de Londres doit avoir le monopole de la Loi. Cependant, le RU n'a jamais été aussi centralisé qu'en France et a toujours toléré une forme de Gouvernement local. Le RU reposait sur les élites locales pour appliquer le droit britannique alors qu'en France, on envoyait des préfets. \\
Cette autonomie locale relevait d'abord de la coutume puis, au XIXe siècle, le Parlement a pris une Loi clarifiant les choses: gouvernement local élu, ayant pour mission de gérer un certain nombre de services publics et pourrait percevoir des impôts locaux. \\
Plusieurs échelons: Comtés, Bourgs, Districts. Ils se sont vus délégués un certain nombre de compétences administratives: police, santé, etc. \\
Juridiquement, le Parlement londonien peut défaire les prérogatives locales. Elles ont d'ailleurs été considérablement réduites dans les années 80 par Thatcher: conseils locaux supprimés dans les grandes villes, échelons locaux ont perdus certaines compétences (éducation), voir ont été transférés vers des entreprises privés. \\
Depuis quelques années, des réformes sur le statuts de certaines nations remettent en cause le caractère unitaire du RU.

\subsection{Le Pays de Galles et l'Écosse}

C'était au départ des nations indépendantes. Ils ont étés intégrés assez tôt dans l'ensemble britannique: le Pays de Galles à la fin du XIIIe siècle, et politiquement soumis assez vite: aujourd'hui, les revendications portent sur la langue et la culture. L'Écosse a été rattaché au RU au XVIIe siècle, au hasard d'une succession, cela a été formalisé un siècle plus tard avec l'acte d'union, fusionnant les deux parlements. \\
L'Écosse a été le théâtre de plusieurs mouvements autonomistes voir indépendantistes, notamment par le Scotish National Party (SNP). \\
Quand Tony Blair est arrivé au pouvoir, il a accepté une dévolution de compétences à l'Écosse et au Pays de Galles, en 1997. Cela a été accepté par les deux nations par référendum local (ce qui était purement consultatif). \\
L'Écosse a un parlement élu pouvant adopter des lois dans un certain nombre de secteurs. Ils n'ont pas de compétences dans le domaine fiscal. L'application de ces lois ont été attribués à des gouvernements locaux qui ont un premier ministre (first minister). \\
Si cela ressemble à un État fédéral, la loi de dévolution peut être remise en cause par le Parlement de Londres: il n'y pas de Constitutions de ces nations. \\
Au Pays de Galles, c'est une assemblée, mais ont beaucoup moins de compétences qu'en Écosse. 


Dans un premier temps, cette dévolution a freiné les revendications nationalistes. Dans le cas de l'Écosse, cela n'a pas duré. En 2007, le SNP est arrivé en première position aux élections régionales. Le chef de ce parti, Alex Salmond est devenu premier ministre de l'Écosse. Il en a profité pour faire voté une loi par son parlement organisant un référendum local pour l'indépendance de l'Écosse. \\
Cameron a accepté la tenue du référendum. Il a eu lieu en 2014, le non l'a emporté à plus de 55\% des voix. Salmond a démissionné à la suite de cela et a été remplacé par Nicola Sturgeon. Suite à cette défaite, le SNP et le gouvernement londonien, ont accepté de ne plus tenir de référendum sur ce sujet pendant au moins une génération. \\
Une autonomie supplémentaire a été donné à l'Écosse, ce qui a privé le parlement londonien de cette compétence, ce qui gêne les anglais, car les écossais ont des députés à Londres où ils votent sur des questions anglaises. \\
En 2015, les conservateurs ont fait voter une loi "English votes for English laws", les députés Anglais ont un droit de veto sur les PJL qui ne concernent que l'Angleterre. En conséquences, certains parlent de créer un Parlement spécialement anglais, ce qui serait un pas supplémentaire vers une forme de fédéralisme. 


Le cas du Brexit a relancé les mouvements nationalistes des écossais (ils étaient 77\% d'écossais à être pour le maintien). Sturgeon menace de faire un nouveau référendum, pendant qu'elle essaye de négocier avec des pays membres de l'UE, ce que certains pays refusent. \\
Il sera difficile pour écossais d'avoir leur indépendance sans l'accord du Parlement londonien. Autre couac pour les indépendantistes, malgré le Brexit, la cause indépendantiste n'a que très peu progressé depuis 2014. 

\subsection{Le cas de l'Irlande du Nord}

Toute l'Irlande avait été conquise par les anglais entre le XII et le XVIIIe siècle. Mais les irlandais ont continués de lutter pour leur indépendance. En 1921, il y avait eu une semi indépendance de l'Irlande, qui avait pour chef d'État le Roi d'Angleterre, mais était globalement indépendant de lui, sauf 6 comtés du Nord. \\
En 1937, l'indépendance totale a été obtenu, c'est l'Eire, la République d'Irlande, la tutelle du monarque du RU a cessé. Mais les 6 comtés du Nord sont restés sous le joug du RU. Ils appellent cette région l'Ulster. \\
Parmi les Irlandais du Nord, il y a toujours des nationalistes virulents, catholiques, et en parallèle, les descendants des colons écossais et anglais, protestants, favorable à la monarchie. \\
La communauté protestante a toujours été plus nombreuses que la communauté catholique, d'où le fait que les comtés soient restés sous le joug du RU. Avec les avancés démographiques, cela pourrait se renverser à moyen terme. \\
Les relations sont exacerbés par des partis politiques disposant de branches armées comme le SinnFein. \\
Les partis qui n'ont aucunes appartenances confessionnelles n'ont aucun électorat, ce qui montre la présence d'un problème. \\
Les américains d'origine Irlandaise, très nombreux, ont fait en sorte que les US jouent un rôle d'arbitre. \\
Dans la constitution de l'Eire, il y avait un article qui prévoyait la réunification, il a été supprimé et le RU a donné à un droit de regarde à l'Irlande.


En 1969, une guerre civile éclate, qui finira en 1998. L'armée britannique a exercée une répression violente. Les extrémistes de chaque camp s'attaquaient à leurs propres membres pour les dissuader de côtoyer "l'ennemi". \\
Tony Blair a mis en place un désarmement parallèle des deux camps. Il a accepté la négociation avec le Sinnfein. Des accords ont étés trouvés et approuvés largement par référendum local. Il n'y a pas d'indépendance, mais une forte autonomie, comparable à celle de l'Écosse avec un premier ministre et une assemblée. \\
Des mesures ont été prévues pour assurer l'égalité entre les catholiques et les protestants. Le Parlement est issu à la proportionnelle intégrale, obligeant une coopération forcée avec l'autre moitié confessionnelle. \\
L'accord a rétabli la paix. Cependant son application s'est heurté à certains problèmes. Il a fallu attendre 2006 pour que l'IRA se désarme et pour que catholiques et protestants gouvernent ensemble, il faut attendre 2008 pour la dévolution des pouvoirs de police soient dévolus et pour que les troupes britanniques se retirent. \\
Si la guerre est finie, la violence n'a pas totalement disparu, notamment à l'occasion de manifestations religieuses. \\
Le fossé existe toujours et est profond: éducation séparé, mariage mixte rare, relation de voisinage inexistantes (ségrégation de fait). 


Le Brexit a là aussi renforcé la cause des indépendantistes, les Irlandais du Nord ayant votés en faveur du maintien. Certains protestants sont même en faveur de cette indépendance. \\
Le leave de l'UE remettrait aussi le processus de paix, l'UE permettant des contacts entre les deux Irlande, qui ne seraient plus possibles post-Brexit.

\section{Une monarchie parlementaire}

Le RU est devenue parlementaire au XVIIIe siècle. Le régime est moniste depuis le XIXe siècle. 

\subsection{Les institutions symboliques}

\subsubsection{La couronne}

À l'origine, c'était la seule institution à détenir la souveraineté, ce n'est plus le cas. Cependant, ce n'est plus le cas mais reste au centre de la vie politique (elle est menée au nom de la couronne). Le monarque n'a plus aucune influence réelle sur la politique. Elle est politiquement irresponsable et doit être neutre. \\
La reine a des pouvoirs nominaux, elle les a en théorie mais ne les exerces pas dans les faits: elle nomme le premier ministre et les ministres. La coutume l'oblige à choisir comme PM le chef du parti majoritaire à la chambre des communes. Une fois un gouvernement établi, tous les actes de la reine sont soumis au contreseign des ministres ou du PM. \\
Le Roi disposait d'un droit de veto qui n'a plus été utilisé depuis le XVIIIe siècle, elle est donc obligée de promulgué les lois. Elle ouvre les sessions, etc. \\
Elle dispose en théorie de la dissolution de la chambre des communes mais selon la coutume, elle le fait sur la proposition du PM. \\
Même décerné des titres de noblesses, elle ne peut pas le faire de son propre chef. \\
Elle est gouverneur de l'Église anglicane, mais c'est là aussi le PM qui lui dit qui nommer. L'anglican est d'ailleurs la religion officielle en Angleterre. Sont donc exclue de la succession royale tout non anglican. \\
La Reine a un rôle de conseil au PM. Elle le voit tout les mardis à l'heure du thé. 


On peut donc se demander pourquoi il y a encore une reine alors qu'elle ne sert à rien. \\
Le monarque sert à incarner la continuité du pouvoir, l'intemporalité, l'unité de la Nation, etc. Il est censé aussi jouer un rôle de garde fou par son autorité morale, qui peut parfois être décisive comme en Espagne où le Roi a joué un rôle dans la transition démocratique, ou comme en Belgique où le Roi est le seul symbole de l'unité. \\
Dans un contexte de crise économique, la permanence de la monarchie peut avoir quelque chose de rassurant. \\
Les règles de succession ont récemment changés pour que les femmes héritent de la même manière du trône. De même pour flexibiliser l'obligation d'être anglican. \\
Si la Reine meurt, ce serait le prince Charles, 67 ans, qui hériterait du trône. \\
Globalement, la monarchie est acceptée au RU. 

\subsubsection{La chambre des lords}

Elle a survécu à travers les siècle un peu comme la monarchie. Cependant, elle n'a aucun pouvoir, normal car elle n'a aucune légitimité démocratique. Tony Blair a changé la composition de la chambre. Il y avait quatre catégories de lords: 700 lords héréditaires, 26 lords spirituels, 12 lords légistes, 300 lords nommés à vie par la Reine sur proposition du PM. \\
Tony Blair a voulu rendre la chambre plus moderne, il a procédé par étape. Il a réduit le nombre de lords héréditaires, passant de 700 à 92, il a enlevé les lords légiste, une cour suprême ayant été créée. 800 lords sont désormais nommés à vie. \\
Blair voulait supprimer entièrement les lords héréditaires, et donc rendre cette chambre pas du tout légitime du tout. Il était question de l'élire, mais elle aurait été aussi légitime que la chambre des communes, ce qui aurait posé problème. Cameron voulait renommer la chambre des lords en Sénat avec 80\% de lords nommés et 20\% de lords élus. Le PJL n'aura finalement même pas été mis aux votes. \\
Depuis 2014, les lords peuvent démissionné, ou être destitues. 


Au départ, la chambre des lords avait un droit de veto absolue, puis réduit au début du XXe siècle, il n'avait plus ce droit pour les lois de finance, puis il n'était que suspensif d'un an ensuite pour un PJL qu'il refusait. Les lords utilisent ce droit de veto avec parcimonie, leur existence étant précaire. \\
Pour certains auteurs, la chambre des lords a son utilité car dépose des amendements, améliore des texte. \\
Dans les années 50, les lords avaient réussi à enterré un PJL sur l'abolition de la peine de mort pendant 10 ans en ayant usé de leur veto un an avant une alternance politique. \\
Globalement, les britanniques sont d'accords pour réformés cette chambre mais ne savent pas pour quoi la changer. 

\subsection{Les institutions fondamentales: de la collaboration des pouvoirs au "Gouvernement de cabinet"}

\subsubsection{La chambre des communes}

C'est une des plus vieilles assemblées parlementaires qui existent aujourd'hui. Elle conserve d'ailleurs des modes de fonctionnement qui peuvent être archaïque. Les députés sont assis face à face, opposé entre majorité et opposition. Il n'y a pas de vote électronique, pas de délégation de vote. \\
C'est aujourd'hui 650 sièges. Les députés sont élus pour 5 ans. Le scrutin majoritaire à un tour est assez rare en Europe. \\
Les débats sont très organisés par rapport aux US. La common house élit un speaker, qui ne participe pas aux votes mais a beaucoup d'influence. \\
Une fois par semaine, le mercredi, des questions sont posés au PM. Il y a des groupes parlementaires, très disciplinés et encadrés par des chefs de files: les whips. L'indiscipline peut être sanctionnée par des blâmes, des sanctions voir une exclusion des partis, ce qui réduit largement les chances d'être élus. \\
Cette discipline partisane s'est un peu réduite ces dernières années. En 2003, une partie des travaillistes avaient votés contre l'intervention en Irak. Cameron, demandant à intervenir en Syrie s'est fais complètement désavoué par la chambre. La réforme sur la mariage homosexuel, en 2013, est passée grâce aux votes de gauche et des centristes, Cameron est donc allé contre son propre parti. \\
La chambre des communes n'est plus le centre des décisions politiques.

\subsubsection{Le Premier ministre et le Cabinet}

Le PM a une forte légitimité, pas autant que pour un Président élu, mais forte quand même. C'est ainsi que les législatives sont fortement personnalisé, car indirectement, c'est cette élection qui fait le PM. \\
Le PM dirige toujours son parti et son groupe parlementaire, il y a une forte imbrication de l'exécutif et du législatif. Il dirige aussi son gouvernement, choisit ses ministres. Le PM peut révoquer ses ministres de manière assez discrétionnaire. \\
Dans le cabinet, on trouve bien sûr le PM. Entre 2010 et 2015, il y avait un vice premier ministre car les conservateurs n'avaient pas la majorité à eux seuls. \\
Le cabinet compte une vingtaine d'autres personnes, des secrétaires (ministres). Parmi les membres les plus important, on compte le "Chancelier de l'échiquier" qui est le ministre de l'économie et des finances. 

\subsubsection{Les mécanismes de collaboration et de révocation mutuelle}

Le régime britannique repose sur une collaboration constante entre l'exécutif et le législatif. De coutume, il ne peut pas y avoir de ministres qui ne seraient pas élus à la chambre des communes. On peut toujours être au gouvernement si on est à la chambre des lords. \\
Ce statut parlementaire renforce la légitimité des élus et leur permet l'accès à leur chambre. \\
L'initiative des lois est partagée, mais c'est le gouvernement à l'origine de la plupart des textes législatifs. Comme aux US, le Gouvernement a le monopole du PJL Finances. \\
Le Gouvernement a la quasi-certitude de faire adopter ses lois vu qu'il dispose de la majorité.


Il y aussi l'existence d'un droit de révocation mutuelle. Le gouvernement est politiquement responsable devant la chambre des communes depuis le XVIIIe siècle. Cette responsabilité est unie, le Gouvernement doit être solidaire, ce qui explique que les débats du cabinet sont confidentiels. Cette solidarité a été mise à mal par le cas du Brexit. \\
La chambre des communes peut renverser le gouvernement, de manière explicite (motion de défiance) ou de façon implicite en refusant de voter une loi comme le budget. \\
Depuis la seconde guerre mondiale, un seul gouvernement a été renversé, celui de Callaghan, qui a demandé la dissolution de la chambre et les élections ont menés Thatcher à prendre le pouvoir. Plusieurs PM ont quand même démissionné à cause de conflits partisans, comme Thatcher. \\
En 2011, une loi a été adoptée pour limiter les dissolutions tactiques. Le PM peut dissoudre seulement si la chambre a refusé la confiance ou voté une motion de défiance ; elle peut décidé aussi de s'auto dissoudre, aux deux tiers des voix. \\
Aujourd'hui, il y a une concentration des pouvoirs entre les mains de l'exécutif (bi partisme, discipline partisane), la chambre devient une chambre d'enregistrement. 














\end{document}
