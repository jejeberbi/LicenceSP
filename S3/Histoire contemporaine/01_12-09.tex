\documentclass[10pt, a4paper, openany]{book}

\usepackage[latin1]{inputenc}
\usepackage[T1]{fontenc}
\usepackage[francais]{babel}
\usepackage{bookman}
\usepackage{fullpage}
\setlength{\parskip}{5px}
\date{}
\title{Cours d'histoire politique contemporaine (UFR Amiens)}
\pagestyle{plain}



\begin{document}
\maketitle
\tableofcontents

\chapter{Introduction}

Le mot politique est un terme polysémique. Cela se présente comme un ensemble de forces institutionnalisés qui interagissent entre elles dans ce que l'on pourrait appeler le champ politique. Cette fonction de régulation sociale du politique se traduit par la mise en oeuvre de politiques publiques qui sont des dispositifs d'actions publiques qui visent à produire un certain nombre d'effets sociaux. \\
Dans le cadre de ce cours, la définition qui nous intéresse est la politique que l'on entend comme être la scène politique sur laquelle s'affrontent sous les yeux du public et des citoyens une série d'acteurs qui cherchent à conquérir et exercer le pouvoir. \\
Bourdieu: "C'est le lieu où s'engendre dans la concurrence entre les agents qui s'y trouvent engagés des produits politiques entre lesquelles les citoyens ordinaires réduit au statut de consommateur doivent choisir". Telle que défini par Bourdieu, la politique serai un marché, un marché économique. On remarque que pour lui, les citoyens ne jouent qu'un rôle mineur et sont réduits à être des consommateurs. \\
Les marchés politiques sont des lieux où s'échangent des produits politiques (comme un programme) contre des soutiens matériels ou immatériels et évidemment, des votes. Cette métaphore économique est donc tout à fait pertinente: il y a donc un marché politique. \\
Comme dans un marché économique, il y a donc un jeu de l'offre et de la demande. La demande étant l'attente des électeurs et leurs préoccupations. Si le marché politique est véritablement un marché économique au sens strict du terme, les citoyens ont un rôle restreint. Dans la mesure où la rationalité du citoyen politique est bien inférieur à la rationalité du citoyen économique. \\
On va, dans ce cours, s'intéresser à un siècle de vie politique. On va s'appuyer sur de grands événements de la vie politique française, sur les acteurs, sur les évolutions des tendances, des sensibilités politiques, sur les luttes électorales, sur les résultats, sur les enjeux politiques qui se constituent, sur les clivages, sur les règles du jeu.


La concurrence et l'interdépendance politique a lieu sous le joug d'un certain nombre de règles politiques: le système partisan, les coalitions, le système de scrutin, la nature des "trophées" qui sont en jeu (degré de parlementarisme, hyper présidentialisation), les normes constitutionnelles etc. \\
Ces règles ne sont pas univoques. La politique n'est pas réglée une bonne fois pour toutes par les constitutions. Ce qui compte, c'est l'usage qu'en font les acteurs politiques. Ces règles sont susceptibles de changer, d'évoluer. Il ne faut donc pas donner à ces règles plus d'importance qu'elles n'en ont.


Les traditions politiques sont des courants, des forces, des sensibilités qui ont une certaine constance dans le temps. On entendra par tradition plus précisément, la permanence à travers le temps d'un système relativement cohérent de représentations et d'images, de souvenirs et de comportements, de fidélité et de refus. \\
Ces traditions s'incarnent dans des organisations, principalement dans des partis politiques. Ces traditions n'existent pas en soi. Il faut se déprendre de l'illusion qui présente ces traditions comme des essences intemporelles. \\
Les traditions doivent être appréhendés comme des constructions historico-sociales. Ces constructions résultent d'un travail historique. \\
Par ailleurs, les traditions politiques ont des fonctions de légitimation. S'approprier le monopole de l'héritage permet de stigmatiser un opposant politique. 


\chapter{L'invention de la politique moderne}

\section{L'instauration de la République}

Le 4 Septembre 1870, la IIIe République est proclamée par Gambetta suite à la capture de Napoléon III par les Prussiens. La IIIe disparaîtra en 1940 quand les pleins pouvoirs seront donnés à Pétain. \\
La IIIe est une période fondamentale pour comprendre la vie politique française. C'est à partir des années 1870 que se sont posés les fondements de la démocratie représentative telle qu'on la connaît encore aujourd'hui. C'est sous cette République que le régime Républicain s'impose. La pratique du vote se généralise à partir des années 1870, il existait depuis 1848 mais était encadré pendant le second empire. La lutte électorale s'intensifie, des marchés électoraux se créent. La politique va devenir une sphère de plus en plus autonome et va se professionnalisé. \\
Le sentiment d'appartenance nationale progresse.


Comment expliquer ces transformations profondes pendant la IIIe République ? \\
Certaines de ces transformations peuvent être imputés notamment à l'introduction du suffrage universel masculin en 1848. \\
Les régimes d'avant la IIIe sont essentiellement des monarchies. C'est donc le suffrage censitaire qui domine dans ces régimes. Le cens est variable selon les élections: 1 Franc pour les élections municipales, 200 ou 400 Francs pour des élections législatives. Comme tout le monde ne peut pas payer, les marchés électoraux sont restreints, peu concurrentiel, dominés par les notables. \\
Restreints car peu d'électeurs. Les trois quarts des collèges électoraux étaient composés de moins de 600 inscrits. Peu concurrentiel car peu de candidats. Plus de 80\% des députés sont élus avec moins de 400 voix. Enfin, ils sont dominés par les notables qui ont les ressources personnelles qu'ils peuvent investir dans le champ politique. Les biens qu'investissent les notables sont essentiellement non spécifiquement politique (clientélisme).


L'introduction du suffrage universel a contribué à accroître considérablement le nombre d'électeurs. On passe de 250 000 électeurs à plus de 10 millions. Il est à noter que tous n'intègre pas le vote du jour au lendemain. Le suffrage universel ne va réellement que s'épanouir sous la IIIe République. Il en va de même pour les candidats qui vont renouveler leurs stratégies, contraints par leur nouvel électorat. \\
Dans un premier temps, ils cherchent à continuer la relation clientéliste, mais il est impossible de rémunérer l'ensemble du corps électoral. \\
Progressivement, le vote va s'individualiser et devenir l'expression d'une opinion personnelle. Ce processus s'initie sous la IIIe République et s'intensifie via la mise en place de politiques volontariste, pour apprendre aux citoyens à voter. Il va s'individualiser car de nouveaux acteurs vont émerger et concurrencer les notables et contester leurs pouvoirs en proposant des visions politiques et en redéfinissant les relations électorales.

\section{La République dans les moeurs}

La République comme régime s'est progressivement affirmé dans deux directions différentes.

\subsection{La consécration dans la classe politique}

Au cours du XIXe siècle, le régime est un débat central. Près d'un siècle après la Révolution française, les monarchistes sont toujours nombreux et mobilisés. \\
Au départ, la République s'impose comme un régime par défaut puis va commencer à devenir dans les esprits, le régime qui divise le moins. On parlait à cette époque de "République d'attente".


En 1870 éclate la guerre avec la Prusse, la défaite se fait deux mois plus tard avec la capture de Napoléon III. À Paris, se constitue un gouvernement de défense nationale qui proclame la République. \\
Des élections législatives sont donc organisés à la va vite. L'enjeu central est bien évidemment la guerre avec la Prusse. Les Royalistes sont favorables à la paix tandis que les Républicains souhaitent poursuivre la guerre. Les partisans de la République obtiendront 200 députés contre 400 pour les royalistes. \\
C'est ainsi que Adolf Thiers devient le chef du pouvoir exécutif dans cette République. De manière assez paradoxale, l'épisode de la commune de Paris va renforcer la République. Certains considèrent que la commune de Paris est une première manifestation du communisme. À l'époque, la problématique des républicains et des monarchistes est la stabilité du régime. En réprimant un mouvement insurrectionnel, la République a fait sa preuve d'un régime stable. En 1872, Thiers se rallie à la République.


C'est le régime qui divise le moins car les monarchistes sont, à l'époque, très divisés. Il y a les Orléanistes, héritiers de la monarchie de Juillet et la monarchie constitutionnelle et parlementaire, modéré et d'autre part, les légitimistes qui revendiquent les héritages des ultras de la première restauration. \\
C'est pour cette raison que Thiers sera renversé en 1873, et Mac Mahon sera élu président. Les divisions persistent chez les monarchistes et rendent impossible une éventuelle restauration. \\
Ce n'est qu'en 1875 que la République est juridiquement renforcée. Le mot République apparaît dans trois lois constitutionnelles. C'est presque par hasard pendant les débats parlementaires qu'un amendement introduit le mot République dans la loi constitutionnelle. 



\subsection{La "républicanisation" de la société}


\section{Vers une politique en sphère autonome}


\chapter{Le clivage gauche/droite}

\chapter{}













\end{document}
