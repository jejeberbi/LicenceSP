\documentclass[10pt, a4paper, openany]{book}

\usepackage[utf8x]{inputenc}
\usepackage[T1]{fontenc}
\usepackage[francais]{babel}
\usepackage{bookman}
\usepackage{fullpage}
\setlength{\parskip}{5px}
\date{}
\title{Cours d'histoire politique contemporaine (UFR Amiens)}
\pagestyle{plain}



\begin{document}
\maketitle
\tableofcontents

\chapter{Introduction}

Le mot politique est un terme polysémique. Cela se présente comme un ensemble de forces institutionnalisés qui interagissent entre elles dans ce que l'on pourrait appeler le champ politique. Cette fonction de régulation sociale du politique se traduit par la mise en oeuvre de politiques publiques qui sont des dispositifs d'actions publiques qui visent à produire un certain nombre d'effets sociaux. \\
Dans le cadre de ce cours, la définition qui nous intéresse est la politique que l'on entend comme être la scène politique sur laquelle s'affrontent sous les yeux du public et des citoyens une série d'acteurs qui cherchent à conquérir et exercer le pouvoir. \\
Bourdieu: "C'est le lieu où s'engendre dans la concurrence entre les agents qui s'y trouvent engagés des produits politiques entre lesquelles les citoyens ordinaires réduit au statut de consommateur doivent choisir". Telle que défini par Bourdieu, la politique serai un marché, un marché économique. On remarque que pour lui, les citoyens ne jouent qu'un rôle mineur et sont réduits à être des consommateurs. \\
Les marchés politiques sont des lieux où s'échangent des produits politiques (comme un programme) contre des soutiens matériels ou immatériels et évidemment, des votes. Cette métaphore économique est donc tout à fait pertinente: il y a donc un marché politique. \\
Comme dans un marché économique, il y a donc un jeu de l'offre et de la demande. La demande étant l'attente des électeurs et leurs préoccupations. Si le marché politique est véritablement un marché économique au sens strict du terme, les citoyens ont un rôle restreint. Dans la mesure où la rationalité du citoyen politique est bien inférieur à la rationalité du citoyen économique. \\
On va, dans ce cours, s'intéresser à un siècle de vie politique. On va s'appuyer sur de grands événements de la vie politique française, sur les acteurs, sur les évolutions des tendances, des sensibilités politiques, sur les luttes électorales, sur les résultats, sur les enjeux politiques qui se constituent, sur les clivages, sur les règles du jeu.


La concurrence et l'interdépendance politique a lieu sous le joug d'un certain nombre de règles politiques: le système partisan, les coalitions, le système de scrutin, la nature des "trophées" qui sont en jeu (degré de parlementarisme, hyper présidentialisation), les normes constitutionnelles etc. \\
Ces règles ne sont pas univoques. La politique n'est pas réglée une bonne fois pour toutes par les constitutions. Ce qui compte, c'est l'usage qu'en font les acteurs politiques. Ces règles sont susceptibles de changer, d'évoluer. Il ne faut donc pas donner à ces règles plus d'importance qu'elles n'en ont.


Les traditions politiques sont des courants, des forces, des sensibilités qui ont une certaine constance dans le temps. On entendra par tradition plus précisément, la permanence à travers le temps d'un système relativement cohérent de représentations et d'images, de souvenirs et de comportements, de fidélité et de refus. \\
Ces traditions s'incarnent dans des organisations, principalement dans des partis politiques. Ces traditions n'existent pas en soi. Il faut se déprendre de l'illusion qui présente ces traditions comme des essences intemporelles. \\
Les traditions doivent être appréhendés comme des constructions historico-sociales. Ces constructions résultent d'un travail historique. \\
Par ailleurs, les traditions politiques ont des fonctions de légitimation. S'approprier le monopole de l'héritage permet de stigmatiser un opposant politique. 


\chapter{L'invention de la politique moderne}

\section{L'instauration de la République}

Le 4 Septembre 1870, la IIIe République est proclamée par Gambetta suite à la capture de Napoléon III par les Prussiens. La IIIe disparaîtra en 1940 quand les pleins pouvoirs seront donnés à Pétain. \\
La IIIe est une période fondamentale pour comprendre la vie politique française. C'est à partir des années 1870 que se sont posés les fondements de la démocratie représentative telle qu'on la connaît encore aujourd'hui. C'est sous cette République que le régime Républicain s'impose. La pratique du vote se généralise à partir des années 1870, il existait depuis 1848 mais était encadré pendant le second empire. La lutte électorale s'intensifie, des marchés électoraux se créent. La politique va devenir une sphère de plus en plus autonome et va se professionnalisé. \\
Le sentiment d'appartenance nationale progresse.


Comment expliquer ces transformations profondes pendant la IIIe République ? \\
Certaines de ces transformations peuvent être imputés notamment à l'introduction du suffrage universel masculin en 1848. \\
Les régimes d'avant la IIIe sont essentiellement des monarchies. C'est donc le suffrage censitaire qui domine dans ces régimes. Le cens est variable selon les élections: 1 Franc pour les élections municipales, 200 ou 400 Francs pour des élections législatives. Comme tout le monde ne peut pas payer, les marchés électoraux sont restreints, peu concurrentiel, dominés par les notables. \\
Restreints car peu d'électeurs. Les trois quarts des collèges électoraux étaient composés de moins de 600 inscrits. Peu concurrentiel car peu de candidats. Plus de 80\% des députés sont élus avec moins de 400 voix. Enfin, ils sont dominés par les notables qui ont les ressources personnelles qu'ils peuvent investir dans le champ politique. Les biens qu'investissent les notables sont essentiellement non spécifiquement politique (clientélisme).


L'introduction du suffrage universel a contribué à accroître considérablement le nombre d'électeurs. On passe de 250 000 électeurs à plus de 10 millions. Il est à noter que tous n'intègre pas le vote du jour au lendemain. Le suffrage universel ne va réellement que s'épanouir sous la IIIe République. Il en va de même pour les candidats qui vont renouveler leurs stratégies, contraints par leur nouvel électorat. \\
Dans un premier temps, ils cherchent à continuer la relation clientéliste, mais il est impossible de rémunérer l'ensemble du corps électoral. \\
Progressivement, le vote va s'individualiser et devenir l'expression d'une opinion personnelle. Ce processus s'initie sous la IIIe République et s'intensifie via la mise en place de politiques volontariste, pour apprendre aux citoyens à voter. Il va s'individualiser car de nouveaux acteurs vont émerger et concurrencer les notables et contester leurs pouvoirs en proposant des visions politiques et en redéfinissant les relations électorales.

\section{La République dans les moeurs}

La République comme régime s'est progressivement affirmé dans deux directions différentes.

\subsection{La consécration dans la classe politique}

Au cours du XIXe siècle, le régime est un débat central. Près d'un siècle après la Révolution française, les monarchistes sont toujours nombreux et mobilisés. \\
Au départ, la République s'impose comme un régime par défaut puis va commencer à devenir dans les esprits, le régime qui divise le moins. On parlait à cette époque de "République d'attente".


En 1870 éclate la guerre avec la Prusse, la défaite se fait deux mois plus tard avec la capture de Napoléon III. À Paris, se constitue un gouvernement de défense nationale qui proclame la République. \\
Des élections législatives sont donc organisés à la va vite. L'enjeu central est bien évidemment la guerre avec la Prusse. Les Royalistes sont favorables à la paix tandis que les Républicains souhaitent poursuivre la guerre. Les partisans de la République obtiendront 200 députés contre 400 pour les royalistes. \\
C'est ainsi que Adolf Thiers devient le chef du pouvoir exécutif dans cette République. De manière assez paradoxale, l'épisode de la commune de Paris va renforcer la République. Certains considèrent que la commune de Paris est une première manifestation du communisme. À l'époque, la problématique des républicains et des monarchistes est la stabilité du régime. En réprimant un mouvement insurrectionnel, la République a fait sa preuve d'un régime stable. En 1872, Thiers se rallie à la République.


C'est le régime qui divise le moins car les monarchistes sont, à l'époque, très divisés. Il y a les Orléanistes, héritiers de la monarchie de Juillet et la monarchie constitutionnelle et parlementaire, modéré et d'autre part, les légitimistes qui revendiquent les héritages des ultras de la première restauration. \\
C'est pour cette raison que Thiers sera renversé en 1873, et Mac Mahon sera élu président. Les divisions persistent chez les monarchistes et rendent impossible une éventuelle restauration. \\
Ce n'est qu'en 1875 que la République est juridiquement renforcée. Le mot République apparaît dans trois lois constitutionnelles. C'est presque par hasard pendant les débats parlementaires qu'un amendement introduit le mot République dans la loi constitutionnelle. \\
Les républicains vont progresser aux élections, leur audience est de plus en plus important à travers le pays et ils finissent par remporter une majorité aux élections de 1876. McMahon dissoudra la chambre et de nouvelles élections seront organisées en 1877, qui donnera, de nouveau, une majorité au camp des Républicains. C'est lors de cette élection que le leader républicain Gambetta, aurait lancé une formule restée célèbre "Quand la France aura fait entendre sa voix il faudra se soumettre ou se démettre". \\
En Janvier  1879, McMahon démissionnera suite à la victoire des Républicains aux élections sénatoriales. 9 ans après la proclamation de la République, c'est la première fois qu'un Républicain est chef de l'État: Jules Grévy. \\
La République est donc désormais acceptée comme seul régime légitime. 

\subsection{La "républicanisation" de la société}

L'affirmation de la République passe également par la mise en oeuvre de politiques publiques volontaristes que les républicains vont mettre en place. On peut parler d'un processus de "républicanisation" de la société française. Processus car l'idée républicaine n'a pu s'imposer que progressivement dans la société française. \\
C'est en grande partie grâce aux politiques volontaristes, grâce aux symboles républicains qui se diffusent (La Marseillaise comme hymne national en 1879 et le 14 Juillet comme fête national). L'éducation civique apparaît avec l'école et permet de développer un attachement à la République aux plus jeunes. \\
Le projet des républicain qui repose sur la participation des citoyens à la vie politique ne prend sens que si le citoyen est en mesure de participer à la vie politique. L'école est donc censé formé des citoyens et consolide ainsi la République. Les républicains sont essentiellement anti-cléricaux et le développement de l'école laïque permet de concurrencer directement l'Église. 

\section{Le processus de politisation}

"On désigne par politisation le processus par lequel un intérêt pour la politique se développe dans la population". \\
Dans la France du XIXe, la population est essentiellement rural. Au début du XXe siècle, plus de 50\% de la population vit en milieu rural et près de 60\% des actifs sont des agriculteurs. Il y a donc un clivage important entre les villes et les campagnes. Ville où l'intérêt des citoyens pour la politique est un peu plus précoce et campagne où le développement de l'intérêt est plus tardif. \\
Dans le milieu rural, le niveau d'éducation est plus faible, les lieux du pouvoir sont plus visible pour les urbains. En ruralité, les entités locales sont très fortes, cependant, il faut manier cette thèse avec précaution, des historiens considèrent que la politique a toujours été très présente dans les villages mais peu tournés vers les arènes nationales. 


Pour Eugen Weber, un historien américain, il montre que la France va avoir tendance à s'uniformiser, les entités locales vont peu à peu s'effacer et le sentiment d'appartenance nationale, va, en contre partie, progresser. Pour lui, la société française va se nationaliser. Cette nationalisation est favorisée par quatre facteurs essentiels: le développement des communications (transport, essentiellement) ; progrès lié à la scolarisation ; le suffrage universel ; la première guerre mondiale. \\
Cette analyse d'Eugen Weber est devenu tout à fait classique mais n'est pas la seule.


Pour Maurice Aghulon, un historien français, propose pour sa part, une autre lecture. Il considère que les républicains vont s'appuyer sur les identités locales. Ce serai donc un processus moins direct. Il insiste sur le poids et l'importance du suffrage universel qui est le principal facteur d'éducation politique des citoyens. Cela signifie que l'intérêt pour la politique découle de l'exercice du droit de vote. Pour lui, les lieux de sociabilité local constitue des lieux de socialisation à la vie politique. Ce serai donc grâce à ces structures que la politique se propage jusque dans les plus petits village de France. 


\section{L'apprentissage du vote}

Le vote est au centre de la vie politique française. Cependant, cela n'a pas toujours été de soi. Le vote a donc subit un processus de naturalisation. "Si l'électeur aujourd'hui fait l'élection, c'est parce que préalablement, l'élection a fait l'électeur" Alain Garrigou. \\
Le vote va devenir peu à peu un comportement codifié, présenté comme l'expression d'une opinion personnelle et non plus comme la marque d'une appartenance à un groupe. La salle de vote va peu à peu perdre son caractère de lieu de délibération et va être érigé en un espace neutre et démocratique. \\
Dans ce lieu, l'électeur doit accomplir dans le calme son devoir civique. L'état d'ivresse est interdit dans un lieu de vote en 1874. Les urnes, les bulletins, l'isoloir, l'interdiction de débattre, d'exercer la moindre pression sur un électeur ne sont pas anecdotiques, et représentent l'idée que les républicains se font du vote. \\
Quelques dates clés de l'apprentissage du vote:
\begin{itemize}
\item 1848: listes permanentes et alphabétiques de l'ensemble des électeurs ;
\item 1852: organisation de la procédure électoral ;
\item 1860: standardisation des urnes ;
\item 1884: carte des électeurs obligatoires ;
\item 1913: isoloir ;
\item 1923: bulletins ;
\item 1988: paraphe sur les listes d'émargement obligatoire.
\end{itemize}
Les républicains tentent de promouvoir une vision individualiste de l'électeur. En votant, les citoyens participent à une communauté nationales. Dis autrement, les citoyens deviennent français en votant et inversement.

\section{La professionnalisation de la politique}

Parallèlement, l'activité politique va devenir un champ autonome. On observe un processus de professionnalisation. Une figure idéal-typique du professionnel de la politique apparaît à la toute fin du XIXe siècle, qui conteste le pouvoir des notables, qui représente des groupes sociaux jusque là ignorés dans la sphère politique. Ces nouveaux professionnels appartiennent souvent à des professions libérales, et à la bourgeoisie de capacité bien souvent Mais aussi des ouvriers, et des représentants du monde ouvrier. \\
Ces nouveaux professionnels ont peu de ressource personnelles donc propose plutôt de la propagande électorale, vont se constituer leurs propres ressources: collectives et organisationnels (partis, mobilisations collectives...). D'un certain point de vue, ils opposent la force du nom par rapport à la force du nom. Ils vont collectivement, développer de nouvelles techniques de campagne. Ces nouveaux entrepreneurs tentent de redéfinir la relation électorale.\\
L'élection va devenir ce qu'elle est, un échange de programmes électoraux contre des suffrages. \\
À noter que le processus va être double, les notables vont se professionnaliser et les professionnels vont se notabilisés. 

\section{La naissance des partis politiques}

À la fin du XIXe naissent les premiers partis politiques au sens moderne du terme, au sens où on l'entend encore aujourd'hui: "Une organisation durable, qui participe à la lutte pour la conquête de postes électifs en mobilisant des soutiens populaires grâce à une organisation ramifiée et territorialement organisée". \\
Pendant longtemps, le mot parti était au sens prendre parti pour tel ou tel idéologie. C'est donc au XIXe que les partis deviennent des acteurs centraux. Ils acquièrent progressivement le monopole de l'activité politique. Certains pensent que les partis politiques sont les enfants du suffrage universel, car celui-ci rend impératif l'organisation de la conquête des suffrages. \\
Jusqu'en 1901, les partis ne sont que tolérés par la réglementation, ce n'est qu'en 1901 et la loi sur les associations qu'ils vont se doter d'un cadre légal. C'est une reconnaissance assez tardive des partis politiques. \\
Les formes que peuvent prendre les partis politiques sont très variables. Elles sont variables historiquement, dans la position sur l'échiquier politique etc.


Les premiers partis politiques sont issus des mouvements ouvriers, et se constituent dans les années 1870: ils s'appuient donc sur le nombre et la mobilisation de leurs adhérents, sur les réseaux de sociabilité ouvrières. \\
Jules Guesde et son parti, ont des premières victoires électorales au début des années 1890. \\
Cependant, les partis modernes n'apparaissent que réellement au début du XXe siècle: le parti radical en 1901. Dans les années suivantes, les principaux courants politiques vont s'organiser en partis politiques. \\
La vie politique tend à se partisanisé: les signes et emblèmes partisans prennent peu à peu du sens pour les électeurs ; les individus commencent à s'identifier aux partis politiques, aux programmes. Les partis exercent des fonctions de plus en plus centrales: désignation des candidats. Ils structurent l'offre politique. Ils encadrent la mobilisation électorale. \\
En bref, ils structurent la vie politique dans son ensemble. 

\chapter{Le clivage gauche/droite}

C'est un clivage qui structure la vie politique française et cela depuis la révolution française, pour au moins deux raisons. D'abord parce que la production savante entretient ce clivage. La deuxième raison est parce que les hommes politiques eux mêmes se positionnent dans le clivage gauche/droite. \\
Ce clivage existe dans de nombreuses démocraties occidentales. Aux USA, c'est par exemple les Républicains contre les Démocrates, et au Royaume-Uni, les Travaillistes et les Conservateurs. \\
Ce clivage a certes structuré la vie politique française, mais n'a jamais cessé d'être contesté. Au cours de son histoire, le clivage gauche/droite n'a pas toujours renvoyé aux mêmes enjeux. C'est une sorte de paradoxe, c'est justement parce que le contenu de ce clivage varie qu'il est toujours efficace. Le clivage ne renvoie pas à des oppositions immuable ni à des oppositions identiques. Comme pour les traditions politiques, il faut veiller à ne pas essentialiser ce clivage: on ne peut pas considérer qu'il existe par essence des problématiques de gauche ou des problématiques de droite. Ce sont des produits de l'histoire, qui sont façonnés par les acteurs et les observateurs de la vie politique. \\
Le clivage se compose et se recompose au fil du temps. À noter que le clivage est toujours multi-dimensionnel. 

\section{La naissance du clivage gauche/droite}

On peut considérer qu'il naît avec la révolution Française. La première étape serait en 1789, où les député du tiers État, décident de se réunir à Versailles contre la volonté du Roi. Ils sont rejoints par une grande partie des membres du clergé et de la noblesse. Spontanément, ces députés vont laisser les chaises les plus honorifiques aux membres du clergé et de la noblesse. \\
Le Vendredi 28 Août 1789, l'Assemblée nationale constituante organise des débats quotidien et lorsque le débat en arrive à la question du veto royal, les députés y étant favorables se placent à la droite de l'hémicycle et ceux qui s'y opposent, se placent à gauche. \\
En réalité, le clivage va se former progressivement et devenir de plus en plus prégnant au cours du XIXe et du XXe siècle. Le premier débat central qui va façonner le clivage gauche/droite, c'est la question du régime politique. 

\section{La question du régime}

Le XIXe siècle est traversé par cette question du régime politique. De la révolution française jusqu'aux années 1880, cette question rythme la vie politique française. Le débat se cristallise sur la question des institutions. La gauche revendique l'héritage de la Révolution Française, elle s'inscrit dans sa filiation et elle défend la République. Les hommes et les femmes de gauche accordent une grande importance à l'héritage de la révolution et à la philosophie des lumières. Ils veulent en finir avec l'absolutisme monarchique et la République serait le régime le plus adapté. \\
La droite, elle, est attachée au Roi et appelle au retour de la monarchie. Elle défend l'ordre moral, voire même l'ordre social. Elle s'oppose à l'individualisme de la philosophie des lumières. Cette droite est divisée, d'une part les partisans de la monarchie absolu, les légitimistes, et d'autre part, les partisans de la monarchie constitutionnelle, les orléanistes. Plus tard, la droite connaît une sorte de renouveau et on pourrait considérer le Bonapartisme comme un nouveau courant de droite, très soucieux de l'ordre moral et social et qui est en même temps favorable à une forte autorité étatique mais qui accepte dans le même temps le principe du suffrage universel sous une forme largement plébiscitaire. \\
Après la restauration, la monarchie absolue n'a plus vraiment d'actualité. Le débat s'organise essentiellement autour des partisans de la monarchie constitutionnelle et des partisans de la République. \\
La troisième République s'impose face à une droite divisée. La République s'impose donc et s'installe. La droite va se rallier à cette République, par défaut. On peut considérer que se termine un épisode politico-historique particulier, ouvert par la révolution française. \\
Il y a deux marqueurs du ralliement de la droite à la République: en Novembre 1890, lorsque les catholiques admettent la République. Confirmé deux ans plus tard, par le Pape, qui invite les catholiques à accepter la Constitution, pour changer de l'intérieur les institutions. Un deuxième ralliement se produit après la première guerre mondiale avec la victoire du bloc national (coalition de droite), qui vient consacrer l'intégration de la droite dans la République. En 1926, le Pape condamne l'Action Française. 

\section{La question de la religion}

En pratique, la question religieuse et la question du régime sont très fortement liées. C'est évident puisque la monarchie est de droit divin et puise sa légitimité dans la religions. Cette question religieuse ne disparaît pas du jour au lendemain. Elle reste très longtemps un point d'affrontement entre la droite et la gauche. Même si ce clivage a aujourd'hui largement disparu, on en trouve régulièrement des apparitions lors des manifestations pour les écoles libres dans les années 80, dans les manifs de la manif pour tous en 2012. \\
Au XIXe et au début du XXe, l'opposition entre les cléricaux et les anti cléricaux, est très forte et très prégnante. Chaque camp est porteur de visions très différentes du monde social. La question centrale est celle de l'influence sociale de l'Église: quelle place doit occuper l'Église dans la société française ? Jusqu'à quel point peut elle avoir une emprise sur les comportement collectifs ou privés ? \\
La gauche, plutôt rationaliste, se revendiquant de la philosophie des lumières, est très critique à l'égard du rôle social du clergé. Elle va chercher à soustraire la société de l'emprise conservatrice de la Religion, elle défend la laïcité: l'idée selon laquelle le champ politique doit être dissocié, indépendant, de toute religion. \\
Face à cela, la droite défend la tradition chrétienne de la France. Elle souhaite préserver les liens privilégiés entre le pouvoir et le clergé. La droite considère que la Religion est le ciment de la société et que ce ciment permet de maintenir l'ordre social et moral. \\
Du XIXe siècle à nos jours, cette question religieuse disparaît progressivement, la laïcité l'a emporté. Le politique et le religieux sont séparés dans plusieurs domaines de la société: c'est le cas dans l'éducation, où l'école est gratuite, obligatoire et laïque (1882). C'est la cas dans d'autres activités de l'État. Enfin, la loi de 1905 proclame la séparation de l'Église et de l'État. \\
La question va cesser d'être un enjeu politique saillant dans l'entre deux guerres.


La question scolaire était une forme de prolongement de cette question religieuse. Il coexistait en France deux écoles, l'école publique mais aussi l'enseignement privé. Face à cette question, la gauche a conservée son caractère laïque et y est globalement opposé. D'une part, la gauche a une prise de position maximaliste qui consiste à combattre l'enseignement privé confessionnel et une position minimaliste qui consisterai à conserver l'école privé mais lui supprimer tout financement public. \\
Quant à la Droite, elle défend la liberté d'enseignement. Elle ne s'oppose plus comme ce fut le cas à l'enseignement laïque mais défend la possibilité pour ceux qui le souhaiterait de donner une éducation confessionnel à leurs enfants. La droite préfère parler d'école libre. \\
À la libération, les subventions aux écoles privés sont supprimés. C'est en 1959 que la loi Debré rétablit les subventions aux écoles privés mais les lies à l'État par l'intermédiaire d'une politique de contrat. Il s'agit en quelques sortes de soumettre des établissements privés à des obligations de service publics en contrepartie de la participation financière de l'État. De plus, l'établissement doit accepté un contrôle rigoureux de son fonctionnement, de ses programmes et de ses enseignants. \\
Cette dimension du clivage gauche/droite se manifeste dans les années 80. Mitterrand cherche à créer un grand service public de l'enseignement unifié et laïque avec son ministre Savary. Les réformes proposées mettent le feux aux poudres, une manifestation a lieu en 1984, la droite soutient cette manifestation. 1 millions et demi de personnes défilent pour défendre l'école libre. Elle débouche sur une crise politique très importante: le Gouvernement retire son projet, le PM est conduit à la démission etc. Cependant, il est à noter que le rapport à l'école n'est plus confessionnalisé, on met notre enfant à l'école privé parce qu'on est mécontent du service public. \\
En 1992, Jack Lang signe un protocole avec le secrétaire général de l'enseignement catholique, mais pourtant, l'année suivante, éclate une nouvelle mobilisation et c'est François Bayrou qui est alors ministre du gouvernement Balladur, tente de remettre en cause une loi de 1850, la loi "Falloux". La Droite tente de permettre aux collectivités locales de financer l'enseignement privé. Cette fois-ci, c'est le camp laïque qui se mobilise et force la droite à battre en retrait. 

\section{La question sociale et économique}

À partir du milieu du XIXe siècle, la question sociale tend à remplacer progressivement la question du régime et la question religieuse. C'est à partir du XIXe siècle que la dimension socio-économique commence à apparaître. L'économie est révolutionné grâce à la révolution industrielle. Il s'agit donc d'un nouveau conflit social. \\
L'industrialisation produit un nouveau groupe social que Marx appelle le prolétariat. Très vite, le mouvement ouvrier cherche à défendre les intérêts de cette classe sociale. C'est une période où émerge les nouveaux entrepreneurs politiques. C'est donc une période où émergent sur la scène politique les classes populaires. On parle à cette époque de masse ouvrière ou de classe laborieuse. Ce sont des préjugés qui s'appuient sur la prétendue dangerosité des foules. \\
On peut se dire que le clivage gauche/droite devient la traduction dans le paysage politique des conflits de classe dans la société. Selon Marx, il y a un clivage possédant/non possédant. Ce clivage se recompose à de nombreuses reprises pour se traduire dans un clivage libéraux contre dirigiste, etc. \\
Le mouvement ouvrier se structure en France dans les années 1880. Les formes d'expression du mouvement ouvrier sont diverses: syndicats, mouvements, etc. Les républicains qui sont alors la Gauche, se diversifient et sont plus ou moins modérés. Certains sont réformistes: souhaitent contrôler, réguler, réglementer l'économie capitaliste ; ils cherchent à améliorer la vie des ouvriers grâce à des lois sociales. \\
D'autres sont révolutionnaires comme les socialistes, à la fin du XIXe. Ils sont contre l'idée d'accompagner le développement de l'économie capitaliste. Ils cherchent à transformer radicalement l'ordre social pour renverser l'exploitation des ouvriers par les patrons. \\
Ce n'est que très progressivement que le rapport de la gauche à l'économie de marché va évolué. On le voit notamment avec le rapport que la Gauche entretient à l'État. \\
Si l'on observe le rapport de la Gauche à l'État, l'on observe des évolutions sensibles. C'est le cas notamment dans les années 30 lorsque pour les socialistes, les nationalisations commencent à devenir possible et voient là une opportunité de réduire les inégalités. Ils considèrent que l'État et son implication dans l'économie peut représenter une certaine vertu. La Gauche commence donc à percevoir un certain intérêt pour l'État. C'est la SFIO qui deviendra plus tard le PS, qui incarne cette gauche étatique. La SFIC y voit une gauche d'accompagnement du libéralisme. 


La Droite, elle, est attachée à la propriété. Elle est attachée aux mécanismes du marché. Elle est favorable aux jeux du marchés. Elle est donc libéral sur le plan économique. Elle défend l'idée que les marchés tendent à s'auto réguler pour aller vers une situation optimale. \\
C'est sur cette question de l'interventionnisme que se structure le clivage gauche/droite. La Droite est plutôt défavorable à l'intervention de l'État. C'est dans les années 1970 que le clivage gauche/droite est le plus fortement structure par la question socio-économique. C'est en 1972 que le PCF et le PS signe le programme commun dans lequel il est fait mention de la rupture avec le capitalisme. La victoire de la Gauche en 1981 amène à la mise en oeuvre d'une politique de relance économique: augmentation des salaires, du SMIG, des retraites, des allocations familiales, vague d'embauche dans la fonction publique, généralisation de la retraite à 60 ans, réduction du temps de travail, mais aussi de nombreuses nationalisations. Pourtant, le tournant de la rigueur, en 1982, marque une rupture avec ces orientations. Cela marque le début d'un brouillage sur le clivage sur les questions sociales et économiques. \\
Lors de la première cohabitation, la droite prend le pouvoir et accentue encore cette politique néo-libérale. Lorsque Mitterrand revient au pouvoir, le thème des nationalisations est totalement abandonnés. Les politiques de droite et de gauche tendent à s'indifférenciée dans le clivage. Depuis 1997, il n'y a plus vraiment eu de différences sensibles entre le gauche et la droite sur le plan socio-économique.


\section{Les sujets passés de gauche à droite ou inversement} 

La nation est marqueur de la droite ou de l'extrême droite. Pourtant, la Nation est un thème de gauche qui apparaît dès la Révolution française. Le thème du nationalisme passe à droite à la fin du XIXe siècle, et c'est un nationalisme plutôt conservateur, identitaire. Cela s'explique par le fait que le mouvement ouvrier commençait à se structurer dans la deuxième partie du XIXe siècle dans une perspective internationaliste. Cela laisse libre le thème de la nation et ce vide conduit à la création de l'extrême droite. \\
Plus tard, au XXe siècle, les lignes autour de cette question sont toujours aussi floues car le PCF a été un parti qui a prôné une certaine forme de nationalisme et de patriotisme en attisant notamment l'anti-germanisme et en exprimant une forme de rejet des USA ainsi que de l'organisation européenne. \\
Une certaine forme de nationalisme, assez jacobine, persiste à gauche comme c'est le cas chez Jean-Pierre Chevènement. Le Gaullisme a aussi une forme de nationalisme en ayant une certaine idée de la France. 


La décentralisation est aussi un sujet qui a changé de clivage. À la révolution, la gauche jacobine est centralisatrice et défend l'unité de la République et son indivisibilité alors que la droite monarchiste est plutôt attachée aux identités locales, aux communautés naturelles. Cette droite monarchiste refuse les règles universelles des jacobins. \\
Au XXe siècle, la question de la décentralisation divise la gauche. Il y a une gauche centralisatrice, étatiste et une autre qui défend la décentralisation. Cette question divise aussi la droite entre les libéraux et les gaullistes, qui sont attachés, respectivement, à la décentralisation et au pouvoir de l'État central. \\
Ceci étant, c'est la Gauche qui met en oeuvre les grandes lois de la décentralisation, sous le premier septennat de François Mitterrand. On peut dire que la question de la décentralisation n'est plus un enjeu constitutif d'un clivage, il y a un certains consensus sur la question. 


Le libéralisme culturel est un enjeu beaucoup plus récent, et désigne un ensemble d'attitude et d'opinions qui sont liées aux moeurs, à la vie privée, à la sexualité ou encore au pluralisme culturel. On peut considérer que ce clivage apparaît lorsque que VGE cherche à asseoir son image de président jeune. Il donne le droit de vote à 18 ans, il légalise la contraception, l'avortement. Même si c'est un président de droite, ce sont toujours les électeurs de gauche qui sont toujours aujourd'hui attaché à ces acquis ainsi qu'à la suppression de la peine de mort, et qui sont aussi les moins opposés à l'immigration. \\
Aujourd'hui, le clivage se fait sur un pôle humaniste et universaliste plutôt marqué à gauche et un autre pôle nationaliste et traditionaliste, marqué à droite. 


On observe la disparition progressive de la plupart des enjeux à partir desquels le clivage gauche/droite s'est structuré. Le clivage a perdu aujourd'hui de sa force, mais pas nécessairement sa pertinence. Le clivage gauche/droite se brouille à partir des années 80 sous l'effet de la politique de Mitterrand mais aussi parce que le jeu de l'alternance politique devient la règle: très peu de majorité législative ont été reconduites depuis 1981. \\
La plupart des référents idéologiques de la gauche apparaissent comme en crise alors qu'en face, la question du libéralisme économique est tellement présente qu'elle n'apparaît même plus comme une idéologie en tant que tel. \\
Aujourd'hui, le clivage gauche/droite n'a pas disparu, ne serait-ce que pour des questions sociologiques. Aujourd'hui, le clivage gauche droite a une réalité sociologique qu'il est difficile de nier. 

\chapter{Le radicalisme politique}

\section{La naissance du parti}

Aujourd'hui, les radicaux sont une force politique marginale. Cela n'a pas toujours été le cas. Lors de la IIIe et la IVe République, les radicaux constituaient un acteur majeur de la vie politique, qui a donc connu finalement, un déclin. \\
Les radicaux défendent des positions républicaines et au fur et à mesure que s'affirme la République, le radicalisme, lui, décline et perd de son influence à mesure que la République en gagne. Le poids politique des radicaux reste important longtemps après l'avènement de la République, notamment grâce à leur maîtrise du jeu institutionnel et leur position centrale dans le spectre politique.


Les idées radicales ont longtemps défendu les idées de progrès social et d'égalité des chances. Ils sont très attachés à l'instruction publique, qui doit permettre le progrès social et l'égalité des chances. Ils défendent la laïcité, qui est l'un de leur marqueur politique. On peut enfin définir le radicalisme comme une doctrine politique qui prône une rupture institutionnelle la plus complète. \\
Le radicalisme est avant tout une sensibilité politique avant d'être un parti, qui sera malgré tout le plus grand parti, fondé en 1901.


La radicalisme apparaît bien avant la naissance du parti radical et désigne, au XIXe, un courant de pensée qui défend les institutions républicaines et l'héritage de la révolution française. Le mot apparaît sous la monarchie de Juillet et sert d'euphémisme pour désigner les républicains. Le radicalisme s'inscrit très nettement à gauche de l'échiquier politique. Ce courant de pensée va se structure progressivement au début de la IIIe République et Gambetta est l'une des figures du radicalisme et en décline les principaux thèmes dans un discours tenu à Belleville en 1869. Parmi ces thèmes, il y a l'attachement à la République, la défense du suffrage universel, la liberté de la presse et d'association et enfin l'anti-cléricalisme et la défense de la laïcité. 


Les radicaux, à la différence des socialistes, mettent les questions sur les institutions. Ils considèrent que la démocratisation du régime va permettre l'émancipation sociale des citoyens. \\
Plus la République va gagner en audience et stabilité, plus on va observer une redéfinition de ce qu'est le radicalisme. À partir de 1879, certains Républicains prônent une politique de prudence et de réalisme, c'est le cas notamment de Gambetta. Cette prudence ne plaît pas à tout le monde, notamment Georges Clémenceau, qui va rompre avec les radicaux et va incarner l'intransigeance initiale des radicaux et défend l'idée d'une laïcité intégrale et défend aussi comme l'instauration d'un impôt sur le revenu progressif. \\
Les radicaux développent une philosophie centrée sur l'égalité des droits et sur une solidarité redistributive.


L'affaire Dreyfus, à la fin du XIXe joue un rôle très important dans la structuration des radicaux. Ils sont le centre de gravité de l'ensemble des défenseurs de la cause de Dreyfus. Ils participent au pouvoir en 1899 avec le bloc des gauches qui rassembles les radicaux, les modérés et certains socialistes. C'est en Juin 1901 que se tient le congrès fondateur du parti radical, il compte alors 280 parlementaires, 476 comités locaux et il va devenir dès lors, un parti charnière pour toutes les coalitions gouvernementales qui dirigent la République. \\
L'importance croissante des mouvements sociaux amènent les radicaux à intégrer des questions sociales et défendent donc l'impôt progressif sur le revenu mais ils avancent aussi d'autres idées comme l'assurance sociale, l'idée de retraites pour les ouvriers, etc. \\
Malgré tout, les radicaux restent très attachés à la propriété individuelle qui est selon eux, gage d'indépendance et de protection dans un système capitaliste. Les radicaux vont progresser sur le plan électoral et deviennent un parti de gouvernement. Ils font voter des lois importantes, la loi sur les congrégations religieuse qui sont interdites, ils mettent en application la séparation de l'Église et de l'État. Clémenceau devient président du conseil en 1906, le climat social est très agité, il est taxé de briseur de grèves et s'oppose à la SFIO de Jaurès sur plusieurs points, notamment au pacifisme, et les mesures sociales que veulent les socialistes. Clémenceau réfute la lutte de classe mais les radicaux son divisés entre deux franges, une plutôt de gauche, cherchant l'union avec les socialistes et une de droite qui cherche l'alliance avec les modérés. 


Les radicaux, pendant l'entre deux guerres, continuent d'occuper une position centrale. Le parti radical devient pratiquement indispensable à la formation de toute coalition gouvernementale et participe à des gouvernement tantôt de gauche, tantôt de droite. \\
En 1919, il s'allie en partie à la droite pour former le bloc national. En 1924, l'alliance se renverse et les radicaux s'associent à la gauche pour former le cartel des gauches. Cette majorité se désagrège petit à petit et se reforme en 1932 sous le nom de néo-cartel des gauches. Il faut noter qu'à cette période (1932), les socialistes commencent à dépasser en nombre de voix les radicaux. \\
Il y a au sein du parti radical, un mouvement qui se fait appeler "les jeunes turcs", qui défend les mêmes prises de positions que le parti du même nom qui a pris le pouvoir en Turquie. Ils défendent un ancrage plus à gauche du parti radical et ce courant va prendre l'ascendant dans le parti à partir de 1935. C'est ainsi que le parti radical va prendre part à une nouvelle coalition, le Front Populaire. C'est Edouard Daladier qui dirige le parti à cette époque qui signe l'alliance avec Leon Blum mais le Front Populaire finira en 1937. \\
Le score du parti radical avoisine les 20\% des voix en moyenne lors des élections législatives, ce qui fait d'eux une force politique majeur. Ils participent à pratiquement l'ensemble des Gouvernements de l'entre deux guerres et présideront 13 des 42 Gouvernements d'entre deux guerres. \\
Les radicaux maîtrisent parfaitement les rouages institutionnels, ils sont au centre de l'échiquier, et la faiblesse de la discipline partisane font que ce parti était le parti central ayant participé à presque tous les Gouvernements.

\section{Un parti politique peu structuré}

Le parti radical est un mouvement de mobilisation électoral qui dépend beaucoup de ses structures locales. Dans le même temps, les socialistes et les communistes développent leurs organisations avec un pouvoir central beaucoup plus important. La discipline de vote au parlement est très faible et ne sont pas respecté. L'autorité du parti se heurte à l'autonomie de ses parlementaires. \\
Le nombre de militants est assez limité, il y a environ 100 000 cotisants dans les années 30. En réalité, l'implantation du parti dépend beaucoup de réseaux auquel il est affilié, il s'appuie notamment sur les loges maçonniques, la ligue des droits de l'Homme, et aussi sur de nombreux journaux qui sont liés de manière plus ou moins direct au parti radical. \\
Sur le plan sociologique, le parti radical est implémenté dans les classes intermédiaires: fonctionnaires, artisans et commerçants mais aussi chez les professions libérales, ainsi que chez les journalistes. 


\section{Le déclin du radicalisme}

Le parti radical enregistre à la Libération un net recul de ses résultats électoraux: 11\% des voix en 1946. Plus explications existent pour ce déclin. \\
La troisième République apparaît comme un régime daté à la fin de la guerre et cette réputation rejaillit sur les radicaux alors que la libération est synonyme de vif renouveau politique. Certains radicaux ont aussi votés les pleins pouvoirs à Pétain, ce qui a contribué à précipité leur déclin. Le parti conserve néanmoins quelques bastions avec quelques fortes personnalités. Ils vont contribuer au maintien du parti radical: Pierre Mendès France et Edgar Faure, sous la IVème République. Sur 22 Présidents du conseil sous la IVe République, 9 seront radicaux. \\
Il y a toujours deux tendances au parti radical, une tendance conservatrice représenté par Edgar Faure, et une aile gauche qu'incarne Pierre Mendès France. Ce dernier va tenter de ré-encrer le parti radical à gauche dans les années 50. Il est nommé président du conseil en 1954 et prend la tête du parti radical en 1956. Ne parvenant pas à rénover le parti radical, il le quitte en 1958 et rejoint le PS. \\
Le déclin du parti radical s'accroît encore aux législatives de 1958 et ne font que 7,3\% des voix et à la fin des années 60, il ne compte plus que 20 000 adhérents. Une nouvelle tentative de rénovation du parti est fait par Servan-Shreiber, qui échoue également en 1971. Dans le climat particulier des années 70, où la bipolarisation devient importante, le parti éclate et se divise avec d'une part le MRG qui va aller signer le programme commun et d'autre part le parti radical valoisiens qui rejoindra l'UDF. Les radicaux ne survivent que comme force d'appoint, à gauche ou à droite. Ils vivent grâce à leurs élus locaux. Bernard Tapie contribue à redonner vie au MRG et réalise un score très important de 12\% aux européennes mais cette progression n'est pas suivie. 


\chapter{Le socialisme}

L'histoire du socialisme est intéressante car au départ, les socialistes occupent une fonction, un rôle, de parti d'opposition qui n'exerce pas le pouvoir, qui le rejette même. Aujourd'hui, le PS est devenu un parti de Gouvernement qui occupe une place centrale dans le système politique. Le socialisme français est marqué par le Marxisme qui le caractérise à ses origines et c'est notamment ce qui va expliquer les fortes spécificités du socialisme français. \\
En effet, le PS français, contrairement à ses homologues européens, n'abandonne que très tardivement son intransigeance aux valeurs originelles. Le PS va entretenir un rapport très ambigu au pouvoir, hésitant entre l'exercice ou non du pouvoir. 

\section{La constitution du socialisme français}

\subsection{L'hétérogénéité du socialisme et l'unification de 1905}

Les socialistes se revendiquent les représentants de la classe ouvrière, le prolétariat. Cette référence à la classe ouvrière va rester un marqueur idéologique important pour les socialistes. À mesure que progresse les revendications ouvrières, les socialistes eux aussi, progressent, ils sont divisés en plusieurs tendances dans les années 1880. Plusieurs groupes se disputent le label socialiste. Certains décident de participer aux élections. 1893, les socialistes comptent 37 députés. \\
D'autres, refusent, c'est le cas notamment des Guesdistes de Jules Guesde qui a fondé le parti ouvrier en 1880. Il incarne la tendance marxiste et la plus intransigeante du socialisme. Ils refusent catégoriquement de participer aux coalitions gouvernementales, y compris les gouvernements radicaux-socialistes. Ils sont cependant favorable à une prise de pouvoir local. Ils ne sont pas pour une révolution immédiate, ils sont pour préparer la révolution méthodiquement en créant une lutte des classes. Les conditions de réussite serait la création d'une structure partisane, forte, organisée. Ils sont installés massivement dans le Nord. Lors de la création de la SFIO, ils sont majoritaires jusqu'en 1936.


Autre courant, les Possibilistes, qui sont réformistes, favorables à la prise de pouvoir de l'État, ils sont prêt à accepter des alliances, une collaboration avec les partis "bourgeois", de manière à obtenir des réformes rapidement qui serviraient la classe ouvrière. \\
Troisième courant, les Allemanistes de Jean Allemane, ils sont proches des anarchistes, ils défendent comme mode d'action privilégié la grève générale. Ils ont été pendant un temps, proche des possibilistes. \\
Quatrième courant: les blanquistes qui considèrent que la Révolution doit naître d'une étincelle qui mettra le feu aux poudres, et la révolution doit être organisée par des individus quasi professionnels. Idéologiquement, on peut situer les blanquistes entre les guesdistes et les socialistes républicains qui participent aux élections. \\
Cinquième courant: les socialistes indépendants, qui ne souhaitent pas adhérer à un parti qui existe, ce sont souvent des intellectuels qui se revendiquent socialiste mais sans appartenir à une organisation politique. Le plus célèbre d'entre eux est Jean Jaurès. 


Tous ces courant socialistes sont différents les uns des autres, il existe entre eux des désaccords très important, notamment sur les questions de stratégie de conquête du pouvoir. Leurs objectifs sont sensiblement les même, ce qui les réunit derrière l'identité socialiste. Malgré toutes ces divergences, un processus d'union s'enclenche au début du XXe siècle, qui conduit à la formation de la SFIO. C'est un parti qui se définit comme un parti de classe, celui de la classe ouvrière. \\
Les différentes tendances y perdurent et Jean Jaurès devient une figure socialiste dominante et ce jusqu'à sa mort en 1914. 

\subsection{La spécificité du socialisme français}

Des spécificités commencent à apparaître et vont le marquer durablement. On en compte trois, mise en avant par Rémi Lefebvre. \\
La première spécificité est la faiblesse de son enracinement social. Il y a peu de connexions entre la classe ouvrière et les socialistes. Il y a très peu de liens entre les socialistes et le monde du travail, comme avec les syndicats. Et, contrairement au parti communiste, la SFIO ne parvient jamais à se constituer en contre société, en parti "milieu de vie". \\
La deuxième spécificité est la faiblesse organisationnelle du SFIO. L'appareil partisan socialiste se manifeste par une bureaucratie très faible et limité. Il présente une forte décentralisation: les élus locaux ont un poids très important au sein du parti, mais son très difficilement contrôlé par l'état major national. Enfin, il n'existe presque pas de formes développés du militantisme. Il ne parviendra jamais à mobiliser massivement la classe ouvrière ou au delà. Cela constitue une particularité par rapport aux autres partis européens qui eux, développent la structure sociale-démocrate. \\
La troisième spécificité est le rapport ambigu du parti avec le pouvoir. La SFIO a dès ses débuts, entretenus un rapport extérieur au pouvoir. Ce n'est donc que très tardivement que les socialistes français acceptent leur intégration dans le système politique, acceptent les règles du jeu du système politique. Le socialisme a toujours manifesté une grande méfiance envers la démocratie représentative. Ce rapport d'extériorité au pouvoir va durer pendant un certain temps avec une prise ponctuelle du pouvoir comme en 1936. Ils abandonneront la référence révolutionnaire très tard. 


Une approche beaucoup plus localisée montre que le socialisme est beaucoup plus diversifié qu'on ne le pense. C'est le travail de Frédéric Sawicki. Il montre que les fédérations socialistes sont très différentes les une des autres. Lefebvre, lui aussi, revisite ces trois grandes spécificités. \\
Néanmoins, la SFIO, dès ses débuts, constituent avant tout une force électorale plutôt qu'une forme de mobilisation. En 1914, il compte 80 000 militants, ce qui est assez peu. On compte une centaine de députés, ce qui est plutôt important. Ce sont majoritairement des députés issus de professions libérales ou intellectuelle. Les élus jouent un rôle très important dans le parti. 

\section{La SFIO dans l'entre-deux-guerres}

\subsection{La progression socialiste}

Après la première guerre mondiale, en 1920, le parti connaît une scission très importante, celle-ci va voir la création d'un nouveau parti: la SFIC au congrès de Tour. Lors du congrès de Tour, la SFIO se divise, la SFIC apparaît sous l'impulsion de la IIIe internationale. Une grande majorité des socialistes acceptent les propositions de Lénine alors qu'une minorité décide de garder la SFIO. Parmi eux, Léon Blum. \\
Même si la SFIO est touchée par la scission, la SFIC peine à décoller alors que la SFIO commence à progresser. \\
Aux législatives de 1924, la SFIO enregistre 1,5 millions de voix et une centaine de députés. Aux législatives de 1928, ils ont 1,7 millions de voix et une centaine de députés élus. En 1934, ils ont 1,9 millions de voix et 130 députés élus. \\
En 1933, la SFIO compte 133 000 adhérents quand la SFIC en compte 30 000. Néanmoins, des tensions apparaissent et continuent de s'accroître. La référence au marxisme est toujours très présente et dans le même temps, l'électorat l'est de plus en plus aussi. Le parti est toujours aussi intransigeant vis à vis de la politique du pouvoir et est en même temps de plus en plus réformiste. L'ouvriérisme affiché de la SFIO fait contraste avec le fait que la SFIO est de moins en moins ancré dans la classe ouvrière. Ils développent essentiellement leur électorat dans la classe intermédiaire ou la petite bourgeoisie. Il y a bien sûr quelques bastions ouvriers: le Nord notamment.


Dans les années 30, deux figures dominent la SFIO: Léon Blum et Paul Faure. Faure est le secrétaire général du parti, il est représentant d'une ligne intransigeante. Blum, quant à lui, est un intellectuel, il était l'un des leaders pour défendre la vieille SFIO lors du congrès de Tour. Il incarne un socialisme qui s'inscrit très bien dans la tradition républicaine et il n'a pas de responsabilité dans le parti mais a des responsabilités au sein du groupe parlementaire socialiste et dirige le journal de la SFIO: le Populaire. \\
La question de la participation au pouvoir est central dans l'entre deux guerres. Blum va essayer de clarifier cette tension idéologique. Pour lui, les socialistes peuvent exercer le pouvoir si ils sont majoritaires au Parlement mais leur objectif restent de conquérir le pouvoir par la voie révolutionnaire. Blum distingue donc la conquête et l'exercice du pouvoir. L'exercice implique une adhésion au système représentatif qui ne peut qu'être provisoire. \\
Les communistes, bien qu'ils soient minoritaires, occupent l'aile gauche de la SFIO est les socialistes se trouve repoussé vers le centre de l'espace politique. Si ils ne peuvent pas s'allier avec les communistes à leur gauche, il y a les radicaux à leur droite. Plusieurs fois, la SFIO sera allié avec les radicaux comme en 1924 où la SFIO soutient mais ne participe pas. Elle soutient de loin des actions comme les nationalisations, soutiennent la semaine de 40h mais sans participer au gouvernement. \\
À partir de 1934, la donne va changé puisque la SFIC va sortir de son isolement pour adopter une stratégie de front unique qui cherche à rassembler la classe ouvrière et la classe moyenne dans la lutte contre le fascisme. 


\subsection{Les socialistes entre conquête et exercice du pouvoir}

Le front populaire, comportant la SFIO, la SFIC et les radicaux de gauche, va remporter en 1936 les élections. Pour la première fois, un gouvernement majoritairement socialiste est créé. Il est dirigé par Léon Blum. Pour la première fois, les socialistes dépassent les radicaux aux élections. \\
Le gouvernement de Léon Blum met en place une nouvelle politique économique audacieuse, avec de grandes réformes sociales. Ce sont les congés payés, la semaine de 40 heures etc. Tout ça dans un contexte social très mouvementé avec des grèves et des occupations d'usine très importantes. \\
Les communistes soutiennent sans participer. Le PCF profite également de ce contexte, il décolle électoralement et va compter 300 000 adhérents en 1936. \\
Le front populaire est une référence mythique au vu du nombre de conquêtes sociales qui font date. Néanmoins, la SFIO va entamer sous la IVe et la Ve République un long déclin.

\section{Déclin et refondation du socialisme sous les IVe et Ve République}

\subsection{Un déclin paradoxal jusqu'à la fin des années 60}

La SFIO va participer à de nombreux gouvernements sous la IVe République. Malgré toutes ces participations importantes, le déclin de la SFIO s'accentue. En 1946, la SFIO obtient 18\% des voix aux élections législatives. En 1962, seulement 12\% des voix. \\
La figure socialiste qui domine lors de cette période d'après guerre est Guy Mollet jusqu'en 1969. Il a assumé à la libération une ligne intransigeante contre la volonté réformiste de Léon Blum. Il a cherché à incarner la pureté originelle du socialisme. En pratique, sa direction prise s'est avérée finalement très électoraliste. \\
Guy Mollet, en tant que président du conseil en 1946, va mener une politique très répressive en Algérie, ce qui choque une large partie de l'opinion publique de gauche. Enfin, Mollet va finir par se rallier à la solution de la Ve République de De Gaulles, ce qui va achever totalement de diviser la SFIO. Les effectifs de la SFIO vont décroître rapidement: moins de 100 000 adhérents en 1960 et dès septembre 1958, la SFIO redevient un parti d'opposition, le mouvement continue d'éclater et de se disperser. C'est une période où la SFIO est très clairement discrédité, elle apparaît comme un parti incapable de s'adapter aux nouvelles règles du jeu de la Ve République. \\
Un rapprochement est amorcé entre la SFIO et le PCF dans les années 60. C'est un rapprochement qui a abouti à une candidature commune aux élections de 1965. 


En 1969, les élections sont un fiasco pour la gauche après la crise politique qu'a présentée Mai 68, la gauche était divisée et n'a pas trouvée de terrain d'entente pour une candidature commune. Le candidat socialiste obtient à peine 5\%. \\
Il s'ouvre alors un cycle nouveau pour la SFIO, une rénovation profonde s'impose pour fonder un nouveau PS et partir à la conquête du pouvoir. 

\subsection{Le nouveau PS en conquête du pouvoir}

Les années 70 constituent une période de renouveau très profond pour le socialisme français. En 1969, la SFIO change de nom et s'appellera désormais Parti Socialiste. Guy Mollet est remplacé. Il faut attendre le congrès d'Epinay en 1971 pour que le PS connaisse un véritable renouveau avec le retour de nombreux groupes extérieurs au parti qui le rejoigne. \\
Le congrès d'Epinay est perçu aujourd'hui comme le renouveau du PS, renouveau qui va le conduire jusqu'à la victoire de 1981. C'est d'ailleurs lors du congrès d'Epinay que François Mitterrand va prendre la tête du PS. Il entre au PS à ce moment, jusque là, il dirigeait un groupe extérieur. C'est avec le soutien de l'aile gauche qu'il est désigné premier secrétaire. Il est soutenu par Jean Pierre Chevènement et également par deux grandes fédérations socialistes: celle du Nord et celle des Bouches du Rhône. \\
Mitterrand n'est pas socialiste de tradition, il a été ministre centre droit sous la IVe République et va pourtant doter le PS d'une nouvelle stratégie de conquête du pouvoir, celle de l'union à gauche. Il faut, pour lui, réaliser l'union de la gauche pour gagner l'élection présidentielle. Du même coup, il redéfinit l'emplacement idéologique du PS en le marquant à gauche. \\
Mitterrand va ouvrir le PS sur la société civile. Il va cherché à ramené au PS de nouvelles classes sociales: les classes moyennes salariées notamment. Il va réussir à augmenter sensiblement les effectifs de militants du PS. Il entreprend une rénovation profonde de l'appareil. Plus largement, les années 70 vont être marqués par le renversement des alliances à gauche avec notamment l'adoption du Programme commun, signé en juin 1972 entre le PS, le PCF et le MRG. C'est la première fois que le PCF passe des accords électoraux avec le PS pour l'élection présidentielle. C'est un programme économique très ambitieux qui est adopté.


Une large fraction du PSU rejoint avec Michel Rocard le PS en 1974. Ce sont de nouvelles orientations programmatiques qui s'affirment. Le PS va tenter une synthèse entre une approche dirigiste et marxiste de l'économie et une approche des revendications des classes moyennes. Cela correspond à l'émergence dans la société post-68 à des revendications non matérielles: droit des femmes, écologie, démocratie locale etc. \\
1974 est aussi une année d'élection présidentielle avec un candidat unique pour la gauche. C'est François Mitterrand qui est le candidat. Il échoue de peu face à VGE qui est élu avec moins de 51\% des voix. Le PS tire un véritable avantage de cette situation d'alliance qu'est le Programme commun pour deux raisons: l'alliance donne au PS l'ancrage à gauche qu'il avait perdu dans les décennies précédentes ; dans le cadre d'un système électoral majoritaire à deux tours, l'alliance favorise l'allié le plus modéré. \\
Le PC va vite prendre conscience de l'avantage que tire le PS du programme commun et va demander en 1977 à ré-actualisé le programme commun et va cherché à le marqué encore plus à gauche. Les discussions n'aboutissent pas et le Programme Commun est rompu dès 1977. La Gauche échoue de nouveau aux législatives de 1978, néanmoins, et pour la première fois, le PS dépasse le PCF lors de ces élections.


Il y a plusieurs tendances qui se créent au PS et qui conditionne la vie du parti. Il contrôle des fédérations, affilie des nouveaux adhérents et François Mitterrand dirige le PS en arbitrant cette concurrence en s'appuyant tantôt sur l'un ou l'autre des courants. \\
Parmi les courants, on peut compter celui de Pierre Mauroy, très proche des élus locaux. Il y a le CERES, pour Centre d'Etudes et de Recherche sur l'Economie Socialiste qui est le courant de Jean-Pierre Chevènement (jacobin et Marxiste). Et depuis 1974, il y a le courant de Michel Rocard qui a contribué à la création de la "deuxième gauche" qui tente de désétatisé le parti et qui développe des réflexions nouvelles comme l'autogestion, etc. \\
Les courants font la force du PS et font de lui un parti capable d'incarné à la fois les revendications traditionnelles ouvrières et les aspirations des nouvelles classes moyennes salariés.


Il se développe dans le même temps un réseau de jeunes cadres, très diplômé dont beaucoup sortent de l'ENA. C'est le cas de Fabius ou de Jospin. Ils arriveront des les cabinets ministériels lorsque Mitterrand accédera au pouvoir. \\
Le PS passera de 60 000 militants en 1970 à plus de 160 000 en 1981. 


La présidentielle de 1981 va venir couronner cette longue période de reconstruction du PS. Marchais était candidat au premier tour pour le PCF en 1981 et obtient 15\% des voix, soit bien moins que le PS. Tous les leviers du pouvoir sont aux mains du PS, plus que jamais, et c'est un nouveau cycle politique qui s'ouvre pour le PS: celui de la pratique du pouvoir. 

\section{Le PS, un parti de gouvernement}

\subsection{Les mutations idéologiques}

La pratique du pouvoir va conduire les socialités à une actualisation idéologique déchirante, douloureuse. L'épreuve du pouvoir ébranle l'identité doctrinale du Parti. Très rapidement après la prise de pouvoir de Mitterrand, la politique de rupture avec le capitalisme est abandonnée. C'est essentiellement sous la pression de Rocard et Delors qu'est entamée une politique de modernisation économique en 1984 par Laurent Fabius qui devient Premier Ministre. \\
Mitterrand est malgré tout réélu en 1988 avec 54\% des voix. Le PS entre dans une logique purement institutionnelle. Le parti perd tout rôle critique envers les gouvernements socialistes (Gouvernement Rocard, Cresson, Bérègovoy). Ce dernier est celui qui incarne le virage du PS sur le plan économique: politique libérale, lutte contre les déficits publiques. \\
Dans le même temps, les socialistes commencent à se déchirer pour l'après Mitterrand. Au congrès de Rennes, en 1990, est mis en évidence une lutte entre les courants qui est fratricide entre Jospin, Rocard, et Fabius et qui témoigne de l'errance idéologique du PS de cette époque. En 1991, lors du congrès de l'Arche, le PS abandonne sa doctrine originelle et amorce une modernisation et abandonne la référence au marxisme. Désormais, pour le PS, le capitalisme demeure un horizon indépassable, il reste cependant critique. \\
Le parti s'engage dans un cycle de défaite très importante comme en 1993, où il enregistre une défaite historique qui va marqué durablement. Aux européennes de 1994, le PS atteint le niveau le plus bas de ses résultats électoraux en ne dépassant pas 13\% des suffrages. C'est une période au cours de laquelle sont mis en place des dispositifs censé donnés la parole aux adhérents (comme le pouvoir donné aux adhérents de désigner le candidat à l'élection présidentielle). Jospin va être l'artisan d'une recomposition du PS. Il est désigné lors d'une primaire interne candidat à l'élection présidentielle de 1995 face à Emmanuelli. Il échoue avec 47,5\% des voix au second tour.


Jospin prendra la tête du PS. C'est à la faveur de la dissolution de l'AN prononcée par Chirac en 1997, que Jospin sera nommé Premier Ministre. \\
Il a travaillé à la mise en place d'une nouvelle union de la gauche qui comptait le PS, le PC, et les verts (c'est la "gauche plurielle"). Au gouvernement, Jospin adopte le "réalisme de gauche". Il revendique un droit d'inventaire sur le bilan de Mitterrand. Il tente de fixer une nouvelle doctrine pour le PS: "oui à l'économie de marché mais non à la société de marché", selon une formule de Jospin. \\
Le chômage baisse de manière historique sous Jospin. Il tente de mener une politique de gauche sur le plan économique (les 35h, la couverture maladie universelle) mais c'est également une politique qui tente de mener la vie sociale avec des lois sur la parité, l'instauration du PACS. Jospin ne passera pas le cap du premier tour à la présidentielle de 2002, c'est donc un nouveau séisme pour le PS. 

\subsection{Sociologie des adhérents du PS}

La pratique du pouvoir n'a pas fondamentalement changé l'organisation du Parti. Il reste un parti d'élus, un parti qui a peu de prises sur les catégories populaires, et un parti où sont surreprésentés les enseignants, qui sont 25 à 35\% des délégués aux congrès du PS. Ils ont peu de poids sur les classes populaires mais ont beaucoup de candidats. \\
Ils ont globalement peu de militants, 213 000 en 1982, 93 000 en 1995, 118 000 en 1999, 120 000 en 2004, 230 000 en 2008. \\
Le PS est extrêmement dépendant du financement public des partis politiques. Il était financé à 53\% par le financement public en 1993 contre 13\% par les adhérents et le reste par les cotisations d'élus. 


Si l'on s'intéresse à la défaite de Jospin en 2002, l'une des interprétation de cette défaite est certainement le décrochage du PS des classes populaires. Sa défaite semble en effet sanctionner une rupture très importante avec les groupes sociaux les plus fragilisés sur le plan économique. Seul 12\% des ouvriers auraient voter pour Jospin au premier tour contre 14\% pour Chirac et 26\% pour JMLP. Les cadres constituent au contraire une base électorale pour le PS. 24\% des cadres supérieurs ont votés pour Jospin en 2002 contre 13\% pour Chirac et 8\% pour JMLP. \\
Jospin, en 1993 et 2002 a maintenu son score chez les cadres mais l'a divisé chez les ouvriers. \\
Le poids des catégories populaires apparaît extrêmement faible dans l'électorat socialiste alors qu'il prétend toujours défendre ses intérêts et se battre pour l'égalité sociale. \\
Le phénomène n'est pas nouveau en 2002 mais particulièrement marqué. Le PC a quant à lui, pendant très longtemps, cherché à promouvoir des adhérents d'origine populaire/ouvrière, ce qui n'est pas le cas du PS.


Le CEVIPOF mène, depuis 1995, des enquêtes sur les adhérents des partis. Les résultats font apparaître un net embourgeoisement des militants depuis 1980. Le nombre des employés et des cadres supérieurs est stable de 1985 à 1998. On observe une diminution de la part des instituteurs chez les socialistes. Ceux-ci sont remplacés par des professeurs du secondaire. Le recrutement en milieu ouvrier est très faible (10\% en 1985, 5\% en 1998). Le sentiment d'appartenance de classe est de moins en moins répandu chez les adhérents. \\
Ces enquêtes montrent un renforcement des couches moyennes salariés. Les salariés précaire qui sont de plus en plus présent dans la société française sont sous représenté au PS (3\% de chômeurs, 4\% de précaires). Les statuts du PS stipulent que tout adhérent du PS doit être adhérent d'un syndicat, 40\% des adhérents en 1998 ne sont affiliés à aucun syndicat. \\
Les adhérents du PS sont de plus en plus diplômés, 10\% n'ont aucuns diplômes, 21\% ont un CAP ou équivalent en 1998, alors que les détenteurs d'un diplôme de l'enseignement supérieur sont à 33\% des adhérents. \\
En 2011, 50\% des adhérents sont des cadres supérieurs, 3\% sont ouvriers, 14\% sont employés. \\
Il y a donc un désajustement idéologique, sociologique et organisationnelle. 


\chapter{Le communisme}

Le communisme a largement marqué la vie politique française via l'intermédiaire du PCF qui a certainement été le seul véritable parti de masse que la France ait connu. Le PC est un parti très important, notamment après la seconde guerre mondiale où près d'un quart des électeurs votent pour le PCF. \\
Le communisme c'est un phénomène multiforme qui relève d'un corpus doctrinal très important: le marxisme. C'est aussi un modèle organisationnel très particulier, fortement hiérarchisé, très centralisé. Ce modèle se transmet internationalement sur le modèle du Parti communiste soviétique. Le PCF, bien qu'il n'ait été que très rarement au pouvoir, reste un parti très important en France. Il possède un dispositif identitaire, il a une contre-culture, contre la culture officielle ou légitime. \\
Le PCF a fait l'objet de beaucoup d'études, sociologiques ou non. C'est un parti qui a beaucoup fasciné ceux qui l'ont étudiés. Être communiste signifie bien plus que de cotisé afin d'être dans un parti politique. C'est un ensemble de pratiques et de visions du monde. \\
Certains analystes du PCF l'ont qualifié de parti milieu de vie, où ses membres sont communistes "du berceau jusqu'à la tombe". \\
On peut distinguer deux dimensions qui sont constitutives du communisme: une téléologique (l'histoire a un sens), qui renvoie à l'idéologie défendu par le PCF (le projet révolutionnaire). Cela assure l'unité et la cohésion du mouvement autour d'un certain nombre de lignes d'actions. \\
Dimension sociétale: renvoie à l'ancrage du PCF dans la société, aux liens sociaux complexes que le PC a su nouer, en particulier avec les ouvriers. \\
Ces analyses démontrent qu'il y a une pluralité des formes que peut prendre le communisme, tant sur le plan national qu'international. Les formes d'implantation sont différentes géographiquement. Le PCF est un parti qui a su se territorialiser. Le PCF a une propension à se donner à voir comme un ensemble homogène, uni comme un seul homme derrière son chef. 


Comment le PCF a pu devenir une force politique aussi importante ? Quelles sont ses mutations, voir son déclin ?

\section{La fondation du PCF}

Il est fondé en 1920 à l'occasion du congrès de Tours qui donne naissance à la SFIC. C'est dans le contexte de l'après première guerre mondiale que la SFIO se divise, mais surtout à la suite de la Révolution bolchevique de 1917. Les tensions se portent surtout sur l'adhésion à l'internationale communiste. Une tendance très majoritaire à la SFIO (67\% des délégués du congrès) se prononcent en faveur de l'adhésion et en faveur des 21 conditions fixés par Lénine parmi lesquelles il faut créer des partis révolutionnaires fortement structuré et discipliné. \\
La scission qui a lieu au congrès de Tours renvoie certes à un clivage d'ordre idéologique mais renvoie aussi à un clivage social entre groupes militants et élus qui eux, resteront à la SFIO. Ceux qui décident de quitter la SFIO sont essentiellement jeunes, ils aspirent à un véritable renouveau politique que l'internationale communiste a su incarner. \\
10 000 adhérents fondent la SFIC et emportent avec eux l'Humanité. Cette scission est suivie, l'année d'après, d'une scission syndical vu que la CGT se divise pour former la CGTU proche de la SFIC. Cette scission perdu jusqu'en Mars 1936, année de réunification des deux CGT. \\


Dans les années 20, le PC entre dans une phase de Bolchevisation, le parti va acquérir les attributs du parti bolchevique, c'est le début d'une organisation très stricte, hiérarchisé, discipliné, quasi-militaire. Le parti contrôle très strictement ses membres et les individus les moins révolutionnaires sont mêmes exclus du Parti si il le faut. Surtout, dans les années 20, la SFIC va s'ouvriérisé, la politique de recrutement de la SFIC va viser très clairement les ouvriers, c'est une politique volontariste. \\
La SFIC forme également ces ouvriers dans les écoles du Parti. Ils sont mêmes parfois promu pour à des postes de responsabilité. En 1932, près de trois quart des députés communistes sont d'origine ouvrière. Deux tiers des membres du comité central élu en 1936 sont ouvriers. Cette élite politique est d'autant dévoué au parti qu'elle lui doit tout, sa promotion politique, son éducation, sa promotion sociale. Pierre Bourdieu appelle cela être le rapport de remise de soi au Parti. \\
Autre attribut de la bolchevisation: la mise en place du centralisme démocratique. Ce centralisme démocratique assure la prééminence du comité central sur la base du Parti. C'est l'idée que la ligne du parti peut être débattu, peut faire l'objet de discussion, mais une fois que la décision est actée, il n'est plus question de la remettre en cause. C'est la distinction entre le temps du débat et le temps de l'action politique. C'est ce que Lénine appelait la liberté de discussion contrebalancé par l'unité dans l'action. \\
Le PC adopte donc, d'un point de vue politique, une stratégie de lutte de classe, n'acceptant aucune alliance avec les autres partis. Ces orientations provoquent une crise dans le parti, en une dizaine d'année, le PC va perdre deux tiers de ses adhérents. Sur le plan électoral, il en va de même, largement dominé par la SFIO. En 1924, le PC a 19 députés et moins de 10\% des voix. En 1932, il possède 11 députés et 8,5\% des voix. \\
Ce parti devient un groupuscule avec des militants très organisés, professionnalisés mais un groupuscule ayant au final peu d'influence dans le pays.


Maurice Thorez est désigné secrétaire général du parti en 1930 et le restera jusqu'à sa mort en 1964. C'est sous Maurice Thorez que le PCF va adopter une nouvelle ligne: l'association avec d'autres partis, la stratégie du front populaire. Le PC soutiendra donc le gouvernement de Léon Blum mais sans y participer. Le PC va jouer un rôle dans les grèves de 1936. C'est à ce moment là que Thorez prononcera "Il faut savoir terminé une grève dès que satisfaction a été obtenu" après les accords de Matignon.\\
Cette nouvelle ligne est un succès. Sur le plan électoral, le PC gagne des positions. En 1936, c'est la première percée électorale du PC qui obtient 72 députés avec 1.5 millions de voix. Le PC atteint 300 000 adhérents en 1937. \\
Grâce à cette nouvelle ligne politique, le PC récolte le fruit des réformes organisationnels opérés dans les années 20. Dans les années 30, le PC revendique le monopole de la représentation ouvrière dans le milieu politique, à raison. \\
Le Parti s'implante dans le Nord de la France et dans la couronne parisienne. 

\section{L'âge d'or du PCF}

Le PC se construit autour de bastions dans l'entre deux guerres. C'est à la libération que le PCF va prendre une ampeur par rapport à ces bastions. Le PCF devient sous la IVe le plus grand partie de France. Il avoisine régulièrement les 25\% des voix sous la IVe. Il y a plus d'un million d'adhérents à la libération. Les zones d'implantation électorale et militante s'étendent: Nord, banlieue rouge, autour du massif central (dans l'allier), la bordure méditerranéenne, la région de Saint-Nazaire, autour des chantiers. \\
Ce développement fulgurant s'explique par le sortir de la guerre, où le PC est auréolé de sa participation à la résistance. Le PC ne s'est pas compromis avec Vichy, d'une part, mais a aussi été très actif dans la résistance. Il se revendiquait comme le parti des "70 000 fusillés", à tort probablement, mais la participation du PC à la résistance ne fait pas de doutes. La contribution de l'URSS à la défaite de l'Allemagne Nazi aide aussi. \\
Ce sont les années 50 et 60 qui marquent l'apogée du PCF en France. Le parti est plus qu'un parti, c'est une organisation, très puissante, avec tout un ensemble d'organe de presse (l'Humanité, plus d'autres, spécialisés), un ensemble d'association, de clubs sportifs, etc. \\
La prise en charge des adhérents est donc total du berceau jusqu'à la tombe, dans toutes les étapes de leur vie. La socialisation communiste est très prégnante. \\
Le succès du communisme dans les milieux intellectuels et artistiques est très marqué (Picasso, ayant été un compagnon de route du Parti par exemple). Malgré tout, l'assise sociale du PC reste la classe ouvrière. \\
C'est aussi un parti contre société, fournissant à ses membres des réseaux de sociabilités, une identité communiste et ouvrière très forte. Cette dimension identitaire est vraiment très forte au PC, qui valorise une manière d'être sociale qui renvoie à son appartenance de classe (manière de s'habiller, comportement, etc.). Il contribue ainsi à donner une dignité à la classe ouvrière et aux ouvriers en règles générales. \\
Le PC a su monopoliser la représentation de la classe ouvrière qu'il a lui même contribué à objectivé (à façonner, voire à créer). \\
Sur le plan politique, au sortir de la guerre, le PC participe aux affaires gouvernementales dans le cadre du tri-partisme, une alliance entre le PCF, la SFIO et le MRP, une sorte de gouvernement d'Union nationale avec pour objectif très clair de reconstruire le pays. \\
Une assemblée constituante vote la constitution de la IVe République sous l'impulsion du tri-partisme. \\
En 1947, le PC commence à parler de ses divergences politiques: la répression intense menée en Indochine, de plus, les communistes soutiennent les salariés en grève contre le gouvernement de 1947. \\
Finalement, le PC quitte le Gouvernement en Mai 1947, devenant parti d'opposition et de contestation avec comme bras armé la CGT qui représente 25\% des syndiqués. 


Dans le contexte de la guerre froide, le PCF va adopter une ligne politique offensive, contre la "tutelle" américaine. Le PC se présente également comme promoteur de paix face aux US, comme des pacifistes convaincus. \\
Staline meurt en 1953, Khrouchtchev engage la déstalinisation en 1956. À la fin des années 50, la balance du PC dans la politique va radicalement changé. \\
Maurice Thorez meurt en 1964, c'est d'abord Waldeck Rochet qui le remplace puis George Marchais de 1969 à 1994. Le PC va s'engager très progressivement vers l'union de la gauche. \\
Le PC va abandonner sa référence à la dictature du prolétariat en 1976. Mais il n'y a pas de réelle distance entre le PCF et le PC Soviétique. \\
Le PC va entrer dans une phase de déclin à partir des années 80. 

\section{Le déclin du PCF}

À partir des années 80, on observe une désagrégation progressive des structures communistes qui assurait les assises du PC dans les communautés ouvrières. \\
D'un point de vue électoral, le déclin se fait très progressivement, mais très régulièrement. Le premier palier est en 1981 où Marchais n'obtient que 15\% des voix, ce qui est le plus faible score enregistré depuis la libération. Lors des législatives suivantes, le déclin se confirme, ils font moins de 10\% des voix. En 1988, ils ont 11\% des voix, puis à peine 9\% en 1993. En 1997, le PCF réalise un peu moins de 10\% des voix aux législatives. \\
C'est à partir de 2002 que l'on enregistre un troisième palier de descente, lors de la présidentielle de 2002, Robert Hue dépasse à peine les 3\%. Aux législatives de 2002, ils font moins de 5\% des voix. C'est de justesse que les communistes forment des groupes parlementaires. \\
En 2007, Marie-George Buffet fait un très mauvais score, à peine 2\% des voix. Le PC obtient 4,3\% des suffrages aux législatives de 2007. \\
Ensuite, à partir de 2008, le PCF s'engage dans une coalition partisane, le FdG. Les scores enregistrés par la coalition sont bien plus important, mais cela n'empêche pas un tournant: le candidat soutenu par le PC n'est pas communiste. \\
Les bastions communistes sont aussi de plus en plus faible. Les municipales restent néanmoins une place assez forte pour le PC qui garde globalement la main sur la couronne parisienne. 


En parallèle au déclin électoral, il y a un déclin identitaire et organisationnel. Il est lié notamment au déclin syndical que la CGT connaît aussi: 5\% de salariés cotisent à la CGT à la fin des années 80, début des années 90. Il en va de même pour le tirage de l'Huma pour qui les ventes sont en chutes libre. Le Parti est vieillissant, un quart des adhérents est âgé de 60 ans et plus. Des tensions apparaissent entre les ouvriers traditionnels et des militants plus jeunes, souvent originaire de classe moyenne et ayant rejoint le PC tardivement, ceux-ci sont assez sensible aux mutations du parti engagés dans les années 70. Dans les années 80, des mouvements de dissidence apparaissent dans les rangs du parti. \\
Plusieurs facteurs entrent en ligne de compte. Le régime de la cinquième République n'est pas du tout favorable au Parti Communiste. L'élection du Président de la République au scrutin uninominal à deux tours favorise une logique des reports de voix et une logique de "vote utile", ce qui accentue une concurrence avec le PCF. La présidentialisation du régime va défavoriser le PC. Ce sont les socialistes qui tirent tous les bénéfices du programme commun. \\
Un deuxième facteur est la crise du référent soviétique. À partir des années 60 et dans les années 70 se développent une thématique anti-totalitaire, très critique envers l'URSS et qui vient ternir son image. Les intellectuels désertent le PC. À partir de 1989, le communisme n'apparaît plus comme une alternative viable/fiable. \\
Un troisième facteur est l'érosion progressive de la classe ouvrière. Le sentiment d'appartenance à la classe ouvrière se dilue. Le monde ouvrier tend à se fragmenter, voir à se désagréger. Le PC n'était pas qu'un parti électoral mais un parti cadre de vie, et la dislocation de ces groupes a été fatale au PCF. Les cellules d'entreprise sont progressivement démantelées. La sociabilité ouvrière tend à s'éroder. La classe ouvrière n'a plus la même consistance qu'elle avait dans le passé. Les modes d'organisation du travail ont aussi joué un rôle dans la dislocation du monde ouvrier: les salariés travaillent de plus en plus individuellement.


Julian Mischi a récemment publié une étude qui apporte un point de vue original pour expliquer le déclin du PCF. Il ajoute une approche lié aux dynamiques interne du déclin du PCF. Il s'intéresse aux choix stratégiques, aux mutations idéologiques. D'un certain point de vue, le PC est responsable de son propre déclin. Il s'intéresse aux logiques endogènes au déclin communiste. Il met en exergue un processus de désouvriérisation du PC. Il considère que ce n'est pas seulement la conséquence du déclin du PC, c'en est une cause. La désouvriérisation se fait en plusieurs étapes. Il y en a au moins deux, d'abord dans les discours, ensuite en pratique dans le recrutement d'individus. \\
À partir des années 70, il y a une inflexion du discours du PCF qui se présente comme le parti de la classe ouvrière. Cette inflexion est dû à l'émergence d'un discours misérabiliste. Après la référence à la classe ouvrière combattante de 1936 et de la résistance, le PC tend à chercher à devenir le porte parole non pas de cette classe ouvrière héroïque mais celui des "pauvres" en général. Les ouvriers se reconnaissent assez peu dans cette image assez dévalorisante. C'est pour Julian Mischi une erreur stratégique du PCF. \\
Dans un second temps, dans les années 90, le PCF ne vise même plus à donner la priorité à la représentation des classes populaires, il élargit son discours en croyant qu'il va élargir son audience. Le PC cherche désormais à représenter l'ensemble de la société française, dans toute sa diversité. Pour Mischi, c'est véritablement le discours de lutte des classes qui disparaît dans les années 90. \\
Concernant le recrutement ouvrier, il diminue sensiblement dans les années 70 et diminue encore plus dans les années 90. La suppression des écoles du parti dans les années 90 permet au phénomène de prendre de l'ampleur. \\
Le résultat en est donc une désouvriérisation à tous les niveaux de l'organisation. Ceux qui manquent de capital politique se retrouvent auto-exclus du Parti. \\
Progressivement, on passe d'une élite partisane composée d'anciens ouvriers à une élite issu de la classe moyenne, qui sont des professionnels ou des semi-professionnels (employés dans une collectivité et membre d'un parti pour le contact). On notera que le dernier secrétaire général du PCF était George Marchais. C'est d'ailleurs le dernier à avoir été à la CGT. Par la suite, tous les dirigeants du PCF ont évité l'usine et la CGT. \\
Robert Hue est infirmier de profession et son recrutement à la tête du PC marque une véritable tournant. C'est un élu local, n'a jamais été permanent du Parti et à partir du Robert Hue, tous les dirigeants connaîtront des trajectoires similaires. Marie-Georges Buffet était employé dans une collectivité et Pierre Laurent est journaliste de profession. \\
La socialisation militante et syndicale de ces leaders se limite exclusivement dans des mouvement sociaux étudiants, notamment à l'UNEF. \\
Aux élections européennes de 2004, les listes du PCF étaient ouvertes à la société civile. \\
On note une individualisation du rapport au PCF, une affirmation de la primauté de l'adhérent sur l'organisation.  


\chapter{La Droite}

Il est plus facile de définir la gauche que la droite dans la mesure où la gauche accepte plus facilement une vision conflictuelle de la politique. C'est plutôt la gauche qui politise les rapports sociaux quand la droite est plus encline à les dépolitiser. \\
On serait tenté de définir la droite par son prétendu conservatisme. On peut tenter par contre de définir la Droite par sa pratique du pouvoir, plus nombreuse dans l'histoire. On pourrait donc considéré comme étant de Droite ce qui est exclu par la gauche et qui accepte volontiers cette exclusion. \\
La droite a aussi beaucoup moins été étudié. Il y a une plus faible affinité entre la droite et les milieux intellectuels. La Droite se donne moins souvent à voir comme organisation politique ou partisane. La droite est en effet marquée par sa pluralité, on devrait d'ailleurs parler des droites. La droite est constamment en train de changer. \\
De manière classique, René Raimond distingue trois traditions de la Droite. La Droite légitimiste d'abord, celle qui apparaît avec la restauration, très traditionaliste, proche du catholicisme, elle est monarchiste, et rejette la Révolution Française et son héritage. On a la Droite orléaniste qui naît avec la monarchie de Juillet, une Droite modérée, favorable à la Monarchie constitutionnelle, modérée ; ils sont libéraux sur le plan économique, accepte le pluralisme et une conception clivée de la politique. La troisième droite est la droite bonapartiste qui apparaît avec Napoléon, une Droite autoritaire, attachée à l'ordre, à l'État, à l'idée du volontarisme, qui fait appel au peuple via le référendum et qui pratique un certain culte du chef. \\
Cette typologie peut être utile et peut nous aider à y voir plus clair dans cette nébuleuse composite. 


Le problème des droites est qu'elles doivent gérer cet héritage historique qui est pluriel et donc les conséquences que cela implique d'un point de vue organisationnel. \\
De plus, la droite a toujours eu un rapport très particulier à la forme de l'organisation partisane. Les partis de droite sont généralement peu structuré, peu formalisé, peu centralisé. Cela ne signifie pas pour autant que la droite n'a aucune force sociale. 


\section{La Droite lors de la première partie du XXe siècle}

C'est une tendance assez hétérogène, les différents groupes ont des références tantôt catholiques, tantôt laïque, ils sont tardivement ralliés à la République ou sont des républicains de la première heure. Cette droite va connaître des recompositions tout au long du XXe siècle, au gré des échéances électorales avec des alliances qui peuvent s'élargir jusqu'aux radicaux. \\
La principale caractéristique de ces mouvements est la faiblesse de leur organisation. Contrairement à la SFIO et à la SFIC, ils ne s'appuient pas sur des mobilisations, ils s'appuient sur leurs élus, sur leurs notables. Ce sont des partis qui sont des rassemblements de personnalités qu'on appelle des notables car ils disposent de ressources fortes qui sont indépendantes des organisations politiques qui les fédères. La fédération républicaine ne compte que 3 000 cotisants par exemple. \\
L'organisation en forme partisane donne une envergure nationale à leur représentation. Pour autant, bien qu'ils acceptent la forme partisane, ils refusent toutes les formes de servitudes ou de dépendances qui lui sont liées (l'homogénéité des votes à l'AN par exemple). 


Après la seconde guerre mondiale, la Droite française va connaître une forte recomposition, liée au discrédit de la plupart des partis de Droite qui se sont compromis en votant les pleins pouvoirs à Pétain ou même qui ont collaboré avec Pétain ou avec l'occupant. Plus de 90\% des députés de la fédération républicaine sont déclarés inéligibles. \\
La gauche domine l'échiquier politique au sortir de la guerre, avec un PCF très puissant. Si on rajoute le MRP, le Parti de De Gaulle,  ils constituent les trois quarts des suffrages aux élections de 1946. \\
Cette situation n'est pas durable, la résurgence de la Droite va se faire. Le mythe de la résistance unifiée va très rapidement s'émietter. Des divergences profondes vont apparaître entre les communistes et le MRP. C'est à partir de 1947 quand les communistes quittent le Gouvernement que le clivage gauche/droite va ré-apparaître. \\
Les Républicains Indépendants, issus de la fédération républicaine constituent une force importante de la Droite. Cette formation est fondée en 1949, elle prend un peu plus tard le nom de CNIP (Centre National des Indépendants et Paysans). C'est un parti conservateur et traditionaliste.  \\
La Droite va revenir au pouvoir à partir des années 1950. Le CNIP obtient 13,5\% des voix en 1951. Près de 15\% des voix en 1956. Le Président du Conseil de 1952 est issu du CNIP: c'est Antoine Pinet. Il est investi par les radicaux, le MRP et les modérés.


La seconde force politique de Droite est le MRP, né en 1944, ce sont des démocrates chrétiens. C'est un mouvement politique issu de la Résistance derrière George Bidault. \\
Le MRP devient le premier parti de France avec 28\% des voix et 166 députés en 1946. Il déclinera progressivement, moins de 11\% en 1956. \\
Le MRP participe aux coalitions gouvernementales. Robert Schuman est aussi un grand nom du MRP. \\
Le MRP peut être vu comme un catholicisme social, se voulant une troisième voie entre la SFIO à gauche et les libéraux à Droite. C'est un mouvement ouvert sur le monde ouvrier, défend la liberté, le respect des individus ou encore l'idée de la défense d'une certaine justice sociale. \\
C'est donc sous bien des aspects un mouvement de Gauche. Certains leaders du MRP soutiennent d'ailleurs la pertinence de la lutte des classes, acceptent les revendications des ouvriers. D'un certain point de vue, le succès du MRP relève d'un malentendu historique, on observe un décalage très important entre les discours, marqués à gauche et l'électorat qui, sociologiquement est marqué à droite. L'explication tient sans doute au fait du sortir de la guerre et le fait qu'il se présente aussi comme la seule alternative à droite. Il constitue donc en quelque sorte le seul débouché politique qui semble viable et fiable pour tout un ensemble d'individus. Le MRP permettait aussi de faire barrage au PC.


L'émergence du CNIP va saper l'audience du MRP. L'émergence du RPF Gaulliste va contribuer également à saper l'audience du MRP et à brouiller son positionnement idéologique. \\
Entre 1946 et 1956, le MRP perd les trois quarts de ses membres, 200 000 militants en 1946, 40 000 en 1956.

\subsection{Le gaullisme}

La droite Gaulliste émerge à partir de 1947 avec le rassemblement du peuple français. Le Gaullisme ne souhaite pas se faire voir comme une idéologie à proprement parler. Le Gaullisme s'apparente à bien des aspects à une forme de pragmatisme mais conserve une définition assez malléable. On peut cependant y identifier un certain nombre de valeur propre: la grandeur nationale, la légitimité de l'État, les vertus du rassemblement national, la souveraineté nationale, le volontarisme, le culte du peuple et aussi du chef. L'idée du Gaullisme est que le système des partis politiques ne doit pas prendre le pas sur la souveraineté populaire. Il s'agit aussi d'une forme de nationalisme, plutôt ouvert, non renfermé sur lui même. En définitive, le Gaullisme est un mélange d'étatisme, de libéralisme et de dirigisme. \\
À partir de 1958, le Gaullisme va connaître son hégémonie. 

\subsubsection{Première séquence du Gaullisme: le RPF}

La première séquence du Gaullisme s'ouvre avec la fondation du RPF en 1947. De Gaulle abandonne le Gouvernement de La Libération en 1946. Il espère être sollicité de nouveau, ce qui n'est finalement pas le cas en réalité. Il se retrouve donc relégué au second plan de la vie politique, isolé dans le champ politique. Il entre en opposition franche et claire de la IVe République, lors du célèbre discours de Bayeux. Dans ce discours, en Juin 1946, De Gaulle critique fortement la IVe République et appelle à une nouvelle Constitution dont il définit les grandes lignes. De Gaulle va devenir, paradoxalement, avec les communistes, le principal opposant à la IVe République. \\
Lorsqu'il fonde le RPF, De Gaulle espèce transformer sa notoriété en un mouvement durable, c'est un véritable succès dans un premier temps puisque le RPF s'appuie sur d'anciens réseaux de résistants. Aux élections municipales de 1947, le parti du Général obtient près de 40\% des voix, essentiellement au dépens du MRP. \\
Si le RPF peut se voir comme un Parti, De Gaulle revendique le fait que ce n'est pas un parti, vu qu'il ne porte pas ces organisations dans son coeur. Le RPF ne souhaite pas être identifier aux partis classiques. \\
En 1947, le RPF compterai plus d'un million d'adhérents, une estimation récente rabaisse le chiffre à 400 000 adhérents. Le RPF est massif. En termes organisationnel et de nombre d'adhérents, le RPF est le deuxième parti politique de France derrière les communistes. 


Le RPF est un parti qui regroupe des acteurs de différentes classes sociales: on y compte beaucoup d'ouvriers. Très rapidement, les résultats électoraux du RPF vont dégringoler. De Gaulle ne parvient pas à traduire sur le plan électoral l'image positive dont il bénéficie. Aux législatives suivantes, il obtient 21\% des voix. En Mars 1952, les parlementaires gaullistes vont se diviser sur la question de l'investiture d'Antoine Pinay. La discipline partisane éclate, marquant le début de la décroissance brutale des résultats électoraux du RPF. \\
En 1953, aux municipales, le RPF subit un revers cinglant et De Gaulle se retire peu à peu du mouvement, qui sera mis au sommeil en 1955. 

\subsubsection{Le Gaullisme de Gouvernement: de l'UNR à l'UDR}

Suite à l'insurrection Algérienne, le Général de Gaulle est rappelé. Il y a toute un conjoncture particulière: crise de la IVe République, crise algérienne. De Gaulle est rappelé en Mai 1958 par l'AN qui lui demande de présidé le Conseil des Ministres et rédiger une nouvelle Constitution. \\
Le 4 Octobre 1958, la Ve République est proclamé, et dès Novembre 1958, aux législatives, les Gaullistes recueillent plus de 20\% des suffrages et 212 sièges de députés. En Décembre 1958 se tient la première élection du Président de la République au suffrage universel indirect avec un collège de 280 000 électeurs qui élit De Gaulle avec plus de 70\% des voix. \\
Une nouvelle formation Gaulliste partisane est créée: l'UNR, l'Union pour la Nouvelle République. L'un des principaux animateurs de ce mouvement est Jacques Chaban-Delmas. L'UNR ne se constitue pas du tout comme un parti de masse sur le plan organisationnel, c'est au contraire une organisation partisane, directement lié à la prise du pouvoir. L'existence même de l'UNR découle de la conquête du pouvoir par les gaullistes. \\
Lors des élections législatives de 1962, l'UNR obtient 32\% des suffrages, et en 1967, 38\% des suffrages. \\
L'effondrement parallèle du MRP met l'UNR en situation de quasi-monopole à droite, devenant le parti représentant la droite. 


En 1967, le parti change de nom et devient l'UDR (Union des Démocrates pour la Ve République). Cela est fait avec l'objectif d'élargir la base électorale. Une nouvelle équipe dirigeante est mise en place, mais le mouvement est stoppé par la crise de Mai 1968. Après les évènements de Mai 1968, les Gaullistes remportent une victoire historique ("élections de la peur"), qui remporte 43-44\% des suffrages. \\
En 1969, De Gaulle démissionnera à la suite d'un référendum sur sa réforme régionale. \\
Ce départ va ouvrir une période d'instabilité où l'héritage de De Gaulle va être très disputé. 

\subsection{Les mutations du néo-Gaullisme: le RPR}

Georges Pompidou, Premier Ministre de De Gaulle, une sorte de conservateur éclairé, apparaît nettement moins tranché que le général De Gaulle. Il souhaite le rassemblement, la synthèse entre une droite autoritaire et une droite plus libérale. C'est sous sa présidence que les dissensions vont apparaître. Elles se traduisent essentiellement par un conflit entre Pompidou lui même et son PM, Chaban-Delmas. Ce dernier va d'ailleurs démissionner en 1972. Chaban-Delmas montre des signes de volonté d'ouverture, de libéralisation... Les gaullistes de Gauche quittent le Parti, Pompidou disparaît en Avril 1974, lançant une guerre de succession. \\
Deux candidats se présentent suite à la mort de Pompidou, Chaban-Delmas et VGE, qui lui, appartient à la Droie Modérée. L'ensemble des gaullistes n'apportent qu'un soutient très timide à Chaban-Delmas. À quelques mois du scrutin, 43 personnalités gaullistes marquent publiquement leur défiance envers Chaban-Delmas. À la tête de ces 43, se trouve un ministre, Jacques Chirac. Chaban-Delmas est éliminé dès le premier tour, et ce sera VGE qui sera le représentant de la Droite au second tour. Jacques Chirac devient son PM. C'est ainsi que Chirac devient le leader de la Droite Gaulliste.


Cependant, Chirac, en 1976, manifeste son désaccord avec VGE et démissionne de Matignon. Chirac entend en effet, préparer les élections présidentielles à venir et fonde en 1976 une nouvelle formation politique: le RPR, qu'il va présidé. Il place des proches aux postes clés et suscite une vague d'adhésion, les Chiraquiens vont tenter de remplacer progressivement les Gaullistes historiques. Ces nouveaux responsables sont en général un peu plus jeunes, les vertus de l'engagement militant sont mis en avant au sein du RPR. Les Gaullistes historiques, progressivement s'efface, au profit des chiraquiens, le plus souvent formé dans des grandes écoles, comme Alain Juppé. \\
Il s'agit pour Chirac de préparer la reconquête du pouvoir. Les Gaullistes se retrouvent dans une situation particulière, pas tout à fait dans l'opposition mais pas tout à fait au pouvoir, vu qu'il a démissionné. Il s'agit donc d'organiser la reconquête du pouvoir. \\
Chirac adopte une position assez originale, il se dit favorable à un travaillisme à la française, il est donc plutôt hostile au libéralisme économique, il se démarque aussi par ses positions concernant la construction européenne, le parti est très critique sur l'Europe. Pendant les élections européennes de 1979, Chirac dénonce les positions de l'UDF, parti centriste, parti pro-européen. \\
Chirac est par ailleurs très opposé au libéralisme culturel de VGE. 


Aux élections présidentielles de 1981, Chirac n'obtient que 18\% des voix au premier tour, soutient très timidement VGE pour le second tour. \\
1981 marque dans l'histoire du Gaullisme une phase nouvelle, celle du néo-Gaullisme d'opposition. La situation à Droite va être bouleversée par la victoire de la Gauche, puisqu'elle va entrer en opposition, ce qui est nouveau pour elle. \\
Le RPR s'impose progressivement comme le principal parti de la droite à la faveur d'un renouveau idéologique important. Le RPR se convertit au libéralisme économique, ce qui est nouveau pour un courant qui se réclame du Gaullisme. Sous Chirac, à partir des années 80, le RPR se convertit au libéralisme économique mais aussi à la construction européenne. \\
Du même coup, les différences entre droite et centre vont s'amenuir voire disparaître. Ils adoptent une stratégie d'union programmatique aux européennes de 1984. Le RPR et l'UDF font liste commune en 1984, conduite par Simone Veil. La pensée néo-libérale se développe au sein du RPR. Elle tranche avec le discours traditionnellement interventionniste du Gaullisme. On critique l'État providence, on encourage l'initiative privée, la liberté d'entreprendre et on commence à promouvoir les normes et les réformes managériales. Reagan et Thatcher deviennent les idoles du RPR. 


Entre 1986 et 1988, lors de la première cohabitation, Jacques Chirac mène une politique très libérale. Il s'agit d'un libéralisme économique, alors que le plan culturel, il est conservateur. \\
La radicalisation d'une partie de l'opinion face au pouvoir socialiste et communiste, peut expliquer cette conversion au libéralisme, qui semble être le seul salut face au socialo-communisme. \\
Il faut aussi dire que sur le plan sociologique, le RPR a évolué. Le déclin du militant populaire est très important au RPR. C'est également vrai au niveau de l'électorat, le RPR connaît un embourgeoisement de son électorat. 


En 1988, Chirac est de nouveau candidat à la présidentielle mais subit de nouveau une défaite. Les Gaullistes se retrouvent donc une nouvelle fois dans l'opposition. La Droite n'est donc plus le parti du pouvoir, c'est une petite révolution culturelle, profonde pour elle. Les néo-gaullistes ne doutaient pas d'être majoritaire dans le pays et doivent donc s'acclimater à ce statut d'opposants. \\
À partir des années 80, des courants s'organisent au sein du RPR, ce qui est inédit au vu de la culture du chef qui existe chez les néo-gaullistes. \\
D'un côté on a un courant néo-gaulliste, qui dénonce la libéralisation du Parti, rappelle à un ressource aux sources (Pasqua, Séguin), de l'autre on a un courant moderniste, plus libéral, et pro-européen, plus proche des centristes et de l'UDF. Le RPR va par exemple se déchirer à l'occasion du réferendum à l'occasion du traité de Maastricht en 1992. Le RPR n'a pas de position officielle. À titre personnel, le Président défend le Oui alors que Séguin dirige une campagne pour le Non. \\
Les législatives de 1993 donnent une large majorité à la droite, alliée avec les centristes. Les résultats montrent plutôt une victoire d'une opposition à la gauche qu'un soutien clair et précis à la droite. 


À la seconde cohabitation, Chirac souhaite prendre du recul et préparer les prochaines présidentielles. C'est donc Balladur qui deviendra PM. Il capitalise rapidement grâce à cette fonction de nombreux soutiens à Droite et sa popularité très forte dans l'opinion en fait un présidentiable potentiel pour l'élection de 1995. \\
Balladur et Chirac sont tous deux candidats aux élections de 1995. Balladur est soutenu par l'UDF, alors que Chirac est soutenu par le RPR. Balladur fait la course en tête des sondages. Cependant, il ne remporte pas cette élection. \\
Pendant cette campagne présidentielle, Chirac renoue avec un discours volontariste sur le plan économique. Il fait de la fracture sociale et la lutte contre le chômage son cheval de bataille. Il opère lors de cette campagne une sorte de retour au source du Gaullisme. Alain Juppé qui devient PM, mettra cependant en place une politique libérale et fera face à un des conflits sociaux les plus importants depuis 1995. \\
Chirac décide de conforter sa majorité parlementaire en 1997, en dissolvant l'AN. La Gauche remportera les élections législatives, provocant une troisième cohabitation qui durera jusqu'en 2002. 


Pendant ce temps là, les luttes internes s'intensifient au sein du RPR dirigé par Séguin. Séguin démissionnera suite à l'échec du RPR aux élections européennes de 1999, et sera remplacé par Michelle Alliot-Marie qui pour la première fois, est élu par les militants du RPR. Le RPR se banalise, devient un parti de Droite sans spécificités particulières. Il perd également sa spécificité organisationnelle car devient dominé par les élus locaux. \\
Au début des années 2000, le RPR compte environ 80 000 militants. 


En 2002, le RPR se transforme en UMP. Tout au long de son histoire, la droite a été divisée comme en 1988 ou en 1995. En 2002, la situation de la Droite est à nouveau la même, elle part divisée, Chirac est candidat pour le RPR, Madelin pour Démocratie Libéral, Bayrou pour l'UDF. \\
Dans l'entre deux tour de l'élection présidentielle de 2002, le résultat du premier tour permet à Chirac d'imposer un parti unique à Droite et il crée entre les deux tours de l'élection présidentielle l'UMP, l'Union pour la Majorité Présidentielle. \\
Aux législatives de 2002, Chirac oblige les candidats de Droite à se présenter sous l'étiquette de l'UMP. Chirac aura un grand nombre de députés sous son étiquette: 365 députés UMP sont élus en 2002. Sur ces 365, 210 viennent du RPR, 73 de l'UDF et 62 de Démocratie Libérale. \\
Le congrès fondateur de l'UMP n'a lieu que quelques mois plus tard, en Novembre 2002. Les organisations fondatrices sont destinées à disparaître, mais certains ont des réticences, notamment Bayrou qui refusera de rejoindre pleinement l'UMP. \\
C'est Alain Juppé qui est désigné à la tête du mouvement. Si c'est textuellement un parti de masse qui laisse la parole aux militants, qui doivent désigner le chef du parti, il n'y a pas de réel courant à l'UMP: Juppé se fait élire à la tête du parti avec plus de 75\% des voix. \\
L'UMP revendique 200 000 adhérents alors qu'il n'y a que 50 000 votants au congrès de 2002. L'UMP a ensuite assis son hégémonie à Droite. Sarkozy prend les rennes du parti en 2004 et il va entreprendre un certain nombre de réforme en profondeur de l'UMP, notamment l'importation de techniques managériales qui visent à augmenter sensiblement le recrutement. Il y parvient car en 2007, le parti compte 300 000 adhérents. \\
Sarkozy propose une nouvelle synthèse idéologique à Droite, faite de libéralisme économique, emprunt du modèle anglo-saxon. Faite aussi de volontarisme républicain, et enfin, d'autoritarisme, avec l'idée qu'il vaut mieux une bonne répression qu'une inutile prévention. \\
C'est sous Sarkozy que l'UMP va pour la première fois faire désigner son candidat à la présidentielle par les militants. Cela sera reproduit en 2016.  

\chapter{L'extrême droite}

Sur le plan historique, c'est une constellation avec de nombreuses ramification. Il ne faut pas donner à l'extrême droite l'homogénéité qu'elle n'a pas même si on peut identifier un certain nombre de constante: l'opposition à la culture républicaine, une opposition au parlementarisme, le refus des valeurs de tolérance et de progrès, un nationalisme très important, exacerbé, fondé sur l'existence d'une communauté de sang. \\
De nombreux historiens ont cherchés à créer des typologies pour étudier la droite, c'est le cas de Pierre Milza. Il identifie d'abord ce qu'il appelle le fascisme à la française, le traditionalisme ultra-conservateur, le national-populisme. 


Pour le fascisme à la française, il faut se méfier de ce terme, le manier avec beaucoup de précaution car renvoie à une séquence historique précise, mais a souvent été utilisé par la gauche pour discréditer les groupes d'extrême droite. Ce terme renvoie à une composante très marginale de l'extrême droite que l'on rencontre seulement dans les années 30 et sous l'occupation. C'est le cas notamment du Parti Populaire Français, fondé en 1936 par Jacques Doriot. C'est un parti très anti-communiste, et a joué un rôle important pendant l'occupation dans la collaboration. 


Le traditionalisme ultra-conservateur est issu de la tradition légitimiste de la droite. C'est un parti royaliste, rallié tardivement à la République. Il est aujourd'hui incarné par Philippe de Villiers. \\
Ils ont longtemps été attachés à l'ancienne France et à la monarchie, il est aussi l'héritier de l'action française, ligue d'extrême droite, née des suites de l'affaire Dreyfus. Maurras, théoricien de ce mouvement est partisan d'un nationalisme intégral et d'une monarchie héréditaire. 


Le national populisme est incarné en France par Jean-Marie Le Pen, qui a permis à l'extrême droite d'atteindre une audience et une longévité sans équivalent en France. Le national populisme s'inscrit dans la tradition du courant plébiscitaire bonapartiste mais également dans le boulangisme, mouvement politique qui souhait organiser un Gouvernement fort pour que la France prenne sa revanche sur l'Allemagne.


Bien que toutes ces manifestations de l'extrême droite aient chacune leurs particularités, on peut identifier un certain nombre de traits communs: le culte du chef charismatique, la référence permanente au peuple et à son rassemblement, le nationalisme et enfin, le recours à des thématiques xénophobes, racistes, ou antisémites. \\
Ce découpage que propose Milza est pertinent d'une certaine manière, cependant, le risque de cette démarche est de légitimer des mouvements et des luttes internes et donc de légitimer des mouvements qui sont finalement le fruit de bricolages historiques. 


En dehors de ça, l'extrême droite peut se caractériser à un répertoire d'action particulier: l'activisme, l'agitation voire même la violence. Certaines formes organisationnelles peuvent aussi être caractéristique: les ligues notamment, qui sont des mouvements politiques d'extrême droite apparaissant des les années 20 et 30. \\
Ces ligues sont fondées sur une discipline très forte, quasi militaire et aussi sur une part de clandestinité et d'ombres. 


Le FN a la particularité d'être un parti d'extrême droite qui a réussi à s'institutionnalisé, à durer, et à quitter le statut de parti de protestation.

\section{L'extrême Droite de la fin du XIXe jusqu'aux années 80}

L'histoire de l'extrême droite est cyclique. L'affaire Dreyfus marque une première apparition significative de l'extrême Droite. Dans l'affaire Dreyfus, le débat se cristallise surtout sur les affrontements sociaux et idéologiques entre les intellectuels d'une part et les patriotes d'une part, entre les partisans des droits de l'homme et les partisans de la raison d'État.


L'action française est la ligue la plus importante fondée à la fin du XIXe. C'est une ligue très structurée, très militante. Elle a un écho dans le milieu étudiant surtout. Le mouvement est assez radical, voire violent. La ligue repose sur un journal, l'action française. \\
Au début du XXe siècle, l'AF incarne le principal mouvement "foyer de contestation anti-républicain". Il ne faut pas minimiser l'importance de ce mouvement mais reste assez minoritaire. \\
Charles Maurras est un royaliste. Il considère qu'il faut revenir à une société plus traditionnelle, organisée autour d'ordre plus organisée. Il dit que l'esprit de la Révolution a détruit la famille et les corporations professionnelles. L'AF est le résultat d'une réaction d'une part de la droite catholique à l'anti-cléricalisme de la Gauche. L'AF est très liée aux catholiques, même si les relations n'ont pas toujours été simple. \\ 
Maurras déclarera qu'il voit dans le régime de Vichy une "divine surprise". La violence est le mode d'action privilégié du groupe. En 1908, ils se dotent des Camelots du Roi, qui n'hésitent pas à aller se battre avec les autorités. 


Dans les années 20 et 30, on observe un nouveau développement de l'extrême droite. Une critique populiste, nationaliste et xénophobe des institutions parlementaires se développent de nouveau. C'est le développement des ligues. Ce ne sont pas des partis politiques. Ces ligues cherchent à influencer le pouvoir sans nécessairement le conquérir. \\
Il y a de nombreuses ligues d'extrême droite comme les croix de feu, dirigé par le colonel De La Rocque, regroupant des anciens combattants essentiellement et compte 100 000 adhérents en 1922. Ils se transforment en Parti à la veille de la seconde guerre mondiale: PSF (Parti Social Français), qui comptera jusqu'à 800 000 membres. \\
Ces ligues ne franchissent jamais le pas  d'une action révolutionnaire. Il y a malgré tout une controverse sur les événements du 6 Février 1934, où une manifestation de différentes ligues dégénèrent et se transforme en une émeute à côté du Palais Bourbon. Certains historiens considèrent qu'il n'y a jamais eu de volonté de prendre d'assaut le palais et de renversé le pouvoir. Néanmoins, la Gauche a utilisée cette date pour cimenter son Union du front populaire en 1936. 


Il semble à première vue que en France, dans les années 30, il y ait aussi eu une montée du fascisme. Cependant, la tradition Républicaine est déjà fortement ancrée et le catholicisme ayant rallié la République, cela freine la montée de ces mouvements. Certains groupuscules émergent mais ont une audience assez réduite et marginale, et joueront un rôle pendant l'occupation. \\
La Révolution Nationale est clairement inspirée des ligues fascistes. On trouve aux côtés de Pétain des Morassiens, des membres des croix de feu. Pourtant, l'État Français, proclamé à Vichy, peut difficilement être considéré comme un régime fasciste. Le régime de Vichy ne promeut pas l'idée du parti unique, alors que le fascisme, si. C'est donc bien sûr, un régime de collaboration mais pas nécessairement un régime fasciste, les ligues fascistes prenant d'ailleurs leurs distances avec Vichy en 1942. \\
À la Libération, l'extrême droite, comme la droite, est décimée. Les militants entrent en clandestinité. C'est en 1956, avec la poussée du Poujadisme que réapparais sous une forme nouvelle l'extrême droite. À la base, c'est une contestation fiscale, contre l'État. Poujade étant anti-parlementaire, cela en fait un parti d'extrême droite. Poujade obtient 11\% des voix en 1956 avec une cinquantaine de Députés dont J.M Lepen. Ces résultats sont éphémères car n'aura plus aucune victoire à venir. \\
La Guerre d'Algérie va constituer un autre contexte favorable à la résurgence de l'extrême droite. Le mouvement Algérie Française est hostile à l'indépendance et il est proche de l'extrême Droite. 

\section{Le Front National}

Le FN est créé en 1972, par Ordre Nouveau, qui est d'abord une revue, qui a une ligne éditoriale anti-parlementaire, opposé au capitalisme, et le FN va être le rassemblement de plusieurs groupuscules d'extrême droite. \\
L'objectif est de donner à Ordre Nouveau une vitrine électorale et de réintégré l'extrême Droite dans le jeu politique. C'est Jean-Marie Lepen qui s'impose comme leader de ce mouvement, il a été député des Poujadisme, il a été président d'une corpo de droit à Paris. C'est donc un vieux routier de la politique, rompu aux pratiques militantes et politiques. \\
Il se présente aux présidentielles de 1974 et obtient 0.8\% des voix. Son électorat est essentiellement de pieds-noirs, sur quelques zones géographiques précises: le littoral méditerranéen. En 1981, il ne peut pas se présenter, il n'obtient pas les 500 signatures et appelle à voter Jeanne d'Arc. \\
Dans les années 80, le FN connaît une véritable poussée électorale. C'est le résultat d'une conjonction de plusieurs facteurs: l'alternance de 1981, le mode de scrutin proportionnelle, l'émergence du sujet de l'immigration, et plus largement, les effets de la crise économique. Le premier élément est une élection partielle dans la ville de Dreux, où le couple Stirbois se présente pour le FN et obtient 17\% des suffrages au premier tour et le FN entre au Conseil Municipal en faisant alliance avec la Droite au second tour. \\
Lors des élections européennes de 1984, le FN va connaître une fort montée de son électoral. Le FN acquiert alors 11\% des voix (2 millions d'électeurs). C'est donc véritablement à ce moment que l'électorat s'émancipe de l'électorat de base "Algérie Française", c'est donc à ce moment que le FN va nationaliser son audience. \\
Plusieurs éléments expliquent cette avancée: la radicalisation d'une partie de l'électorat de Droite choquée par la victoire de Mitterrand ; de plus, la liste de Droite est dirigée par Veil, qui symbolise les dérives du libéralisme culturel de VGE ; en 1984, le contexte est particulier puisqu'il y a des manifestations très importantes avec l'école libre qui fait débat et qui fait une Droite très mobilisée et politisée. \\
En 1986, c'est la première fois de la Ve République que l'AN est élue à la proportionnelle, le FN obtient donc une trentaine de députés. \\
En 1986, des lois sur l'immigration vont encore favorisées le FN qui se saisit de cet enjeux et de ce débat. Lors des élections présidentielles de 1988, J.M Lepen réalise un score historique de 14.9\% des voix (4 millions d'électeurs). Ils réalisent ensuite 14\% aux régionales de 1992, 12.5\% aux législatives de 1993. Le FN s'impose donc comme un acteur central de la scène politique, qui impose des débats sur ses propres thèmes. \\
En 1995, il est candidat pour la troisième fois à l'élection présidentielle, il obtient 15\% des voix avec un slogan "Ni Droite ni Gauche". En 1995, il y a des élections municipales, trois villes importantes tombent sous le giron frontiste, Toulon, Orange, Marignane. Cela montre que le FN connaît un ancrage de plus en plus fort de son électorat. Il se maintient aussi au second tour dans de nombreuses villes. \\
Le pouvoir de nuisance devient fort, notamment pour la Droite, car le maintien du FN au second tour d'élections législatives fait perdre 20 sièges à la Droite traditionnelle. 


La structure de l'organisation frontiste se développe elle aussi. Cette dynamique électorale se traduit également par le développement d'une organisation partisane. Le FN recrute de nouvelles forces militantes, se dote de son propre service de sécurité, des espaces de sociabilité émergent dans le parti. Le FN commence à offre un entre soi partisan à des individus marginalisés socialement. \\
L'organisation des espaces de sociabilité permet de faire face aux regards stigmatisant. Le FN s'implante dans des quartiers populaires difficile, désertés par les forces de Gauche. Le FN devient à certains endroits un parti lieu de vie, sur le modèle du Parti Communiste, qui favorise la diffusion d'une véritable contre-culture et qui renforce les contre-culture. Ces lieux sont très limités. \\
Le FN compte 40 000 adhérents à la fin des années 90, ce sont souvent des adhérents actifs. Il commence à attiré des anciens responsables de la Droite Parlementaire, comme Bruno Mégret, un énarque issu du RPR. \\
En Décembre 1998, la scission entre Lepen et Mégret va affaiblir momentanément le FN. Les Mégretistes quittent le FN pour fonder le MNR (Mouvement National Républicain) ; Mégret emporte avec lui près de la moitié des élus et des responsables locaux du FN. Cette scission montre qu'il y a des querelles idéologiques très importantes, comme dans tous les partis politiques. Le divorce entre Lepen et Mégret s'est fait sur des choix tactiques, Mégret voulant s'allié avec la Droite traditionnelle, alors que Lepen veut faire l'inverse. Les Mégretistes constituent la frange la plus "bourgeoise" du FN, la plus moderniste quand Lepen est la mouvance historique, catholique, intégriste. \\
Aux européennes de 1999, quelques mois après la scission, le FN obtient 5\% des voix et le MNR, 3,3\% des voix. J.M Lepen obtient un électorat populaire alors que Mégret obtient un électorat bourgeois, votant occasionnellement pour le FN. En 2002, les autres responsables politiques se rendront compte que l'extrême droite a été tuée trop vite. 


L'électorat FN a beaucoup évolué au fil du temps. Dans les années 90, le FN de vient un parti attrape tout, qui agrège des groupes sociaux aux intérêts très différents alors qu'au départ, la poussée du FN était lié à la radicalisation de la droite classique, de petits commerçants. Cette base va s'élargir en épargnant aucune classe sociale. \\
On observe par ailleurs à partir des années 90 une prolétarisation du vote FN. Aux présidentielles de 1995, un électeur sur deux du FN est un ouvrier, et 30\% des ouvriers ont votés pour le FN. Certains spécialistes, vont alors évoquer une thèse, celle du gaucho lepenisme, selon laquelle des groupes sociaux, traditionnellement ancrée à gauche, votent pour l'extrême Droite, mais tend à négliger qu'une partie des ouvriers ont toujours étés de Droite et que ce sont précisément ceux-là qui votent FN. \\
Il est notable qu'un parti d'extrême droite, pour la première fois dans l'histoire, réussit à s'inscrire durablement dans les couches populaires de la population. \\
Ensuite, on note que les hommes votent plus FN que les femmes, les jeunes votent assez massivement pour le FN, surtout ceux qui sont en situation d'échec scolaire soit de déclassement sociale. Le diplôme constitue un rempart plus important contre le FN. Ce sont aujourd'hui des variables sociales qui définissent le mieux les électeurs du FN: chômage, inégalité de richesse dans un territoire de vie, etc. \\
À partir de 2002, le FN va confirmé sa percée dans le monde rural, généralement très peu affecté par les problèmes d'insécurité et d'immigration. 


Le FN, à ses débuts, reprends des thématiques qui ont étés laissés à déserrance par la Droite traditionnelle: la question de la Nation, l'ordre, la sécurité, l'opposition à l'UE. Le discours du FN est polysémique, défense de la nation, lutte contre l'immigration, contre la corruption, contre l'insécurité. \\
Sur le plan idéologique, le FN est une sorte de bric à brac qui mélange à la fois des thèmes de gauche comme la protection de l'État, l'augmentation des salaires, et des thèmes de droite comme la diminution des impôts. Avec ces nombreux thèmes, le FN séduit des électorats très différents. \\
Les dérapages de Lepen étaient utilisés comme stratégique, il usait d'une rhétorique transgressive lui permettant de conserver une frange extrémiste tout en ayant une audience plus large. 












\end{document}
