\documentclass[10pt, a4paper, openany]{book}

\usepackage[utf8x]{inputenc}
\usepackage[T1]{fontenc}
\usepackage[francais]{babel}
\usepackage{bookman}
\usepackage{fullpage}
\setlength{\parskip}{5px}
\date{}
\title{Cours d'histoire politique contemporaine (UFR Amiens)}
\pagestyle{plain}



\begin{document}
\maketitle
\tableofcontents

\chapter{Introduction}

Le mot politique est un terme polysémique. Cela se présente comme un ensemble de forces institutionnalisés qui interagissent entre elles dans ce que l'on pourrait appeler le champ politique. Cette fonction de régulation sociale du politique se traduit par la mise en oeuvre de politiques publiques qui sont des dispositifs d'actions publiques qui visent à produire un certain nombre d'effets sociaux. \\
Dans le cadre de ce cours, la définition qui nous intéresse est la politique que l'on entend comme être la scène politique sur laquelle s'affrontent sous les yeux du public et des citoyens une série d'acteurs qui cherchent à conquérir et exercer le pouvoir. \\
Bourdieu: "C'est le lieu où s'engendre dans la concurrence entre les agents qui s'y trouvent engagés des produits politiques entre lesquelles les citoyens ordinaires réduit au statut de consommateur doivent choisir". Telle que défini par Bourdieu, la politique serai un marché, un marché économique. On remarque que pour lui, les citoyens ne jouent qu'un rôle mineur et sont réduits à être des consommateurs. \\
Les marchés politiques sont des lieux où s'échangent des produits politiques (comme un programme) contre des soutiens matériels ou immatériels et évidemment, des votes. Cette métaphore économique est donc tout à fait pertinente: il y a donc un marché politique. \\
Comme dans un marché économique, il y a donc un jeu de l'offre et de la demande. La demande étant l'attente des électeurs et leurs préoccupations. Si le marché politique est véritablement un marché économique au sens strict du terme, les citoyens ont un rôle restreint. Dans la mesure où la rationalité du citoyen politique est bien inférieur à la rationalité du citoyen économique. \\
On va, dans ce cours, s'intéresser à un siècle de vie politique. On va s'appuyer sur de grands événements de la vie politique française, sur les acteurs, sur les évolutions des tendances, des sensibilités politiques, sur les luttes électorales, sur les résultats, sur les enjeux politiques qui se constituent, sur les clivages, sur les règles du jeu.


La concurrence et l'interdépendance politique a lieu sous le joug d'un certain nombre de règles politiques: le système partisan, les coalitions, le système de scrutin, la nature des "trophées" qui sont en jeu (degré de parlementarisme, hyper présidentialisation), les normes constitutionnelles etc. \\
Ces règles ne sont pas univoques. La politique n'est pas réglée une bonne fois pour toutes par les constitutions. Ce qui compte, c'est l'usage qu'en font les acteurs politiques. Ces règles sont susceptibles de changer, d'évoluer. Il ne faut donc pas donner à ces règles plus d'importance qu'elles n'en ont.


Les traditions politiques sont des courants, des forces, des sensibilités qui ont une certaine constance dans le temps. On entendra par tradition plus précisément, la permanence à travers le temps d'un système relativement cohérent de représentations et d'images, de souvenirs et de comportements, de fidélité et de refus. \\
Ces traditions s'incarnent dans des organisations, principalement dans des partis politiques. Ces traditions n'existent pas en soi. Il faut se déprendre de l'illusion qui présente ces traditions comme des essences intemporelles. \\
Les traditions doivent être appréhendés comme des constructions historico-sociales. Ces constructions résultent d'un travail historique. \\
Par ailleurs, les traditions politiques ont des fonctions de légitimation. S'approprier le monopole de l'héritage permet de stigmatiser un opposant politique. 


\chapter{L'invention de la politique moderne}

\section{L'instauration de la République}

Le 4 Septembre 1870, la IIIe République est proclamée par Gambetta suite à la capture de Napoléon III par les Prussiens. La IIIe disparaîtra en 1940 quand les pleins pouvoirs seront donnés à Pétain. \\
La IIIe est une période fondamentale pour comprendre la vie politique française. C'est à partir des années 1870 que se sont posés les fondements de la démocratie représentative telle qu'on la connaît encore aujourd'hui. C'est sous cette République que le régime Républicain s'impose. La pratique du vote se généralise à partir des années 1870, il existait depuis 1848 mais était encadré pendant le second empire. La lutte électorale s'intensifie, des marchés électoraux se créent. La politique va devenir une sphère de plus en plus autonome et va se professionnalisé. \\
Le sentiment d'appartenance nationale progresse.


Comment expliquer ces transformations profondes pendant la IIIe République ? \\
Certaines de ces transformations peuvent être imputés notamment à l'introduction du suffrage universel masculin en 1848. \\
Les régimes d'avant la IIIe sont essentiellement des monarchies. C'est donc le suffrage censitaire qui domine dans ces régimes. Le cens est variable selon les élections: 1 Franc pour les élections municipales, 200 ou 400 Francs pour des élections législatives. Comme tout le monde ne peut pas payer, les marchés électoraux sont restreints, peu concurrentiel, dominés par les notables. \\
Restreints car peu d'électeurs. Les trois quarts des collèges électoraux étaient composés de moins de 600 inscrits. Peu concurrentiel car peu de candidats. Plus de 80\% des députés sont élus avec moins de 400 voix. Enfin, ils sont dominés par les notables qui ont les ressources personnelles qu'ils peuvent investir dans le champ politique. Les biens qu'investissent les notables sont essentiellement non spécifiquement politique (clientélisme).


L'introduction du suffrage universel a contribué à accroître considérablement le nombre d'électeurs. On passe de 250 000 électeurs à plus de 10 millions. Il est à noter que tous n'intègre pas le vote du jour au lendemain. Le suffrage universel ne va réellement que s'épanouir sous la IIIe République. Il en va de même pour les candidats qui vont renouveler leurs stratégies, contraints par leur nouvel électorat. \\
Dans un premier temps, ils cherchent à continuer la relation clientéliste, mais il est impossible de rémunérer l'ensemble du corps électoral. \\
Progressivement, le vote va s'individualiser et devenir l'expression d'une opinion personnelle. Ce processus s'initie sous la IIIe République et s'intensifie via la mise en place de politiques volontariste, pour apprendre aux citoyens à voter. Il va s'individualiser car de nouveaux acteurs vont émerger et concurrencer les notables et contester leurs pouvoirs en proposant des visions politiques et en redéfinissant les relations électorales.

\section{La République dans les moeurs}

La République comme régime s'est progressivement affirmé dans deux directions différentes.

\subsection{La consécration dans la classe politique}

Au cours du XIXe siècle, le régime est un débat central. Près d'un siècle après la Révolution française, les monarchistes sont toujours nombreux et mobilisés. \\
Au départ, la République s'impose comme un régime par défaut puis va commencer à devenir dans les esprits, le régime qui divise le moins. On parlait à cette époque de "République d'attente".


En 1870 éclate la guerre avec la Prusse, la défaite se fait deux mois plus tard avec la capture de Napoléon III. À Paris, se constitue un gouvernement de défense nationale qui proclame la République. \\
Des élections législatives sont donc organisés à la va vite. L'enjeu central est bien évidemment la guerre avec la Prusse. Les Royalistes sont favorables à la paix tandis que les Républicains souhaitent poursuivre la guerre. Les partisans de la République obtiendront 200 députés contre 400 pour les royalistes. \\
C'est ainsi que Adolf Thiers devient le chef du pouvoir exécutif dans cette République. De manière assez paradoxale, l'épisode de la commune de Paris va renforcer la République. Certains considèrent que la commune de Paris est une première manifestation du communisme. À l'époque, la problématique des républicains et des monarchistes est la stabilité du régime. En réprimant un mouvement insurrectionnel, la République a fait sa preuve d'un régime stable. En 1872, Thiers se rallie à la République.


C'est le régime qui divise le moins car les monarchistes sont, à l'époque, très divisés. Il y a les Orléanistes, héritiers de la monarchie de Juillet et la monarchie constitutionnelle et parlementaire, modéré et d'autre part, les légitimistes qui revendiquent les héritages des ultras de la première restauration. \\
C'est pour cette raison que Thiers sera renversé en 1873, et Mac Mahon sera élu président. Les divisions persistent chez les monarchistes et rendent impossible une éventuelle restauration. \\
Ce n'est qu'en 1875 que la République est juridiquement renforcée. Le mot République apparaît dans trois lois constitutionnelles. C'est presque par hasard pendant les débats parlementaires qu'un amendement introduit le mot République dans la loi constitutionnelle. \\
Les républicains vont progresser aux élections, leur audience est de plus en plus important à travers le pays et ils finissent par remporter une majorité aux élections de 1876. McMahon dissoudra la chambre et de nouvelles élections seront organisées en 1877, qui donnera, de nouveau, une majorité au camp des Républicains. C'est lors de cette élection que le leader républicain Gambetta, aurait lancé une formule restée célèbre "Quand la France aura fait entendre sa voix il faudra se soumettre ou se démettre". \\
En Janvier  1879, McMahon démissionnera suite à la victoire des Républicains aux élections sénatoriales. 9 ans après la proclamation de la République, c'est la première fois qu'un Républicain est chef de l'État: Jules Grévy. \\
La République est donc désormais acceptée comme seul régime légitime. 

\subsection{La "républicanisation" de la société}

L'affirmation de la République passe également par la mise en oeuvre de politiques publiques volontaristes que les républicains vont mettre en place. On peut parler d'un processus de "républicanisation" de la société française. Processus car l'idée républicaine n'a pu s'imposer que progressivement dans la société française. \\
C'est en grande partie grâce aux politiques volontaristes, grâce aux symboles républicains qui se diffusent (La Marseillaise comme hymne national en 1879 et le 14 Juillet comme fête national). L'éducation civique apparaît avec l'école et permet de développer un attachement à la République aux plus jeunes. \\
Le projet des républicain qui repose sur la participation des citoyens à la vie politique ne prend sens que si le citoyen est en mesure de participer à la vie politique. L'école est donc censé formé des citoyens et consolide ainsi la République. Les républicains sont essentiellement anti-cléricaux et le développement de l'école laïque permet de concurrencer directement l'Église. 

\section{Le processus de politisation}

"On désigne par politisation le processus par lequel un intérêt pour la politique se développe dans la population". \\
Dans la France du XIXe, la population est essentiellement rural. Au début du XXe siècle, plus de 50\% de la population vit en milieu rural et près de 60\% des actifs sont des agriculteurs. Il y a donc un clivage important entre les villes et les campagnes. Ville où l'intérêt des citoyens pour la politique est un peu plus précoce et campagne où le développement de l'intérêt est plus tardif. \\
Dans le milieu rural, le niveau d'éducation est plus faible, les lieux du pouvoir sont plus visible pour les urbains. En ruralité, les entités locales sont très fortes, cependant, il faut manier cette thèse avec précaution, des historiens considèrent que la politique a toujours été très présente dans les villages mais peu tournés vers les arènes nationales. 


Pour Eugen Weber, un historien américain, il montre que la France va avoir tendance à s'uniformiser, les entités locales vont peu à peu s'effacer et le sentiment d'appartenance nationale, va, en contre partie, progresser. Pour lui, la société française va se nationaliser. Cette nationalisation est favorisée par quatre facteurs essentiels: le développement des communications (transport, essentiellement) ; progrès lié à la scolarisation ; le suffrage universel ; la première guerre mondiale. \\
Cette analyse d'Eugen Weber est devenu tout à fait classique mais n'est pas la seule.


Pour Maurice Aghulon, un historien français, propose pour sa part, une autre lecture. Il considère que les républicains vont s'appuyer sur les identités locales. Ce serai donc un processus moins direct. Il insiste sur le poids et l'importance du suffrage universel qui est le principal facteur d'éducation politique des citoyens. Cela signifie que l'intérêt pour la politique découle de l'exercice du droit de vote. Pour lui, les lieux de sociabilité local constitue des lieux de socialisation à la vie politique. Ce serai donc grâce à ces structures que la politique se propage jusque dans les plus petits village de France. 


\section{L'apprentissage du vote}

Le vote est au centre de la vie politique française. Cependant, cela n'a pas toujours été de soi. Le vote a donc subit un processus de naturalisation. "Si l'électeur aujourd'hui fait l'élection, c'est parce que préalablement, l'élection a fait l'électeur" Alain Garrigou. \\
Le vote va devenir peu à peu un comportement codifié, présenté comme l'expression d'une opinion personnelle et non plus comme la marque d'une appartenance à un groupe. La salle de vote va peu à peu perdre son caractère de lieu de délibération et va être érigé en un espace neutre et démocratique. \\
Dans ce lieu, l'électeur doit accomplir dans le calme son devoir civique. L'état d'ivresse est interdit dans un lieu de vote en 1874. Les urnes, les bulletins, l'isoloir, l'interdiction de débattre, d'exercer la moindre pression sur un électeur ne sont pas anecdotiques, et représentent l'idée que les républicains se font du vote. \\
Quelques dates clés de l'apprentissage du vote:
\begin{itemize}
\item 1848: listes permanentes et alphabétiques de l'ensemble des électeurs ;
\item 1852: organisation de la procédure électoral ;
\item 1860: standardisation des urnes ;
\item 1884: carte des électeurs obligatoires ;
\item 1913: isoloir ;
\item 1923: bulletins ;
\item 1988: paraphe sur les listes d'émargement obligatoire.
\end{itemize}
Les républicains tentent de promouvoir une vision individualiste de l'électeur. En votant, les citoyens participent à une communauté nationales. Dis autrement, les citoyens deviennent français en votant et inversement.

\section{La professionnalisation de la politique}

Parallèlement, l'activité politique va devenir un champ autonome. On observe un processus de professionnalisation. Une figure idéal-typique du professionnel de la politique apparaît à la toute fin du XIXe siècle, qui conteste le pouvoir des notables, qui représente des groupes sociaux jusque là ignorés dans la sphère politique. Ces nouveaux professionnels appartiennent souvent à des professions libérales, et à la bourgeoisie de capacité bien souvent Mais aussi des ouvriers, et des représentants du monde ouvrier. \\
Ces nouveaux professionnels ont peu de ressource personnelles donc propose plutôt de la propagande électorale, vont se constituer leurs propres ressources: collectives et organisationnels (partis, mobilisations collectives...). D'un certain point de vue, ils opposent la force du nom par rapport à la force du nom. Ils vont collectivement, développer de nouvelles techniques de campagne. Ces nouveaux entrepreneurs tentent de redéfinir la relation électorale.\\
L'élection va devenir ce qu'elle est, un échange de programmes électoraux contre des suffrages. \\
À noter que le processus va être double, les notables vont se professionnaliser et les professionnels vont se notabilisés. 

\section{La naissance des partis politiques}

À la fin du XIXe naissent les premiers partis politiques au sens moderne du terme, au sens où on l'entend encore aujourd'hui: "Une organisation durable, qui participe à la lutte pour la conquête de postes électifs en mobilisant des soutiens populaires grâce à une organisation ramifiée et territorialement organisée". \\
Pendant longtemps, le mot parti était au sens prendre parti pour tel ou tel idéologie. C'est donc au XIXe que les partis deviennent des acteurs centraux. Ils acquièrent progressivement le monopole de l'activité politique. Certains pensent que les partis politiques sont les enfants du suffrage universel, car celui-ci rend impératif l'organisation de la conquête des suffrages. \\
Jusqu'en 1901, les partis ne sont que tolérés par la réglementation, ce n'est qu'en 1901 et la loi sur les associations qu'ils vont se doter d'un cadre légal. C'est une reconnaissance assez tardive des partis politiques. \\
Les formes que peuvent prendre les partis politiques sont très variables. Elles sont variables historiquement, dans la position sur l'échiquier politique etc.


Les premiers partis politiques sont issus des mouvements ouvriers, et se constituent dans les années 1870: ils s'appuient donc sur le nombre et la mobilisation de leurs adhérents, sur les réseaux de sociabilité ouvrières. \\
Jules Guesde et son parti, ont des premières victoires électorales au début des années 1890. \\
Cependant, les partis modernes n'apparaissent que réellement au début du XXe siècle: le parti radical en 1901. Dans les années suivantes, les principaux courants politiques vont s'organiser en partis politiques. \\
La vie politique tend à se partisanisé: les signes et emblèmes partisans prennent peu à peu du sens pour les électeurs ; les individus commencent à s'identifier aux partis politiques, aux programmes. Les partis exercent des fonctions de plus en plus centrales: désignation des candidats. Ils structurent l'offre politique. Ils encadrent la mobilisation électorale. \\
En bref, ils structurent la vie politique dans son ensemble. 

\chapter{Le clivage gauche/droite}

C'est un clivage qui structure la vie politique française et cela depuis la révolution française, pour au moins deux raisons. D'abord parce que la production savante entretient ce clivage. La deuxième raison est parce que les hommes politiques eux mêmes se positionnent dans le clivage gauche/droite. \\
Ce clivage existe dans de nombreuses démocraties occidentales. Aux USA, c'est par exemple les Républicains contre les Démocrates, et au Royaume-Uni, les Travaillistes et les Conservateurs. \\
Ce clivage a certes structuré la vie politique française, mais n'a jamais cessé d'être contesté. Au cours de son histoire, le clivage gauche/droite n'a pas toujours renvoyé aux mêmes enjeux. C'est une sorte de paradoxe, c'est justement parce que le contenu de ce clivage varie qu'il est toujours efficace. Le clivage ne renvoie pas à des oppositions immuable ni à des oppositions identiques. Comme pour les traditions politiques, il faut veiller à ne pas essentialiser ce clivage: on ne peut pas considérer qu'il existe par essence des problématiques de gauche ou des problématiques de droite. Ce sont des produits de l'histoire, qui sont façonnés par les acteurs et les observateurs de la vie politique. \\
Le clivage se compose et se recompose au fil du temps. À noter que le clivage est toujours multi-dimensionnel. 

\section{La naissance du clivage gauche/droite}

On peut considérer qu'il naît avec la révolution Française. La première étape serait en 1789, où les député du tiers État, décident de se réunir à Versailles contre la volonté du Roi. Ils sont rejoints par une grande partie des membres du clergé et de la noblesse. Spontanément, ces députés vont laisser les chaises les plus honorifiques aux membres du clergé et de la noblesse. \\
Le Vendredi 28 Août 1789, l'Assemblée nationale constituante organise des débats quotidien et lorsque le débat en arrive à la question du veto royal, les députés y étant favorables se placent à la droite de l'hémicycle et ceux qui s'y opposent, se placent à gauche. \\
En réalité, le clivage va se former progressivement et devenir de plus en plus prégnant au cours du XIXe et du XXe siècle. Le premier débat central qui va façonner le clivage gauche/droite, c'est la question du régime politique. 

\section{La question du régime}

Le XIXe siècle est traversé par cette question du régime politique. De la révolution française jusqu'aux années 1880, cette question rythme la vie politique française. Le débat se cristallise sur la question des institutions. La gauche revendique l'héritage de la Révolution Française, elle s'inscrit dans sa filiation et elle défend la République. Les hommes et les femmes de gauche accordent une grande importance à l'héritage de la révolution et à la philosophie des lumières. Ils veulent en finir avec l'absolutisme monarchique et la République serait le régime le plus adapté. \\
La droite, elle, est attachée au Roi et appelle au retour de la monarchie. Elle défend l'ordre moral, voire même l'ordre social. Elle s'oppose à l'individualisme de la philosophie des lumières. Cette droite est divisée, d'une part les partisans de la monarchie absolu, les légitimistes, et d'autre part, les partisans de la monarchie constitutionnelle, les orléanistes. Plus tard, la droite connaît une sorte de renouveau et on pourrait considérer le Bonapartisme comme un nouveau courant de droite, très soucieux de l'ordre moral et social et qui est en même temps favorable à une forte autorité étatique mais qui accepte dans le même temps le principe du suffrage universel sous une forme largement plébiscitaire. \\
Après la restauration, la monarchie absolue n'a plus vraiment d'actualité. Le débat s'organise essentiellement autour des partisans de la monarchie constitutionnelle et des partisans de la République. \\
La troisième République s'impose face à une droite divisée. La République s'impose donc et s'installe. La droite va se rallier à cette République, par défaut. On peut considérer que se termine un épisode politico-historique particulier, ouvert par la révolution française. \\
Il y a deux marqueurs du ralliement de la droite à la République: en Novembre 1890, lorsque les catholiques admettent la République. Confirmé deux ans plus tard, par le Pape, qui invite les catholiques à accepter la Constitution, pour changer de l'intérieur les institutions. Un deuxième ralliement se produit après la première guerre mondiale avec la victoire du bloc national (coalition de droite), qui vient consacrer l'intégration de la droite dans la République. En 1926, le Pape condamne l'Action Française. 

\section{La question de la religion}

En pratique, la question religieuse et la question du régime sont très fortement liées. C'est évident puisque la monarchie est de droit divin et puise sa légitimité dans la religions. Cette question religieuse ne disparaît pas du jour au lendemain. Elle reste très longtemps un point d'affrontement entre la droite et la gauche. Même si ce clivage a aujourd'hui largement disparu, on en trouve régulièrement des apparitions lors des manifestations pour les écoles libres dans les années 80, dans les manifs de la manif pour tous en 2012. \\
Au XIXe et au début du XXe, l'opposition entre les cléricaux et les anti cléricaux, est très forte et très prégnante. Chaque camp est porteur de visions très différentes du monde social. La question centrale est celle de l'influence sociale de l'Église: quelle place doit occuper l'Église dans la société française ? Jusqu'à quel point peut elle avoir une emprise sur les comportement collectifs ou privés ? \\
La gauche, plutôt rationaliste, se revendiquant de la philosophie des lumières, est très critique à l'égard du rôle social du clergé. Elle va chercher à soustraire la société de l'emprise conservatrice de la Religion, elle défend la laïcité: l'idée selon laquelle le champ politique doit être dissocié, indépendant, de toute religion. \\
Face à cela, la droite défend la tradition chrétienne de la France. Elle souhaite préserver les liens privilégiés entre le pouvoir et le clergé. La droite considère que la Religion est le ciment de la société et que ce ciment permet de maintenir l'ordre social et moral. \\
Du XIXe siècle à nos jours, cette question religieuse disparaît progressivement, la laïcité l'a emporté. Le politique et le religieux sont séparés dans plusieurs domaines de la société: c'est le cas dans l'éducation, où l'école est gratuite, obligatoire et laïque (1882). C'est la cas dans d'autres activités de l'État. Enfin, la loi de 1905 proclame la séparation de l'Église et de l'État. \\
La question va cesser d'être un enjeu politique saillant dans l'entre deux guerres.


La question scolaire était une forme de prolongement de cette question religieuse. Il coexistait en France deux écoles, l'école publique mais aussi l'enseignement privé. Face à cette question, la gauche a conservée son caractère laïque et y est globalement opposé. D'une part, la gauche a une prise de position maximaliste qui consiste à combattre l'enseignement privé confessionnel et une position minimaliste qui consisterai à conserver l'école privé mais lui supprimer tout financement public. \\
Quant à la Droite, elle défend la liberté d'enseignement. Elle ne s'oppose plus comme ce fut le cas à l'enseignement laïque mais défend la possibilité pour ceux qui le souhaiterait de donner une éducation confessionnel à leurs enfants. La droite préfère parler d'école libre. \\
À la libération, les subventions aux écoles privés sont supprimés. C'est en 1959 que la loi Debré rétablit les subventions aux écoles privés mais les lies à l'État par l'intermédiaire d'une politique de contrat. Il s'agit en quelques sortes de soumettre des établissements privés à des obligations de service publics en contrepartie de la participation financière de l'État. De plus, l'établissement doit accepté un contrôle rigoureux de son fonctionnement, de ses programmes et de ses enseignants. \\
Cette dimension du clivage gauche/droite se manifeste dans les années 80. Mitterrand cherche à créer un grand service public de l'enseignement unifié et laïque avec son ministre Savary. Les réformes proposées mettent le feux aux poudres, une manifestation a lieu en 1984, la droite soutient cette manifestation. 1 millions et demi de personnes défilent pour défendre l'école libre. Elle débouche sur une crise politique très importante: le Gouvernement retire son projet, le PM est conduit à la démission etc. Cependant, il est à noter que le rapport à l'école n'est plus confessionnalisé, on met notre enfant à l'école privé parce qu'on est mécontent du service public. \\
En 1992, Jack Lang signe un protocole avec le secrétaire général de l'enseignement catholique, mais pourtant, l'année suivante, éclate une nouvelle mobilisation et c'est François Bayrou qui est alors ministre du gouvernement Balladur, tente de remettre en cause une loi de 1850, la loi "Falloux". La Droite tente de permettre aux collectivités locales de financer l'enseignement privé. Cette fois-ci, c'est le camp laïque qui se mobilise et force la droite à battre en retrait. 

\section{La question sociale et économique}

À partir du milieu du XIXe siècle, la question sociale tend à remplacer progressivement la question du régime et la question religieuse. C'est à partir du XIXe siècle que la dimension socio-économique commence à apparaître. L'économie est révolutionné grâce à la révolution industrielle. Il s'agit donc d'un nouveau conflit social. \\
L'industrialisation produit un nouveau groupe social que Marx appelle le prolétariat. Très vite, le mouvement ouvrier cherche à défendre les intérêts de cette classe sociale. C'est une période où émerge les nouveaux entrepreneurs politiques. C'est donc une période où émergent sur la scène politique les classes populaires. On parle à cette époque de masse ouvrière ou de classe laborieuse. Ce sont des préjugés qui s'appuient sur la prétendue dangerosité des foules. \\
On peut se dire que le clivage gauche/droite devient la traduction dans le paysage politique des conflits de classe dans la société. Selon Marx, il y a un clivage possédant/non possédant. Ce clivage se recompose à de nombreuses reprises pour se traduire dans un clivage libéraux contre dirigiste, etc. \\
Le mouvement ouvrier se structure en France dans les années 1880. Les formes d'expression du mouvement ouvrier sont diverses: syndicats, mouvements, etc. Les républicains qui sont alors la Gauche, se diversifient et sont plus ou moins modérés. Certains sont réformistes: souhaitent contrôler, réguler, réglementer l'économie capitaliste ; ils cherchent à améliorer la vie des ouvriers grâce à des lois sociales. \\
D'autres sont révolutionnaires comme les socialistes, à la fin du XIXe. Ils sont contre l'idée d'accompagner le développement de l'économie capitaliste. Ils cherchent à transformer radicalement l'ordre social pour renverser l'exploitation des ouvriers par les patrons. \\
Ce n'est que très progressivement que le rapport de la gauche à l'économie de marché va évolué. On le voit notamment avec le rapport que la Gauche entretient à l'État. \\
Si l'on observe le rapport de la Gauche à l'État, l'on observe des évolutions sensibles. C'est le cas notamment dans les années 30 lorsque pour les socialistes, les nationalisations commencent à devenir possible et voient là une opportunité de réduire les inégalités. Ils considèrent que l'État et son implication dans l'économie peut représenter une certaine vertu. La Gauche commence donc à percevoir un certain intérêt pour l'État. C'est la SFIO qui deviendra plus tard le PS, qui incarne cette gauche étatique. La SFIC y voit une gauche d'accompagnement du libéralisme. 


La Droite, elle, est attachée à la propriété. Elle est attachée aux mécanismes du marché. Elle est favorable aux jeux du marchés. Elle est donc libéral sur le plan économique. Elle défend l'idée que les marchés tendent à s'auto réguler pour aller vers une situation optimale. \\
C'est sur cette question de l'interventionnisme que se structure le clivage gauche/droite. La Droite est plutôt défavorable à l'intervention de l'État. C'est dans les années 1970 que le clivage gauche/droite est le plus fortement structure par la question socio-économique. C'est en 1972 que le PCF et le PS signe le programme commun dans lequel il est fait mention de la rupture avec le capitalisme. La victoire de la Gauche en 1981 amène à la mise en oeuvre d'une politique de relance économique: augmentation des salaires, du SMIG, des retraites, des allocations familiales, vague d'embauche dans la fonction publique, généralisation de la retraite à 60 ans, réduction du temps de travail, mais aussi de nombreuses nationalisations. Pourtant, le tournant de la rigueur, en 1982, marque une rupture avec ces orientations. Cela marque le début d'un brouillage sur le clivage sur les questions sociales et économiques. \\
Lors de la première cohabitation, la droite prend le pouvoir et accentue encore cette politique néo-libérale. Lorsque Mitterrand revient au pouvoir, le thème des nationalisations est totalement abandonnés. Les politiques de droite et de gauche tendent à s'indifférenciée dans le clivage. Depuis 1997, il n'y a plus vraiment eu de différences sensibles entre le gauche et la droite sur le plan socio-économique.


\section{Les sujets passés de gauche à droite ou inversement} 

La nation est marqueur de la droite ou de l'extrême droite. Pourtant, la Nation est un thème de gauche qui apparaît dès la Révolution française. Le thème du nationalisme passe à droite à la fin du XIXe siècle, et c'est un nationalisme plutôt conservateur, identitaire. Cela s'explique par le fait que le mouvement ouvrier commençait à se structurer dans la deuxième partie du XIXe siècle dans une perspective internationaliste. Cela laisse libre le thème de la nation et ce vide conduit à la création de l'extrême droite. \\
Plus tard, au XXe siècle, les lignes autour de cette question sont toujours aussi floues car le PCF a été un parti qui a prôné une certaine forme de nationalisme et de patriotisme en attisant notamment l'anti-germanisme et en exprimant une forme de rejet des USA ainsi que de l'organisation européenne. \\
Une certaine forme de nationalisme, assez jacobine, persiste à gauche comme c'est le cas chez Jean-Pierre Chevènement. Le Gaullisme a aussi une forme de nationalisme en ayant une certaine idée de la France. 


La décentralisation est aussi un sujet qui a changé de clivage. À la révolution, la gauche jacobine est centralisatrice et défend l'unité de la République et son indivisibilité alors que la droite monarchiste est plutôt attachée aux identités locales, aux communautés naturelles. Cette droite monarchiste refuse les règles universelles des jacobins. \\
Au XXe siècle, la question de la décentralisation divise la gauche. Il y a une gauche centralisatrice, étatiste et une autre qui défend la décentralisation. Cette question divise aussi la droite entre les libéraux et les gaullistes, qui sont attachés, respectivement, à la décentralisation et au pouvoir de l'État central. \\
Ceci étant, c'est la Gauche qui met en oeuvre les grandes lois de la décentralisation, sous le premier septennat de François Mitterrand. On peut dire que la question de la décentralisation n'est plus un enjeu constitutif d'un clivage, il y a un certains consensus sur la question. 


Le libéralisme culturel est un enjeu beaucoup plus récent, et désigne un ensemble d'attitude et d'opinions qui sont liées aux moeurs, à la vie privée, à la sexualité ou encore au pluralisme culturel. On peut considérer que ce clivage apparaît lorsque que VGE cherche à asseoir son image de président jeune. Il donne le droit de vote à 18 ans, il légalise la contraception, l'avortement. Même si c'est un président de droite, ce sont toujours les électeurs de gauche qui sont toujours aujourd'hui attaché à ces acquis ainsi qu'à la suppression de la peine de mort, et qui sont aussi les moins opposés à l'immigration. \\
Aujourd'hui, le clivage se fait sur un pôle humaniste et universaliste plutôt marqué à gauche et un autre pôle nationaliste et traditionaliste, marqué à droite. 


On observe la disparition progressive de la plupart des enjeux à partir desquels le clivage gauche/droite s'est structuré. Le clivage a perdu aujourd'hui de sa force, mais pas nécessairement sa pertinence. Le clivage gauche/droite se brouille à partir des années 80 sous l'effet de la politique de Mitterrand mais aussi parce que le jeu de l'alternance politique devient la règle: très peu de majorité législative ont été reconduites depuis 1981. \\
La plupart des référents idéologiques de la gauche apparaissent comme en crise alors qu'en face, la question du libéralisme économique est tellement présente qu'elle n'apparaît même plus comme une idéologie en tant que tel. \\
Aujourd'hui, le clivage gauche/droite n'a pas disparu, ne serait-ce que pour des questions sociologiques. Aujourd'hui, le clivage gauche droite a une réalité sociologique qu'il est difficile de nier. 

\chapter{Le radicalisme politique}

\section{La naissance du parti}

Aujourd'hui, les radicaux sont une force politique marginale. Cela n'a pas toujours été le cas. Lors de la IIIe et la IVe République, les radicaux constituaient un acteur majeur de la vie politique, qui a donc connu finalement, un déclin. \\
Les radicaux défendent des positions républicaines et au fur et à mesure que s'affirme la République, le radicalisme, lui, décline et perd de son influence à mesure que la République en gagne. Le poids politique des radicaux reste important longtemps après l'avènement de la République, notamment grâce à leur maîtrise du jeu institutionnel et leur position centrale dans le spectre politique.


Les idées radicales ont longtemps défendu les idées de progrès social et d'égalité des chances. Ils sont très attachés à l'instruction publique, qui doit permettre le progrès social et l'égalité des chances. Ils défendent la laïcité, qui est l'un de leur marqueur politique. On peut enfin définir le radicalisme comme une doctrine politique qui prône une rupture institutionnelle la plus complète. \\
Le radicalisme est avant tout une sensibilité politique avant d'être un parti, qui sera malgré tout le plus grand parti, fondé en 1901.


La radicalisme apparaît bien avant la naissance du parti radical et désigne, au XIXe, un courant de pensée qui défend les institutions républicaines et l'héritage de la révolution française. Le mot apparaît sous la monarchie de Juillet et sert d'euphémisme pour désigner les républicains. Le radicalisme s'inscrit très nettement à gauche de l'échiquier politique. Ce courant de pensée va se structure progressivement au début de la IIIe République et Gambetta est l'une des figures du radicalisme et en décline les principaux thèmes dans un discours tenu à Belleville en 1869. Parmi ces thèmes, il y a l'attachement à la République, la défense du suffrage universel, la liberté de la presse et d'association et enfin l'anti-cléricalisme et la défense de la laïcité. 


Les radicaux, à la différence des socialistes, mettent les questions sur les institutions. Ils considèrent que la démocratisation du régime va permettre l'émancipation sociale des citoyens. \\
Plus la République va gagner en audience et stabilité, plus on va observer une redéfinition de ce qu'est le radicalisme. À partir de 1879, certains Républicains prônent une politique de prudence et de réalisme, c'est le cas notamment de Gambetta. Cette prudence ne plaît pas à tout le monde, notamment Georges Clémenceau, qui va rompre avec les radicaux et va incarner l'intransigeance initiale des radicaux et défend l'idée d'une laïcité intégrale et défend aussi comme l'instauration d'un impôt sur le revenu progressif. \\
Les radicaux développent une philosophie centrée sur l'égalité des droits et sur une solidarité redistributive.


L'affaire Dreyfus, à la fin du XIXe joue un rôle très important dans la structuration des radicaux. Ils sont le centre de gravité de l'ensemble des défenseurs de la cause de Dreyfus. Ils participent au pouvoir en 1899 avec le bloc des gauches qui rassembles les radicaux, les modérés et certains socialistes. C'est en Juin 1901 que se tient le congrès fondateur du parti radical, il compte alors 280 parlementaires, 476 comités locaux et il va devenir dès lors, un parti charnière pour toutes les coalitions gouvernementales qui dirigent la République. \\
L'importance croissante des mouvements sociaux amènent les radicaux à intégrer des questions sociales et défendent donc l'impôt progressif sur le revenu mais ils avancent aussi d'autres idées comme l'assurance sociale, l'idée de retraites pour les ouvriers, etc. \\
Malgré tout, les radicaux restent très attachés à la propriété individuelle qui est selon eux, gage d'indépendance et de protection dans un système capitaliste. Les radicaux vont progresser sur le plan électoral et deviennent un parti de gouvernement. Ils font voter des lois importantes, la loi sur les congrégations religieuse qui sont interdites, ils mettent en application la séparation de l'Église et de l'État. Clémenceau devient président du conseil en 1906, le climat social est très agité, il est taxé de briseur de grèves et s'oppose à la SFIO de Jaurès sur plusieurs points, notamment au pacifisme, et les mesures sociales que veulent les socialistes. Clémenceau réfute la lutte de classe mais les radicaux son divisés entre deux franges, une plutôt de gauche, cherchant l'union avec les socialistes et une de droite qui cherche l'alliance avec les modérés. 


Les radicaux, pendant l'entre deux guerres, continuent d'occuper une position centrale. Le parti radical devient pratiquement indispensable à la formation de toute coalition gouvernementale et participe à des gouvernement tantôt de gauche, tantôt de droite. \\
En 1919, il s'allie en partie à la droite pour former le bloc national. En 1924, l'alliance se renverse et les radicaux s'associent à la gauche pour former le cartel des gauches. Cette majorité se désagrège petit à petit et se reforme en 1932 sous le nom de néo-cartel des gauches. Il faut noter qu'à cette période (1932), les socialistes commencent à dépasser en nombre de voix les radicaux. \\
Il y a au sein du parti radical, un mouvement qui se fait appeler "les jeunes turcs", qui défend les mêmes prises de positions que le parti du même nom qui a pris le pouvoir en Turquie. Ils défendent un ancrage plus à gauche du parti radical et ce courant va prendre l'ascendant dans le parti à partir de 1935. C'est ainsi que le parti radical va prendre part à une nouvelle coalition, le Front Populaire. C'est Edouard Daladier qui dirige le parti à cette époque qui signe l'alliance avec Leon Blum mais le Front Populaire finira en 1937. \\
Le score du parti radical avoisine les 20\% des voix en moyenne lors des élections législatives, ce qui fait d'eux une force politique majeur. Ils participent à pratiquement l'ensemble des Gouvernements de l'entre deux guerres et présideront 13 des 42 Gouvernements d'entre deux guerres. \\
Les radicaux maîtrisent parfaitement les rouages institutionnels, ils sont au centre de l'échiquier, et la faiblesse de la discipline partisane font que ce parti était le parti central ayant participé à presque tous les Gouvernements.

\section{Un parti politique peu structuré}

Le parti radical est un mouvement de mobilisation électoral qui dépend beaucoup de ses structures locales. Dans le même temps, les socialistes et les communistes développent leurs organisations avec un pouvoir central beaucoup plus important. La discipline de vote au parlement est très faible et ne sont pas respecté. L'autorité du parti se heurte à l'autonomie de ses parlementaires. \\
Le nombre de militants est assez limité, il y a environ 100 000 cotisants dans les années 30. En réalité, l'implantation du parti dépend beaucoup de réseaux auquel il est affilié, il s'appuie notamment sur les loges maçonniques, la ligue des droits de l'Homme, et aussi sur de nombreux journaux qui sont liés de manière plus ou moins direct au parti radical. \\
Sur le plan sociologique, le parti radical est implémenté dans les classes intermédiaires: fonctionnaires, artisans et commerçants mais aussi chez les professions libérales, ainsi que chez les journalistes. 


\section{Le déclin du radicalisme}

Le parti radical enregistre à la Libération un net recul de ses résultats électoraux: 11\% des voix en 1946. Plus explications existent pour ce déclin. \\
La troisième République apparaît comme un régime daté à la fin de la guerre et cette réputation rejaillit sur les radicaux alors que la libération est synonyme de vif renouveau politique. Certains radicaux ont aussi votés les pleins pouvoirs à Pétain, ce qui a contribué à précipité leur déclin. Le parti conserve néanmoins quelques bastions avec quelques fortes personnalités. Ils vont contribuer au maintien du parti radical: Pierre Mendès France et Edgar Faure, sous la IVème République. Sur 22 Présidents du conseil sous la IVe République, 9 seront radicaux. \\
Il y a toujours deux tendances au parti radical, une tendance conservatrice représenté par Edgar Faure, et une aile gauche qu'incarne Pierre Mendès France. Ce dernier va tenter de ré-encrer le parti radical à gauche dans les années 50. Il est nommé président du conseil en 1954 et prend la tête du parti radical en 1956. Ne parvenant pas à rénover le parti radical, il le quitte en 1958 et rejoint le PS. \\
Le déclin du parti radical s'accroît encore aux législatives de 1958 et ne font que 7,3\% des voix et à la fin des années 60, il ne compte plus que 20 000 adhérents. Une nouvelle tentative de rénovation du parti est fait par Servan-Shreiber, qui échoue également en 1971. Dans le climat particulier des années 70, où la bipolarisation devient importante, le parti éclate et se divise avec d'une part le MRG qui va aller signer le programme commun et d'autre part le parti radical valoisiens qui rejoindra l'UDF. Les radicaux ne survivent que comme force d'appoint, à gauche ou à droite. Ils vivent grâce à leurs élus locaux. Bernard Tapie contribue à redonner vie au MRG et réalise un score très important de 12\% aux européennes mais cette progression n'est pas suivie. 


\chapter{Le socialisme}

L'histoire du socialisme est intéressante car au départ, les socialistes occupent une fonction, un rôle, de parti d'opposition qui n'exerce pas le pouvoir, qui le rejette même. Aujourd'hui, le PS est devenu un parti de Gouvernement qui occupe une place centrale dans le système politique. Le socialisme français est marqué par le Marxisme qui le caractérise à ses origines et c'est notamment ce qui va expliquer les fortes spécificités du socialisme français. \\
En effet, le PS français, contrairement à ses homologues européens, n'abandonne que très tardivement son intransigeance aux valeurs originelles. Le PS va entretenir un rapport très ambigu au pouvoir, hésitant entre l'exercice ou non du pouvoir. 

\section{La constitution du socialisme français}

\subsection{L'hétérogénéité du socialisme et l'unification de 1905}

Les socialistes se revendiquent les représentants de la classe ouvrière, le prolétariat. Cette référence à la classe ouvrière va rester un marqueur idéologique important pour les socialistes. À mesure que progresse les revendications ouvrières, les socialistes eux aussi, progressent, ils sont divisés en plusieurs tendances dans les années 1880. Plusieurs groupes se disputent le label socialiste. Certains décident de participer aux élections. 1893, les socialistes comptent 37 députés. \\
D'autres, refusent, c'est le cas notamment des Guesdistes de Jules Guesde qui a fondé le parti ouvrier en 1880. Il incarne la tendance marxiste et la plus intransigeante du socialisme. Ils refusent catégoriquement de participer aux coalitions gouvernementales, y compris les gouvernements radicaux-socialistes. Ils sont cependant favorable à une prise de pouvoir local. Ils ne sont pas pour une révolution immédiate, ils sont pour préparer la révolution méthodiquement en créant une lutte des classes. Les conditions de réussite serait la création d'une structure partisane, forte, organisée. Ils sont installés massivement dans le Nord. Lors de la création de la SFIO, ils sont majoritaires jusqu'en 1936.


Autre courant, les Possibilistes, qui sont réformistes, favorables à la prise de pouvoir de l'État, ils sont prêt à accepter des alliances, une collaboration avec les partis "bourgeois", de manière à obtenir des réformes rapidement qui serviraient la classe ouvrière. \\
Troisième courant, les Allemanistes de Jean Allemane, ils sont proches des anarchistes, ils défendent comme mode d'action privilégié la grève générale. Ils ont été pendant un temps, proche des possibilistes. \\
Quatrième courant: les blanquistes qui considèrent que la Révolution doit naître d'une étincelle qui mettra le feu aux poudres, et la révolution doit être organisée par des individus quasi professionnels. Idéologiquement, on peut situer les blanquistes entre les guesdistes et les socialistes républicains qui participent aux élections. \\
Cinquième courant: les socialistes indépendants, qui ne souhaitent pas adhérer à un parti qui existe, ce sont souvent des intellectuels qui se revendiquent socialiste mais sans appartenir à une organisation politique. Le plus célèbre d'entre eux est Jean Jaurès. 


Tous ces courant socialistes sont différents les uns des autres, il existe entre eux des désaccords très important, notamment sur les questions de stratégie de conquête du pouvoir. Leurs objectifs sont sensiblement les même, ce qui les réunit derrière l'identité socialiste. Malgré toutes ces divergences, un processus d'union s'enclenche au début du XXe siècle, qui conduit à la formation de la SFIO. C'est un parti qui se définit comme un parti de classe, celui de la classe ouvrière. \\
Les différentes tendances y perdurent et Jean Jaurès devient une figure socialiste dominante et ce jusqu'à sa mort en 1914. 

\subsection{La spécificité du socialisme français}

Des spécificités commencent à apparaître et vont le marquer durablement. On en compte trois, mise en avant par Rémi Lefebvre. \\
La première spécificité est la faiblesse de son enracinement social. Il y a peu de connexions entre la classe ouvrière et les socialistes. Il y a très peu de liens entre les socialistes et le monde du travail, comme avec les syndicats. Et, contrairement au parti communiste, la SFIO ne parvient jamais à se constituer en contre société, en parti "milieu de vie". \\
La deuxième spécificité est la faiblesse organisationnelle du SFIO. L'appareil partisan socialiste se manifeste par une bureaucratie très faible et limité. Il présente une forte décentralisation: les élus locaux ont un poids très important au sein du parti, mais son très difficilement contrôlé par l'état major national. Enfin, il n'existe presque pas de formes développés du militantisme. Il ne parviendra jamais à mobiliser massivement la classe ouvrière ou au delà. Cela constitue une particularité par rapport aux autres partis européens qui eux, développent la structure sociale-démocrate. \\
La troisième spécificité est le rapport ambigu du parti avec le pouvoir. La SFIO a dès ses débuts, entretenus un rapport extérieur au pouvoir. Ce n'est donc que très tardivement que les socialistes français acceptent leur intégration dans le système politique, acceptent les règles du jeu du système politique. Le socialisme a toujours manifesté une grande méfiance envers la démocratie représentative. Ce rapport d'extériorité au pouvoir va durer pendant un certain temps avec une prise ponctuelle du pouvoir comme en 1936. Ils abandonneront la référence révolutionnaire très tard. 


Une approche beaucoup plus localisée montre que le socialisme est beaucoup plus diversifié qu'on ne le pense. C'est le travail de Frédéric Sawicki. Il montre que les fédérations socialistes sont très différentes les une des autres. Lefebvre, lui aussi, revisite ces trois grandes spécificités. \\
Néanmoins, la SFIO, dès ses débuts, constituent avant tout une force électorale plutôt qu'une forme de mobilisation. En 1914, il compte 80 000 militants, ce qui est assez peu. On compte une centaine de députés, ce qui est plutôt important. Ce sont majoritairement des députés issus de professions libérales ou intellectuelle. Les élus jouent un rôle très important dans le parti. 

\section{La SFIO dans l'entre-deux-guerres}

\subsection{La progression socialiste}

Après la première guerre mondiale, en 1920, le parti connaît une scission très importante, celle-ci va voir la création d'un nouveau parti: la SFIC au congrès de Tour. Lors du congrès de Tour, la SFIO se divise, la SFIC apparaît sous l'impulsion de la IIIe internationale. Une grande majorité des socialistes acceptent les propositions de Lénine alors qu'une minorité décide de garder la SFIO. Parmi eux, Léon Blum. \\
Même si la SFIO est touchée par la scission, la SFIC peine à décoller alors que la SFIO commence à progresser. \\
Aux législatives de 1924, la SFIO enregistre 1,5 millions de voix et une centaine de députés. Aux législatives de 1928, ils ont 1,7 millions de voix et une centaine de députés élus. En 1934, ils ont 1,9 millions de voix et 130 députés élus. \\
En 1933, la SFIO compte 133 000 adhérents quand la SFIC en compte 30 000. Néanmoins, des tensions apparaissent et continuent de s'accroître. La référence au marxisme est toujours très présente et dans le même temps, l'électorat l'est de plus en plus aussi. Le parti est toujours aussi intransigeant vis à vis de la politique du pouvoir et est en même temps de plus en plus réformiste. L'ouvriérisme affiché de la SFIO fait contraste avec le fait que la SFIO est de moins en moins ancré dans la classe ouvrière. Ils développent essentiellement leur électorat dans la classe intermédiaire ou la petite bourgeoisie. Il y a bien sûr quelques bastions ouvriers: le Nord notamment.


Dans les années 30, deux figures dominent la SFIO: Léon Blum et Paul Faure. Faure est le secrétaire général du parti, il est représentant d'une ligne intransigeante. Blum, quant à lui, est un intellectuel, il était l'un des leaders pour défendre la vieille SFIO lors du congrès de Tour. Il incarne un socialisme qui s'inscrit très bien dans la tradition républicaine et il n'a pas de responsabilité dans le parti mais a des responsabilités au sein du groupe parlementaire socialiste et dirige le journal de la SFIO: le Populaire. \\
La question de la participation au pouvoir est central dans l'entre deux guerres. Blum va essayer de clarifier cette tension idéologique. Pour lui, les socialistes peuvent exercer le pouvoir si ils sont majoritaires au Parlement mais leur objectif restent de conquérir le pouvoir par la voie révolutionnaire. Blum distingue donc la conquête et l'exercice du pouvoir. L'exercice implique une adhésion au système représentatif qui ne peut qu'être provisoire. \\
Les communistes, bien qu'ils soient minoritaires, occupent l'aile gauche de la SFIO est les socialistes se trouve repoussé vers le centre de l'espace politique. Si ils ne peuvent pas s'allier avec les communistes à leur gauche, il y a les radicaux à leur droite. Plusieurs fois, la SFIO sera allié avec les radicaux comme en 1924 où la SFIO soutient mais ne participe pas. Elle soutient de loin des actions comme les nationalisations, soutiennent la semaine de 40h mais sans participer au gouvernement. \\
À partir de 1934, la donne va changé puisque la SFIC va sortir de son isolement pour adopter une stratégie de front unique qui cherche à rassembler la classe ouvrière et la classe moyenne dans la lutte contre le fascisme. 








\subsection{Les socialistes entre conquête et exercice du pouvoir}

Le front populaire, comportant la SFIO, la SFIC et les radicaux de gauche, va remporter en 1936 les élections. Pour la première fois, un gouvernement majoritairement socialiste est créé. Il est dirigé par Léon Blum. Pour la première fois, les socialistes dépassent les radicaux aux élections. \\
Le gouvernement de Léon Blum met en place une nouvelle politique économique audacieuse, avec de grandes réformes sociales. Ce sont les congés payés, la semaine de 40 heures etc. Tout ça dans un contexte social très mouvementé avec des grèves et des occupations d'usine très importantes. \\
Les communistes soutiennent sans participer. Le PCF profite également de ce contexte, il décolle électoralement et va compter 300 000 adhérents en 1936. \\
Le front populaire est une référence mythique au vu du nombre de conquêtes sociales qui font date. Néanmoins, la SFIO va entamer sous la IVe et la Ve République un long déclin.

\section{Déclin et refondation du socialisme sous les IVe et Ve République}

\subsection{Un déclin paradoxal jusqu'à la fin des années 60}

La SFIO va participer à de nombreux gouvernements sous la IVe République. Malgré toutes ces participations importantes, le déclin de la SFIO s'accentue. En 1946, la SFIO obtient 18\% des voix aux élections législatives. En 1962, seulement 12\% des voix. \\
La figure socialiste qui domine lors de cette période d'après guerre est Guy Mollet jusqu'en 1969. Il a assumé à la libération une ligne intransigeante contre la volonté réformiste de Léon Blum. Il a cherché à incarner la pureté originelle du socialisme. En pratique, sa direction prise s'est avérée finalement très électoraliste. \\
Guy Mollet, en tant que président du conseil en 1946, va mener une politique très répressive en Algérie, ce qui choque une large partie de l'opinion publique de gauche. Enfin, Mollet va finir par se rallier à la solution de la Ve République de De Gaulles, ce qui va achever totalement de diviser la SFIO. Les effectifs de la SFIO vont décroître rapidement: moins de 100 000 adhérents en 1960 et dès septembre 1958, la SFIO redevient un parti d'opposition, le mouvement continue d'éclater et de se disperser. C'est une période où la SFIO est très clairement discrédité, elle apparaît comme un parti incapable de s'adapter aux nouvelles règles du jeu de la Ve République. \\
Un rapprochement est amorcé entre la SFIO et le PCF dans les années 60. C'est un rapprochement qui a abouti à une candidature commune aux élections de 1965. 


En 1969, les élections sont un fiasco pour la gauche après la crise politique qu'a présentée Mai 68, la gauche était divisée et n'a pas trouvée de terrain d'entente pour une candidature commune. Le candidat socialiste obtient à peine 5\%. \\
Il s'ouvre alors un cycle nouveau pour la SFIO, une rénovation profonde s'impose pour fonder un nouveau PS et partir à la conquête du pouvoir. 

\subsection{Le nouveau PS en conquête du pouvoir}

Les années 70 constituent une période de renouveau très profond pour le socialisme français. En 1969, la SFIO change de nom et s'appellera désormais Parti Socialiste. Guy Mollet est remplacé. Il faut attendre le congrès d'Epinay en 1971 pour que le PS connaisse un véritable renouveau avec le retour de nombreux groupes extérieurs au parti qui le rejoigne. \\
Le congrès d'Epinay est perçu aujourd'hui comme le renouveau du PS, renouveau qui va le conduire jusqu'à la victoire de 1981. C'est d'ailleurs lors du congrès d'Epinay que François Mitterrand va prendre la tête du PS. Il entre au PS à ce moment, jusque là, il dirigeait un groupe extérieur. C'est avec le soutien de l'aile gauche qu'il est désigné premier secrétaire. Il est soutenu par Jean Pierre Chevènement et également par deux grandes fédérations socialistes: celle du Nord et celle des Bouches du Rhône. \\
Mitterrand n'est pas socialiste de tradition, il a été ministre centre droit sous la IVe République et va pourtant doter le PS d'une nouvelle stratégie de conquête du pouvoir, celle de l'union à gauche. Il faut, pour lui, réaliser l'union de la gauche pour gagner l'élection présidentielle. Du même coup, il redéfinit l'emplacement idéologique du PS en le marquant à gauche. \\
Mitterrand va ouvrir le PS sur la société civile. Il va cherché à ramené au PS de nouvelles classes sociales: les classes moyennes salariées notamment. Il va réussir à augmenter sensiblement les effectifs de militants du PS. 











\end{document}
