\documentclass[10pt, a4paper, openany]{book}

\usepackage[utf8x]{inputenc}
\usepackage[T1]{fontenc}
\usepackage[francais]{babel}
\usepackage{bookman}
\usepackage{fullpage}
\setlength{\parskip}{5px}
\date{}
\title{Cours d'histoire politique contemporaine (UFR Amiens)}
\pagestyle{plain}



\begin{document}
\maketitle
\tableofcontents

\chapter{Introduction}

Le mot politique est un terme polysémique. Cela se présente comme un ensemble de forces institutionnalisés qui interagissent entre elles dans ce que l'on pourrait appeler le champ politique. Cette fonction de régulation sociale du politique se traduit par la mise en oeuvre de politiques publiques qui sont des dispositifs d'actions publiques qui visent à produire un certain nombre d'effets sociaux. \\
Dans le cadre de ce cours, la définition qui nous intéresse est la politique que l'on entend comme être la scène politique sur laquelle s'affrontent sous les yeux du public et des citoyens une série d'acteurs qui cherchent à conquérir et exercer le pouvoir. \\
Bourdieu: "C'est le lieu où s'engendre dans la concurrence entre les agents qui s'y trouvent engagés des produits politiques entre lesquelles les citoyens ordinaires réduit au statut de consommateur doivent choisir". Telle que défini par Bourdieu, la politique serai un marché, un marché économique. On remarque que pour lui, les citoyens ne jouent qu'un rôle mineur et sont réduits à être des consommateurs. \\
Les marchés politiques sont des lieux où s'échangent des produits politiques (comme un programme) contre des soutiens matériels ou immatériels et évidemment, des votes. Cette métaphore économique est donc tout à fait pertinente: il y a donc un marché politique. \\
Comme dans un marché économique, il y a donc un jeu de l'offre et de la demande. La demande étant l'attente des électeurs et leurs préoccupations. Si le marché politique est véritablement un marché économique au sens strict du terme, les citoyens ont un rôle restreint. Dans la mesure où la rationalité du citoyen politique est bien inférieur à la rationalité du citoyen économique. \\
On va, dans ce cours, s'intéresser à un siècle de vie politique. On va s'appuyer sur de grands événements de la vie politique française, sur les acteurs, sur les évolutions des tendances, des sensibilités politiques, sur les luttes électorales, sur les résultats, sur les enjeux politiques qui se constituent, sur les clivages, sur les règles du jeu.


La concurrence et l'interdépendance politique a lieu sous le joug d'un certain nombre de règles politiques: le système partisan, les coalitions, le système de scrutin, la nature des "trophées" qui sont en jeu (degré de parlementarisme, hyper présidentialisation), les normes constitutionnelles etc. \\
Ces règles ne sont pas univoques. La politique n'est pas réglée une bonne fois pour toutes par les constitutions. Ce qui compte, c'est l'usage qu'en font les acteurs politiques. Ces règles sont susceptibles de changer, d'évoluer. Il ne faut donc pas donner à ces règles plus d'importance qu'elles n'en ont.


Les traditions politiques sont des courants, des forces, des sensibilités qui ont une certaine constance dans le temps. On entendra par tradition plus précisément, la permanence à travers le temps d'un système relativement cohérent de représentations et d'images, de souvenirs et de comportements, de fidélité et de refus. \\
Ces traditions s'incarnent dans des organisations, principalement dans des partis politiques. Ces traditions n'existent pas en soi. Il faut se déprendre de l'illusion qui présente ces traditions comme des essences intemporelles. \\
Les traditions doivent être appréhendés comme des constructions historico-sociales. Ces constructions résultent d'un travail historique. \\
Par ailleurs, les traditions politiques ont des fonctions de légitimation. S'approprier le monopole de l'héritage permet de stigmatiser un opposant politique. 


\chapter{L'invention de la politique moderne}

\section{L'instauration de la République}

Le 4 Septembre 1870, la IIIe République est proclamée par Gambetta suite à la capture de Napoléon III par les Prussiens. La IIIe disparaîtra en 1940 quand les pleins pouvoirs seront donnés à Pétain. \\
La IIIe est une période fondamentale pour comprendre la vie politique française. C'est à partir des années 1870 que se sont posés les fondements de la démocratie représentative telle qu'on la connaît encore aujourd'hui. C'est sous cette République que le régime Républicain s'impose. La pratique du vote se généralise à partir des années 1870, il existait depuis 1848 mais était encadré pendant le second empire. La lutte électorale s'intensifie, des marchés électoraux se créent. La politique va devenir une sphère de plus en plus autonome et va se professionnalisé. \\
Le sentiment d'appartenance nationale progresse.


Comment expliquer ces transformations profondes pendant la IIIe République ? \\
Certaines de ces transformations peuvent être imputés notamment à l'introduction du suffrage universel masculin en 1848. \\
Les régimes d'avant la IIIe sont essentiellement des monarchies. C'est donc le suffrage censitaire qui domine dans ces régimes. Le cens est variable selon les élections: 1 Franc pour les élections municipales, 200 ou 400 Francs pour des élections législatives. Comme tout le monde ne peut pas payer, les marchés électoraux sont restreints, peu concurrentiel, dominés par les notables. \\
Restreints car peu d'électeurs. Les trois quarts des collèges électoraux étaient composés de moins de 600 inscrits. Peu concurrentiel car peu de candidats. Plus de 80\% des députés sont élus avec moins de 400 voix. Enfin, ils sont dominés par les notables qui ont les ressources personnelles qu'ils peuvent investir dans le champ politique. Les biens qu'investissent les notables sont essentiellement non spécifiquement politique (clientélisme).


L'introduction du suffrage universel a contribué à accroître considérablement le nombre d'électeurs. On passe de 250 000 électeurs à plus de 10 millions. Il est à noter que tous n'intègre pas le vote du jour au lendemain. Le suffrage universel ne va réellement que s'épanouir sous la IIIe République. Il en va de même pour les candidats qui vont renouveler leurs stratégies, contraints par leur nouvel électorat. \\
Dans un premier temps, ils cherchent à continuer la relation clientéliste, mais il est impossible de rémunérer l'ensemble du corps électoral. \\
Progressivement, le vote va s'individualiser et devenir l'expression d'une opinion personnelle. Ce processus s'initie sous la IIIe République et s'intensifie via la mise en place de politiques volontariste, pour apprendre aux citoyens à voter. Il va s'individualiser car de nouveaux acteurs vont émerger et concurrencer les notables et contester leurs pouvoirs en proposant des visions politiques et en redéfinissant les relations électorales.

\section{La République dans les moeurs}

La République comme régime s'est progressivement affirmé dans deux directions différentes.

\subsection{La consécration dans la classe politique}

Au cours du XIXe siècle, le régime est un débat central. Près d'un siècle après la Révolution française, les monarchistes sont toujours nombreux et mobilisés. \\
Au départ, la République s'impose comme un régime par défaut puis va commencer à devenir dans les esprits, le régime qui divise le moins. On parlait à cette époque de "République d'attente".


En 1870 éclate la guerre avec la Prusse, la défaite se fait deux mois plus tard avec la capture de Napoléon III. À Paris, se constitue un gouvernement de défense nationale qui proclame la République. \\
Des élections législatives sont donc organisés à la va vite. L'enjeu central est bien évidemment la guerre avec la Prusse. Les Royalistes sont favorables à la paix tandis que les Républicains souhaitent poursuivre la guerre. Les partisans de la République obtiendront 200 députés contre 400 pour les royalistes. \\
C'est ainsi que Adolf Thiers devient le chef du pouvoir exécutif dans cette République. De manière assez paradoxale, l'épisode de la commune de Paris va renforcer la République. Certains considèrent que la commune de Paris est une première manifestation du communisme. À l'époque, la problématique des républicains et des monarchistes est la stabilité du régime. En réprimant un mouvement insurrectionnel, la République a fait sa preuve d'un régime stable. En 1872, Thiers se rallie à la République.


C'est le régime qui divise le moins car les monarchistes sont, à l'époque, très divisés. Il y a les Orléanistes, héritiers de la monarchie de Juillet et la monarchie constitutionnelle et parlementaire, modéré et d'autre part, les légitimistes qui revendiquent les héritages des ultras de la première restauration. \\
C'est pour cette raison que Thiers sera renversé en 1873, et Mac Mahon sera élu président. Les divisions persistent chez les monarchistes et rendent impossible une éventuelle restauration. \\
Ce n'est qu'en 1875 que la République est juridiquement renforcée. Le mot République apparaît dans trois lois constitutionnelles. C'est presque par hasard pendant les débats parlementaires qu'un amendement introduit le mot République dans la loi constitutionnelle. \\
Les républicains vont progresser aux élections, leur audience est de plus en plus important à travers le pays et ils finissent par remporter une majorité aux élections de 1876. McMahon dissoudra la chambre et de nouvelles élections seront organisées en 1877, qui donnera, de nouveau, une majorité au camp des Républicains. C'est lors de cette élection que le leader républicain Gambetta, aurait lancé une formule restée célèbre "Quand la France aura fait entendre sa voix il faudra se soumettre ou se démettre". \\
En Janvier  1879, McMahon démissionnera suite à la victoire des Républicains aux élections sénatoriales. 9 ans après la proclamation de la République, c'est la première fois qu'un Républicain est chef de l'État: Jules Grévy. \\
La République est donc désormais acceptée comme seul régime légitime. 

\subsection{La "républicanisation" de la société}

L'affirmation de la République passe également par la mise en oeuvre de politiques publiques volontaristes que les républicains vont mettre en place. On peut parler d'un processus de "républicanisation" de la société française. Processus car l'idée républicaine n'a pu s'imposer que progressivement dans la société française. \\
C'est en grande partie grâce aux politiques volontaristes, grâce aux symboles républicains qui se diffusent (La Marseillaise comme hymne national en 1879 et le 14 Juillet comme fête national). L'éducation civique apparaît avec l'école et permet de développer un attachement à la République aux plus jeunes. \\
Le projet des républicain qui repose sur la participation des citoyens à la vie politique ne prend sens que si le citoyen est en mesure de participer à la vie politique. L'école est donc censé formé des citoyens et consolide ainsi la République. Les républicains sont essentiellement anti-cléricaux et le développement de l'école laïque permet de concurrencer directement l'Église. 

\section{Le processus de politisation}

"On désigne par politisation le processus par lequel un intérêt pour la politique se développe dans la population". \\
Dans la France du XIXe, la population est essentiellement rural. Au début du XXe siècle, plus de 50\% de la population vit en milieu rural et près de 60\% des actifs sont des agriculteurs. Il y a donc un clivage important entre les villes et les campagnes. Ville où l'intérêt des citoyens pour la politique est un peu plus précoce et campagne où le développement de l'intérêt est plus tardif. \\
Dans le milieu rural, le niveau d'éducation est plus faible, les lieux du pouvoir sont plus visible pour les urbains. En ruralité, les entités locales sont très fortes, cependant, il faut manier cette thèse avec précaution, des historiens considèrent que la politique a toujours été très présente dans les villages mais peu tournés vers les arènes nationales. 


Pour Eugen Weber, un historien américain, il montre que la France va avoir tendance à s'uniformiser, les entités locales vont peu à peu s'effacer et le sentiment d'appartenance nationale, va, en contre partie, progresser. Pour lui, la société française va se nationaliser. Cette nationalisation est favorisée par quatre facteurs essentiels: le développement des communications (transport, essentiellement) ; progrès lié à la scolarisation ; le suffrage universel ; la première guerre mondiale. \\
Cette analyse d'Eugen Weber est devenu tout à fait classique mais n'est pas la seule.


Pour Maurice Aghulon, un historien français, propose pour sa part, une autre lecture. Il considère que les républicains vont s'appuyer sur les identités locales. Ce serai donc un processus moins direct. Il insiste sur le poids et l'importance du suffrage universel qui est le principal facteur d'éducation politique des citoyens. Cela signifie que l'intérêt pour la politique découle de l'exercice du droit de vote. Pour lui, les lieux de sociabilité local constitue des lieux de socialisation à la vie politique. Ce serai donc grâce à ces structures que la politique se propage jusque dans les plus petits village de France. 


\section{L'apprentissage du vote}

Le vote est au centre de la vie politique française. Cependant, cela n'a pas toujours été de soi. Le vote a donc subit un processus de naturalisation. "Si l'électeur aujourd'hui fait l'élection, c'est parce que préalablement, l'élection a fait l'électeur" Alain Garrigou. \\
Le vote va devenir peu à peu un comportement codifié, présenté comme l'expression d'une opinion personnelle et non plus comme la marque d'une appartenance à un groupe. La salle de vote va peu à peu perdre son caractère de lieu de délibération et va être érigé en un espace neutre et démocratique. \\
Dans ce lieu, l'électeur doit accomplir dans le calme son devoir civique. L'état d'ivresse est interdit dans un lieu de vote en 1874. Les urnes, les bulletins, l'isoloir, l'interdiction de débattre, d'exercer la moindre pression sur un électeur ne sont pas anecdotiques, et représentent l'idée que les républicains se font du vote. \\
Quelques dates clés de l'apprentissage du vote:
\begin{itemize}
\item 1848: listes permanentes et alphabétiques de l'ensemble des électeurs ;
\item 1852: organisation de la procédure électoral ;
\item 1860: standardisation des urnes ;
\item 1884: carte des électeurs obligatoires ;
\item 1913: isoloir ;
\item 1923: bulletins ;
\item 1988: paraphe sur les listes d'émargement obligatoire.
\end{itemize}
Les républicains tentent de promouvoir une vision individualiste de l'électeur. En votant, les citoyens participent à une communauté nationales. Dis autrement, les citoyens deviennent français en votant et inversement.

\section{La professionnalisation de la politique}

Parallèlement, l'activité politique va devenir un champ autonome. On observe un processus de professionnalisation. Une figure idéal-typique du professionnel de la politique apparaît à la toute fin du XIXe siècle, qui conteste le pouvoir des notables, qui représente des groupes sociaux jusque là ignorés dans la sphère politique. Ces nouveaux professionnels appartiennent souvent à des professions libérales, et à la bourgeoisie de capacité bien souvent Mais aussi des ouvriers, et des représentants du monde ouvrier. \\
Ces nouveaux professionnels ont peu de ressource personnelles donc propose plutôt de la propagande électorale, vont se constituer leurs propres ressources: collectives et organisationnels (partis, mobilisations collectives...). D'un certain point de vue, ils opposent la force du nom par rapport à la force du nom. Ils vont collectivement, développer de nouvelles techniques de campagne. Ces nouveaux entrepreneurs tentent de redéfinir la relation électorale.\\
L'élection va devenir ce qu'elle est, un échange de programmes électoraux contre des suffrages. \\
À noter que le processus va être double, les notables vont se professionnaliser et les professionnels vont se notabilisés. 

\section{La naissance des partis politiques}

À la fin du XIXe naissent les premiers partis politiques au sens moderne du terme, au sens où on l'entend encore aujourd'hui: "Une organisation durable, qui participe à la lutte pour la conquête de postes électifs en mobilisant des soutiens populaires grâce à une organisation ramifiée et territorialement organisée". \\
Pendant longtemps, le mot parti était au sens prendre parti pour tel ou tel idéologie. C'est donc au XIXe que les partis deviennent des acteurs centraux. Ils acquièrent progressivement le monopole de l'activité politique. Certains pensent que les partis politiques sont les enfants du suffrage universel, car celui-ci rend impératif l'organisation de la conquête des suffrages. \\
Jusqu'en 1901, les partis ne sont que tolérés par la réglementation, ce n'est qu'en 1901 et la loi sur les associations qu'ils vont se doter d'un cadre légal. C'est une reconnaissance assez tardive des partis politiques. \\
Les formes que peuvent prendre les partis politiques sont très variables. Elles sont variables historiquement, dans la position sur l'échiquier politique etc.


Les premiers partis politiques sont issus des mouvements ouvriers, et se constituent dans les années 1870: ils s'appuient donc sur le nombre et la mobilisation de leurs adhérents, sur les réseaux de sociabilité ouvrières. \\
Jules Guesde et son parti, ont des premières victoires électorales au début des années 1890. \\
Cependant, les partis modernes n'apparaissent que réellement au début du XXe siècle: le parti radical en 1901. Dans les années suivantes, les principaux courants politiques vont s'organiser en partis politiques. \\
La vie politique tend à se partisanisé: les signes et emblèmes partisans prennent peu à peu du sens pour les électeurs ; les individus commencent à s'identifier aux partis politiques, aux programmes. Les partis exercent des fonctions de plus en plus centrales: désignation des candidats. Ils structurent l'offre politique. Ils encadrent la mobilisation électorale. \\
En bref, ils structurent la vie politique dans son ensemble. 








\chapter{Le clivage gauche/droite}













\chapter{}













\end{document}
