\documentclass[10pt, a4paper, openany]{book}

\usepackage[utf8x]{inputenc}
\usepackage[T1]{fontenc}
\usepackage[francais]{babel}
\usepackage{bookman}
\usepackage{fullpage}
\setlength{\parskip}{5px}
\date{}
\title{Cours de Sociologie politique: police, justice, prison (UFR Amiens)}
\pagestyle{plain}

\begin{document}
\maketitle
\tableofcontents

\chapter{Introduction}

Le climat actuel est propice aux pré-notions et empêchent de prendre les faits sociaux comme des choses au sens de la méthodologie de Durkheim. Les pré-notions, selon Durkheim, sont formés par la pratique et pour elle. D'un point de vue théorique, les pré-notions peuvent être fausses mais aussi dangereuses. Nous cherchons, en sociologie, à comprendre et expliquer les faits sociaux. \\

\section{Perspective socio-historique de l'approche scientifique du crime}

Le crime a souvent été délaissé par les milieux scientifiques et la progression d'une approche étiologique (recherche de cause) à une approche sociologique a été lente.  


On distingue trois aspects dans l'étude sociologique du crime, élaborés par Burges et reprises par Sutherland. \\
Premièrement, la criminalisation primaire, le criminalisation par le droit de certains comportements. \\
Deuxièmement, la transgression de cette criminalisation. \\
Troisièmement, la criminalisation secondaire qui renvoie à la répression pénale et à la punition. 


Beccaria inspire la rationalité pénale moderne en proposant une définition et un sens à la peine mais également en proposant des principe fondateurs du système pénal, à savoir l'égalité et la proportionnalité des peines. \\
Fin XIXe, l'école positiviste italienne ne fait des recherches que sur le crime et cherche à tout savoir à propos de cela. L'école Durkheimienne propose une première définition sociologique du crime. Ils parlent de la normalité, de la relativité et de l'utilité du crime. En effet, le crime est un phénomène normal au sens où le crime est un fait social comme un autre. Le crime est relatif et en cela, il se réfère à la relativité de la norme pénale qui change selon le temps et l'espace. Face à la relativité, Durkheim propose une définition méthodologique du crime: "Le crime est tout acte qui, à un degré quelconque détermine pour son auteur une réaction qu'on appelle la peine". Le crime est utile et socialement nécessaire car il est utile dans les évolutions de la morale et du droit. \\
L'école positiviste italienne cherche l'anomalie chez l'homme criminel. Ce savoir sur le crime implique une certaine conception des politiques répressives, à savoir la "défense sociale". 





Incrimination primaire
Déviance
Définition du crime de P. Robert
La place de la norme pénal dans l'univers normatif
Approche constructiviste des problèmes publics
Intérêt protégé dans le processus de la création de la loi
La dimension symbolique des lois pénales
L'inflation législative



























\end{document}
