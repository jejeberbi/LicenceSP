\documentclass[10pt, a4paper, openany]{book}

\usepackage[utf8x]{inputenc}
\usepackage[T1]{fontenc}
\usepackage[francais]{babel}
\usepackage{bookman}
\usepackage{fullpage}
\setlength{\parskip}{5px}
\date{}
\title{Cours de Droit Pénal Général (UFR Amiens)}
\pagestyle{plain}


\begin{document}
\maketitle
\tableofcontents

\chapter{Introduction}

\section{La notion de droit pénal}

\subsection{Le droit pénal, un droit de punir}

Pour affirmer cela, il suffit de prendre la racine du mot pénal, pena, qui signifie punir. Le droit pénal est donc le droit de la sanction. Ce droit désigne donc ce qui est autorisé de faire et ce qu'il est interdit de faire et va renvoyer à la peine si infraction il y a. \\
Cette peine est un mal infligée à une personne et doit être proportionnée au mal qui a été fait à la société. Si un individu commet donc une infraction, on inflige une peine selon l'acte commis et non selon la dangerosité. \\
Le droit pénal a une fonction intimidatrice. Il faut qu'il fasse peur pour inciter à ne pas commettre d'infractions. \\
La prison a un objectif de rééducation, de réintégration, de reclassement social. Dès qu'un individu arrive en prison, on doit préparer sa sortie et donc aménager sa peine de sorte à former voir guérir celui qui la subit. 


En réalité, en matière criminelle, la prison est fréquemment utilisée, pareil en matière délictuelle. Or, les prisons sont surpeuplés. \\
La première difficultés est que cette surpopulation est génératrice de phénomènes: violence entre détenu et pour le personnel des prisons. Différentes études qui ont eu lieu pour réduire la récidive montrent que la courte peine est inefficace. 

\subsection{Le droit pénal général et les autres branches du droit}

Par rapport aux autres branches du droit, le droit pénal général a des mécanismes différends. Le droit pénal général renvoie souvent aux rapports inter-individuels, c'est pour ça qu'on rattache le droit pénal au droit privé. Cependant, il existe des infractions commises contre le bien public: évasion fiscale, corruption... \\
Le droit pénal, de façon générale est le droit sanctionateur des autres droits. \\
Il existe tout un tas de branches qui sont reliés au droit pénal: du droit pénal international, de la criminologie etc.

\section{Les caractères du droit pénal}

\subsection{Les sources historiques}

\subsubsection{Les sources de l'ancien régime}

Le droit pénal est le premier droit qui s'est formé. On trouvera du droit pénal dans les formes primitives de sociétés humaines. Dans la société primitive, lorsqu'est commise une infraction, celle ci donne d'abord lieu à une vengeance privée. Progressivement, on va avoir un passage à la justice publique car on se rend compte que les guerres sont épuisantes et que tout dommage ne nécessite pas nécessairement d'en déclencher une. \\
La première peine publique à exister est l'amende, permettant de dédommager la victime. Lorsqu'elle ne peut pas être payée, on séquestre la personne jusqu'à ce que sa famille paye pour lui, c'est la création de la prison. \\
La justice publique va ensuite s'affirmé et développé un certain nombre de moyen et la période du moyen âge va être marqué par l'arbitraire des juges. Il y a donc une grande disparité des peines sur tout le royaume selon la personnalité des juges. \\
À la veille de la révolution, le système de preuve est très critiquée. Après les ordani, ce sont les duels judiciaires qui permettent de prouver innocence ou culpabilité. On se dit finalement que la seule preuve parfaite est l'aveu. C'est ainsi que la torture se développera. À la veille de la Révolution, tout ces aspects sont critiqués. Beccaria propose une réforme du droit pénal dans "Traité des délits et des peines" en 1764, il prône la disparition de la torture et de la peine de mort, et introduit le principe de légalité. 

\subsubsection{Du droit révolutionnaire à l'ancien code pénal}

Le droit révolutionnaire va s'inspirer du droit de Beccaria. Il y a une définition précise des infractions, liés à une peine. Le rôle du juge n'est donc que de prononcer la peine si l'infraction a été constatée. \\
Ce droit est assez peu efficace. Napoléon va donc créer l'ancien code pénal qui entrera en vigueur en 1811. Celui ci s'inspire essentiellement des idées de Benthan qui défend que le code pénal doit inspiré la crainte. Cette théorie s'appelle la théorie de l'utilitarisme. Cependant le législateur prévoit pour chaque infraction une peine minimale et une peine maximale. Le juge doit prononcer la peine dans cette fourchette. Sont prévues des circonstances atténuantes et des circonstances aggravantes permettant d'aller au delà ou en deçà de la peine légale. \\
L'ancien code pénal maintient trois peines corporelles. 

\subsubsection{De l'ancien code pénal au nouveau code pénal}

Les premières réformes vont faire disparaître, en 1832, les peines corporelles. En 1891, on instaure le sursis. C'est une possibilité pour le juge qui constate l'infraction et qui condamne la personne de suspendre l'exécution de la peine. Cette exécution peut se faire si l'individu revient devant le juge. \\
Dès la fin du XIXe siècle, le code pénal ne contient pas tout le droit pénal. De nombreux textes, codifiés ou non, font leur apparition et prévoit des sanctions pénales voire des régimes de responsabilité spécifique. La loi du 29 Juillet 1881 par exemple, sur la liberté de la presse, prévoit de nombreuses sanctions pénales comme la diffamation ou l'injure et qui n'apparaisse pas dans le code pénal. C'est aussi le cas d'une ordonnance de 1945 sur les enfants délinquants qui n'a jamais été codifiés. \\
Le code pénal va vieillir en raison des inspirations de la société. Juste après la seconde guerre mondiale est créé un courant de pensé "La défense sociale nouvelle" dont le chef de file est Marc Ancel qui défend que le délinquant est un malade social. La peine doit donc guérir le délinquant de sa maladie. \\
Le législateur va dans des sens contradictoire. Par exemple, en 1981, la loi sécurité et la loi sur l'abolition de la peine de mort font contraste.

\subsubsection{Depuis le nouveau code pénal}

L'idée d'un nouveau code pénal est ancienne. La première fois que l'on en parle, c'est en 1930, or la guerre arriva. Après la seconde guerre mondiale, on se concentre sur le code de procédure pénale. \\ 
Un avant-projet est présenté en 1978. Puis en 1981, c'est l'alternance et le projet est abandonné. On représentera quelque chose en 1986, or, l'assemblée subira une alternance. \\
Un consensus est trouvé et en Juillet 1992, le nouveau code pénal est adopté et entrera en vigueur en 1994. L'inconvénient du consensus est qu'il manque certains éléments dans le nouveau code pénal. Pour l'entrée en vigueur, c'est une bonne chose qu'il soit retardé de deux ans car permet de former les juristes avant d'appliquer les règles. \\
Même après l'adoption du nouveau code pénal et avant son entrée en vigueur, le nouveau code pénal est réformé plusieurs fois. Depuis 20 ans, le nombre de lois en matière pénale est extrêmement dense. La loi du 10 Juillet 2000 redéfinit la faute pénale, en 2003, une loi sur la sécurité intérieure ; en Décembre 2005, loi généralisant la possibilités d'imposer un bracelet électronique aux prisonniers ; 5 lois de lutte contre la récidive entre 2005 et 2010 ; instauration des peines planchers en 2007 ; loi pénitentiaire en 2009 ; loi du 6 Août 2012 redéfinissant le harcèlement sexuel ; loi du 15 Août 2014 créant la contrainte pénale ; loi du 3 juin 2016 renforçant la lute contre le terrorisme. \\
39 lois ont étés produites depuis l'adoption du nouveau code pénal. Ce grand nombre de lois provoque une insécurité juridique. Le mouvement législatif n'est pas néfaste en lui même, cependant un grand nombre de ces lois sont soit inutiles, soit maladroites, et sont souvent des lois de réactions. 

\subsection{Les sources formelles}

Dans les sources formelles, on pense de suite au code pénal, cependant, ce n'est pas la seule source. 

\subsubsection{Sources internationales}

C'est par exemple une convention du 9 Décembre 1948 qui va définir le génocide. Le traité de Rome crée la CPI en 1998, chargé en vertu de ce traité, de poursuivre les auteurs de crimes internationaux. 

\subsubsection{Droit de l'UE}

En 1957, on ne trouve pas de traces du droit pénal dans les traités. Progressivement, le droit de l'UE est devenu une source directe du droit pénal. Dans les domaines où il était nécessaire d'avoir un réglementation, une réglementation a été adoptée. Par exemple, en droit pénal de la pêche, en raison des réglementations internes, il y avait entre pêcheurs espagnols et français des filets de maillage différents, ce qui avait pour effet de produire des différences de quantité pêché, ce qui a été réglementé par l'UE. Depuis le traité de Lisbonne en 2009, le législateur en est venu à transposer des directives de droit européens. \\
Parfois, le droit de l'Union produit un effet de neutralisation ou d'appel du droit pénal. Pour le premier, lorsque la norme pénale s'apparente à une mesure que l'on peut considérer comme étant discriminatoire par rapport aux citoyens de l'UE, la norme européenne va neutraliser cette norme pénale. Pour le second, on peut relever trois types d'exemples, en matière de corruption, blanchiment ou mutilation sexuelle. Il n'existait aucune norme pénale réprimant la possibilité de sanctionner un fonctionnaire communautaire corrompu, on a créé une norme par effet d'appel. 

\subsubsection{Le droit de la CEDH}

La CEDH a une influence directe sur le droit pénal. Elle a été adopté en 1950 et sont liés par la convention, 57 États, donc presque toute l'Europe, sauf la Biélorussie. La Cour Européenne des droits de l'homme, lorsqu'elle interprète la convention, elle crée certaines innovations juridiques. \\
L'article 2, relatif au droit à la vie, prévoit également un certain nombre de tempérament avec des protocoles additionnels, qui, aujourd'hui, prohibent totalement la peine de mort. \\
L'article 3 prohibe les mauvais traitements, inhumains ou dégradants et la torture. Il est à noter que la France a été le deuxième pays à être condamné sur la base de cet article en raison d'interrogatoire en garde à vue car l'individu qui est passé en garde à vue a été passé à tabac. La France a également été condamné pour torture car une personne en garde à vue a été plus que passé à tabac. \\
Reynolde contre France, CEDH, 2008, un détenu faisant une tentative de suicide a été sauvé par un surveillant qui a été blessé dans la manoeuvre. Le prisonnier a été envoyé en quartier disciplinaire et sera retrouvé pendu. La CEDH considerera une atteinte à l'article 2 et 3. \\
L'article 4 prohibe le travail forcé. Les article 5 et 6 concernent la procédure pénale. L'article 7, lui, est particulièrement important. Il prévoit le principe de légalité des crimes et délits ainsi que le principe de la non rétroactivité. \\
L'article 4 du protocole additionnel 7, veut qu'un individu ne peut pas être jugé deux fois pour les mêmes faits

\subsubsection{Le bloc de constitutionnalité}

Il a une influence en droit pénal. La DDHC est plus importante sur le droit pénal car pose en ses article 5, 7 et 8 également de principe de légalité, de proportionnalité des peines et de non rétro activité de la loi pénale plus sévère.

\subsubsection{Le code pénal}

Il contient l'essentiel des règles de droit pénal. Il contient l'essentiel du droit pénal, excepté l'ordonnance de Février 1945 sur la délinquance des enfants. \\
Le législateur a adopté un numérotation décimale, le premier chiffre correspond au livre, le deuxième au titre, le troisième au chiffre. \\
Le livre I contient les dispositions générales, le livre II contient les atteintes aux personnes, le III contre les biens, le IV contre la chose publique, le V contient des règles de bio éthiques et de droit animal. \\
L'innovation majeure de ce nouveau code pénal est l'instauration de la responsabilité pénale des personnes mortes. Il supprime aussi les peines minimales, et donc aussi les circonstances atténuantes. Il codifie aussi certaines jurisprudences, notamment l'état de nécessité. 

\section{Les classifications des infractions}

\subsection{La distinction en crime, délit et contravention}

\subsubsection{L'énoncé de la distinction}

On repose sur l'article 111-1 du code pénal, qui dispose que selon leurs gravités, les infractions sont classés en crime, délit, ou contravention. Les crimes sont les infractions qui sont punis de la réclusion ou de la détention criminelle soit à perpétuité soit 30 ans, soit 20 ans, soit 15 ans. Le délit est puni d'une peine d'emprisonnement inférieur ou égal à 10 ans et/ou une peine d'amende supérieur ou égal à 3750 euros. \\
Les contraventions sont divisés en cinq classes de contraventions. Il n'y a qu'une peine d'amende qui va de 38 euros pour la première classe à 1500 euros pour la cinquième classe.

\subsubsection{L'intérêt de la distinction}

Le législateur est compétent pour définir les crimes et les délits. Le règlement est compétent pour définir les contraventions. La tentative d'un crime est toujours punissable. La tentative d'un délit n'est punissable que si la loi le prévoit, mais la tentative d'une contravention n'est jamais punissable. \\
Tout crime repose sur une intention. En principe, l'élément moral d'un délit est aussi une intention, toutefois, lorsque la loi le prévoit, l'élément moral du délit peut être une simple faute. Enfin, il n'y a pas d'élément moral pour les contraventions. \\
Toute la procédure pénale repose sur cette distinction. Les juridictions compétentes ne sont pas les mêmes. 

\subsection{La distinction entre infraction de commission et d'omission}

Soit l'infraction réside dans le fait de punir un acte positif, perceptible, soit réside dans le fait de punir une abstention. Il existe essentiellement des infractions de commission. Il y a quelques infractions d'omission, comme la non assistance à personne en danger, ou délaissement de mineur. \\
L'intérêt est de dire qu'il n'y a pas d'infraction de commission par omission. Si les effets d'une omission s'apparent à une commission, ce n'est pas pour autant qu'elle est punissable. Dans la décision de la CA Poitiers, 1901, la CA a puni une forme d'omission. \\
Dans un arrêt de la C.Cass, C.Crim, 2 septembre 2005, la cour dira que la présence de personnes dans un combat participe à l'affaiblissement psychologique de l'adversaire. Sur la complicité, il existe un principe d'abstention participative. Si des personnes se battent et qu'un attroupement a lieu, alors les spectateurs peuvent être considérés comme complices. \\

\subsection{La distinction entre infraction simple et infraction complexe}

Il y a plusieurs intérêts à faire cette distinction. S'agissant d'une infraction complexe, si, entre le début de l'infraction et le début de l'infraction, est adoptée une loi nouvelle, même plus sévère, cette loi va s'appliquée à l'infraction. \\
Tous les tribunaux dans les ressorts desquels une infraction complexe a été commise sont compétents. Le délai de prescription, c'est le délai entre le début de l'infraction et le début de l'action en justice. Le point de départ du délai de l'infraction est lorsque l'infraction est terminée.

\subsection{La distinction entre infractions instantanées et infractions continues}

Le vol est une infraction instantanée, elle se fait en un instant. \\
L'infraction continu, continu dans le temps, avec donc, la réitération de la volonté du coupable. Comme la séquéstration.


Si une loi plus sévère est adoptée, elle s'appliquera à l'infraction continue. Tous les tribunaux dans lesquels l'infraction est faite sont compétent, et le départ du délai de prescription est la fin de l'infraction. 

\subsection{La distinction entre infractions simples et infraction d'habitude}

Les infractions simples sont lorsque le législateur incrimine une action ou une omission. L'infraction d'habitude est lorsqu'il y a répétition dans le temps, par exemple l'exercice illégal de la médecine car l'exercice pré suppose une continuité dans le temps de l'exercice. \\
Le point de départ du délai de prescription est lorsque l'habitude prend fin. 

\subsection{La distinction entre infraction formelle et infraction matérielle}

Cette distinction suppose de regarder ce que fait le législateur. Soit le législateur prévoit que l'infraction est réprimée si on parvient à un certain résultat. Soit le législateur incrimine le cheminement vers un résultat, sans prendre en compte celui-ci. La première est une infraction matérielle, la deuxième une infraction formelle. \\
Dans le deuxième cas, l'hypothèse type est l'empoisonnement, et l'infraction est commise dès que l'on a administré la substance, sans s'occuper de si la personne est morte ou non. \\
La tentative d'infraction matérielle intervient juste avant la commission de l'infraction.

\subsection{La distinction entre infraction de criminalité et de bande organisée et de droit commun}

Avant 2004, on évoquait une distinction entre les infraction terroristes, et les autres. Aujourd'hui, les infractions terroristes sont des infractions de criminalité et de délinquance organisée. 706-73 du CPP dresse une liste d'infractions qui sont soit particulièrement graves qui reposent toutes sur un réseau comme le terrorisme, mais aussi le trafic de stups, le blanchiment d'argent. \\
La bande organisée est défini dans l'article 132-71 qui dispose "La bande organisée est le groupement formée ou l'entente établi caractérisée par un ou plusieurs faits matériels en vu de commettre une infraction". À noter que la bande organisée commence à trois personnes. \\
Depuis 2004, des textes se sont multipliés pour renforcer la singularité du terrorisme. \\
Le 9 Octobre 2014, le CC indique qu'il va falloir distinguer les infractions qui sont contre les personnes ou celles qui sont contre les biens. 


S'agissant de l'ensemble des infractions de bande organisée, il existe des moyens de preuve supplémentaires. Il y a par exemple, la possibilité de pratiquer des écoutes et des enregistrements. \\
On va également avoir la possibilité de pratiquer une infiltration. \\
Celui qui va empêcher de commettre l'infraction et qui fait partie de la bande pourra être exempté de peine. Celui qui minimisera les conséquences, aura une peine réduite de moitié. \\
Par rapport au terrorisme, il suppose aussi des moyens de preuve encore plus dérogatoire, dont un, les IMSI Catcher. \\
S'agissant de la GAV, il y a plusieurs délais. Celui de droit commun est de 48h. En criminalité organisée contre les bien, il en est de même. Cependant, le délai de GAV pour de la criminalité organisée contre les personnes, il est de 96h, et en terrorisme, il est de 6 jours. 

\part{Livre I: l'infraction}

\chapter{L'élément légal}

\section{Le principe de légalité}

\subsection{L'existence d'un texte}

Pour qu'il y ait une condamnation, il faut qu'un texte préalable existe. \\
Par exemple, dans le cas du "revenge porn", le législateur n'a jamais prévu de tel cas et a du donc le régler. Cela est de même avec le cas de la zoophilie.


\subsubsection{L'affirmation du principe}

Le principe de légalité a une valeur constitutionnelle, il résulte de la DDHC (Art. 5, 7 et 8). Il est repris dans les article 34 et 37 de la Constitution. \\
Le CC, le 28 Novembre 1973, va interprété l'article 8 de la DDHC, qui prévoit que "nul ne peut être détenu que dans les cas déterminés par la loi", or, à l'époque, certaines contraventions prévoient des peines privatives de libertés. Le CC dira que le pouvoir réglementaire peut définir les contraventions mais ne doit pas prévoir de peines privatives de libertés. \\
La C.Cas et le CE ont donc appliqué la théorie dite de la loi écran. Ils ne sont pas juges de la constitutionnalité des lois, et ne peuvent qu'appliquer les règlements. En conséquence, cette décision va rester lettre morte jusque dans le nouveau code pénal. Le pouvoir réglementaire est désormais compétent pour définir les contraventions mais ne peut plus leur attribuer de peines privatives de libertés. \\
Aujourd'hui, le problème se pose dans l'autre sens. La loi a créée une contravention. Le CC devrait donc en logique censuré cette loi. Cependant, au cours du processus législatif, lorsqu'il s'agit d'un PJL, le Gouvernement peut s'opposer à tout moment à l'adoption d'un texte qui empiéterait à ses prérogatives, et si il ne le fait pas, il est considéré qu'implicitement, il autorise l'empiétement du pouvoir législatif sur son pouvoir réglementaire. Cependant, dans notre cas, c'est une PPL, proposé par un parlementaire devenu membre du Parlement, et il reste à savoir si le CC censurera ou non. \\
Le principe de légalité est également prévu à l'article 7 de la CEDH. Il est encore rappelé dans le code pénal.

\subsubsection{Signification du principe}

Pour le législateur, exiger un tel principe peut paraître assez simple, une loi pour un crime ou un délit, un règlement pour les infractions. \\
Pour qu'une loi puisse prévoir une infraction, il faut que la loi ait une certaine qualité. CEDH, "Sunday times c/ Royaume-Uni", 26 Avril 1979, la cour estime qu'une loi pénale doit être de qualité: précise, prévisible, et accessible. Précise car il ne faut pas qu'il y ait le moindre doute sur le comportement à adopter ou à ne pas adopter, prévisible pour savoir ce que l'on encourt, accessible car tous les justiciables doivent comprendre la Loi. \\
Pendant longtemps, une règle de common law, au RU, prévoyait que le viol sur son épouse n'était pas constitutif d'une infraction. Dans les années 80, le juge change la règle et l'applique immédiatement. La CEDH dira que la règle était prévisible car elle estime que les moeurs ont évolués. \\
CEDH, 15 Novembre 1996 "Cantony c/ France", il fallait savoir si un produit était un médicament au sens de la Loi. Le code de la santé publique prévoyait plusieurs définitions, par renvois. La question était de savoir si la norme était facilement accessible, la CEDH dira que oui. \\
On a commencé à utiliser ce principe en droit interne, en regard de l'article 10 de la CEDH, qui prévoit la liberté d'expression. Il faut, pour limiter la liberté d'expression, un but légitime, que ce soit nécessaire, et que la limitation soit prévue par la Loi. \\
Une loi de 1931 interdisait de faire état d'une plainte avec constitution de partie civile. CEDH "Du roy et Malorie c/ France", 3 Octobre 2000, va indiquer que cette infraction là ne se justifie plus dans une société démocratique. Le 16 Janvier 2001, la C.Cass refuse d'appliquer le texte car la CEDH avait dis que le texte n'était pas légitime, et enfin, le législateur abrogera le texte. \\
Une infraction de 1881, interdisait de faire état des circonstances d'une infraction. C'était la deuxième fois qu'on faisait état de cette infraction. La première fois était dans l'affaire dite du japonais cannibale et la deuxième fois dans l'affaire de l'attentat du RER en 1995. La C.Cas refusera d'appliquer le texte car la loi aurait été trop imprécise. Depuis, la loi a été revu par le législateur. \\
Aujourd'hui, n'importe quelle loi peut être écartée par le juge en invoquant la CEDH. 


À partir de l'article 8 de la DDHC, le CC a estimé que toute infraction pénale doit être clairs et précise. \\
De nombreuses lois pénales n'ont pas été soumis au CC, ce pour des raisons politiques. Le CC n'a pas été saisi par exemple pour les quatre lois qui forment le nouveau code pénal. \\
Par exemple, une loi de 2010 inscrit l'inceste dans le code et on adopte une définition très large. 16 Septembre 2011, et 17 Février 2012, dans ces deux décisions, le CC constate la définition et cette définition est trop large. Le texte est donc abrogé. Le 8 Février 2016, une loi plus précise a été adoptée. \\
Le 17 Janvier 2002, une loi est adoptée pour définir le harcèlement sexuel. Le CC dira le 4 Mai 2012 que le harcèlement n'est pas suffisamment défini, la loi est donc abrogée. L'abrogation a conduit à ne plus pouvoir juger les gens pour harcèlement. Pendant quelques mois, l'infraction a été remplacée ; au lieu de poursuivre pour harcèlement, on poursuivait pour agression sexuelle ou viol, harcèlement moral, exhibition sexuelle, violence morale ou psychologique.


L'article 111-4 du code pénal dispose que la loi pénale est d'interprétation stricte. L'interprétation analogique n'est pas possible, sauf dans une hypothèse se fait dans une façon favorable par rapport à la personne poursuivie. \\
La jurisprudence indique régulièrement que l'interprétation doit avoir un sens clair et strict du texte. Le foeticide est la façon dont la C.Cas s'est prononcée sur le commencement de la vie. Assemblée, 29 Juin 2001, C.Cas, développe ce qu'est le foeticide: en l'espèce, c'est une femme enceinte percuté dans un accident de voiture, ce qui conduira à la mort de foetus. La C.Cas se demande si il y a homicide involontaire. La C.Cas considérera avec l'article 111-4 du Code pénal. L'homicide est défini comme "dès que quelqu'un a involontairement occasionné le décès d'une autre personne". Cela renvoie à une définition sociologique de la famille et l'homicide involontaire est possible que si elle est commise sur autrui. Entre l'auteur et la victime, il n'y a pas d'individualisation possible, il n'y a donc pas d'homicide involontaire sur le foeutus. \\
La jurisprudence va s'affiner, en 2003. Hypothèse similaire mais au moment de l'accident, le foetus n'est pas viable, et au moment de l'accident, le foetus va être expulsé du corps de la mère et son coeur va battre pendant 1 heure. Il y a donc bien homicide involontaire. \\
Ces cas ont néanmoins montrés qu'il y avait peut être une lacune du droit. C'est une lacune où il y avait une absence de règle spécifique par rapport au foetus. Alors on a créé une infraction: le délit d'interruption de grossesse. Ce délit fait double emploi avec le délit de violence. CA Amiens, 1er septembre 2014, se penche sur cette infraction. Le juge pénal fait peu souvent oeuvre de création. 


Quand on traite des rapports du juge et d'un acte administratif, on évoque l'article 111-5 du code pénal qui met fin à un conflit positif de compétence. TC, 5 Avril 1951, "Avranche et Démarês", il indique que le juge pénal est compétent pour interpréter les actes administratifs mais n'a pas le compétence pour en apprécier la légalité. Il faut attendre le nouveau code pénal pour avoir un solution définitive dans le 111-5: le juge pénal est compétent pour interpréter les actes administratifs et pour en apprécier la légalité lorsque la solution du litige en dépend.

\subsection{Le contrôle du texte}

En principe, le juge est compétent pour interpréter le texte. Le juge pénal est susceptible d'écarter un texte. Il peut l'écarter sur le fondement de la CEDH. Le juge pénal si il est saisi à cette fin, peut décider de renvoyer une QPC. Lorsqu'il le fait, le juge pénal fait un contrôle à minima pour vérifier si la question est nouvelle et sérieuse. \\
Le juge pénal a donc des compétences multiples.

\section{L'application de la Loi pénale dans l'espace}

Tous les cas de compétence, sauf un, figurent dans le code pénal. Il existe une unité des compétences législatives et juridictionnelles. Le juge pénal français est compétent pour appliquer le droit pénal français. Le juge pénal français peut appliquer du droit international. \\
Il y a une tendance impérialiste du droit pénal français, qui se donne compétent pour beaucoup de chose. D'un point de vue pratique, il existe un certain nombre de cas où le juge va au delà de ce qui se trouve dans les textes.

\subsection{Le système de territorialité}

\subsubsection{Le territoire de la République}

Le territoire de la République est la combinaison du territoire terrestre, maritime et aérien. Pour l'espace terrestre, c'est la France métropolitaine, les DOM, les TOM, les collectivités d'outre mer et ceux à statuts spéciaux. Pour l'espace maritime, il est fixé par des normes internationales. C'est d'abord 12 000 mile marins (22-23km), des conventions étendent une zone écologique exclusive à 200 000 marins, la France est également compétente s'agissant de tous les navires battant un pavillon français. La France n'est pas compétente pour les infractions commises par les navires militaires étranger dans les eaux territoriales, et donc est compétente pour tous ses navires militaires. L'espace aérien, c'est tout ce qui se situe au dessus de l'espace maritime et de l'espace terrestre. La France est compétente à bord des aéronefs immatriculés en France. De même pour les aéronefs militaires français, et on exclut la compétences pour les aéronefs militaires étrangers.

\subsubsection{Le rattachement territorial}

Le principe du fait constitutif, c'est un élément constitutif. Pour les infractions complexes, il y a plusieurs éléments constitutifs, comme dans l'escroquerie. Pour que la France soit compétente, il faut qu'un élément constitutif ait lieu en France. Plusieurs pays peuvent donc être compétentes. \\
Pour qu'il y ait compétence, il peut y avoir aussi condition préalable dans le pays. Par exemple, dans l'abus de confiance, la condition préalable est la signature d'un contrat: si le contrat a été fait en France, elle sera compétente. \\
Enfin, la France est compétente si l'infraction a des effets en France, comme dans l'escroquerie à la carte bancaire. Un problème se pose quand on a accès par les réseaux sociaux à des contenus présentant des injures à l'encontre de français. La C.Cas estimera que c'est pas parce qu'on peut accéder à ces propos, qu'on peut les juger. En 2016, le législateur adopte le point de vue diamétralement inverse: désormais, la France est compétente dès lors qu'une infraction commise à l'étranger est accessible depuis la France en utilisant les réseaux de communication. Cela pose un soucis d'effectivité de la Loi.


Première extension: l'extension légale. 113-5 C.Pénal, lorsqu'on est complice en France d'une infraction commise à l'étranger, la France est compétente pour la complicité et pour l'infraction sous la réserve que l'auteur de l'infraction n'a pas été jugé définitivement à l'étranger. \\
Il y a des extensions jurisprudentielles. La France est compétente pour les infractions connexes et les faits indivisibles. Une infraction connexe est une infraction qui est la suite nécessaire de l'autre. L'indivisibilité est lorsqu'on a des faits intimement liés entre eux. \\
Arrêt de principe: C.Crim, 3 Avril 1995, "Mc Ruby", navire immatriculé aux Bahamas, équipage exclusivement Ukrainien, il fait le tour de l'Afrique où il récupère sans s'en rendre compte 9 passagers clandestins et les marins découvrent cela lors de la nuit. Ils les tuent et les lance dans l'eau, sauf qu'il y a un survivant, qui s'enfuira lorsque le bateau accostera en France et préviendra les autorités françaises. Il y a indivisibilité car le plan est de tuer les clandestins, il y en a un qui a pris la fuite. Comme les marins le recherche dans les eaux territoriales, il y a indivisibilité avec les assassinats donc elle a la compétence. \\
Sur le trafic de stups, la C.Cas va avoir un raisonnement en deux temps: si un navire a pour objectif d'atteindre un pays en particulier, celui-ci est susceptible d'approvisionner la France, donc elle est compétente. De plus, il y a connexité entre le trafiquant à l'étranger qui a pour objectif d'importer de la drogue. 

\subsection{Les infractions commises hors du territoire de la République}

\subsubsection{Les systèmes de personnalité}

On parle de personnalité active lorsque l'auteur est français, et de personnalité passive lorsque la victime est française. 


Conditions communes: comme les infractions ont eu lieu à l'étranger, il faut qu'il y ait une plainte officielle du pays dans lequel l'infraction a eu lieu ou encore de la victime. Il faut que les poursuites soient effectués à la requête du ministère public. Il faut que la personne n'ait pas été définitivement jugée à l'étranger. \\
La C.Cas a du s'y reprendre à plusieurs fois. En Juin 2009, dans l'affaire Kalinka, l'Allemagne estime à 4 reprises que la mort était accidentelle, et classe l'affaire sans suite. En France, on aboutit à des conclusions opposées. La C.Crim dira qu'un classement sans suite n'a pas autorité de chose jugée. À noté que la C.Cas dira ça alors qu'en Allemagne, un classement sans suite a autorité de chose jugée. Ensuite, la C.Cas dira, sur la base de l'article 111-4, l'article 311-8, on ne peut pas condamner si quelqu'un a déjà été définitivement jugée, ce que ne fait pas le classement sans suite. \\
Au regard de la CEDH, Art 4, PA 7, qui donne la règle qu'on juge pas une personne deux fois pour les mêmes faits. Mais cette disposition n'est pas applicable. L'article 54 des accords de Schengen en revanche, empêche cela entre deux juridictions étrangères. La France va estimer que dans une tel hypothèse, cela ne concernait que les jugements sur le fond et donc la C.Cas va maintenant sa position en considérant qu'on pouvait bien juger en France. 


Conditions spécifiques à la personnalité passive: la personne doit être française au moment des faits ; l'infraction doit être un crime ou un délit punit d'emprisonnement. \\
Conditions spécifiques à la personnalité active: l'auteur des faits doit être français et l'être au moment des poursuites ; il faut que l'infraction soit un crime ou si il s'agit d'un délit, qu'il y ait réciprocité d'incrimination (délit puni par la France et dans le pays dans lequel le délit a eu lieu). 


Il y a plusieurs cadres dérogatoires. Le premier est le cas du tourisme sexuel avec une loi du 17 Juin 1998, 222-22, alinéa 3 qui prévoit la compétence en la matière. Dans cette matière, il n'y a pas besoin d'une plainte de la victime ou d'une dénonciation du pays. Cela s'applique même à la personne qui n'habite pas en France dès lors qu'il habite habituellement en France. Le clonage et le terrorisme sont aussi des cadres dérogatoires. 

\subsubsection{Le système de réalité}

Prévu à l'article 113-10, qui prévoit la compétence française alors même que l'infraction a été commise à l'étranger, par des étrangers, sur les étrangers. La France se reconnaît compétente car ces infractions touchent les intérêts de la France: faux monnayage, attaques d'ambassades. 

\subsubsection{Le système alternatif à l'extradition}

Il faut une plainte de la victime ou une dénonciation et que ce soit grave: crime ou délit puni au minimum de 5 ans d'emprisonnement. Soit la France extrade, soit elle juge. \\
La France peut refuser d'extrader car le pays en question ne respecte pas les règles du procès équitable, ou la peine encoure est une peine contraire aux peines française. Ou encore, c'est une infraction politique, ou alors l'extradition aura sur la personne des conséquences sur la santé d'une extrême gravité. 

\subsubsection{Le système de l'universalité}

C'est un système prévu par le CPP. La France va définir ce pour quoi elle est compétente en renvoyant à des textes internationaux. \\
Les conditions de la compétence universelle sont multiple. Il faut supposer que l'individu en question n'a pas été définitivement jugée à l'étranger. Il faut qu'une convention internationale prévoit le cas. Il faut qu'une loi transpose le cas. Pour que la France puisse exercer sa compétence, il faut que la personne se trouve en France. 


Il existe quatre compétences universelles principales. Le plus utilisé est le 689-2 du C.Pénal, la torture, défini dans des conventions internationales. Il y a aussi le terrorisme et le financement du terrorisme. La piraterie est aussi une compétence universelle. Il y a compétence universelle dans tout ce qui est dans la compétence de la CPI et il y a un principe de subsidiarité où on demande à la CPI de juger et si elle refus, le juge se fait dans le pays où la personne a été arrêtée. Il existe un cas de compétence universelle jamais utilisé: le trafic de matière nucléaire. 

\section{L'application de la loi pénale dans le temps}

\subsection{Précisions terminologiques}

La loi pénale a une spécificité, elle peut être paralysée par l'existence d'une loi d'amnistie. Ces lois ont pour objet de couvrir du droit ou de l'oubli la rigueur du droit pénal. Ces lois ont été régulièrement adopté car permettent la réconciliation. Le contraire de l'amnistie c'est la terreur. \\
Au cours du XXe siècle, l'amnistie était régulièrement utilisé par les Présidents pour augmenter leur popularité à des moments stratégiques. La dernière grande loi d'amnistie a été celle du 6 août 2002. Celle-ci permet au PR, par décret nominatif, d'amnistier notamment les personnes qui se sont illustrés au nom de la France dans tout une série de domaine dont les championnat internationaux, la culture, etc. Cette loi est toujours en vigueur mais a peu été utilisée.


La rétro-activité signifie appliquer une loi à une infraction intégralement commise avant l'entrée en vigueur de la Loi. \\
La loi pénale de fond est ce qui touche à la définition de l'infraction, à la responsabilité pénale et à la peine. La loi pénale de forme est ce qui reste: la preuve, les voies de recours, les juridictions, la prescription d'action publique, l'exécution de la sanction. \\
La loi pénale, au regard du droit interne, relève à priori de quelque chose de simple, qui relèverait du domaine pénal. Pessino c/ France, CEDH, 10 Octobre 2006, la Cour revient sur la CEDH qui réunit des pays romano-germanique où la loi est la principale source de droit et des pays de la common law. En conséquence, quand, dans la CEDH, on utilise le terme loi, on va devoir englober non seulement la Loi mais aussi la jurisprudence. Partant de là, toutes les exigences qui sont posés à l'égard de la Loi dans la convention s'appliquent également à la jurisprudence. Ainsi l'article 7 de la CEDH pose le principe d'une non rétro-activité de la peine plus sévère, que ce soit une loi ou une jurisprudence. Or, la C.Cas, quand elle procède à un revirement de jurisprudence, applique directement. La CEDH condamnera ces revirements plus sévères. C.Cas, assemblée, 2006, la cour accepte de moduler dans le temps un arrêt. En 2009, soit la C.Cas accepte de faire ce qu'enjoint de faire la CEDH, mais si elle le fait, elle réécrit le code pénal et méconnaît l'article 5 du C.Civ, soit elle entre en résistance. La C.Cas décidera plutôt de faire un revirement plus doux. 

\subsection{L'applicabilité de la loi pénale de fond}

\subsubsection{Le principe de non rétro-activité des lois pénales de fond plus sévère}

Le principe est à valeur constitutionnelle, il résulte de l'art. 8 de la DDHC. C'est aussi un principe à valeur conventionnelle, dans la CEDH notamment. Il est encore repris dans le code pénal et irrigue donc l'ensemble du droit pénal. Si il y a infraction complexe ou continue dont une partie a lieu sous l'empire du texte nouveau, alors le nouveau texte s'appliquera. 


Il existe quatre exceptions au principe. \\
Sur les lois interprétatives, elles font corps avec le texte qu'elles interprètent. On en a très peu d'exemples en droit positif. \\
Les lois déclaratives reprennent une règle pré-existante. Il y en a peu aussi. Une loi du 26 Décembre 1964 déclare l'imprescriptibilité des crimes contre l'humanité. \\
Les lois relatives aux valeurs fondamentales, il résulte de l'article 7, §2 qui prévoit que le principe de non rétroactivité de la loi pénale plus sévère ne fait pas obstacle au jugement et à la punition d'actes qui sont contraires aux principes généraux du droit reconnus par les Nations civilisées. \\
Les lois instituant des mesures de sûreté. Il faut distinguer les peines des mesures de sûreté. Les peines servent à sanctionner, à punir, alors que les mesures de sûreté sont elles, destinées à protéger. La mesure de sûreté type est l'interdiction d'exercice d'une profession. Cependant, les deux sont étroitement liées. Selon le CC, une peine est une quelconque sanction ayant caractère de punition. Mais la sureté sera reconnu par le CC dans la Loi instituant des bracelets électroniques de manière rétroactives. Rétention de sûreté: dans l'année qui précède la sortie de prison, on va évaluer la sortie de prison de criminel. Si la dangerosité est élevée, la juridiction va prononcer une rétention de sûreté où le prisonnier va suivre un traitement médico-social, pour deux ans, renouvelables. Dans la mesure où il y a privation de liberté, ce n'est pas une sûreté donc pas de rétroactivité. 

\subsubsection{Rétroactivité de la loi pénale plus douce}

Le principe n'est pas prévu par la DDHC. Mais, selon le CC, le 20 janvier 1981, le principe se dégage à contrario de l'article 8 de la DDHC. Le principe a donc valeur constitutionnel. Il n'est pas non plus prévu par la CEDH mais est déduit par la CEDH et figure à l'article 15 du pacte civil et politique. Le principe est aussi rappelé à l'article 112-1 du code pénal. On a donc là encore un principe cardinal du droit pénal. Il existe aussi un article 112-4 qui prévoit que si la loi pénale plus douce est une loi qui abroge une incrimination, dans ce cas, même si l'affaire a été définitivement jugée, il est prévu que la peine cesse d'être exécuté. \\
Il y a une exception qui concerne la réglementation économique. Par définition, celle-ci est changeante. La C.Cas a donc aménagé le principe. Désormais, si le texte d'incrimination est un décret d'application soit d'une loi soit d'un réglementent de l'UE et que cette loi et ce règlement reste en vigueur, dans cette hypothèse, cette personne sera jugée en fonction du droit existant lors de la commission des faits. 

\subsubsection{Application du caractère plus doux ou plus sévère}

Souvent, il s'agit d'un panachage des deux principes. Dans ce cas, les règles vont différé si les règles édictés sont divisibles ou indivisibles. \\
Critère du lien nécessaire: y-a-t-il un lien nécessaire entre les peines plus douces et plus sévères. Si la loi a un thème générale, c'est un indice d'indivisibilité, alors que si elle contient diverses dispositions, c'est un critère de divisibilité. Ensuite, on peut se demander si il existe une structure interne qui serait un indice d'indivisibilité de la loi. Enfin, on va regarder au fond si les deux aspects sont liés ou si ils sont étrangers. \\
Dans les lois divisibles, les dispositions plus douces rétroagissent, les dispositions plus sévères ne rétroagissent pas. \\
Dans les lois indivisibles, le critère général est celui de l'appréciation globale: est-ce qu'elle est globalement plus douce ou plus sévère ? Si elle est globalement plus douce, elle aura une rétroactivité, si c'est le contraire, ce ne sera pas le cas. Depuis 2001, toutes les lois pénales ont étés plus sévères sauf une exception. 

\subsection{L'applicabilité de la loi pénale de forme}

Le principe est que ces lois pénales sont d'applicabilité directe sauf lorsque ça nuit aux intérêts de la personne concernée. 

\chapter{L'élément matériel}

121-4, définit la responsabilité pénale. Il y a responsabilité lorsque, soit on a commis l'agissement décrit par le texte, soit on a tenté de le commettre, soit on en est complice. 

\section{La tentative}

La tentative est un comportement dangereux. C'est un sujet sur lequel on doit user de criminologie. \\
Étape première: pensée ou représentation du crime. \\
Étape seconde: résolution. \\
Étape trois: actes préparatoires. \\
Étape quatre: commencement de l'exécution. \\
Étape cinq: l'exécution. 


La tentative c'est l'échec de l'infraction, elle est manquée. Infraction impossible, deux tendances existent: il ne faut pas la punir car il est impossible de produire ses effets ; l'auteur d'une infraction impossible a manifesté sa dangerosité: il est dangereux donc il faut le punir. La Loi ne dit rien ou pas grand chose, dans l'ancien code pénal, on incriminait l'avortement sur une femme supposé enceinte. \\
La question a donc été posée à la jurisprudence. Elle a variée. On pensait qu'il était trop sévère de punir totalement, mais pas assez de ne pas punir. On a donc choisi de distinguer dans l'infraction impossible l'impossibilité de droit et l'impossibilité de fait. Dans l'impossibilité de droit, lorsque l'infraction ne peut pas juridiquement produire ses effets, il y a impunité. L'impossibilité de fait c'est que factuellement, l'infraction ne peut pas parvenir à son objectif. Cette distinction n'est que théorique, les deux se mélangeant en pratique. \\
La jurisprudence a adopté la distinction entre l'impossibilité absolue et relative. En cas d'impossibilité absolue, il y a  impunité, en cas de relatif, on assimile à la tentative. Impossibilité absolu: les moyens en eux même sont inefficace. Relative: l'objet de l'infraction existe mais pas là où l'auteur des faits croyaient. C.Crim, 7 Janvier 1986 Perdereau, un individu essaye de tuer un homme qui vient tout juste de décéder: la C.Cas considérera qu'il y a tentative et assimilation totale. 

\subsection{Les conditions de la tentatives punissables}

Deux conditions: commencement de l'exécution et absence de désistement volontaire. 

\subsubsection{Le commencement de l'exécution}

Il n'est pas défini par la loi. C.Crim, 25 Octobre 1962, Lacour, le commencement de l'exécution, ce sont des actes devant avoir pour conséquences directes et immédiates de consommer l'infraction, celle-ci étant entré dans sa phase d'exécution (consécration de la théorie de l'acte univoque). On parle parfois de "point de non retour" ou "point break" en anglais. \\
Le législateur va parfois décider d'incriminer certains actes préparatoires: ce sont des infractions obstacles. Le port d'arme est illégal par exemple. L'association de malfaiteur permet cela aussi, c'est le groupement formé caractérisé par un ou plusieurs éléments matériels en vue de commettre un crime ou un délit puni au minimum de 5 ans d'emprisonnement.

\subsubsection{L'absence de désistement volontaire}

Elle est non défini par la loi non plus. Le désistement est volontaire si il est spontané. Si l'intervention d'un tiers se réalise sans contrainte, le désistement est volontaire. Si il est imposé par la contrainte, le désistement est involontaire et la tentative est punissable. \\
En cas de circonstances extérieures, il faut chercher à savoir si c'est la circonstance qui arrête l'action ou le libre arbitre. \\
C.Crim, 10 Janvier 1996, le commencement de l'exécution peut se composer de plusieurs actes (faisceau d'indice) qui caractérisent un acte univoque.  

\subsubsection{La répression de la tentative}

La tentative d'une infraction est théoriquement puni de la même manière que l'infraction. En pratique, la peine prononcée est moins importante. En effet, le juge doit tenir compte des circonstances de l'infraction. 

\section{La complicité}

Le complice s'associe à l'infraction et commet un acte de complicité. Les co-auteur commettent ensemble tous les éléments constitutifs, le complice aide à réaliser l'infraction. Sous l'ancien code pénal, cette distinction n'était pas respectée, la jurisprudence considérait le complice comme co-auteur pour pouvoir le réprimer. On disait du complice qu'il était co-auteur pour appliquer une circonstance aggravante de réunion qui supposait plusieurs auteurs de l'infraction. L'ancien code pénal n'avait pas prévu la responsabilité pénale de l'instigateur: dire qu'ils sont co-auteur permet de retenir la responsabilité du complice média. \\
Les raisons d'être de ces confusions ont aujourd'hui disparu. Si le nouveau code pénal n'a toujours pas réglé la responsabilité de l'instigateur, la jurisprudence a réglé ce problème. 

\subsection{Les conditions de la complicité}

\subsubsection{L'élément légal de la complicité}

On peut être complice de toutes sortes d'infractions. Le législateur a pris l'habitude d'incriminer des complicités incomplète. 

\subsubsection{L'élément matériel de la complicité}

Élément commun à tous les cas de complicité. Le premier élément est le fait principal punissable: on est donc complice d'une infraction ou d'une tentative d'infraction (la jurisprudence a permis la punition du complice si punir l'auteur est impossible). C.Crim 8 Janvier 2003, la C.Cas a mit fin à la théorie de l'auteur média (l'hypothèse dans laquelle une personne utilise une autre personne de bonne foi, celui qui utilise est l'auteur de l'infraction): l'auteur de l'infraction matérielle, de bonne foi, ne sera pas puni, alors que le "véritable auteur", sera considéré comme complice, disposant de l'intention de commettre l'infraction principal et sera puni. \\
Le deuxième élément est que la complicité repose sur l'exigence d'un acte positif. Il n'y a donc pas de complicité par abstention sauf si le législateur le prévoit. La jurisprudence tempère cette exigence à l'égard des professionnels du chiffre et de la finance: si ils laissent passer des infractions dans les comptes qu'ils supervisent, ils sont complices. Autre exception: l'abstention participative, C.Crim, 21 Janvier 1992, deux personnes se battent, il y a attroupement: tout le monde sera jugé complice. \\
Troisième élément, il faut que l'action du complice ait facilité ou permit la commission de l'infraction. On peut distinguer l'aide morale et l'aide matérielle. Les deux caractérisent un cas de complicité et ne sont pas cumulatifs. \\
Quatrième élément: il faut que l'acte soit antérieur ou concomitant mais on peut se contenter d'un acte postérieur si celui-ci résulter d'un accord antérieur à l'infraction. 


L'art 121-7 distingue quatre cas de complicités: complicité par aide (matérielle ou morale), l'assistance, provocation, instructions. 

\subsubsection{L'élément moral de la complicité}

Toute complicité est nécessairement intentionnel. Difficultés lorsqu'il y a une discordance entre l'infraction imaginé par le complice et celle réalisée par l'auteur. C.Crim Nicolaï, 13 Janvier 1955, dans cet arrêt, un créancier demande à un gorille d'impressionner un débiteur. Le gorille tue le débiteur. L'auteur des instructions voulaient bien être complice de menaces, pas d'un meurtre. La complicité ne sera pas retenu. \\
Cela va être complété par la C.Crim, 21 Mai 1996, où, dans le même cas de figure, le débiteur a demandé à son gorille d'effectuer des menaces légères. Il y a eu aggravation de conséquences et le donneur d'ordre sera jugé complice, ayant ordonné une infraction de même nature à l'aggravation. 

\subsection{La répression de la complicité}

Entre l'ancien et le nouveau code pénal, il y a une révolution juridique. Dans l'ancien, le complice est puni comme l'auteur, maintenant il est puni comme auteur. On est passé de la théorie de l'emprunt de criminalité à l'emprunt de pénalité. \\
Sous l'ancien, le complice s'incarnait comme auteur, dans le nouveau, seul la peine est partagé. \\
Trois catégories de circonstances aggravantes: circonstances aggravantes réelles, qui s'applique indifféremment à l'auteur et au complice ; circonstances aggravantes personnelles, propre à chaque individu (récidive par exemple) ; circonstances aggravantes mixtes, dans l'ancien, elles affectaient la situation du complice, ce n'est plus le cas maintenant. \\
Ces circonstances aggravantes mixtes sont au nombre de trois: le parricide, la préméditation, le guet-apens. 

\chapter{L'élément moral}

\section{Les innovations du nouveau code pénal}

\subsection{L'exigence d'une intention pour les crimes}

Principe intangible, sans exceptions. Tous les crimes sont intentionnels. On ne peut pas commettre un crime par mégarde, imprudence ou en raison d'une faute. Cela vient rompre avec l'ancien code pénal où des crimes pouvaient être commis par faute comme la divulgation par mégarde d'un secret de défense nationale. 

\subsection{L'exigence d'une attention ou d'une faute pour les délits}

Pour les délits, c'était plus complexe dans l'ancien code pénal. Avant, pour les délits, soit il y avait une intention soit il y avait une faute soit aucun élément moral. On allait déduire l'élément moral des termes employés par l'infraction. Par exemple, si elle utilisait le mot sciemment, de mauvaise foie, frauduleusement, en connaissances de causes etc. ces termes suggéraient que c'était des infractions intentionnels. \\
Il pouvait y avoir des délits matériels, où l'élément moral n'était pas requis pour commettre l'infraction. 


Pour les contraventions, il y a un schisme doctrinal. Pour certains auteurs, ils vont estimer qu'il n'y a pas d'élément moral pour les contraventions, et pour d'autres, il existe mais est présumé. Une autre partie de la doctrine estime que ce n'est pas important car dès qu'il y a l'agissement, il y a la responsabilité pénale.


Toutes les infractions pénales ne sont pas dans le code pénal. Dans chaque autre code, on trouve des dispositions pénales. Pour faire en sorte que toutes les infractions pénales soient impactés avec le nouveau CP, il y a eu une loi d'adaptation, celle du 7 Décembre 1992 dont l'article 339 prévoit que désormais, là où il existait un délit matériel, est désormais exigé une faute. 

\section{L'intention et la faute}

\subsection{L'intention}

\subsubsection{Le dol général}

Le dol est le même pour toutes les infractions. On peut parler de dol abstrait ou de dol objectif. C'est la volonté de commettre l'infraction. Il doit être caractérisé systématiquement. Si il existe une erreur de fait portant sur le dol général, cela va faire disparaître l'infraction. \\
La jurisprudence est relativement sévère. C.Crim, 6 novembre 1963 et C.Crim, 2006, dans ces deux cas, pour que le dol soit caractérisée, il faut que la personne ait conscience d'avoir des rapports avec un mineur de 15 ans. La jurisprudence pose comme principe que si il y a discussion, il faut qu'il y ait recherche d'informations sur l'âge de l'individu. 

\subsubsection{Les dols spécifiés}

Il va parfois être exigé en plus du dol général, un dol spécial. Il va varier d'une infraction à l'autre si il est exigé. Pour qu'il y ait meurtre, il faut qu'il y ait l'animus necandi, l'intention de tuer. C'est ce qui différencie les violences du meurtre. S'agissant du vol, son dol spécial était l'intention de se comporter comme propriétaire de la chose. \\
S'agissant du meurtre, on déduit du fait de s'attaquer à des parties exposés du corps humain comme intention de tuer: si il s'attaque à des parties vitales, il a l'intention de tuer. \\
Pour le vol, on peut parler du vol d'usage. La jurisprudence a donc condamné, C.Crim, 19 Février 1959, le dol spécial est redéfini comme l'intention de se comporter, même momentanément comme propriétaire de la chose. 

\subsubsection{Discordances entre le dol et le résultat de l'infraction}

Dol aggravé: raison avec laquelle on a agit. 


Longtemps, le droit pénal a été gouverné par un principe d'indifférence des mobiles, peu importe le mobile qui a animé l'auteur. C.Crim, Gorguloff, 20 Août 1932, il souhaitait non pas faire un meurtre individuel mais commettre un meurtre politique. Ça avait son importance car la peine de mort avait été abolie pour les infractions politiques. La jurisprudence n'a pas cherché si le mobile était politique ou non. 


Pour un certain nombre d'infractions, le législateur va prendre en compte les motivations de l'auteur du délit. Dans une loi de 1986, on nous dit que les infractions sont terroristes, quand il s'agit d'infractions de droit commun avec la volonté d'intimider ou de terroriser des populations. \\
Il y a des circonstances aggravantes fondées sur les mobiles. Elles ont étés dégagés progressivement. La loi du 18 Mars 2003 crée des circonstances aggravantes si les faits sont commis en raison d'un mobile raciste ou homophobe et sont définis strictement (positivement et négativement) dans le code pénal. Des lois plus récentes rendent cela plus large, tellement large qu'on peut en venir à des situations bizarres. \\
Cas d'irresponsabilité pénale: légitime défense, état de nécessité. Journaliste, faisant un article diffamatoire, si il utilise des pièces auxquelles il n'a pas normalement accès pour justifier ses dires, il ne peut pas être condamné pour cela. \\
C.Crim, 11 Mai 2004, le salarié qui vole des documents dans son entreprise afin de se défendre dans une instance prud’homale, ce salarié va être considéré comme étant irresponsable pénalement. \\
Loi du 2 Février 2016, consacre certaines formes d'euthanasie. Elle crée un droit nouveau, celui de la sédation profonde, pour les personnes atteintes de maladies graves, qui peuvent donc demandées à être progressivement endormi. 

\subsubsection{Les autres dols}

Il existe d'autres dols comme le dol indéterminé: l'action est voulu mais les conséquences sont inconnues. Il existe essentiellement à l'égard des violences. Les violences sont volontaires mais les conséquences sont ignorés. Dans ces cas là, c'est non en fonction de la gravité des faits mais des conséquences que la peine est attribuée. \\
Le dol dépassé est l'action voulue et consciente, mais les conséquences sont écartés. Exemple type: violence ayant donné la mort, sans intention de la donner. 

\subsection{La faute}

La faute est étroitement définie par le législateur. 121-3, alinéa 2 à l'alinéa 4. Quatre grandes catégories de fautes. Trois rédactions: issus du nouveau code pénal, rédaction de la loi du 13 Mai 1996, puis celle de la loi du 10 Juillet 2000. \\
La loi du 13 Mai 1996 a été mal rédigée. La C.Crim, juste après, a neutralisée la Loi en disant que le texte était mal écrit et que cela ne changeait rien à sa jurisprudence. Il y a quatre cas de fautes. 

\subsubsection{Le lien de causalité}

Causalité directe, deux types de fautes. Indirecte, deux types de fautes aussi. \\
Il existe deux grandes théories de la causalité. La première est l'équivalence des conditions, toutes les causes ayant précédés la réalisation du dommage. C'est le moyen le plus répressif. \\
La deuxième hypothèse est la proximité des causes, on ne retient que comme cause, la dernière ayant précédée la réalisation du dommage. \\
121-3, alinéa 4, le législateur indique qu'il y a causalité indirecte envers ceux qui ont contribués à la réalisation du dommage et ceux qui n'ont pas pris des mesures permettant de l'éviter. Circulaire 11 Octobre 2000, dans celle-ci, on va dire que si il y a un contact direct entre l'auteur et la victime, on va parler de causalité direct. S'agissant des causes plus éloignés, dans l'espace et dans le temps, la causalité est dite indirecte. \\
La C.Cas dira dans un rapport, que l'auteur direct est l'auteur de la faute essentielle et déterminante. Dans les causes qui ont conduites au dommage, sont les causalités directes. 

\subsubsection{Les catégories de faute}

Les fautes retenus en causalité directe. \\
La faute de mise en danger délibérée. Soit c'est une très grosse faute, soit une série de fautes. Dans ce cas la peine est aggravée. \\
Faute simple, faute d'imprudence, de négligence, ou encore un manquement à une obligation de sécurité affirmé par une loi ou un règlement. C'est la faute la plus facile à déterminée. 


Les fautes retenus en causalité indirecte. \\
La violation manifestement délibérée d'une obligation particulière de prudence ou de sécurité imposée par la loi ou le règlement. \\
La faute caractérisé exposant autrui à un risque d'une particulière gravité que l'auteur ne peut ignorer. \\
Cette faute a des caractéristiques particulières: elle doit être d'une particulière netteté, évidence, et repose sur la violation d'un texte spécifique qui pose une obligation particulière de prudence ou de sécurité. \\
Appréciation de la faute subjective: sous l'empire du Code Pénal, on engageait la responsabilité face à une faute, quel que soit la faute. Aujourd'hui, l'appréciation est dite subjective car on apprécie la faute en l'espèce. La responsabilité pénale des décideurs, qu'elle soit publique ou privée, s'est allégée, alors que la responsabilité des instituteurs ou des médecins s'est aggravées. 

\part{Livre II: Le délinquant}

\chapter{Les régimes de responsabilités}

\section{Le principe de responsabilité pénale personnelle}

Figure à l'article 121-1 du Code Pénal. 

\subsection{L'affirmation et la force du principe}

Le 121-1 annonce que "nul n'est pénalement responsable que de son propre fait", c'est une innovation du nouveau code pénal. Le CC, 16 Juin 1999, consacre ce principe et lui confère valeur constitutionnelle. Le 121-1 revêt un caractère purement dogmatique, théorique, les implications n'y sont pas mentionnés. Initialement, cet article a plusieurs alinéas, il devait avoir un alinéa 2 consacré à la responsabilité de l'instigateur et un autre du chef d'entreprise, or, sans consensus, il est resté à énoncé ce principe, revêtant un caractère constitutionnel. \\
Il n'y a donc pas de responsabilité pénale d'autrui (les parents ne sont pas responsables pénalement de leurs enfants). 

\subsection{Les applications du principe}

Si il existe une responsabilité pénale personnelle, c'est que le code exclut toutes formes de responsabilité pénale collective, ainsi que du fait d'autrui. 

\subsubsection{Exclusion de la responsabilité pénale collective}

L'affirmation de cette exclusion résulte de plusieurs décisions. C.Crim, 17 Décembre 2002, hypothèse d'une ville se dotant d'un nouveau maire décidant d'appliquer le programme de son parti, prévoyant d'ailleurs une prime de naissance consistant en une discrimination ; si il existait une responsabilité collective, tous les membres du conseil municipal ayant voté favorablement pourrait être poursuivi, or, seul le Maire et son adjoint ayant fait voté la mesure et appliqué peuvent être poursuivis. \\
Faute pénale commune: se retrouve dans l'hypothèse où une faute est partagée par tous, l'ensemble des participants a une responsabilité pénale. Scène unique de violence: lorsqu'il y a sur le même lieu plusieurs personnes dont certaines ne font rien. Ceux qui ne font rien sont aussi condamnés. 

\subsubsection{Exclusion de la responsabilité pénale du fait d'autrui}

Parfois, en droit positif, il existe des présomptions de droit ou de faits qui conduisent à faire peser sur une personne une forme atténué de présomption de responsabilité. 


Présomption de fait: utilisation de la méthode inductive ou indiciaire. On va prendre un certains nombre d'éléments factuel et en induire que telle personne a réalisée les faits. La présomption de fait est une oeuvre de l'esprit.


Présomption de droit: il existe en droit positif un certains nombre, relativement élevé, de présomptions de droit. Par exemple, le code de la route prévoit que le conducteur est pénalement responsable, mais ce même code indique par la suite un certain nombre de présomptions qui posent l'exigence selon laquelle le titulaire du certificat d'immatriculation est présumé responsable pécuniairement d'un certain nombre d'infractions qui sont listés (stationnement, dépassement de vitesse, etc). En cas de force majeur, la présomption tombe (comme un vol de voiture), un autre alinéa prévoit que le propriétaire peut rapporter la preuve qu'il n'est pas l'auteur de l'infraction. Dans ce dernier alinéa, y-a-t-il une obligation de délation ? Le législateur a tranché et pose une distinction: si le titulaire du certificat d'immatriculation est une personne physique, on peut se contenter d'une preuve négative (qu'il n'est pas l'auteur, donc pas d'obligation de délation), par contre, si le titulaire est une personne morale, dans ce cas, celle-ci devra fournir l'identité du véritable auteur de l'infraction ou la responsabilité va peser sur son représentant légal. \\
On ne peut pas, par analogie, étendre ces présomptions à d'autres infractions. 


Ces présomptions sont-elles conformes aux exigences des droits fondamentaux ? La CEDH, le CC, la C.Cas se sont penchés sur la question. \\
La CEDH, Salabiaku contre France, 7 octobre 1988, conformité de ces présomptions par rapport à la présomption d'innocence. La CEDH va accepter de façon limité ces présomptions. D'abord, elles sont admises uniquement si celles-ci sont prévues par la loi, qu'elles préservent les droits de la défense, et qu'elles soient volatiles. La CEDH prend en compte la gravité de l'enjeu. \\
Doit préserver les droits de la défense: facilité de la possibilité de s'exonérer de ces activités. Préserve les droits de la défense dans le sens où ces présomptions doivent être sur des sujets minimes: les contraventions. \\
Le CC indique que ces présomptions doivent reposer sur la vraisemblance de l'imputabilité. C'est à dire qu'elles ne peuvent peser que sur une personne qui est habituellement l'auteur de l'infraction. CC, 10 Juin 2009, le CC va se pencher sur les présomptions de la loi HADOPI, que le CC va censurer, le titulaire de l'accès internet n'est pas forcément l'auteur de l'infraction, de plus, les présomptions sont contraires aux lois de la défense. 


Les présomptions de droit ou de fait s'applique indifféremment aux personnes morales ou aux personnes physiques. 

\section{La responsabilité pénale du chef d'entreprise}

Ce système de responsabilité est prétorien, il s'applique au sens d'un dirigeant de droit au sein d'une structure privée, en tant qu'employeur, etc. Le chef d'entreprise englobe beaucoup de choses. Cette responsabilité s'est construite autour de plusieurs théories de la fin du XIXe et du début du XXe. \\
Théorie du risque et théorie du profit: comme c'est le chef d'entreprise qui gère son entreprise, il fait courir des risques à son entreprise donc la responsabilité lui incombe. Le chef d'entreprise bénéficie financièrement du travail de ses salariés donc il est logique qu'il endosse la responsabilité de leurs agissements. \\
Le code pénal ne condamne pas la responsabilité et ne la consacre pas non plus. 

\subsection{Les conditions d'engagement de la responsabilité pénale du chef d'entreprise}

\subsubsection{Une infraction commise un préposé}

L'infraction doit être une infraction non intentionnelle. Il y a des décisions où malgré que l'infraction soit intentionnelle, on engage la responsabilité du chef d'entreprise. Au moment où cette responsabilité a été formée, il n'y avait pas de distinction intentionnelle ou non, c'était au juge de le décider. \\
Le préposé et le chef d'entreprise peuvent être poursuivis tous les deux. 


Le chef d'entreprise doit avoir commis une faute. Si le salarié commet une infraction, c'est que le chef d'entreprise est nécessairement fautif, depuis les années 70, cette présomption est irréfragable. Le 6 Décembre 1976, le législateur est intervenu pour rappeler l'exigence d'une faute personnelle, ce qui a atténué la présomption. \\
Il y a chez le chef d'entreprise, l'exigence d'une faute qui est présumée. \\
Loi du 10 Juin 2000, on adopte une distinction entre causalité directe et causalité indirecte: la causalité à l'égard du chef d'entreprise est nécessairement indirect parce que entre le chef d'entreprise et le dommage est interposé le salarié. En cas de causalité indirecte, pour engager la responsabilité pénale, il est nécessaire qu'il y ait une faute plus grave: violation d'une obligation en matière de prudence ou de sécurité imposée par une loi ou le règlement soit une faute caractérisé exposant autrui à un risque d'une particulière gravité que l'auteur ne peut ignorer. \\
L'avènement de cette loi ne peut pas avoir impacté les règles prétoriennes qui préexistait. Aujourd'hui, depuis la Loi de 2000, la faute ne suffit plus à engager la responsabilité pénale. \\
Première conséquence: cela ne concerne que la responsabilité pénale, la responsabilité civile peut l'être. La responsabilité pénale de l'entreprise peut tout de même être engagée. 

\section{La responsabilité pénale des personnes morales}

C'est une innovation du nouveau code pénal. Les ordonnances du 5 Mai 1945 ont fait une exception, reconnaissant la presse française coupable d'avoir collaboré avec l'ennemi, ce qui permettait de purger les journaux français. 

\subsection{Le champ d'application des personnes morales}

\subsubsection{Une responsabilité étendue à presque toutes les personnes morales}

Pour être pénalement responsable, il faut être une personne morale. Les art. 1871 et 1873 du C.Civ nient la personnalité juridique à certaines formes de structures. 


Les personnes morales de droit privé (Société, Association). \\
Elles sont en principe pénalement responsable. Un certain nombre de choses peuvent affecter cela. \\
La personne morale de droit privé connaît une période de constitution plus ou moins longue qui est propice à la commission d'infraction. La personnalité juridique commence au moment de l'immatriculation, ce qui se passe avant est propice aux infractions. Il faut trouver avant cela, un nom d'entreprise, un siège social (corruption pour une adresse prestigieuse), un capital social (recel, argent sale). Dans ces cas là, ce n'est pas la personne morale qui est responsable, mais la personne morale, une fois constituée, peut être retenu comme responsable des infractions dites de conséquences liées aux infractions commise pendant la période de constitution. \\
Concernant la période de mutation, il y en a plusieurs possibles comme la fusion. Si une des entreprises a commis une infraction avant la fusion, elle ne peut pas être tenue responsable car est étrangère. Concernant la fusion absorption, la société qui a absorbé l'autre ne peut pas être tenue pour responsable des infractions de la société qu'elle a absorbé. Concernant la disparition de la société, dans ce cas, à la fois le C.Civ (1844-8) et le CP (133-1), prévoient une survie de la personnalité juridique donc il y a une survie de la responsabilité pénale durant la période de liquidation. 


Les personnes morales de droit public. \\
L'État est irresponsable pénalement, les autres collectivités ont une responsabilité limité. Cela car l'État détient le monopole de la force publique et ne peut pas s'appliquer à lui même une sanction. La justice est de plus, donnée au nom de l'État. 



















\end{document}
