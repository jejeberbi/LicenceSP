\documentclass[10pt, a4paper, openany]{book}

\usepackage[utf8x]{inputenc}
\usepackage[T1]{fontenc}
\usepackage[francais]{babel}
\usepackage{bookman}
\usepackage{fullpage}
\setlength{\parskip}{5px}
\date{}
\title{Cours de Droit Pénal Général (UFR Amiens)}
\pagestyle{plain}


\begin{document}
\maketitle
\tableofcontents

\chapter{Introduction}

\section{La notion de droit pénal}

\subsection{Le droit pénal, un droit de punir}

Pour affirmer cela, il suffit de prendre la racine du mot pénal, pena, qui signifie punir. Le droit pénal est donc le droit de la sanction. Ce droit désigne donc ce qui est autorisé de faire et ce qu'il est interdit de faire et va renvoyer à la peine si infraction il y a. \\
Cette peine est un mal infligée à une personne et doit être proportionnée au mal qui a été fait à la société. Si un individu commet donc une infraction, on inflige une peine selon l'acte commis et non selon la dangerosité. \\
Le droit pénal a une fonction intimidatrice. Il faut qu'il fasse peur pour inciter à ne pas commettre d'infractions. \\
La prison a un objectif de rééducation, de réintégration, de reclassement social. Dès qu'un individu arrive en prison, on doit préparer sa sortie et donc aménager sa peine de sorte à former voir guérir celui qui la subit. 


En réalité, en matière criminelle, la prison est fréquemment utilisée, pareil en matière délictuelle. Or, les prisons sont surpeuplés. \\
La première difficultés est que cette surpopulation est génératrice de phénomènes: violence entre détenu et pour le personnel des prisons. Différentes études qui ont eu lieu pour réduire la récidive montrent que la courte peine est inefficace. 

\subsection{Le droit pénal général et les autres branches du droit}

Par rapport aux autres branches du droit, le droit pénal général a des mécanismes différends. Le droit pénal général renvoie souvent aux rapports inter-individuels, c'est pour ça qu'on rattache le droit pénal au droit privé. Cependant, il existe des infractions commises contre le bien public: évasion fiscale, corruption... \\
Le droit pénal, de façon générale est le droit sanctionateur des autres droits. \\
Il existe tout un tas de branches qui sont reliés au droit pénal: du droit pénal international, de la criminologie etc.

\section{Les caractères du droit pénal}

\subsection{Les sources historiques}

\subsubsection{Les sources de l'ancien régime}

Le droit pénal est le premier droit qui s'est formé. On trouvera du droit pénal dans les formes primitives de sociétés humaines. Dans la société primitive, lorsqu'est commise une infraction, celle ci donne d'abord lieu à une vengeance privée. Progressivement, on va avoir un passage à la justice publique car on se rend compte que les guerres sont épuisantes et que tout dommage ne nécessite pas nécessairement d'en déclencher une. \\
La première peine publique à exister est l'amende, permettant de dédommager la victime. Lorsqu'elle ne peut pas être payée, on séquestre la personne jusqu'à ce que sa famille paye pour lui, c'est la création de la prison. \\
La justice publique va ensuite s'affirmé et développé un certain nombre de moyen et la période du moyen âge va être marqué par l'arbitraire des juges. Il y a donc une grande disparité des peines sur tout le royaume selon la personnalité des juges. \\
À la veille de la révolution, le système de preuve est très critiquée. Après les ordani, ce sont les duels judiciaires qui permettent de prouver innocence ou culpabilité. On se dit finalement que la seule preuve parfaite est l'aveu. C'est ainsi que la torture se développera. À la veille de la Révolution, tout ces aspects sont critiqués. Beccaria propose une réforme du droit pénal dans "Traité des délits et des peines" en 1764, il prône la disparition de la torture et de la peine de mort, et introduit le principe de légalité. 

\subsubsection{Du droit révolutionnaire à l'ancien code pénal}

Le droit révolutionnaire va s'inspirer du droit de Beccaria. Il y a une définition précise des infractions, liés à une peine. Le rôle du juge n'est donc que de prononcer la peine si l'infraction a été constatée. \\
Ce droit est assez peu efficace. Napoléon va donc créer l'ancien code pénal qui entrera en vigueur en 1811. Celui ci s'inspire essentiellement des idées de Benthan qui défend que le code pénal doit inspiré la crainte. Cette théorie s'appelle la théorie de l'utilitarisme. Cependant le législateur prévoit pour chaque infraction une peine minimale et une peine maximale. Le juge doit prononcer la peine dans cette fourchette. Sont prévues des circonstances atténuantes et des circonstances aggravantes permettant d'aller au delà ou en deçà de la peine légale. \\
L'ancien code pénal maintient trois peines corporelles. 

\subsubsection{De l'ancien code pénal au nouveau code pénal}

Les premières réformes vont faire disparaître, en 1832, les peines corporelles. En 1891, on instaure le sursis. C'est une possibilité pour le juge qui constate l'infraction et qui condamne la personne de suspendre l'exécution de la peine. Cette exécution peut se faire si l'individu revient devant le juge. \\
Dès la fin du XIXe siècle, le code pénal ne contient pas tout le droit pénal. De nombreux textes, codifiés ou non, font leur apparition et prévoit des sanctions pénales voire des régimes de responsabilité spécifique. La loi du 29 Juillet 1881 par exemple, sur la liberté de la presse, prévoit de nombreuses sanctions pénales comme la diffamation ou l'injure et qui n'apparaisse pas dans le code pénal. C'est aussi le cas d'une ordonnance de 1945 sur les enfants délinquants qui n'a jamais été codifiés. \\
Le code pénal va vieillir en raison des inspirations de la société. Juste après la seconde guerre mondiale est créé un courant de pensé "La défense sociale nouvelle" dont le chef de file est Marc Ancel qui défend que le délinquant est un malade social. La peine doit donc guérir le délinquant de sa maladie. \\
Le législateur va dans des sens contradictoire. Par exemple, en 1981, la loi sécurité et la loi sur l'abolition de la peine de mort font contraste.

\subsubsection{Depuis le nouveau code pénal}

L'idée d'un nouveau code pénal est ancienne. La première fois que l'on en parle, c'est en 1930, or la guerre arriva. Après la seconde guerre mondiale, on se concentre sur le code de procédure pénale. \\ 
Un avant-projet est présenté en 1978. Puis en 1981, c'est l'alternance et le projet est abandonné. On représentera quelque chose en 1986, or, l'assemblée subira une alternance. \\
Un consensus est trouvé et en Juillet 1992, le nouveau code pénal est adopté et entrera en vigueur en 1994. L'inconvénient du consensus est qu'il manque certains éléments dans le nouveau code pénal. Pour l'entrée en vigueur, c'est une bonne chose qu'il soit retardé de deux ans car permet de former les juristes avant d'appliquer les règles. \\
Même après l'adoption du nouveau code pénal et avant son entrée en vigueur, le nouveau code pénal est réformé plusieurs fois. Depuis 20 ans, le nombre de lois en matière pénale est extrêmement dense. La loi du 10 Juillet 2000 redéfinit la faute pénale, en 2003, une loi sur la sécurité intérieure ; en Décembre 2005, loi généralisant la possibilités d'imposer un bracelet électronique aux prisonniers ; 5 lois de lutte contre la récidive entre 2005 et 2010 ; instauration des peines planchers en 2007 ; loi pénitentiaire en 2009 ; loi du 6 Août 2012 redéfinissant le harcèlement sexuel ; loi du 15 Août 2014 créant la contrainte pénale ; loi du 3 juin 2016 renforçant la lute contre le terrorisme. \\
39 lois ont étés produites depuis l'adoption du nouveau code pénal. Ce grand nombre de lois provoque une insécurité juridique. Le mouvement législatif n'est pas néfaste en lui même, cependant un grand nombre de ces lois sont soit inutiles, soit maladroites, et sont souvent des lois de réactions. 

\subsection{Les sources formelles}

Dans les sources formelles, on pense de suite au code pénal, cependant, ce n'est pas la seule source. 

\subsubsection{Sources internationales}

C'est par exemple une convention du 9 Décembre 1948 qui va définir le génocide. Le traité de Rome crée la CPI en 1998, chargé en vertu de ce traité, de poursuivre les auteurs de crimes internationaux. 

\subsubsection{Droit de l'UE}

En 1957, on ne trouve pas de traces du droit pénal dans les traités. Progressivement, le droit de l'UE est devenu une source directe du droit pénal. Dans les domaines où il était nécessaire d'avoir un réglementation, une réglementation a été adoptée. Par exemple, en droit pénal de la pêche, en raison des réglementations internes, il y avait entre pêcheurs espagnols et français des filets de maillage différents, ce qui avait pour effet de produire des différences de quantité pêché, ce qui a été réglementé par l'UE. Depuis le traité de Lisbonne en 2009, le législateur en est venu à transposer des directives de droit européens. \\
Parfois, le droit de l'Union produit un effet de neutralisation ou d'appel du droit pénal. Pour le premier, lorsque la norme pénale s'apparente à une mesure que l'on peut considérer comme étant discriminatoire par rapport aux citoyens de l'UE, la norme européenne va neutraliser cette norme pénale. Pour le second, on peut relever trois types d'exemples, en matière de corruption, blanchiment ou mutilation sexuelle. Il n'existait aucune norme pénale réprimant la possibilité de sanctionner un fonctionnaire communautaire corrompu, on a créé une norme par effet d'appel. 

\subsubsection{Le droit de la CEDH}

La CEDH a une influence directe sur le droit pénal. Elle a été adopté en 1950 et sont liés par la convention, 57 États, donc presque toute l'Europe, sauf la Biélorussie. La Cour Européenne des droits de l'homme, lorsqu'elle interprète la convention, elle crée certaines innovations juridiques. \\
L'article 2, relatif au droit à la vie, prévoit également un certain nombre de tempérament avec des protocoles additionnels, qui, aujourd'hui, prohibent totalement la peine de mort. \\
L'article 3 prohibe les mauvais traitements, inhumains ou dégradants et la torture. Il est à noter que la France a été le deuxième pays à être condamné sur la base de cet article en raison d'interrogatoire en garde à vue car l'individu qui est passé en garde à vue a été passé à tabac. La France a également été condamné pour torture car une personne en garde à vue a été plus que passé à tabac. \\
Reynolde contre France, CEDH, 2008, un détenu faisant une tentative de suicide a été sauvé par un surveillant qui a été blessé dans la manoeuvre. Le prisonnier a été envoyé en quartier disciplinaire et sera retrouvé pendu. La CEDH considerera une atteinte à l'article 2 et 3. \\
L'article 4 prohibe le travail forcé. Les article 5 et 6 concernent la procédure pénale. L'article 7, lui, est particulièrement important. Il prévoit le principe de légalité des crimes et délits ainsi que le principe de la non rétroactivité. \\
L'article 4 du protocole additionnel 7, veut qu'un individu ne peut pas être jugé deux fois pour les mêmes faits

\subsubsection{Le bloc de constitutionnalité}

Il a une influence en droit pénal. La DDHC est plus importante sur le droit pénal car pose en ses article 5, 7 et 8 également de principe de légalité, de proportionnalité des peines et de non rétro activité de la loi pénale plus sévère.

\subsubsection{Le code pénal}

Il contient l'essentiel des règles de droit pénal. Il contient l'essentiel du droit pénal, excepté l'ordonnance de Février 1945 sur la délinquance des enfants. \\
Le législateur a adopté un numérotation décimale, le premier chiffre correspond au livre, le deuxième au titre, le troisième au chiffre. \\
Le livre I contient les dispositions générales, le livre II contient les atteintes aux personnes, le III contre les biens, le IV contre la chose publique, le V contient des règles de bio éthiques et de droit animal. \\
L'innovation majeure de ce nouveau code pénal est l'instauration de la responsabilité pénale des personnes mortes. Il supprime aussi les peines minimales, et donc aussi les circonstances atténuantes. Il codifie aussi certaines jurisprudences, notamment l'état de nécessité. 

\section{Les classifications des infractions}

\subsection{La distinction en crime, délit et contravention}

\subsubsection{L'énoncé de la distinction}

On repose sur l'article 111-1 du code pénal, qui dispose que selon leurs gravités, les infractions sont classés en crime, délit, ou contravention. Les crimes sont les infractions qui sont punis de la réclusion ou de la détention criminelle soit à perpétuité soit 30 ans, soit 20 ans, soit 15 ans. Le délit est puni d'une peine d'emprisonnement inférieur ou égal à 10 ans et/ou une peine d'amende supérieur ou égal à 3750 euros. \\
Les contraventions sont divisés en cinq classes de contraventions. Il n'y a qu'une peine d'amende qui va de 38 euros pour la première classe à 1500 euros pour la cinquième classe.

\subsubsection{L'intérêt de la distinction}

Le législateur est compétent pour définir les crimes et les délits. Le règlement est compétent pour définir les contraventions. La tentative d'un crime est toujours punissable. La tentative d'un délit n'est punissable que si la loi le prévoit, mais la tentative d'une contravention n'est jamais punissable. \\
Tout crime repose sur une intention. En principe, l'élément moral d'un délit est aussi une intention, toutefois, lorsque la loi le prévoit, l'élément moral du délit peut être une simple faute. Enfin, il n'y a pas d'élément moral pour les contraventions. \\
Toute la procédure pénale repose sur cette distinction. Les juridictions compétentes ne sont pas les mêmes. 

\subsection{La distinction entre infraction de commission et d'omission}

Soit l'infraction réside dans le fait de punir un acte positif, perceptible, soit réside dans le fait de punir une abstention. Il existe essentiellement des infractions de commission. Il y a quelques infractions d'omission, comme la non assistance à personne en danger, ou délaissement de mineur. \\
L'intérêt est de dire qu'il n'y a pas d'infraction de commission par omission. Si les effets d'une omission s'apparent à une commission, ce n'est pas pour autant qu'elle est punissable. Dans la décision de la CA Poitiers, 1901, la CA a puni une forme d'omission. \\
Dans un arrêt de la C.Cass, C.Crim, 2 septembre 2005, la cour dira que la présence de personnes dans un combat participe à l'affaiblissement psychologique de l'adversaire. Sur la complicité, il existe un principe d'abstention participative. Si des personnes se battent et qu'un attroupement a lieu, alors les spectateurs peuvent être considérés comme complices. \\

\subsection{La distinction entre infraction simple et infraction complexe}

Il y a plusieurs intérêts à faire cette distinction. S'agissant d'une infraction complexe, si, entre le début de l'infraction et le début de l'infraction, est adoptée une loi nouvelle, même plus sévère, cette loi va s'appliquée à l'infraction. \\
Tous les tribunaux dans les ressorts desquels une infraction complexe a été commise sont compétents. Le délai de prescription, c'est le délai entre le début de l'infraction et le début de l'action en justice. Le point de départ du délai de l'infraction est lorsque l'infraction est terminée.

\subsection{La distinction entre infractions instantanées et infractions continues}

Le vol est une infraction instantanée, elle se fait en un instant. \\
L'infraction continu, continu dans le temps, avec donc, la réitération de la volonté du coupable. Comme la séquéstration.


Si une loi plus sévère est adoptée, elle s'appliquera à l'infraction continue. Tous les tribunaux dans lesquels l'infraction est faite sont compétent, et le départ du délai de prescription est la fin de l'infraction. 

\subsection{La distinction entre infractions simples et infraction d'habitude}

Les infractions simples sont lorsque le législateur incrimine une action ou une omission. L'infraction d'habitude est lorsqu'il y a répétition dans le temps, par exemple l'exercice illégal de la médecine car l'exercice pré suppose une continuité dans le temps de l'exercice. \\
Le point de départ du délai de prescription est lorsque l'habitude prend fin. 

\subsection{La distinction entre infraction formelle et infraction matérielle}

Cette distinction suppose de regarder ce que fait le législateur. Soit le législateur prévoit que l'infraction est réprimée si on parvient à un certain résultat. Soit le législateur incrimine le cheminement vers un résultat, sans prendre en compte celui-ci. La première est une infraction matérielle, la deuxième une infraction formelle. \\
Dans le deuxième cas, l'hypothèse type est l'empoisonnement, et l'infraction est commise dès que l'on a administré la substance, sans s'occuper de si la personne est morte ou non. \\
La tentative d'infraction matérielle intervient juste avant la commission de l'infraction.

\subsection{La distinction entre infraction de criminalité et de bande organisée et de droit commun}

Avant 2004, on évoquait une distinction entre les infraction terroristes, et les autres. Aujourd'hui, les infractions terroristes sont des infractions de criminalité et de délinquance organisée. 706-73 du CPP dresse une liste d'infractions qui sont soit particulièrement graves qui reposent toutes sur un réseau comme le terrorisme, mais aussi le trafic de stups, le blanchiment d'argent. \\
La bande organisée est défini dans l'article 132-71 qui dispose "La bande organisée est le groupement formée ou l'entente établi caractérisée par un ou plusieurs faits matériels en vu de commettre une infraction". À noter que la bande organisée commence à trois personnes. \\
Depuis 2004, des textes se sont multipliés pour renforcer la singularité du terrorisme. \\
Le 9 Octobre 2014, le CC indique qu'il va falloir distinguer les infractions qui sont contre les personnes ou celles qui sont contre les biens. 


S'agissant de l'ensemble des infractions de bande organisée, il existe des moyens de preuve supplémentaires. Il y a par exemple, la possibilité de pratiquer des écoutes et des enregistrements. \\
On va également avoir la possibilité de pratiquer une infiltration. \\
Celui qui va empêcher de commettre l'infraction et qui fait partie de la bande pourra être exempté de peine. Celui qui minimisera les conséquences, aura une peine réduite de moitié. \\
Par rapport au terrorisme, il suppose aussi des moyens de preuve encore plus dérogatoire, dont un, les IMSI Catcher. \\
S'agissant de la GAV, il y a plusieurs délais. Celui de droit commun est de 48h. En criminalité organisée contre les bien, il en est de même. Cependant, le délai de GAV pour de la criminalité organisée contre les personnes, il est de 96h, et en terrorisme, il est de 6 jours. 

\part{Livre I: l'infraction}

\chapter{L'élément légal}

\section{Le principe de légalité}

\subsection{L'existence d'un texte}

Pour qu'il y ait une condamnation, il faut qu'un texte préalable existe. \\
Par exemple, dans le cas du "revenge porn", le législateur n'a jamais prévu de tel cas et a du donc le régler. Cela est de même avec le cas de la zoophilie.


\subsubsection{L'affirmation du principe}

Le principe de légalité a une valeur constitutionnelle, il résulte de la DDHC (Art. 5, 7 et 8). Il est repris dans les article 34 et 37 de la Constitution. \\
Le CC, le 28 Novembre 1973, va interprété l'article 8 de la DDHC, qui prévoit que "nul ne peut être détenu que dans les cas déterminés par la loi", or, à l'époque, certaines contraventions prévoient des peines privatives de libertés. Le CC dira que le pouvoir réglementaire peut définir les contraventions mais ne doit pas prévoir de peines privatives de libertés. \\
La C.Cas et le CE ont donc appliqué la théorie dite de la loi écran. Ils ne sont pas juges de la constitutionnalité des lois, et ne peuvent qu'appliquer les règlements. En conséquence, cette décision va rester lettre morte jusque dans le nouveau code pénal. Le pouvoir réglementaire est désormais compétent pour définir les contraventions mais ne peut plus leur attribuer de peines privatives de libertés. \\
Aujourd'hui, le problème se pose dans l'autre sens. La loi a créée une contravention. Le CC devrait donc en logique censuré cette loi. Cependant, au cours du processus législatif, lorsqu'il s'agit d'un PJL, le Gouvernement peut s'opposer à tout moment à l'adoption d'un texte qui empiéterait à ses prérogatives, et si il ne le fait pas, il est considéré qu'implicitement, il autorise l'empiétement du pouvoir législatif sur son pouvoir réglementaire. Cependant, dans notre cas, c'est une PPL, proposé par un parlementaire devenu membre du Parlement, et il reste à savoir si le CC censurera ou non. \\
Le principe de légalité est également prévu à l'article 7 de la CEDH. Il est encore rappelé dans le code pénal.

\subsubsection{Signification du principe}

Pour le législateur, exiger un tel principe peut paraître assez simple, une loi pour un crime ou un délit, un règlement pour les infractions. \\
Pour qu'une loi puisse prévoir une infraction, il faut que la loi ait une certaine qualité. CEDH, "Sunday times c/ Royaume-Uni", 26 Avril 1979, la cour estime qu'une loi pénale doit être de qualité: précise, prévisible, et accessible. Précise car il ne faut pas qu'il y ait le moindre doute sur le comportement à adopter ou à ne pas adopter, prévisible pour savoir ce que l'on encourt, accessible car tous les justiciables doivent comprendre la Loi. \\
Pendant longtemps, une règle de common law, au RU, prévoyait que le viol sur son épouse n'était pas constitutif d'une infraction. Dans les années 80, le juge change la règle et l'applique immédiatement. La CEDH dira que la règle était prévisible car elle estime que les moeurs ont évolués. \\
CEDH, 15 Novembre 1996 "Cantony c/ France", il fallait savoir si un produit était un médicament au sens de la Loi. Le code de la santé publique prévoyait plusieurs définitions, par renvois. La question était de savoir si la norme était facilement accessible, la CEDH dira que oui. \\
On a commencé à utiliser ce principe en droit interne, en regard de l'article 10 de la CEDH, qui prévoit la liberté d'expression. Il faut, pour limiter la liberté d'expression, un but légitime, que ce soit nécessaire, et que la limitation soit prévue par la Loi. \\
Une loi de 1931 interdisait de faire état d'une plainte avec constitution de partie civile. CEDH "Du roy et Malorie c/ France", 3 Octobre 2000, va indiquer que cette infraction là ne se justifie plus dans une société démocratique. Le 16 Janvier 2001, la C.Cass refuse d'appliquer le texte car la CEDH avait dis que le texte n'était pas légitime, et enfin, le législateur abrogera le texte. \\
Une infraction de 1881, interdisait de faire état des circonstances d'une infraction. C'était la deuxième fois qu'on faisait état de cette infraction. La première fois était dans l'affaire dite du japonais cannibale et la deuxième fois dans l'affaire de l'attentat du RER en 1995. La C.Cas refusera d'appliquer le texte car la loi aurait été trop imprécise. Depuis, la loi a été revu par le législateur. \\
Aujourd'hui, n'importe quelle loi peut être écartée par le juge en invoquant la CEDH. 


À partir de l'article 8 de la DDHC, le CC a estimé que toute infraction pénale doit être clairs et précise. \\
De nombreuses lois pénales n'ont pas été soumis au CC, ce pour des raisons politiques. Le CC n'a pas été saisi par exemple pour les quatre lois qui forment le nouveau code pénal. \\
Par exemple, une loi de 2010 inscrit l'inceste dans le code et on adopte une définition très large. 16 Septembre 2011, et 17 Février 2012, dans ces deux décisions, le CC constate la définition et cette définition est trop large. Le texte est donc abrogé. Le 8 Février 2016, une loi plus précise a été adoptée. \\
Le 17 Janvier 2002, une loi est adoptée pour définir le harcèlement sexuel. Le CC dira le 4 Mai 2012 que le harcèlement n'est pas suffisamment défini, la loi est donc abrogée. L'abrogation a conduit à ne plus pouvoir juger les gens pour harcèlement. Pendant quelques mois, l'infraction a été remplacée ; au lieu de poursuivre pour harcèlement, on poursuivait pour agression sexuelle ou viol, harcèlement moral, exhibition sexuelle, violence morale ou psychologique.


L'article 111-4 du code pénal dispose que la loi pénale est d'interprétation stricte. L'interprétation analogique n'est pas possible, sauf dans une hypothèse se fait dans une façon favorable par rapport à la personne poursuivie. \\
La jurisprudence indique régulièrement que l'interprétation doit avoir un sens clair et strict du texte. Le foeticide est la façon dont la C.Cas s'est prononcée sur le commencement de la vie.


































\end{document}
