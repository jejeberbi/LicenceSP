\documentclass[10pt, a4paper, openany]{book}

\usepackage[latin1]{inputenc}
\usepackage[T1]{fontenc}
\usepackage[francais]{babel}
\usepackage{bookman}
\usepackage{fullpage}
\setlength{\parskip}{5px}
\date{}
\title{Cours de Droit Administratif (UFR Amiens)}
\pagestyle{plain}

% GALOP SAMEDI 12 NOVEMBRE 14h - 17h
% "Livre de chevet": LES GRANDS ARRÊTS DES JURIDICTIONS ADMINISTRATIVES

\begin{document}
\maketitle
\tableofcontents

\chapter{Introduction}

Le droit administratif est loin du cliché qu'on peut avoir. \\
Le DA est une relation entre l'Administration et d'autres personnes juridiques, qu'elle soit publique, privé, ou morale. Le DA vise l'intérêt général. C'est un droit de conciliation entre l'intérêt général et les libertés individuelles. Le DA est la matière qui permet de savoir, par exemple, si on peut suspendre la dissémination d'OGM dans une commune ; on lie dans cet exemple la liberté d'entreprendre de l'agriculteur et l'intérêt général. Comme le DA est un droit d'intérêt général, c'est un droit éminemment politique. \\
Le DA est un droit théorique car certaines questions ne sont pas résolues. En l'absence de principes de droit, c'est à la doctrine de trouver des solutions. \\
Le DA est un droit important dans le droit public et contrairement en droit privée ou en droit des affaires, c'est un domaine qui est moins prisé des avocats et juristes. \\
Pour certains, le DA serait inintelligible et très complexe. Cette critique est certainement vrai. On notera que c'est une matière qui s'est complexifié récemment. 


Bibliographie:
\begin{itemize}
\item Les Grands Arrêts des Juridictions Administratives - Dalloz ;
\item Le traité de droit administratif - Dalloz, 2 tomes ;
\item Manuel de droit administratif - Plessix ;
\item Manuel de droit administratif - Lombard, Sirinelli ;
\item Manuel de droit administratif - Truchet ;
\item Actualité Juridique du Droit Administratif - Dalloz, hebdomadaire ;
\item Revue Française de Droit Administratif ;
\item Revue de Droit Administratif - Lexis Nexis, mensuel.
\end{itemize}


À priori, on pourrait penser que le DA est le droit de l'administration. Cette définition est incomplète car il faut se demander ce qu'est l'administration. Ce n'est qu'après avoir défini l'administration qu'on pourrait envisager le droit administratif et le juge administratif.

\section{L'administration}

C'est un terme polysémique, on peut le voir d'un point de vue matériel et d'un point de vue organique. \\
D'un point de vue matériel, l'administration est une activité qui consiste à gérer une affaire. Du point de vue organique, c'est l'organe qui gère cette affaire. Ce terme peut très bien s'utiliser en droit privé comme le Conseil d'Administration d'une entreprise. C'est pour cela que nous parlerons, nous, d'administration publique, l'organe qui gère les activités publiques.

\subsection{L'administration et l'action des particuliers}

L'activité publique se distingue de l'activité des particuliers. L'activité publique poursuit un but particulier: celui de l'intérêt général. Ce but, l'administration le remplit grâce à des moyens spécifiques: des prérogatives de puissance publique. Le particulier va agir pour satisfaire son propre intérêt, et parfois, peut coïncider à l'intérêt commun. Parfois l'intérêt privé ne veut pas prendre en charge les intérêts publics, notamment ceux qui ne sont pas profitables, ou alors ils ne peuvent pas. Ces activités sont celles qui sont prises en charge par l'administration. \\
On doit maintenant définir ce qu'est l'intérêt général pour comprendre l'action de l'administration, et ce concept est impossible à définir. On est sûr que ce n'est en tout cas pas la somme d'intérêts commun. C'est une notion évolutive dans le temps: le CE considérait en 1916 (Arrêt Astruc) que la construction d'un théâtre n'était pas de l'intérêt général (en 1916, pendant la guerre, la puissance publique a d'autres priorités), mais en dehors des périodes de guerre, la culture relève de l'intérêt général. \\
Il n'est pas possible de définir l'intérêt général, mais on peut  l'identifier. Il est d'intérêt général ce que le législateur considère comme intérêt général et ce que le juge considère comme tel.


Contrairement au droit privé, on ne demande pas aux administrés leur accord avant d'imposer une décision. Pour imposer ses vues, l'administration utilise des prérogatives de puissances publiques. Par exemple, pour construire une autoroute, l'État a besoin d'un terrain appartenant à une personne privée et possède donc la capacité de négocier à l'amiable la vente ou alors peut l'exproprier pour faire primer l'intérêt général. \\
L'administration n'a pas l'obligation de contraindre les personnes privés et peut utiliser les outils du droit privé, comme des contrats de droit privés. 


"L'administration est une activité par laquelle les autorités publiques pourvoient à la satisfaction de besoins d'intérêts publics en utilisant, le cas échéant, des prérogatives de puissance publique" Jean Walin. \\
Depuis 2015, il existe une définition juridique de l'administration dans le CRPA, qui définit les finalités de l'administration: agir dans l'intérêt général. Ce code identifie également l'Administration L100-3: sont l'administration l'Administration de l'État ainsi que celle des collectivités territoriales, établissements publics administratif et toutes autres personnalités juridiques chargés d'une mission de service public administratif.

\subsection{L'administration et les autres activités publiques}

Administrer n'est pas légiférer. Le législateur pose des règles générales et impersonnelles, qui s'imposent à tous. C'est le cas aussi de l'Administration. L'activité législative est une activité ponctuelle alors que l'activité de l'administration est continue et plus concrète. Un acte législatif a une valeur législative alors qu'un acte administratif a une valeur infra-législative. \\
Administrer n'est pas juger. L'administration est aussi soumise au droit mais n'a pas pour fonction de trancher des litiges. \\
On peut faire une distinction entre l'Administration et le Gouvernement, celle ci n'a pas de portée juridique. Il y a une différence factuelle entre administré et gouverné, gouverné consiste à prendre les décisions importantes, à diriger la politique de la nation (art. 20 Constitution). Le Gouvernement dispose de l'Administration (Art. 20 Constitution), ce qui signifie que le Gouvernement a un pouvoir hiérarchique. Les décisions du Gouvernement s'imposent à l'administration, à l'inverse, les décisions de l'administration sont celle du gouvernement. Enfin, les actes du gouvernement sont aussi des actes administratifs. 


\section{Le droit administratif}

\subsection{La consécration du droit administratif}

Le droit administratif est un droit assez jeune, il a approximativement deux siècles.

\subsubsection{La soumission de l'administration au droit}

L'administration n'a pas toujours été soumise au droit car dans l'État de police, l'administration était soumise à des règles qui n'avaient de valeur qu'à l'intérieur de l'administration, n'avait pas de valeur juridique, ce qui conduisait à une administration non soumise à un droit. \\
Depuis la Révolution, l'État de droit est apparu progressivement, ce qui a eu deux conséquences: l'administration est liée par les règles de droit et celles qu'elle a posée elle même ; il est possible de contester juridiquement la plupart des actes de l'administrations. \\

\subsubsection{La spécificité du droit de l'administration}

Dans d'autres systèmes juridiques, l'administration est soumise au droit privé et ce n'est que par dérogations qu'elle n'y est pas soumise. En France, l'administration peut se mettre de lui même sous le joug du droit privé (on parle alors de gestion privée de l'administration). Quoi qu'il arrive, l'administration a besoin de règles particulières qui est le droit administratif. \\
En France, il existe le principe de la séparation des autorités administratives et judiciaires. Tout juge ne peut pas trancher des affaires de l'administration. C'est progressivement un juge spécial qui va régler ces affaires de manière différente que les juges judiciaires. Les juges administratifs ont donc construit progressivement le droit administratif.


La décision Blanco du juge des conflits (TC 8 Février 1873), consacre explicitement l'autonomie du juge administratif. Il le fait dans une affaire banale. Blanco est une fillette de 5 ans, renversée par un wagonnet poussé par des ouvriers d'une manufacture publique de tabac. Mr Blanco souhaite donc engager la responsabilité de l'État. Le juge des conflits va se demander quel droit est applicable pour définir la juridiction. Le TC va dire que l'administration doit être soumis à un droit particulier. Cette autonomie se justifie, selon le juge, par les besoins du service (service public). \\
Cet arrêt est souvent cité comme l'arrêt qui crée le droit administratif. \\
C'est un raisonnement moniste. Qui dit droit administratif dit compétence au juge administratif. Il y a un principe de liaison de la compétence et du fond, mais ce principe a été dérogé récemment et il est arrivé que le juge judiciaire applique des règles de droit administratifs. \\
La notion de service public est une notion importante, mais le lien qui existe entre service public et application du droit administratif peut parfois ne pas exister. Les services publics industriels et commerciaux sont par exemple régies par le droit privé. La SNCF remplit une mission de service public mais si on se fait écraser par un wagon c'est devant le juge judiciaire que l'on ira.

\subsubsection{Définition du droit administratif}

La doctrine a tentée de définir le droit administratif. Il y avait un débat entre l'école du service public et l'école de la puissance publique. \\
Pour la première, que l'on appelait aussi l'école de Bordeaux, la notion de service public permet de définir ce qu'est le droit administratif afin de déterminer la compétence du juge administratif, de définir d'autres notions du droit administratifs. Cette tentative doctrinale a échouée pour trois raisons: l'administration exerce d'autres activités que le service public (comme la police administrative, la réglementation, gestion des biens) ; la notion de service public est particulièrement floue, c'est une mission d'intérêt général, or l'intérêt général n'est pas définissable ; qui dit service public ne dit pas forcément application du droit administratif. \\
Pour la deuxième, dont le principal représentant est Hauriau, c'est la prérogative de puissance publique qui détermine la compétence de la juridiction administrative. Cette école a échouée aussi car le juge et le législateur se réfère parfois à ces deux notions, parfois de manières cumulatives, parfois, non. \\
Le droit administratif ne peut donc pas se définir à travers un seul critère. 

\subsection{Les caractères du droit administratif}

\subsubsection{Un droit initialement jurisprudentiel}

C'est le juge administratif qui a dégagé de nombreuses règles du droit administratif car il y a été contraint, aucun droit écrit n'existant. \\
Ce caractère jurisprudentiel a un désavantage: il n'est pas forcément clair. Les décisions sont souvent laconiques, les juges en écrivent le moins possible pour éviter de se lier, pour éviter que toutes les décisions soient des décisions de principes. Cela pose un problème de sécurité juridique, car c'est une source juridique rétroactive. Le CE a trouvé une parade, en 2007: le CE peut moduler dans le temps les revirements de jurisprudence et peut décider d'appliquer la jurisprudence que dans l'avenir, par dérogation (CE Ass, 2007, Tropic travaux signalisation). 

\subsubsection{Un droit de plus en plus écrit}

Depuis 50 ans, les règles écrites sont de plus en plus importantes: la constitution, les traités internationaux, la prolifération de la loi et des règlements. Il existe de plus en plus de textes relatifs au droit administratif. \\
Cependant, cela pose un problème: on a peut être trop de sources écrites. L'inflation législative pose un soucis de droit qui devient inintelligible. Le législateur essaye d'adopter une nouvelle attitude: codifier les textes pour permettre de clarifier et de simplifier le droit. Cette codification a bien sûr toucher le droit administratif: le CGCT (Code général des collectivités territoriales), le CJA (code de justice administrative), le CG3P (Code général de la propriété des personnes publiques), le CRPA (Code des relations entre le public et l'administration), le futur Code de la commande publique. \\
Depuis 2015, la codification se fait souvent à droit inconstant, le législateur modifie donc le droit, notamment jurisprudentiel en codifiant le droit. 

\section{Le juge administratif}

\subsection{La construction du dualisme juridictionnel}

Le dualisme juridictionnel est apparu progressivement. On peut identifier 5 temps dans l'évolution du juge administratif.


Le premier temps est la période révolutionnaire et commence par l'adoption "16 et 24 Août 1790" qui vient poser le principe de séparation des autorités administratives et judiciaires. Le législateur va interdire à tout juge de trancher les litiges de l'administration. Ce texte apparaît à la suite du fait que des instances juridiques (les parlements) bloquaient les ordonnances royales. L'administration elle même tranche donc ses propres litiges. On considérait que "juger l'administration, c'est encore administrer".


Le deuxième temps est l'an VIII, 1799 donc, qui est la période de la création des juridictions administratives. Le Conseil d'État est créé ainsi que les conseils de préfecture. Si l'on crée des juridictions administratives, il est à noter que ces juridictions ne tranchent encore aucun litige. Elles donnaient seulement des avis aux ministres qui jugeaient (ministre juge, on parlait de justice retenu).


Le troisième temps est l'indépendance du juge administratif qui a été consacré par la loi du 24 Mai 1872 qui met fin à la justice retenu et met en place la justice déléguée aux juridictions administratives. Ce texte est important car crée aussi le tribunal des conflits. Il n'a pas non plus supprimé totalement la justice retenu, ce sera donc le CE, 13 Décembre 1889, Arrêt Cadot, qui y mettra fin lui même. 


Le quatrième temps est le XXe siècle où sont apparus des réformes importantes de la juridiction administrative. La première est la création des Tribunaux Administratifs qui remplace les conseils de préfecture ; la deuxième est la création des CAA, en 1997 pour désencombrer le Conseil d'État. Le CE est donc devenu principalement un juge de cassation. Au niveau des réformes procédurales, la première est l'attribution d'un pouvoir d'injonction au juge administratif en 1995 qui lui permet d'imposer à l'administration un comportement déterminé. Un loi du 30 Juin 2000 réforme les procédures d'urgence devant la justice administrative qui crée des procédures rapides et efficaces et remplaçant les anciennes procédures qui étaient difficiles à mettre en oeuvre.


Le cinquième temps sont les réformes contemporaines qui vont dans deux sens distincts.\\ 
Le premier sens est l'idée de garantir le droit au procès équitable devant le juge administratif. C'est ce qui a été fait notamment à propos de l'ancien commissaire au gouvernement, que l'on appelle depuis 2009 un rapporteur public. Celui-ci a un rôle très spécifique, il a pour mission de donner son avis sur le litige en cours, il rend ce qu'on appelle des conclusions (il donnera la question de droit, les solutions possible et celle qu'il préconise en toute indépendance et impartialité). Ce sont les conclusions de ces rapporteurs qui ont aidés à construire la justice administrative. Pendant longtemps, les requérants ne pouvaient lui répondre, et le rapporteur avait droit au délibéré, il posait donc des problème d'apparence de partialité. La CEDH a relevé ce soucis de partialité et a conduit à une réforme du commissaire au gouvernement. \\
Le deuxième sens vise l'efficacité du juge administratif. Pendant longtemps, le juge pouvait mettre des années à rendre des décisions, certaines réformes ont donc été faites et ont permis de réduire le temps des affaires traités. Aujourd'hui, le CE met en moyenne 8 mois pour trancher une affaire. Il y a comme une volonté de bannaliser le juge administratif: le rendre similaire au juge judiciaire. 

\subsection{La consécration constitutionnelle du juge administratif}

Pendant longtemps, il n'y avait aucune mention du juge administratif alors que le juge judiciaire était mentionné (art. 66). La Constitution traitait du CE mais seulement en tant qu'organe consultatif (art. 39). \\
Le CC a rendu 3 décisions pour progressivement consacré constitutionnellement la juridiction administrative. \\
La première décision est la consécration de l'indépendance du juge administratif: CC, 22 Juillet 1981 "loi portant validation d'actes administratifs", il dégage un PFRLR: l'indépendance de la juridiction administrative. Le fondement qui permet de dégager ce PFRLR est la loi de 1872. Le juge ne se réfère pas à l'article 16 de la DDHC et a préféré dégager un principe pour conforter la position de la justice administrative. Curieusement, son indépendance est consacré alors que son existence ne l'est pas encore.


La deuxième décision du 23 Janvier 1987 "Conseil de la concurrence". Une loi avait pour objet de transférer au juge judiciaire une compétence en matière administrative. La question en l'espèce était de savoir si un tel transfert était possible. Le CC dégagera un nouveau PFRLR qui définit constitutionnellement la compétence du juge administratif. Le juge administratif est compétent pour traiter de tous les actes de puissance publiques. En consacrant une compétence au niveau constitutionnel, il consacre l'existence de la juridiction. \\
Ce PFRLR pourrait être fondé dans la loi de 1790. Cependant, ce n'est pas le cas car la loi de 1790 ne permettait pas de fonder ce principe car c'était une loi non républicaine. Le CC va donc se référer "à la conception française de la séparation des pouvoirs". \\
Concernant le champ de la compétence constitutionnelle du juge administratif, il est compétent dès que sont en cause des prérogatives de puissance publique. Cela ne concerne pas toutes les prérogatives car, selon le CC, la compétence concerne seulement "l'annulation ou la réformation d'actes administratifs". Le contentieux contractuel ne relève donc pas du juge administratif pour le CC, cette compétence n'est donc pas constitutionnelle. Des personnes privés peuvent aussi édicter des actes administratifs et ne relèvent pas du juge administratif. \\
En 1987, le CC dit qu'il est possible de déroger à ce principe et de confier au juge judiciaire de traiter d'un contentieux administratif, sous deux conditions, dans l'intérêt d'une bonne administration de la justice ; l'aménagement doit être "précis et limité". \\
Certains contentieux administratifs très importants ont étés confiés à des juges judiciaires comme des affaire de maltraitance à l'école par des profs. La plupart des contentieux des AAI sont aussi traités au judiciaire. 


La troisième décision, du 3 Décembre 2009 "Loi organique relative à la QPC", le CC a interprété les disposition et a dit que "la C.Cas comme le CE sont les juridictions placés au sommet de chacun des deux ordres juridictionnels". \\
Cette décision consacre donc le dualisme juridictionnel.


Au niveau politique, ou même en doctrine, le JA est contesté notamment à cause de la complexité de la répartition des compétences. La question de comment supprimer le JA pourrait donc se poser. Une simple loi est donc impossible pour supprimer le JA car il est consacré constitutionnellement. Une LC est donc nécessaire pour se faire. 


\subsection{Les attributions de la juridiction administrative}

Le JA exerce une fonction juridictionnelle qui consiste à trancher des litiges. Cependant, il exerce aussi une fonction consultative qui lui permet d'émettre des avis: c'est le cas des TA et des CAA qui peuvent donner des avis aux préfets mais aussi le CE qui donne des avis au profit du Gouvernement. Dans certains cas, le CE doit être obligatoirement consulté, notamment dans les PJL, les projets d'ordonnance, ou encore les décrets en conseil d'État. \\
Sur l'hypothèse des décrets en CE, lorsqu'il est obligatoirement consulté, le CE est considéré comme co-auteur. Le gouvernement a donc un choix très limité, soit il adopte le projet initial, soit le projet modifié par le CE. Le Gouvernement ne pourra donc pas adopter d'autres versions de son projet. \\
Cela a des effets contentieux très important lorsque le Gouvernement ne consulte pas le CE lorsqu'il devrait l'être. Dans une décision "SCI Boulevard Marago", le CE crée le vice d'incompétence. \\
Le contrôle en matière consultative est très différent du contrôle en matière contentieuse. En matière contentieuse, il vérifie que l'acte est conforme aux règles de droit supérieur (contrôle de légalité). En matière consultative, il vérifie avec la légalité la question de l'opportunité politique. 



















\end{document}
