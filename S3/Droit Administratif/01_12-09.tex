\documentclass[10pt, a4paper, openany]{book}

\usepackage[utf8x]{inputenc}
\usepackage[T1]{fontenc}
\usepackage[francais]{babel}
\usepackage{bookman}
\usepackage{fullpage}
\setlength{\parskip}{5px}
\date{}
\title{Cours de Droit Administratif (UFR Amiens)}
\pagestyle{plain}

% GALOP SAMEDI 12 NOVEMBRE 14h - 17h

\begin{document}
\maketitle
\tableofcontents

\chapter{Introduction}

Le droit administratif est loin du cliché qu'on peut avoir. \\
Le DA est une relation entre l'Administration et d'autres personnes juridiques, qu'elle soit publique, privé, ou morale. Le DA vise l'intérêt général. C'est un droit de conciliation entre l'intérêt général et les libertés individuelles. Le DA est la matière qui permet de savoir, par exemple, si on peut suspendre la dissémination d'OGM dans une commune ; on lie dans cet exemple la liberté d'entreprendre de l'agriculteur et l'intérêt général. Comme le DA est un droit d'intérêt général, c'est un droit éminemment politique. \\
Le DA est un droit théorique car certaines questions ne sont pas résolues. En l'absence de principes de droit, c'est à la doctrine de trouver des solutions. \\
Le DA est un droit important dans le droit public et contrairement en droit privée ou en droit des affaires, c'est un domaine qui est moins prisé des avocats et juristes. \\
Pour certains, le DA serait inintelligible et très complexe. Cette critique est certainement vrai. On notera que c'est une matière qui s'est complexifié récemment. 


Bibliographie:
\begin{itemize}
\item Les Grands Arrêts des Juridictions Administratives - Dalloz ;
\item Le traité de droit administratif - Dalloz, 2 tomes ;
\item Manuel de droit administratif - Plessix ;
\item Manuel de droit administratif - Lombard, Sirinelli ;
\item Manuel de droit administratif - Truchet ;
\item Actualité Juridique du Droit Administratif - Dalloz, hebdomadaire ;
\item Revue Française de Droit Administratif ;
\item Revue de Droit Administratif - Lexis Nexis, mensuel.
\end{itemize}


À priori, on pourrait penser que le DA est le droit de l'administration. Cette définition est incomplète car il faut se demander ce qu'est l'administration. Ce n'est qu'après avoir défini l'administration qu'on pourrait envisager le droit administratif et le juge administratif.

\section{L'administration}

C'est un terme polysémique, on peut le voir d'un point de vue matériel et d'un point de vue organique. \\
D'un point de vue matériel, l'administration est une activité qui consiste à gérer une affaire. Du point de vue organique, c'est l'organe qui gère cette affaire. Ce terme peut très bien s'utiliser en droit privé comme le Conseil d'Administration d'une entreprise. C'est pour cela que nous parlerons, nous, d'administration publique, l'organe qui gère les activités publiques.

\subsection{L'administration et l'action des particuliers}

L'activité publique se distingue de l'activité des particuliers. L'activité publique poursuit un but particulier: celui de l'intérêt général. Ce but, l'administration le remplit grâce à des moyens spécifiques: des prérogatives de puissance publique. Le particulier va agir pour satisfaire son propre intérêt, et parfois, peut coïncider à l'intérêt commun. Parfois l'intérêt privé ne veut pas prendre en charge les intérêts publics, notamment ceux qui ne sont pas profitables, ou alors ils ne peuvent pas. Ces activités sont celles qui sont prises en charge par l'administration. \\
On doit maintenant définir ce qu'est l'intérêt général pour comprendre l'action de l'administration, et ce concept est impossible à définir. On est sûr que ce n'est en tout cas pas la somme d'intérêts commun. C'est une notion évolutive dans le temps: le CE considérait en 1916 (Arrêt Astruc) que la construction d'un théâtre n'était pas de l'intérêt général (en 1916, pendant la guerre, la puissance publique a d'autres priorités), mais en dehors des périodes de guerre, la culture relève de l'intérêt général. \\
Il n'est pas possible de définir l'intérêt général, mais on peut  l'identifier. Il est d'intérêt général ce que le législateur considère comme intérêt général et ce que le juge considère comme tel.


Contrairement au droit privé, on ne demande pas aux administrés leur accord avant d'imposer une décision. Pour imposer ses vues, l'administration utilise des prérogatives de puissances publiques. Par exemple, pour construire une autoroute, l'État a besoin d'un terrain appartenant à une personne privée et possède donc la capacité de négocier à l'amiable la vente ou alors peut l'exproprier pour faire primer l'intérêt général. \\
L'administration n'a pas l'obligation de contraindre les personnes privés et peut utiliser les outils du droit privé, comme des contrats de droit privés. 


"L'administration est une activité par laquelle les autorités publiques pourvoient à la satisfaction de besoins d'intérêts publics en utilisant, le cas échéant, des prérogatives de puissance publique" Jean Walin. \\
Depuis 2015, il existe une définition juridique de l'administration dans le CRPA, qui définit les finalités de l'administration: agir dans l'intérêt général. Ce code identifie également l'Administration L100-3: sont l'administration l'Administration de l'État ainsi que celle des collectivités territoriales, établissements publics administratif et toutes autres personnalités juridiques chargés d'une mission de service public administratif.

\subsection{L'administration et les autres activités publiques}

Administrer n'est pas légiférer. Le législateur pose des règles générales et impersonnelles, qui s'imposent à tous. C'est le cas aussi de l'Administration. L'activité législative est une activité ponctuelle alors que l'activité de l'administration est continue et plus concrète. Un acte législatif a une valeur législative alors qu'un acte administratif a une valeur infra-législative. \\
Administrer n'est pas juger. L'administration est aussi soumise au droit mais n'a pas pour fonction de trancher des litiges. \\
On peut faire une distinction entre l'Administration et le Gouvernement, celle ci n'a pas de portée juridique. Il y a une différence factuelle entre administré et gouverné, gouverné consiste à prendre les décisions importantes, à diriger la politique de la nation (art. 20 Constitution). Le Gouvernement dispose de l'Administration (Art. 20 Constitution), ce qui signifie que le Gouvernement a un pouvoir hiérarchique. Les décisions du Gouvernement s'imposent à l'administration, à l'inverse, les décisions de l'administration sont celle du gouvernement. Enfin, les actes du gouvernement sont aussi des actes administratifs. 


\section{Le droit administratif}

\subsection{La consécration du droit administratif}

Le droit administratif est un droit assez jeune, il a approximativement deux siècles.

\subsubsection{La soumission de l'administration au droit}

L'administration n'a pas toujours été soumise au droit car dans l'État de police, l'administration était soumise à des règles qui n'avaient de valeur qu'à l'intérieur de l'administration, n'avait pas de valeur juridique, ce qui conduisait à une administration non soumise à un droit. \\
Depuis la Révolution, l'État de droit est apparu progressivement, ce qui a eu deux conséquences: l'administration est liée par les règles de droit et celles qu'elle a posée elle même ; il est possible de contester juridiquement la plupart des actes de l'administrations. \\

\subsubsection{La spécificité du droit de l'administration}

Dans d'autres systèmes juridiques, l'administration est soumise au droit privé et ce n'est que par dérogations qu'elle n'y est pas soumise. En France, l'administration peut se mettre de lui même sous le joug du droit privé (on parle alors de gestion privée de l'administration). Quoi qu'il arrive, l'administration a besoin de règles particulières qui est le droit administratif. \\
En France, il existe le principe de la séparation des autorités administratives et judiciaires. Tout juge ne peut pas trancher des affaires de l'administration. C'est progressivement un juge spécial qui va régler ces affaires de manière différente que les juges judiciaires. Les juges administratifs ont donc construit progressivement le droit administratif.


La décision Blanco du juge des conflits (TC 8 Février 1873), consacre explicitement l'autonomie du juge administratif. Il le fait dans une affaire banale. Blanco est une fillette de 5 ans, renversée par un wagonnet poussé par des ouvriers d'une manufacture publique de tabac. Mr Blanco souhaite donc engager la responsabilité de l'État. Le juge des conflits va se demander quel droit est applicable pour définir la juridiction. Le TC va dire que l'administration doit être soumis à un droit particulier. Cette autonomie se justifie, selon le juge, par les besoins du service (service public). \\
Cet arrêt est souvent cité comme l'arrêt qui crée le droit administratif. \\
C'est un raisonnement moniste. Qui dit droit administratif dit compétence au juge administratif. Il y a un principe de liaison de la compétence et du fond, mais ce principe a été dérogé récemment et il est arrivé que le juge judiciaire applique des règles de droit administratifs. \\
La notion de service public est une notion importante, mais le lien qui existe entre service public et application du droit administratif peut parfois ne pas exister. Les services publics industriels et commerciaux sont par exemple régies par le droit privé. La SNCF remplit une mission de service public mais si on se fait écraser par un wagon c'est devant le juge judiciaire que l'on ira.

\subsubsection{Définition du droit administratif}

La doctrine a tentée de définir le droit administratif. Il y avait un débat entre l'école du service public et l'école de la puissance publique. \\
Pour la première, que l'on appelait aussi l'école de Bordeaux, la notion de service public permet de définir ce qu'est le droit administratif afin de déterminer la compétence du juge administratif, de définir d'autres notions du droit administratifs. Cette tentative doctrinale a échouée pour trois raisons: l'administration exerce d'autres activités que le service public (comme la police administrative, la réglementation, gestion des biens) ; la notion de service public est particulièrement floue, c'est une mission d'intérêt général, or l'intérêt général n'est pas définissable ; qui dit service public ne dit pas forcément application du droit administratif. \\
Pour la deuxième, dont le principal représentant est Hauriau, c'est la prérogative de puissance publique qui détermine la compétence de la juridiction administrative. Cette école a échouée aussi car le juge et le législateur se réfère parfois à ces deux notions, parfois de manières cumulatives, parfois, non. \\
Le droit administratif ne peut donc pas se définir à travers un seul critère. 

\subsection{Les caractères du droit administratif}

\subsubsection{Un droit initialement jurisprudentiel}

C'est le juge administratif qui a dégagé de nombreuses règles du droit administratif car il y a été contraint, aucun droit écrit n'existant. \\
Ce caractère jurisprudentiel a un désavantage: il n'est pas forcément clair. Les décisions sont souvent laconiques, les juges en écrivent le moins possible pour éviter de se lier, pour éviter que toutes les décisions soient des décisions de principes. Cela pose un problème de sécurité juridique, car c'est une source juridique rétroactive. Le CE a trouvé une parade, en 2007: le CE peut moduler dans le temps les revirements de jurisprudence et peut décider d'appliquer la jurisprudence que dans l'avenir, par dérogation (CE Ass, 2007, Tropic travaux signalisation). 

\subsubsection{Un droit de plus en plus écrit}

Depuis 50 ans, les règles écrites sont de plus en plus importantes: la constitution, les traités internationaux, la prolifération de la loi et des règlements. Il existe de plus en plus de textes relatifs au droit administratif. \\
Cependant, cela pose un problème: on a peut être trop de sources écrites. L'inflation législative pose un soucis de droit qui devient inintelligible. Le législateur essaye d'adopter une nouvelle attitude: codifier les textes pour permettre de clarifier et de simplifier le droit. Cette codification a bien sûr toucher le droit administratif: le CGCT (Code général des collectivités territoriales), le CJA (code de justice administrative), le CG3P (Code général de la propriété des personnes publiques), le CRPA (Code des relations entre le public et l'administration), le futur Code de la commande publique. \\
Depuis 2015, la codification se fait souvent à droit inconstant, le législateur modifie donc le droit, notamment jurisprudentiel en codifiant le droit. 

\section{Le juge administratif}

\subsection{La construction du dualisme juridictionnel}

Le dualisme juridictionnel est apparu progressivement. On peut identifier 5 temps dans l'évolution du juge administratif.


Le premier temps est la période révolutionnaire et commence par l'adoption "16 et 24 Août 1790" qui vient poser le principe de séparation des autorités administratives et judiciaires. Le législateur va interdire à tout juge de trancher les litiges de l'administration. Ce texte apparaît à la suite du fait que des instances juridiques (les parlements) bloquaient les ordonnances royales. L'administration elle même tranche donc ses propres litiges. On considérait que "juger l'administration, c'est encore administrer".


Le deuxième temps est l'an VIII, 1799 donc, qui est la période de la création des juridictions administratives. Le Conseil d'État est créé ainsi que les conseils de préfecture. Si l'on crée des juridictions administratives, il est à noter que ces juridictions ne tranchent encore aucun litige. Elles donnaient seulement des avis aux ministres qui jugeaient (ministre juge, on parlait de justice retenu).


Le troisième temps est l'indépendance du juge administratif qui a été consacré par la loi du 24 Mai 1872 qui met fin à la justice retenu et met en place la justice déléguée aux juridictions administratives. Ce texte est important car crée aussi le tribunal des conflits. Il n'a pas non plus supprimé totalement la justice retenu, ce sera donc le CE, 13 Décembre 1889, Arrêt Cadot, qui y mettra fin lui même. 


Le quatrième temps est le XXe siècle où sont apparus des réformes importantes de la juridiction administrative. La première est la création des Tribunaux Administratifs qui remplace les conseils de préfecture ; la deuxième est la création des CAA, en 1997 pour désencombrer le Conseil d'État. Le CE est donc devenu principalement un juge de cassation. Au niveau des réformes procédurales, la première est l'attribution d'un pouvoir d'injonction au juge administratif en 1995 qui lui permet d'imposer à l'administration un comportement déterminé. Un loi du 30 Juin 2000 réforme les procédures d'urgence devant la justice administrative qui crée des procédures rapides et efficaces et remplaçant les anciennes procédures qui étaient difficiles à mettre en oeuvre.


Le cinquième temps sont les réformes contemporaines qui vont dans deux sens distincts.\\ 
Le premier sens est l'idée de garantir le droit au procès équitable devant le juge administratif. C'est ce qui a été fait notamment à propos de l'ancien commissaire au gouvernement, que l'on appelle depuis 2009 un rapporteur public. Celui-ci a un rôle très spécifique, il a pour mission de donner son avis sur le litige en cours, il rend ce qu'on appelle des conclusions (il donnera la question de droit, les solutions possible et celle qu'il préconise en toute indépendance et impartialité). Ce sont les conclusions de ces rapporteurs qui ont aidés à construire la justice administrative. Pendant longtemps, les requérants ne pouvaient lui répondre, et le rapporteur avait droit au délibéré, il posait donc des problème d'apparence de partialité. La CEDH a relevé ce soucis de partialité et a conduit à une réforme du commissaire au gouvernement. \\
Le deuxième sens vise l'efficacité du juge administratif. Pendant longtemps, le juge pouvait mettre des années à rendre des décisions, certaines réformes ont donc été faites et ont permis de réduire le temps des affaires traités. Aujourd'hui, le CE met en moyenne 8 mois pour trancher une affaire. Il y a comme une volonté de bannaliser le juge administratif: le rendre similaire au juge judiciaire. 

\subsection{La consécration constitutionnelle du juge administratif}

Pendant longtemps, il n'y avait aucune mention du juge administratif alors que le juge judiciaire était mentionné (art. 66). La Constitution traitait du CE mais seulement en tant qu'organe consultatif (art. 39). \\
Le CC a rendu 3 décisions pour progressivement consacré constitutionnellement la juridiction administrative. \\
La première décision est la consécration de l'indépendance du juge administratif: CC, 22 Juillet 1981 "loi portant validation d'actes administratifs", il dégage un PFRLR: l'indépendance de la juridiction administrative. Le fondement qui permet de dégager ce PFRLR est la loi de 1872. Le juge ne se réfère pas à l'article 16 de la DDHC et a préféré dégager un principe pour conforter la position de la justice administrative. Curieusement, son indépendance est consacré alors que son existence ne l'est pas encore.


La deuxième décision du 23 Janvier 1987 "Conseil de la concurrence". Une loi avait pour objet de transférer au juge judiciaire une compétence en matière administrative. La question en l'espèce était de savoir si un tel transfert était possible. Le CC dégagera un nouveau PFRLR qui définit constitutionnellement la compétence du juge administratif. Le juge administratif est compétent pour traiter de tous les actes de puissance publiques. En consacrant une compétence au niveau constitutionnel, il consacre l'existence de la juridiction. \\
Ce PFRLR pourrait être fondé dans la loi de 1790. Cependant, ce n'est pas le cas car la loi de 1790 ne permettait pas de fonder ce principe car c'était une loi non républicaine. Le CC va donc se référer "à la conception française de la séparation des pouvoirs". \\
Concernant le champ de la compétence constitutionnelle du juge administratif, il est compétent dès que sont en cause des prérogatives de puissance publique. Cela ne concerne pas toutes les prérogatives car, selon le CC, la compétence concerne seulement "l'annulation ou la réformation d'actes administratifs". Le contentieux contractuel ne relève donc pas du juge administratif pour le CC, cette compétence n'est donc pas constitutionnelle. Des personnes privés peuvent aussi édicter des actes administratifs et ne relèvent pas du juge administratif. \\
En 1987, le CC dit qu'il est possible de déroger à ce principe et de confier au juge judiciaire de traiter d'un contentieux administratif, sous deux conditions, dans l'intérêt d'une bonne administration de la justice ; l'aménagement doit être "précis et limité". \\
Certains contentieux administratifs très importants ont étés confiés à des juges judiciaires comme des affaire de maltraitance à l'école par des profs. La plupart des contentieux des AAI sont aussi traités au judiciaire. 


La troisième décision, du 3 Décembre 2009 "Loi organique relative à la QPC", le CC a interprété les disposition et a dit que "la C.Cas comme le CE sont les juridictions placés au sommet de chacun des deux ordres juridictionnels". \\
Cette décision consacre donc le dualisme juridictionnel.


Au niveau politique, ou même en doctrine, le JA est contesté notamment à cause de la complexité de la répartition des compétences. La question de comment supprimer le JA pourrait donc se poser. Une simple loi est donc impossible pour supprimer le JA car il est consacré constitutionnellement. Une LC est donc nécessaire pour se faire. 


\subsection{Les attributions de la juridiction administrative}

Le JA exerce une fonction juridictionnelle qui consiste à trancher des litiges. Cependant, il exerce aussi une fonction consultative qui lui permet d'émettre des avis: c'est le cas des TA et des CAA qui peuvent donner des avis aux préfets mais aussi le CE qui donne des avis au profit du Gouvernement. Dans certains cas, le CE doit être obligatoirement consulté, notamment dans les PJL, les projets d'ordonnance, ou encore les décrets en conseil d'État. \\
Sur l'hypothèse des décrets en CE, lorsqu'il est obligatoirement consulté, le CE est considéré comme co-auteur. Le gouvernement a donc un choix très limité, soit il adopte le projet initial, soit le projet modifié par le CE. Le Gouvernement ne pourra donc pas adopter d'autres versions de son projet. \\
Cela a des effets contentieux très important lorsque le Gouvernement ne consulte pas le CE lorsqu'il devrait l'être. Dans une décision "SCI Boulevard Marago", le CE crée le vice d'incompétence. \\
Le contrôle en matière consultative est très différent du contrôle en matière contentieuse. En matière contentieuse, il vérifie que l'acte est conforme aux règles de droit supérieur (contrôle de légalité). En matière consultative, il vérifie avec la légalité la question de l'opportunité politique. 


Le principe de dédoublement fonctionnel peut poser un problème de partialité. Par exemple, si le CE donne un avis sur un décret pris en CE (il en est donc le co-auteur) et que quelques mois plus tard, un requérant conteste le dit décret, c'est le CE qui devra régler le litige. La CEDH, 28 Septembre 1995 "Procola contre Luxembourg", dit, à propos de l'usage successif de fonctions consultatives puis contentieuse par la même personne et dans la même affaire viole le principe d'impartialité garanti par l'article 6 de la CEDH. Le principe de dédoublement fonctionnel n'est donc pas remis en cause. \\
On a considéré qu'en pratique, on pouvait faire en sorte que jamais une même personne exerce les deux fonctions dans une même affaire, en effet il y a plus de 300 conseillers d'État. En 2006, la CEDH rappellera la France à l'ordre et qu'il faut un texte qui sera pris en 2008. Un décret du 6 Mars 2008 vient séparé les fonctions consultatives et contentieuse du CE. Il est désormais formellement interdit à un conseiller d'État de travailler à la fois au consultatif et au contentieux dans la même affaire du CE. Ce décret a aussi modifié les formations de jugement: les conseillers qui ont participé à la consultation ne siègent plus au jugement en contentieux. \\
Un décret du 23 Septembre 2011 interdit aux conseillers siégeant au contentieux de prendre un éventuel avis d'un autre conseiller ayant siégé au consultatif. Ce décret n'a plus aucune utilité depuis 2015 car tous les avis sont désormais publics. 



\part{Les sources du droit administratif}


\chapter{Les sources constitutionnelles du droit administratif}

Un arrêt d'assemblée du 30 Octobre 1998, "Sarran", confirmé par la C.Cas en assemblée plénière, 2 Juin 2000 "Mlle Fraisse" affirment que le droit administratif se doit de respecter la Constitution et que c'est au juge de contrôler le droit administratif au regard de la Constitution. 

\section{Bloc de constitutionnalité}

Le bloc de constitutionnalité a été étendue essentiellement par le CC, mais aussi par le juge administratif. Il existe des principes jurisprudentiels qui ont valeur constitutionnel. CC, 17 Mai 2013, "Mariage pour tous", le droit naturel n'est pas reconnu. 

\subsection{Les articles de la Constitution}

Certains articles concernent directement le droit administratif. On y trouve des règles de compétence et de procédure qui sont très importantes. L'article 21 de la Constitution par exemple, donne le premier ministre comme le détenteur du pouvoir réglementaire. Les articles 13 et 21 donnent des pouvoirs de nomination. Les articles 19 et 22 qui concerne le contreseing. L'article 37 délimite le domaine de compétence du pouvoir réglementaire. L'article 53 détermine les règles de ratification des traités internationaux. \\
Des articles peuvent donner des principes de fond, comme l'article 2 qui pose le principe de l'égalité devant la loi, l'article 55 qui pose le principe de supériorité sur la Loi. L'article 72 qui pose le principe de libre administration des collectivités territoriales. 

\subsection{Le préambule de la Constitution}

Le préambule renvoie à d'autres textes: la DDHC, le préambule de 1946, la charte de l'environnement. Le préambule de 1946 évoque aussi les PFRLR mais aussi aux "Principes politiques, économiques et sociaux particulièrement nécessaire à notre temps". \\
Pour le CE, il s'est contenté, dans un premier temps, de reprendre les principes du préambule pour en faire des principes jurisprudentiels. Le CE a aussi dit, le 16 Juillet 1950, en assemblée, "Dehanne", que le préambule de 1946 avait valeur constitutionnelle ainsi que ses normes auxquelles elle renvoie, notamment donc les PFRLR. Le 11 Juillet 1956, "Amicale des annamites de Paris", le CE dira encore que les PFRLR ont valeur constitutionnelles. En 1957, arrêt "Condamine", consacre la DDHC ayant valeur constitutionnelle. Après 1958, le CE, le 12 Février 1960, Arrêt "société Eky", dira que le nouveau préambule est constitutionnel lui aussi. \\
Le 16 Juillet 1971, le CC confirmera tout cela dans sa décision "Liberté d'association". \\
CE, 3 Octobre 2008, "Commune d'Annecy", le CE confirme la constitutionnalité de l'intégralité du préambule de 1958. 


Si le préambule a valeur constitutionnelle, cela ne signifie pas que toutes les normes sont opposables. En effet, certaines dispositions ne sont pas suffisamment précise. \\
Le 29 Novembre 1968, CE, arrêt "Tallagrand", une loi doit venir préciser les normes imprécises des préambules, comme par exemple le droit au logement, le droit à la santé. 

\subsection{La charte de l'environnement}

Elle a été inscrite dans la constitution en 2005. Elle contient deux grandes parties: un préambule qui ne comporte que des dispositions proclamatrices, et qui sont donc symboliques et non pas juridiques. \\
Elle contient des articles précis qui ont valeur juridique. Principe de prévention des atteintes à l'environnement, principe de réparation des atteintes, principe de précaution, droit de participation des administrés aux décisions qui ont un impact sur l'environnement. \\
Le CE dira en 2008 (commune d'Annecy) que toutes les dispositions de la charte ont valeur constitutionnelle. Deux questions se sont posés rapidement: quelle est l'autorité compétente pour mettre en oeuvre cette charte ? \\
En principe, c'est le législateur qui doit mettre en oeuvre ces principes. Le pouvoir réglementaire n'est pas compétent tant qu'une loi n'est pas adopté. Il y a une nuance sur l'article 5 de la charte, qui peut être mise en oeuvre par le pouvoir réglementaire même en l'absence de texte législatif. \\
Deuxième question: les principes sont elles suffisamment précis pour être opposable. Le principe de précaution est suffisamment précis pour être invocable en justice: arrêt du 19 Juillet 2010, CE, "Association quartier Les haut du choiseil". Pour toutes les autres dispositions, elles ne sont pas suffisamment précises selon le juge, et on ne peut pas les invoquer directement. CE, 6 Juin 2006 "Eau et rivière de Bretagne". Le CE a quelque peau nuancé cet arrêt dans 3 décisions. La première évolution résulte de l'arrêt "Commune d'Annecy" où le CE décide que l'article 7 (principe de participation des administrés) est directement opposable. La deuxième évolution, Assemblée 12 juillet 2013 "Fédération nationale de la pêche en France", où le juge décide que l'article 3 est invocable contre un acte administratif même si une loi l'a déjà mise en oeuvre, à la condition que l'acte administratif aille plus loin que la loi. La troisième évolution, dans un arrêt du 26 Février 2014 "Ban asbestos", le CE reprend le même raisonnement qu'en 2013 à propos de l'article premier de la charte. \\
On se demande aujourd'hui si ce raisonnement pourrait être étendue à tous les principes de la charte. 

\subsection{Principes jurisprudentiels}

Ce sont des principes dégagés par des juridictions et qui ont une valeur constitutionnelle. On peut distinguer deux catégories de principes: les principes et objectifs à valeur constitutionnels. Le principe pose une obligation de résultat au législateur comme le principe de continuité du service public. L'objectif ne fixe qu'une obligation de moyens ; le législateur doit mettre tous les moyens en oeuvre pour y parvenir et ne sera pas sanctionné si il n'y parvient pas. On peut citer dans ces objectif le principe d'intelligibilité de la norme. 


Concernant les PFRLR qui sont dégagés principalement par le CC (11 depuis une cinquantaine d'année) peuvent aussi être dégagés par le CE. CE, 3 Juillet 1996, "Konei", interdit l'extradition pour des motifs politiques. \\
Il existe plusieurs critères pour qu'un juge dégage un PFRLR, il faut que le principe soit dans une loi républicaine, antérieur à 1946, constant depuis sa consécration. Depuis 2013, le juge constitutionnel a fait évolué ses critères.


En 2013, s'est posé une question avec la loi sur le mariage pour tous. À cette occasion, il y a eu un grand débat chez les juristes qui se sont demandés si il existait un PFRLR qui interdirait le mariage entre personne du même sexe. Pour certains, il existerait, avec pour origine le Code Civil, qui a été adopté quelques jours avant l'empire. D'autres auteurs avaient une position contraire ; ceux ci se concentre sur l'objet des PFRLR, ceux-ci ayant pour objet les libertés fondamentales, et donc un PFRLR ne pourrait pas interdire quelque chose. \\
D'autres auteurs démontrent que ce débat n'est pas vraiment juridique et que ce sont des gens qui viennent débattre politique à l'aide d'arguments pseudo-juridique. Ils font remarquer une chose importantes: tant que le juge ne l'a pas identifié, il ne sert à rien d'essayer d'identifier un principe. Si le juge souhaite dégager un PFRLR, il trouvera les arguments pour, et si il ne le veut pas, il les trouvera aussi. \\
Cependant, le CC vient dégager un nouveau critère pour identifier un PFRLR. Pour le CC, un PFRLR ne peut concerner que la protection des droits fondamentaux, soit la souveraineté nationale, soit l'organisation des pouvoirs publics.


Il est à noter que le juge n'est jamais lié par ce qu'il dit. Il peut très bien décidé, demain, de changer ses critères. Des PFRLR dégagent des principes qui n'ont pas pour objet les 3 évoqués par le CC. En 2011, par exemple, le CC dégage le PFRLR de l'existence du droit local d'Alsace-Moselle. C'est un principe non constant, car le législateur peut réduire le champ du droit local, il est aussi limité à une partie du territoire. 


\section{Étendu du contrôle de constitutionnalité}

\subsection{L'écran législatif}

L'écran législatif est une limite au contrôle de constitutionnalité. Elle se tient en ce que le juge ne contrôlera pas l'acte administratif au regard de la constitution si le dit acte a été pris en application d'une loi. Si c'était le cas, cela reviendrai à contrôler la loi, ce que ne peut pas faire le juge administratif. \\
CE, 6 Novembre 1936, "Arrighi", le juge affirme qu'il n'est pas compétent pour juger de la constitutionnalité d'une loi. Le commissaire du gouvernement de l'époque disait dans ses conclusions que cela n'était pas possible à cause de l'idée du légicentrisme: le législateur est souverain et le juge ne peut donc pas interférer. \\
CE, 5 Janvier 2005, "Mlle Debrez", le juge affirme que si il ne contrôle pas la constitutionnalité des lois, c'est parce que le juge constitutionnel a la compétence exclusive. 

\subsection{Les limites à l'écran législatif}

\subsubsection{La théorie de l'écran transparent}

CE, 19 Novembre 1986, "Smanor", confirmé par CE, 17 Mai 1991, "Quintin". \\
Le juge peut contrôler l'acte administratif si il va plus loin que la loi. 

\subsubsection{Contrôle de constitutionnalité des lois antérieures à la constitution du 4 Octobre 1958}

Une loi peut devenir inconstitutionnelle suite au changement de constitution ou encore si une loi a été adoptée avant une révision, la révision nouvelle pouvant rendre non constitutionnelle la loi ancienne. \\
Le juge administratif peut donc constater qu'une loi est devenue inconstitutionnelle et va donc constater l'abrogation implicite de la loi. Cette théorie a été élaboré le 12 Février 1960, arrêt "Eky". CE, 16 Décembre 2005, "Syndicat national des huissiers de justice", confirmera ce contrôle. \\
Pour le juge administratif, ce n'est qu'un constat, il ne le décide pas. Quand la loi antérieure est devenu inconstitutionnelle, le juge ne l'appliquera pas et pourra donc contrôler l'acte administratif au regard de la constitution. \\
La question de la pérennité de ce contrôle peut se poser depuis l'avènement de la QPC. Pour l'instant, la réponse est positive et le CE continue d'appliquer ce contrôle. Cependant, il pourrait y avoir des conflits de point de vue, et certains auteurs suggèrent au CE d'abandonner ce contrôle.

\section{La QPC}

La QPC a été introduite lors de la révision de Juillet 2008 et permet un contrôle de constitutionnalité à postériorie par voie d'exception. La procédure est effective depuis 2010.


\subsection{Rappel de procédure de la QPC}

\subsubsection{Les étapes de la procédure}

La question est posée devant le juge ordinaire, par tout requérant. Le juge ordinaire va, dans un premier temps, renvoyer la question au juge suprême de son ordre (CE ou C.Cas) qui lui pourra envoyer la question au CC. La question est donc filtrée, d'abord par le juge ordinaire puis par le juge suprême de l'ordre juridictionnel en question. \\
Il existe des exigences procédurales. L'article 6, §1 de la CEDH s'applique à la procédure de la QPC et cette procédure a été modifiée pour y être conforme. Cependant, la question de la composition du CC peut poser des problèmes quant à l'exigence de l'impartialité. 

\subsubsection{Le caractère prioritaire de la QPC}

Le terme prioritaire semble indiqué que la question est prioritaire sur les autres. Cependant, ce caractère officiellement prioritaire ne l'est peut être plus. \\
La C.Cas s'est demandé, le 16 Avril 2010 "Melki et Abdeli", si la QPC était conforme au droit de l'Union. Elle dira que la QPC permet de maintenir des normes éventuellement contraire aux conventions européennes, or, maintenir temporairement une loi active alors qu'elle est contraire au droit de l'union, c'est violer le principe de primauté. De plus, la procédure de la QPC empêche de poser une question préjudicielle à la CJUE. Or, poser cette question, est une obligation pour les juridictions: art 267 du TFUE. \\
La C.Cas considère donc que la QPC est contraire au droit européen. La C.Cas posera donc une question préjudicielle à la CJUE pour savoir si la QPC est contraire au droit de l'Union. La doctrine estime que cette question a été posée par une volonté de la C.Cas de ne pas mettre en oeuvre la QPC. Cependant, la question n'est pas dénuée de sens, car un arrêt de la CJCE, 9 Mars 1978 "Simmenthal", considère qu'une question préjudicielle doit primer sur la question constitutionnelle. \\
Assez étonnamment, c'est le CC qui va répondre à la question le 12 Mai 2010 en profitant d'un contrôle à priori pour donner une leçon de droit à la C.Cas. Il rédigera deux pages de considérant abondants, et dira que la QPC est compatible avec le droit de l'Union. Le CC dira que tout juge peut prendre des normes provisoires et notamment suspendre toutes normes contraires au droit de l'Union, et le juge peut, à tout moment, poser une question préjudicielle à la CJUE. Suivant cette décision, la QPC n'est plus prioritaire sur une question de conventionnalité. \\
Cette solution du CC a été reprise deux jours plus tard par le CE, 14 Mai 2010 "Rujovic", et a été reprise elle même par la CJUE, 22 Juin 2010 "Melki et Abdeli", qui va reprendre à l'identique le raisonnement du CC: elle est compatible car le juge peut suspendre à tout moment une norme contraire au droit de l'union, et le juge peut poser une question préjudicielle à tout moment. \\
Cependant, il est rare qu'une QPC et qu'une question de conventionnalité se pose en même temps. Cela a été fait une fois par le CE le 31 Mai 2016. À noter que la question préjudicielle est tranchée en minimum 3 mois, si procédure d'urgence, or, la juridiction suprême (CE ou C.Cas), dispose de trois mois pour transmettre ou non la QPC. Le CE a rejeté la QPC, demandant au requérant de reposer la QPC après la question préjudicielle. 

\subsection{Les conditions de fond la QPC}


Elles sont posées par la constitution elle même (art. 61-1 de la constitution). La première condition est que seule les lois peuvent être contrôlées en QPC. \\
La deuxième est la norme à laquelle la loi va être contrôlée. Et c'est au regard des droits et libertés garantis que la loi va être contrôlée. \\
On peut se demander si on peut poser une QPC sur une loi abrogée ou modifiée. Cela est effectivement possible, QPC, 23 Juillet 2010 "Philippe E." car une loi abrogée peut toujours être applicable dans un litige. On appelle cela un retro-contrôle. \\
Concernant les ordonnances de l'article 38, on doit distinguer deux cas de figures. Si elle a été ratifiée, elle pourra faire l'objet d'une QPC. Sinon, si elle n'a pas été ratifiée, elle n'a qu'une valeur réglementaire et ne peut faire l'objet d'une QPC: CE, 11 Mars 2011, "Benzoni". C'est au juge administratif d'en contrôler la constitutionnalité, comme tout acte réglementaire. \\
Les lois d'habilitation, les lois de ratification, les lois de programmation, ne peuvent pas faire l'objet d'une QPC. Pour cette dernière, c'est ce qu'a jugé le CE, 18 Juillet 2011 "Fédération nationale des chasseurs". \\
Concernant les lois organiques, qui, en principe, peuvent faire l'objet d'une QPC. Cependant, les lois organiques sont forcément contrôlé sur le contrôle à priori. Dès lors qu'elles ont déjà été contrôlées, la QPC ne sera pas nouvelle sauf si les circonstances ont changées, et donc si la constitution a changée: CE, 29 Juin 2011, "Président de l'assemblée de la Polynésie française". 

\subsubsection{Les normes de contrôle}

Il s'agit uniquement des droits et libertés garantis par la constitution. On peut tenter de les identifier à la fois de manière positive et négative. \\
De manière positive, on peut trouver ces libertés dans toutes les sources constitutionnelles, mais principalement dans le préambule, donc DDHC, préambule de 1946, dans la jurisprudence. On les trouvera moins dans le texte lui même de la constitution. \\
Une exception: le principe de laïcité, assimilé aux droits et libertés, CC, QPC, 21 Février 2013 "Association pour la promotion et l'extension de la laïcité". \\
Négativement, ce qui ne relève pas de ces droits et libertés, sont les principes constitutionnels qui n'ont pas lien avec ces droits et libertés. L'organisation décentralisée de la République ne peut pas être invoquée dans une QPC par exemple. L'ensemble des règles procédurales prévues par la Constitution ne peut pas être invoquée, et notamment la procédure législative. Le CC ne contrôlera jamais la procédure législative en QPC, alors qu'elle le fait à priori. \\
Un objectif de valeur constitutionnelle peut être invoquée. Tout dépendra de l'objet, de l'objectif. L'objectif de bonne administration de la justice ne garantit aucuns droits ou aucunes libertés en lui même, alors que l'objectif de la parité garantit des droits et libertés. On peut avoir le même raisonnement pour les PFRLR. \\
Le contrôle peut-il être fait au regard de l'article 34 ? Peut-on invoquer l'incompétence négative dans une QPC ? L'incompétence négative est quand le législateur est resté en deçà de sa compétence. Il est possible d'invoquer cette incompétence, à une condition, à la condition que l'incompétence est elle même portée atteinte aux droits et libertés, CC, QPC, 18 Juin 2010, "SNC Kimberley Clark". 

\subsection{Les conditions de recevabilité}

Seules les juridictions suprêmes contrôlent les conditions, le CC se refuse à les contrôler. \\
La question doit être nouvelle, donc il faut que la loi n'ait pas fait l'objet d'un contrôle de constitutionnalité. Il faut que la loi n'ait pas été déclaré conforme dans les motifs ou les dispositifs. Le CC est sensé statuer ultra petita, il statue au delà des moyens: le CC est sensé avoir examiné toute la loi au regard de la constitution. Or, le CC ne le fait pas systématiquement: en contrôle à priori, il examine seulement les moyens invoqués. Si la disposition de la loi mise en cause n'apparaît pas dans les motifs, la QPC sera donc considérée comme nouvelle. \\
Il existe l'hypothèse selon laquelle la question peut être nouvelle même si la loi a déjà été contrôlée, en cas de changement de circonstances. Il y a trois hypothèses de changement de circonstance. Le premier est la modification de la norme contrôlée. Le deuxième est la nouveauté de la norme de contrôle, un changement de la constitution. Le troisième a été admis par la C.Cas, C.Crim, 20 Aout 2014, "Mouvement raelien international", qui considère qu'une décision CEDH qui condamne la France est un changement de circonstance et qui permet de poser la QPC. \\
Troisième condition de recevabilité, la question doit être sérieuse. Vérifier que la question est sérieuse, consiste à faire un pré contrôle de constitutionnalité. Pour le juge administratif, c'est très nouveau depuis 2010. \\
Quatrième condition, la QPC doit être posée dans un écrit distinct et motivée. 

\subsection{Le contrôle du CC dans le cadre de la QPC}

Le CC peut formuler des réserves d'interprétation. Le CC n'exerce qu'un contrôle restreint sur l'activité du législateur, il ne censure que les horreurs grossières du législateur.

\subsubsection{Le contrôle de constitutionnalité de l'interprétation de la loi}

Il est réalisé depuis deux décisions, une du 6 Octobre 2010, et une du 14 Octobre 2010. \\
Le fondement de cette solution est qu'il se réfère à la théorie du droit vivant. C'est une théorie réaliste du droit. Suivant cette théorie, c'est que le sens d'une norme ne se comprend pas à sa seule lecture. Le sens ne résulte pas de la norme mais de son interprétation. Des auteurs estiment que la norme n'existe pas tant qu'elle n'a pas été interprété. \\
Le CC ne contrôlera pas toutes les interprétations législatives, mais seulement celles des juridictions suprêmes, les interprétations constantes, il ne contrôlera que les interprétations fondées sur un acte législatif. \\
Certains auteurs ont eu peur que le CC deviennent la juridiction suprême qui asservirai les juridictions suprêmes. Cette crainte n'est pas fondée car ce sont le CE et la C.Cas qui filtrent, et peuvent donc décider de ne pas transmettre la question. CE, 14 Septembre 2011, "Mr Pierre", à l'occasion d'une transmission d'une QPC, le CE a changée son interprétation afin de la rendre conforme à la constitution. On peut se demander ici si il n'y a pas un soucis d'impartialité. Cela n'en pose pas, d'après le CE, 12 Septembre 2011, "Dion". 

\subsubsection{Les effets du contrôle}

Ils sont précisés dans l'article 62 de la Constitution. Si la loi est contraire à la constitution, elle est abrogée, et donc annulée pour l'avenir et non pas pour le passé. \\
Le CC peut aussi moduler dans le temps les effets de sa décision. Cette modulation permet au législateur de prendre le temps de régler son erreur et de prendre de nouvelles dispositions. \\
Dès lors que la loi est abrogée, la loi ne peut plus être appliquée dans l'instance en cours ni dans toutes les autres en cours. C'est ce qu'a jugé le CE en assemblée, le 13 Mai 2011, "M'Rmeda". À l'inverse, si le juge module dans le temps sa décision, la loi est encore applicable à l'instance en cours. Le CE l'a rappelé dans un arrêt du 14 Novembre 2012 "Association France nature environnement". \\
En QPC, le CC peut lui aussi poser une question préjudicielle à la CJUE. C'est ce qu'il a fait le 4 Avril 2013 "Jeremy Forrest". \\
CC, 24 Juillet 2015 "Déchéance de nationalité", le CC refuse de poser une question préjudicielle. 

\chapter{Les conventions internationales}

Les traités n'ont pas toujours fais parti de la légalité. En 1946, on considérait le respect du traité comme naturel et non pas comme juridique. L'article 26 de la Constitution de 1946 donne "force de loi" aux traités internationaux. Le juge administratif va alors accepter la conformité d'un acte administratif aux traités. CE, 1952 "Kirkwoud". \\
Les choses ont encore changées en 1958. L'article 55 donne une valeur supérieure aux lois aux traités. \\
Si un traité a une valeur supra-législative, elle a toujours une valeur infra-constitutionnelle. Le CE le dit dans son arrêt "Saran". 

\section{Les conditions d'applicabilité des normes internationales}

De l'applicabilité va dépendre l'invocabilité: la possibilité d'appliquer une norme internationale devant une juridiction. Il existe trois conditions: l'effet direct, la ratification régulière, le principe de réciprocité.

\subsection{L'effet direct}

L'effet direct est caractérisé dans Assemblée CE, 11 Avril 2012 "GISTI": pour identifier un effet direct, il faut se baser sur les termes du traités mais aussi sur l'intention générale des partis ou encore l'économie générale du traité. Il faut donc se référer dans à la lettre qu'à l'esprit du traité. \\
Nous notons deux critères et une réserve. Pour qu'une norme soit d'effet direct, il faut qu'elle soit "inconditionnelle et suffisamment précise". Le deuxième critère tient au destinataire de la norme: pour être d'effet direct, la norme ne doit pas avoir pour objet de régir uniquement les relations entre États. Si, il y a dans un traité "L'État signataire s'engage à respecter tel droit ou telle liberté", ce dit traité créera nécessairement un droit ou une liberté. Ces normes sont d'applicabilité directe depuis 2012. \\
Concernant la nuance, elle concerne les conventions auxquelles l'UE est parti. En 2012, au CE, le juge considère qu'il n'est pas compétent pour juger de l'effet direct de ces conventions, seule la CJUE serait compétente. En 2012, la CJUE dans une affaire "Iata", elle dit que le juge national est le juge de droit commun du droit de l'union. Le CE a donc corrigé sa maladresse le 6 Décembre 2012 "Air Algérie". En cas de difficulté sérieuse d'interprétation, le CE peut poser une question préjudicielle à la CJUE.


Deux remarques sont à faire: le juge apprécie la convention, non pas dans sa globalité, mais stipulations par stipulations. Vu que le CE a étendu l'effet direct en 2012, cela s'est appliqué à des conventions qui ont effectivement des effets directs, comme en 2014 où il a reconnu des normes d'effet direct dans la charte sociale européenne. 

\subsection{La régularité de la ratification}

L'article 55 de la Constitution dispose qu'un traité doit être régulièrement ratifié et publié. Le CE, pendant longtemps, ne contrôlait pas cette condition. Pour lui, la ratification constituait un acte de gouvernement et qui sont donc insusceptible de recours au contentieux. Suivant l'article 53 de la Constit., c'est en principe un décret présidentiel qui ratifie un traité. Cependant, il y a des hypothèses où il faut une loi pour ratifier un traité, notamment ceux qui engagent les finances de l'État. \\
Le CE a d'abord accepté de contrôler la publication du traité le 30 Octobre 1964, "Société Porosagor". Finale, à la fin des années 90, le CE va accepter de contrôler l'acte de ratification, au regard de l'article 53 de la Constitution, CE Assemblée, 18 Décembre 1998, "SARL Parc d'activité Blodzheym". \\
Le juge ne peut contrôler que les décrets de ratification et peut simplement dire, si, à la place du décret, il fallait une loi. Au contraire, il ne peut pas effectuer un contrôle de la ratification via une loi, ce refus date d'un arrêt du 8 Juillet 2002 "Commune de Porta". Ce contrôle peut être effectué par voie d'action (quand un requérant conteste directement le décret de ratification) mais aussi par voie d'exception d'illégalité: CE, Assemblée, 5 Mars 2003 "Aggoun". 

\subsection{Condition de réciprocité}

Cette condition est aussi contenue à l'article 55 de la Constitution. Pour être appliqué, un traité doit être appliqué par l'autre État. Pendant longtemps, le CE n'a pas contrôlé cette condition. Le CE renvoyait cette question au ministre des affaires étrangères, qui serait le mieux placé pour savoir si la condition est rempli, CE, Assemblée 19 Mai 1981 "Rekhou". \\
In fine, c'est donc le ministre qui décidait d'appliquer une norme dans l'ordre interne. Ministre qui serait donc juge et parti dans l'affaire. \\
La CEDH a remis en cause cela le 13 Février 2003, "Chevrol", elle ne condamne pas la demande d'avis en elle même mais considère que le JA ne doit pas être lié par l'avis du ministre. Si c'était le cas, il y aurait une ingérence du ministre dans la fonction de juger. Suite à cet arrêt, le CE n'a pas réagi rapidement, il a fait de la résistance. CE, Assemblée, 9 Juillet 2010 "Cheriet-Binseghir", le CE se déclare compétent pour apprécier lui même la condition de réciprocité et ne demande plus l'avis du ministre. \\
Cette condition de réciprocité n'est pas exigée dans un cas: dans les traités protégeant les droits fondamentaux, qui visent justement à sanctionner les États qui ne respectent pas ces droits. CE, Assemblée, 21 Décembre 1990, "Confédération nationale des associations familiales catholiques", solution confirmée par le CC lui même en 1999: "traité portant statut de la CPI". 

\section{Comment le juge administratif garantit le respect des normes internationales ?}

\subsection{Quelle est l'étendue du contrôle de conventionnalité ?}

Depuis les années 1950, le JA effectue un contrôle de conventionnalité au regard des conventions internationales depuis l'arrêt "Kirkoud" en 1952. Une difficulté s'est présenté assez rapidement, quand un acte administratif est fondée sur une loi: contrôler la conventionnalité de cet acte revient à contrôler la conventionnalité de la loi, ce qu'il ne peut faire que à voie d'exception.

\subsubsection{Affirmation du principe du contrôle de conventionnalité des lois par voie d'exception}

Il y a eu une évolution, qui a débuté par une décision du CC, le 15 Janvier 1975, "IVG", dans laquelle le CC se déclare incompétent pour contrôler une loi vis à vis des normes internationales car la constitution ne lui donnerai pas cette prérogative. Le Conseil explique aussi que le contrôle de conventionnalité n'a pas la même nature que le contrôle de constitutionnalité qui serait absolu et définitif (il est considéré comme tel car pas de QPC en 1975). \\
Le contrôle de conventionnalité ne l'est pas car la condition de réciprocité elle même est variable. \\
Aujourd'hui, les moyens du conseil sont anachroniques car le contrôle de constitutionnalité n'est plus absolu et définitif depuis l'avènement de la QPC. Depuis quelques années, le CC ne reprend plus cette motivation et se fonde exclusivement sur la motivation textuelle: la constitution ne lui donne pas compétence pour ce contrôle. \\
Le CC contrôle parfois, depuis 2006, des lois de transposition au regard des directives que la loi transpose: CC, 27 Juillet 2006, "DADVSI". \\
Le CC pourrait se décider à contrôler la conventionnalité des lois. Jean-Louis Debré a proposé certaines choses dont ce contrôle de conventionnalité. Il y a donc de fortes chances que le CC fasse évolué sa jurisprudence. 


La décision IVG était en quelques sortes un "appel du pied" aux autres juridictions ordinaires d'effectuer un tel contrôle. Les juridictions ordinaires vont suivre la décision du CC. La C.Cas va rapidement suivre le CC, C.Cas, chambre mixte, 24 Mai 1975 "Jacques Vabre". Si la C.Cas a directement accepté de suivre le CC, ce n'était pas le cas du CE. \\
Le CE a toujours accepté de contrôler une loi antérieure à un traité au regard de ce traité. Mais, à l'inverse, le CE ne contrôlait pas la loi postérieure au regard de ce traité: CE, 1er Mars 1958 "Semoule". Deux raisons ont motivés cette décisions, d'abord, le CE ne veut pas remettre en cause la dernière expression de la souveraineté nationale. La deuxième raison est que le CE pensait que faire primer un traité sur une loi postérieure, ça revient à dire que la loi est contraire à l'article 55 de la Constitution donc à faire un contrôle de constitutionnalité de la Loi. \\
Pendant longtemps, le CE a mis en oeuvre la théorie de la loi écran, lorsque la Loi postérieure est entre l'acte administratif et le traité, il n'effectue pas le contrôle de conventionnalité. \\
Le CE a mis près de 15 ans à abandonner cette position et a accepté de contrôler toutes les lois au regard des conventions le 20 Octobre 1999, "Nicolo". CE, 3 Décembre 2001, "SNIP", le CE accepte de contrôler une loi au regard des PGDC. \\
Le CE accepte même de contrôler les lois organiques au regard des traité, le 16 Avril 2016 "CSM".  \\
Ce contrôle a un effet sur les autorités administratives qui ne peuvent pas édicter des actes d'une loi qui serait contraires aux conventions internationales. CE, 24 Février 1999 "Association des patients de la médecine".

\subsubsection{Les limites du contrôle de conventionnalité}

La première limite est que le juge ne contrôle pas les lois au regard de toutes les normes internationales car certaines normes n'ont pas la même valeur que les traités comme les principes généraux du droit international: CE, 28 Juillet 2000, "Paulin". La coutume internationale n'a pas la même valeur que les traités, et par conséquent, cette coutume ne prime pas sur le droit interne. CE, Assemblée, 6 Juin 1997 "Aquarone". \\
La valeur juridique de la coutume est donc probablement législative voire infra-législative comme le laisse pensé un arrêt du 14 Octobre 2011, "Saleh". Il y a une idée qui serait que la coutume internationale serait un droit général, qui serait détrôné par un droit spécial. \\
La deuxième limite de ce contrôle tient aux limites de l'étendue du contrôle. Il y a en effet un contrôle de compatibilité et non de conformité. C'est une limite importante mais permet au juge d'avoir une grand marge de manoeuvre. \\
Il y a une troisième limite qui tenait à ce que le juge des référés n'effectue pas un tel jugement. Le CE considérait qu'il n'était pas compétent pour un tel contrôle: CE, Assemblée, 2002 "Carminatti". C'est une solution qui remettait en cause la primauté du droit de l'UE. Le CE a abandonné cette jurisprudence et l'a abandonné solennelement le 31 Mai 2016, assemblée référée "Gonzales Gommes". Cet arrêt vient mettre en place un contrôle de conventionnalité "in concreto"

\subsubsection{L'évolution du contrôle: un contrôle in concreto}

Le juge ne vérifie pas seulement si la loi est compatible avec les conventions, mais si les actes administratifs sont compatible avec la convention. Dans l'affaire "Gonzales Gommes", la loi sera conforme aux conventions mais pas l'acte qui met en oeuvre la loi. \\
Juridiquement, cet arrêt ne mériterait que des critiques, notamment au regard de la hiérarchie des normes, la loi étant compatible, mais pas l'acte. Cela pourrait même revenir à dire que l'acte est plus important que la loi. En l'espèce, le juge fait primer un droit individuel sur une règle générale. Dans cette décision, le juge porte plus d'importance à la loi étrangère qu'à la loi française. Si l'on suit le raisonnement du CE, cela signifierait que l'administration, si elle cherche à respecter la convention, serait obligé déroger à la Loi. 


\subsection{Comment le juge interprète une convention internationale ?}

\subsubsection{La reconnaissance du pouvoir d'interprétation d'une convention internationale}

Pendant longtemps, le CE s'est déclaré incompétent pour interprété une convention internationale. Il renvoyait donc la question au Ministre des affaires étrangères comme la condition de réciprocité. \\
Là aussi, le CE a évolué, en Assemblée, le 29 Juin 1990 "Gisti", le CE accepte d'interprété lui même un traité international. Le juge peut parfois encore demander son avis au ministre mais il n'est plus lié par celui-ci. \\
Ce revirement a entraîné quelques difficultés. Certaines interprétations qu'il faisait était parfois contraire à l'interprétation de certaines juridictions internationales.

\subsubsection{La question préjudicielle à la CJUE}

En matière de droit de l'UE, les choses sont différentes, la CJUE dispose d'un monopole d'interprétation des normes de l'UE. En principe, si se pose une difficulté d'interprétation du droit de l'Union, il faut poser une question préjudicielle à la CJUE. Cette question est obligatoire, surtout pour les juridictions suprêmes comme le CE. Le CE n'aimait pas poser ces questions car ne souhaitait pas être lié par la position du juge européen. \\
Pour éviter de poser ces questions, le CE avait développé la théorie de l'acte clair, selon laquelle tous les actes de l'Union sont claires. \\
Le CE a abandonné cette théorie au début des années 2000. Le CE respecte totalement l'interprétation du juge de l'Union depuis l'arrêt dit "des oignons".


Le CC accepte aussi parfois de poser des questions préjudicielles à la CJUE depuis la QPC du 4 Avril 2013 "Jeremy Forest" et le mandat européen. Soit le droit de l'Union ne laissait aucune marge de manoeuvre et donc le juge constitutionnel n'avait aucun contrôle à faire. Soit il y a une marge de manoeuvre laissé à l'État membre pour la transposition, et dans ce cas là, le juge constitutionnel peut contrôler la conformité de la loi à la constitution. Le CC a donc posé une question préjudicielle pour savoir si la décision cadre laissait une marge de manoeuvre de transposition aux États membre. La CJUE dira que oui et donc le CC a contrôlé la disposition sur le mandat européen au regarde de la constitution. \\
C'est la première fois que le CC pose une question préjudicielle. Il le fait dans un contexte très particulier, celui du mandat européen dont le respect est constitutionnel. D'habitude il n'en pose pas à cause des délais. Cette évolution tient à une évolution au niveau européen: un dialogue se crée entre les cours constitutionnelles et le juge de l'Union: la même année, la cour constitutionnelle Espagnole et Allemande ont aussi posé des questions préjudicielles. Ces nouvelles relations permettent de mieux prendre en compte le droit de l'Union. \\
Cette décision du CC montre que la question de conventionnalité prime sur la question de constitutionnalité. \\
Le CC a pris un risque en posant cette question. Selon le TFUE, seule les juridictions (donc impartiales et indépendantes) peuvent poser des questions à la CJUE. Vu les critiques qui sont faites au CC, la CJUE aurait pu considérer que le CC n'est pas une juridiction. 

\subsubsection{La future procédure d'avis consultatif devant la CEDH}

Cette future procédure est prévu par le protocole numéro 16, qui met en place une sorte de question préjudicielle que pourront poser les plus hautes juridictions. Comme à la CJUE, les plus hautes juridictions pourront poser des questions concernant l'interprétation de la convention. \\
Le protocole 16 entrera en vigueur dès qu'il sera ratifié par au moins 10 États membres, ce qui n'est pas le cas. \\
Cette question pourrait être posé par les plus hautes juridictions mais en réalité, il reviendrai aux États membres de définir les juridictions qui pourraient formuler cette demande d'avis. En France, ce pourrait être la C.Cas et le CE. Mais le CC pourrait ne pas être concerné car il ne juge pas de la conventionnalité. \\
L'avis serait seulement consultatif, il ne serait donc pas contraignant mais on peut penser que ces avis seraient respectés par crainte de voir la CEDH condamner le pays qui ne respecterait pas la convention. 


\subsection{Le contrôle des normes internationales}

\subsubsection{L'exception d'inconventionnalité d'une norme internationale}

CE, Assemblée, 23 Décembre 2011 "Kandyrine". À l'occasion de son recours, Mr Kandyrine considérait qu'une convention franco-russe était incompatible avec la CEDH et demandait donc de l'écarter. Se pose une question importante: le CE peut il apprécié l'exception d'inconventionnalité. \\
En principe, ce contrôle ne peut pas exister, mais si il ne peut pas exister, le juge pourrait essayer de les concilier.


Le juge ne peut pas contrôler un traité au regard d'un autre traité. Cette absence de contrôle a déjà été affirmé en 1998 "SARL Bloodzeim", et l'avait encore confirmé en 2002 "Commune de Porta". Cette solution de principe s'explique par deux raisons principales: il n'y a pas de relations hiérarchiques entre les conventions internationales ; les conventions ont un effet relatif, une convention ne peut pas lié les États qui n'ont pas signé cette convention. \\
Dans les années 2000, le CE avait parfois accepté de contrôler une norme communautaire au regard des traités européen. Cela revient déjà à contrôler une norme internationale par rapport à une autre.


Le juge peut essayer de concilier les conventions internationales. Très concrètement, le CE va apprécié de la compatibilité du traité franco-russe avec la CEDH. \\
Il est à préciser que c'est une conciliation et non un contrôle, même si cela aboutit à cela, au final. Ce contrôle est fait par voie d'exception. Ce contrôle n'est pas une condition de validité du traité, c'est au contraire une condition d'applicabilité du traité: pour que le traité franco-russe soit applicable, il faut qu'il soit conciliable avec la CEDH. Le juge réserve une hypothèse dans laquelle le traité relèverait du droit de l'Union et comme il existe une obligation constitutionnelle du respect du droit de l'Union, celui ci prime sur les autres conventions. \\
Enfin, le juge explique quelles sont les modalités de conciliation entre les traités. Il va se référé aux principes de droits coutumiers relatif à la combinaison des normes de droit international. \\
Le juge national a un rôle nouveau, il se permet de contrôler des normes internationales. Le rôle du CE a donc bien évolué. 

\subsubsection{L'exception d'inconstitutionnalité d'une norme internationale}

Assemblée, 1996 "Coné", interpréter une convention conformément à la constitution, cela consiste à donner un sens qui soit conforme à la Constitution. Le JA ne peut jamais contrôler la constitutionnalité d'un traité. 2002, "Commune de Porta", et le réaffirme le 9 Juillet 2010 "Fédération nationale de la libre pensée". \\
Le rapporteur public préconisait d'effectuer un tel contrôle, celui-ci n'a pas été suivi. Le CE ne l'a pas fait car cela relève de la compétence exclusive du CC. La deuxième raison tient à ce que le contenu d'un traité relève des relations diplomatiques de la France et cela rentre dans la catégorie des actes de Gouvernement, qui ne sont pas susceptibles d'être contrôlés par un juge. La troisième raison est que ce contrôle ne peut se faire qu'à priori, c'est ce qu'a décidé le CC lui même le 25 Janvier 1985 où il refuse de contrôler un traité déjà promulgué. 


\section{La responsabilité de l'État du fait du non respect du droit international}

Si l'État ne respecte pas un traité et si ce traité porte préjudice à un administré, cela conduit à une faute qui engage la responsabilité de l'État français. Le CE a mis du temps à reconnaître cette responsabilité. Dans l'hypothèse où une loi était contraire au droit de l'Union, le CE n'a pas voulu reconnaître la faute du législateur. En 1984, le CE admet la responsabilité de l'État mais reconnaît une responsabilité sans fautes alors que l'acte réglementaire était manifestement contraire: Assemblée, 23 Mars 1984 "Alivar". \\
Face à une loi contraire au droit de l'Union, le CE a considéré que c'est l'acte réglementaire qui était en faute: Assemblée, 28 Février 1992, "Arizona tobaco products". \\
La CJCE a, au contraire, considéré qu'une loi contraire au droit de l'Union constituait une faute: CJCE, 20 Octobre 1996 "Dillen Kofer". \\
Il a fallu attendre 2007 pour que le CE évolue. CE, Assemblée le 8 Février 2007 "Gardedieu", dans lequel le CE reconnaît la responsabilité de la puissance publique du fait d'une loi contraire à une norme internationale. Cette décision est étonnante car le CE parle de responsabilité mais ne précise pas si il y a faute ou non. Si une loi est contraire à la norme internationale, il devrait y avoir une responsabilité pour faute et le CE refuse de l'admettre. 

\chapter{Le droit de l'Union Européenne}

\section{La spécificité du droit européen}

Le droit de l'Union présente aujourd'hui des spécificités particulières, qui concernent les sources même du droit européen, et les spécificités de l'ordre juridique de l'UE. 

\subsection{La spécificité des sources}

On distingue d'une part le droit originaire, et le droit dérivé. Le premier droit vient des traités, le TUE et le TFUE. À côté des traités, on trouve le droit dérivé, édicté par les institution de l'Union Européenne. Il y en a plusieurs catégories: les règlements de l'Union Européenne, obligatoire dans tous leurs éléments et sont directement applicables. \\
Les directives sont bien différentes des règlements: elles lient les États membres à un résultats, les États sont libre des formes et des moyens pour atteindre ces résultats. Les États transposent donc les directives dans leur droit interne via une loi ou par un règlement. \\
On trouve aussi en droit dérivé des décisions individuelles comme des décisions de la commission européenne qui sont obligatoires pour leur destinataire. \\
Il y a enfin du droit souple, de la soft law, qui n'est pas obligatoire pour les destinataires de ce droit. 


Deux remarques: les PJDUE dégagés par la jurisprudence européenne ont, pour le CE, la même valeur que les traités: CE, 3 Décembre 2001, SNIP. \\
Le droit communautaire n'existe plus depuis 2009. Le TFUE a supprimé la communauté européenne, on devrait parler de droit de l'Union. 

\subsection{La spécificité de l'ordre juridique de l'Union}

Concernant cet ordre juridique, la CJCE à l'époque avait dégagé un principe de primauté du droit communautaire (et donc maintenant du droit de l'Union): CJCE, 1964, Costa c. Enel. Raisonnement en deux temps: les traités ont "institués un ordres juridique propre, intégré aux ordres juridiques des États membres et qui s'imposent à leurs juridictions". Une norme interne ne peut donc pas être contraire au droit communautaire. \\
CJCE, 1978, Simmenthal, le juge explique que toute autorité nationale doit laisser inappliqué une norme nationale qui serait contraire au droit communautaire (et donc au droit de l'union aujourd'hui). Si un État contrevient à cette règle, l'État membre peut être responsable de cette violation: CJCE, 1991, Francovitch. 


Le principe posé par la CJCE est contraire à un principe qui s'y oppose: le principe selon lequel c'est la Constitution qui s'impose à tout dans l'ordre juridique interne: arrêt Sarran. \\
Les juridictions nationales ont tentés de concilier ces deux principes antagonistes. D'après le CE et le CC, le droit de l'Union a toujours une valeur conventionnelle mais les juridictions considèrent que le droit de l'Union est une obligation prévue par la constitution Française. \\
CC, 10 Juin 2004, Économie numérique, dans laquelle le CC dégage une obligation constitutionnelle du respect de certaines norme de l'Union, et de transposition des directives, résultant de l'article 88-1 de la Constitution. \\
TC, 17 Octobre 2011, SCEA du Cheneau, le juge considère que le respect de l'ensemble du droit européen est une obligation constitutionnelle. 


\section{Le contrôle de la transposition des directives}

\subsection{L'obligation de transposition des directives}

Transposé une directive de l'UE est une obligation conventionnelle. Transposer une directive est aussi une obligation constitutionnelle (88-1 Constitution). L'État membre n'est obligé qu'à un résultat concernant la directive, l'État est libre de la forme et des moyens. C'est ce que l'on appelle le principe d'autonomie procédurale. 


Même si l'État est libre, il est à noté que le juge de l'Union ne se satisfait pas d'une jurisprudence. \\
La transposition se fait aussi dans un certain délai (entre 6 mois et 2 ans). CE, Ass, 3 Février 1989, Alitalia, le CE considère que si une norme est contraire à une directive non transposée, cette norme peut être abrogée par l'autorité administrative. \\
En pratique, les directives sont transposées par les ordonnances de l'article 38. 

\subsection{L'effet direct des directives}

La norme nationale n'est pas compatible avec la directive soit si elle a pas été transposée ou si elle a été mal transposée. \\
La question concerne donc le contrôle d'un acte administratif au regard d'une directive. \\
Cela ne pose pas de problème au regard d'un règlement car il s'impose dans sa globalité à l'État membre. Ce n'est pas le cas d'une directive. 


Le juge a pendant longtemps distinguer deux situations: la première est si l'acte contesté est réglementaire. Un particulier peut invoquer la directive contre l'acte: CE, 7 Décembre 1984, FFSPN ; précisé en 1992, on peut invoquer une directive contre un acte réglementaire même si il est pris en application d'une loi: CE, Ass, 28 Février 1992, Rothlans. 


Peut-on invoquer une directive contre un acte individuel de la puissance publique ? \\
Avant 2009, le juge considérait que la directive n'avait pas d'effet direct à l'égard des particuliers tant que qu'elle ne créait pas de droit et d'obligations à l'égard de ceux-ci: CE, Ass, 22 Décembre 1978, Cohn-Bendit. Cette solution était largement contraire au droit de l'Union. \\
La CJCE, le 4 Décembre 1974, Van duyne, explique qu'une directive peut avoir un effet direct si elle est inconditionnelle et suffisamment précise. On parle ici d'un effet direct vertical, puisque concerne le droit de l'union d'une part et le particulier d'autre part. \\
Effet direct horizontal: Union et État membre. \\
Le CE a atténué sa décision intolérable à partir des années 90. Si la directive n'a toujours pas d'effet direct vertical, un particulier peut invoquer la directive contre un acte individuel si celui-ci est fondé sur une autre norme nationale. CE, Ass, 30 Octobre 1996, Revert et Badonon, dans laquelle il est possible d'invoquer la directive contre un acte individuel dès lors que l'acte est pris sur le fondement d'une loi. \\
CE, 6 Février 1998, Mr Tete, dans laquelle la directive peut être invoqué si l'acte individuel est pris sur le fondement d'une règle jurisprudentiel. 


En pratique, on peut toujours contester l'acte individuel, mais en théorie, le CE, avec son arrêt Cohn-Bendit a violé le droit de l'Union pendant plus de 20 ans. \\
CE, Ass, 30 Octobre 2009, Perreux, dans lequel le juge rappelle deux choses: l'obligation de transposition de la directive ; l'on peut invoquer une directive à l'encontre d'un acte réglementaire (FFSPN), évolution: mise à mort de l'arrêt Cohn-Bendit puisqu'il est désormais possible d'invoquer une directive à l'encontre d'un acte individuel si deux conditions sont réunies: si la directive est précise et inconditionnelle et il faut que la directive n'ait pas été transposé dans les délais. 

\section{Le contrôle de constitutionnalité des actes de transpositions des directives}

\subsection{Le contrôle des lois de transposition}

Contrôler une loi de transposition, c'est contrôler une directive à l'égard de la constitution. Cela irait à l'encontre du principe de primauté. \\
Le CC a dégagé en 2004 une obligation constitutionnelle de transposé les directives. Suivant cette obligation, transposé est une obligation constitutionnelle. \\
CC, 27 juillet 2006, DADVSI, le CC va contrôler une loi de transposition au regard de la directive qu'elle transpose, c'est contrôler l'exacte transposition de la directive. Le CC va considérer que ce contrôle est constitutionnel car la transposition est constitutionnalisé. Ce contrôle a l'apparence d'un contrôle de conventionnalité mais est fondé sur l'art. 88-1 de la constitution. \\
Ce contrôle est limité à un double point de vue: quant à son étendue, le juge limite son contrôle à l'incompatibilité manifeste ; la deuxième limite est une hypothèse où le CC n'effectuera pas son contrôle si la directive est contraire à Principe Inhérent à l'Identité Constitutionnelle de la France: dans cette hypothèse, ce principe doit primer. \\
Les juridictions n'ont jamais défini les P2ICF. Pour la doctrine, ce serai des principes constitutionnels qui n'auraient pas d'équivalent dans les autres pays. Les exemples types sont la laïcité et la continuité du service publique. \\
Dans l'hypothèse normal, le CC fait primer le droit de l'Union alors que dans la logique du P2ICF, il fera primer celui-ci. \\
Ce contrôle des lois de transpositions n'est jamais effectuées en QPC, CC, 2010 "jeu de hasard". Il ne le fait pas car l'article 88-1 n'est pas invocable car ne relève pas des droits et libertés garantis par la constitution. \\
Le CC, dans l'affaire "Jeremy Forest", avait à contrôler une loi de transposition d'une décision cadre. Dès lors que l'acte de l'Union ne laisse aucune marge de manoeuvre, le juge ne contrôlera pas la norme de transposition au regard de la constitution. 


\subsection{Le contrôle de constitutionnalité des actes administratifs de transposition}

CE, Ass, 8 Février 2007, Arcelor. Comme dans la loi, contrôler l'acte revient à contrôler la directive. Le CE rappelle la suprématie de la Constitution, et donc l'obligation constitutionnelle de transposition des directives. Le contrôle de constitutionnalité des actes de transpositions doit donc se faire selon des modalités particulières. \\
Le juge va distinguer deux situations: 1. Le juge constate qu'il existe un principe européen équivalent au principe constitutionnel invoqué, le JA va contrôler la directive au regard du principe européen, et si il y a une difficulté sérieuse, il posera une question à la CJUE. \\
2. Si, à l'inverse, il n'y a pas de principe équivalent en droit de l'Union, le CE contrôlera directement l'acte de transposition au regard de la constitution. \\
Il y a très clairement ici une conciliation entre la suprématie de la constitution et la primauté du droit de l'Union. Dans la première situation, le juge cherche très clairement à éviter la confrontation avec la constitution. Dans la deuxième situation, le juge va faire primer la constitution: c'est l'hypothèse des P2ICF. \\
Première application de cet arrêt: CE, 12 Décembre 2011, Air Algérie. Le CE refusera d'appliquer la solution Arcelor, car il s'agissait ici d'une loi de transposition et non d'un règlement. \\
Deuxième application dans l'affaire des OGM en 2016. 

\section{Le contrôle de conventionnalité du droit dérivé par le juge interne}

Le JA peut-il contrôler les directives au regard d'autres normes internationales ? Ce contrôle a pendant longtemps été refusé, car il refusait de contrôler une norme internationale au regarde d'une autre. Deuxième raison: seul la CJUE peut censurer une norme du droit de l'Union, c'est une compétence exclusive qui résulte d'un arrêt de la CJCE, 22 Octobre 1987, Foto-Frost. \\
Le premier argument, le CE est passé outre en 2011, quand il a concilié deux traités ensemble. Certes, seul la CJUE a le monopole de l'annulation, mais la CJUE n'a pas le monopole de l'interprétation de ces normes, mais plus encore, le juge national est le juge de droit commun et c'est précisément en tant que tel que le CE va accepter progressivement de contrôler au regard des conventions. \\
Le premier temps de cette évolution est l'arrêt Arcelor, le CE contrôlant une directive au regard d'un principe de droit européen. \\
Deuxième temps: CE, 10 Avril 2008, CNB, était en cause une directive de l'union, transposé par une loi et appliqué par un règlement, contesté. Le CE a considéré que le règlement, appliqué en l'espèce était incompatible avec la CEDH. Pour la première fois, le CE va contrôler une directive de l'union par rapport à la CEDH. Il est à noter que tous les membres de l'UE sont membre de la CEDH, l'inverse n'est pas vrai, mais ça reste une raison pour que le CE accepte un tel contrôle. Les droits garantis par la CEDH sont aussi garantis dans le droit de l'union, ce sont des PJD, d'après le TUE. \\
Troisième temps, affaire "Air Algérie", le juge contrôle une directive au regard d'un traité conclu par l'UE. Depuis l'affaire Gisti de 2012, le CE apprécie lui même l'effet direct des conventions conclues par l'Union. Ce contrôle est assez logique car les conventions conclues par l'Union sont supérieures au droit de l'Union. \\
Quatrième temps, CE, 1er Août 2013, AGPM, concernant les OGM. Était en cause un règlement de l'UE que le requérant considérait comme contraire au principe de précaution de l'article 5 de la Charte de l'environnement. Ce type de contrôle aurait conduit à violer le principe de primauté. Le CE va rechercher une norme équivalente en droit européen: l'article 191 du TFUE. 


\chapter{La loi et les règlements}

La définition de la loi comme "norme générale et impersonnelle" est trop simple. On peut lui apposer une définition organique: la norme adoptée par le Parlement. Mais il existe des textes législatifs qui ne vienne pas du Parlement, comme les ordonnances. On peut donc aussi avoir une approche formelle: la loi est un texte qui a valeur législative. On peut aussi lui donner une définition matérielle, l'objet de la loi: ce qui relève du champ d'application de l'article 37 de la constitution. \\
Concernant le règlement, les choses sont plus simple, la loi le définit à l'art. L200-1 du CRPA: l'acte réglementaire est un acte décisoire (qui produit des effets juridiques) ; c'est un acte qui vise des personnes de manière abstraite (au contraire de l'acte individuel qui désigne des personnes de manière nominative). L'acte réglementaire est susceptible d'un recours pour excès de pouvoir. \\
Il existe une dégradation de la norme: l'inflation législative, il y a trop de lois, cela contrarie un objectif à valeur constitutionnelle, l'accessibilité et l'intelligibilité de la norme ; tous ces textes n'ont pas la même qualité normative, les textes devraient être général mais les textes sont de plus en plus précis et vont très loin, tout en prenant même des dispositions qui n'ont aucune portée juridique (ce sont les dispositions non normative), que le CC censure depuis le 21 Avril 2005 "Loi pour l'avenir de l'école".

\section{Les domaines de la loi et du règlement}

Les domaines sont posés aux articles 34 et 37 de la constitution. L'article 37 définit la portée du pouvoir réglementaire, qui est une compétence de principe: le gouvernement a compétence dans tous les domaines où le législateur n'est pas compétent. \\
Le législateur, à l'article 34, a des compétences d'attributions. \\
Le CC contrôle l'incompétence négative du législateur, c'est à dire l'hypothèse où le législateur n'exerce pas toute sa compétence et la délaisse au pouvoir réglementaire. Cela est censuré par le CC, depuis le 17 Janvier 2002: "Loi relative à la Corse". \\
La répartition va céder devant une disposition constitutionnelle spéciale qui peut déroger à cette répartition comme l'article 1er de la constitution qui donne à la loi la compétence de régler la parité. CE, Ass, 7 Mai 2013, CFTC-AGRI, le CE a annulé un décret du PM mettant en place la parité dans des chambres de commerce, jugeant qu'il était incompétent.  

\section{Les empiétements}

\subsection{L'empiétement du législateur sur le domaine réglementaire}

En cas d'empiétement de la loi sur les règlements, le Gouvernement dispose de deux procédures. Une avant le vote, une après. \\
La procédure à priori est disposé à l'article 41 de la Constitution et le gouvernement peut déposer une exception d'irrecevabilité devant le président de l'AN, il peut saisir le CC. \\
La procédure à posteriori est celle de l'article 37.2 de la constitution, qui pose une procédure de dé-légalisation, le gouvernement va saisir le CC et si celui-ci constate un empiétement, la disposition est dé-légalisée et n'a plus qu'une valeur réglementaire.


Sur la question de l'obligation pour le gouvernement de mettre en oeuvre ces procédures, il y a une évolution en trois temps. \\
Dans sa décision du 30 Juillet 1982 "blocage des prix", le CC considère que la procédure de l'article 37 est facultative et que le gouvernement peut accepter l'empiétement et ne le censure donc pas. \\
Dans sa décision du 21 Avril 2005 "Loi sur l'école", le CC refuse toujours de censuré une disposition qui aurait empiété mais le Conseil accepte de dé-légalisé directement la disposition litigieuse. \\
Dans sa décision du 15 Mars 2012 "Loi de simplification du droit", le CC en est revenu à sa jurisprudence de 1982, le CC ne censure pas l'empiétement et ne dé-légalise pas. 

\subsection{L'empiétement du pouvoir réglementaire sur le domaine législatif}

Cet empiétement est en principe prohibé et toujours sanctionné par le Juge administratif. Il est parfois autorisé par dérogation, dans le cadre des ordonnances de l'article 38. 

\section{Le pouvoir réglementaire}

Pendant longtemps, les actes réglementaires n'étaient pas considérés comme des actes administratifs, in-susceptible de recours. Mais le 13 Mai 1872, CE "Brac de la Periere", le CE refusait de contrôler les RAP (Réglementent d'Administration Publique), considérant que c'était semblable à la loi mais il a fait une revirement de jurisprudence le 6 Décembre 1907 "Compagnie des chemins de fer de l'Est", considérant que les RAP émanaient d'une autorité administrative. \\
Sous la Constitution de 1958, le CE a rappelé que tous les actes réglementaires sont susceptibles de recours y compris les réglements autonomes de l'article 37: CE, 26 Juin 1959, Syndicat général des ingénieurs conseils. 

\subsection{Les titulaires du pouvoir réglementaire}

Le Premier Ministre est le titulaire du pouvoir réglementaire au niveau national: article 21 de la Constitution. Les actes réglementaires du PM doivent être contresignés par les ministres compétents, ce qui ne fait pas d'eux les co-auteurs. \\
Le Chef de l'État participe au pouvoir réglementaire en contresignant les décrets signés en Conseil des Ministres (art. 13 Constitution), ce qui en fait un auteur. Pendant longtemps, le Chef de l'État disposait du pouvoir de police administrative au niveau national (CE, 1919, Labonne), celui a été transféré au PM depuis la Constitution de 1958. Les ministres, en principe, ne disposent pas du pouvoir réglementaire mais il y a une nuance et deux dérogations: les ministres participent à l'exercice du pouvoir réglementaire du PM en contresignant les actes ; les lois peuvent donner compétence réglementaire à un ministre, limité et précis dans son champ et qui s'exerce souvent en matière de police spéciale ; les ministres disposent d'un pouvoir réglementaire à l'intérieur de leur service en tant que chef de service (dégagé par le CE, 7 Février 1936, Jamart). \\
Les collectivités territoriales disposent d'un pouvoir réglementaire pour l'exercice de leurs compétences, c'est ce que prévoit depuis 2003 l'article 72 de la Constitution. En pratique, les collectivités disposaient déjà d'un pouvoir réglementaire, mais seulement donné par la loi. \\
Il existe d'autres titulaires du pouvoir réglementaire. Une loi peut autoriser une autre autorité administrative à exercer un tel pouvoir. Le CC l'a admis le 17 janvier 1989 "liberté de communication", l'habilitation ne doit concerner que des mesures à portée limité tant dans leurs champs d'application que dans leurs contenus. Autre limite posée par le CC: le pouvoir réglementaire du PM ne peut jamais être subordonné à celui d'une autre autorité administrative. 

\subsection{La typologie des actes réglementaires}

Il faut rappeler la distinction entre les règlements autonomes et ceux d'application. \\
Les premiers sont pris dans le champ de l'article 37 de la Constitution, et peuvent être édictés même en l'absence d'une loi. \\
Dans un règlement d'application ou d'exécution, la loi prévoit la prise de ce dit règlement. Le pouvoir réglementaire a toujours la possibilité, à posteriori, de modifier ou d'abroger un règlement: CE, 6 Décembre 1907 "Compagnie des chemins de fer de l'Est". Si le règlement n'est pas pris, la loi ne peut pas être appliquée, ce qui permet au gouvernement de paralyser l'action d'une loi.

\subsection{L'obligation d'exercer le pouvoir réglementaire}

Il existe trois hypothèses dans lesquelles le pouvoir réglementaire sera obligé d'exercer son pouvoir. \\
1. L'obligation d'exercer son pouvoir de police administrative et donc obligation d'adopter des règlements de police administrative: CE, 1962, Doublet. \\
2. L'obligation d'adopter des règlements d'application des lois dans un délai raisonnable, obligation posée par le CE, 28 Juin 2000, Association France Nature Environnement. Cette obligation a été posée par le juge afin de contrer l'action (ou plutôt l'inaction) de certains gouvernements. Le délai raisonnable est apprécié par le juge, et a déjà dit que 4 ans, ce n'était pas raisonnable ; un délai raisonnable, ce serai entre 6 mois et un an et demi. Si le règlement n'est pas pris, la puissance publique engage sa responsabilité pour fautes si il y a un dommage. \\
Si la loi est suffisamment précise et même si elle a prévu un décret d'application, la loi peut s'appliquer sans obligation d'un décret d'application, CE, 23 Novembre 2011, Monsieur A.  \\
3. L'obligation d'abroger les règlements illégaux. CE, Ass., 1989, Ali Talia, fait de cette obligation une PGD. Prévu désormais au CRPA. 

\section{Les ordonnances}

Procédure particulière: le Parlement va déléguer son pouvoir législatif au Gouvernement, pour qu'il prenne des mesures relevant de l'article 34 de la Constitution. 

\subsection{Procédure d'édiction des ordonnances}

C'est une procédure qui commence par le vote d'une loi d'habilitation, pour autoriser le Gouvernement à prendre des mesures qui relèvent du champ de l'article 34. La loi d'habilitation a des conditions et produit des effets.

\subsubsection{Les conditions de la loi d'habilitation}

La première condition tient à l'initiative de la loi d'habilitation, qui ne peut être demandée que par le Gouvernement. \\
La deuxième condition tient au contenu. Elle doit énumérer les matières dans lesquelles le gouvernement va intervenir mais aussi les objectifs que va poursuivre le gouvernement. \\
La troisième condition concerne le délai. Elle doit fixer deux séries de délais. Elle doit d'abord fixer un délai d'habilitation, pendant lequel le Gouvernement peut édicter une ordonnance. Outre ce délai, il faut préciser aussi le délai de dépôt d'un projet de loi de ratification. La ratification n'est jamais obligatoire et peut décider de ne jamais ratifier l'ordonnance. Si le Gouvernement ne dépose pas le projet de ratification dans les délais, les ordonnances deviennent caduques. 

\subsubsection{Les effets de la loi d'habilitation}

Effet positif: elle permet d'étendre les compétences du Gouvernement, de façon temporaire, pendant le délai d'habilitation. Note: les ordonnances sont en général adopté en conseil des ministres après avis du Conseil d'État. \\
Négativement, l'ordonnance limite la compétence du Parlement, qui est dé-saisi de sa compétence pendant le délai de l'habilitation. \\
L'habilitation n'est pas sans limites. Elles ont étés posées le 24 Février 1961, Fédération nationale des syndicats de police: l'habilitation ne peut pas être permanente ; une loi d'habilitation ne peut pas avoir pour effet de valider des actes administratifs ou d'empêcher les juridictions de contrôler les actes administratifs. 

\subsection{Le régime juridique des ordonnances}

\subsubsection{Le régime des ordonnances ratifiées}

Dès lors qu'elle est ratifiée, l'ordonnance acquiert rétroactivement une valeur législative, et n'est donc plus susceptible de recours devant le juge administratif. \\
Avant 2008, le juge admettait les ratifications explicites qui impliquait le vote des projets de loi de ratification. Il admettait aussi la ratification implicite, qui admettait que l'acte du Parlement impliquait nécessairement. TC, 2007, Préfet de l'Essonne. Depuis la révision de 2008, la ratification ne peut plus qu'être explicite. Il y a là la volonté du constituant d'assurer la sécurité juridique du citoyen car la ratification implicite est difficile à montrer. 

\subsubsection{Le régime des ordonnances non ratifiées}

L'ordonnance qui n'est pas ratifiée n'a qu'une valeur réglementaire et peut donc, être contestée devant le juge administratif. Cette solution vaut pour toutes les ordonnances, que ce soit l'article 37 ou celle de l'article 16: CE, Ass, 19 Octobre 1962, Canal. 

\subsection{La modification des ordonnances}

Deux hypothèses. \\
Si on se situe avant le délai d'expiration de l'habilitation, le gouvernement peut modifier l'ordonnance. \\
Si on se situe après, il faut distinguer deux situations, en fonction de la nature des dispositions en cause. La première est si les dispositions relèvent du domaine de la loi, le gouvernement ne pourra pas les modifier: CE, Ass, 11 Décembre 2006, Conseil national de l'ordre des médecins. À l'inverse, si la disposition relève du domaine de l'article 37, le Gouvernement pourra modifier l'ordonnance non ratifiée, même après le délai d'expiration: CE, 30 Juin 2003, Fédération régionale ovine du Sud-Est. 

\chapter{Les principes jurisprudentiels}

Ce sont des principes que le juge consacre, révèle, ou dégage. Il dégage des PGD. C'est une technique qui remonte à la fin de la seconde guerre mondiale en dégageant le principe des droits de la défense, CE, Ass, 26 Octobre 1945, Aramu. Le juge avait déjà dégagé des PGD un an avant, sans le dire explicitement: CE, 5 Mai 1944, Trompier-Gravier. \\
C'est une technique qui pourrait surprendre car le juge n'a pas compétence créatrice. L'article 5 du Code Civil prohibe les arrêts de règlement. Ceci étant, il est interdit aux juges de ne pas rendre justice, il a l'obligation de statuer (art. 4 du C.Civ), et pour le juge administratif, il a existé pendant très longtemps très peu de sources écrites du droit, et donc il a eu l'obligation de dégager ces principes avant de trancher les litiges qui lui étaient soumis. \\
Deux questions se posent sur les PGD: comment sont-ils élaborés ? Cette technique a-t-elle encore un avenir ? 

\section{L'élaboration des principes jurisprudentiels}

\subsection{Les auteurs des principes}

Outre le juge administratif, d'autres juridictions dégagent ce genre de PGD. 


Quantitativement, c'est le CE qui a dégagé le plus de principes et qui continue de le faire. Certaines juridictions du fond se sont mêmes permises de parfois le faire. Précision: le juge n'invente pas un principe, il le révèle, pour lui, il est déjà dans la conscience sociale et existerai déjà ; c'est une manière pour justifier le fait que le juge crée du droit.  \\
CE. Ass. 1973, Dame Penet, dans lequel le CE dégage le principe qu'il est interdit de licencier une femme enceinte de la fonction publique, qui existait déjà en droit du travail privé. Il utilise une formule spécifique comme quoi le principe est "un principe dont s'inspire le code du travail", comme si la Loi s'inspirait d'un principe déjà existant alors que le CE s'inspire de la loi et non pas l'inverse.


Les juges constitutionnels et internationaux, mais aussi les juges ordinaires comme le juge judiciaire. La C.Cas a dégagé un principe: Civ 1ère, 21 Décembre 1987, BRGM, consacrant le principe de l'insaisabilité des bien des personnes publiques. \\
Le CC dégage beaucoup de principes, soit des principes à valeur constitutionnelle, soit des objectifs à valeur à constitutionnel ou encore des PFRLR. À noter que les PGDA ne sont pas à valeur constitutionnel. \\
Il y a pu y avoir des différences de point de vue entre le CC et je le CE mais cela fait 20 ans que l'on ne voit plus de contradiction chez ces deux juges. \\
La CJUE dégage les PGDUE suivant des inspirations assez variables: les traités, la nature des traités, des principes existant dans des États membres, de leur tradition constitutionnelle, et la CEDH. Les PGDUE ont la même valeur que le droit originaire: CE, 2001, SNIP. Les PGDUE ne sont pas les mêmes que les PGDI qui n'ont pas de valeur supra-législative: CE, 2000, Paulin.

\subsection{La valeur des principes}

La valeur n'a jamais été explicitée par le juge, c'est donc la doctrine qui a tenté d'en trouver la valeur. Il existe aujourd'hui trois positions doctrinales. \\
La position majoritaire est celle de René Chapus où les principes ont une valeur infra-législative et supra-décretale. Cela s'explique par la place du juge dans la hiérarchie des normes, il doit se subordonner à la Loi mais peut censurer les actes administratifs. \\
Mecheriakoff propose une deuxième position, pour les PGD ont une valeur législative en faisait application du principe selon lequel le droit spécial supplante le droit général. Pour lui, les PGD sont la loi général qui vient céder devant toute loi spéciale. Elle semble plus cohérente car le juge interprète la Loi et n'y est donc pas totalement soumis. \\
Une troisième position est celle de Pierre Brunet qui considère que le juge accorde aux PGD la valeur qu'il souhaite et dont il a besoin, en fonction des circonstances du litige. Le CE, en 1996, Ass, Koné, le juge dégage un PFRLR. \\
Le CE distingue les  PGD et les règles générales (comme celles appliqués aux contrats administratifs). Ces règles générales n'ont pas la même valeur que les principes, ils ont au maximum, une valeur réglementaire. 

\section{Le domaine des principes jurisprudentiels}

Il existait un domaine traditionnel de ces principes mais s'est élargi ces dernières années. 

\subsection{Le domaine traditionnel des principes}

Ce sont ceux qui ont été consacrés dans l'immédiat de la fin de la seconde guerre mondiale: ils posent des principes essentiels, dit fondamentaux. Comme, par exemple, le principe de non rétro-activité des actes réglementaires (CE, Ass, 25 Juin 1948, Société du Journal l'Aurore), le principe des droits de la défense, le principe général de l'existence du REP (CE, Ass, 19 Février 1950, Lamotte). \\
Le CE va intégrer le préambule de la constitution dans le bloc de légalité. Il le fait de manière indirecte en y intégrant dans le droit positif via des PGD. Dans les années 50, le CE a posé comme PGD toutes les  libertés garantis par la DDHC: liberté du commerce et de l'industrie (CE, 22 Juin 1951, Daudignac), liberté d'aller et venir (CE, 20 Mai 1955, Société Lucien), l'égalité devant le service public (CE, 9 Mars 1951, Société des concert des conservatoires), l'égal accès aux emplois publics (CE, Ass, 28 Mai 1954, Barel), l'égalité devant les charges publiques (CE, 7 Février 1958, Syndicat des propriétaires des chêne-lièges d'Algérie).  

\subsection{L'extension du domaine des PGD}

On aurait pu penser à un certain déclin des PGD. La doctrine a pu penser que l'essentiel des principes avaient été consacrés, il a été aussi invoqué que les sources écrites du droit administratif se multipliait, ce qui rendait ces principes moins utiles, mais encore, il y aurait la concurrence d'autres juridictions qui dégagent d'autres principes et certains auteurs ont craint la concurrence du juge constitutionnel ou encore celui de l'Union européenne. \\
Le déclin ne s'est pas produit d'abord parce que le juge se réfère toujours aux principes qu'il a déjà pu constater. Plus encore, le CE a pu dégager de nouveaux principes dans de nouveaux domaines avec des principes qui se spécialisent. Le CE dégage aussi des principes en les reprenant à d'autres juridictions (ce qu'on pourrait appeler "nationalisation" des PGD). 


Le juge dégage des principes dans des domaines plus spécialisés, plus précis, moins généraux. On peut relever comme domaines: le droit des étrangers (principes applicables aux demandeurs d'asile: CE, 1er Avril 1988, Bereciartua-Echarri ; droit de mener une vie familiale normale: CE, Ass, 8 Décembre 1978, GISTI), dans le domaine médical (droit du patient à consentir à son traitement ; droit du patient à ne pas subir un traitement qui serait le résultat d'une obstination déraisonnable: Ass Référé, 14 Février 2014, Vincent Lambert.  


Le juge refuse parfois de consacré certains principes. Le premier est celui qui interdit de donner des souffrances aux animaux: CE, 30 Décembre 1998, Ligue Française du droit à l'animal. Pas de principes non plus pour l'anonymat des copies dans l'enseignement supérieur: CE, 1er Avril 1998, Jolivet. 


L'idée de nationalisation des principes est que le juge s'approprie un principe dégagé par une autre juridiction, voir d'un autre ordre juridique. Le juge fait cela pour ne pas être lié par le principe dégagé par l'autre juridiction et pour pouvoir donner sa propre portée. \\
Deux exemples: CE, Ass, 1999, Didier sur le principe d'impartialité: "qui résulte notamment de l'article 6, §1 CEDH", le juge consacre lui même son principe et lui accorde son principe qui peut varier de la portée donné par le juge européen. \\
Principe de sécurité juridique: CE, Ass. 2006, KPMG. Le principe est le même que la confiance légitime trouvé en principe du droit de l'union européenne, qui n'a jamais été repris par le CE (il avait refusé en 2001, Freyluth), qui préfère consacré son propre principe de sécurité juridique qui a quasiment la même portée. 


\part{Les missions de l'administration}

\chapter{Le service public}

La question du service public est le sujet le plus politiquement sensible car c'est l'expression d'une vision sociale, politique, qui incarne des valeurs (la solidarité, l'efficacité, etc.), c'est une conception des rapports entre les individus. Le service public est, en France, une référence commune, à défendre car apparaît comme menacé, par le droit de l'UE notamment. Le service public est donc une notion importante en dehors du droit. \\
Pourtant, le service public a été une question centrale du droit administratif pendant longtemps. C'est la notion qui a permis de construire le DA. Le service public est le coeur de la réflexion de l'arrêt Blanco. \\
Il est compliqué de définir le service public. On peut essayer de le définir avec trois éléments. C'est avant tout une société de prestations de services (matériels, financière, gratuite), ces prestations sont parfois facultatif (école de musique, piscine municipale etc.), parfois obligatoire (police, pompiers). Deuxième élément, qui tient à la finalité de la mission du service public: l'intérêt général mais toutes missions d'intérêt général n'est pas nécessairement un service public. Troisième élément, il y a une forte intervention de la puissance publique. \\
On envisageait, il y a un siècle, le service public comme cela, avant qu'il entre en crise, notamment de la notion, dès le début du XXe siècle, où l'on s'est rendu que la définition était flou. L'intérêt général est une notion floue donc on a du mal à savoir ce qu'est le service public. On peut se demander aussi qui prend en charge le service public. Dans les années 30, le juge a admis qu'une personne privée pouvait gérer un service public, ce qui a participé à ébranler cette notion. \\
Aujourd'hui, la notion de service public ne détermine plus l'applicabilité du DA. Il existe des SPIC, qui sont majoritairement soumis au droit privés mais qui sont des services publics.


Dans les années 50, le juge s'est remis à utilisé la notion qui permet de savoir ce qu'est un acte administratif unilatéral, permet de définir les contrats administratifs, le domaine public, les personnes publiques, les travaux publics. \\
Cela n'est pas anodin que cette notion revienne à ce moment car c'est une période de fort interventionnisme étatique. \\
Depuis 30 ans, la notion subit de nouveaux des "attaques", elle est bouleversée par le droit de l'UE, celui de la concurrence notamment qui conçoit le service public comme une activité d'entreprise qui doit être soumis à la concurrence. Cela conduit donc à des craintes du démantèlement du service public à la française. Le droit de l'UE ne connaît pas la notion de service public. \\
Si il y a un terme pour qualifier le service public, c'est variété car il n'existe pas un service public mais des services publics. Non seulement la notion est variable, mais le cadre juridique l'est aussi.

\section{Une notion variable}

\subsection{Les critères du service public}

Si un loi a décidé que quelque chose était de service public, il n'y a pas à tergiverser, c'est de service public. Le L6112-1 du CSP définit le service public hospitalier et ses missions. \\
Des règlements peuvent créer des services publics mais le juge peut très bien requalifier si c'est bien un service public ou non, selon certains critères. On peut noter trois critères jurisprudentiels qui ne sont pas tout à fait clair aujourd'hui. 

\subsubsection{Le critère matériel: une mission d'intérêt général}

Le service public est une activité d'intérêt général, c'est lui même une notion flou, fluctuante, et même subjectif. Les premiers auteurs à en discuter il y a un siècle souhaitait en faire une notion objective. Bac d'eloka, le CDG Batter disait que le service public relevait de la nature et de l'essence même de l'État. \\
Pour un autre auteur, influencé par le positivisme sociologique, le service public est une activité assuré par la puissance publique. \\
Ces conceptions objectives ont profondément échouées. Le service public est pris en charge par la puissance publique, c'est à elle de définir ce qui est d'intérêt général, ce qu'il faut définir comme tel. C'est donc un choix politique, et donc subjectif. Rien n'interdit à la puissance publique de se défaire de toutes ses activités (mis à part peut être les compétences régaliennes), ou de prendre en charge toutes les activités.


Le service public une notion aussi contingente de l'intérêt général qui varie selon les époques. En 1916, le CE considère par exemple que la construction d'un théâtre ne relève pas de l'intérêt général, vu qu'on est en pleine guerre. Le CE considérera après la guerre que la culture relève de l'intérêt général (comme une festival de BD, CE, 25 Mars 1988, Commune d'Hyeres). \\
Concernant l'organisation d'un jeu de hasard, ça a été pendant longtemps un monopole de la puissance publique. Cela visait plusieurs buts d'intérêt général: lutter contre le blanchiment d'argent, contre la fraude, il y a 80 ans, les fonds des loteries finançaient un fond de solidarité agricole. En 1999, le CE n'a pas eu cette vision des choses à propos de la FDJ, il a considéré qu'elle n'exerçait pas de missions d'intérêt général CE, 27 Octobre 1999, Mr Rolain. L'intérêt général n'est pas le jeu mais le contrôle de ces jeux: c'est pour ça qu'a été créé l'ARJEL en 2010 lors de l'ouverture à la concurrence. \\
Le CE dira pareil à propos des courses de chevaux, CE, 25 Septembre 1996, Bellenger. \\
Il y a un cas particulier à propos des casinos, qui peuvent exercer leur activité après autorisation du ministère. Ils peuvent être implanter que dans des stations thermales. Le CE dira que les jeux de casinos ne sont pas des services publics: CE, 19 Mars 2012, SA Groupe Partouche. Mais les casinos participent à un service public, à celui du développement économique, touristique et culturel des communes, ce qui a pour conséquence que le contrat passé entre la commune et le casino est une délégation de service public. \\
Les activités touristiques relèvent elles aussi de l'intérêt général, comme c'est le cas de l'organisation des plages. Le JA a considéré qu'un contrat qui liait la commune et une entreprise louant des transats sur la plage par exemple est un contrat de délégation de service public. 

\subsubsection{Le critère organique: le contrôle de la personne publique}

Il se peut que ce soit une personne publique qui gère des missions de service publics. C'était exclusivement le cas il y a un siècle. \\
Depuis à peu près 70, le JA a admis que la personne privée peut gérer un service public. Soit par délégation de service public via un contrat, ou alors au travers d'une loi ou d'un règlement. 


Le juge l'a admis depuis bien longtemps qu'une personne privée puisse gérée un service public via un contrat. CE, 6 Février 1903, Terrier, le JA va qualifier implicitement la mission d'une personne privée de service public. CE, 4 Mars 1910, Thérond, le JA va conclure la compétence de la juridiction administrative car le contrat qui lie la ville de Montpellier à Mr Thérond est une mission de service public. 


Une personne privée peut être chargée d'une mission de service public par une décision unilatérale. CE, Ass, 13 Mai 1938, Caisse primaire et des protections, le CE a considéré que les caisses primaires sont un service public que la Loi a confié à des personnes privées. CE, 31 Juillet 1942 "Montbeurt", décision concernant un organisme corporatif de droit privé qui se voit confier des missions par la Loi, le juge va considéré que celui-ci participe à des missions de service public, donc ses actes sont de natures administratives. \\
TC, 15 Janvier 1968 "Air France contre époux Barbier", concernant un règlement intérieur d'Air France, qui interdisait aux hotesses de l'air de se marier. Le TC dira que Air France est une société privé mais gère un service public et que son règlement met en oeuvre des prérogatives de puissance publique, le règlement est donc de nature administrative. 

\subsubsection{Le critère formel: la détention de prérogatives de puissance publique}

Ce critère a souvent été en question. Les réponses ont évolués et cette question ne concerne que les personnes privés, les personnes publiques détenant forcément de telles prérogatives. \\
Dans un premier temps, le juge a exigé la détention de prérogatives pour qualifier de service public: CE, 28 Juin 1963, Narcy. Le CE considérait qu'une personne privé était bien un service public car remplissait une mission d'intérêt général, elle a été attribué par les pouvoirs publics, enfin, le juge relève que la personne privée détient des prérogatives de puissance publique. \\
Dans un deuxième temps, le juge a été plus hésitant, notamment au début des années 80. Le juge admettait qu'une personne privée puisse gérer un service public sans disposer de prérogative, TC, 1978, Bernardi, concernant un hôpital privé, qui contribue au service public hospitalier alors qu'il ne détient pas de prérogative. TC, 25 Janvier 1982, Consuel, deuxième exemple. \\
Dans un troisième temps, le juge va clarifier sa position, CE, 20 Juillet 1990, Ville de Melun, la personne privée peut gérer un service public sans prérogative à condition qu'il existe un contrôle étroit de la personne publique sur la personne privée. Était en cause en l'espèce une association qui exerçait une activité d'intérêt général, elle ne détient pas de prérogatives de puissance publiques mais le juge va relever le contrôle étroit de la collectivité sur l'association: elle a été créée à l'initiative de la commune, administrée principalement par la commune, subventionnée par la commune. En pratique, les communes créent souvent des associations pour gérer des missions, probablement parce que la mise en oeuvre du droit privé est plus simple que la mise en oeuvre du droit public. Le juge considère que ce sont des personnes privés transparentes, agissant comme un service de la commune et a donc des conséquences sur la qualification des actes fait par l'association. \\
Dans un quatrième temps, le juge va reprendre les solutions précédentes dans son arrêt du 22 Février 2007, APREI. Une personne privée peut gérer un service public dans trois situation, la première quand la loi le prévoit, la deuxième si la personne privée détient des prérogatives de service public, la troisième quand il n'y a pas de PPP mais un contrôle étroit de la personne publique. 


Il y a des hypothèses où la personne privée gère des missions d'intérêt général sans que la mission soit qualifié de service public. Tout ce qui est d'intérêt général n'est pas service public. CE, 5 Octobre 2007, UGC, une société privée exploitait un cinéma: le juge dira que la personne privée exerce bien une mission d'intérêt général mais cette société ne détient pas de prérogatives de puissance publique et il n'y pas de contrôle étroit, la société ne gère donc pas une mission de service public. \\
La qualification de l'activité dépend souvent du mode de gestion de l'activité. 


Le critère organique n'est pas déterminant, ni le formel, le seul critère toujours exigé, c'est celui de l'intérêt général. 


\subsection{Les catégories de service public}

\subsubsection{Les services publics industriels et commerciaux (SPIC)}

C'est une notion assez récente. TC, 22 Janvier 1921, Société commerciale de l'ouest africain, plus connu comme l'arrêt du "bac d'éloka", reliant deux rives d'un fleuve en côté d'Ivoire. Le TC va conclure la compétence du juge judiciaire en l'espèce, car "la colonie exploite le bac dans la même condition qu'un industriel ordinaire". 


Le juge appréhende la notion de SPIC différemment aujourd'hui qu'il y a 70 ans. Matter considérait, dans ses conclusions, que la nature même des activités allaient naturellement aux industriels et d'autres à la puissance publique. Le juge va plutôt se fonder sur la gestion contrairement à Matter. \\
Le CE va clarifier les choses en 1956, 16 Novembre, en assemblée, Union syndicale des industries aéronautiques. Le juge va dégager trois indices pour dégager la nature de l'activité. Il faut d'abord vérifier si il existe une qualification législative de l'activité le juge peut requalifier une qualification réglementaire en utilisant le faisceau d'indice. Le premier indice est la nature de l'activité, le SPIC est donc une activité pouvant être entreprise par n'importe quel entrepreneur. Deuxième indice: le mode d'organisation (similaire à d'autres entreprises privés, régime applicable aux agents). Troisième indice: le mode de financement de l'activité (par l'impôt ou par les usagers). \\
Sur ces trois indices, l'un est plus déterminant que les autres. Le premier, la nature, n'est jamais déterminant pour le juge car difficile à mettre en oeuvre, il n'est pas clair. Le deuxième indice a très peu d'effectivité car est fortement tautologique donc on tourne en rond. Le troisième indice est donc le plus déterminant, CE, Avis, 10 Avril 1992 "SARL Hoffmiller", concernant l'enlèvement des ordures. Le juge considérera que c'est un SPIC en se fondant sur le mode de financement qui est par redevance, ce qui en fait un SPIC. \\
Parfois le juge peut se référer aux autres indices. Dans de rares cas, la nature du service peut servir. Le juge considère que l'exploitation des autorités est un service administratif même si elles sont gérées par une personne privée et financés par redevance. Cette décision s'explique car touche aux fonctions régaliennes. 

\subsubsection{Le régime du SPIC}

C'est essentiellement un service qui utilise le droit privé. Le droit du travail est applicable aux agents de SPIC, sous deux limites. Le directeur du service (CE, 1923, De Robert Lafregeyre), ainsi que le comptable public si il existe (CE, 1957, Jalenques de Labeau), sont soumis au droit public. \\
Les relations entre le SPIC et les usagers sont des rapports de droit privé (droit civil et commercial). Ce sont des rapports de droit privé même si les contrats liant l'usager au service contient une clause exorbitante au droit commun (CE, 13 Octobre 1961, Campanon-rey).


Deux situations dans lesquelles le droit public s'appliquent: dans le cas de certains biens, qui appartiennent au domaine public, régime très protecteur. Il en va de même de certains règlements, si édictés par des personnes privés chargés d'un SPIC, qui sont pris pour l'organisation de ce service, qui mettent en oeuvre des prérogatives de puissance publique. 

\section{Un régime juridique varié}

Le régime peut varier selon plusieurs critères, les règles de création sont très variables, elles ne sont pas les mêmes en fonction de la collectivité. Les modes de gestion peuvent être variables. Certains services publics sont soumis au droit de la concurrence.

\subsection{Le régime de la création des services publics}

\subsubsection{Les services publics obligatoires}

Les services publics sont obligatoire dès lors que leurs créations sont imposés par un texte, qui peut être constitutionnel ou législatif. \\
Il existe des services publics imposés par la Constitution elle même. CC: "Service public dont la nécessité découle de principes ou de règles à valeurs constitutionnelles". Cependant, le CC se contente de dénier les services publics sans les consacrer. En doctrine, il est considéré que ces services publics concernent les activités régaliennes: défense, justice, police, fiscalité, enseignement public. \\
Parfois une simple Loi impose la création d'un service public. Cela concerne principalement les services locaux: les collectivités territoriales sont obligés de remplir des missions de service public donnés par la Loi: service des pompes funèbres, ramassage des ordures, assainissement, voies communales, enseignement primaire et secondaire. 

\subsubsection{Les services publics facultatifs - le cas des services économiques}

Une personne public peut-elle créer un service public industriel et commerciale librement comme le ferait un entrepreneur privé ? \\
La réponse a été pendant longtemps négative. Une réponse qui a trouvé sa source dans la liberté du commerce et de l'industrie: non concurrence qui interdit aux personnes publiques de venir concurrencer les personnes privées. Cette solution se fonde en réalité sur le fait que la puissance publique dispose de moyens exorbitants, de financement en théorie illimité, etc. \\
Ce principe a beaucoup été assoupli. Aujourd'hui, par dérogation, une personne publique peut créer une activité économique. 


Premier temps, CE, 24 Mars 1901, Casanova, le CE considère que la PP ne peut pas créer d'activités économiques sauf en cas de circonstances exceptionnelles, sans les préciser. \\
Deuxième temps, CE, 30 Mai 1930, Syndicat du commerce en détails de la ville de Nevers. Le CE rappelle le principe précédent, mais va l'assouplir en admettant que la PP puisse créer une activité commerciale si deux conditions sont remplies: si l'activité est d'intérêt général (il faut qu'il existe des circonstances particulières de temps et de lieu) et il faut qu'il n'y ait aucune initiative privée (carence de celle-ci). \\
Troisième temps, c'est l'application souple du précédent, CE, 20 Novembre 1964, Ville de Nanterre. Il accepte très facilement qu'une PP crée une activité économique. Trois assouplissements: le juge prend en compte la carence qualitative de l'initiative privée, le juge admet parfois la création en l'absence de telles carences dans l'hypothèse d'une auto-prestation (CE, 29 Avril 1970, Unipain), dans l'hypothèse où un service rentre dans les compétences traditionnelles du service public (Arrêt Blanc, 1970, CE). \\
Le juge admet facilement l'extension de services déjà existant si deux conditions sont existantes: deux  services publics liés; nouvelle activité permettant d'améliorer le service pour l'usager (CE, 18 Décembre 1959). \\
Quatrième temps, le juge a reformulé le principe. CE, Ass, 31 Mai 2006, Ordre des avocats au barreau de Paris. Le juge rappelle le principe initialement posé, Ceci étant le juge opère deux évolutions: assouplir la création et donc imposer un nouvel encadrement. L'intérêt public, selon le juge, peut résulter de la carence d'une initiative privée.  Deuxième évolution: dès lors qu'il existait un intérêt public et qu'il a été créé, l juge va considéré que le service créé doit respecter le droit de la concurrence. Le juge va s'attaché à vérifier l'égalité de concurrence entre l'acteur public et privé  et étudie donc le régime fiscal, l'origine sociale, bref que l'acteur public n'ai pas d'avantages. 

\subsection{Les modes de gestion du service public}

CE, 6 Avril 2007, Aix En Provence. Le juge va rappeller les différentes modalités de gestion du service public. 

\subsubsection{Le choix discrétionnaire du mode de gestion}

Question réglée, CE, 18 Mars 1988, Loupias. Confirmé par la décision commune d'Aix. La personne publique choisit librement le mode de gestion du service public. Soit le gérer elle même, soit la collectivité confie la gestion du service à un tiers, et donc externaliser le service. \\
Les personnes publiques n'ont pas l'obligation d'externaliser la gestion des services publics, ce que certains ont crains dans le droit de l'Union.


CE, 2007, Aix en Provence, le CE explique que la collectivité peut gérer le service public elle même de trois façons. \\
La première façon est la régie. Il y a gestion en régie quand la collectivité qui crée le service, la gère, avec son propre personnel, ses propres moyens. Il y a deux types de régies: la régie simple et la régie directe. On considère qu'il y a régie simple quand les dépenses et les recettes de la régie sont dans le budget de la collectivité. Dans cette hypothèse, ce n'est pas possible de faire ce genre de régie dans des SPIC. La régie directe est dotée de l'autonomie financière, peut disposer d'une comptabilité propre. \\
La deuxième façon est ce que le CE appelle la régie autonome, que l'on appelle aussi la régie spécialisé. C'est dans l'hypothèse dans la création d'un établissement public, une autre personne public, pour personnifier le service public. Il y a en droit français, deux types d'établissement public, les établissements administratif et les établissements industriels et commerciaux (EPA et EPIC). Cette différenciation n'est pas claire pour deux raisons: car il existe des établissements publics à double visage (qui sont qualifié de l'un ou de l'autre mais qui exerce une pluralité de missions) ; il existe des établissements à visage inversé, qualifié d'EPIC mais qui sont en réalité des EPA: ils sont EPIC par commodité dans le but de les soumettre au droit privé (cela ne peut plus avoir cours aujourd'hui car le CC interdit cela). \\
La troisième façon est celui de l'organisme dédié. C'est une personne morale distincte de la collectivité mais qui répond à deux critères bien précis: cet organisme exerce son activité pour le compte de la collectivité ; il faut que la collectivité exerce sur cet organisme un contrôle comparable à celui qu'elle exercerait sur ses propres services. On parle de gestion du service "in house". CJCE, 1999, Teckal. \\
Le droit de l'UE tend à encadré ces modes de gestion via le droit des aides de l'Union, qui prohibent les aides qui rompent la libre concurrence. Pour l'UE, financer des établissements publics, qui gèrent des services publics, est suspect. Les établissements publics sont avantagés par rapport aux autres (garanti illimité de l'État). Pour la CJUE, 3 Avril 2014, Commission contre France, à propos de l'ancien statut de La Poste: cette décision a amené les EPIC à se transformé en Société Anonyme (La Poste, EDF, GDF-Suez, etc.). 


La gestion du service public par un tiers a été admise dès le XXe siècle par le CE. C'est de plus en plus courant en pratique aujourd'hui. \\
Dans la décision commune d'Aix en Provence, cela est encadré par une première contrainte: toutes les activités ne sont pas délégables: les missions de polices administratives notamment, comme le pouvoir réglementaire des collectivités. Deuxième contrainte: il faut une habilitation d'une personne publique qui peut prendre la forme d'un contrat ou d'un acte unilatéral. Le juge dégage dans la Commune d'Aix une hypothèse dans laquelle une personne peut commencer à gérer une mission d'intérêt général de sa propre initiative, et dans un deuxième, elle reconnaît la mission de service public. Le TC a confirmé cette hypothèse le 13 Octobre 2014, Axa France IARD. 

\subsubsection{Le choix encadré du gestionnaire du service}

Ce choix ne pose pas de problème quand la personne publique gère elle même le service. Le choix pose question quand le service est externalisé. Dans cette hypothèse, un ensemble de règles va encadrer le choix du gestionnaire, c'est le droit de la commande publique (ou encore droit de la mise en concurrence), qui impose à la personne publique de mettre en concurrence les opérateurs avant de le choisir. Cela vise deux objectifs: préserver les deniers publics et limiter la dépense. Mais aussi pour permettre à tous les opérateurs économiques d'avoir accès à la commande publique. 


Quand la personne publique confère la gestion du service à un tiers, elle doit le faire via un contrat. Cela ne peut se faire qu'après une procédure de mise en concurrence dans laquelle l'opérateur sera choisi selon des critères pré-définis. \\
Il existe deux catégories de contrats, d'une part les marchés publics, d'autre part les concessions de service public (depuis 2016, avant délégation de service public). Ces contrats ont des procédures de passations différentes, prévues par des textes distincts. Le marché public était encadré par le code des marchés publics, abrogé en Janvier 2016 par l'ordonnance du 23 Juillet 2015. Les critères prédéfinis permettent de choisir l'opérateur qui remplit le mieux les critères. \\
Jusqu'en 2016, c'est la loi Sapin qui encadrait les délégations de service public avec l'idée que la personne publique disposait du choix du gestionnaire et il pouvait y avoir négociation, ce qui n'est pas possible dans les marchés publics. Une ordonnance du 27 Janvier 2016 remplace la loi Sapin et encadre plus fortement la passation des concessions. La personne publique aura moins le choix, elle devra publier des critères de sélections, limitant très fortement les négociations. 


Il existe des dérogations à la mise en concurrence. Trois hypothèses peuvent être dégagées. \\
La première hypothèse est quand un texte écarte une telle mise en concurrence. CE, Commune d'Aix, 2007. Cette possibilité n'est probablement pas en accord avec le droit de l'Union. Cette hypothèse n'a jamais été appliquée. \\
La deuxième dérogation est l'hypothèse de l'organisme dédié, un tiers par rapport à la personne publique, très lié. La personne publique n'a pas à faire de procédure de mise en concurrence, ce qu'on appelle en droit de l'Union une exception "in house". \\
La troisième dérogation est l'hypothèse de la coopération entre personnes publiques. C'est quand une personne publique va en aider une autre avec son personnel ou ses moyens financiers. Cette hypothèse a été admise par la CJCE et reprise par le CE. 

\subsection{Les "lois" du service public}

Ces "lois" sont en fait des PGD, parfois consacrés par des textes législatifs, parfois non, parfois consacrés au niveau constitutionnel. On parle de principes communs à tous les services publics. 

\subsubsection{Mutabilité}

Principe d'adaptation constante des services publics aux nécessités de l'intérêt général. Le service public doit donc s'adapter aux besoin des populations, s'adapter au contexte politique, etc. \\
Ce principe a deux conséquences juridiques, la première conséquence tient à ce que les administrés n'ont pas le droit à la création et au maintien des services publiques, CE, 27 Janvier 1961, Vannier. Cela passe par un autre principe: l'administré n'a pas de droit acquis au maintien d'un règlement (la personne publique peut modifier ou abroger son règlement comme elle le souhaite). Ce principe s'exprime tant qu'il n'y pas de lois contraires à ce principe. \\
La personne publique peut adapter à tout moment l'organisation de son service et peut donc modifier unilatéralement le contrat qui la lie au gestionnaire du service. Cela est admis depuis le 10 Janvier 1902, CE, Gaz de ville et Rouen. La personne publique peut aussi à tout moment décidé de supprimer un service public.  

\subsubsection{Égalité}

C'est un principe général du DA. Le CE le consacre le 9 Mars 1951, Société des concerts du conservatoire. \\
C'est un corollaire du principe constitutionnel d'égalité devant la loi (Art. 6 de la DDHC). \\
CE, 10 Mai 1974, Denoyez et Chorques, le principe impose de traiter de manière identique deux individus placés dans la même situation. L'inverse n'est pas vrai, en cas de situation différentes, l'administration n'a pas l'obligation de traiter différemment les situations: CE, Ass, 28 Mars 1997, Société Baxter, confirmé par le CC, QPC 17 Juin 2011, Fédération nationale des association tutélaires. 


Il peut y avoir des dérogations à ce principe. CE, Denoyez et Chorques, le CE donne trois situations où c'est possible. \\
La première possibilité est quand la Loi le prévoit. La deuxième est quand il y a des différences de situation. La troisième est si l'intérêt général le commande, qui doit être en rapport avec les conditions d'exploitation du service. \\
Le CE a précisé les deux derniers cas, CE, 18 Décembre 2002, Mr Villemain. Dans les deux cas, le CE pose deux conditions nouvelles. La première est qu'il faut que la différence de traitement soit en rapport direct avec l'objet du service. La deuxième est qu'il faut que la différence de traitement ne soit pas excessive, elle ne doit pas être manifestement disproportionnée. \\
Une question se pose: à partir de quand il y a changement de situation ? L'appréciation de celle-ci est très variable et très subjective. Parfois, les autres juridictions ont une idée de la différenciation qui n'est pas la même qu'une juridiction donnée. Plusieurs questions se sont posés sur la différenciation à propos des SPA facultatifs, notamment sociaux, qui associent parfois des services culturels, s'est posée la question des tarifs, et à partir de quand il y avait différence de situation pour justifier des écarts de tarifs. \\
Le CE a considéré que la situation géographique constitue une justification de traitement tarifaire particulier, CE, 5 octobre 1984, Lavalonais. Le CE a assoupli sa jurisprudence le 13 Mai 1994, Commune de Dreux. Le CE revalide donc la différence de traitement, mais justifie encore un autre traitement des usagers qui auraient des "liens suffisants avec la commune" (travailler dans une commune par exemple). \\
La question s'est posée pour savoir si la différences de ressources de l'usager justifie une demande de traitement. C'est la question du "quotient familial" qui fait varier le coup du service en fonction des ressources. Le CE a commencé à rejeter l'idée de ce quotient le 2 Avril 1985, Ville de Tarbes. Le CE a évolué et admis le QF pour les services sociaux uniquement: CE, 18 Mars 1994, Mme Dejonckeer, qui admet la différence de traitement dans les services d'une cantine. Ici, c'est l'intérêt général qui justifie la différence de traitement, notamment que le service soit utilisé par le plus grand nombre d'usagers. Le tarif le plus élevé doit rester inférieur au coût du service. \\
CE, 29 Décembre 1997, Commune de Genevilliers et Commune de Nanterre (deux espèces différentes). Dans une loi de 1998, relative à la lutte contre les exclusions, le législateur a codifié la JP administrative. \\
CE, 10 octobre 2014, Région Nord-Pas-De-Calais, qui contestait la tarification de la liaison Paris-Lille. Elle considérait qu'elle avait un traitement différent qui n'était pas justifié, ni par l'intérêt général ni par une différence de situation. Le CE lui, va rejeter le recours qui considère qu'il existe une différence de situation qui justifie cette tarification. 

\subsubsection{La neutralité}

C'est un corollaire du principe d'égalité. Principe qui interdit de traiter différemment les agents ou usagers du service en fonctions de leurs convictions politiques ou religieuses. CE, Ass, 28 Mai 1954, Barel: était en causse l'accès à l'ENA d'un élève qui avait été refusé en raison de ses convictions politiques (il était communiste), le CE a jugé que le refus d'accès de cet élève était contraire au principe de neutralité. \\
C'est parce que l'État et ses agents sont neutres que la liberté de conscience est possible. CE, 2 Octobre 2014, Confédération nationale des associations familiales catholiques, concernant des campagnes de lutte contre l'homophobie dans les collèges et lycées, et le fait que le ministre invite les recteurs à relayer une campagne incitant en sous lettre à user de drogues, etc, était contraire au principe de neutralité.


Le principe de laïcité est un élément de ce principe de neutralité. C'est un principe à valeur constitutionnel, inscrit dans l'article 2 de la Constitution, qui trouve son origine dans un texte législatif: la loi de 1905. Cela concerne la neutralité de l'État et uniquement de lui. Il doit être neutre, pour garantir la liberté de conscience des administrés. \\
Le CC a étendu ce principe le 21 Février 2013, QPC "Association pour la promotion et l'extension de la Laïcité". Le CC commence par nous expliquer que c'est un droit ou une liberté garanti au sens de la QPC. Ce n'est pas un droit subjectif mais le principe de laïcité garantit la liberté de conscience des administrés. Deuxième élément, le CC va redéfinir la portée du principe de laïcité, il va constitutionnaliser la plupart des corollaires de ce principe, pour la plupart issu de la Loi de 1905. Il y a cinq conséquences: le respect de toutes les croyances et l'égalité des citoyens devant la loi sans considération de la religion ; garantit le libre exercice du culte ; la République ne reconnaît aucun culte (aucune religion d'État, principe d'ignorance des cultes) ; cela implique enfin que l'État ne subventionne ou salarie aucun culte.


En droit positif se posent aujourd'hui quatre questions: \\
Dans quelles conditions on peut porter un signe appartenant à une religion ? Jusqu'à quel point l'État doit garantir la liberté d'exercer le culte ? Dans quelle mesure la laïcité implique la neutralité des bâtiments publics ? Question du financement du culte par l'État, en principe prohibé, mais autorisé de plus en plus souvent par dérogation. 


Un usager ou un agent du service public, peut-il porter un signe religieux ? Jusqu'à il y a 15 ans, la réponse variait de si la personne concernée était un agent ou un usager. \\
L'agent du service public ne peut pas le faire car, représentant l'État, il avait l'obligation d'être neutre. \\
Pour les usagers, cela a été pendant longtemps une liberté totale dans l'usage du service public. \\
Depuis une quinzaine d'années cela n'est plus vrai car la laïcité tend à s'appliquer aux usagers ou aux citoyens. Cela conduit donc à une neutralisation des opinions religieuses. 


Concernant le principe de laïcité et les agents. Les agents sont les représentants de l'État et soumis à une stricte obligation de neutralité: CE, avis, 3 Mai 2000, Mlle Marteaux, le CE commence par nous rappeler que les agents bénéficient d'une liberté de conscience dans l'accès aux fonctions mais aussi dans le déroulement de carrière: il n'est pas possible de discriminer l'agent en fonction de ses opinions politiques ou religieuses. Le CE rajoute un deuxième élément: lorsqu'ils exécutent le service public, les agents sont soumis à une stricte neutralité. \\
Ces dispositions sont compatibles avec la CEDH, 26 Novembre 2015, Ebrahimian contre France. La CEDH exerce un contrôle limité sur la question religieuse et laisse une grande marge de manoeuvre à l'État membre. \\
S'est posé une autre question quant aux agents: peuvent-ils disposer d'une dispense d'assiduité pour questions religieuses ? La réponse est négative. CE, Référé, 16 Février 2004, Amhed B. Cette solution se justifie par le principe de continuité du service public qui s'oppose à ce qu'un agent s'absente. 


Concernant le principe de laïcité et les usagers. Le droit positif est en train d'évoluer depuis une dizaine d'années. Il y a encore 20 ans, tous les usagers, de tous les services publics, pouvaient manifester si ils le souhaitaient une opinion religieuse au sein du service public. Cela découlait directement du principe de laïcité: c'est parce que l'État est neutre que la liberté de conscience est possible. \\
CE, 2 Novembre 1992, Kherouaa, dans le contexte de l'apparition de certains foulards dans les collèges et lycées. Le CE a censuré l'exclusion d'élèves au motif qu'elles portaient des signes religieux. Le CE dira que les usagers ne peuvent pas être  discriminés dans son accès au service public en fonction de ses opinions politiques et religieuses. Le CE ne fixe que deux limites: le prosélytisme qui est interdit et la sécurité publique. \\
Les choses ont basculés avec la loi du 15 Mars 2004, codifié en L141-5-1 du code de l'Éducation qui vient prohiber pour certains usagers (ceux des écoles, collèges et lycées) le port de signes manifestant ostensiblement l'appartenance religieuse (Kippa, Voile, turban, croix de grand dimensions). Des textes s'opposent à l'extension de cela aux université, par exemple la franchise universitaire qui interdit, par principes, aux forces de l'ordre de rentrer dans une université. La question s'est posé aussi pour des championnats sportifs. \\
Un usager peut-il bénéficier d'une dispense d'assiduité pour motifs religieux ? Oui si deux conditions sont remplies: la dispense doit être nécessaire à l'exercice du culte, il faut aussi que la dispense soit compatible avec le bon fonctionnement du service public CE, 14 Avril 1995, Kohen. 


Concernant la laïcité et les personnes privées chargées de service publics. L'obligation de neutralité s'applique aussi à ces personnes là, notamment à ses agents qui doivent être neutres. La C.Cas, C.Soc, 19 Mars 2013, CPAM Saint-Denis l'a rappelé récemment: dès qu'une personne privé remplit un service public, il représente aussi l'État. \\
Des questions ont pu se poser: une personne privée chargée de service public peut-elle disposer d'une dispense d'assiduité ? La personne peut bénéficier d'une telle dispense, si deux conditions sont remplies: dispense compatible avec le fonctionnement, et nécessaire à l'exercice du culte. CE, 23 Décembre 2011, Mr Halfon. \\
Deuxième question: la neutralité s'applique-t-elle aux collaborateurs occasionnels ? Notamment des bénévoles, les bonnes soeurs faisant des visites en prison, un parent d'élève accompagnant une sortie scolaire, etc. En droit positif, rien n'interdit d'un tel port. TA Montreuil, 22 Novembre 2011, Osman, le TA a considéré qu'en tant que collaborateur du service public, on ne peut pas porter de signes religieux. CE, 27 Juillet 2001, Syndicat national pénitentiaire FO: le CE devait se prononcer sur la participation de soeurs au service pénitentiaire. Le CE a admis que ces bonnes soeurs participent, avec leurs signes religieux. \\
Après une circulaire du ministre de l'éducation de 2011, le CE dira en 2013 qu'en l'état actuel du droit positif, rien ne dit que les collaborateurs occasionnels sont soumis au principe de neutralité. \\
Le TA de Nice, le 9 Juin 2015, Mme Dahi, qui, en substance, a admis que l'accompagnateur puisse porter un signe religieux. Le TA de Nice ne considère pas que l'accompagnateur n'est pas un collaborateur mais est un usager et dispose donc de sa liberté de conscience. Deuxième apport: les restrictions à la liberté religieuse ne peut intervenir que sous le joug d'un texte particulier, qui en l'espèce, n'existe pas.


S'est enfin posée la question de savoir si le principe de neutralité peut s'appliquer aux personnes privées. En principe, on devrait répondre par la négative, et malgré quelques tentatives, la C.Cas continuera d'aller dans ce sens: C.Cas C.Soc, 19 Mars 2013, Fatima x. contre Babibou, les juges du fond essayaient désespérément d'appliquer la laïcité, la C.Cas rappelant que non, cela ne s'applique qu'à l'État. \\
Cependant, d'après la C.Cas, selon des besoins particuliers d'entreprise, il peut être instauré une certaine neutralité, C.Cas, 25 Juin 2014, Ass. plénière, Fatima x. contre Babibou. La C.Cas va valider le licenciement de Mme Fatima x. \\
Ce point de vue est aussi celui de la CEDH, 15 Janvier 2013, Eweida, la CEDH rappelle la liberté de conscience mais admet des restrictions au port d'un signe religieux sur des motifs de santé ou sécurité publique ou bien l'image de marque de l'entreprise. La CEDH dira en l'espèce que l'interdiction était trop forte. 


Une dernière question se pose: peut-on étendre la laïcité à tout signe religieux au delà du service public ? \\
L'affaire "emblématique" et d'actualité est celle du Burkini qui représente très bien la question qui a été posée au CE. Pour les maires ayant pris l'arrêté anti-burkini, leurs arguments étaient la sécurité publique, l'hygiène, le principe de laïcité. Le TA, en se prononçant au fond, autorise l'interdiction avec une motivation étonnante: "les plages doivent rester un lieu de neutralité religieuse", il redéfinit le statut des plages. Le CE a été saisi en cassation et a rendu son ordonnance de principe le 26 Août 2016, LDH, le CE suspend les arrêtés anti-burkini en considérant que ces arrêtés portent atteintes à trois libertés fondamentales, liberté de conscience, liberté d'aller et venir, la liberté personnelle. Pour le CE, l'arrêté n'est pas pris pour la sécurité. Le juge indique que la laïcité n'est pas une composante de l'ordre public. Le juge rajoute "l'émotion et les incertitudes résultant d'attentats terroristes ne peuvent résulter par une interdiction", en clair, la peur n'est pas une composante de la sécurité publique. \\
Une autorité de police administrative n'est pas compétente pour interdire le burkini sur une plage. Cette décision signifie aussi qu'on ne peut pas interdire un signe religieux ni sur une plage donc ni dans un quelconque espace public. \\
Deux jours après l'ordonnance du CE, le président du CE, le PM donc, s'est permis de critiquer la décision du CE et a donc proposé de faire une loi pour interdire le burkini. Si un tel texte visait à être adopté, il y a de fortes chances pour qu'il ne soit pas conforme à la Constitution car devrait être considéré soit comme discriminatoire, soit comme étant une atteinte excessive aux libertés. 


Il y a sans doute une dérive aujourd'hui dans le principe de la laïcité qui servirai à justifier des lois contre une population religieuse particulière.


Une deuxième question se pose: la question de l'exercice du culte et de l'implication de l'État. La liberté d'exercice du culte est un corollaire de la laïcité. L'État doit mettre tous les moyens en oeuvre pour garantir le libre exercice du culte. \\
La première obligation de la personne publique est de fournir un lieu de culte. Le CE a précisé cette possibilité, CE, Ass, 19 Juillet 2011, Commune de Montpellier, il doit y avoir le respect du principe d'égalité, la mise à disposition est temporaire, l'association cultuelle ne peut pas être financé, la mise à disposition se fait donc nécessairement en contrepartie pécuniaire. \\
Les collectivités peuvent faire des BEA, c'est un bail de très longue durée (99 ans), permettant au bailleur d'obtenir des droits réels sur la parcelle qu'il occupe, lui permettant de construire des édifices.


Une autre question est la pratique d'un culte dans les service publics pénitentiaires. L'État doit-il organiser une visite des ministres du culte en prison ? La réponse est en principe positive mais certains établissements avaient refusés en disant qu'il y avait trop peu de détenus concernés. Le CE a censuré une telle approche, la seule limite étant la sécurité et le bon ordre de l'établissement. CE, 16 Octobre 2013, Ministre de la Justice. \\
Une autre question s'est posée: la prison doit-elle fournir une alimentation conforme à la religion des détenus ? 7 Novembre 2013, TA Grenoble, il considère que les établissements pénitentiaires ont l'obligation de fournir ce genre de repas, résultant de la situation de contrainte des détenus. Le CE dira l'inverse en prenant en compte les contraintes matérielles qui selon le juge, fait que la fourniture de repas spéciaux a un coût, mais surtout l'obligation de fournir la liberté d'exercice du culte n'a pas été violé car les détenus n'ont pas l'obligation de manger ces repas mais surtout parce que les détenus de certaines religions peuvent avoir d'autres aliments en payant, satisfaisant selon le CE la liberté d'exercice du culte. \\
Une dernière question se pose sur les menus de substitutions. Le TA de Dijon s'est prononcé le 21 Octobre 2015, Ligue de Défense Judiciaire des Musulmans, il n'y pas de problème juridique à supprimé ces menus. 


Concernant la laïcité des bâtiments publics. \\
C'est une question qui n'en était pas une depuis longtemps, car réglé par l'article 28 de la loi de 1905, qui interdit d'apposer un signe religieux sur les monuments publics ou en quelque emplacement que ce soit: c'est l'existence d'un principe de neutralité des bâtiments publics. Il y a quelques exceptions, inscrites directement après, qui concerne les sépultures dans les cimetières, pour les monuments funéraires, pour les musées ou expositions. \\
TA Rennes, 2 Mai 2015, Fédération de la libre pensée, concernant une petite commune où le maire avait décidé d'édifier une statue de Jean-Paul II. Le TA de Rennes considère que la statue est contraire au principe de neutralité, mais dira que c'est la croix qui pose problème et non pas l'ensemble de la statue. Cela est audacieux, car en application des droits de propriété intellectuel, la statue ne pouvait pas être modifié et a donc été entièrement démonté. \\
CEDH, Grande chambre, 18 Mars 2011, Lautsi c/ Italie, la Cour se prononçait à propos de la présence de Crucifix dans les salles de classes italienne car ça ne pose pas de contraintes excessives à la liberté de conscience des Italiens. La solution se fonde sur la liberté d'appréciation des États membres, de leur marge de manoeuvre pour régler ces questions. Cette décision a ouvert une brèche car pour la Cour de Strasbourg, la tradition peut justifier cette présence. \\
La question s'est reposée en 2014 en France à propos de l'installation de crèches. Le TA de Montpellier rejette le recours pour défaut d'urgence, le 19 Décembre 2014. Au TA de Nantes, la crèche est bien un symbole religieux et doit donc être retiré. Pour le TA de Melun, la crèche a perdu son caractère religieux et peut donc être installée. Les CAA ont des décisions aussi éclatées. Le CE a rendu un arrêt dé principe le 9 Novembre 2016, Fédération départementale des libres penseurs de Seine et Marne et Fédération de la libre pensée de Vendée. Le juge rappelle les termes de l'article 28 de la loi de 1905, et ses exceptions, dont il retient le titre d'exposition et les sites religieux apposés avant 1905. Dans un deuxième temps, le CE analyse la crèche, qui peut revêtir deux significations, le symbole religieux mais aussi le symbole de décoration traditionnelle "sans signification religieuse  particulière". Le CE va donc en tirer une conséquence, c'est en principe interdit sauf si c'est dans un but culturel, artistique ou festif. Le CE va plus loin concernant les emplacements publics en disant que les installations de fête de fin d'année sont nécessairement festif. \\
On note que le CE interprète très extensivement la loi de 1905. Pour des commentateurs, la décision serait contra legem. Cette décision est à la fois politique et une décision de compromis. 


Concernant le financement des cultes. \\ 
C'est en principe prohibé pour la puissance publique par l'article 2 de la loi de 1905. Valeur Constitutionnelle depuis QPC 2013, association de lutte pour l'extension de la laïcité. \\
Limites traditionnelles: ces limites se trouvent dans le texte même de la loi de 1905. Concernant les édifices de cultes, elles sont en principe propriété de la puissance publique mais laissés gratuitement à la disposition des ministères du culte. Deuxième limite, la personne publique peut entretenir et assurer la conservation des édifices du culte qu'elle possède. La loi de 1905 considère dans son article 19 que l'argent dédié à cet entretien n'est pas considéré comme une subvention. La collectivité a la possibilité de garantir un prêt bancaire d'une association cultuelle pour la construction d'un édifice. Les départements d'Alsace et de Moselle ont un droit local qui permet le financement du culte par la puissance publique ; la constitutionnalité de ce droit local a été posé au CC qui va considéré que le droit local n'est pas contraire à l'article premier de la Constitution. Le CC se réfère aux travaux préparatoires et va considéré que l'article premier contient une dérogation implicite en faveur de l'Alsace et la Moselle. \\
Les nouvelles limites: elles ont été posées par le CE le 19 Juillet 2011. Le CE admet que les collectivités territoriales peuvent intervenir en faveur des cultes, pour plusieurs raisons. Le CE fait évoluer les règles afin de garantir l'exercice effectif de certains cultes car le juge note l'apparition de nouvelles religion et la progression d'autres. Pour le CE, c'est donc le principe d'égalité qui justifie cette décision. Pour le CE, les dépenses liées au culte doivent être le fruit d'un besoin public local CE, Trelazet, le CE dira que la subvention de l'installation d'un orgue remplit un besoin public local, non seulement pour le culte mais aussi pour l'école de musique locale. CE, 17 Février 2016, Association de la libre pensée du Rhône, dans laquelle le juge admet qu'une collectivité française puisse financer la rénovation d'un lieu de culte à l'étranger (une église en Algérie), ce financement étant légal car le juge considère qu'il y a un intérêt public touristique. 

\subsubsection{La continuité}

Le principe de continuité est un corollaire du principe de responsabilité de l'État et implique le droit des administrés au fonctionnement normal des services publics. Cela implique donc une continuité temporelle du service public mais aussi spatiale et géographique. \\
CE, 19 Février 2010, Mr Pierre. Le juge se prononçait à propos de la carte judiciaire qui conduisait à la suppression de tribunaux en France. Le CE censure partiellement cette carte, considérant que la suppression de certains tribunaux portaient atteinte au principe de continuité du service public. \\
La plupart du temps, la continuité est défini par la Loi elle même. Une loi du 9 Février 2010 rappelle par exemple les missions de service publics de la Poste, concernant l'aménagement du territoire où le législateur apprécie la dimension du maillage territorial de la Poste et signale donc qu'il ne peut pas exister moins de 17 000 points de vente en France. 


Concernant la conciliation de la continuité du service public avec le droit de grève. \\
Le principe de continuité doit être concilié avec un autre principe à valeur constitutionnel: le droit de grève. Celui ci doit s'exercer dans le cadre des lois qui le réglemente. \\
Limitation par les autorités administratives: cette possibilité a été prévue par le CE lui même, CE, Ass, 7 Juillet 1950 Dehaene qui interprète le préambule de 1946, qui conduit le CE à admettre qu'une simple autorité administrative peut limiter le droit de grève même en l'absence de dispositions législatives. Selon le juge, elle ne peut le faire que pour certaines raisons: pour des raisons d'ordre public, en vue d'éviter un usage abusif du droit de grève, ou pour les besoins essentiels du pays. Le CE donne aussi les autorités compétentes pour une telle limitation, il y en a trois si on regarde aussi les jurisprudences postérieures: l'autorité en charge du bon fonctionnement du service public (le chef de service donc) ; organe dirigeant d'un EPIC ou d'une personne privée chargée de service public qui doivent pouvoir agir en vertu "des pouvoirs généraux des services placées sous leur autorité" (CE, Ass, 12 Avril 2013, FFO Énergie mine) ; enfin, le préfet qui dispose d'un pouvoir de réquisition qui peut porter sur des biens mais aussi des personnels, si l'ordre public le nécessite, il peut aussi réquisitionner des personnels privés si le maintien d'une activité est fondamental pour l'économie du pays (notamment personnel de raffinerie de pétrole). \\
Le juge vérifie si ces limitations sont proportionnées. Le juge contrôle la justification du droit de grève et si il existe un motif qui justifie la limitation. Il va surtout contrôler la proportion d'une telle limitation. CE, 6 Juillet 2016, Syndicat CGT Cadres et techniciens parisiens, la ville de Paris avait imposée à ses agents un préavis de grève de 48h avant que l'agent ne commence à faire grève, empêchant les agents de rejoindre un mouvement de grève. \\
Le CC admet des limitations par le législateur, CC, 25 Juillet 1979 "Pont à péage", le législateur peut aller jusqu'à l'interdiction du droit de grève lorsque l'agent travaille dans un secteur essentiel aux besoins du pays. Le législateur a donc interdit la grève dans les services publics régaliens notamment, les gardiens de prisons ne peuvent pas faire grève, les magistrats, les militaires, la police. Dans d'autres cas, le législateur peut encadrer le droit de grève, ce qu'il fait depuis quelques années avec le "service minimum", institué d'abord dans les transports (loi du 21 août 2007), puis étendu dans l'éducation (service minimum d'accueil en maternelle et en élémentaire, Loi du 20 août 2008), puis dans le transport aérien (se contentant d'améliorer la prévisibilité des grèves, 19 Mars 2012). 

\subsubsection{Les nouveaux principes ?}

Le principe de gratuité n'a jamais été reconnu par le juge, CE, Ass, 10 Juillet 1996, direct mail promotion. Cependant, certains textes législatifs peuvent instituer une telle gratuité, certains textes constitutionnels le prévoient (le préambule de 1946 évoque l'enseignement public, qui est gratuit et laïque). On peut se poser la question de la tarification du service public. Dans les SPIC, la gratuité n'est pas de mise, mais la plupart du temps, le tarif sera supérieur au coût du service. CE, Ass, 7 Juillet 2007, Syndicat national de défense de l'exercice libéral de la médecine à l'hôpital, le juge indique que le SPIC peut mettre en place une redevance excédant le coût du service afin de tenir compte de l'avantage économique procuré au bénéficiaire du service. Concernant les SPA, il faut distinguer les services facultatifs ou obligatoires, les obligatoires sont gratuits, à l'inverse, les facultatifs peuvent faire l'objet d'une redevance, qui est encadrée et nécessairement inférieur au coût réel. \\
Il n'existe pas de principe de qualité du service public, ni de principe de transparence de service public. Il n'existe pas de principe de participation aux usagers au service public. 


\subsection{La soumission des services publics au droit du marché}

Le droit du marché ce sont principalement le droit de la concurrence et le droit de la consommation. Ce sont des droits privés, qui, depuis 20 ans sont applicables aux services publics.

\subsubsection{La soumission du service public au droit de la concurrence}

Le droit de la concurrence est contenu dans le code de commerce principalement et prohibe deux types de pratiques: les abus de position dominante ou encore les ententes entre opérateurs économiques (L420-1 et L420-2 du C.Com). Il existe aussi du droit européen qui prohibe les mêmes pratiques (Art. 101 et 102 du TFUE). Le droit européen est plus large, car prohibe aussi les aides d'État (Art. 107 TFUE). \\
La CJUE adopte une interprétation très large de l'activité économique: toute activité de prestations de service ou d'offres de produit, donc toute activité d'entreprise exercée sur un marché. Dès lors qu'elles exercent une activité économique, les service publics sont sous le joug du droit de la concurrence. \\
Les SPIC sont des activités économiques. Les SPA peuvent être considérés comme tel, certains SPA sont donc aussi soumis au droit de la concurrence. Il est à noter que des considérations d'intérêt général peuvent permettre de déroger à ces règles de concurrence, prévue au TFUE, le CE l'a confirmé, CE, 3 Novembre 1997 "Millon et Marais". 


Une première conséquence de cela, c'est une conséquence concernant la gestion des services publics, certains modes de gestion sont attentatoires aux règles de la concurrence, comme une mode de gestion en monopole. Les règles de la concurrence ont aussi une conséquence sur les financements comme les subventions croisées, qui permettent de financer un service public avec l'argent d'activités concurrentiels, cela est commun en France mais est considéré comme suspect par le juge de l'Union. \\
Une deuxième conséquence est l'ouverture à la concurrence de certains services publics, autrefois gérés en monopole par l'État, préconisé par le droit européen. Il s'agit essentiellement aux services gérés en réseaux: électricité, gaz, télécommunication, Poste, transport. Toutes ces activités sont progressivement ouvertes à la concurrence depuis 25 ans. Il est à noter que l'ouverture à la concurrence ne supprime pas le service public, l'idée est que le service public est réparti entre tous les opérateurs, ou son coût. Il y a la plupart du temps une autorité de régulation comme la CREU, l'ARCEP, l'ARAFER. L'objectif de ces libéralisations étaient de réduire le prix de revient aux consommateurs, ce qui a pu marché dans certains secteurs (télécommunications), mais pas dans d'autres (énergie). \\
Une troisième conséquence est que les règles de concurrences ne sont pas absolu pour le service public, le droit de l'Union permet d'y déroger dans l'intérêt général. Le TFUE lui-même le prévoit, si la dérogation se fait dans un SIEG, des Services d'Intérêt Économique Générale, la dérogation doit être proportionnée et demande donc l'atteinte la moins forte.


Depuis 1997, non seulement les services publics sont soumis aux règles de concurrence mais les actes administratifs le sont soumis aussi. CE, 3 Novembre 1997, Millon et Marais, le juge considère que les règles de la concurrence sont opposables aux actes administratifs, qui ne peut donc pas avoir pour objet ou pour effet de placer un opérateur en situation anti-concurrentiel. \\
Depuis 1997, ce sont donc tous les actes de la puissance publique qui doivent donc respecter les règles de la concurrence. Il y a une recherche de conciliation entre l'intérêt général et l'intérêt du marché. \\
On note d'ailleurs que l'intérêt général, en droit positif, contient la concurrence qui est donc aussi un impératif d'intérêt général. 

\subsubsection{La soumission des services publics au droit de la consommation}

Des dispositions comme celles contre les clauses abusives dans les contrats, sont opposables aux services publics: CE, 11 Juillet 2001, Syndicat des eaux du Nord. Le juge administratif est donc aussi un juge des clauses abusives. \\
Comme dans le droit de la concurrence, l'application du droit de la consommation n'est pas absolue, le service public peut là aussi y déroger pour des motifs d'intérêt général. 


\chapter{Police administrative}

C'est une mission de l'administration, qui tend à préserver et garantir l'ordre public. Cette mission va donc avoir pour conséquence de limiter les libertés individuelles des administrés. Cette mission est précisément nécessaire mais aussi dangereuse, raison pour laquelle ces missions de police sont encadrés afin d'atteindre le moins possible aux libertés individuelles. \\
Le JA contrôle d'abord la finalité de l'acte, la compétence de l'autorité qui édicte l'acte et enfin, la proportionnalité de la mesure de police. Dans certains cas, les pouvoirs de polices peuvent être étendus de manière exceptionnelle. \\
À noter que le JA contrôle la mesure de police comme tout autre acte administratif, et contrôle donc cet acte à toutes les normes supérieures, y compris le droit de la concurrence: CE, avis, 22 Novembre 2000, L\&P Publicité, le CE va concilier l'intérêt général de limiter les pubs extérieurs avec les règles de concurrences. 

\section{Le but de la police administrative}

La police tend à préserver et à protéger l'ordre public. Sa mission est donc nécessairement préventive.

\subsection{Le caractère préventif de la police administrative}

C'est par ce caractère préventif que l'on peut voir la différence entre la police administrative et la police judiciaire, ce qui est fondamental puisque permet de définir le droit applicable. La police administrative relève du droit public alors que la police judiciaire relève bien souvent du Pénal. \\
La police administrative vise à prévenir les infractions, elle est préventive en cela. La police judiciaire a elle, une mission répressive car intervient dès lors qu'une infraction a été commise. La police judiciaire a été défini par le CE, 1960, Frampar, selon le juge, la police judiciaire tend "à constater des crimes et des délits ainsi que livrer les auteurs de ces infractions aux tribunaux judiciaires". TC, 7 Juin 1951, Noualek, Madame Noualek a été blessé par une décharge de plomb tiré par des OPJ, elle va demander indemnisation devant un tribunal civil. Le TC, pour définir la compétence, va rechercher la finalité de la mission des OPJ, qui était hors de tout cadre d'enquête judiciaire et donnera compétence à l'administratif. \\
Deux raisons qui expliquent la complexité de la différenciation, la première est que la mission de la police administrative peut se prolonger en mission de police judiciaire. La seconde difficulté est que ce sont souvent les mêmes autorités qui remplissent les deux missions. \\
TC, 12 Juin 1978, Société Le Profil, qui avait subi une attaque à main armé, a cherché à se faire indemniser de son préjudice en attaquant l'État qui n'aurait pas suffisamment prévenu l'atteinte possible, et n'aurait pas poursuivi les auteurs de l'infraction. Le TC, pour trancher la juridiction compétente, a recherché le but de la mission et finira par donner l'affaire au JA, le préjudice trouvant son origine essentiellement dans la mission de la PA. 

\subsection{La finalité d'ordre public}

L'ordre public est une notion plus précise que l'intérêt général. L'ordre public comprend traditionnellement un ordre public matériel, qui renvoie à trois composantes: la sécurité publique, la tranquillité publique, la salubrité publique. C'est la finalité historique de la police administrative. \\
Le JA a développé un ordre public immatériel, comportant notamment la dignité de la personne humaine. 

\subsubsection{L'ordre public matériel}

Sécurité, tranquillité, salubrité. Ces trois composantes sont rappelés dans le CGCT à l'art L.2212-2. CE, Sect, 12 Novembre 1997, association communauté tibétaine de France, le préfet de police de Paris avait interdit toute manifestation pendant la visite du Président Chinois, afin de préserver les relations internationales de la France. Le juge va censurer un tel arrêté, qui ne relève pas de l'ordre public. \\
CE, Référé 2016, LDH, le juge censure l'arrêté anti-burkini, n'étant pas motivé par la protection de l'ordre public. \\
L'autorité de police prévient les atteintes à l'ordre public et intervient donc sur tous les lieux publics. Elle ne peut donc pas intervenir dans un domicile privé, sauf quand une activité privé aura pour effet de déborder sur l'extérieur, CE, 2 Juillet 1997, Mr Bricq, le CE dira que le Maire peut interdire l'usage d'une tondeuse à gazon le matin. \\
L'espace public, ce sont aussi les lieux privés ouvert au public (un magasin par exemple). \\
Pendant un temps, il existait une quatrième composante à l'ordre public matériel: l'esthétique. L'esthétique peut être le but d'une police spéciale: la police de l'urbanisme. 

\subsubsection{L'ordre public immatériel}

On compte trois composantes principales. La moralité, puis la dignité de la personne humaine, les "exigences minimales de la vie en société"


Concernant la moralité publique, la moralité ne peut pas être un but de la police, la morale étant une notion subjective. Une telle mission serait largement liberticide. La morale n'est donc pas en elle même un but de la police administrative. Cependant, la police administrative peut parfois poursuivre un tel but, à une condition importante: l'autorité doit démontrer des circonstances locales particulières. CE, Sect. 18 Décembre 1959, Société Les films Lutecia, était en cause une projection de film dans une commune que le Maire a souhaité interdire. Le CE va valider l'interdiction à cause des circonstances locales, tenant à la forte population jeune, et aux risques de protestation des habitants de la commune. \\
CE, Sect, 8 Décembre 1997, Commune d'Arcueil, le JA va censuré une interdiction de publicité pour les messageries roses, en l'absence de circonstances locales particulières. 


La dignité humaine est un principe à valeur constitutionnel depuis 1994 et une composante de l'ordre public depuis 1995, CE, 27 Octobre 1995, Commune Orson sur Orge, le CE a admis l'interdiction d'un spectacle qui consiste en un lancer de nains. \\
La notion même de la dignité de la personne humaine est une notion profondément floue qui peut signifier tout et son contraire. Il existe deux conceptions de la dignité de la personne humaine, révélée par le CE dans une rapport de 2010, il y voit une conception objective et une subjective. La conception objective est celle qui est mis en oeuvre dans la décision de 1995, qui vise à protéger des personnes contre elle même (le nain qui voulait participer, les mannequins anorexiques). D'un point de vue subjectif, la dignité se confond avec la liberté en générale, pouvoir agir libre sans que la puissance publique délivre un modèle comportement. En droit allemand, cette conception est un principe fondamental du droit allemand. On note que les deux points de vue sont contraires et peut être liberticide. \\
C'est en ces raisons que le CE utilise cette notion de façon parcimonieuse. Le CE l'a utilisé à quatre reprises, après la première fois, le CE a attendu 12 ans, CE, Référé, 7 janvier 2007, Association solidarité des français, le JA avait interdit ces soupes, considérant qu'ils étaient contraire à la dignité de la personne humaine, excluant une partie de la population. \\
CE, Refus de concours de la puissance publique en cas d'expulsion peut être accordé si un tel concours poserait un problème d'ordre public ou si était contraire à la dignité de la personne humaine.


CE, Référé, 9 Janvier 2014, Ministre de l'intérieur contre Société de production de la plume et Mr M'bala m'bala. Le CE valide l'interdiction du spectacle de Dieudonné, pour le juge, annuler le spectacle est la seule solution pour protéger la dignité humaine. Cette décision met en place un régime préventif en matière de liberté d'expression. Le droit positif ne semblait pas permettre une tel interdiction, certains juristes pensaient que la circulaire de Valls ordonnant aux préfets d'interdire ses spectacles allait être annulés. Dans un premier temps, le TA de Nantes annule l'interdiction du spectacle. Dans la même journée, un appel est interjeté, et le CE valide l'interdiction dans la journée même. La durée exceptionnellement réduite peut poser des questions d'impartialité, mais on peut aussi penser que l'appel était prévisible donc que le CE a préparé son sujet. Fait étonnant, dans cet arrêt, le CE met en visa ses propres décisions. \\
On peut voir une nouvelle conception de la dignité humaine par le CE dans l'arrêt précédent. On ne protège pas ici la dignité d'une personne contre elle même mais un groupe de personne tiers. Le juge considère que la dignité justifie l'interdiction. Le CE dégage d'autres composantes à l'ordre public qui justifieraient l'interdiction: la cohésion nationale, la tradition républicaine. Ce sont des références jusqu'alors inconnu. Elles ne seront pas reprises dans des ordonnances suivantes. \\
On peut interroger les conséquence sur les libertés fondamentales. La liberté d'expression est fondée sur un régime répressif, tout le monde peut s'exprimer comme il le souhaite, et si quelqu'un dépasse les limites, il sera condamné. C'est donc un régime libéral contrairement à l'ordonnance de 2014 qui permet un régime préventif en matière de liberté d'expression. \\
On pouvait penser que d'autres solutions étaient possible en espèce. L'arrêt Benjamin notamment (1933), le CE avait refusé une interdiction à priori en la considérant comme disproportionné, le juge considérant qu'il y avait d'autres moyens plus doux de conserver l'ordre public comme déployer des forces de polices. En 1933, on considère donc là qu'il faut laisser les gens s'exprimer puis gérer ensuite les débordements, alors qu'aujourd'hui, on considère l'inverse en se basant sur une protection d'un ordre public immatériel et non matériel comme dans l'arrêt Benjamin. \\
L'autorité de police préjuge donc en considérant qu'untel va commettre une infraction. \\
La circulaire Valls est probablement le fruit d'un avis consultatif sur lequel s'est basé le ministre pour y écrire sa circulaire. Cela peut poser un problème de partialité. \\
TA Nice, 2015, CRAN, affaire des pâtisseries racistes, le CRAN avait demandé au maire d'intervenir pour interdire l'exposition de pâtisserie raciste, car était contraire à la dignité humaine. L'activité privé débordait sur la rue, donc sur le public, le pouvoir de police était donc compétent. Le TA de Nice a accédé à la demande des requérants. Le TA considérera que les pâtisseries sont contraire à la dignité, et que celle-ci est une Liberté fondamentale, ce qui est étonnant, la dignité permet de limiter les libertés. Le TA de Nice considère aussi que le principe de la dignité était dans la DDHC, ce qui est faux, le CC ayant élevé le principe à valeur constitutionnelle et non en tant que PFRLR. Le CE censurera en appel l'ordonnance du TA de Nice, considérant que le juge des référés ne peut pas intervenir en cas de carence.  


Le juge constitutionnel a peut être dégagé une troisième composante de l'ordre public immatériel, les "exigences minimales de la vie en société". Cette composante a été dégagée le 11 Octobre 2010, CC, "Loi interdisant la dissimulation du visage dans l'espace public". \\
Le CE disait qu'une telle interdiction ne pouvait être justifiée par la protection de la dignité. Pour le CE, la sécurité était en elle même un argument permettant l'interdiction. \\
Le CC propose lui, plusieurs fondements à cette interdiction. Le CC considère que les impératifs de sécurité publiques justifient une telle interdiction. Mais le CC va aussi se référer aux exigences minimales de la vie en société. La formule est à questionner. Comme avec le principe de la dignité, la notion est là aussi floue et permettrai d'interdire tout et n'importe quoi. La notion est donc utilisée comme un couteau suisse dont on ne connaît pas le contenu. \\
Le CC va aller plus loin dans l'extension de l'ordre public, le CC va considéré que porter un objet qui dissimule sont visage porte atteinte à la liberté et l'égalité, qui ne sont pas des composantes de l'ordre public. \\
CEDH, 1 Juillet 2014, SAS c. France, la Cour de Strasbourg se fonde sur la marge de manoeuvre dont dispose les États partis, et que cette interdiction relève d'un choix de société, et donc que l'État français dispose d'une marge de manoeuvre importante. \\
Si le CC devait valider une interdiction de tout signe religieux dans l'espace public, il le ferait avec ce principe. Néanmoins, cela pourrait contrevenir à la liberté religieuse consacrée par la CEDH.  

\section{Les autorités de police administrative}

\subsection{Identification}

\subsubsection{Les autorités de police générale}

Il en existe trois catégorie. La première est le pouvoir de police du Premier Ministre, qui, par sa détention du pouvoir réglementaire, peut prendre des règlements de police applicables au territoire entier. Dans la Constitution précédente, c'était le Chef de l'État. CE, 7 Mai 2008, Collectif pour la défense des loisirs verts. Le ministre de l'intérieur n'est pas une autorité de police générale contrairement à ce que l'on pourrait penser, cependant, c'est le chef des services de police, qui sont donc sous son autorité. Le ministre de l'intérieur peut donc donner des instructions à ses services ou aux préfets pour l'exercice des pouvoirs de police. \\
La deuxième catégorie est le pouvoir de police des préfets qui sont chargés du maintien de l'ordre dans les départements. Cela pose la question des répartitions de compétences entre les préfets et les maires, qui ont aussi un pouvoir de police administrative. L2215-1, le préfet peut intervenir dans trois hypothèses: si il prend des mesures dépassant le territoire d'une seule commune ; en cas de carence du Maire, à une condition: mettre en demeure le Maire d'intervenir (c'est le pouvoir de substitution d'action) ; dans les communes de plus de 10 000 habitants dans lesquelles à été étatisé la police administrative, où la sécurité est la mission de la police nationale, sous l'autorité du préfet. Il y a une exception dans la ville de Paris où il existe un préfet de police. La Maire de Paris souhaite obtenir une extension des pouvoirs de police de la Maire, et un texte législatif pourrait remettre en cause le statut de préfet de police. \\
La troisième catégorie est le pouvoir de police des maires qui a autorité sur la police municipale. Celle-ci a pouvoir sur les zones urbaines comme non urbaines de la commune. 

\subsubsection{Les polices spéciales}

La distinction n'a jamais été systématisée par le juge. René Chapus voit trois critères de distinction entre police générale et police spéciale. On a d'abord l'autorité compétente, une police spéciale peut être confiée à toutes personnes publiques alors que la police générale est limitée. \\
Deuxièmement, la finalité de la police spéciale est une finalité à la fois plus large et plus précise. Une police spéciale peut avoir pour finalité un intérêt général et pas seulement la protection de l'ordre public. Elle est plus précise dans le sens où la police spéciale a une finalité très particulière. \\
Troisièmement, la police spéciale est très souvent encadrée par des textes particuliers, suivant des procédures particulières. CE, Sect, 30 Juin 2000, Association Promouvoir, a conduit à une modification de la classification des films, qui est un pouvoir de police spéciale. 

\subsection{Les concours de police}

Ces concours interviennent lorsque des autorités de police différentes interviennent à propos d'un même objet sur un même territoire. 

\subsubsection{Le concours de police générale}

Il est possible mais très encadré. CE, Néris Les Bains, posent trois conditions à un tel concours: seule l'autorité inférieure peut venir compléter une intervention ; pour que l'intervention de l'autorité inférieur soit valable, il faut qu'elle vienne aggravé l'atteinte aux libertés ; il faut qu'il existe des circonstances locales pour justifier l'intervention de police de l'autorité inférieur. \\
Exemple: si un préfet limite à 50 km/h la circulation en agglomération, le Maire peut aggraver la limite, près d'une zone résidentiel ou d'une école. 

\subsubsection{Les concours de police générale et spéciale}

En 1959, un maire peut interdire la diffusion d'un film alors que le ministre de la culture avait accordé un visa d'exploitation. Le concours de police générale et spéciale est donc toléré en 1959. Il existe aujourd'hui trois conceptions d'un tel concours. \\
Le CE considère d'abord que la police spéciale est exclusive et que cela exclut toute compétence de la police générale. CE, 1914, Gurnez. \\
Deuxième conception, la police spéciale a une exclusivité limitée. L'exclusivité n'interdit pas d'intervention de la police générale, notamment en cas de péril imminent alors même que l'autorité spéciale est déjà intervenue. \\
La troisième conception est que le concours serait autorisé.


Le CE semble s'orienter à la première conception. CE, Ass, 26 Octobre 2011, SFR: un Maire peut-il interdire certaines antennes de radiophonie, en haut d'une maternelle ou d'un hôpital sachant que ces antennes sont d'abord autorisé par une police spéciale ? Police spéciale: ARCEP, ANF, le ministre en charge des communications électroniques. Le CE répondra par la négative, l'autorité de police générale ne peut pas intervenir si l'autorité de police spéciale est exclusive: le CE dit que plus la police spéciale est exclusive, plus le concours est impossible. Le CE, pour savoir sir la police spéciale est plus ou moins exclusive, se fonde sur le champ d'application de la police spéciale (national ou local), si la police spéciale est technique (plus elle l'est, plus elle sera exclusive), enfin, le juge prend en compte le caractère plus ou moins complet de la police en cause (en l'espèce, dans l'arrêt SFR, les autorités encadrent entièrement les communications). \\
On peut donc se demander que peut faire un maire face à de tels hypothèses. Ils n'ont qu'une voie de droit: contester les autorisation en justice. \\
Dans cette décisions, les requérants invoquaient le principe de précaution, qui, à lui seul, pouvait justifier l'intervention du maire. Le CE rappellera que ce principe n'est applicable que dans le cadre des compétences de l'autorité administrative. \\
Cette décision SFR semble exclure tout concours de police, mais le CE n'a sans doute pas encore réellement adopté de principe et va probablement continuer à faire varier ses solutions en fonction des contextes. \\
Le même genre de problème s'est posée sur la dissémination d'OGM: un maire peut-il l'interdire alors que la ministre de l'environnement l'avait autorisé ? Le CE répondra par la négative comme dans l'arrêt SFR. 

\subsection{L'obligation d'action des titulaires du pouvoir de police}

L'autorité a l'obligation de prendre toutes mesures pour protéger l'ordre public. Le fait de ne pas prendre de mesure de police peut provoquer des préjudice aux administrés, il y a alors carence. La responsabilité de l'État peut donc être engager en cas d'inertie. CE, 1918, époux Lemonnier. \\
Le refus d'intervention des pouvoirs de police peut être invoqués et le juge dispose d'un pouvoir d'injonction dont il peut user pour forcer une autorité de police d'édicter une mesure. 

\subsection{L'interdiction de déléguer les fonctions de police administrative}

C'est un principe qui tend de plus en plus à être assoupli par le juge. \\

\subsubsection{La consécration du principe}

CE, Ass, 1932, Commune de Castinodary, les compétences en matière de police sont en principe indisponible, donc ne peuvent pas être délégué à des personnes privées, et en particulier via un contrat. On ne peut pas le faire car cette autorité de police doit être exercé par des individus qui sont sous l'autorité de l'administration. TA Montpellier, 19 Janvier 2016, préfet de l'Hérault, où était en cause la création d'une garde à Béziers, composé de bénévoles qui devaient faire des gardes statiques devant les bâtiments public ou des déambulations dans la voie public. Le TA censure car c'est la mission que de la police municipale. \\
Plusieurs justifications à ce principe, jamais vraiment explicité par le CE. D'après la doctrine, la police ne peut pas se déléguer car c'est une mission régalienne, qui appartient par nature à l'État, ou encore qui relève de la souveraineté. \\
Le CC, le 10 Mars 2011, LOPPSI 2, le juge constitutionnel constitutionnalise le principe d'interdiction de la délégation des services de police en se basant sur l'article 12 de la DDHC, qui institue une force publique à l'avantage de tous. 

\subsubsection{La portée du principe}

Interdiction de déléguer la compétence normative: CE, 30 Septembre 1983, Fédération départementale des associations agréés des bêches de lins. Le juge a considéré pendant longtemps que l'activité matérielle de la police ne pouvait pas non plus être délégué. CE, 1er Avril 1994, Commune de Menton, le juge rappelle l'interdiction de déléguer la police du stationnement. \\
CAA Versailles, 19 Mai 2009, Société Jean Lefebvre, était en cause un contrat qui donnait à la société privé une mission de prévention contre le terrorisme, que le juge a censuré.

\subsubsection{L'assouplissement du principe}

Depuis quelques années, le CE admet de plus en plus la délégation de prestations matérielles d'exécution. CE, 24 Mai 1968, Chambrin, qui concernait l'activité de la fourrière, que le juge considère non comme un service de police mais comme un service public. \\
Le CE a confirmé cette approche dans les années 80, puis enfin, a fait un arrêt de principe le 10 Octobre 2011, Ministre de l'agriculture. Le préfet de l'Allier avait pris une mesure de police administrative, ordonnant l'abattage des troupeaux. La mission d'abattre les troupeaux avait été déléguée à un tiers privé. Le CE l'admet car trois conditions sont remplies: ces missions sont détachables de la souveraineté car le prestataire ne fait qu'exécuter, et non décider ; il faut que ce soit seulement des tâches d'exécution, de mise en oeuvre ; la mission doit être exécutée sous le contrôle et la responsabilité de l'administration. Si la personne privée cause un dommage à un tiers, la responsabilité de l'autorité de police pour faute peut être engagée.


Le juge constitutionnel a lui aussi fait évolué le principe. Loi du 10 Juillet 1989, tend à renforcer la sécurité des aérodrome, autorisant des agents de sécurité privé à procédé à des fouilles de personnes et de bagage. Loi du 28 Mars 2003, autorisant des sociétés privés à mettre en oeuvre des palpations sur des spectateurs. \\
CC, 20 Novembre 2003, maîtrise de l'immigration. Était en cause une loi qui donnait à une société privé la mission de transporter des détenus, ce que le CC a admis. Plus récemment, le CC a peut être durci son interprétation: LOPPSI 2, était en cause la délégation de vidéosurveillance de la voie publique, censuré par le CC alors que le Conseil avait admis une telle délégation en 2006. 

\section{La proportion des mesures de police}

Exigence posée par le CE, 1933, Benjamin, le CE va censuré l'interdiction d'une conférence littéraire: le trouble à l'ordre public n'était pas suffisamment grave pour justifier une interdiction, jugée excessive par le CE qui signale que des mesures plus douces auraient permis de rétablir l'ordre. \\
La proportionnalité signifie que la mesure de police ne dépasse pas ce qui est nécessaire pour atteindre l'objectif visé. Le CE recherche donc si des mesures plus douces sont possibles. \\
Première précision: si la mesure de police doit être proportionnée alors une interdiction générale et absolue est par définition suspecte, car c'est la mesure la plus extrême, la plus liberticide. Pour être considéré comme proportionné, une mesure de police doit être limité dans le temps, dans l'espace, quant aux personnes concernées. On retrouve ces précisions dans les jurisprudences de couvre-feu pour mineur: CE, Ord, 9 Juillet 2001, Préfet du Loiret. Le couvre feu est proportionné si il est triplement limité: concerne que les mineurs de moins de 13 ans, ne concerne que certains quartiers, est limité dans le temps. Quand il y a atteinte à la dignité de la personne humaine, l'interdiction est la seule chose possible. CE, 9 Juillet 2003, Lecomte, le juge admet les arrêts anti-mendicité à la condition qu'il soit strictement proportionné: limité dans le temps, dans l'espace, que certaines personnes. \\
CE, Daudignac, Le Maire avait subordonné les prises de photos à une autorisation préalable délivrée par la mairie. Le CE censure et dit que seul le législateur peut mettre en oeuvre ce genre de chose. CE, 2002, Consoeurs Leroy, des soeurs avaient décidés de conserver leur mère congelé, et demandé l'autorisation au préfet, qui a refusé. Le juge dit que le préfet a la capacité d'autoriser une inhumation par la loi, mais pas d'autoriser une congélation.

\section{L'extension exceptionnelle des pouvoirs de police}

Ils peuvent être étendus dans des circonstances particulières pendant les périodes de crises, comme l'article 16 de la Constitution. Mais, outre ces périodes de crise, la Constitution prévoit d'autres régimes spécifiques: l'état de siège dans son article 36. Le législateur a complété ce régime par l'état d'urgence. Le juge a lui aussi posé un régime particulier: les circonstances exceptionnelles. 

\subsection{L'état de siège}

Une loi de 1849 mettait déjà en place ce régime. Il est mis en place par un décret en conseil des ministres en cas de péril imminent, qui résulte soit d'une guerre étrangère ou d'une insurrection à main armée. Cet état de siège doit être prolongée par le législateur au delà de douze jours. \\
Sous le régime de l'état de siège, les pouvoirs de police sont transmis à l'armée mais aussi étendu: les perquisitions sont autorisés de jour comme de nuit. L'armée peut aussi interdire toute publication ou réunion. \\
CE, 28 Février 1919, Dol et Laurent, étaient des prostituées, et le préfet avait interdit cette activité, de peur que les militaires se confient sur l'oreiller.

\subsection{L'état d'urgence}

Loi du 3 Avril 1955, modifiée à plusieurs reprises. C'est un régime mis en place une dizaine de fois sous la Ve République: guerre d'Algérie, émeutes de 2005, attentats de 2015. \\









\part{Les recours au contentieux}




























\end{document}
