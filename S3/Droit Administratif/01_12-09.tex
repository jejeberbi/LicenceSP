\documentclass[10pt, a4paper, openany]{book}

\usepackage[utf8x]{inputenc}
\usepackage[T1]{fontenc}
\usepackage[francais]{babel}
\usepackage{bookman}
\usepackage{fullpage}
\setlength{\parskip}{5px}
\date{}
\title{Cours de Droit Administratif (UFR Amiens)}
\pagestyle{plain}

% GALOP SAMEDI 12 NOVEMBRE 14h - 17h
% "Livre de chevet": LES GRANDS ARRÊTS DES JURIDICTIONS ADMINISTRATIVES

\begin{document}
\maketitle
\tableofcontents

\chapter{Introduction}

Le droit administratif est loin du cliché qu'on peut avoir. \\
Le DA est une relation entre l'Administration et d'autres personnes juridiques, qu'elle soit publique, privé, ou morale. Le DA vise l'intérêt général. C'est un droit de conciliation entre l'intérêt général et les libertés individuelles. Le DA est la matière qui permet de savoir, par exemple, si on peut suspendre la dissémination d'OGM dans une commune ; on lie dans cet exemple la liberté d'entreprendre de l'agriculteur et l'intérêt général. Comme le DA est un droit d'intérêt général, c'est un droit éminemment politique. \\
Le DA est un droit théorique car certaines questions ne sont pas résolues. En l'absence de principes de droit, c'est à la doctrine de trouver des solutions. \\
Le DA est un droit important dans le droit public et contrairement en droit privée ou en droit des affaires, c'est un domaine qui est moins prisé des avocats et juristes. \\
Pour certains, le DA serait inintelligible et très complexe. Cette critique est certainement vrai. On notera que c'est une matière qui s'est complexifié récemment. 


Bibliographie:
\begin{itemize}
\item Les Grands Arrêts des Juridictions Administratives - Dalloz ;
\item Le traité de droit administratif - Dalloz, 2 tomes ;
\item Manuel de droit administratif - Plessix ;
\item Manuel de droit administratif - Lombard, Sirinelli ;
\item Manuel de droit administratif - Truchet ;
\item Actualité Juridique du Droit Administratif - Dalloz, hebdomadaire ;
\item Revue Française de Droit Administratif ;
\item Revue de Droit Administratif - Lexis Nexis, mensuel.
\end{itemize}


À priori, on pourrait penser que le DA est le droit de l'administration. Cette définition est incomplète car il faut se demander ce qu'est l'administration. Ce n'est qu'après avoir défini l'administration qu'on pourrait envisager le droit administratif et le juge administratif.

\section{L'administration}

C'est un terme polysémique, on peut le voir d'un point de vue matériel et d'un point de vue organique. \\
D'un point de vue matériel, l'administration est une activité qui consiste à gérer une affaire. Du point de vue organique, c'est l'organe qui gère cette affaire. Ce terme peut très bien s'utiliser en droit privé comme le Conseil d'Administration d'une entreprise. C'est pour cela que nous parlerons, nous, d'administration publique, l'organe qui gère les activités publiques.

\subsection{L'administration et l'action des particuliers}

L'activité publique se distingue de l'activité des particuliers. L'activité publique poursuit un but particulier: celui de l'intérêt général. Ce but, l'administration le remplit grâce à des moyens spécifiques: des prérogatives de puissance publique. Le particulier va agir pour satisfaire son propre intérêt, et parfois, peut coïncider à l'intérêt commun. Parfois l'intérêt privé ne veut pas prendre en charge les intérêts publics, notamment ceux qui ne sont pas profitables, ou alors ils ne peuvent pas. Ces activités sont celles qui sont prises en charge par l'administration. \\
On doit maintenant définir ce qu'est l'intérêt général pour comprendre l'action de l'administration, et ce concept est impossible à définir. On est sûr que ce n'est en tout cas pas la somme d'intérêts commun. C'est une notion évolutive dans le temps: le CE considérait en 1916 (Arrêt Astruc) que la construction d'un théâtre n'était pas de l'intérêt général (en 1916, pendant la guerre, la puissance publique a d'autres priorités), mais en dehors des périodes de guerre, la culture relève de l'intérêt général. \\
Il n'est pas possible de définir l'intérêt général, mais on peut  l'identifier. Il est d'intérêt général ce que le législateur considère comme intérêt général et ce que le juge considère comme tel.


Contrairement au droit privé, on ne demande pas aux administrés leur accord avant d'imposer une décision. Pour imposer ses vues, l'administration utilise des prérogatives de puissances publiques. Par exemple, pour construire une autoroute, l'État a besoin d'un terrain appartenant à une personne privée et possède donc la capacité de négocier à l'amiable la vente ou alors peut l'exproprier pour faire primer l'intérêt général. \\
L'administration n'a pas l'obligation de contraindre les personnes privés et peut utiliser les outils du droit privé, comme des contrats de droit privés. 


"L'administration est une activité par laquelle les autorités publiques pourvoient à la satisfaction de besoins d'intérêts publics en utilisant, le cas échéant, des prérogatives de puissance publique" Jean Walin. \\
Depuis 2015, il existe une définition juridique de l'administration dans le CRPA, qui définit les finalités de l'administration: agir dans l'intérêt général. Ce code identifie également l'Administration L100-3: sont l'administration l'Administration de l'État ainsi que celle des collectivités territoriales, établissements publics administratif et toutes autres personnalités juridiques chargés d'une mission de service public administratif.

\subsection{L'administration et les autres activités publiques}

Administrer n'est pas légiférer. Le législateur pose des règles générales et impersonnelles, qui s'imposent à tous. C'est le cas aussi de l'Administration. L'activité législative est une activité ponctuelle alors que l'activité de l'administration est continue et plus concrète. Un acte législatif a une valeur législative alors qu'un acte administratif a une valeur infra-législative. \\
Administrer n'est pas juger. L'administration est aussi soumise au droit mais n'a pas pour fonction de trancher des litiges. \\
On peut faire une distinction entre l'Administration et le Gouvernement, celle ci n'a pas de portée juridique. Il y a une différence factuelle entre administré et gouverné, gouverné consiste à prendre les décisions importantes, à diriger la politique de la nation (art. 20 Constitution). Le Gouvernement dispose de l'Administration (Art. 20 Constitution), ce qui signifie que le Gouvernement a un pouvoir hiérarchique. Les décisions du Gouvernement s'imposent à l'administration, à l'inverse, les décisions de l'administration sont celle du gouvernement. Enfin, les actes du gouvernement sont aussi des actes administratifs. 


\section{Le droit administratif}

\subsection{La consécration du droit administratif}

Le droit administratif est un droit assez jeune, il a approximativement deux siècles.

\subsubsection{La soumission de l'administration au droit}

L'administration n'a pas toujours été soumise au droit car dans l'État de police, l'administration était soumise à des règles qui n'avaient de valeur qu'à l'intérieur de l'administration, n'avait pas de valeur juridique, ce qui conduisait à une administration non soumise à un droit. \\
Depuis la Révolution, l'État de droit est apparu progressivement, ce qui a eu deux conséquences: l'administration est liée par les règles de droit et celles qu'elle a posée elle même ; il est possible de contester juridiquement la plupart des actes de l'administrations. \\

\subsubsection{La spécificité du droit de l'administration}

Dans d'autres systèmes juridiques, l'administration est soumise au droit privé et ce n'est que par dérogations qu'elle n'y est pas soumise. En France, l'administration peut se mettre de lui même sous le joug du droit privé (on parle alors de gestion privée de l'administration). Quoi qu'il arrive, l'administration a besoin de règles particulières qui est le droit administratif. \\
En France, il existe le principe de la séparation des autorités administratives et judiciaires. Tout juge ne peut pas trancher des affaires de l'administration. C'est progressivement un juge spécial qui va régler ces affaires de manière différente que les juges judiciaires. Les juges administratifs ont donc construit progressivement le droit administratif.


La décision Blanco du juge des conflits (TC 8 Février 1873), consacre explicitement l'autonomie du juge administratif. Il le fait dans une affaire banale. Blanco est une fillette de 5 ans, renversée par un wagonnet poussé par des ouvriers d'une manufacture publique de tabac. Mr Blanco souhaite donc engager la responsabilité de l'État. Le juge des conflits va se demander quel droit est applicable pour définir la juridiction. Le TC va dire que l'administration doit être soumis à un droit particulier. Cette autonomie se justifie, selon le juge, par les besoins du service (service public). \\
Cet arrêt est souvent cité comme l'arrêt qui crée le droit administratif. \\
C'est un raisonnement moniste. Qui dit droit administratif dit compétence au juge administratif. Il y a un principe de liaison de la compétence et du fond, mais ce principe a été dérogé récemment et il est arrivé que le juge judiciaire applique des règles de droit administratifs. \\
La notion de service public est une notion importante, mais le lien qui existe entre service public et application du droit administratif peut parfois ne pas exister. Les services publics industriels et commerciaux sont par exemple régies par le droit privé. La SNCF remplit une mission de service public mais si on se fait écraser par un wagon c'est devant le juge judiciaire que l'on ira.

\subsubsection{Définition du droit administratif}

La doctrine a tentée de définir le droit administratif. Il y avait un débat entre l'école du service public et l'école de la puissance publique. \\
Pour la première, que l'on appelait aussi l'école de Bordeaux, la notion de service public permet de définir ce qu'est le droit administratif afin de déterminer la compétence du juge administratif, de définir d'autres notions du droit administratifs. Cette tentative doctrinale a échouée pour trois raisons: l'administration exerce d'autres activités que le service public (comme la police administrative, la réglementation, gestion des biens) ; la notion de service public est particulièrement floue, c'est une mission d'intérêt général, or l'intérêt général n'est pas définissable ; qui dit service public ne dit pas forcément application du droit administratif. \\
Pour la deuxième, dont le principal représentant est Hauriau, c'est la prérogative de puissance publique qui détermine la compétence de la juridiction administrative. Cette école a échouée aussi car le juge et le législateur se réfère parfois à ces deux notions, parfois de manières cumulatives, parfois, non. \\
Le droit administratif ne peut donc pas se définir à travers un seul critère. 

\subsection{Les caractères du droit administratif}

\subsubsection{Un droit initialement jurisprudentiel}

C'est le juge administratif qui a dégagé de nombreuses règles du droit administratif car il y a été contraint, aucun droit écrit n'existant. \\
Ce caractère jurisprudentiel a un désavantage: il n'est pas forcément clair. Les décisions sont souvent laconiques, les juges en écrivent le moins possible pour éviter de se lier, pour éviter que toutes les décisions soient des décisions de principes. Cela pose un problème de sécurité juridique, car c'est une source juridique rétroactive. Le CE a trouvé une parade, en 2007: le CE peut moduler dans le temps les revirements de jurisprudence et peut décider d'appliquer la jurisprudence que dans l'avenir, par dérogation (CE Ass, 2007, Tropic travaux signalisation). 

\subsubsection{Un droit de plus en plus écrit}

Depuis 50 ans, les règles écrites sont de plus en plus importantes: la constitution, les traités internationaux, la prolifération de la loi et des règlements. Il existe de plus en plus de textes relatifs au droit administratif. \\
Cependant, cela pose un problème: on a peut être trop de sources écrites. L'inflation législative pose un soucis de droit qui devient inintelligible. Le législateur essaye d'adopter une nouvelle attitude: codifier les textes pour permettre de clarifier et de simplifier le droit. Cette codification a bien sûr toucher le droit administratif: le CGCT (Code général des collectivités territoriales), le CJA (code de justice administrative), le CG3P (Code général de la propriété des personnes publiques), le CRPA (Code des relations entre le public et l'administration), le futur Code de la commande publique. \\
Depuis 2015, la codification se fait souvent à droit inconstant, le législateur modifie donc le droit, notamment jurisprudentiel en codifiant le droit. 

\section{Le juge administratif}

\subsection{La construction du dualisme juridictionnel}

Le dualisme juridictionnel est apparu progressivement. On peut identifier 5 temps dans l'évolution du juge administratif.


Le premier temps est la période révolutionnaire et commence par l'adoption "16 et 24 Août 1790" qui vient poser le principe de séparation des autorités administratives et judiciaires. Le législateur va interdire à tout juge de trancher les litiges de l'administration. Ce texte apparaît à la suite du fait que des instances juridiques (les parlements) bloquaient les ordonnances royales. L'administration elle même tranche donc ses propres litiges. On considérait que "juger l'administration, c'est encore administrer".


Le deuxième temps est l'an VIII, 1799 donc, qui est la période de la création des juridictions administratives. Le Conseil d'État est créé ainsi que les conseils de préfecture. Si l'on crée des juridictions administratives, il est à noter que ces juridictions ne tranchent encore aucun litige. Elles donnaient seulement des avis aux ministres qui jugeaient (ministre juge, on parlait de justice retenu).


Le troisième temps est l'indépendance du juge administratif qui a été consacré par la loi du 24 Mai 1872 qui met fin à la justice retenu et met en place la justice déléguée aux juridictions administratives. Ce texte est important car crée aussi le tribunal des conflits. Il n'a pas non plus supprimé totalement la justice retenu, ce sera donc le CE, 13 Décembre 1889, Arrêt Cadot, qui y mettra fin lui même. 


Le quatrième temps est le XXe siècle où sont apparus des réformes importantes de la juridiction administrative. La première est la création des Tribunaux Administratifs qui remplace les conseils de préfecture ; la deuxième est la création des CAA, en 1997 pour désencombrer le Conseil d'État. Le CE est donc devenu principalement un juge de cassation. Au niveau des réformes procédurales, la première est l'attribution d'un pouvoir d'injonction au juge administratif en 1995 qui lui permet d'imposer à l'administration un comportement déterminé. Un loi du 30 Juin 2000 réforme les procédures d'urgence devant la justice administrative qui crée des procédures rapides et efficaces et remplaçant les anciennes procédures qui étaient difficiles à mettre en oeuvre.


Le cinquième temps sont les réformes contemporaines qui vont dans deux sens distincts.\\ 
Le premier sens est l'idée de garantir le droit au procès équitable devant le juge administratif. C'est ce qui a été fait notamment à propos de l'ancien commissaire au gouvernement, que l'on appelle depuis 2009 un rapporteur public. Celui-ci a un rôle très spécifique, il a pour mission de donner son avis sur le litige en cours, il rend ce qu'on appelle des conclusions (il donnera la question de droit, les solutions possible et celle qu'il préconise en toute indépendance et impartialité). Ce sont les conclusions de ces rapporteurs qui ont aidés à construire la justice administrative. Pendant longtemps, les requérants ne pouvaient lui répondre, et le rapporteur avait droit au délibéré, il posait donc des problème d'apparence de partialité. La CEDH a relevé ce soucis de partialité et a conduit à une réforme du commissaire au gouvernement. \\
Le deuxième sens vise l'efficacité du juge administratif. Pendant longtemps, le juge pouvait mettre des années à rendre des décisions, certaines réformes ont donc été faites et ont permis de réduire le temps des affaires traités. Aujourd'hui, le CE met en moyenne 8 mois pour trancher une affaire. Il y a comme une volonté de bannaliser le juge administratif: le rendre similaire au juge judiciaire. 

\subsection{La consécration constitutionnelle du juge administratif}

Pendant longtemps, il n'y avait aucune mention du juge administratif alors que le juge judiciaire était mentionné (art. 66). La Constitution traitait du CE mais seulement en tant qu'organe consultatif (art. 39). \\
Le CC a rendu 3 décisions pour progressivement consacré constitutionnellement la juridiction administrative. \\
La première décision est la consécration de l'indépendance du juge administratif: CC, 22 Juillet 1981 "loi portant validation d'actes administratifs", il dégage un PFRLR: l'indépendance de la juridiction administrative. Le fondement qui permet de dégager ce PFRLR est la loi de 1872. Le juge ne se réfère pas à l'article 16 de la DDHC et a préféré dégager un principe pour conforter la position de la justice administrative. Curieusement, son indépendance est consacré alors que son existence ne l'est pas encore.


La deuxième décision du 23 Janvier 1987 "Conseil de la concurrence". Une loi avait pour objet de transférer au juge judiciaire une compétence en matière administrative. La question en l'espèce était de savoir si un tel transfert était possible. Le CC dégagera un nouveau PFRLR qui définit constitutionnellement la compétence du juge administratif. Le juge administratif est compétent pour traiter de tous les actes de puissance publiques. En consacrant une compétence au niveau constitutionnel, il consacre l'existence de la juridiction. \\
Ce PFRLR pourrait être fondé dans la loi de 1790. Cependant, ce n'est pas le cas car la loi de 1790 ne permettait pas de fonder ce principe car c'était une loi non républicaine. Le CC va donc se référer "à la conception française de la séparation des pouvoirs". \\
Concernant le champ de la compétence constitutionnelle du juge administratif, il est compétent dès que sont en cause des prérogatives de puissance publique. Cela ne concerne pas toutes les prérogatives car, selon le CC, la compétence concerne seulement "l'annulation ou la réformation d'actes administratifs". Le contentieux contractuel ne relève donc pas du juge administratif pour le CC, cette compétence n'est donc pas constitutionnelle. Des personnes privés peuvent aussi édicter des actes administratifs et ne relèvent pas du juge administratif. \\
En 1987, le CC dit qu'il est possible de déroger à ce principe et de confier au juge judiciaire de traiter d'un contentieux administratif, sous deux conditions, dans l'intérêt d'une bonne administration de la justice ; l'aménagement doit être "précis et limité". \\
Certains contentieux administratifs très importants ont étés confiés à des juges judiciaires comme des affaire de maltraitance à l'école par des profs. La plupart des contentieux des AAI sont aussi traités au judiciaire. 


La troisième décision, du 3 Décembre 2009 "Loi organique relative à la QPC", le CC a interprété les disposition et a dit que "la C.Cas comme le CE sont les juridictions placés au sommet de chacun des deux ordres juridictionnels". \\
Cette décision consacre donc le dualisme juridictionnel.


Au niveau politique, ou même en doctrine, le JA est contesté notamment à cause de la complexité de la répartition des compétences. La question de comment supprimer le JA pourrait donc se poser. Une simple loi est donc impossible pour supprimer le JA car il est consacré constitutionnellement. Une LC est donc nécessaire pour se faire. 


\subsection{Les attributions de la juridiction administrative}

Le JA exerce une fonction juridictionnelle qui consiste à trancher des litiges. Cependant, il exerce aussi une fonction consultative qui lui permet d'émettre des avis: c'est le cas des TA et des CAA qui peuvent donner des avis aux préfets mais aussi le CE qui donne des avis au profit du Gouvernement. Dans certains cas, le CE doit être obligatoirement consulté, notamment dans les PJL, les projets d'ordonnance, ou encore les décrets en conseil d'État. \\
Sur l'hypothèse des décrets en CE, lorsqu'il est obligatoirement consulté, le CE est considéré comme co-auteur. Le gouvernement a donc un choix très limité, soit il adopte le projet initial, soit le projet modifié par le CE. Le Gouvernement ne pourra donc pas adopter d'autres versions de son projet. \\
Cela a des effets contentieux très important lorsque le Gouvernement ne consulte pas le CE lorsqu'il devrait l'être. Dans une décision "SCI Boulevard Marago", le CE crée le vice d'incompétence. \\
Le contrôle en matière consultative est très différent du contrôle en matière contentieuse. En matière contentieuse, il vérifie que l'acte est conforme aux règles de droit supérieur (contrôle de légalité). En matière consultative, il vérifie avec la légalité la question de l'opportunité politique. 


Le principe de dédoublement fonctionnel peut poser un problème de partialité. Par exemple, si le CE donne un avis sur un décret pris en CE (il en est donc le co-auteur) et que quelques mois plus tard, un requérant conteste le dit décret, c'est le CE qui devra régler le litige. La CEDH, 28 Septembre 1995 "Procola contre Luxembourg", dit, à propos de l'usage successif de fonctions consultatives puis contentieuse par la même personne et dans la même affaire viole le principe d'impartialité garanti par l'article 6 de la CEDH. Le principe de dédoublement fonctionnel n'est donc pas remis en cause. \\
On a considéré qu'en pratique, on pouvait faire en sorte que jamais une même personne exerce les deux fonctions dans une même affaire, en effet il y a plus de 300 conseillers d'État. En 2006, la CEDH rappellera la France à l'ordre et qu'il faut un texte qui sera pris en 2008. Un décret du 6 Mars 2008 vient séparé les fonctions consultatives et contentieuse du CE. Il est désormais formellement interdit à un conseiller d'État de travailler à la fois au consultatif et au contentieux dans la même affaire du CE. Ce décret a aussi modifié les formations de jugement: les conseillers qui ont participé à la consultation ne siègent plus au jugement en contentieux. \\
Un décret du 23 Septembre 2011 interdit aux conseillers siégeant au contentieux de prendre un éventuel avis d'un autre conseiller ayant siégé au consultatif. Ce décret n'a plus aucune utilité depuis 2015 car tous les avis sont désormais publics. 



\part{Les sources du droit administratif}


\chapter{Les sources constitutionnelles du droit administratif}

Un arrêt d'assemblée du 30 Octobre 1998, "Sarran", confirmé par la C.Cas en assemblée plénière, 2 Juin 2000 "Mlle Fraisse" affirment que le droit administratif se doit de respecter la Constitution et que c'est au juge de contrôler le droit administratif au regard de la Constitution. 

\section{Bloc de constitutionnalité}

Le bloc de constitutionnalité a été étendue essentiellement par le CC, mais aussi par le juge administratif. Il existe des principes jurisprudentiels qui ont valeur constitutionnel. CC, 17 Mai 2013, "Mariage pour tous", le droit naturel n'est pas reconnu. 

\subsection{Les articles de la Constitution}

Certains articles concernent directement le droit administratif. On y trouve des règles de compétence et de procédure qui sont très importantes. L'article 21 de la Constitution par exemple, donne le premier ministre comme le détenteur du pouvoir réglementaire. Les articles 13 et 21 donnent des pouvoirs de nomination. Les articles 19 et 22 qui concerne le contreseing. L'article 37 délimite le domaine de compétence du pouvoir réglementaire. L'article 53 détermine les règles de ratification des traités internationaux. \\
Des articles peuvent donner des principes de fond, comme l'article 2 qui pose le principe de l'égalité devant la loi, l'article 55 qui pose le principe de supériorité sur la Loi. L'article 72 qui pose le principe de libre administration des collectivités territoriales. 

\subsection{Le préambule de la Constitution}

Le préambule renvoie à d'autres textes: la DDHC, le préambule de 1946, la charte de l'environnement. Le préambule de 1946 évoque aussi les PFRLR mais aussi aux "Principes politiques, économiques et sociaux particulièrement nécessaire à notre temps". \\
Pour le CE, il s'est contenté, dans un premier temps, de reprendre les principes du préambule pour en faire des principes jurisprudentiels. Le CE a aussi dit, le 16 Juillet 1950, en assemblée, "Dehanne", que le préambule de 1946 avait valeur constitutionnelle ainsi que ses normes auxquelles elle renvoie, notamment donc les PFRLR. Le 11 Juillet 1956, "Amicale des annamites de Paris", le CE dira encore que les PFRLR ont valeur constitutionnelles. En 1957, arrêt "Condamine", consacre la DDHC ayant valeur constitutionnelle. Après 1958, le CE, le 12 Février 1960, Arrêt "société Eky", dira que le nouveau préambule est constitutionnel lui aussi. \\
Le 16 Juillet 1971, le CC confirmera tout cela dans sa décision "Liberté d'association". \\
CE, 3 Octobre 2008, "Commune d'Annecy", le CE confirme la constitutionnalité de l'intégralité du préambule de 1958. 


Si le préambule a valeur constitutionnelle, cela ne signifie pas que toutes les normes sont opposables. En effet, certaines dispositions ne sont pas suffisamment précise. \\
Le 29 Novembre 1968, CE, arrêt "Tallagrand", une loi doit venir préciser les normes imprécises des préambules, comme par exemple le droit au logement, le droit à la santé. 

\subsection{La charte de l'environnement}

Elle a été inscrite dans la constitution en 2005. Elle contient deux grandes parties: un préambule qui ne comporte que des dispositions proclamatrices, et qui sont donc symboliques et non pas juridiques. \\
Elle contient des articles précis qui ont valeur juridique. Principe de prévention des atteintes à l'environnement, principe de réparation des atteintes, principe de précaution, droit de participation des administrés aux décisions qui ont un impact sur l'environnement. \\
Le CE dira en 2008 (commune d'Annecy) que toutes les dispositions de la charte ont valeur constitutionnelle. Deux questions se sont posés rapidement: quelle est l'autorité compétente pour mettre en oeuvre cette charte ? \\
En principe, c'est le législateur qui doit mettre en oeuvre ces principes. Le pouvoir réglementaire n'est pas compétent tant qu'une loi n'est pas adopté. Il y a une nuance sur l'article 5 de la charte, qui peut être mise en oeuvre par le pouvoir réglementaire même en l'absence de texte législatif. \\
Deuxième question: les principes sont elles suffisamment précis pour être opposable. Le principe de précaution est suffisamment précis pour être invocable en justice: arrêt du 19 Juillet 2010, CE, "Association quartier Les haut du choiseil". Pour toutes les autres dispositions, elles ne sont pas suffisamment précises selon le juge, et on ne peut pas les invoquer directement. CE, 6 Juin 2006 "Eau et rivière de Bretagne". Le CE a quelque peau nuancé cet arrêt dans 3 décisions. La première évolution résulte de l'arrêt "Commune d'Annecy" où le CE décide que l'article 7 (principe de participation des administrés) est directement opposable. La deuxième évolution, Assemblée 12 juillet 2013 "Fédération nationale de la pêche en France", où le juge décide que l'article 3 est invocable contre un acte administratif même si une loi l'a déjà mise en oeuvre, à la condition que l'acte administratif aille plus loin que la loi. La troisième évolution, dans un arrêt du 26 Février 2014 "Ban asbestos", le CE reprend le même raisonnement qu'en 2013 à propos de l'article premier de la charte. \\
On se demande aujourd'hui si ce raisonnement pourrait être étendue à tous les principes de la charte. 

\subsection{Principes jurisprudentiels}

Ce sont des principes dégagés par des juridictions et qui ont une valeur constitutionnelle. On peut distinguer deux catégories de principes: les principes et objectifs à valeur constitutionnels. Le principe pose une obligation de résultat au législateur comme le principe de continuité du service public. L'objectif ne fixe qu'une obligation de moyens ; le législateur doit mettre tous les moyens en oeuvre pour y parvenir et ne sera pas sanctionné si il n'y parvient pas. On peut citer dans ces objectif le principe d'intelligibilité de la norme. 


Concernant les PFRLR qui sont dégagés principalement par le CC (11 depuis une cinquantaine d'année) peuvent aussi être dégagés par le CE. CE, 3 Juillet 1996, "Konei", interdit l'extradition pour des motifs politiques. \\
Il existe plusieurs critères pour qu'un juge dégage un PFRLR, il faut que le principe soit dans une loi républicaine, antérieur à 1946, constant depuis sa consécration. Depuis 2013, le juge constitutionnel a fait évolué ses critères.


En 2013, s'est posé une question avec la loi sur le mariage pour tous. À cette occasion, il y a eu un grand débat chez les juristes qui se sont demandés si il existait un PFRLR qui interdirait le mariage entre personne du même sexe. Pour certains, il existerait, avec pour origine le Code Civil, qui a été adopté quelques jours avant l'empire. D'autres auteurs avaient une position contraire ; ceux ci se concentre sur l'objet des PFRLR, ceux-ci ayant pour objet les libertés fondamentales, et donc un PFRLR ne pourrait pas interdire quelque chose. \\
D'autres auteurs démontrent que ce débat n'est pas vraiment juridique et que ce sont des gens qui viennent débattre politique à l'aide d'arguments pseudo-juridique. Ils font remarquer une chose importantes: tant que le juge ne l'a pas identifié, il ne sert à rien d'essayer d'identifier un principe. Si le juge souhaite dégager un PFRLR, il trouvera les arguments pour, et si il ne le veut pas, il les trouvera aussi. \\
Cependant, le CC vient dégager un nouveau critère pour identifier un PFRLR. Pour le CC, un PFRLR ne peut concerner que la protection des droits fondamentaux, soit la souveraineté nationale, soit l'organisation des pouvoirs publics.


Il est à noter que le juge n'est jamais lié par ce qu'il dit. Il peut très bien décidé, demain, de changer ses critères. Des PFRLR dégagent des principes qui n'ont pas pour objet les 3 évoqués par le CC. En 2011, par exemple, le CC dégage le PFRLR de l'existence du droit local d'Alsace-Moselle. C'est un principe non constant, car le législateur peut réduire le champ du droit local, il est aussi limité à une partie du territoire. 


\section{Étendu du contrôle de constitutionnalité}

\subsection{L'écran législatif}

L'écran législatif est une limite au contrôle de constitutionnalité. Elle se tient en ce que le juge ne contrôlera pas l'acte administratif au regard de la constitution si le dit acte a été pris en application d'une loi. Si c'était le cas, cela reviendrai à contrôler la loi, ce que ne peut pas faire le juge administratif. \\
CE, 6 Novembre 1936, "Arrighi", le juge affirme qu'il n'est pas compétent pour juger de la constitutionnalité d'une loi. Le commissaire du gouvernement de l'époque disait dans ses conclusions que cela n'était pas possible à cause de l'idée du légicentrisme: le législateur est souverain et le juge ne peut donc pas interférer. \\
CE, 5 Janvier 2005, "Mlle Debrez", le juge affirme que si il ne contrôle pas la constitutionnalité des lois, c'est parce que le juge constitutionnel a la compétence exclusive. 

\subsection{Les limites à l'écran législatif}

\subsubsection{La théorie de l'écran transparent}

CE, 19 Novembre 1986, "Smanor", confirmé par CE, 17 Mai 1991, "Quintin". \\
Le juge peut contrôler l'acte administratif si il va plus loin que la loi. 

\subsubsection{Contrôle de constitutionnalité des lois antérieures à la constitution du 4 Octobre 1958}

Une loi peut devenir inconstitutionnelle suite au changement de constitution ou encore si une loi a été adoptée avant une révision, la révision nouvelle pouvant rendre non constitutionnelle la loi ancienne. \\
Le juge administratif peut donc constater qu'une loi est devenue inconstitutionnelle et va donc constater l'abrogation implicite de la loi. Cette théorie a été élaboré le 12 Février 1960, arrêt "Eky". CE, 16 Décembre 2005, "Syndicat national des huissiers de justice", confirmera ce contrôle. \\
Pour le juge administratif, ce n'est qu'un constat, il ne le décide pas. Quand la loi antérieure est devenu inconstitutionnelle, le juge ne l'appliquera pas et pourra donc contrôler l'acte administratif au regard de la constitution. \\
La question de la pérennité de ce contrôle peut se poser depuis l'avènement de la QPC. Pour l'instant, la réponse est positive et le CE continue d'appliquer ce contrôle. Cependant, il pourrait y avoir des conflits de point de vue, et certains auteurs suggèrent au CE d'abandonner ce contrôle.

\section{La QPC}

La QPC a été introduite lors de la révision de Juillet 2008 et permet un contrôle de constitutionnalité à postériorie par voie d'exception. La procédure est effective depuis 2010.


\subsection{Rappel de procédure de la QPC}

\subsubsection{Les étapes de la procédure}

La question est posée devant le juge ordinaire, par tout requérant. Le juge ordinaire va, dans un premier temps, renvoyer la question au juge suprême de son ordre (CE ou C.Cas) qui lui pourra envoyer la question au CC. La question est donc filtrée, d'abord par le juge ordinaire puis par le juge suprême de l'ordre juridictionnel en question. \\
Il existe des exigences procédurales. L'article 6, §1 de la CEDH s'applique à la procédure de la QPC et cette procédure a été modifiée pour y être conforme. Cependant, la question de la composition du CC peut poser des problèmes quant à l'exigence de l'impartialité. 

\subsubsection{Le caractère prioritaire de la QPC}

Le terme prioritaire semble indiqué que la question est prioritaire sur les autres. Cependant, ce caractère officiellement prioritaire ne l'est peut être plus. \\
La C.Cas s'est demandé, le 16 Avril 2010 "Melki et Abdeli", si la QPC était conforme au droit de l'Union. Elle dira que la QPC permet de maintenir des normes éventuellement contraire aux conventions européennes, or, maintenir temporairement une loi active alors qu'elle est contraire au droit de l'union, c'est violer le principe de primauté. De plus, la procédure de la QPC empêche de poser une question préjudicielle à la CJUE. Or, poser cette question, est une obligation pour les juridictions: art 267 du TFUE. \\
La C.Cas considère donc que la QPC est contraire au droit européen. La C.Cas posera donc une question préjudicielle à la CJUE pour savoir si la QPC est contraire au droit de l'Union. La doctrine estime que cette question a été posée par une volonté de la C.Cas de ne pas mettre en oeuvre la QPC. Cependant, la question n'est pas dénuée de sens, car un arrêt de la CJCE, 9 Mars 1978 "Simmenthal", considère qu'une question préjudicielle doit primer sur la question constitutionnelle. \\
Assez étonnamment, c'est le CC qui va répondre à la question le 12 Mai 2010 en profitant d'un contrôle à priori pour donner une leçon de droit à la C.Cas. Il rédigera deux pages de considérant abondants, et dira que la QPC est compatible avec le droit de l'Union. Le CC dira que tout juge peut prendre des normes provisoires et notamment suspendre toutes normes contraires au droit de l'Union, et le juge peut, à tout moment, poser une question préjudicielle à la CJUE. Suivant cette décision, la QPC n'est plus prioritaire sur une question de conventionnalité. \\
Cette solution du CC a été reprise deux jours plus tard par le CE, 14 Mai 2010 "Rujovic", et a été reprise elle même par la CJUE, 22 Juin 2010 "Melki et Abdeli", qui va reprendre à l'identique le raisonnement du CC: elle est compatible car le juge peut suspendre à tout moment une norme contraire au droit de l'union, et le juge peut poser une question préjudicielle à tout moment. \\
Cependant, il est rare qu'une QPC et qu'une question de conventionnalité se pose en même temps. Cela a été fait une fois par le CE le 31 Mai 2016. À noter que la question préjudicielle est tranchée en minimum 3 mois, si procédure d'urgence, or, la juridiction suprême (CE ou C.Cas), dispose de trois mois pour transmettre ou non la QPC. Le CE a rejeté la QPC, demandant au requérant de reposer la QPC après la question préjudicielle. 

\subsection{Les conditions de fond la QPC}


Elles sont posées par la constitution elle même (art. 61-1 de la constitution). La première condition est que seule les lois peuvent être contrôlées en QPC. \\
La deuxième est la norme à laquelle la loi va être contrôlée. Et c'est au regard des droits et libertés garantis que la loi va être contrôlée. \\
On peut se demander si on peut poser une QPC sur une loi abrogée ou modifiée. Cela est effectivement possible, QPC, 23 Juillet 2010 "Philippe E." car une loi abrogée peut toujours être applicable dans un litige. On appelle cela un retro-contrôle. \\
Concernant les ordonnances de l'article 38, on doit distinguer deux cas de figures. Si elle a été ratifiée, elle pourra faire l'objet d'une QPC. Sinon, si elle n'a pas été ratifiée, elle n'a qu'une valeur réglementaire et ne peut faire l'objet d'une QPC: CE, 11 Mars 2011, "Benzoni". C'est au juge administratif d'en contrôler la constitutionnalité, comme tout acte réglementaire. \\
Les lois d'habilitation, les lois de ratification, les lois de programmation, ne peuvent pas faire l'objet d'une QPC. Pour cette dernière, c'est ce qu'a jugé le CE, 18 Juillet 2011 "Fédération nationale des chasseurs". \\
Concernant les lois organiques, qui, en principe, peuvent faire l'objet d'une QPC. Cependant, les lois organiques sont forcément contrôlé sur le contrôle à priori. Dès lors qu'elles ont déjà été contrôlées, la QPC ne sera pas nouvelle sauf si les circonstances ont changées, et donc si la constitution a changée: CE, 29 Juin 2011, "Président de l'assemblée de la Polynésie française". 

\subsubsection{Les normes de contrôle}

Il s'agit uniquement des droits et libertés garantis par la constitution. On peut tenter de les identifier à la fois de manière positive et négative. \\
De manière positive, on peut trouver ces libertés dans toutes les sources constitutionnelles, mais principalement dans le préambule, donc DDHC, préambule de 1946, dans la jurisprudence. On les trouvera moins dans le texte lui même de la constitution. \\
Une exception: le principe de laïcité, assimilé aux droits et libertés, CC, QPC, 21 Février 2013 "Association pour la promotion et l'extension de la laïcité". \\
Négativement, ce qui ne relève pas de ces droits et libertés, sont les principes constitutionnels qui n'ont pas lien avec ces droits et libertés. L'organisation décentralisée de la République ne peut pas être invoquée dans une QPC par exemple. L'ensemble des règles procédurales prévues par la Constitution ne peut pas être invoquée, et notamment la procédure législative. Le CC ne contrôlera jamais la procédure législative en QPC, alors qu'elle le fait à priori. \\
Un objectif de valeur constitutionnelle peut être invoquée. Tout dépendra de l'objet, de l'objectif. L'objectif de bonne administration de la justice ne garantit aucuns droits ou aucunes libertés en lui même, alors que l'objectif de la parité garantit des droits et libertés. On peut avoir le même raisonnement pour les PFRLR. \\
Le contrôle peut-il être fait au regard de l'article 34 ? Peut-on invoquer l'incompétence négative dans une QPC ? L'incompétence négative est quand le législateur est resté en deçà de sa compétence. Il est possible d'invoquer cette incompétence, à une condition, à la condition que l'incompétence est elle même portée atteinte aux droits et libertés, CC, QPC, 18 Juin 2010, "SNC Kimberley Clark". 

\subsection{Les conditions de recevabilité}

Seules les juridictions suprêmes contrôlent les conditions, le CC se refuse à les contrôler. \\
La question doit être nouvelle, donc il faut que la loi n'ait pas fait l'objet d'un contrôle de constitutionnalité. Il faut que la loi n'ait pas été déclaré conforme dans les motifs ou les dispositifs. Le CC est sensé statuer ultra petita, il statue au delà des moyens: le CC est sensé avoir examiné toute la loi au regard de la constitution. Or, le CC ne le fait pas systématiquement: en contrôle à priori, il examine seulement les moyens invoqués. Si la disposition de la loi mise en cause n'apparaît pas dans les motifs, la QPC sera donc considérée comme nouvelle. \\
Il existe l'hypothèse selon laquelle la question peut être nouvelle même si la loi a déjà été contrôlée, en cas de changement de circonstances. Il y a trois hypothèses de changement de circonstance. Le premier est la modification de la norme contrôlée. Le deuxième est la nouveauté de la norme de contrôle, un changement de la constitution. Le troisième a été admis par la C.Cas, C.Crim, 20 Aout 2014, "Mouvement raelien international", qui considère qu'une décision CEDH qui condamne la France est un changement de circonstance et qui permet de poser la QPC. \\
Troisième condition de recevabilité, la question doit être sérieuse. Vérifier que la question est sérieuse, consiste à faire un pré contrôle de constitutionnalité. Pour le juge administratif, c'est très nouveau depuis 2010. \\
Quatrième condition, la QPC doit être posée dans un écrit distinct et motivée. 

\subsection{Le contrôle du CC dans le cadre de la QPC}

Le CC peut formuler des réserves d'interprétation. Le CC n'exerce qu'un contrôle restreint sur l'activité du législateur, il ne censure que les horreurs grossières du législateur.

\subsubsection{Le contrôle de constitutionnalité de l'interprétation de la loi}

Il est réalisé depuis deux décisions, une du 6 Octobre 2010, et une du 14 Octobre 2010. \\
Le fondement de cette solution est qu'il se réfère à la théorie du droit vivant. C'est une théorie réaliste du droit. Suivant cette théorie, c'est que le sens d'une norme ne se comprend pas à sa seule lecture. Le sens ne résulte pas de la norme mais de son interprétation. Des auteurs estiment que la norme n'existe pas tant qu'elle n'a pas été interprété. \\
Le CC ne contrôlera pas toutes les interprétations législatives, mais seulement celles des juridictions suprêmes, les interprétations constantes, il ne contrôlera que les interprétations fondées sur un acte législatif. \\
Certains auteurs ont eu peur que le CC deviennent la juridiction suprême qui asservirai les juridictions suprêmes. Cette crainte n'est pas fondée car ce sont le CE et la C.Cas qui filtrent, et peuvent donc décider de ne pas transmettre la question. CE, 14 Septembre 2011, "Mr Pierre", à l'occasion d'une transmission d'une QPC, le CE a changée son interprétation afin de la rendre conforme à la constitution. On peut se demander ici si il n'y a pas un soucis d'impartialité. Cela n'en pose pas, d'après le CE, 12 Septembre 2011, "Dion". 

\subsubsection{Les effets du contrôle}

Ils sont précisés dans l'article 62 de la Constitution. Si la loi est contraire à la constitution, elle est abrogée, et donc annulée pour l'avenir et non pas pour le passé. \\
Le CC peut aussi moduler dans le temps les effets de sa décision. Cette modulation permet au législateur de prendre le temps de régler son erreur et de prendre de nouvelles dispositions. \\
Dès lors que la loi est abrogée, la loi ne peut plus être appliquée dans l'instance en cours ni dans toutes les autres en cours. C'est ce qu'a jugé le CE en assemblée, le 13 Mai 2011, "M'Rmeda". À l'inverse, si le juge module dans le temps sa décision, la loi est encore applicable à l'instance en cours. Le CE l'a rappelé dans un arrêt du 14 Novembre 2012 "Association France nature environnement". \\
En QPC, le CC peut lui aussi poser une question préjudicielle à la CJUE. C'est ce qu'il a fait le 4 Avril 2013 "Jeremy Forrest". \\
CC, 24 Juillet 2015 "Déchéance de nationalité", le CC refuse de poser une question préjudicielle. 

\chapter{Les conventions internationales}

Les traités n'ont pas toujours fais parti de la légalité. En 1946, on considérait le respect du traité comme naturel et non pas comme juridique. L'article 26 de la Constitution de 1946 donne "force de loi" aux traités internationaux. Le juge administratif va alors accepter la conformité d'un acte administratif aux traités. CE, 1952 "Kirkwoud". \\
Les choses ont encore changées en 1958. L'article 55 donne une valeur supérieure aux lois aux traités. \\
Si un traité a une valeur supra-législative, elle a toujours une valeur infra-constitutionnelle. Le CE le dit dans son arrêt "Saran". 

\section{Les conditions d'applicabilité des normes internationales}

\section{Comment le juge administratif garantit le respect des normes internationales ?}





















\chapter{Le droit de l'Union Européenne}

\chapter{Les lois et réglements}

\chapter{Les principes généraux du droit administratif (principes jurisprudentiels)}

\part{Les missions de l'administration}


\part{Les recours au contentieux}




























\end{document}
