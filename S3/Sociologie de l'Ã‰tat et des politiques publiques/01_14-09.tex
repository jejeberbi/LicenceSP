\documentclass[10pt, a4paper, openany]{book}

\usepackage[utf8x]{inputenc}
\usepackage[T1]{fontenc}
\usepackage[francais]{babel}
\usepackage{bookman}
\usepackage{fullpage}
\setlength{\parskip}{5px}
\date{}
\title{Cours de Sociologie de l'État et des politiques publiques (UFR Amiens)}
\pagestyle{plain}

% Rattrapage 13h-16h - 20 septembre

\begin{document}
\maketitle
\tableofcontents

\part{Sociologie de l'État}

\chapter{Introduction}

État nation ; État providence ; État producteur. Depuis peu, il y a un recul de l'État alors que dans les années 60, il y a une recherche de développement équitable des territoires (DATAR). \\
Pierre Rosanvallon crée le concept d'État régulateur qui est la forme actuelle de l'État. C'est un État minimisé par rapport à l'État providence ou producteur. Il est désormais un arbitre et non plus un acteur central comme c'était le cas auparavant. \\
L'État est un fait social au sens de Durkheim car l'État remplit les propriétés du fait social (extériorité, antériorité, coercition). De plus, l'État s'est forgé sans que l'on s'en rende compte. Il n'a pas été pensé à priori mais à posteriori. \\
L'État peut être pensé comme une abstraction philosophique mais aussi comme un ensemble d'acteurs producteur d'actions publiques. Il faut éviter l'anthropomorphisme vis à vis de l'État. On va chercher à savoir précisément qui décide, car l'État ne décide pas. \\
Dans ce cours, on va penser l'État comme un composé d'acteurs en interaction, qui peuvent être en conflit ou dans des alliances. L'État n'est pas un ensemble homogène, il subit des divisions. On étudiera l'État comme dans la démarche de Jean-Gustave Padioleau "L'État au concret". 


Le pouvoir est une notion importante. Pour certains, la science politique est la science du pouvoir. Une première conception du pouvoir est essentialiste ou naturalisante qui postule que le pouvoir fait partie de la nature du porteur de pouvoir. On ne peut pas se satisfaire de cette définition. \\
On oppose à la conception essentialiste la conception relationnelle du pouvoir qui considère, à la suite de Robert Dahl, que le pouvoir s'inscrit dans une relation entre au moins deux personnes: un détenteur du pouvoir et celui qui accepte de se soumettre. Le pouvoir est la capacité qu'un individu A fasse faire une action à B qu'il n'aurait pas fait sans A. \\
Le pouvoir politique est un pouvoir qui s'exerce sur un groupe social, un collectif. Ce pouvoir s'exerce dans un but collectif, celui de la cohésion du groupe. Pour perdurer, il est obligé de s'appuyer sur un discours d'accompagnement, des systèmes de valeurs: le pouvoir doit être reconnu par ceux qui s'y soumettent comme étant légitime (la légitimité est une affaire de croyance, on croit en la pertinence ou au bien fondé de quelque chose). Goffman dans "Stigmate" et "Mise en scène de la vie quotidienne", explique que l'État, la société, la démocratie, etc. ne sont pas des institutions figées. Comme la croyance est la source de la légitimité, les institutions se doivent de se renouveler, de changer ses discours. \\
La légitimation ne s'arrête pas aux idéaux-types de Max Weber. La science par exemple est un pourvoyeur de légitimité. La démocratie participative peut aussi être un puissant vecteur de légitimation des politiques publiques.


Il a existé des formes de pouvoirs politiques spécifiques, notamment dans des sociétés non étatiques. En effet, l'État n'a pas toujours existé. Il faut se méfier d'un double préjugé très répandu qui consisterait à croire que l'État est une nécessité naturelle ou croire que c'est une nécessité historique. C'est une forme d'ethnocentrisme qui place l'État comme au dessus des autres. \\
On distingue schématiquement quatre types de pouvoirs qui précèdent l'État, le premier est le pouvoir politique indifférencié. Dans celui-ci, le pouvoir politique se confond avec les autres pouvoirs. \\
Le second est le pouvoir politique personnalisé, incarné dans une personne particulière (chef de clan, chef de tribu), il tient sa position à la faveur des dieux ou de ses ancêtres. Ce type de pouvoir n'est pas propre aux sociétés sans États. \\
Le troisième est le pouvoir politique patrimonial. Il se caractérise par une certaine indifférenciation et une forte personnalisation dans le sens où le pouvoir est le patrimoine des ou du détenteur. C'est Max Weber qui met en exergue ce concept. Le pouvoir est une propriété. Dans ce type de régime, on ne distingue pas le budget public et le budget privé. Les empires, les sultanats sont des pouvoirs patrimoniaux. On distingue deux types de pouvoir politique patrimoniaux: il se manifeste par une forte centralisation dans certains cas et peut être, à contrario, très éclaté. Empire, Sultanat, sont des pouvoirs patrimoniaux centralisés. Le pouvoir patrimonial fragmenté, paradoxalement, est celui qui préexiste directement à la création de l'État, comme dans la société féodale. En occident, les seigneuries se sont multipliées avec l'éclatement des carolingiens. 


Dans l'État, le pouvoir politique est centralisé ; différencié ; spécialisé ; institutionnalisé. 



\chapter{La formation de l'État en occident}




\chapter{Les trajectoires multiples de l'État}




\chapter{Les analyses des États contemporains}










\end{document}
