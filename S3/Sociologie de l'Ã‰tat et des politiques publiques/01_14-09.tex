\documentclass[10pt, a4paper, openany]{book}

\usepackage[utf8x]{inputenc}
\usepackage[T1]{fontenc}
\usepackage[francais]{babel}
\usepackage{bookman}
\usepackage{fullpage}
\setlength{\parskip}{5px}
\date{}
\title{Cours de Sociologie de l'État et des politiques publiques (UFR Amiens)}
\pagestyle{plain}

% Rattrapage 13h-16h - 20 septembre

\begin{document}
\maketitle
\tableofcontents

\part{Sociologie de l'État}

\chapter{Introduction}

État nation ; État providence ; État producteur. Depuis peu, il y a un recul de l'État alors que dans les années 60, il y a une recherche de développement équitable des territoires (DATAR). \\
Pierre Rosanvallon crée le concept d'État régulateur qui est la forme actuelle de l'État. C'est un État minimisé par rapport à l'État providence ou producteur. Il est désormais un arbitre et non plus un acteur central comme c'était le cas auparavant. \\
L'État est un fait social au sens de Durkheim car l'État remplit les propriétés du fait social (extériorité, antériorité, coercition). De plus, l'État s'est forgé sans que l'on s'en rende compte. Il n'a pas été pensé à priori mais à posteriori. \\
L'État peut être pensé comme une abstraction philosophique mais aussi comme un ensemble d'acteurs producteur d'actions publiques. Il faut éviter l'anthropomorphisme vis à vis de l'État. On va chercher à savoir précisément qui décide, car l'État ne décide pas. \\
Dans ce cours, on va penser l'État comme un composé d'acteurs en interaction, qui peuvent être en conflit ou dans des alliances. L'État n'est pas un ensemble homogène, il subit des divisions. On étudiera l'État comme dans la démarche de Jean-Gustave Padioleau "L'État au concret". 


Le pouvoir est une notion importante. Pour certains, la science politique est la science du pouvoir. Une première conception du pouvoir est essentialiste ou naturalisante qui postule que le pouvoir fait partie de la nature du porteur de pouvoir. On ne peut pas se satisfaire de cette définition. \\
On oppose à la conception essentialiste la conception relationnelle du pouvoir qui considère, à la suite de Robert Dahl, que le pouvoir s'inscrit dans une relation entre au moins deux personnes: un détenteur du pouvoir et celui qui accepte de se soumettre. Le pouvoir est la capacité qu'un individu A fasse faire une action à B qu'il n'aurait pas fait sans A. \\
Le pouvoir politique est un pouvoir qui s'exerce sur un groupe social, un collectif. Ce pouvoir s'exerce dans un but collectif, celui de la cohésion du groupe. Pour perdurer, il est obligé de s'appuyer sur un discours d'accompagnement, des systèmes de valeurs: le pouvoir doit être reconnu par ceux qui s'y soumettent comme étant légitime (la légitimité est une affaire de croyance, on croit en la pertinence ou au bien fondé de quelque chose). Goffman dans "Stigmate" et "Mise en scène de la vie quotidienne", explique que l'État, la société, la démocratie, etc. ne sont pas des institutions figées. Comme la croyance est la source de la légitimité, les institutions se doivent de se renouveler, de changer ses discours. \\
La légitimation ne s'arrête pas aux idéaux-types de Max Weber. La science par exemple est un pourvoyeur de légitimité. La démocratie participative peut aussi être un puissant vecteur de légitimation des politiques publiques.


Il a existé des formes de pouvoirs politiques spécifiques, notamment dans des sociétés non étatiques. En effet, l'État n'a pas toujours existé. Il faut se méfier d'un double préjugé très répandu qui consisterait à croire que l'État est une nécessité naturelle ou croire que c'est une nécessité historique. C'est une forme d'ethnocentrisme qui place l'État comme au dessus des autres. \\
On distingue schématiquement quatre types de pouvoirs qui précèdent l'État, le premier est le pouvoir politique indifférencié. Dans celui-ci, le pouvoir politique se confond avec les autres pouvoirs. \\
Le second est le pouvoir politique personnalisé, incarné dans une personne particulière (chef de clan, chef de tribu), il tient sa position à la faveur des dieux ou de ses ancêtres. Ce type de pouvoir n'est pas propre aux sociétés sans États. \\
Le troisième est le pouvoir politique patrimonial. Il se caractérise par une certaine indifférenciation et une forte personnalisation dans le sens où le pouvoir est le patrimoine des ou du détenteur. C'est Max Weber qui met en exergue ce concept. Le pouvoir est une propriété. Dans ce type de régime, on ne distingue pas le budget public et le budget privé. Les empires, les sultanats sont des pouvoirs patrimoniaux. On distingue deux types de pouvoir politique patrimoniaux: il se manifeste par une forte centralisation dans certains cas et peut être, à contrario, très éclaté. Empire, Sultanat, sont des pouvoirs patrimoniaux centralisés. Le pouvoir patrimonial fragmenté, paradoxalement, est celui qui préexiste directement à la création de l'État, comme dans la société féodale. En occident, les seigneuries se sont multipliées avec l'éclatement des carolingiens. 


Dans l'État, le pouvoir politique est centralisé ; différencié ; spécialisé ; institutionnalisé. 


\chapter{La formation de l'État en occident}

Ce que nous appelons État correspond à une forme d'organisation politique qui s'est imposé en occident. Pour saisir le processus historique de formation de ces États, on peut essayer de constater les caractéristiques les plus repérables. \\
Bernard Lacroix dans "Le traité de Science Politique", 1985, volume 1, dit "le mot État est apparu et a été généralisé en Europe à la fin du XVe siècle". Cette apparition tardive montre que l'État a été créé récemment. \\
L'État exerce un pouvoir sur un territoire élargi. \\
L'État se distingue des détenteurs du pouvoir. \\
Il y a une nette séparation entre les règles de fonctionnement de l'État et de la société civile. \\
L'État suppose une laïcité minimale. \\
Les fonctions de Gouvernement sont remplis par des organes spécialisés, qui se distinguent des autres pouvoirs (notamment économiques et religieux). \\
L'État s'appuie sur une administration institutionnalisée. L'État est donc légal rationnel. 

\section{La centralisation du pouvoir politique et la crise de la féodalité}

C'est la crise de la féodalité qui va permettre un début de formation de l'État. \\
La crise féodale a deux dimensions, une dimension socio-économique et une dimension politique.


La dimension socio-économique renvoie à l'expansion économique en occident à partir de la fin du XIe siècle, qui va s'étendre jusqu'au début du XIIe. Cette expansion va avoir deux conséquences importantes: l'essor des villes et avec ceci, le développement d'une classe sociale: la bourgeoisie urbaine. Cet essor s'effectue en dehors du système féodal car les villes possèdent une relative autonomie. \\
Le développement du commerce entraînent donc la création de la bourgeoisie urbaine qui n'est pas contrôlée par la noblesse, ce qui en fait une entorse au système féodal. \\
Une partie de la paysannerie va s'intégrer aux circuits commerciaux en relation avec les villes et échappent donc en partie au contrôle de la noblesse. Cela provoque aussi un premier exode rural. \\
L'agriculture devient une matière commerciale. On assiste à une monétarisation de l'économie et donc à une perte de contrôle économique et sociale des paysans par les seigneurs. 


La dimension politique est le fait que le pouvoir féodal est affaibli. Cet affaiblissement est lié aux concurrences de plus en plus fortes entre les seigneurs dû à la croissance démographique. \\
Le morcellement de l'autorité en seigneurie entraîne des concurrences et donc des conflits entre seigneurs pour le contrôle d'un territoire. \\
Ce processus a été analysé par Norbert Elias dans "La dynamique de l'occident". Elias souligne le rôle prédominant et décisif joué par le Roi qui a cherché à étendre son autorité par la voie de la conquête militaire. Cette stratégie n'est pas propre au Roi. \\
Tout au long du XIIe siècle, des luttes incessantes entre seigneur vont se produire et déboucher sur une concentration progressive du pouvoir aux mains notamment des seigneurs les plus puissants. \\
Quatre phases conduisant au pouvoir politique centralisé: 
\begin{itemize}
\item Phase de concurrence libres et unification du domaine dynastique (XIIe siècle).
\item Lutte des maisons princières (XIIIe et XIVe siècle).
\item Phase des apanages (XIVe et XVe siècle): les affrontements vont être interne aux grandes familles. Ce phénomène va réduire la dynamique monopolistique, le Roi devant morceller son pouvoir. 
\item Phase de victoire du monopole Royal (XVe et XVIIe siècle). 
\end{itemize}


Ce que Norbert Elias appelle "Les chances de puissance sociale" qu'on définira comme les possibilités de domination déterminées par le nombre de dépendants que le monopoliste peut s'attacher, sont dans les mains du Roi. \\
Le Roi détient le monopole de la fiscalité et de la contrainte physique. L'État a un degré élevé de monopolisation. Ces deux monopoles entraînent une centralisation. \\
Ces deux monopoles vont générer la formation d'autres monopoles fondamentaux et à la base de la création de l'État. Grâce aux monopoles fiscale et militaire, vont se constituer d'autres monopoles et notamment le monopole juridique et également le monopole d'exercice de la justice. Pour ce dernier, le Roi va tout au long du XVIIe siècle, lutter contre les tribunaux ecclésiastiques. \\
Le monopole de la production juridique va disparaître de l'Église et va apparaître chez des légistes royaux. \\
Cette dépossession des pouvoirs de l'Église vers l'État est ce qui permet de dire que le processus est un minimum laïque. 


À partir du moment où le monopole s'agrandit, il va s'exiger la mise en place d'une administration aux fonctions multiple et une division poussée du travail. \\
La complexification rend la marge de manoeuvre plus faible. À partir d'un certain seuil, le monopole échappe à la main d'un seul individu. Le Roi a une dépendance fonctionnelle vis à vis de sa cour et de son administration. On dira que le monopole se socialise: il s'ouvre à la gestion et au contrôle de couches entières de la population: administration, gestionnaire. À partir du moment où le monopole se socialise, le monopole privé devient public. 

\section{Les facteurs explicatifs de la différenciation du pouvoir politique}

Ce sont surtout des facteurs culturels. Il y a le rôle de la religion ainsi que des doctrines philosophiques et juridiques.


L'État se construit sur la base d'une dissociation et d'une automatisation du politique par rapport au social. Le politique se désencastre du social. \\
Le christianisme a joué un rôle majeur voire accélérateur de différenciation du pouvoir politique. Cette différenciation est inscrite dans la théologie qui prône une séparation entre la spiritualité et le temporel. \\
Cette différenciation se comprend très bien dans la formule "Rendre à César ce qui est à César, rendre à Dieu ce qui est à Dieu". \\
En proclamant l'autonomie du pouvoir spirituel par rapport au pouvoir temporel, l'Église a dessiné en négatif les contours d'un domaine politique spécifique. \\
Comme le montre Max Weber, l'organisation interne de l'Église a contribué a posé le cadre du futur modèle étatique et lui a servi de modèle d'organisation. L'Église va influencer les fondements de la légitimité de l'État. On sait que l'Église va très loin dans une théorie de la souveraineté qui avait pour fonction d'asseoir la légitimité et l'autorité du pape. Cette théorie de la souveraineté va, en quelque sorte, être laïcisé et va servir à l'élaboration d'une théorie de la souveraineté du monarque, et donc, de l'État. C'est là qu'interviennent les légistes royaux qui vont favoriser la différenciation.  


La redécouverte du droit romain entre le XIIe et le XIIIe siècle participent à la différenciation. Ce droit est fondé sur un système juridique qui a la capacité à penser un système politique autonome et détenteur de sa propre légitimité. \\
Il y a une distinction entres institutions privés et institutions publiques. Le domaine politique est entendu dans le cadre du droit romain comme fonctionnant selon les exigences de l'intérêt public. L'État y est pensé comme producteur privilégié de droit, ce n'est plus la société qui impose ses normes et ses valeurs au pouvoir politique, mais au contraire, c'est l'État qui fixe les normes. Ce qui signifie le droit, pour le Roi, de ne pas être soumis à un pouvoir extérieur. \\


La doctrine du double corps du Roi est l'idée que la souveraineté d'un État n'est pas exercée par le Roi en tant que personne physique mais par le Roi en tant que fonction. C'est la fonction qui confère la souveraineté. C'est l'idée que développe Jean Bodin au XVIe siècle. Selon lui, la souveraineté est absolue, n'appartient à personne, mais la souveraineté appartient à l'État ; les droits régaliens seraient donc la propriété de l'État et donc non pas du Roi. \\
Il y là, dépatrimonialisation du pouvoir politique. \\
Le Roi a donc un corps physique et un corps symbolique détenteur de la souveraineté, d'où l'idée de double corps du Roi. \\
Dans cette idée, la souveraineté est perpétuelle. Quelque soit le monarque, la souveraineté existe toujours, d'où la formule "Le Roi est mort, vive la Roi!". 


"La supériorité des lois fondamentales du royaume" signifie que le pouvoir souverain ne peut être arbitraire, il ne peut pas utiliser le droit ou qu'il ne peut pas utiliser le pouvoir politique à des fins personnels. C'est l'idée de l'existence de lois supérieures immuables et inviolables, et autant que ses sujets, le Roi doit s'y soumettre, au risque de perdre sa légitimité. Le droit qu'il édicte doit se conformer à ce droit. \\
Ceci signe le début d'un État de droit ainsi que de la hiérarchie des normes. \\
C'est par cette mutation progressive en État de droit que la différenciation s'opère puisque le pouvoir politique détient ses propres règles de fonctionnement, respectant les lois fondamentales. 

\section{L'institutionnalisation et la spécialisation}

Ce long processus a nécessité la mise en place d'une administration. Weber a distingué cinq traits, dans "Économie et société", caractéristiques de l'appareil d'administration étatique. \\
La spécialisation des fonctions est le premier trait. Les tâches, au sein de l'administration, sont clairement subdivisés et correspondent à des compétences particulières, à des droits et des obligations spécifiques. L'administration publique étatique repose sur une division du travail. \\
La hiérarchisation des fonctions est le deuxième trait. Les différentes tâches sont appliqués dans le cadre d'une division hiérarchique du travail. Chaque poste est sous le contrôle et la subordination hiérarchique d'un poste qui lui est supérieur. Ce contrôle s'exerce selon des règles juridiques, des règlements abstraits et subjectifs ("modèle bureaucratique"), ces règles évitent l'arbitraire. \\ 
La réglementation des fonctions est le troisième trait. Des règles juridiques définissent les conduites à tenir dans le règlement des tâches administratives. \\
La séparation de la fonction d'administration et des moyens d'administration est le quatrième trait. Les postes n'appartiennent pas au fonctionnaire. Le poste est exercé, il n'est pas possédé (vénalité des offices). \\
La professionnalisation est le cinquième trait. L'exercice des fonctions spécialisées nécessite une formation, une compétence, un recrutement des fonctionnaires qui se fait donc sur la compétence et non sur l'hérédité. Les recrutements vont donc se basés sur des concours. C'est l'aptitude qui est décisive dans l'obtention des postes. \\
Weber considère tout cela à une rationalisation du pouvoir politique qui devient légal-rationnel. 


Une telle administration commence à se mettre en place à la fin du XIVe siècle, sous le Roi Philippe le Bel. Tout au long du XVe et du XVIe est créée un impôt unique, la taille. \\
Au cours du XVIe et du XVIIe s'opère une spécialisation et une institutionnalisation de l'État en France. Elle se repère à trois niveaux, d'abord au niveau central (spécialisation dans le conseil du roi), la mise en place de secrétaires d'État. Un gouvernement spécialisé se met en place auprès du Roi. \\
Au niveau local, des commissaires royaux sont mis en place en 1435. Au début du XVIIe siècle est créé l'intendant, c'est l'ancêtre du préfet. Il détient des pouvoirs important au niveau local. \\
Comme l'a étudié Thocqueville, la révolution française ne marque pas une rupture avec ce processus, mais au contraire, favorise la centralisation. En 1790, sont créés les départements. La centralisation administrative est perpétué par Napoléon par la création des préfets nommés par l'État et donc représentant de l'État. Le préfet contrôle le département. \\
Ce n'est qu'en 1982 que la décentralisation va apparaître, avec les lois Deferre. Ces lois vont affaiblir les corps préfectoraux au profit des collectivités territoriales. \\
L'administration se spécialise de façon croissante au début du XVIIIe siècle, et on voit apparaître les postes de commis, recrutés par concours, également des ingénieurs chargés de fonction technique, qui obtiennent des diplômes d'écoles qui se créent dans la lignée du processus (Ponts et chaussé, École des mines, Polytechnique). Ces ingénieurs se forment dans des grands corps techniques. \\
La compétence, le savoir, deviennent des sources de légitimité. Les grands corps techniques existent encore et sont une spécificité française. Ces grands corps deviennent des hauts fonctionnaires.


Le "pantouflage" renvoie à une pratique dans la haute fonction publique. La pantoufle est, historiquement, la somme que devait verser un grand commis de l'État lorsqu'il quittait le service de l'État. Aujourd'hui, cela désigne des hauts fonctionnaires qui font leur carrière en dehors de leur corps d'appartenance. Les hauts fonctionnaires vont dans le privé ou rentrent en politique: Pierre Birnbaum, "Les sommets de l'État". \\
Un membre de haut corps reste à vie titulaire de son poste et peut retourner à n'importe quel moment à son poste. Cela pose un problème de démocratie car l'administration est censé être au service du politique. Mais aussi, cela leur offre une grande sécurité politique, ils peuvent prendre bien plus de risques que les citoyens lambda. De plus, leur ancienneté continue de courir. 


Il existe un certains nombre de formes de carrières politiques. La majorité, historiquement, était la carrière ascendante, ou carrière par le bas. Cette forme est basée sur le militant, et donc aussi les partis. L'individu commence par être un sympathisant, puis devient adhérents, puis militant, et gravit petit à petit les échelons du parti. Cet individu va finir par se présenter à des élections locales pour aller vers les élections nationales. Il pourra finir nommé au Gouvernement. \\
Il existe une carrière par le milieu qui est lié au parti politique. Ces partis ont besoin de permanents et ceux-ci vont profiter de leur place pour briguer des mandats électifs. \\
La carrière dominante aujourd'hui est la carrière descendante (par le haut), où les individus qui sortent d'une grande école se font nominer dans un cabinet ministériel comme conseiller. L'individu finit nommé secrétaire d'État puis ministre etc. Il manque à ces individus des suffrages, donc ceux-ci vont investir le parti qui les a nommés. Ceux-ci vont ensuite briguer des mandats électoraux et vont se faire "parachuter" afin de se greffer une légitimité électorale et ainsi continuer leurs carrières politiques. 


Le pouvoir exécutif possède l'administration. Ce sont juridiquement deux entités distinctes, cependant, il existe une entité entre les deux, ce sont les cabinets. Ceux-ci sont très opaques, certains sont nommés avec publicité au JORF, d'autres, non. Les nominations se font selon une récompense des siens, des fidèles, la compétence, l'interconnaissance. 


\chapter{Les trajectoires multiples de l'État}

Il serait faux de croire que dans le cadre de l'Europe, les trajectoires soient les mêmes.


\section{L'État en Europe}

"Comparative research on cultures and nations", 1968, est une oeuvre de Stein Rokkan, qui utilise une méthode diachronique (comparaison dans le temps). Il constate des différences dans le mode de développement des États en Europe, notamment en terme de centralisation, de différenciation, d'institutionnalisation, spécialisation. Ils remontent dans le temps pour construire les variables succeptibles d'expliquer ces différences. \\
Ces variables se situent dans les temps modernes. Il distingue principalement trois variables, la variable économique, la variable territoriale, la variable culturelle. 


Pour la variable économique, elle explique les différences en terme de différenciation. Il prend en compte des facteurs qui lui permet d'opposer l'Ouest à l'Est de l'Europe. À l'ouest de l'Europe, on constate une construction étatique précoce, une économie marchande sur la base du commerce atlantique. C'est le cas en France, en Angleterre et en Espagne en particulier. Ce dynamisme économique procure les ressources financières nécessaires au développement de l'État. En effet, l'accroissement des richesses favorisent l'accroissement des ressources militaires et des ressources administratives. Tout cela est soutenu par la bourgeoisie urbaine qui a intérêt à avoir un État fort qui peut garantir la sécurité et l'ordre dans le cadre des circuits commerciaux. Cela n'est pas l'intérêt de l'aristocratie foncière. \\
La bourgeoisie urbaine constitue la classe montante, qui va conduire, par l'achat de charges administratives notamment, à l'anoblissement de certains de ces nobles. \\
À l'est, l'économie demeure très agraire, c'est le cas de la Russie, de la Pologne, de la Prusse. Rokkan parle en ce sens d'ordre impérial agrarien. Le poids des seigneurs féodaux demeurent très importants, ce qui freine la différenciation de l'État. De plus, le pouvoir politique est accaparé par l'aristocratie foncière et conserve donc des traits patrimoniaux. C'est ce qui explique des rythmes de constructions étatiques différents à l'est et à l'ouest. \\
Il distingue aussi le centre de l'Europe, qu'il appelle l'épine dorsale, qui va du monde germanique au monde Italien. Dans ce centre de l'Europe, le modèle de construction étatique est tardif car il a été freiné par la résistance d'un système de cité-État. 


Pour la variable territoriale, elle permet de comprendre les différences en terme de centralisation entre les différents pays européens. Le centre de l'Europe, qui va exactement des Pays-Bas à l'Italie du Nord, en passant par la vallée du Rhin, est au centre des échanges économiques depuis le Moyen-âge. Les routes commerciales sont nombreuses, et les échanges économiques se développent dans de nombreux centres urbains. La concurrence entre ces villes, empêche la formation d'un centre politique. En effet, le nombre de villes pouvant prétendre au poste de capitale va freiner la constitution d'un centre politique. C'est pourquoi la centralisation y est plus tardive. XIXe siècle pour l'Allemagne et l'Italie. \\
Si on prend le cas de Londre et Paris, on voit qu'ils acquièrent vite le statut de centre s'imposant sur un vaste territoire, ce qui accélère la formation d'un centre politique. 


Pour la variable culturelle, principalement la variable religieuse, joue surtout sur un axe Nord-Sud. Le protestantisme se développe dans le nord de l'Europe alors que le catholicisme se développe plutôt au Sud. \\
La religion joue un rôle dans la formation des identités nationales mais aussi dans la différenciation du spirituel et du temporel. \\
En Angleterre, l'Église Anglicane, est le fruit d'un conflit entre la hiérarchie catholique, le Roi est déclaré chef de l'Église d'Angleterre. Progressivement, les ordres réguliers sont supprimés, le mariage des prêtres est autorisé et le culte se dirige progressivement vers du Calvinisme, puis du Luthérien. \\
Le rôle de la papauté sera différent selon les situations. Le pape va freiner la construction de l'État en Allemagne, dans sa lutte contre l'Empereur et le développera en France dans le cadre de sa lutte contre les protestants.


La France et l'Angleterre sont deux États qui se distinguent par leur degré de centralisation et de différenciation. À la suite de deux auteurs, qui ont écrit "Sociologie de l'État", ils considèrent la France comme étant un État fort, et l'Angleterre comme un État faible, au sens de peu de différenciation, de spécialisation, de centralisation et d'institutionnalisation. Ces deux pays sont ceux qui prennent le plus tôt la forme d'État, mais ceux ci se distinguent des traits caractéristiques de l'État.


Une différence dans le degré de centralisation s'observe dans des différences du système féodal. C'est en France que le système féodal a été le plus poussé, mais paradoxalement, c'est en France que la centralisation est le plus poussé. La principale explication réside dans le fait qu'en France, la lutte entre le Roi et les seigneurs a été beaucoup plus forte. Cela a requis une forte monopolisation du pouvoir par la Monarchie et a donnée lieu en France à une répression pour donner le pouvoir au domaine royal. En Grande-Bretagne, cette lutte a été beaucoup moins intense, pour au moins deux raisons, un moindre morcellement du territoire et donc du pouvoir politique car l'unité territoriale est réalisé entre le IXe et le Xe siècle car c'est une île. ; la seconde raison est la préservation par le Roi des pouvoirs seigneuriaux et des coutumes. Le sujets du Roi acceptent donc de se soumettre au Roi et ce, dès le XVIe siècle, en dehors de tout lien vassalique. À noter que de manière très tôt, la noblesse peut se faire entendre du Roi via une représentation parlementaire: magna carta, 1215. Cette charte énonce l'indépendance de la justice, la propriété foncière individuelle. Dans le cas britannique, l'aristocratie cherche plus à contrôler le Roi qu'à détruire les institutions royales.


En Grande Bretagne, par rapport à la France, on constate une moindre différenciation de l'État par rapport à la société. Une des explications principales vient du droit, notamment public. En Grande-Bretagne, le corps politique ne peut se différencier car le droit administratif n'existe pas contrairement à la France où s'élabore un droit public, distinct du droit privé, un droit propre à l'État. Ceci s'explique car la Grande-Bretagne a été beaucoup moins influencée par Rome mais reste aussi attachée à la common-law. Ce droit n'accorde aucune prérogative de puissance publique à l'État, celui-ci est donc soumis au droit commun. \\
On constate aussi qu'en France, surtout depuis la Révolution française, l'État se donne un nouveau rôle, une nouvelle mission, qui est de produire du social. Société à solidarité organique: les individus n'ont plus de liens direct qui les unissent, c'est l'État qui réalise la liaison. \\
En Grande-Bretagne, il n'y a pas de construction d'identité nationale. Il y a plusieurs raisons à cela. La première est qu'on a très tôt une forte unité territoriale, ensuite, une très forte unité linguistique, puis le rôle de l'Église Anglicane qui construit un sentiment et une appartenance commune.


Concernant la spécialisation et l'institutionnalisation, la distinction s'observe tant au niveau du monopole administratif, du monopole militaire et de l'interventionnisme économique. \\
Du point de vue du monopole administratif, alors que se développe en France comme en Espagne la monarchie absolue, il y a, en Angleterre, très peu d'agents professionnels. Ce sont des amateurs, des fidèles, qui entourent le Roi et le conseille et non des fonctionnaires rétribués. Il n'y a pas de fonction publique en Grande-Bretagne avant le XIXe siècle. Les gouvernements britanniques font appels à des commissions ad hoc, spécialisés mais temporaire, pour traiter de manière ponctuelle des problèmes et partir une fois le problème réglé. Cela évite la constitution d'une élite administrative. On trouve plutôt en Grande-Bretagne ce qu'on appelle "l'establishment" composé à la fois de l'aristocratie, de la bourgeoisie et des classes moyennes. L'aristocratie n'a pas de privilèges fiscaux contrairement à la France. \\
Concernant le monopole militaire, le pouvoir central, en Grande-Bretagne, n'a pas de monopole militaire car il n'y a plus de guerre à mener. Très tôt, les rapports sociaux intérieurs sont pacifiés. La police n'est jamais devenu centralisée et est resté très peu professionnalisée. \\
Enfin, au niveau de l'interventionnisme économique, il est resté très faible jusqu'à aujourd'hui: on y trouve bien moins d'entreprises nationalisées. L'individualisme, le libre jeu du marché, est très promu en Grande-Bretagne, et a réduit donc son interventionnisme. Il y a un développement très précoce du capitalisme en Grande-Bretagne. C'est donc le marché qui est premier et non l'État, alors, qu'en France, l'État organise le marché, historiquement. 

\section{L'État post-colonial: l'État néo-patrimonial}

Schmuel Eisenstadt parle du concept de "néo-patrimonialisme". \\
Nous allons nous intéresser aux effets de la décolonisation en Afrique noire et dans le monde musulman. Dans les deux cas, le processus de colonisation vient se produire pendant la seconde moitié du XIXe siècle et la première moitié du XXe. Elle est le fait des États Européens, constitué de longue date, parmi lesquels la France ou encore l'Angleterre au Proche-Orient et en Afrique Noire. Les États colonisateurs sont fortement constitués au moment où ils colonisent d'autres aires géographiques. \\
Ces États vont exporter le modèle occidental de l'État dans les aires géographiques qu'ils colonisent. Aux indépendances, les nouveaux États qui vont se constituer vont poursuivre ce phénomène de reproduction du modèle de l'ancien colonisateur. Cependant, et notamment du fait des structures sociales et culturelles pré-existantes, cela aboutit à une forme d'hybridation du modèle occidental, c'est cette hybridation qu'on appelle l'État "néo-patrimonial". 


\subsection{L'effet de diffusion et reproduction du modèle occidental}

La colonisation européenne a principalement deux effets sur les formes d'organisation du pouvoir politique: le premier est ce qu'on appelle la territorialisation du pouvoir politique (le fait d'associer le pouvoir politique à un territoire). Une telle association favorise un phénomène de centralisation du pouvoir politique. \\
Le tracé de ces frontières a été faite de manière très arbitraire, "à la règle", au congrès de Berlin en 1875. Cet arbitraire a été tel, cette totale négation des ethnies, qu'il a morcelé des ethnies co-existante et a regroupé des ethnies concurrentes. Avant la colonisation, dans ces aires géographiques, la territorialisation n'était pas achevée. \\
Le pouvoir se fondait sur les ethnies et les communautés. Il se définit par rapport à la société, à sa structuration sociale et aux ethnies. 


Au lendemain des indépendances, en 1963, la conférence Addis-Abbedha va créer l'OUA (l'Organisation de l'Unité Africaine), qui va prononcer l'intangibilité des frontières issus de la colonisation. Cela a été fait dans le but d'éviter le bain de sang, ce qui a repoussé le problème. \\
La colonisation se traduit par l'imposition des structures administratives sur le modèle de l'État occidental. Cependant, l'imposition de ces structures administratives du centre n'a pas totalement supprimée les structures pré-existantes. Les colonisateurs ont par exemple installé leurs propres fonctionnaires afin de faire appliquer le droit positif européen. \\
Soucieuse d'asseoir leur autorité, les métropoles ont fait appel aux chefs locaux ayant un ascendant sur les populations pour les utiliser comme courroie de transmission entre la métropole et les territoires colonisés. Ils ont aussi été utilisés pour réduire les éventuels poches de résistance en utilisant les ressorts de légitimité admis et partagé par les populations locales. Ces chefs locaux assurent l'emprise du colonisateur en même temps qu'ils illustrent la persistance de principes traditionnels d'organisation de la société locale. \\
Il y a deux niveaux administratifs imbriqués, juxtaposés. À l'indépendance, les nouveaux États se trouvent devant des structures administratives hétérogènes reposant sur deux héritages bien différents. De manière surprenante, la décolonisation ne va pas interrompre ce mouvement de diffusion du modèle occidental. Il y a un puissant vecteur de mimétisme, l'imitation du modèle: drapeaux, hymnes, constitutions. Le mimétisme n'est pas seulement engendré par la colonisation, il est aussi renforcé par les coopérations qui font que les individus africains viennent faire leurs études en occident et vont donc transmettre un système de valeur qu'ils  ont acquis dans un pays anciennement colonisateur. La coopération international ne reconnaît que des États souverains qui ont des frontières claires et délimités: la coopération internationale est un puissant facteur de reproduction de l'État. La forme de l'État permet la reconnaissance de l'ONU, les prêts internationaux, etc. 


L'auteur part de la situation de l'État patrimonial. L'État néo-patrimonial est donc dans la continuité de celui-ci. La principale différence entre l'État occidental et l'État néo-patrimonial, se situe de la différenciation. Elle est très faible dans le cadre de l'État néo-patrimonial. \\
La première caractéristique de l'État patrimonial est l'appropriation privée du pouvoir politique. En effet, aux indépendances, se mettent souvent en place, des régimes de parti unique qui va permettre la constitution d'une élite qui se fait sur une base soit professionnel (comme les militaires), une élite technique (dictature technocratique), sur une base religieuse, ethnique. Cette élite restreinte s'accapare le pouvoir politique et cherche à limiter l'accès des autres groupes aux pouvoirs politiques et notamment aux ressources qu'impliquent la détention du pouvoir, essentiellement économiques, sous la forme de matières premières. Le détenteur du pouvoir est aussi le détenteur de l'économie. \\
L'appropriation des richesses se fait sur le plan internet et externe. Sur le plan interne, tout le processus de développement économique du pays est contrôlé par l'État et sont accaparés par les détenteurs du pouvoir politique. Sur le plan externe, l'accès des pays occidentaux aux ressources naturelles des États en développement se réalise en échange de versement financier aux détenteurs du pouvoir politique. On constate aussi un détournement important des aides internationales à des fins personnelles. \\
Cette confusion montre la faible différenciation. L'appropriation du pouvoir politique est l'appropriation du pouvoir économique: non différenciation du privé/public.


Il y a aussi une faible différenciation de l'appareil administratif. La bureaucratisation atteint des niveaux records dans les États issus de la colonisation. Le degré d'institutionnalisation et spécialisation est élevé mais l'administration est très peu différenciée. L'administration s'est très largement développée mais ne s'est pas autonomisée du pouvoir politique, au contraire, l'administration est un moyen pour l'élite au pouvoir de faire perdurer son pouvoir. Le pouvoir politique cherche à éviter l'élaboration d'une nouvelle élite. Le gouvernement pléthorique de l'administration permet d'absorber les jeunes diplômés dans les fonctions de l'État. \\
La faiblesse de la différenciation, on peut l'observer aussi du point de vue des rapports entre l'administration et la société. Le nombre élevé de fonctionnaires a pour effet la faiblesse de leurs revenus. Cela génère un effet de corruption et de clientélisme important entre la fonction publique et la société privé. 


Il y a des modes de légitimation spécifiques. Il y a une très forte présence de la légitimation traditionnelle, notamment via la religion, l'appartenance à un dynastie. Fondée sur le charisme avec les héros des révolutions nationales. Il y a une forte personnalisation du pouvoir politique avec des cultes de la personnalité. 


La catégorie de l'État néo-patrimonial est très général, trop large. 


\chapter{Les analyses des États contemporains}

Il y a eu trois formes de l'État dans son long processus de création. Il y a l'État providence, l'État producteur, l'État régulateur. 

\section{L'État providence et les développements de l'interventionnisme étatique}

Au tournant du XIXe siècle, tous les États européens ont été obligés de reconnaître les droit sociaux des citoyens. Ce que T.H.Marchal appelle une citoyenneté sociale. Il distingue trois étapes dans la construction de l'État moderne, correspondant à trois dimensions de la citoyenneté. Chacune de ces étapes a permis la reconnaissance de trois ensembles de droits. \\
Les premiers droits sont les libertés individuelles: au XVIIe siècle en Grande-Bretagne, au XVIIIe en France. C'est la dimension civile de la citoyenneté. C'est la liberté de la personne, d'expression, de foie, de la justice. \\
Les seconds droits sont les droits politiques, qui correspondent à la citoyenneté politique. Cela s'est fait au cours du XIXe siècle. Elle implique un citoyen à l'exercice du pouvoir via le biais de l'élection. \\
L'étape la plus tardive correspond aux droits sociaux qui correspondent à la citoyenneté sociale qui passerait par un minimum de bien être économique, de la reconnaissance du droit à la sécurité face aux aléas et aux risques de l'existence (la mort, la maladie, le chômage, le veuvage). La mise en place des États providence en Europe a permis la reconnaissance de cette citoyenneté sociale. 


L'extension de la sphère d'intervention étatique est le produit de l'État providence depuis la seconde moitié du XIXe siècle. \\
L'avènement de l'État providence amène à une révolution de la place de l'État. Jusqu'à la révolution industrielle, l'État se cantonnait aux fonctions régaliennes. La mise en place de cet État s'explique par deux causes principales: des causes sociales d'abord, et politiques ensuite. \\
Les causes sociales sont les effets de la révolution industrielle sur le prolétariat urbain. Cela bouleverse les structures sociales comme le traduise des phénomènes comme l'exode rural ou l'urbanisation massive ; l'approfondissement de la division du travail ; la dislocation de la structure familiale qui n'est plus le lieu de la création économique. Ce prolétariat urbain a des conditions de vie très dures. \\
Les causes politiques sont l'accroissement du prolétariat qui entraînent la multiplication de conflits sociaux qui traduit des antagonismes de classes. Il y a une montée en puissance du mouvement socialiste ou des mouvements ouvriers dans les usines dont le patronat ou encore l'État a peur. L'État va rechercher à recréer de la cohésion sociale. La réponse principale qui va être donnée est la prise en charge des effets du développement de l'économie capitalisme via l'intermédiaire des assurances sociales. On passe de l'assistance à l'assurance collective. La responsabilité de l'accident du travail n'est pas individuel, il est le produit de la société industrielle qui est donc responsable.


L'Allemagne de Bismarck joue un rôle de précurseur. Il va faire voter des lois sociales dans les années 1880 et apparaît donc comme un précurseur. Dans la seconde moitié du XIXe siècle, l'Allemagne connaît une accélération de l'industrialisation accompagné par le succès du mouvement ouvrier et socialiste. Bismarck décide d'accompagner sa politique de répression du mouvement ouvrier par une politique de réforme sociale. \\
1871: première loi sur les accidents du travail, approfondi par la loi de 1884, avec une assurance maladie, assurance invalidité et vieillesse en 1889. Ainsi naissait le premier système d'assurance sociale obligatoire, qui garantit un revenu faible de compensation en cas de perte de salaire liés à un risque social. \\
Les caractéristiques du système Bismarckien va avoir beaucoup d'influence dans le reste de l'Europe. \\
En France, c'est sous la IIIe République qu'apparaît la question de l'interventionnisme de l'État dans l'économie. Mais les libéraux sont très farouches face aux républicains. Il faut attendre 1945 pour qu'un vrai système de sécurité sociale s'impose en France. \\
En 1942 Beveridge publie un rapport dont l'objectif était de penser le système social britannique. Il a eu beaucoup d'influence en Europe. Il met en avant le principe d'universalité, il rejette le principe d'une assurance relevé aux seuls salariés. Le but de sont rapport est d'éliminer la pauvreté donc l'ensemble des citoyens sont couvert par le système, qui verserait tous des prestations d'un même montant. Ce système, pour lui, doit être financé par l'impôt. \\
Bismarck: objectif de compensation de revenu pour les salariés alors que l'ensemble de la population est concernée chez Beveridge. Pour Bismarck, l'ouverture aux droits est conditionnés par le versement d'une cotisation alors que chez Beveridge c'est par l'impôt. \\
Le Plan français de Sécurité Sociale, de 1945 se situe entre Bismarck et Biveridge. Il veut atteindre les objectifs de Biveridge avec les moyens de Bismarck. Le plan est à l'époque fondé sur les 3U, Unité, Universalité, Uniformité. Les missions données à ce plan relèvent pour une part de Biveridge, notamment via le ciblage des prestations depuis 1970. En revanche, la sécu mise en place en 1945, jusqu'aux trente glorieuses relève de Bismarck. 

\section{L'État producteur}

C'est en 1944 que l'État producteur se met en place. C'est l'engagement de l'État en tant qu'agent économique. C'est l'État investisseur, dans le domaine industriel, financier, économique. À partir de 1945, le rôle de l'État s'accroît dans l'économie. Cela montre une véritable rupture dans la conception de l'État en France. \\
Les nationalisations entre 1944 et 1946 sont un levier (sanction, stratégique), le deuxième est la planification. Il y a une étatisation de l'économie française. 


















\end{document}
