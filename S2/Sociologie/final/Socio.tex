\documentclass[10pt, a4paper, openany]{book}

\usepackage[utf8x]{inputenc}
\usepackage[T1]{fontenc}
\usepackage[francais]{babel}
\usepackage{bookman}
\setlength{\parskip}{5px}
\date{}
\title{Cours de Sociologie (UFR Amiens)}
\pagestyle{plain}

\begin{document}
\maketitle
\tableofcontents

\part{Les inégalités de genre}

\chapter{Qu'est-ce que le genre ?}

Le concept de genre est moins institutionnalisé en France que dans de nombreux autres pays. Cependant, le concept est présent et acclimaté en France. \\
Le genre peut être compris comme la construction sociale et historique des différences entre femmes et hommes. C'est un système de catégorisation hiérarchisé entre les sexes et entre les valeurs et les représentations qui leurs sont associées. \\ 
Le concept de genre se fonde sur quatre éléments:
\begin{itemize}
\item Perspective constructiviste ;
\item Approche relationnelle ;
\item Considère les relations sociales entre les sexes comme un rapport de pouvoir (étude de la domination) ;
\item Imbriqué dans d'autres rapports de pouvoir, d'autres rapports sociaux.
\end{itemize}

\section{Construction sociale du genre: différence entre femmes et hommes}

\subsection{Simone de Beauvoir comme précurseur}

Elle publie "Le deuxième sexe" en 1949, elle écrit "On ne naît pas femme, on le devient". Elle dissocie donc le sexe biologique d'un sexe social. \\
Ce texte est important car c'est la première fois que théoriquement le genre est déconnecté du sexe, même si le terme de genre (qui apparaît dans les années 70) n'apparaît pas ici. \\
"Tout le monde s'accorde à reconnaître qu'il y a dans l'espèce humaine des femelles. Elles constituent aujourd'hui comme autrefois à peu près la moitié de l'humanité ; et pourtant, on nous dit que la féminité est en péril, on nous exhorte: soyez femmes, restez femmes, devenez femmes. Cette exhortation permet de tout de suite posée qu'être une femelle, ce n'est pas nécessairement être une femme." \\
Selon Beauvoir, les femmes sont les seules à se définir comme étant femme, contrairement aux hommes qui seraient "positif et neutre" donc universel. Cette subjectivité serait le produit de la nature et le corps féminin est défini par le manque. Le sexe féminin serait donc relatif à l'homme "Il est le sujet, il est l'absolu, elle est l'autre". \\
C'est cette argumentaire et cette façon de penser qui ont déconnecté les femmes du droit de vote.

\subsection{Margaret Mead}

Elle publie "Moeurs et sexualité en Océanie" en 1930. Elle pense les rôles sexuels comme des rôles sociaux qui ne découlent pas des sexes biologiques mais sont diversement construit selon les sociétés. \\
Elle travaille en Papouasie où il y a de nombreuse ethnies, peu distante mais très différentes dans leurs organisations sociales. \\
Elle s'intéresse au tempérament, un ensemble de traits de caractère. Elle travaille chez les Arapesh où il y a un tempérament homme comme femme "doux et sensible". Cependant, dans un groupe peu distant, les Mundugumor, le tempérament des hommes et des femmes est "violent et agressif". Chez les Chambulu, les hommes "doux et sensible" sont mariés à des femmes "violentes". \\
La variabilité des rôles montrent que ces rôles ne sont pas déterminés par le sexe biologique mais par l'organisation sociale de la société dans laquelle on se situe. \\
Cette variabilité est donc mis en exergue comme une construction sociale mais elle ne pense pas les relations de domination. 

\subsection{L'émergence du genre comme notion critique}

La notion de genre apparaît aux États-Unis et en Angleterre, mais c'est une sociologue anglaise du nom de Anne Oakley qui publie un ouvrage en 1972 qui s'appelle "Sex, gender et society" qui posera clairement le concept de genre comme sexe social. Le genre regroupe les aptitudes, les attitudes, les tâches considérées comme masculine ou féminine, socialement déterminées et donc variable. \\
Ce qui est intéressant dans la perspective de Oakley, c'est qu'elle pense ce qui est variable et permet donc de remettre en question ces rôles, ces différences. 


Irving Goffman, "L'arrangement des sexes", publié en 1977, nous dit que les différences biologiques entre homme et femme sont très peu nombreuse et donc pas si importante que ça. \\ 
Il dit "Pour que ces faits matériels de la vie n'ait pas d'appréciable conséquence sociale, il suffirait d'un peu d'organisation, mais relativement peu, selon les normes modernes". Ce qui intéresse donc Goffman, c'est comment ces deux classes peuvent se maintenir et se reproduire sachant que ces différences ne sont pas importantes dans la société actuelle. \\
Il va chercher comment, dès le "triage initial", ces deux classes sexuelles peuvent se maintenir. Pour lui, dans nos sociétés, la première catégorisation se fait à la naissance où l'on trie par le sexe: l'attribution du sexe est la catégorisation première des individus dans la société. Il va donc s'intéressé aux effets de ce classement et donc les différences de traitement entre filles et garçons: les agencements de ségrégation sexuée. \\ 
Goffman parle de ségrégation périodique entre les sexes. "Plus qu'un produit de la différence biologique entre les classes, il s'agit d'une manière de la produire". 


La notion de genre permet d'appréhender les rôles sociaux en dénaturalisant, de déconnecter les rôles sociaux du biologique. Cela permet donc de comprendre les différences qui sont socialement construite. Cela va donc remettre en question les caractéristiques des hommes et des femmes: la notion va à l'encontre de la vision essentialiste qui donne des caractéristiques immuables à la caractéristique biologique. \\
C'est cette portée critique qui fait que la notion de genre a de nombreux ennemis. 

\subsection{Approche relationnelle}

Les caractéristiques qui sont associées à chaque sexe sont construites socialement dans une relation d'opposition. Ils sont le produit d'un rapport social. On ne peut donc pas étudier un groupe de sexe sans le rapporter à l'autre. \\
Goffman a écrit un texte "La ritualisation de la féminité" dans lequel il analyse le traitement de la femme dans les magazines. Ce qu'il montre dans ses analyses, c'est où en est la ritualisation de la féminité par rapport à l'homme: les femmes sont souvent représentées comme subalterne, docile, infantile, inférieure à l'Homme.

\section{Le genre comme rapport social asymétrique, comme rapport de pouvoir}

Quand on se situe dans la perspective des rapports de pouvoir, on ne va pas se contenter de montrer qu'il y a une différence de construite socialement, mais comment les différences conduisent à des inégalités. La différence entre homme et femme a été culturellement associée à une hiérarchisation. Même si les rapports de pouvoir sont multiforme et très variable d'une société à l'autre, la quasi totalité des sociétés étudiées présentent une distribution inégale des ressources au profit des hommes au détriment des femmes ainsi qu'une valorisation systématique du masculin au détriment du féminin. \\
Les rôles sociaux féminin et masculin sont donc associés à des valeurs et à une hiérarchie dans l'espace social. \\
Joan Scott a écrit "Gender, a usefull categorie of historical analysis", elle dit "Le genre est un élément constitutif des rapports sociaux fondés sur des différences perçus entre les sexes, et le genre est une façon première de définir des rapports de pouvoir". Elle repère quatre éléments constitutif du genre:
\begin{itemize}
\item Les symboles culturellement disponible ;
\item Les concepts normatifs (dans le droit, la religion, la psychanalyse etc.)
\item Les institutions, l'organisation sociale (organisation de la parenté, marché du travail etc.) ;
\item L'identité subjective. 
\end{itemize}

\section{L'articulation des différents rapports de pouvoir}

Il ne faut pas analyser les rapports de genre indépendamment des autres rapports de pouvoir. Les catégories de sexe ne sont pas homogènes. \\
S. Truth, ancienne esclave, déclarait lors de son discours "Ain't not a woman ?" à la convention des droits des femmes en 1851 aux États-Unis "Cet homme là bas dit que l'on doit aider les femmes à monter dans une calèche[...]" \\
Bourvois "En quête de respect", une enquête sur des dealers dans un ghetto américain dit qu'au départ il parait efféminé à cause de sa manière de parler etc. et est catalogué comme non viril donc gay. Il montre donc qu'un certain nombre de dealer, qui cherche à être très viril, ne peuvent pas se reconvertir dans des emplois légaux car sont justement trop attachés à leur masculinité.

\chapter{Genre et socialisation}

\section{La socialisation}

Muriel Daron dit de la socialisation: "C'est l'ensemble des processus par lesquels un individu est construit par la société globale et locale dans laquelle il vit". Ce sont donc des processus au cours desquels l'individu intériorise des façons de faire, de penser, et d'être qui sont situées socialement. \\
Elle dit aussi: "La socialisation, c'est la façon dont la société forme et transforme les individus". \\
La socialisation est dynamique, elle n'est jamais achevée, elle continue tout au long de la vie. La socialisation a donc un caractère très dynamique. \\
La socialisation n'est pas de l'ordre de l'éducation ou l'inculcation, elle se fait de manière inconsciente. \\
La notion d'habitus est très proche de la notion de socialisation. Bourdieu, dans "La domination masculine" parle d'habitus masculin et d'habitus féminin. 

\subsection{La socialisation de genre}

On peut isoler trois dimensions principales:
\begin{itemize}
\item La dimension de l'apprentissage des rôles de sexe ;
\item La dimension de la ségrégation sexuée ;
\item L'identité de genre que développe les enfants. 
\end{itemize}

Il y a autant de socialisation de genre qu'il y a d'instances de socialisation, pareil pour les espaces de socialisation.

\section{La socialisation de genre primaire: la famille}

\subsection{Les parents comme modèle sexués}

Les deux parents offrent deux modèles d'identification distincts dans la famille. Les mères, dans l'éducation des enfants, ont plutôt des activités verbales ou visuelles ; alors que les pères se chargent de l'activité ludique ponctuelle, plutôt physique, psychomoteur. \\
Par ailleurs, les pères et mères ne s'adressent pas de la même manière aux enfants: les pères parlent davantage comme des adultes à leurs enfants tandis que les mères parlent davantage à ceux-ci avec un "langage bébé". Ce serait donc plutôt le père qui font le pont avec l'extérieur. \\
Pour des enfants un peu plus âgés, une sociologue a fait une enquête sur l'argent de poche: le rapport à l'argent serait sexué. Pour les plus jeunes, l'argent est d'abord donné par la mère par de petites fractions, de main en main. C'est ensuite le père qui prend le relais pour des sommes plus importantes et rationalisés. On notera aussi que ce sont plutôt les mères qui s'occupent des devoirs. \\
Les parents ne font pas la même chose et constituent donc un modèle d'identification différent. 

\subsection{Des comportements parentaux différenciés}

Un des premiers ouvrages dédiés à cette question s'intitule "Du côté des petites filles", de Elena Bellotti, en 1973. Elle montrait que dès la grossesse, la conduite n'est pas identique suivant le sexe de l'enfant. Cette différence de conduite tend à valoriser les garçons. \\
Par exemple, les femmes allaitent plus souvent les garçons que les filles, de plus, les parents tolèrent que les garçons soient voraces. Les cris des garçons sont plus tolérés etc. \\

\subsection{Des environnements matériels différenciés}

Ces différences sexuées s'incarnent aussi dans l'environnement des enfants: on va offrir des environnement différents aux filles et aux garçons. \\ 
Alors que l'égalité entre homme et femme adulte s'améliore, les différenciations sexués se font de plus en plus, et de plus en plus jeunes. \\
Si on s'intéresse aux vêtements, on se rend compte que la différenciation entre petites filles et petits garçons est très récente sur une perspective longue: sous l'ancien régime, cette différenciation n'existait pas, que ce soit garçon ou fille, ils étaient habillés exactement de la même manière. Ce n'est qu'au début du XXe siècle que les habits des garçons et filles ont commencés à se différencier. \\ 
Cela est pareil pour les couleurs, la différenciation est là aussi très récente vu que l'essentiel des vêtements étaient blancs. La différenciation date des années 60. \\
Pour ce qui est des jouets, ce sont des agents périphériques de socialisation qui véhiculent des normes de genre. Souvent, pour un enfant, jouer, c'est imiter des adultes. Les jeux ritualisent les rôles sexués et accentue les différences. Au XVIIIe et XIXe siècle, fille comme garçon jouent à la poupée. C'est au XIXe que l'on va commencer à manufacturé les jouets et donc créer des différences. Différences que l'on connaît aujourd'hui.

\section{La socialisation de genre à l'école}

Malgré une idéologie d'égalité entre le sexes, les enquêtes d'observation menées en classe, montrent une différence dans le comportement des enseignants. \\
Les enseignants interagissent plus avec les garçons qu'avec les filles. Les élèves garçons sont davantage perçus comme des individualités, et les filles, plus indifférenciées. Les sociologues ont aussi montrés, que, sur le contenu des interactions, les enseignants adressent davantage de remarques sur le fond aux garçons et sur la forme aux filles. \\
Les enseignants attribuent plus la réussite des filles à leurs efforts et celle des garçons à leurs qualités intellectuelles. \\
Les enseignants ont aussi des attentes stéréotypés sur les sexes: ils s'attendent à ce que les garçons soient dissipés et que les filles soient sages. Ils ont aussi un comportement différencié en matière d'orientation: ils proposent plus d'orientation en matière scientifique aux garçons qu'aux filles. 

\subsection{La réussite scolaire des filles}

"Allez les filles" est un ouvrage de Baudelot qui montre que les filles ont de meilleurs résultats, de plus, il y a plus d'étudiantes que d'étudiants. \\
Les explications avancées dans l'ouvrage sont que la socialisation primaire des filles les place en meilleure condition de conformité avec l'univers scolaire. De plus, comme ce sont les mères qui aident principalement aux devoirs, l'identification à la mère par les filles permettraient une plus grande aisance dans la scolarité. Les filles seraient aussi plus sensible aux attentes institutionnelles. Elles sont moins absentes, plus assidues, etc. \\
Jamais les orientations ne sont à la hauteur des réussite. Les différences d'orientation ne sont pas dus à une différence de réussite. Pour comprendre cette différence, il faut enquêter à un palier d'orientation (seconde), et on remarque que si les résultats sont semblables, les filles ont moins de culture scientifique personnelle. Il y a donc une différence dans l'auto-évaluation: à même niveau, les garçons s'estiment bon, les filles, non, et n'envisage donc pas de passage dans une filière scientifique. \\
Malgré la mixité scolaire, les modèles de sexe continuent à prévaloir. "Il est exceptionnel que les filles issus d'une catégorie sociale donnée dépasse le niveau des garçons de la catégorie immédiatement supérieure". 

\section{Pratiques culturelles}

\subsection{Le sport}

Mennesson a fait une enquête sur les femmes boxeuses et footballeuse dans le top niveau chez les amateurs. \\
Elle va chercher à comprendre comment elles ont réussi à transgresser les normes de genre et s'imposer dans un sport masculin. Elle va se rendre compte que la majorité des enquêtés ont constitués des dispositions sexuées inversés au cours de l'enfance et de l'adolescence dans la socialisation sportive et la participation au groupe de père masculin. Elles ont donc été socialisés par les hommes (père, frère) de leur famille où le sport a une certaine importance. \\
Ces filles ont donc généralement refusé d'intégré le groupe de filles. Cette socialisation spécifique s'inscrit dans des configurations familiales spécifiques: par exemple, une fratrie de filles où la plus jeune endosse le rôle du garçon pour répondre aux attentes du père d'avoir une descendance masculine (modèle du garçon manquant). Le deuxième modèle est au sein de familles nombreuses et la socialisation d'une fille serait confiée à ses frères les plus proches en âge. \\
Ces sportives intègrent plus ou moins le modèle de "garçon manqué" suivant l'intensité et la précocité de la socialisation sportive. 

\subsection{La lecture}

Les produits culturels disposent des stéréotypes sexués. Des études montrent que, au niveau des romans, les livres d'enfants etc. les stéréotypes sont fortement présents. Les femmes ont tendance à être peu présentes, passives, et les hommes à être les héros. \\
Detrez travaille sur la lecture des mangas, et montre que les garçons s'identifient parfaitement aux shonen seulement, alors que les filles montreront qu'elles lisent shojo comme shonen. Les lecteurs de mangas, une fois qu'ils ont affirmés leur identité sexué, vont prendre plus de distance et lire aussi des shojos: leur discours devient moins stéréotypé. 




\chapter{Genre et travail}

\section{Travail domestique et division sexuée du travail}

\subsection{Qu'est-ce que le travail domestique ?}

On appelle ça aussi le travail de reproduction: cela recouvre un certain nombre de tâches, qui sont relative au soins aux personnes et qui sont réalisés de manière quotidienne et gratuite, dans le cadre de la famille afin d'assurer l'entretien et le bien être des différents membres de cette famille. \\
On inclut dans le travail domestique le travail parental: soins aux enfants et aux adultes dépendants (personnes âgées) mais aussi les tâches administratives: courrier, impôts, etc. \\
Christine Delphy a publié un recueil d'articles des années 70 et 80 dans "L'ennemi principal - Tome 1: économie politique du patriarcat" qui montre que les tâches domestiques sont surtout caractérisé par leur gratuité. En effet, un grand nombre de tâches domestique pourrai être marchandes.


\subsection{Une répartition des tâches domestiques toujours inégalitaire}

La mesure du travail domestique date des années 70: on manque donc d'éléments pour comparé à long terme. \\
L'INSEE fait les premières mesure en 1981. Deux sociologues vont travaillées sur cette enquête et diront du travail domestique: "Tout activité réalisée au domicile et qui a un substitut marchand".  \\
En 1974, une femme inactive accomplit 40h de travail domestique par semaine, ce qui représente une part significative du PIB. Les sociologues montrent qu'une femme qui se dit inactive travaille donc. Elles montrent aussi que la mise en couple augmente le travail domestique des femmes mais a un effet nul sur les maris. Quand il y a un ou des enfants, cela augmente encore le travail domestique des femmes, ça a un impact nul sur les maris qui, au contraire, augmente leur temps de travail salarié. \\
La participation des femmes au marché du travail ne s'accompagne pas d'une redistribution égalitaire des temps de travail domestiques: une femme à temps plein travaille environ 28h par semaine aux travaux domestiques contre 16h pour les hommes. \\
Ces pratiques changent peu et lentement, par jour, les femmes consacrent en moyenne 3h30 aux travaux domestiques contre 2h pour les hommes. Les évolutions sont très lente car le temps domestique des hommes a augmenté de 6 minutes entre 1974 et 1999, et de 1 minute entre 1999 et 2010. 


Les hommes et les femmes ne consacrent donc pas les mêmes temps mais ne font pas du tout les mêmes choses. \\
Les hommes s'adonnent au bricolage, jardinage, soin aux animaux. Les femmes au ménage, aux courses, à la cuisine, le linge, prendre en charge les adultes dépendants. Le fait de s'occuper des enfants est assez partagé. \\
Même si les femmes des classes supérieures sont déchargés de certaines tâches, ce sont tout de même elles qui vont s'en occuper dans le sens où ce sont elles qui vont recruter des personnes pour s'occuper du ménage, du linge etc. \\
Il existe donc des tâches féminisées et d'autres qui sont plutôt masculines. 

\subsection{Division sexuée du travail}


D'après Danièle Kergoat: il y a deux principes  organisateurs de la division sexuée du travail: le principe de séparation (tâche masculine et tâche féminine) et le principe hiérarchique (les tâches masculines sont d'une valeur supérieure aux féminines). \\
Il est à noté que ce qui est considéré comme travail masculin ou féminin varie fortement selon les sociétés, mais la séparation est toujours présente. \\
"La division sexuelle du travail a pour caractéristiques l'assignation prioritaire des hommes à la sphère productive et des femmes à la sphère reproductive ainsi que simultanément, la captation par les hommes des fonctions à forte valeur sociale ajoutée". \\
La spécificité des tâches féminines est leur invisibilité alors que les tâches masculines sont visibles et valorisés. \\
Ce sont les femmes ou les filles aînées qui réalisent, dans les classes populaires, la majorité du travail administratif.


\section{Un accès inégal à l'emploi}

L'analyse de la division sexuée du travail est essentiel pour comprendre les inégalités de genre dans la sphère professionnel, d'abord parce que les femmes sont contraintes dans l'accès à l'emploi et d'autre part parce que cette division organise l'accès à l'emploi: les femmes et les hommes n'occupent pas les mêmes métiers, et, quand c'est le cas, on se rend compte qu'ils ne s'approprient pas le métier de la même manière et n'ont pas les même carrières. \\
Il y a un inégal accès à l'emploi et une inégalité dans l'exercice concret des métiers. 


\subsection{L'accroissement du salariat des femmes}

Les femmes travaillent depuis longtemps (évident si on prend en compte le travail domestique) mais c'est vrai aussi si on parle de travail marchand. Les conjointes d'agriculteurs, de marchands, etc réalisent un travail marchand non reconnu. \\
On recense 11,6 millions d'hommes actifs et 5 millions de femmes actives à la fin du XIXe siècle. Le salariat féminin existe donc depuis pas mal de temps. Si on ajoute à ces 5 millions les femmes qui font du travail marchand non reconnu, on arrive à 8,1 millions de femmes actives. De ce fait, l'écart n'est donc pas si important que ça. \\
Il y a une croissance très importante au XXe siècle.


Le taux d'activité des femmes croît très rapidement dans les années 60 et 70, et continue de progresser dans les années 80. Pour les femmes entre 24 et 57 ans, on passe de 59,9\% à 82,3\% de femmes qui travaillent. \\
Entre 1954 et 1982, le taux d'activité des mères de deux enfants passent de 24,7\% à 57,7\%, en 1997, on est 74\%, et enfin, en 2007, à 84\%. \\
D'après Margaret Maruani, le modèle dominant n'est plus le choix de la maternité ou du travail, ni le modèle de l'alternance (alterner entre période professionnel et maternage), mais le modèle dominant est le modèle du cumul dans lequel les maternités n'induisent pas d'interruption de l'activité professionnel. \\
Actuellement en France, les mères de deux enfants sont majoritairement actives ; ce n'est qu'au troisième enfant que le taux d'activité baisse. \\
Sur la même période, quand on s'intéresse au taux d'activité des hommes, il est très peu sensible à leur situation familiale. Il est de 96\% sans enfant, 98\% avec enfants.  


\subsection{Un accès à l'emploi qui demeure inégalitaire}

Il demeure des importantes disparités d'insertion avec notamment la question du temps partiel. Le temps partiel est une forme d'emploi féminin: en 2012, 30,2\% des femmes travaillent à temps partiel contre 6,9\% des hommes et 80\% des emplois à temps partiel sont occupés par des femmes. Dans un contexte de crise économique, cette forme d'emploi a été encouragée par les pouvoirs publiques. Le commerce, les services à la personne etc. se sont emparées de cette forme de travail pour gérer la main d'oeuvre. La plupart du temps, le temps partiel est imposé par l'employeur. \\
Seul 34\% des emplois à temps partiel féminin ont été choisis par les femmes pour des raisons familiales. C'est important car le temps partiel imposé concerne souvent des femmes jeunes, peu diplômée, peu qualifiée et en contrat précaire ; c'est une source importante de sous emploi. \\
Le temps partiel choisi concerne des femmes plus qualifiées et souvent dans la fonction publique. 


Pendant très longtemps, on avait un chômage des femmes qui était plus important que celui des hommes. Depuis quelques années, l'écart est très serré, voir même tend à s'inversé. L'écart n'est donc plus significatif. \\
Il y a de forte disparité selon l'âge, les diplômes ou la PCS. \\
Il peut y avoir une sous évaluation du chômage féminin car la frontière entre inactivité et chômage est plus poreuse: les femmes préféreront déclarer qu'elles sont femmes au foyer plutôt que chômeuse et l'inverse pour les hommes. 


\section{Un marché du travail segmenté: métiers de femmes et métiers d'hommes}

\subsection{Des métiers différents et inégalement prestigieux}

L'accroissement du salariat des femmes s'est produit au moment de la tertiarisation de l'économie. Les femmes se sont donc retrouvés concentrés dans le secteur tertiaire. Les femmes ont principalement pris des emplois dans des secteurs qui étaient déjà anciennement féminisés: éducation, santé, social. \\
De ce fait, on a 80\% de la population active féminine qui est concentré dans 5 secteurs d'activités: éducation, santé, action sociale, service aux particuliers, service aux entreprises. \\
À l'inverse, certains secteurs sont très masculins: l'industrie, le BTP, les transports. \\
C'est la ségrégation horizontale du marché du travail qui se double d'une ségrégation verticale: les hommes et les femmes ne sont pas représentés de la même façon dans les échelons.


Les métiers du "care" sont les métiers relatifs aux soins et à la prise en charge des jeunes enfants et des personnes dépendantes. Ces métiers permettent de montrer qu'il y a un déplacement des métiers féminins plutôt qu'une véritable remise en question de la division sexuée du travail. \\
Il y a une continuité de ces métiers avec le rôle des femmes dans la sphère privée. Ces métiers sont peu reconnus socialement et économiquement. \\
Ces métiers montrent un mécanisme: le déni de qualification. C'est à dire que les compétences des femmes acquises dans le travail domestique sont valorisées par les employeurs mais que ces compétences sont considérées comme des qualités féminines (comme quelque chose d'inné) et non pas comme des qualifications qui auraient fait l'objet d'un apprentissage. Ce déni de qualification entraîne notamment le fait que les métiers "féminins" sont bien moins payés. \\
Hoschild parle de travail émotionnel: il y a beaucoup de métiers où il y a une part importante d'émotions, faire paraître les émotions fait parti du travail (empathie etc.). Ce travail représente un coup psychique pour les femmes. Il est à la fois attendu de le faire et en même temps non reconnu dans le travail. Les femmes sont beaucoup moins protégées des sollicitations émotionnelles et sexuelles. 


\section{Métiers mixtes et carrières sexuées}

La mixité s'accompagne d'une recomposition des différences plutôt qu'une disparition.

\subsection{Féminisation des professions}

On a souvent associé féminisation et dévalorisation, à tort. Dans un certain nombre de cas, il y a effectivement dévalorisation mais liée à d'autres facteurs. \\
Plus un secteur d'emploi est ancien, moins il se féminise facilement. Ce qui facilite la féminisation d'une profession, c'est quand l'entrée est régi par un diplôme. Pour ce qui est des secteurs techniques, l'insertion des femmes a été plus facile dans les secteurs récents comme l'électronique où il n'y a pas de tradition ouvrière masculine à contrario de l'industrie, du bâtiment etc. \\
Les pionnières dans un domaine sont très souvent sur-sélectionnées socialement et scolairement. Souvent, celles-ci utilisent des techniques de "neutralisation", en euphémisant leur apparence de femme. Cependant, pour les femmes suivantes, c'est beaucoup plus variable, voire même les pousse à mettre en valeur leur féminité. Par exemple, les femmes chirurgiens font face à des pratiques sexistes très importantes et vont donc marqués leur féminité de manière symbolique (maquillage par exemple). 

\subsection{Mixité et différenciation sexuée}

La mixité est ici au sens de coexistence des deux sexes ne signifie pas égalité, car on constate que au sein de professions mixtes, il tend à se redéfinir des spécialités féminines et masculines. \\
Si on prend l'exemple de la magistrature qui était une discipline masculine (les femmes n'avaient pas le droit de passer le concours), mais cependant, une sociologue française a montrée que la discipline est mixte et paritaire, mais les femmes et les hommes n'aspirent pas à la même fonction. La fonction à laquelle aspire le plus les magistrates est de devenir juge des enfants, pour les magistrats, ils souhaitent surtout aller au parquet. 72\% des juges des enfants sont des femmes, cependant, elles sont minoritaires au parquet. \\
Chez les médecins, l'analyse est la même que chez les magistrats. Les femmes sont surtout présentes en pédiatrie, dermatologie etc.


Isabelle Bertaux Wianne a travaillée sur des petits indépendants: artisans, commerçants etc. qui ont la spécificité de s'installer en couple. Elle s'est intéressée à qui faisait quoi. Les conditions d'installation sont asymétriques, c'est souvent un projet de l'homme dans la continuité de sa formation alors que pour les femmes, c'est un projet étranger car souvent elles vont accepter le projet de leur époux et abandonner leur projet antérieur. \\
Ces femmes ont souvent un niveau scolaire plus important que leur mari, mais en même temps, elles anticipent leur valeur sur le marché du travail (la situation des femmes est plus incertaine et moins rémunératrice), mais ces femmes ont surtout intériorisée que la dimension professionnel est très important pour un homme. Dans ces cas, l'homme garde le statut principal, alors que la femme a le statut secondaire (femme de boulanger, de boucher etc.). \\
Celles qui n'adhèrent pas à ce modèle, ce sont celle qui ont un emploi régulier et bien payée alors que celle qui adhère sont des femmes de milieu populaire qui voient une possibilité d'ascension sociale. 


Dans le secteur de l'imprimerie, un secteur très ancien, il n'y avait que des hommes. En 1969, une imprimerie embauche des femmes pour être claviste sur de nouvelles machines. Le syndicat des ouvriers typographes négocie avec la direction que les femmes n'aient pas accès à la machine qui compose le texte (la plus importante de l'usine). Ce que montre les sociologues, c'est que peu à peu, les techniques évoluent (avec l'arrivée de l'informatique), mais à chaque fois que des changements techniques sont introduits, les ouvriers négocient des différences avec la direction. \\
Dans les années 80, les sociologues décrivent la situation suivante: les ouvriers qualifiés saisissent le texte et le corrige et font 180 lignes à l'heure. Les femmes saisissent le texte bien plus rapidement (300 lignes à l'heure) mais théoriquement, elles ne font pas de travail de correction et n'ont pas le droit à plus de 5\% d'erreurs. La différence de traitement entre tâche et salaire devient trop évidente, une grève des femmes permettra d'égaliser la situation. \\
Cette histoire dans une imprimerie nous montre réellement les conséquences de l'arrive de la mixité sexuée dans un secteur masculin.


La différenciation sexuée varie très fortement selon les professions et les organisations. Catherine Marry a travaillée sur les ingénieurs et montre qu'il y a une banalisation des femmes dans ce milieu. Mais selon les secteurs, il y a un maintient de la différenciation sexuée. \\
D'habitude, les stéréotypes agissent contre les femmes, mais parfois, ils jouent en faveur des femmes. Dans la police par exemple, le fait que le recours à la violence soit vu comme masculin joue dans l'intérêt des femmes car elles accèdent plus facilement à des postes de direction. 


\subsection{Le plafond de verre: Ségrégation verticale}

Les femmes et les hommes ne gravissent pas les échelons de la même manière: c'est la ségrégation verticale. Il y a une concentration des femmes dans les métiers et les postes du bas de l'échelle hiérarchique. \\
Ce qu'on appelle le plafond de verre, c'est le fait quand dans une profession donnée, les femmes sont de moins en moins nombreuse au fur et à mesure qu'on s'élève dans la hiérarchie. Laufelr écrit que le plafond de verre, c'est l'ensemble des obstacles visibles et invisibles qui séparent les femmes du sommet des hiérarchies professionnelles et organisationnelles. \\
Ce concept est particulièrement utilisé dans l'étude des carrières des femmes cadres et diplômé qui, dans le public comme dans le privé, n'accèdent pas à des postes de direction. \\
Les femmes représentent 58\% des salariés du public mais seulement 37,5\% des cadres et 12,5\% des cadres supérieurs. Il n'y a que 16\% de femmes qui sont chef d'entreprise. 


Les facteurs qui forment ce plafond de verre sont à la fois de l'ordre de la discrimination directe ou indirecte ainsi que de l'auto-exclusion. \\
Cécile Guillaume et une autre sociologue ont fait une étude sur une grosse entreprise et montre qu'il y a un plafond de verre qui résulte de la combinaison de cinq facteurs. \\
Le premier facteur est la politique de gestion des ressources humaines dans l'entreprise. Les RH favorisent la promotion de cadres issus de certaines grandes écoles d'ingénieurs dans lesquelles les femmes demeurent très minoritaires. \\
Le deuxième facteur, ce sont les freins à la mobilité géographique des femmes. La progression dans la carrière passe par la mobilité géographique et les femmes hésitent ou renoncent à cette mobilité car pense avoir des difficultés à l'imposer à leur conjoint. \\
Troisième facteur, les salariés qui aspirent à une promotion doivent faire preuve d'un investissement total qui est difficilement compatible avec des charges familiales. \\
Quatrième facteur, il y a un certain calendrier de promotion et donc rater une promotion du à une maternité par exemple, handicapent les femmes qui deviennent trop âgé pour être promu. \\
Cinquième facteur, les réseaux de sociabilité et de cooptation qui permettent de grimper dans la hiérarchie sont très marqués par des réseaux masculins. Les femmes n'ont donc pas accès à ces réseaux. \\
Les normes pratiques de ces organisations sont la source de discrimination sans qu'il y ait la volonté de discriminations. C'est de la discrimination indirecte. 


Il existe des processus qui relèvent d'une réelle discrimination. Un argument mobilisé est que les femmes seraient soumise à la charge domestique ou encore les grossesses. \\
Une enquête a été réalisée sur ce sujet. Cette perception discrimine les femmes dans l'accession aux postes supérieures. Battagliola a mené cette enquête et remarque l'absence de promotion des femmes à ancienneté égale avec les hommes. \\
La direction du centre justifie cet écart par la moindre disponibilité des femmes en raison de leurs charges familiales. De nombreuses femmes ont fait preuve d'une grande disponibilité dans leur travail et ont, pour autant, eu aucune promotion. 

\subsection{Carrière masculine et arrangement conjugaux}

On va s'appuyer sur l'enquête de Catherine Mary sur les ingénieurs et réalise qu'il y a une corrélation statistique entre la réussite professionnel des hommes ingénieurs et l'importance de leur descendance. Elle donne deux explications: cette plus grande réussite renvoie à la division du travail domestique qui existe dans les couples d'ingénieurs de familles nombreuses, ces pères sont dégagés de toute contrainte domestique par des épouses qui sont très diplômée mais femme au foyer. \\
La deuxième explication est le renvoie aux normes et aux stéréotypes de genre des employeurs qui associent aux pères de familles stabilisés le sens des responsabilité et l'autorité et qui donc, parmi les hommes ingénieurs, seront favorisés pour monter dans la hiérarchie.


Dans l'exemple de la haute fonction publique, le sociologue Jacquemart a écrit "J'ai une femme exceptionnelle" et montre que la conciliation entre vie pro et vie perso n'est un problème que pour les femmes, les hommes, eux, parviennent à s'investir dans leur vie personnelle sans avoir à résoudre des problèmes de conciliation. Les femmes de la haute fonction publique articule calendrier des postes et calendrier familial: elles font en sorte d'avoir une seule grossesse par poste. \\
Dans les entretiens, on voit que ces femmes ont du mal à se plier à l'amplitude horaire et la mobilité géographique que leur poste requiert. Elles ne postulent donc pas dans des postes plus élevés par crainte de séparation conjugale. \\
Les hommes eux, ne se sente pas de devoir planifier les naissances en fonction de leur calendrier pro, et d'ailleurs, dans les entretiens, les hommes font très rarement référence à leur vie domestique, familiale dans leur récit de carrière. Ils décrivent des arrangements conjugaux qui leur permet d'investir pleinement la sphère professionnel, d'être disponible, d'être mobile géographiquement car la vie familiale est prise en charge par leur conjointe. \\
"Pour l'instant, elle reste à la maison car elle a eu un deuxième enfant. Elle a eu une petite interruption de trois ans." Les entretiens montrent l'état de la situation familiale et les arrangements conjugaux chez les hauts-fonctionnaires. Ces modèles conjugaux sont majoritaire parmi les enquêtés, quelque soit la génération ou la classe sociale d'origine. \\
Sur 95 enquêtés, il y a deux couples qui ont un modèle égalitaire. C'est un modèle de couple à double carrière. 


\chapter{Genre et politique}

\section{La République masculine: histoire de l'exclusion politique des femmes}

\subsection{Les femmes en politique sous l'ancien régime}

À l'époque moderne, après le Moyen-Âge, les femmes disposent d'un certain nombre d'accès aux politiques: les femmes monarques, les femmes dans les assemblées provinciales, les femmes aristocrates chef de guerre, les femmes de classe populaire émeutière. \\
L'accès des femmes au trône est limité par rapport à celui des hommes et il est spécifique. Il existe une loi salique qui progressivement est considéré comme l'une des lois fondamentales du Royaume. Pour autant, il y a tout de même des cas où les femmes règnent: c'est le cas d'une régence par exemple (De Médicis). \\
Sous l'ancien régime, certaines femmes peuvent siéger dans les assemblées provinciales et élire des députés aux États-Généraux. Les femmes qui peuvent siéger sont les abbesses du clergé, les héritières des fiefs de la noblesse, les membres des corporations féminines dans le tiers État. Ces femmes siègent car on leur reconnaît un statut indépendamment de leur sexe. \\
Une dernière figure sont les femmes chef de guerres (aristocrates) et les émeutières. Elles participent à la guerre, au moment de la fronde (milieu du XVIIe) notamment. La duchesse De Longueville soutient les parlements contre Mazarin par exemple. Le fait que des femmes participent à la guerre est quelque chose d'accepté. Il y a une sorte d'indifférence. Cependant, celle-ci est reconnue à des exceptions: elles sont de hautes aristocrates, leur statut social prime sur leur sexe. \\
Toujours sous l'ancien régime, les femmes de classes populaires participent aussi à la violence politique, ce sont les femmes émeutières. Pharge montre que les femmes de classe populaire participent aux émeutes, et ça n'a rien d'anodin, elles sont souvent considérés comme sous responsable et risquent moins devant les tribunaux. De plus, elles prennent aussi la parole. Souvent, ce sont les femmes qui vont faire les missions de sabotages. Quand elles participent aux émeutes, souvent, elles se travestissent. 

\subsection{L'ère démocratique n'est pas, à priori, favorable aux femmes}

Godineau s'intéresse à la place des femmes dans la Révolution dans l'article "Fille de la liberté et citoyenne révolutionnaire". \\
Elle nous montre que les femmes participent activement aux mouvements populaires et investissent l'espace public. Elles sont très présentes dans les émeutes dès 1789 et sont très encouragés à le faire "Les femmes commenceront le mouvement, les hommes viendront à l'appui", un député de l'AN. \\
On notera que les armes sont maniées par les hommes. Les femmes sont donc très présentes dans la parole publique, notamment aux séances des assemblées. Elles sont donc aussi présentes dans le mouvement contre révolutionnaire. \\
Les femmes rejoignent des clubs mixtes mais créent aussi des clubs féminins (plus de 50 clubs féminins à Paris et en province au début de la Révolution). \\
Cependant, malgré tout, elles vont être considérées comme des citoyennes passives par les hommes révolutionnaires et les clubs féminins seront interdits tout comme leurs droits citoyens. Dès 1789, les lois sur les élections excluent les femmes et les révolutionnaires votent aussi une Loi qui interdit les femmes de participer à la garde nationale. La Constitution de 1791 classe les femmes comme citoyennes passives: elles sont privées de droit politique. Les clubs féminins sont interdits en 1793. En 1795, après des émeutes où les femmes ont été très présentes, on leur interdit l'accès aux tribunes de l'AN, interdiction d'assister à toute assemblée politique et de s'attrouper à plus de 5. \\
À partir du moment où on accède à un régime démocratique fondé sur l'égalité des droits, on ne peut plus toléré des exceptions concernant les femmes dans l'espace politique. Joan Scott dans "La citoyenne paradoxale, les féministes françaises et les droits de l'homme", met en évidence la manière dont la référence aux différences naturelles entre les sexes justifie l'exclusion politique des femmes. \\
Condorcet est le seul philosophe à ne pas véhiculer ces idées. 


\subsection{L'inclusion des femmes dans la citoyenneté}

C'est dans la deuxième moitié du XIXe siècle que vont naître des mouvements suffragistes en Angleterre, en France et aux États-Unis. \\
Dans la première moitié du XIXe, on assiste à des prémices à ces mouvements mais restes marginaux. C'est pendant la IIIe République qu'apparaît le terme de féminisme qui se divise en branche réformiste (réclamant droit politique et réforme du statut civil), et une branche radicale qui souhaite une transformation complète des relations entre les sexes. \\
La revendication suffragiste s'institutionnalise et prend forme dans les années 1870. La première grande leader qui porte la cause est Hubertine Auclert qui fonde en 1877 une association appelée "Le droit des femmes" qui s'appelle ensuite "Le suffrage des femmes" et fonde en 1881 un journal appelé "La citoyenne". \\
Auclert sera suivi par Madeleine Pelletier qui fonde en 1908 "La suffragiste". Cette revendication est de plus en plus reprise pas nombre de mouvements politiques modérés. En 1910 a lieu le premier meeting suffragiste qui réunit 1200 personnes. Une partie des animatrices de ce mouvement se présente même si elles n'ont pas le droit et obtiennent 4\% des voix. \\
Cécile Brunchvick crée l'union du droit des femmes et sera la première secrétaire d'État sous le gouvernement de Léon Blum. \\
Les leaders du parti radical soutiennent le droit de vote des femmes. \\
À l'entre deux guerres, c'est le moment où les femmes obtiennent le droit de vote dans de nombreux pays européens. C'est un résultat des mouvements suffragistes, une récompense de l'engagement des femmes dans la guerre ou encore une manière de lutter contre le communisme. \\
En 1919 est préparé le projet de Loi sur l'égalité des sexes, voté à l'AN mais rejeté par le Sénat. \\
Les femmes auront finalement le droit de vote en 1944, cela s'explique par la participation des femmes dans la résistance, combler le retard de la France au regard des pays alliés.  

\subsection{Conclusion}

L'universalisme Républicain n'a jamais été neutre, il s'agit d'un universalisme masculin. Il y a donc eu un conditionnement de la citoyenneté par le genre.


\section{La construction genrée des rôles politiques}

\subsection{Le genre des institutions politiques}

En France, les rôles politiques ont étés historiquement définis sur l'exclusion des femmes et ont étés longtemps incarnés uniquement par les hommes. Alain Corbin disait "Jusqu'au coeur du XXe siècle, parler politique, participer à des réunions politiques, s'engager est une affaire d'homme". \\
Dans le métier politique, il y a des exigences qui sont présentées comme universelles mais qui sont en réalité des exigences masculines au sens où ce sont des qualités, des pratiques, qui ont été inculqués comme tel au cours de la socialisation. Ce sont ce qu'on appelle des qualités genrées. Ces qualités sont par exemple de savoir prendre la parole en publique, s'imposer en public, faire preuve d'autorité, ne pas montrer ses émotions. \\
On se rend compte que les institutions politique ont un corps, et pas n'importe lequel. Par exemple, le corps des femmes en politique a été longtemps absent et souvent allégoriques (Marianne par exemple). 


Bertini a travaillé sur la place des femmes dans l'espace public et dans les médias. Elle analyse le discours médiatique contemporain sur le statut des femmes dans la société Française. Elle montre que la place des femmes dans les médias est réduite, les femmes sont sous représentés chez les journalistes mais aussi dans les objets des médias. Un tiers des femmes sont citées sans que l'on précise leur profession, contre un homme sur vingt. Pour 20\% des femmes, on va préciser le lien de parenté. \\
Achin et Dorlin ont identifiées trois figures stéréotypées des femmes en politique. Ces figures ont jouées un rôle de repoussoir des hautes sphères du pouvoir. Il y a la figure de la favorite intrigante (sur érotisé, sexualité débridée), la king: la femme homme et enfin la figure de la matriarche (régente). 


Si on regarde la campagne de Ségolène Royal, elle a occupé des poste traditionnellement féminins (éducation...) et s'est elle même mise en avant en tant que mère. Cependant, ce n'est pas suffisant pour mener à la victoire. Au départ, Royal bat Sarkozy dans tous les sondages, cependant, début 2007, cela se renverse et est concomitant avec un recadrage médiatique qui lui est défavorable, en particulier, la presse insiste beaucoup sur les gaffes de la candidate. \\
On voit un traitement stéréotypé par l'usage du prénom au lieu du nom, l'insistance sur les rôles féminins, l'attention récurrente sur l'apparence et bien sûr les blagues sur les qualités d'oratrices en dépit de la ferveur que suscitait Royal à ce moment là. \\
Le traitement médiatique des femmes politiques est tendanciellement différent de celui des hommes politiques: on les définit souvent à leur statut de femme de, fille de etc. et on va beaucoup plus rapporté les affaires privées des femmes. 


Si on regarde la campagne de Lepen de 2012, toutes ces règles fonctionnent aussi. Elle est qualifiée de "fille du chef", "benjamine" etc. et est souvent qualifiée par rapport à son père. \\
Ses coupes de cheveux, sa voix, son apparence etc. sont sans cesse commentées. 

\section{La représentation politique des femmes}

\subsection{La stagnation de la représentation politique des femmes}

On compte 7\% de femmes députées au début de la IVe République, cela baisse jusqu'à la fin des années 70 pour se stabilisé à 5\% et ensuite remonter petit à petit (surtout en 2002) pour atteindre 27\% en 2012. \\
En Allemagne, les femmes sont beaucoup plus présentes au Parlement qu'en France. 

\subsection{Les obstacles à la représentation féminine politique}

La domination des hommes dans le champs politique s'auto-entretient car le fait de déjà détenir des positions de pouvoir augmente les chances d'en obtenir de nouvelles. \\
Les règles électorales sont un obstacle. Les régimes majoritaire et uninominaux sont plus défavorables aux femmes que les scrutins proportionnels. 

\section{La parité}

La loi sur la parité est induite par les prescription internationales de l'ONU et de l'UE. De plus, il y a une forte mobilisation dans les années 90 en faveur de l'adoption de la parité. 


\chapter{L'âge}

\section{Introduction}

L'âge est quelque chose de très objectif, mais c'est aussi une façade sociale. Derrière l'âge, il y a un espace et une position sociale, c'est ce qui intéresse le sociologue. Derrière le mot jeune par exemple, nous vient directement l'idée d'une certaine précarité par exemple. \\
La mentalité, le caractère, etc. relève de la psychologie et ne nous intéresse donc pas. Il faut dégager les pré-notions qui découlent rapidement de l'âge. \\
Quand on fait de la sociologie, on cherche à croiser des variables et on s'intéresse surtout aux inégalités. On s'intéresse à l'âge comme variable de discrimination au même titre que le genre, l'ethnicité etc. L'âge est aussi une question d'époque, en rapport avec l'Histoire. Être jeune n'est pas la même chose lors de Mai 68 ou de la guerre d'Algérie. \\
On distingue généralement l'effet d'âge, l'effet de conjoncture et l'effet de génération. Parce que les différences d'âge correspondent à des moments dans le cycle de vie, et prédispose à certaines activités ou non. On distingue ainsi des jeunes et des moins jeunes en fonction d'un processus biologique, celui du vieillissement, et les cycles de vie ne nous intéresse que parce qu'ils correspondent à des processus sociaux ou des positions sociales. \\ 
Il existe aussi des générations très claires, et la question que le sociologue se pose sont sur leurs productions, leurs reproductions. Il y a deux manières de s'intéresser à la génération: l'effet de conjoncture, les opportunités qui sont offertes au moment où l'individu atteint un certain âge, comme l'entrée sur le marché du travail, l'entrée dans le mariage, le moment du premier enfant etc. La deuxième manière d'appréhender la génération est l'effet de génération au sens propre, c'est quelque chose de culturel ayant rapport avec la socialisation (génération Mai 68, Casimir etc.), des générations ont été socialisés par des événements politiques particuliers (une guerre, une révolution, un grand champ culturel). \\
Des choses se transmettent de génération en génération, c'est un phénomène de reproduction, et il y a un phénomène de ré-actualisation. \\
Pour penser tout ça, il faut connaître les espaces sociaux et les trajectoires des individus. 

\section{L'âge comme appartenance générationnelle: incorporation et confrontation aux structures socio-historiques}

L'année de naissance d'un individu permet de supposer la vision du monde qu'il peut avoir: quelqu'un né en 1962 n'aura pas la même vision ni la même trajectoire sociale qu'un individu né en 1960, pour la seule raison que 1962 est la fin de la guerre d'Algérie. \\
Il faut bien comprendre que la sociologie ne peut pas prédire la trajectoire, ni le futur des individus. 


Baudelot et Establet ont publié un ouvrage "Avoir 30 ans en 1968 et en 1998". Ils regardent dans quelle état on intègre la marché du travail en 68 et dans quel état on l'intègre 20 ans plus tard. \\
Ils montrent que globalement, il y a un écart de salaire qui s'est beaucoup creusé. On gagne moins bien sa vie en 1998, les diplômes se sont dévalorisés. Il y a donc une précarisation des positions économiques. \\
Les jeunes en 1998, on plus de mal à atteindre des positions monopolisées par les plus anciens car eux même ne grimpent plus les échelons (il y a un bouchon dans l'ascension sociale). \\
Les inégalités viennent du fait que un événement économique ou historique a un impact différent selon la place sociale. Le même événement social n'a pas la même répercussion: l'espace social n'est pas plat.
Il y a un phénomène de massification scolaire (et non de démocratisation). L'espoir des familles ont changés car espèrent une chance d'ascension sociale. \\
À titre scolaire équivalent, à diplôme équivalent, l'entrée sur le marché du travail n'est pas le même et s'est détérioré. 


Louis Chauvel, compare des générations nés de 1920 à 1935 et celles de 1936 à 1950 et celles de 1950 à 1965. \\
Il a plusieurs critères: répartition du pouvoir d'achat, où en 1975, les salariés de 50 ans gagnaient 15\% de plus que les salariés de 30 ans. En 2002, cet écart est de 35\%. \\
Les jeunes sont sur toutes ces générations, les plus touchés par la chômage. Ceci est vrai dans le secteur public comme dans le serveur privé. \\
Le déclassement touche inégalement les individus. \\
Il y a une perception objective et subjective de sa position sociale. 


Karl Manheim va définir de manière sociologique la génération. Il distingue trois manières de l'appréhender. \\
Il y a la situation de génération, une situation qui est identique pour tout le monde à partir du moment où les individus appartiennent à une même classe d'âge: même vision du monde. \\
Il y a l'ensemble générationnel, l'idée est que les membres qui forment cet ensemble partagent des enjeux Historiques qui leur assignent une sorte de destin commun (tous ceux ayant fait un service militaire en Algérie par exemple). Cela n'est pas valable sur l'ensemble des positions sociales, mais c'est bien la confrontation à l'événement qui forment l'ensemble. C'est la perception subjective qui a un rapport sur la trajectoire. \\
Il y a enfin l'unité de génération. Cela a à voir avec l'Histoire et un ensemble d'aspirations sociales, culturelles voir spirituelle (au sens large) qui correspond au milieu social de cette génération. Ce qui fait l'unité de génération, ce sont des individus contemporains qui sont soumis pendant des années de grande réceptivité aux mêmes influences culturelles et qui forment une génération au sens de leur homogénéisation de leurs expériences. \\
Cette situation de faire partie d'une génération, ce sentiment de faire partie d'une génération est nécessaire mais n'est pas suffisante car en fonction des classes sociales, des pays, on aura pas les mêmes expériences. 


La sociologie essaye de comprendre les événements qui font passer d'une génération à une autre. \\
Pour qu'un événement ait cet effet, il faut que cet événement crée une rupture d'intelligibilité dans le monde dans lequel on est. \\
On cherche alors un événement fondateur pour qu'il n'y ait pas seulement une différence d'âge mais aussi une différence de génération. 

\section{L'âge comme position dans le cycle de vie}

L'objectif est de voir comment les cycles de vies se succèdent et entraînent des phénomènes sociologiques. \\
Le texte de Bourdieu, "La jeunesse n'est qu'un mot", permet d'avoir tout de suite en tête que la position dans le cycle de vie est relative: on est toujours le plus vieux ou le plus jeune de quelqu'un. Les positions elles mêmes sont relatives et l'objet de luttes sociales (je suis de gauche car un autre est de droite). \\
Le sociologue doit déconstruire les classes déjà socialement formées car tous les cadres ne sont pas les mêmes cadres, il en va de même pour les jeunes ou les vieux. \\
Bourdieu explique que ces classes socialement formés ne sont que des objets de luttes. \\
Chaque champ est un espace de position et de prise de position dont l'objet de la lutte est la définition de la position et de la concurrence avec les autres.  


Galland dit que la jeunesse comme catégorie n'existe pas de la même manière dans toutes les classes sociales. Cependant, on peut unifier la jeunesse comme une phase moratoire, une sorte d'entre deux, un intermédiaire. La jeunesse, c'est quelque chose entre l'adolescence et l'âge adulte. C'est le moment où on a quitté sa famille mais où on en a pas construit une nous même. \\
Les enfants de catégorie populaire passent plus rapidement de la vie hors famille à la vie en famille car c'est coûteux de vivre seul. \\
Il y a des tendances lourdes qui ont créées cette jeunesse, notamment l'allongement de la scolarité. De fait, cet allongement s'est accompagné d'aspiration à des positions sociales. \\
Tout un tas d'institutions participent à la création de la catégorie jeunesse. Les radios libres par exemple, en 1981 qui ont créés une véritable fracture. \\
Il existe un flou de statut et de positions sociales qui sont propices à des préjugés sociaux. 



\chapter{Des discriminations ethno-raciales}

Les sociologues s'intéresse à la "race" comme un processus de "racialisation" des rapports sociaux. On appréhende donc la "race" comme un groupe construit socialement et on s'intéresse aux processus sociaux par lesquelles des populations quelle qu'elle soit se voit attribué une identité différente, une forme d'altérité sociale, qui peuvent être en partie fondé sur le préjugé de la race. Il y a donc un rapport de domination entre les groupes. \\
On notera qu'il peut y avoir des racismes sans raciste (des perceptions racistes discriminantes sans croire au caractère biologique de la race). La racialisation au quotidien se fait sur la distinction de différences physiques, plus largement, l'esclavage, la colonisation, les phénomènes de migration révèlent une racialisation plus grande des rapports sociaux. 

\section{Racialisation, ethnicisation, discrimination}

\subsection{Des catégories et de leurs usages}

Catégoriser, c'est définir un "nous" et un "eux", cela se fait par une racialisation ou une ethnicisation. \\
La racialisation donne lieu à des catégorisation dans la manière dont on construit les stéréotypes. De fait, un groupe racialisé est défini comme fondé sur l'existence d'un groupe, pratiques communes, caractéristiques physiques similaires. On oppose à la race l'ethnicisation, car la racialisation se base surtout le physique alors que l'ethnicisation voit le groupe comme une culture: histoire commune, géographie commune. \\
Weber a remis en question dans "Économie et société" la vision essentialiste de la société. Pour lui, les groupes ethniques ne se définissent pas par des caractéristiques objectives (histoire, géographie) mais se définissent surtout par une croyance commune en une histoire commune. L'ethnie, comme la race est donc un phénomène relationnel (je ne suis moi que parce qu'il y a des  autres). \\
Barth a écrit "La sociologie des questions raciales: les groupes ethniques et leurs frontières", il s'est intéressé à comment la frontière entre "eux" et "nous" s'est tracé entre les ethnies. La manière dont on trace la frontière est une manière de définir cela. Dans le phénomène d'altérisation (rendre quelqu'un autre que moi), il y a un phénomène d'identification: "c'est parce que j'appartiens à ce groupe que je n'appartiens pas à l'autre". Il y a dans les relations entre ces groupes un enjeu de pouvoir, de domination, d'occupation de l'espace géographique (Israël/Palestine en est un parfait exemple). \\
Les catégorisations sont entretenus par des institutions. \\
La différence supposé entre race et ethnie serait donc qu'il y a une question de biologie, de hiérarchisation morale, d'assignation du côté de la race, alors que du côté de l'ethnie, on insiste bien plus sur la culture, auto identifié. Dans les deux cas il s'agit d'un processus de construction de l'autre dans une relation de pouvoir asymétrique. 


Dans le temps et dans l'espace varie la catégorisation. \\
Wacquant montre que la race aux États-Unis admet plusieurs accessions, le poids, les ancêtres etc. Il s'intéresse aux catégories socio-culturelle aux boxeurs noirs. Il montre que la race est défini sur la base de l'ascendance et comment au Brésil, l'identité raciale est plutôt à la fois les marqueurs physiques mais aussi la position social. Il y a aussi des catégories dans les catégories racialisés.


"Ethno-racial" est une manière de combiner toutes les appréhensions racial et ethniques mais sert surtout à montrer les objets de lutte, de catégorisation dans un contexte historique. \\
C'est aussi une manière de dire que les catégories sont encastrés institutionnellement. Il y a des politiques publiques qui utilisent la racialisation. \\
Cette catégorisation ethno-racial n'existe que parce qu'elle est le fruit de la socialisation: elle s'est transmise de génération en génération.


\subsection{La discrimination comme produit d'une construction sociale}

Le racisme est "un programme d'action consistant à produire de l'altérité dans la société, afin d'alimenter un mécanisme de distinction, de relégation, de stigmatisation et de discrimination" (JF. Schaub). On confond parfois le racisme avec la xénophobie qui est "le rejet de l'altérité qui est défini comme une réalité statique manifeste", l'autre est autre à cause d'un caractère qui ne changera jamais (Schaub). \\
Concernant la stigmatisation, c'est "une hypothèse de fixité naturelle des conditions humaines ou d'une mutation probable dans un sens négatif". \\
Et, concernant la discrimination, c'est "le fait de traiter différemment en raison d'un critère illégitime". Dans le différemment, on entend de manière préjudiciable, cependant on peut faire une discrimination positive (critère illégitime mais traitement avantagé), l'idée étant de corriger quelque chose, de promouvoir, voire éventuellement de désinstitutionnalisé la race.


\subsection{Mesurer les discriminations}

En France, les statistiques ethno-raciales sont interdites contrairement aux États-Unis. \\
Les premières lois anti-discriminations datent de 2001. La HALD a été créée en 2004-2005 puis supprimée en 2011. \\
On a malgré tout des statistiques de l'INSEE, on peut déduire la "race" par le nom ou la langue parlée voire des questions de sondage à la con (qui peuvent être vus comme une fumisterie intellectuel). 


Dispositif du testing: on fait des faux CV et on regarde lesquels sont conservés et qui détient un entretien ou pas. \\
Ce test montre qu'à CV identique, celui qui a un nom "pas français" a trois fois moins de chances, cinq fois moins de chances pour un entretien. Si on croise le lieu d'habitation, on peut obtenir des douze fois moins de chance (quartier nord de Marseille typiquement).


Sur les observations ethnographiques (avec statistique), on a une enquête phare de F. Jobard et P. Levy qui ont enquêté sur les contrôles au faciès. Ils ont suivi la police sur 20 semaines, 525 contrôles, et les statistiques se sont fait suivant la couleur de la peau, du genre, de l'origine, de la tenue, du sac à dos etc. \\
Les résultats montrent que les hommes sont plus contrôlés, encore plus si on a une tenu "jeune" et d'autant plus si on a la capuche, le jean troué, et encore plus si on est noir et/ou maghrébin. 


Lahalle a travaillé sur des mineurs délinquants et montre que la police est beaucoup plus répressive quand il s'agit de jeunes maghrébins et que ceux-ci, à dossier comparé, obtiennent moins d'accompagnement. \\
De plus, les policiers agressés ont plus de chance de se porter partie civile si l'agresseur est soit noir, soit maghrébin.


Salapala a fait une enquête sur le logement social où elle montre que les agents mobilisent de manière routinière des préjugés pour éviter la présence dans les HLM de certains groupes dit "à problème". \\
Jounin a fait une enquête de même type sur les chantiers dans le domaine du bâtiment (il a observé 6 chantiers en 12 mois), et il y a une discrimination à l'encontre des non européens plus forts que contre les est-européens. \\
Chauvel a fait une étude sur des enfants scolarisés et le vécu des discriminations à l'école. Elle conclue que plus de 80\% des personnes âgé de 18 à 35 ans et qui sont sortis diplômés considèrent avoir été traités de manière inégalitaire à l'école. 

\section{Penser l'articulation des inégalités: un racisme institutionnel}

Quelle est le rôle de l'État dans la question de la racialisation ? \\
Le concept de racialisation institutionnelle est né au États-Unis au moment des luttes pour les droits civiques, et ont mis l'accent sur le fait que le racisme institutionnel trouvait sa force dans le fonctionnement même des institutions. On emploi le racisme institutionnel quand "en dehors de toute intention manifeste et directe de nuire à certains groupes ethnique, les institutions ou les acteurs au sein des institutions développent des pratiques dont l'effet est d'exclure ou d'inférioriser les groupes".

\subsection{École et (re)production des inégalités}

Au delà du sentiment d'inégalité scolaire, l'école est une institution de reproduction des inégalités sociales. Elle est aussi une institution de reproduction des inégalités ethno-raciales: l'école discrimine pour des raisons culturelles car plusieurs effets se combinent à l'intérieur de l'institution scolaire et certains groupes ethniques ont un rapport différent à l'accompagnement des études de leurs enfants, la capacité d'accompagnement n'étant pas la même partout (langue, communauté d'aide aux devoirs, effet prof etc). \\
De fait, l'école devient un instrument de racialisation et de stigmatisation.


Contrairement à ce qu'on pourrait penser instinctivement, les premières générations d'immigrés placent dans l'univers scolaire énormément d'espoir d'intégration pour leurs enfants. À inspiration supérieure et à accompagnement équivalent, les élèves racialisés ont de moins bons résultats, sont moins bien orientés que les élèves qui ne font pas l'objet d'une racialisation. \\
Les enfants d'immigrés sont donc particulièrement sensible à l'échec scolaire. 1 enfant sur 3 dont les parents étaient immigrés a redoublé à l'élémentaire, contre seulement 1 sur 5 pour les enfants dont un seul parent est immigré. Les écarts de réussite sont donc très inégaux. Un enfant racialisé qui réussit à l'école est donc sur-sélectionné (ce que le système scolaire a reconnu malgré lui).

\subsection{Marché du travail et conditions d'emploi}

Il y a trois caractéristiques propres à l'étude de l'emploi: l'emploi en lui même, la trajectoire professionnelle (mobile, ascendante, précaire), les conditions de travail. \\
Pour les enfants d'immigrés, les inégalités sont avérés, le chômage est plus fort en cause de la discrimination scolaire et à l'embauche, le niveau de précarité est supérieur, la carrière moins ascendante. \\
La seule manière d'inverser ces variables, c'est de faire partie d'une communauté ou d'un groupe dont l'intégration est extrêmement forte et qui possède un fort capital culturel (c'est notamment le cas des Portugais, des Asiatiques et des Turques), qui créent des communautés de travail palliant à ces inégalités. \\
L'inégalité de l'emploi est à combiné avec l'inégalité scolaire, mais aussi avec l'inégalité devant la justice qui conduit à un casier judiciaire et donc une difficulté d'accès à l'emploi.


L'organisation du travail est donc ethnicisé et les relations au travail sont racialisés. Cela s'oppose à plus ou moins de résistance. \\
Il existe aussi une inégalité de revenus, lié à l'inégalité de l'emploi (plus de 46\% d'écart dans le revenu des ménages). \\


\subsection{Logement et mobilité résidentielle}

12\% des blancs veulent rester dans des quartiers blancs, alors que 3\% des noirs ont envie de rester dans des quartiers ségrégués. Le fait d'habiter dans des territoires ségrégués est contraint par le revenu donc. \\
Les candidats supposément non français ont 50\% de chances de moins d'obtenir un rendez-vous pour un prêt bancaire, les propriétaires louent très peu aux populations racialisés qui sont redirigés de fait vers les logements sociaux. 

\subsection{Accès à la sphère civique: citoyenneté, police, justice, santé...}

Sur la sphère civique, c'est là qu'est le plus visible la racialisation. \\
Les soins sont de moindre qualité, le niveau d'attente est plus important, le suivi médical moins fait. \\
Des enquêtes qu'on essaye de corréler montrent que tout critère confondu, quand le niveau de discours racialisé augmente, l'ensemble des inégalités ethno-raciales augmente. 

\chapter{Les classes sociales dans l'espace physique}

Depuis 2008, plus de 50\% de la population mondiale vit en ville et cela s'accroît, surtout dans les pays en voie de développement qui s'urbanise très rapidement. \\
En France, une cinquantaine de villes réunissent plus de 100 000 habitants. Plus de 85\% des français vivent dans des zones urbaines. \\
Habiter en ville a des conséquences sur l'emploi, les modes de vie etc. Ce qui nous intéresse, ce sont les problématiques que la ville engage et son impact sur les rapports sociaux. \\
La sociologie urbaine a été fondée par la première école de Chicago qui a travaillé sur l'urbanisme, la question des transports, le maillage des commerces, les cultures etc. C'est une sociologie de la ville et des pratiques sociales qui vont avec le fait urbain. \\
R.Park a piloté la première étude paru en 1925: "The City" et qui est un article fondateur car il est clairement proposé de travaillé sur la ville comme dans un laboratoire social. L'impact de ces premiers travaux a porté sur d'autres scientifiques qui se sont posé d'autres questions (qualité/nature de l'habitat, effet des politiques publiques, mouvement des personnes etc.).


\section{La ségrégation urbaine: définitions et mécanismes}

Quand les gens arrivent dans les villes, ils ne s'installent pas n'importe où, mais en plus, ils ne s'installent pas par choix mais par contrainte, et, dans ces contraintes, on peut lire des formes de ségrégation. \\


\subsection{Qu'est-ce que la ville ?}

L. Wirth a aussi produit un article fondateur, en 1938, où le phénomène urbain est décrit comme un mode de vie et du coup, donne une définition de la ville. Il dit "La ville peut être défini comme un établissement relativement important, dense, et permanent d'individus socialement hétérogène". \\
Il trouve trois grandes tendances qui peut qualifier le fait urbaine: la concentration (densité) de population sur un territoire ; une hétérogénéité de population et des mobilités ; la fragmentation urbaine, comme une mosaïque tant dans l'espace que dans le social (frontière symbolique). Pour ceux qui vivent en ville, c'est un cadre de (re)constitution des groupes sociaux dans lequel les individus s'inscrivent. 

\subsection{Qu'est-ce que la ségrégation ?}

C'est une notion polysémique. \\
Pour les sociologues urbains, la ségrégation est la distance spatiale qui existe entre les groupes sociaux qui composent la ville. La ségrégation désigne alors toutes différences de localisation résidentielle, ces groupes sociaux étant défini par la position sociale, l'origine ethnique, le niveau de revenus. Ces différences sont marqués sur le territoire. \\
La tâche du sociologue est de mesurer cette ségrégation, on appelle cela une mesure de dissimilarité. On mesure le degré de concentration ou de dissémination d'un groupe donné sur un territoire. On crée un indice de dissimilarité dont on déduit l'indice de ségrégation. \\
Beaucoup de sociologues utilisent cette approche qui n'explique pas les causes de la ségrégation. À ce stade des recherches, rien ne permet de voir si la ségrégation est volontaire ou non. 


On peut aussi définir la ségrégation urbaine comme une inégalité d'accès aux espaces résidentiels. Quand on aborde le fait ségrégatif sous cet angle, on essaye de voir non pas tant à quel endroit atterrissent les individus mais si ils bougent, vers où, dans quelles conditions, sous quelles contraintes. \\
On étudie alors les parcours résidentiels.


On peut aussi voir la ségrégation comme une relégation dans une enclave. On regarde où les gens sont positionnés et on s'intéresse à ceux qui n'ont aucune mobilité et on s'intéresse à ceux qui ne bougent pas, et pourquoi. Souvent ils sont relégués à cet endroit et on parle alors de ghettos. \\
Le ghetto fait son apparition dans les travaux sociologiques américaines relativement tôt. Ce terme est importé en France dans les années 80-90. Il est défini comme "une forme de regroupement spatial associant étroitement des populations défavorisées à des territoires circonscrits". \\
E. Préteceille passe sa vie à essayer de démonter cette définition du ghetto. 

\subsection{Les mécanismes de la ségrégation}

La ségrégation peut résulter d'inégalités économiques. Dans ce cas, on regarde le niveau de patrimoine des individus, leurs salaires, le prix de l'immobilier et on voit qu'en fonction des quartiers, les prix ne sont pas les mêmes. En fonction du niveau de revenu et du patrimoine, on ne peut pas habiter partout. Une partie du fait ségrégatif urbain est directement causé par cette inégalité. \\
De plus, les activités économiques se distribuent inégalement, les emplois de l'industrie lourde vont être en périphérie, les petits commerces en centre-ville etc. 


La deuxième manière d'appréhender les causes est d'observer si il y a une volonté de ségrégation de certains acteurs sociaux vis à vis d'autres acteurs. \\
Lorsqu'elle est volontaire, c'est une agrégation de petites décisions qui conduisent à la ségrégation. Le fait ségrégatif vient du fait que les groupes sociaux ont une forme d'exigence de ne pas se retrouver en minorité. Cette exigence fait que les individus développent des stratégies résidentielles particulières. 


La troisième cause de ségrégation urbaine est la ségrégation explicite (apartheid). N. Elias dans "Logiques de l'exclusion", a, avec Scotson, montré qu'un groupe donné, en majorité sur son territoire spatial, le domine et empêche les autres groupes d'y rentrer. Elias appelle cela le "conflit entre les établis et les marginaux". \\
Elias ne parle pas de classes sociales mais seulement de perceptions individuelles. 


\subsection{Les effets des politiques publiques de peuplement}

Les effets des politiques publiques sont à prendre en compte, car elles peuvent essayer de réduire cette ségrégation. La construction des logements sociaux en est un parfait exemple. Les maires de villes sont obligés de construire des logements sociaux.


\section{Les théories de la dualisation: approches comparées}

La théorie de la dualisation est apparu aux USA, elle a été énoncée par Castells et Mollenkopf. Les espaces se divisent, se transforment en fonction du fait économique et que, du coup, on a deux types de dualisation, une sociale et une spatiale. La dualisation sociale, on observe une augmentation des cadres diplômés et à côté, une augmentation du "prolétariat". La dualisation spatiale où les activités se concentreraient en certains points augmentant alors les prix fonciers et, inversement, des territoires relégués avec un faible prix foncier. \\
Les sociologues se sont vites penchés sur la question des ghettos aux États-Unis.

\subsection{Ghettos et "Underclass" aux USA}

L. Wacquant a retracé l'histoire des ghettos pour essayer de définir ce qu'est un ghetto et ce qui ne l'est pas. Il montre que les ghettos noirs n'ont pas les mêmes caractéristiques les autres ghettos. La particularité des ghettos noirs est qu'ils se sont fait par relégation et non par choix d'une ethnie de vivre ensemble (contrairement à de nombreux quartiers aux USA). \\
Pour définir et différencier le ghetto noir des autres ghettos, il parle de ghetto institutionnel dans le sens où l'État a joué un rôle actif pour inciter les blancs à aller habiter en banlieue (subventions fédérale). La politique de "zonage" était donc très clair et assumée, dans les années 20 notamment et ont interdit, avec les prix du foncier, à certaines populations, de s'installer dans certaines zones. \\
De fait, dans ce contexte, le ghetto désigne bien des enclaves denses, relégués, où s'accumulent par nécessité et par contraintes les afro-américains au moment même où les classes moyennes américaines accèdent à des niveaux de revenus permettant d'aller dans des banlieues. \\
Wacquant nous dit que les ghettos sont synonymes d'exploitation économique et d'ostracisation sociale. Ce sont aussi des espaces d'ascension sociale, de solidarité, de construction d'identité sociale.


Si on doit caractérisé le ghetto comme concept, on est bien sur une enclave qui contient une population ciblé, mais on peut y voir des institutions duplicatives. Le ghetto est donc bouclé sur lui même, il a donc besoin de ses propres institutions et devient une ville dans la ville. \\
Les autorités sont d'autant plus d'accord que ces institutions empêchent les populations d'y sortir. \\
Il y a une manière d'être heureux dans un ghetto mais c'est aussi un espace sécurisant, d'intégration. \\
Le ghetto est donc biface. 


D. Fassin a travaillé sur l'exclusion sociale et la question de l'underclass aux USA en comparant avec la France. \\
Le débat sur l'underclass s'est ouvert dans les années 70 dans le champ médiatique américain (1974, choc pétrolier). Aux USA, il y a toute une campagne de presse américaine conservatrice qui décrivent les ghettos noirs comme des lieux paupérisés où se développeraient aussi des violences, des naissances hors mariages, etc. \\
Les premiers articles mettent les causes sur des phénomènes psychologiques. À ce moment là, on attribue des caractéristiques psychologiques particulières aux populations paupérisées. \\
Les sociologues vont vite s'inscrire en faux contre ces explications et vont essayer de comprendre ces phénomènes. \\
La pauvreté était d'abord un phénomène rural, qui concernait surtout les personnes âgées. On passe donc d'une pauvreté rurale à une pauvreté urbaine qui touche plus particulièrement les jeunes mais aussi les salariés. Il y a de fait, un phénomène qui fait que dans les lieux de relégation, on peut observer une anomie sociale. À la pauvreté s'ajouterait l'anomie sociale. \\
Contrairement à la nouvelle pauvreté, la pauvreté était cyclique, relativement résiduelle (on se "refaisait" d'une année sur l'autre), et surtout, elle était circonscrite à la classe ouvrière. \\
Aujourd'hui, la pauvreté semble plus permanente et se reproduit d'une génération sur l'autre. \\
De fait, à Chicago par exemple, 16\% des habitants d'un ghetto (en 1996) ont un emploi rémunérés, et la moitié vit en dessous du seuil de pauvreté. Le pire, dit Fassin, c'est que s'il y a une offre d'emploi de disponible, elle ne reviendra pas dans ce ghetto, qui est alors un lieu non pas seulement discriminé mais aussi discriminant. 


\subsection{Une dualisation à la française}

On s'interroge en France sur, notamment, les politiques de la ville et globalement sur les politiques publiques de lutte contre les inégalités territoriales. \\
J. Donzelot a reprit dans un article de 2004 la question de la dualisation en France. Il parle de ville à trois vitesses pour essayer de rendre compte d'évolutions relativement récente. Son hypothèse, c'est qu'il y a une forme de rupture qualifiable à partir d'un triple mouvement de séparation entre les groupes sociaux et qui se combine. \\
Le premier mouvement est le mouvement d'embourgeoisement des centre-villes prestigieux. Le deuxième est le départ des classes moyennes vers le péri-urbain moins coûteux et protégés. Le troisième est la relégation des cités d'habitats social. 


Si on regarde les classes supérieures, qui disposent le plus de latitude en matière de choix, elles se concentrent de plus en plus sur les centre-villes rénovés. \\
Sur les quartiers bourgeois, la tendance est ancienne mais l'arrivée des classes à fort capitaux dans les centre-ville est plus récente. C'est un phénomène de "gentrification" pour montrer comment les classes moyennes supérieures s'approprient des espaces urbains, les rénove, les requalifie et font donc augmenter le prix du foncier changeant la population qui vient s'installer (en somme c'est un changement sociologique d'un quartier). \\
Les gentrificateurs sont plutôt à fort capital culturel, ils ont un capital nouveau qui ne leur a pas été transmis.


Pour les classes moyennes, elles sont chassées des centre-ville par la hausse des prix du loyer et son chassés vers les extérieurs de la ville dans des habitats pavillonnaires. En urbanisme, on appelle ça le mitage pavillonnaire. \\
Pour décrire ce phénomène, Donzelot utilise le terme de sécession. 


Pour les ménages précarisés sont relégués dans des banlieues à l'extérieur de la ville et qui sont symboliquement fortement dévalorisées auquel on a attribué des zones d'anomie sociale. Certains appellent ces zones les territoires perdus de la République. Cependant, ce ne sont pas des ghettos car il n'y a pas d'institutions duplicatives. 


Lapeyronnie a écrit en 2008 et considère qu'il y a bien un effet ghetto dans les quartiers et qu'il est à entendre de deux manières. À la fois, il serait défini par l'extérieur, et une vision de l'intérieur qui considère que leur quartier est devenu, au fil du temps, un ghetto. C'est donc un étiquetage au sens de Becker. \\
Pour Lapeyronnie, le ghetto existe dès qu'on le considère comme une marge. Ce qu'il constate, c'est que, dans ces quartiers, il y a une culture de la consommation de masse qui finalement, attise les déceptions, les frustrations. De plus, ces quartiers ont souvent été populaires, donc dans des emplois industriels et créent une différence entre ceux qui ont un emploi et ceux qui n'ont en pas. \\
Ceux ayant un emploi regrette que le quartier soit devenu ghetto, les autres estimes qu'il n'y avait rien d'autres à faire. \\
Il y a un sentiment partagé de distance par rapport aux institutions républicaines. Le ghetto n'est pas un endroit anomique. Les habitants de ghettos reconnaissent l'État mais considère qu'il n'est pas pour eux. Une fois qu'on est stigmatisé, soit l'individu se sent stigmatisé soit il retourne le stigmate en le transformant en fierté. Le ghetto est donc un contre monde. Dans tous les cas, ces "ghettos urbains" sont des formes paroxystiques. 


\subsection{F/USA: des constellations socio-spatiales distinctes}

Préteceille essaye d'appliquer les modèles américains au monde social Français. Il se demande si la ségrégation sociale a augmenté à Paris. Il répondra à la fois par l'affirmative et la négative. \\
Il dit que les plus ségrégués des classes populaires, ce sont les ouvriers alors que globalement les employés et professions intermédiaires s'en sortent mieux. En réalité, la catégorie la plus ségrégué sur un territoire, ce sont les classes de la très grande bourgeoisie. \\
Il se demande si la ségrégation ethno-raciale existe en France comme aux USA. En France, la ségrégation ethno-raciale a effectivement augmenté depuis les années 80 mais restent faibles. Les plus ségrégués sont les immigrés d'origine maghrébines, subsaharienne et Turque.


Le premier indice de morphologie est la taille. Le ghetto de Chicago a 400 000 habitants, plusieurs centaines de kilomètres carrés. Le quartier noir de New York, c'est plus d'un million d'habitants. \\
En France, les quartiers les plus massifs au sens des indices de ségrégation, sont de l'ordre de un dixième par rapport aux ghettos américains. \\
Le deuxième indice est la structure. En France, les quartiers sont ouverts, restent mélangés, les institutions ne sont pas dupliqués. Aux USA, les ghettos sont relativement unifiés. \\
Un troisième indice est la mobilité, et, en France, on sort des quartiers ségrégués. \\
Donc, oui, il existe en France des quartiers discriminés, mais ce ne sont pas des espaces clos, ce ne sont pas des ghettos.


Un autre indicateur est la violence, on fait le présupposé de la violence dans les quartiers ségrégués français. \\
Il existe certes des phénomènes des violences, mais ils n'ont rien à voir avec les faits américains. \\
Un autre indicateur est le rôle de l'État. Aux USA, l'État n'est pas aussi présent que dans les quartiers français. Au pire, dans certains quartiers, l'État recule. Les USA n'investissent pas dans les ghettos contrairement à la France. 


\section{Des différents types de ségrégations en France}

Quand on parle de ségrégations en France, on parle de classe sociale car elle existe en France et donc on l'enquête. Mais aussi parce qu'il y a une influence marxiste. Cela fait que dans la sociologie européenne, l'espace urbain est le produit historique des rapports de classes. \\

\subsection{Les "beaux quartiers": l'entre soi réservé aux classes supérieures}

En France, les quartiers ségrégués ne sont pas les plus discriminés. Si on regarde comment les individus bougent dans un quartier, on se rend compte que la classe la plus enfermé sur elle même est la grande bourgeoisie (chef d'entreprises, cadres supérieurs à fort niveau de patrimoine, profession des arts et du spectacle, hautes fonctions publiques). \\
Le XVIe, le VIIIe, sont par exemple des arrondissements d'entre soi de grande bourgeoisie. Il y a sur ces territoires très peu de mouvement pendulaire. La construction de Bercy à l'Est a provoqué un mécontentement des hauts fonctionnaires habitant dans l'ouest, le transport fluvial a donc été remis en fonction car ils ne prennent pas le métro ni le bus. \\
Les mobilités professionnels, à part cette exception, sont très rares. \\
Les Pinçon-Charlot ont écrit "Les ghettos du Gotha" consacrés à la grande bourgeoisie. Ils ont passés toute une carrière auprès de la grande bourgeoisie et montrent donc comment, de génération en génération, ils restent sur les mêmes espaces et comment, cette appropriation du territoire est un élément de se maintenir en haut de la hiérarchie sociale car la haute bourgeoisie a les moyens de clore des espaces.


Concernant la stratégie matrimoniale, elle n'est pas spécialement contrôlée dans les classes populaires alors que dans la haute bourgeoisie, les mariages sont arrangés, ou presque. On explique aux femmes quels sont les bons partis. Des rallyes sont organisés pour se faire rencontrer les individus. C'est une manière de clore les groupes sociaux. \\
C'est encore plus vrai dans la noblesse dans le sens strict du terme où les nobles se marient toujours entre eux. \\
Les Pinçon-Charlot montrent qu'en partant du Louvre, on peut dessiner un axe du pouvoir qui passe par l'arc de triomphe et l'arche de la défense. \\
De fait, il existe une stratégie de la haute bourgeoisie pour annexer les espaces et les territoires pour asseoir leur position de pouvoir. 

\subsection{Les quartiers populaires: une cohabitation préservée ?}

La deuxième classe la plus ségrégué est la classe ouvrière pour des raisons inverses aux classes bourgeoises. \\
Les quartiers populaires sont des quartiers très ancrés dans l'histoire des territoires. Les populations des quartiers ouvriers sont liés aux mutations économiques. Le nombre d'ouvriers a fortement décru du fait de la désindustrialisation et la mondialisation. Une des tendances lourdes qui a transformé la classe populaire ouvrière, c'est la tertiarisation de l'économie avec des trajectoires différenciés. \\
Une partie des quartiers populaires ouvriers sont devenus précaire et paupérisés car n'ont pas pu suivre le mouvement économique. C'est dans lequel se concentre les plus précaires et les plus instables professionnellement. \\
Dans les quartiers les plus précarisés, seuls 52\% des chômeurs ouvriers y habitent. Il n'y pas d'uniformité dans les quartiers précarisés. La majorité des ouvriers habitent dans d'autres types d'espace que dans des "ghettos d'ouvriers". 44\% des chômeurs ouvriers habitent dans des espaces socialement mixte (avec classes moyennes). \\
Plus de la moitié des ouvriers habitent donc sur des territoires mixtes.


Ces quartiers ont une image mixte, personne ne les qualifierait de ghetto. On quitte ces quartiers que lorsque l'on change de classes sociales. L'évolution et la situation immobilière des classes populaires est donc très diversifié contrairement aux classes de haute bourgeoisie. \\
De fait, la relégation dans des territoires tout à fait appauvris est assez faibles. \\
Il existe très peu de quartiers sinistrés.

\subsection{Les classes moyennes périurbaines: séparatisme social ou déstabilisation ?}

Les classes moyennes ont une place centrale dans le travail de Danzelot. \\
Elles peuvent être chassées du centre ville par la rénovation de ceux-ci et vont donc allez dans les zones périurbaines. Cela fait bouger les frontières des villes. \\
Les classes moyennes font-elle exprès de se différenciés (à la fois des classes paupérisées et aussi de la haute bourgeoisie) ou est-ce une paupérisation à venir ? \\
Le pavillon de banlieue, permet de se différencié des plus pauvre au sens où c'est une maison, pas un appartement, sans être un logement de la haute. Il y a dans les zones périurbaines des techniciens, des enseignants, des professionnels de la santé etc. Ce sont des espaces très hétérogène car, de fait, ils sont près de quartiers bien plus ouvriers. \\
Il y a des pratiques différenciés dans les zones périurbaines, notamment les déplacements pendulaires en voiture ou en RER. Ces déplacements permettent de connaître comment les classes sociales utilisent l'espace. À chaque fois qu'il y a une mobilisation sur un transport, il va se jouer la frontière des classes sociales sur la gestion de l'espace. Les tracés sont l'enjeu de luttes.

\section{Conclusion}

Sur l'ensemble des phénomènes qui sont de ségrégations, ce qui est en creux comme question, c'est la question de la mixité sociale, de la mixité urbaine. C'est un enjeux fort des classes. \\
Toutes les politiques publiques de mixités n'engendrent pas nécessairement un rapprochement entre les classes sociales. Elles accentuent même parfois les tensions. \\
Pour repérer ce débat, il suffit de repérer le débat autour de la délinquance qui a toujours été la manière d'appréhendé la mixité. 







\end{document}
