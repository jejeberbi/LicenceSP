\documentclass[12pt, a4paper, openany]{book}

\usepackage[latin1]{inputenc}
\usepackage[T1]{fontenc}
\usepackage[francais]{babel}
\date{}
\title{Cours de Sociologie (UFR Amiens)}
\pagestyle{plain}

\begin{document}

\part{Les inégalités de genre}

\chapter{Qu'est-ce que le genre ?}

Le concept de genre est moins institutionnalisé en France que dans de nombreux autres pays. Cependant, le concept est présent et acclimaté en France. \\
Le genre peut être compris comme la construction sociale et historique des différences entre femmes et hommes. C'est un système de catégorisation hiérarchisé entre les sexes et entre les valeurs et les représentations qui leurs sont associées. \\ 
Le concept de genre se fonde sur quatre éléments:
\begin{itemize}
\item Perspective constructiviste ;
\item Approche relationnelle ;
\item Considère les relations sociales entre les sexes comme un rapport de pouvoir (étude de la domination) ;
\item Imbriqué dans d'autres rapports de pouvoir, d'autres rapports sociaux.
\end{itemize}

\section{Construction sociale du genre: différence entre femmes et hommes}

\subsection{Simone de Beauvoir comme précurseur}

Elle publie "Le deuxième sexe" en 1949, elle écrit "On ne naît pas femme, on le devient". Elle dissocie donc le sexe biologique d'un sexe social. \\
Ce texte est important car c'est la première fois que théoriquement le genre est déconnecté du sexe, même si le terme de genre (qui apparaît dans les années 70) n'apparaît pas ici. \\
"Tout le monde s'accorde à reconnaître qu'il y a dans l'espèce humaine des femelles. Elles constituent aujourd'hui comme autrefois à peu près la moitié de l'humanité ; et pourtant, on nous dit que la féminité est en péril, on nous exhorte: soyez femmes, restez femmes, devenez femmes. Cette exhortation permet de tout de suite posée qu'être une femelle, ce n'est pas nécessairement être une femme." \\
Selon Beauvoir, les femmes sont les seules à se définir comme étant femme, contrairement aux hommes qui seraient "positif et neutre" donc universel. Cette subjectivité serait le produit de la nature et le corps féminin est défini par le manque. Le sexe féminin serait donc relatif à l'homme "Il est le sujet, il est l'absolu, elle est l'autre". \\
C'est cette argumentaire et cette façon de penser qui ont déconnecté les femmes du droit de vote.

\subsection{Margaret Mead}

Elle publie "Moeurs et sexualité en Océanie" en 1930. Elle pense les rôles sexuels comme des rôles sociaux qui ne découlent pas des sexes biologiques mais sont diversement construit selon les sociétés. \\
Elle travaille en Papouasie où il y a de nombreuse ethnies, peu distante mais très différentes dans leurs organisations sociales. \\
Elle s'intéresse au tempérament, un ensemble de traits de caractère. Elle travaille chez les Arapesh où il y a un tempérament homme comme femme "doux et sensible". Cependant, dans un groupe peu distant, les Mundugumor, le tempérament des hommes et des femmes est "violent et agressif". Chez les Chambulu, les hommes "doux et sensible" sont mariés à des femmes "violentes". \\
La variabilité des rôles montrent que ces rôles ne sont pas déterminés par le sexe biologique mais par l'organisation sociale de la société dans laquelle on se situe. \\
Cette variabilité est donc mis en exergue comme une construction sociale mais elle ne pense pas les relations de domination. 

\subsection{L'émergence du genre comme notion critique}

La notion de genre apparaît aux États-Unis et en Angleterre, mais c'est une sociologue anglaise du nom de Anne Oakley qui publie un ouvrage en 1972 qui s'appelle "Sex, gender et society" qui posera clairement le concept de genre comme sexe social. Le genre regroupe les aptitudes, les attitudes, les tâches considérées comme masculine ou féminine, socialement déterminées et donc variable. \\
Ce qui est intéressant dans la perspective de Oakley, c'est qu'elle pense ce qui est variable et permet donc de remettre en question ces rôles, ces différences. 


Irving Goffman, "L'arrangement des sexes", publié en 1977, nous dit que les différences biologiques entre homme et femme sont très peu nombreuse et donc pas si importante que ça. \\ 
Il dit "Pour que ces faits matériels de la vie n'ait pas d'appréciable conséquence sociale, il suffirait d'un peu d'organisation, mais relativement peu, selon les normes modernes". Ce qui intéresse donc Goffman, c'est comment ces deux classes peuvent se maintenir et se reproduire sachant que ces différences ne sont pas importantes dans la société actuelle. \\
Il va chercher comment, dès le "triage initial", ces deux classes sexuelles peuvent se maintenir. Pour lui, dans nos sociétés, la première catégorisation se fait à la naissance où l'on trie par le sexe: l'attribution du sexe est la catégorisation première des individus dans la société. Il va donc s'intéressé aux effets de ce classement et donc les différences de traitement entre filles et garçons: les agencements de ségrégation sexuée. \\ 
Goffman parle de ségrégation périodique entre les sexes. "Plus qu'un produit de la différence biologique entre les classes, il s'agit d'une manière de la produire". 


La notion de genre permet d'appréhender les rôles sociaux en dénaturalisant, de déconnecter les rôles sociaux du biologique. Cela permet donc de comprendre les différences qui sont socialement construite. Cela va donc remettre en question les caractéristiques des hommes et des femmes: la notion va à l'encontre de la vision essentialiste qui donne des caractéristiques immuables à la caractéristique biologique. \\
C'est cette portée critique qui fait que la notion de genre a de nombreux ennemis. 

\subsection{Approche relationnelle}

Les caractéristiques qui sont associées à chaque sexe sont construites socialement dans une relation d'opposition. Ils sont le produit d'un rapport social. On ne peut donc pas étudier un groupe de sexe sans le rapporter à l'autre. \\
Goffman a écrit un texte "La ritualisation de la féminité" dans lequel il analyse le traitement de la femme dans les magazines. Ce qu'il montre dans ses analyses, c'est où en est la ritualisation de la féminité par rapport à l'homme: les femmes sont souvent représentées comme subalterne, docile, infantile, inférieure à l'Homme.

\section{Le genre comme rapport social asymétrique, comme rapport de pouvoir}

Quand on se situe dans la perspective des rapports de pouvoir, on ne va pas se contenter de montrer qu'il y a une différence de construite socialement, mais comment les différences conduisent à des inégalités. La différence entre homme et femme a été culturellement associée à une hiérarchisation. Même si les rapports de pouvoir sont multiforme et très variable d'une société à l'autre, la quasi totalité des sociétés étudiées présentent une distribution inégale des ressources au profit des hommes au détriment des femmes ainsi qu'une valorisation systématique du masculin au détriment du féminin. \\
Les rôles sociaux féminin et masculin sont donc associés à des valeurs et à une hiérarchie dans l'espace social. \\
Joan Scott a écrit "Gender, a usefull categorie of historical analysis", elle dit "Le genre est un élément constitutif des rapports sociaux fondés sur des différences perçus entre les sexes, et le genre est une façon première de définir des rapports de pouvoir". Elle repère quatre éléments constitutif du genre:
\begin{itemize}
\item Les symboles culturellement disponible ;
\item Les concepts normatifs (dans le droit, la religion, la psychanalyse etc.)
\item Les institutions, l'organisation sociale (organisation de la parenté, marché du travail etc.) ;
\item L'identité subjective. 
\end{itemize}

\section{L'articulation des différents rapports de pouvoir}

Il ne faut pas analyser les rapports de genre indépendamment des autres rapports de pouvoir. Les catégories de sexe ne sont pas homogènes. \\
S. Truth, ancienne esclave, déclarait lors de son discours "Ain't not a woman ?" à la convention des droits des femmes en 1851 aux États-Unis "Cet homme là bas dit que l'on doit aider les femmes à monter dans une calèche[...]" \\
Bourvois "En quête de respect", une enquête sur des dealers dans un ghetto américain dit qu'au départ il parait efféminé à cause de sa manière de parler etc. et est catalogué comme non viril donc gay. Il montre donc qu'un certain nombre de dealer, qui cherche à être très viril, ne peuvent pas se reconvertir dans des emplois légaux car sont justement trop attachés à leur masculinité.



\end{document}
