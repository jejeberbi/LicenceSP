\documentclass[12pt, a4paper, openany]{book}

\usepackage[latin1]{inputenc}
\usepackage[T1]{fontenc}
\usepackage[francais]{babel}
\date{}
\title{Cours de Sociologie (UFR Amiens)}
\pagestyle{plain}

\begin{document}

\chapter{L'âge}

\section{Introduction}

L'âge est quelque chose de très objectif, mais c'est aussi une façade sociale. Derrière l'âge, il y a un espace et une position sociale, c'est ce qui intéresse le sociologue. Derrière le mot jeune par exemple, nous vient directement l'idée d'une certaine précarité par exemple. \\
La mentalité, le caractère, etc. relève de la psychologie et ne nous intéresse donc pas. Il faut dégager les pré-notions qui découlent rapidement de l'âge. \\
Quand on fait de la sociologie, on cherche à croiser des variables et on s'intéresse surtout aux inégalités. On s'intéresse à l'âge comme variable de discrimination au même titre que le genre, l'ethnicité etc. L'âge est aussi une question d'époque, en rapport avec l'Histoire. Être jeune n'est pas la même chose lors de Mai 68 ou de la guerre d'Algérie. \\
On distingue généralement l'effet d'âge, l'effet de conjoncture et l'effet de génération. Parce que les différences d'âge correspondent à des moments dans le cycle de vie, et prédispose à certaines activités ou non. On distingue ainsi des jeunes et des moins jeunes en fonction d'un processus biologique, celui du vieillissement, et les cycles de vie ne nous intéresse que parce qu'ils correspondent à des processus sociaux ou des positions sociales. \\ 
Il existe aussi des générations très claires, et la question que le sociologue se pose sont sur leurs productions, leurs reproductions. Il y a deux manières de s'intéresser à la génération: l'effet de conjoncture, les opportunités qui sont offertes au moment où l'individu atteint un certain âge, comme l'entrée sur le marché du travail, l'entrée dans le mariage, le moment du premier enfant etc. La deuxième manière d'appréhender la génération est l'effet de génération au sens propre, c'est quelque chose de culturel ayant rapport avec la socialisation (génération Mai 68, Casimir etc.), des générations ont été socialisés par des événements politiques particuliers (une guerre, une révolution, un grand champ culturel). \\
Des choses se transmettent de génération en génération, c'est un phénomène de reproduction, et il y a un phénomène de ré-actualisation. \\
Pour penser tout ça, il faut connaître les espaces sociaux et les trajectoires des individus. 

\section{L'âge comme appartenance générationnelle: incorporation et confrontation aux structures socio-historiques}

L'année de naissance d'un individu permet de supposer la vision du monde qu'il peut avoir: quelqu'un né en 1962 n'aura pas la même vision ni la même trajectoire sociale qu'un individu né en 1960, pour la seule raison que 1962 est la fin de la guerre d'Algérie. \\
Il faut bien comprendre que la sociologie ne peut pas prédire la trajectoire, ni le futur des individus. 


Baudelot et Establet ont publié un ouvrage "Avoir 30 ans en 1968 et en 1998". Ils regardent dans quelle état on intègre la marché du travail en 68 et dans quel état on l'intègre 20 ans plus tard. \\
Ils montrent que globalement, il y a un écart de salaire qui s'est beaucoup creusé. On gagne moins bien sa vie en 1998, les diplômes se sont dévalorisés. Il y a donc une précarisation des positions économiques. \\
Les jeunes en 1998, on plus de mal à atteindre des positions monopolisées par les plus anciens car eux même ne grimpent plus les échelons (il y a un bouchon dans l'ascension sociale). \\
Les inégalités viennent du fait que un événement économique ou historique a un impact différent selon la place sociale. Le même événement social n'a pas la même répercussion: l'espace social n'est pas plat.
Il y a un phénomène de massification scolaire (et non de démocratisation). L'espoir des familles ont changés car espèrent une chance d'ascension sociale. \\
À titre scolaire équivalent, à diplôme équivalent, l'entrée sur le marché du travail n'est pas le même et s'est détérioré. 


Louis Chauvel, compare des générations nés de 1920 à 1935 et celles de 1936 à 1950 et celles de 1950 à 1965. \\
Il a plusieurs critères: répartition du pouvoir d'achat, où en 1975, les salariés de 50 ans gagnaient 15\% de plus que les salariés de 30 ans. En 2002, cet écart est de 35\%. \\
Les jeunes sont sur toutes ces générations, les plus touchés par la chômage. Ceci est vrai dans le secteur public comme dans le serveur privé. \\
Le déclassement touche inégalement les individus. \\
Il y a une perception objective et subjective de sa position sociale. 


Karl Manheim va définir de manière sociologique la génération. Il distingue trois manières de l'appréhender. \\
Il y a la situation de génération, une situation qui est identique pour tout le monde à partir du moment où les individus appartiennent à une même classe d'âge: même vision du monde. \\
Il y a l'ensemble générationnel, l'idée est que les membres qui forment cet ensemble partagent des enjeux Historiques qui leur assignent une sorte de destin commun (tous ceux ayant fait un service militaire en Algérie par exemple). Cela n'est pas valable sur l'ensemble des positions sociales, mais c'est bien la confrontation à l'événement qui forment l'ensemble. C'est la perception subjective qui a un rapport sur la trajectoire. \\
Il y a enfin l'unité de génération. Cela a à voir avec l'Histoire et un ensemble d'aspirations sociales, culturelles voir spirituelle (au sens large) qui correspond au milieu social de cette génération. Ce qui fait l'unité de génération, ce sont des individus contemporains qui sont soumis pendant des années de grande réceptivité aux mêmes influences culturelles et qui forment une génération au sens de leur homogénéisation de leurs expériences. \\
Cette situation de faire partie d'une génération, ce sentiment de faire partie d'une génération est nécessaire mais n'est pas suffisante car en fonction des classes sociales, des pays, on aura pas les mêmes expériences. 


La sociologie essaye de comprendre les événements qui font passer d'une génération à une autre. \\
Pour qu'un événement ait cet effet, il faut que cet événement crée une rupture d'intelligibilité dans le monde dans lequel on est. \\
On cherche alors un événement fondateur pour qu'il n'y ait pas seulement une différence d'âge mais aussi une différence de génération. 

\section{L'âge comme position dans le cycle de vie}

L'objectif est de voir comment les cycles de vies se succèdent et entraînent des phénomènes sociologiques. \\
Le texte de Bourdieu, "La jeunesse n'est qu'un mot", permet d'avoir tout de suite en tête que la position dans le cycle de vie est relative: on est toujours le plus vieux ou le plus jeune de quelqu'un. Les positions elles mêmes sont relatives et l'objet de luttes sociales (je suis de gauche car un autre est de droite). \\
Le sociologue doit déconstruire les classes déjà socialement formées car tous les cadres ne sont pas les mêmes cadres, il en va de même pour les jeunes ou les vieux. \\
Bourdieu explique que ces classes socialement formés ne sont que des objets de luttes. \\
Chaque champ est un espace de position et de prise de position dont l'objet de la lutte est la définition de la position et de la concurrence avec les autres.  


Galland dit que la jeunesse comme catégorie n'existe pas de la même manière dans toutes les classes sociales. Cependant, on peut unifier la jeunesse comme une phase moratoire, une sorte d'entre deux, un intermédiaire. La jeunesse, c'est quelque chose entre l'adolescence et l'âge adulte. C'est le moment où on a quitté sa famille mais où on en a pas construit une nous même. \\
Les enfants de catégorie populaire passent plus rapidement de la vie hors famille à la vie en famille car c'est coûteux de vivre seul. \\
Il y a des tendances lourdes qui ont créées cette jeunesse, notamment l'allongement de la scolarité. De fait, cet allongement s'est accompagné d'aspiration à des positions sociales. \\
Tout un tas d'institutions participent à la création de la catégorie jeunesse. Les radios libres par exemple, en 1981 qui ont créés une véritable fracture. \\
Il existe un flou de statut et de positions sociales qui sont propices à des préjugés sociaux. 



\chapter{Des discriminations ethno-raciales}

Les sociologues s'intéresse à la "race" comme un processus de "racialisation" des rapports sociaux. On appréhende donc la "race" comme un groupe construit socialement et on s'intéresse aux processus sociaux par lesquelles des populations quelle qu'elle soit se voit attribué une identité différente, une forme d'altérité sociale, qui peuvent être en partie fondé sur le préjugé de la race. Il y a donc un rapport de domination entre les groupes. \\
On notera qu'il peut y avoir des racismes sans raciste (des perceptions racistes discriminantes sans croire au caractère biologique de la race). La racialisation au quotidien se fait sur la distinction de différences physiques, plus largement, l'esclavage, la colonisation, les phénomènes de migration révèlent une racialisation plus grande des rapports sociaux. 

\section{Racialisation, ethnicisation, discrimination}

\subsection{Des catégories et de leurs usages}

Catégoriser, c'est définir un "nous" et un "eux", cela se fait par une racialisation ou une ethnicisation. \\
La racialisation donne lieu à des catégorisation dans la manière dont on construit les stéréotypes. De fait, un groupe racialisé est défini comme fondé sur l'existence d'un groupe, pratiques communes, caractéristiques physiques similaires. On oppose à la race l'ethnicisation, car la racialisation se base surtout le physique alors que l'ethnicisation voit le groupe comme une culture: histoire commune, géographie commune. \\
Weber a remis en question dans "Économie et société" la vision essentialiste de la société. Pour lui, les groupes ethniques ne se définissent pas par des caractéristiques objectives (histoire, géographie) mais se définissent surtout par une croyance commune en une histoire commune. L'ethnie, comme la race est donc un phénomène relationnel (je ne suis moi que parce qu'il y a des  autres). \\
Barth a écrit "La sociologie des questions raciales: les groupes ethniques et leurs frontières", il s'est intéressé à comment la frontière entre "eux" et "nous" s'est tracé entre les ethnies. La manière dont on trace la frontière est une manière de définir cela. Dans le phénomène d'altérisation (rendre quelqu'un autre que moi), il y a un phénomène d'identification: "c'est parce que j'appartiens à ce groupe que je n'appartiens pas à l'autre". Il y a dans les relations entre ces groupes un enjeu de pouvoir, de domination, d'occupation de l'espace géographique (Israël/Palestine en est un parfait exemple). \\
Les catégorisations sont entretenus par des institutions. \\
La différence supposé entre race et ethnie serait donc qu'il y a une question de biologie, de hiérarchisation morale, d'assignation du côté de la race, alors que du côté de l'ethnie, on insiste bien plus sur la culture, auto identifié. Dans les deux cas il s'agit d'un processus de construction de l'autre dans une relation de pouvoir asymétrique. 


Dans le temps et dans l'espace varie la catégorisation. \\
Wacquant montre que la race aux États-Unis admet plusieurs accessions, le poids, les ancêtres etc. Il s'intéresse aux catégories socio-culturelle aux boxeurs noirs. Il montre que la race est défini sur la base de l'ascendance et comment au Brésil, l'identité raciale est plutôt à la fois les marqueurs physiques mais aussi la position social. Il y a aussi des catégories dans les catégories racialisés.


"Ethno-racial" est une manière de combiner toutes les appréhensions racial et ethniques mais sert surtout à montrer les objets de lutte, de catégorisation dans un contexte historique. \\
C'est aussi une manière de dire que les catégories sont encastrés institutionnellement. Il y a des politiques publiques qui utilisent la racialisation. \\
Cette catégorisation ethno-racial n'existe que parce qu'elle est le fruit de la socialisation: elle s'est transmise de génération en génération.


\subsection{La discrimination comme produit d'une construction sociale}

Le racisme est "un programme d'action consistant à produire de l'altérité dans la société, afin d'alimenter un mécanisme de distinction, de relégation, de stigmatisation et de discrimination" (JF. Schaub). On confond parfois le racisme avec la xénophobie qui est "le rejet de l'altérité qui est défini comme une réalité statique manifeste", l'autre est autre à cause d'un caractère qui ne changera jamais (Schaub). \\
Concernant la stigmatisation, c'est "une hypothèse de fixité naturelle des conditions humaines ou d'une mutation probable dans un sens négatif". \\
Et, concernant la discrimination, c'est "le fait de traiter différemment en raison d'un critère illégitime". Dans le différemment, on entend de manière préjudiciable, cependant on peut faire une discrimination positive (critère illégitime mais traitement avantagé), l'idée étant de corriger quelque chose, de promouvoir, voire éventuellement de désinstitutionnalisé la race.


\subsection{Mesurer les discriminations}

En France, les statistiques ethno-raciales sont interdites contrairement aux États-Unis. \\
Les premières lois anti-discriminations datent de 2001. La HALD a été créée en 2004-2005 puis supprimée en 2011. \\
On a malgré tout des statistiques de l'INSEE, on peut déduire la "race" par le nom ou la langue parlée voire des questions de sondage à la con (qui peuvent être vus comme une fumisterie intellectuel). 


Dispositif du testing: on fait des faux CV et on regarde lesquels sont conservés et qui détient un entretien ou pas. \\
Ce test montre qu'à CV identique, celui qui a un nom "pas français" a trois fois moins de chances, cinq fois moins de chances pour un entretien. Si on croise le lieu d'habitation, on peut obtenir des douze fois moins de chance (quartier nord de Marseille typiquement).


Sur les observations ethnographiques (avec statistique), on a une enquête phare de F. Jobard et P. Levy qui ont enquêté sur les contrôles au faciès. Ils ont suivi la police sur 20 semaines, 525 contrôles, et les statistiques se sont fait suivant la couleur de la peau, du genre, de l'origine, de la tenue, du sac à dos etc. \\
Les résultats montrent que les hommes sont plus contrôlés, encore plus si on a une tenu "jeune" et d'autant plus si on a la capuche, le jean troué, et encore plus si on est noir et/ou maghrébin. 


Lahalle a travaillé sur des mineurs délinquants et montre que la police est beaucoup plus répressive quand il s'agit de jeunes maghrébins et que ceux-ci, à dossier comparé, obtiennent moins d'accompagnement. \\
De plus, les policiers agressés ont plus de chance de se porter partie civile si l'agresseur est soit noir, soit maghrébin.


Salapala a fait une enquête sur le logement social où elle montre que les agents mobilisent de manière routinière des préjugés pour éviter la présence dans les HLM de certains groupes dit "à problème". \\
Jounin a fait une enquête de même type sur les chantiers dans le domaine du bâtiment (il a observé 6 chantiers en 12 mois), et il y a une discrimination à l'encontre des non européens plus forts que contre les est-européens. \\
Chauvel a fait une étude sur des enfants scolarisés et le vécu des discriminations à l'école. Elle conclue que plus de 80\% des personnes âgé de 18 à 35 ans et qui sont sortis diplômés considèrent avoir été traités de manière inégalitaire à l'école. 

\section{Penser l'articulation des inégalités: un racisme institutionnel}

Quelle est le rôle de l'État dans la question de la racialisation ? \\
Le concept de racialisation institutionnelle est né au États-Unis au moment des luttes pour les droits civiques, et ont mis l'accent sur le fait que le racisme institutionnel trouvait sa force dans le fonctionnement même des institutions. On emploi le racisme institutionnel quand "en dehors de toute intention manifeste et directe de nuire à certains groupes ethnique, les institutions ou les acteurs au sein des institutions développent des pratiques dont l'effet est d'exclure ou d'inférioriser les groupes".

\subsection{École et (re)production des inégalités}

Au delà du sentiment d'inégalité scolaire, l'école est une institution de reproduction des inégalités sociales. Elle est aussi une institution de reproduction des inégalités ethno-raciales: l'école discrimine pour des raisons culturelles car plusieurs effets se combinent à l'intérieur de l'institution scolaire et certains groupes ethniques ont un rapport différent à l'accompagnement des études de leurs enfants, la capacité d'accompagnement n'étant pas la même partout (langue, communauté d'aide aux devoirs, effet prof etc). \\
De fait, l'école devient un instrument de racialisation et de stigmatisation.


Contrairement à ce qu'on pourrait penser instinctivement, les premières générations d'immigrés placent dans l'univers scolaire énormément d'espoir d'intégration pour leurs enfants. À inspiration supérieure et à accompagnement équivalent, les élèves racialisés ont de moins bons résultats, sont moins bien orientés que les élèves qui ne font pas l'objet d'une racialisation. \\
Les enfants d'immigrés sont donc particulièrement sensible à l'échec scolaire. 1 enfant sur 3 dont les parents étaient immigrés a redoublé à l'élémentaire, contre seulement 1 sur 5 pour les enfants dont un seul parent est immigré. Les écarts de réussite sont donc très inégaux. Un enfant racialisé qui réussit à l'école est donc sur-sélectionné (ce que le système scolaire a reconnu malgré lui).

\subsection{Marché du travail et conditions d'emploi}

Il y a trois caractéristiques propres à l'étude de l'emploi: l'emploi en lui même, la trajectoire professionnelle (mobile, ascendante, précaire), les conditions de travail. \\
Pour les enfants d'immigrés, les inégalités sont avérés, le chômage est plus fort en cause de la discrimination scolaire et à l'embauche, le niveau de précarité est supérieur, la carrière moins ascendante. \\
La seule manière d'inverser ces variables, c'est de faire partie d'une communauté ou d'un groupe dont l'intégration est extrêmement forte et qui possède un fort capital culturel (c'est notamment le cas des Portugais, des Asiatiques et des Turques), qui créent des communautés de travail palliant à ces inégalités. \\
L'inégalité de l'emploi est à combiné avec l'inégalité scolaire, mais aussi avec l'inégalité devant la justice qui conduit à un casier judiciaire et donc une difficulté d'accès à l'emploi.


L'organisation du travail est donc ethnicisé et les relations au travail sont racialisés. Cela s'oppose à plus ou moins de résistance. \\
Il existe aussi une inégalité de revenus, lié à l'inégalité de l'emploi (plus de 46\% d'écart dans le revenu des ménages). \\


\subsection{Logement et mobilité résidentielle}

12\% des blancs veulent rester dans des quartiers blancs, alors que 3\% des noirs ont envie de rester dans des quartiers ségrégués. Le fait d'habiter dans des territoires ségrégués est contraint par le revenu donc. \\
Les candidats supposément non français ont 50\% de chances de moins d'obtenir un rendez-vous pour un prêt bancaire, les propriétaires louent très peu aux populations racialisés qui sont redirigés de fait vers les logements sociaux. 

\subsection{Accès à la sphère civique: citoyenneté, police, justice, santé...}

Sur la sphère civique, c'est là qu'est le plus visible la racialisation. \\
Les soins sont de moindre qualité, le niveau d'attente est plus important, le suivi médical moins fait. \\
Des enquêtes qu'on essaye de corréler montrent que tout critère confondu, quand le niveau de discours racialisé augmente, l'ensemble des inégalités ethno-raciales augmente. 






\end{document}
