\documentclass[12pt, a4paper, openany]{book}

\usepackage[utf8x]{inputenc}
\usepackage[T1]{fontenc}
\usepackage[francais]{babel}
\date{}
\title{Cours de Sociologie (UFR Amiens)}
\pagestyle{plain}

\begin{document}

\chapter{L'âge}

\section{Introduction}

L'âge est quelque chose de très objectif, mais c'est aussi une façade sociale. Derrière l'âge, il y a un espace et une position sociale, c'est ce qui intéresse le sociologue. Derrière le mot jeune par exemple, nous vient directement l'idée d'une certaine précarité par exemple. \\
La mentalité, le caractère, etc. relève de la psychologie et ne nous intéresse donc pas. Il faut dégager les pré-notions qui découlent rapidement de l'âge. \\
Quand on fait de la sociologie, on cherche à croiser des variables et on s'intéresse surtout aux inégalités. On s'intéresse à l'âge comme variable de discrimination au même titre que le genre, l'ethnicité etc. L'âge est aussi une question d'époque, en rapport avec l'Histoire. Être jeune n'est pas la même chose lors de Mai 68 ou de la guerre d'Algérie. \\
On distingue généralement l'effet d'âge, l'effet de conjoncture et l'effet de génération. Parce que les différences d'âge correspondent à des moments dans le cycle de vie, et prédispose à certaines activités ou non. On distingue ainsi des jeunes et des moins jeunes en fonction d'un processus biologique, celui du vieillissement, et les cycles de vie ne nous intéresse que parce qu'ils correspondent à des processus sociaux ou des positions sociales. \\ 
Il existe aussi des générations très claires, et la question que le sociologue se pose sont sur leurs productions, leurs reproductions. Il y a deux manières de s'intéresser à la génération: l'effet de conjoncture, les opportunités qui sont offertes au moment où l'individu atteint un certain âge, comme l'entrée sur le marché du travail, l'entrée dans le mariage, le moment du premier enfant etc. La deuxième manière d'appréhender la génération est l'effet de génération au sens propre, c'est quelque chose de culturel ayant rapport avec la socialisation (génération Mai 68, Casimir etc.), des générations ont été socialisés par des événements politiques particuliers (une guerre, une révolution, un grand champ culturel). \\
Des choses se transmettent de génération en génération, c'est un phénomène de reproduction, et il y a un phénomène de ré-actualisation. \\
Pour penser tout ça, il faut connaître les espaces sociaux et les trajectoires des individus. 

\section{L'âge comme appartenance générationnelle: incorporation et confrontation aux structures socio-historiques}

L'année de naissance d'un individu permet de supposer la vision du monde qu'il peut avoir: quelqu'un né en 1962 n'aura pas la même vision ni la même trajectoire sociale qu'un individu né en 1960, pour la seule raison que 1962 est la fin de la guerre d'Algérie. \\
Il faut bien comprendre que la sociologie ne peut pas prédire la trajectoire, ni le futur des individus. 


Baudelot et Establet ont publié un ouvrage "Avoir 30 ans en 1968 et en 1998". Ils regardent dans quelle état on intègre la marché du travail en 68 et dans quel état on l'intègre 20 ans plus tard. \\
Ils montrent que globalement, il y a un écart de salaire qui s'est beaucoup creusé. On gagne moins bien sa vie en 1998, les diplômes se sont dévalorisés. Il y a donc une précarisation des positions économiques. \\
Les jeunes en 1998, on plus de mal à atteindre des positions monopolisées par les plus anciens car eux même ne grimpent plus les échelons (il y a un bouchon dans l'ascension sociale). \\
Les inégalités viennent du fait que un événement économique ou historique a un impact différent selon la place sociale. Le même événement social n'a pas la même répercussion: l'espace social n'est pas plat.
Il y a un phénomène de massification scolaire (et non de démocratisation). L'espoir des familles ont changés car espèrent une chance d'ascension sociale. \\
À titre scolaire équivalent, à diplôme équivalent, l'entrée sur le marché du travail n'est pas le même et s'est détérioré. 


Louis Chauvel, compare des générations nés de 1920 à 1935 et celles de 1936 à 1950 et celles de 1950 à 1965. \\
Il a plusieurs critères: répartition du pouvoir d'achat, où en 1975, les salariés de 50 ans gagnaient 15\% de plus que les salariés de 30 ans. En 2002, cet écart est de 35\%. \\
Les jeunes sont sur toutes ces générations, les plus touchés par la chômage. Ceci est vrai dans le secteur public comme dans le serveur privé. \\
Le déclassement touche inégalement les individus. \\
Il y a une perception objective et subjective de sa position sociale. 


Karl Manheim va définir de manière sociologique la génération. Il distingue trois manières de l'appréhender. \\
Il y a la situation de génération, une situation qui est identique pour tout le monde à partir du moment où les individus appartiennent à une même classe d'âge: même vision du monde. \\
Il y a l'ensemble générationnel, l'idée est que les membres qui forment cet ensemble partagent des enjeux Historiques qui leur assignent une sorte de destin commun (tous ceux ayant fait un service militaire en Algérie par exemple). Cela n'est pas valable sur l'ensemble des positions sociales, mais c'est bien la confrontation à l'événement qui forment l'ensemble. C'est la perception subjective qui a un rapport sur la trajectoire. \\
Il y a enfin l'unité de génération. Cela a à voir avec l'Histoire et un ensemble d'aspirations sociales, culturelles voir spirituelle (au sens large) qui correspond au milieu social de cette génération. Ce qui fait l'unité de génération, ce sont des individus contemporains qui sont soumis pendant des années de grande réceptivité aux mêmes influences culturelles et qui forment une génération au sens de leur homogénéisation de leurs expériences. \\
Cette situation de faire partie d'une génération, ce sentiment de faire partie d'une génération est nécessaire mais n'est pas suffisante car en fonction des classes sociales, des pays, on aura pas les mêmes expériences. 


La sociologie essaye de comprendre les événements qui font passer d'une génération à une autre. \\
Pour qu'un événement ait cet effet, il faut que cet événement crée une rupture d'intelligibilité dans le monde dans lequel on est. \\
On cherche alors un événement fondateur pour qu'il n'y ait pas seulement une différence d'âge mais aussi une différence de génération. 

\section{L'âge comme position dans le cycle de vie}

L'objectif est de voir comment les cycles de vies se succèdent et entraînent des phénomènes sociologiques. \\
Le texte de Bourdieu, "La jeunesse n'est qu'un mot", permet d'avoir tout de suite en tête que la position dans le cycle de vie est relative: on est toujours le plus vieux ou le plus jeune de quelqu'un. Les positions elles mêmes sont relatives et l'objet de luttes sociales (je suis de gauche car un autre est de droite). \\
Le sociologue doit déconstruire les classes déjà socialement formées car tous les cadres ne sont pas les mêmes cadres, il en va de même pour les jeunes ou les vieux. \\
Bourdieu explique que ces classes socialement formés ne sont que des objets de luttes. \\
Chaque champ est un espace de position et de prise de position dont l'objet de la lutte est la définition de la position et de la concurrence avec les autres.  


Galland dit que la jeunesse comme catégorie n'existe pas de la même manière dans toutes les classes sociales. Cependant, on peut unifier la jeunesse comme une phase moratoire, une sorte d'entre deux, un intermédiaire. La jeunesse, c'est quelque chose entre l'adolescence et l'âge adulte. C'est le moment où on a quitté sa famille mais où on en a pas construit une nous même. \\
Les enfants de catégorie populaire passent plus rapidement de la vie hors famille à la vie en famille car c'est coûteux de vivre seul. \\
Il y a des tendances lourdes qui ont créées cette jeunesse, notamment l'allongement de la scolarité. De fait, cet allongement s'est accompagné d'aspiration à des positions sociales. \\
Tout un tas d'institutions participent à la création de la catégorie jeunesse. Les radios libres par exemple, en 1981 qui ont créés une véritable fracture. \\
Il existe un flou de statut et de positions sociales qui sont propices à des préjugés sociaux. 



\chapter{Des discriminations ethno-raciales}

Les sociologues s'intéresse à la "race" comme un processus de "racialisation" des rapports sociaux. On appréhende donc la "race" comme un groupe construit socialement et on s'intéresse aux processus sociaux par lesquelles des populations quelle qu'elle soit se voit attribué une identité différente, une forme d'altérité sociale, qui peuvent être en partie fondé sur le préjugé de la race. Il y a donc un rapport de domination entre les groupes. \\
On notera qu'il peut y avoir des racismes sans raciste (des perceptions racistes discriminantes sans croire au caractère biologique de la race). La racialisation au quotidien se fait sur la distinction de différences physiques, plus largement, l'esclavage, la colonisation, les phénomènes de migration révèlent une racialisation plus grande des rapports sociaux. 

\section{Racialisation, ethnicisation, discrimination}

\subsection{Des catégories et de leurs usages}

Catégoriser, c'est définir un "nous" et un "eux", cela se fait par une racialisation ou une ethnicisation. \\
La racialisation donne lieu à des catégorisation dans la manière dont on construit les stéréotypes. De fait, un groupe racialisé est défini comme fondé sur l'existence d'un groupe, pratiques communes, caractéristiques physiques similaires. On oppose à la race l'ethnicisation, car la racialisation se base surtout le physique alors que l'ethnicisation voit le groupe comme une culture: histoire commune, géographie commune. \\
Weber a remis en question dans "Économie et société" la vision essentialiste de la société. Pour lui, les groupes ethniques ne se définissent pas par des caractéristiques objectives (histoire, géographie) mais se définissent surtout par une croyance commune en une histoire commune. L'ethnie, comme la race est donc un phénomène relationnel (je ne suis moi que parce qu'il y a des  autres). \\
Barth a écrit "La sociologie des questions raciales: les groupes ethniques et leurs frontières", il s'est intéressé à comment la frontière entre "eux" et "nous" s'est tracé entre les ethnies. La manière dont on trace la frontière est une manière de définir cela. Dans le phénomène d'altérisation (rendre quelqu'un autre que moi), il y a un phénomène d'identification: "c'est parce que j'appartiens à ce groupe que je n'appartiens pas à l'autre". Il y a dans les relations entre ces groupes un enjeu de pouvoir, de domination, d'occupation de l'espace géographique (Israël/Palestine en est un parfait exemple). \\
Les catégorisations sont entretenus par des institutions. \\
La différence supposé entre race et ethnie serait donc qu'il y a une question de biologie, de hiérarchisation morale, d'assignation du côté de la race, alors que du côté de l'ethnie, on insiste bien plus sur la culture, auto identifié. Dans les deux cas il s'agit d'un processus de construction de l'autre dans une relation de pouvoir asymétrique. 


Dans le temps et dans l'espace varie la catégorisation. \\
Wacquant montre que la race aux États-Unis admet plusieurs accessions, le poids, les ancêtres etc. Il s'intéresse aux catégories socio-culturelle aux boxeurs noirs. Il montre que la race est défini sur la base de l'ascendance et comment au Brésil, l'identité raciale est plutôt à la fois les marqueurs physiques mais aussi la position social. Il y a aussi des catégories dans les catégories racialisés.


"Ethno-racial" est une manière de combiner toutes les appréhensions racial et ethniques mais sert surtout à montrer les objets de lutte, de catégorisation dans un contexte historique. \\
C'est aussi une manière de dire que les catégories sont encastrés institutionnellement. Il y a des politiques publiques qui utilisent la racialisation. \\
Cette catégorisation ethno-racial n'existe que parce qu'elle est le fruit de la socialisation: elle s'est transmise de génération en génération.


\subsection{La discrimination comme produit d'une construction sociale}

Le racisme est "un programme d'action consistant à produire de l'altérité dans la société, afin d'alimenter un mécanisme de distinction, de relégation, de stigmatisation et de discrimination" (JF. Schaub). On confond parfois le racisme avec la xénophobie qui est "le rejet de l'altérité qui est défini comme une réalité statique manifeste", l'autre est autre à cause d'un caractère qui ne changera jamais (Schaub). \\
Concernant la stigmatisation, c'est "une hypothèse de fixité naturelle des conditions humaines ou d'une mutation probable dans un sens négatif". \\
Et, concernant la discrimination, c'est "le fait de traiter différemment en raison d'un critère illégitime". Dans le différemment, on entend de manière préjudiciable, cependant on peut faire une discrimination positive (critère illégitime mais traitement avantagé), l'idée étant de corriger quelque chose, de promouvoir, voire éventuellement de désinstitutionnalisé la race.


\subsection{Mesurer les discriminations}

En France, les statistiques ethno-raciales sont interdites contrairement aux États-Unis. \\
Les premières lois anti-discriminations datent de 2001. La HALD a été créée en 2004-2005 puis supprimée en 2011. \\
On a malgré tout des statistiques de l'INSEE, on peut déduire la "race" par le nom ou la langue parlée voire des questions de sondage à la con (qui peuvent être vus comme une fumisterie intellectuel). 


Dispositif du testing: on fait des faux CV et on regarde lesquels sont conservés et qui détient un entretien ou pas. \\
Ce test montre qu'à CV identique, celui qui a un nom "pas français" a trois fois moins de chances, cinq fois moins de chances pour un entretien. Si on croise le lieu d'habitation, on peut obtenir des douze fois moins de chance (quartier nord de Marseille typiquement).


Sur les observations ethnographiques (avec statistique), on a une enquête phare de F. Jobard et P. Levy qui ont enquêté sur les contrôles au faciès. Ils ont suivi la police sur 20 semaines, 525 contrôles, et les statistiques se sont fait suivant la couleur de la peau, du genre, de l'origine, de la tenue, du sac à dos etc. \\
Les résultats montrent que les hommes sont plus contrôlés, encore plus si on a une tenu "jeune" et d'autant plus si on a la capuche, le jean troué, et encore plus si on est noir et/ou maghrébin. 


Lahalle a travaillé sur des mineurs délinquants et montre que la police est beaucoup plus répressive quand il s'agit de jeunes maghrébins et que ceux-ci, à dossier comparé, obtiennent moins d'accompagnement. \\
De plus, les policiers agressés ont plus de chance de se porter partie civile si l'agresseur est soit noir, soit maghrébin.


Salapala a fait une enquête sur le logement social où elle montre que les agents mobilisent de manière routinière des préjugés pour éviter la présence dans les HLM de certains groupes dit "à problème". \\
Jounin a fait une enquête de même type sur les chantiers dans le domaine du bâtiment (il a observé 6 chantiers en 12 mois), et il y a une discrimination à l'encontre des non européens plus forts que contre les est-européens. \\
Chauvel a fait une étude sur des enfants scolarisés et le vécu des discriminations à l'école. Elle conclue que plus de 80\% des personnes âgé de 18 à 35 ans et qui sont sortis diplômés considèrent avoir été traités de manière inégalitaire à l'école. 

\section{Penser l'articulation des inégalités: un racisme institutionnel}

Quelle est le rôle de l'État dans la question de la racialisation ? \\
Le concept de racialisation institutionnelle est né au États-Unis au moment des luttes pour les droits civiques, et ont mis l'accent sur le fait que le racisme institutionnel trouvait sa force dans le fonctionnement même des institutions. On emploi le racisme institutionnel quand "en dehors de toute intention manifeste et directe de nuire à certains groupes ethnique, les institutions ou les acteurs au sein des institutions développent des pratiques dont l'effet est d'exclure ou d'inférioriser les groupes".

\subsection{École et (re)production des inégalités}

Au delà du sentiment d'inégalité scolaire, l'école est une institution de reproduction des inégalités sociales. Elle est aussi une institution de reproduction des inégalités ethno-raciales: l'école discrimine pour des raisons culturelles car plusieurs effets se combinent à l'intérieur de l'institution scolaire et certains groupes ethniques ont un rapport différent à l'accompagnement des études de leurs enfants, la capacité d'accompagnement n'étant pas la même partout (langue, communauté d'aide aux devoirs, effet prof etc). \\
De fait, l'école devient un instrument de racialisation et de stigmatisation.


Contrairement à ce qu'on pourrait penser instinctivement, les premières générations d'immigrés placent dans l'univers scolaire énormément d'espoir d'intégration pour leurs enfants. À inspiration supérieure et à accompagnement équivalent, les élèves racialisés ont de moins bons résultats, sont moins bien orientés que les élèves qui ne font pas l'objet d'une racialisation. \\
Les enfants d'immigrés sont donc particulièrement sensible à l'échec scolaire. 1 enfant sur 3 dont les parents étaient immigrés a redoublé à l'élémentaire, contre seulement 1 sur 5 pour les enfants dont un seul parent est immigré. Les écarts de réussite sont donc très inégaux. Un enfant racialisé qui réussit à l'école est donc sur-sélectionné (ce que le système scolaire a reconnu malgré lui).

\subsection{Marché du travail et conditions d'emploi}

Il y a trois caractéristiques propres à l'étude de l'emploi: l'emploi en lui même, la trajectoire professionnelle (mobile, ascendante, précaire), les conditions de travail. \\
Pour les enfants d'immigrés, les inégalités sont avérés, le chômage est plus fort en cause de la discrimination scolaire et à l'embauche, le niveau de précarité est supérieur, la carrière moins ascendante. \\
La seule manière d'inverser ces variables, c'est de faire partie d'une communauté ou d'un groupe dont l'intégration est extrêmement forte et qui possède un fort capital culturel (c'est notamment le cas des Portugais, des Asiatiques et des Turques), qui créent des communautés de travail palliant à ces inégalités. \\
L'inégalité de l'emploi est à combiné avec l'inégalité scolaire, mais aussi avec l'inégalité devant la justice qui conduit à un casier judiciaire et donc une difficulté d'accès à l'emploi.


L'organisation du travail est donc ethnicisé et les relations au travail sont racialisés. Cela s'oppose à plus ou moins de résistance. \\
Il existe aussi une inégalité de revenus, lié à l'inégalité de l'emploi (plus de 46\% d'écart dans le revenu des ménages). \\


\subsection{Logement et mobilité résidentielle}

12\% des blancs veulent rester dans des quartiers blancs, alors que 3\% des noirs ont envie de rester dans des quartiers ségrégués. Le fait d'habiter dans des territoires ségrégués est contraint par le revenu donc. \\
Les candidats supposément non français ont 50\% de chances de moins d'obtenir un rendez-vous pour un prêt bancaire, les propriétaires louent très peu aux populations racialisés qui sont redirigés de fait vers les logements sociaux. 

\subsection{Accès à la sphère civique: citoyenneté, police, justice, santé...}

Sur la sphère civique, c'est là qu'est le plus visible la racialisation. \\
Les soins sont de moindre qualité, le niveau d'attente est plus important, le suivi médical moins fait. \\
Des enquêtes qu'on essaye de corréler montrent que tout critère confondu, quand le niveau de discours racialisé augmente, l'ensemble des inégalités ethno-raciales augmente. 

\chapter{Les classes sociales dans l'espace physique}

Depuis 2008, plus de 50\% de la population mondiale vit en ville et cela s'accroît, surtout dans les pays en voie de développement qui s'urbanise très rapidement. \\
En France, une cinquantaine de villes réunissent plus de 100 000 habitants. Plus de 85\% des français vivent dans des zones urbaines. \\
Habiter en ville a des conséquences sur l'emploi, les modes de vie etc. Ce qui nous intéresse, ce sont les problématiques que la ville engage et son impact sur les rapports sociaux. \\
La sociologie urbaine a été fondée par la première école de Chicago qui a travaillé sur l'urbanisme, la question des transports, le maillage des commerces, les cultures etc. C'est une sociologie de la ville et des pratiques sociales qui vont avec le fait urbain. \\
R.Park a piloté la première étude paru en 1925: "The City" et qui est un article fondateur car il est clairement proposé de travaillé sur la ville comme dans un laboratoire social. L'impact de ces premiers travaux a porté sur d'autres scientifiques qui se sont posé d'autres questions (qualité/nature de l'habitat, effet des politiques publiques, mouvement des personnes etc.).


\section{La ségrégation urbaine: définitions et mécanismes}

Quand les gens arrivent dans les villes, ils ne s'installent pas n'importe où, mais en plus, ils ne s'installent pas par choix mais par contrainte, et, dans ces contraintes, on peut lire des formes de ségrégation. \\


\subsection{Qu'est-ce que la ville ?}

L. Wirth a aussi produit un article fondateur, en 1938, où le phénomène urbain est décrit comme un mode de vie et du coup, donne une définition de la ville. Il dit "La ville peut être défini comme un établissement relativement important, dense, et permanent d'individus socialement hétérogène". \\
Il trouve trois grandes tendances qui peut qualifier le fait urbaine: la concentration (densité) de population sur un territoire ; une hétérogénéité de population et des mobilités ; la fragmentation urbaine, comme une mosaïque tant dans l'espace que dans le social (frontière symbolique). Pour ceux qui vivent en ville, c'est un cadre de (re)constitution des groupes sociaux dans lequel les individus s'inscrivent. 

\subsection{Qu'est-ce que la ségrégation ?}

C'est une notion polysémique. \\
Pour les sociologues urbains, la ségrégation est la distance spatiale qui existe entre les groupes sociaux qui composent la ville. La ségrégation désigne alors toutes différences de localisation résidentielle, ces groupes sociaux étant défini par la position sociale, l'origine ethnique, le niveau de revenus. Ces différences sont marqués sur le territoire. \\
La tâche du sociologue est de mesurer cette ségrégation, on appelle cela une mesure de dissimilarité. On mesure le degré de concentration ou de dissémination d'un groupe donné sur un territoire. On crée un indice de dissimilarité dont on déduit l'indice de ségrégation. \\
Beaucoup de sociologues utilisent cette approche qui n'explique pas les causes de la ségrégation. À ce stade des recherches, rien ne permet de voir si la ségrégation est volontaire ou non. 


On peut aussi définir la ségrégation urbaine comme une inégalité d'accès aux espaces résidentiels. Quand on aborde le fait ségrégatif sous cet angle, on essaye de voir non pas tant à quel endroit atterrissent les individus mais si ils bougent, vers où, dans quelles conditions, sous quelles contraintes. \\
On étudie alors les parcours résidentiels.


On peut aussi voir la ségrégation comme une relégation dans une enclave. On regarde où les gens sont positionnés et on s'intéresse à ceux qui n'ont aucune mobilité et on s'intéresse à ceux qui ne bougent pas, et pourquoi. Souvent ils sont relégués à cet endroit et on parle alors de ghettos. \\
Le ghetto fait son apparition dans les travaux sociologiques américaines relativement tôt. Ce terme est importé en France dans les années 80-90. Il est défini comme "une forme de regroupement spatial associant étroitement des populations défavorisées à des territoires circonscrits". \\
E. Préteceille passe sa vie à essayer de démonter cette définition du ghetto. 

\subsection{Les mécanismes de la ségrégation}

La ségrégation peut résulter d'inégalités économiques. Dans ce cas, on regarde le niveau de patrimoine des individus, leurs salaires, le prix de l'immobilier et on voit qu'en fonction des quartiers, les prix ne sont pas les mêmes. En fonction du niveau de revenu et du patrimoine, on ne peut pas habiter partout. Une partie du fait ségrégatif urbain est directement causé par cette inégalité. \\
De plus, les activités économiques se distribuent inégalement, les emplois de l'industrie lourde vont être en périphérie, les petits commerces en centre-ville etc. 


La deuxième manière d'appréhender les causes est d'observer si il y a une volonté de ségrégation de certains acteurs sociaux vis à vis d'autres acteurs. \\
Lorsqu'elle est volontaire, c'est une agrégation de petites décisions qui conduisent à la ségrégation. Le fait ségrégatif vient du fait que les groupes sociaux ont une forme d'exigence de ne pas se retrouver en minorité. Cette exigence fait que les individus développent des stratégies résidentielles particulières. 


La troisième cause de ségrégation urbaine est la ségrégation explicite (apartheid). N. Elias dans "Logiques de l'exclusion", a, avec Scotson, montré qu'un groupe donné, en majorité sur son territoire spatial, le domine et empêche les autres groupes d'y rentrer. Elias appelle cela le "conflit entre les établis et les marginaux". \\
Elias ne parle pas de classes sociales mais seulement de perceptions individuelles. 


\subsection{Les effets des politiques publiques de peuplement}

Les effets des politiques publiques sont à prendre en compte, car elles peuvent essayer de réduire cette ségrégation. La construction des logements sociaux en est un parfait exemple. Les maires de villes sont obligés de construire des logements sociaux.


\section{Les théories de la dualisation: approches comparées}

La théorie de la dualisation est apparu aux USA, elle a été énoncée par Castells et Mollenkopf. Les espaces se divisent, se transforment en fonction du fait économique et que, du coup, on a deux types de dualisation, une sociale et une spatiale. La dualisation sociale, on observe une augmentation des cadres diplômés et à côté, une augmentation du "prolétariat". La dualisation spatiale où les activités se concentreraient en certains points augmentant alors les prix fonciers et, inversement, des territoires relégués avec un faible prix foncier. \\
Les sociologues se sont vites penchés sur la question des ghettos aux États-Unis.

\subsection{Ghettos et "Underclass" aux USA}

L. Wacquant a retracé l'histoire des ghettos pour essayer de définir ce qu'est un ghetto et ce qui ne l'est pas. Il montre que les ghettos noirs n'ont pas les mêmes caractéristiques les autres ghettos. La particularité des ghettos noirs est qu'ils se sont fait par relégation et non par choix d'une ethnie de vivre ensemble (contrairement à de nombreux quartiers aux USA). \\
Pour définir et différencier le ghetto noir des autres ghettos, il parle de ghetto institutionnel dans le sens où l'État a joué un rôle actif pour inciter les blancs à aller habiter en banlieue (subventions fédérale). La politique de "zonage" était donc très clair et assumée, dans les années 20 notamment et ont interdit, avec les prix du foncier, à certaines populations, de s'installer dans certaines zones. \\
De fait, dans ce contexte, le ghetto désigne bien des enclaves denses, relégués, où s'accumulent par nécessité et par contraintes les afro-américains au moment même où les classes moyennes américaines accèdent à des niveaux de revenus permettant d'aller dans des banlieues. \\
Wacquant nous dit que les ghettos sont synonymes d'exploitation économique et d'ostracisation sociale. Ce sont aussi des espaces d'ascension sociale, de solidarité, de construction d'identité sociale.


Si on doit caractérisé le ghetto comme concept, on est bien sur une enclave qui contient une population ciblé, mais on peut y voir des institutions duplicatives. Le ghetto est donc bouclé sur lui même, il a donc besoin de ses propres institutions et devient une ville dans la ville. \\
Les autorités sont d'autant plus d'accord que ces institutions empêchent les populations d'y sortir. \\
Il y a une manière d'être heureux dans un ghetto mais c'est aussi un espace sécurisant, d'intégration. \\
Le ghetto est donc biface. 


D. Fassin a travaillé sur l'exclusion sociale et la question de l'underclass aux USA en comparant avec la France. \\
Le débat sur l'underclass s'est ouvert dans les années 70 dans le champ médiatique américain (1974, choc pétrolier). Aux USA, il y a toute une campagne de presse américaine conservatrice qui décrivent les ghettos noirs comme des lieux paupérisés où se développeraient aussi des violences, des naissances hors mariages, etc. \\
Les premiers articles mettent les causes sur des phénomènes psychologiques. À ce moment là, on attribue des caractéristiques psychologiques particulières aux populations paupérisées. \\
Les sociologues vont vite s'inscrire en faux contre ces explications et vont essayer de comprendre ces phénomènes. \\
La pauvreté était d'abord un phénomène rural, qui concernait surtout les personnes âgées. On passe donc d'une pauvreté rurale à une pauvreté urbaine qui touche plus particulièrement les jeunes mais aussi les salariés. Il y a de fait, un phénomène qui fait que dans les lieux de relégation, on peut observer une anomie sociale. À la pauvreté s'ajouterait l'anomie sociale. \\
Contrairement à la nouvelle pauvreté, la pauvreté était cyclique, relativement résiduelle (on se "refaisait" d'une année sur l'autre), et surtout, elle était circonscrite à la classe ouvrière. \\
Aujourd'hui, la pauvreté semble plus permanente et se reproduit d'une génération sur l'autre. \\
De fait, à Chicago par exemple, 16\% des habitants d'un ghetto (en 1996) ont un emploi rémunérés, et la moitié vit en dessous du seuil de pauvreté. Le pire, dit Fassin, c'est que s'il y a une offre d'emploi de disponible, elle ne reviendra pas dans ce ghetto, qui est alors un lieu non pas seulement discriminé mais aussi discriminant. 


\subsection{Une dualisation à la française}

On s'interroge en France sur, notamment, les politiques de la ville et globalement sur les politiques publiques de lutte contre les inégalités territoriales. \\
J. Donzelot a reprit dans un article de 2004 la question de la dualisation en France. Il parle de ville à trois vitesses pour essayer de rendre compte d'évolutions relativement récente. Son hypothèse, c'est qu'il y a une forme de rupture qualifiable à partir d'un triple mouvement de séparation entre les groupes sociaux et qui se combine. \\
Le premier mouvement est le mouvement d'embourgeoisement des centre-villes prestigieux. Le deuxième est le départ des classes moyennes vers le péri-urbain moins coûteux et protégés. Le troisième est la relégation des cités d'habitats social. 


Si on regarde les classes supérieures, qui disposent le plus de latitude en matière de choix, elles se concentrent de plus en plus sur les centre-villes rénovés. \\
Sur les quartiers bourgeois, la tendance est ancienne mais l'arrivée des classes à fort capitaux dans les centre-ville est plus récente. C'est un phénomène de "gentrification" pour montrer comment les classes moyennes supérieures s'approprient des espaces urbains, les rénove, les requalifie et font donc augmenter le prix du foncier changeant la population qui vient s'installer (en somme c'est un changement sociologique d'un quartier). \\
Les gentrificateurs sont plutôt à fort capital culturel, ils ont un capital nouveau qui ne leur a pas été transmis.


Pour les classes moyennes, elles sont chassées des centre-ville par la hausse des prix du loyer et son chassés vers les extérieurs de la ville dans des habitats pavillonnaires. En urbanisme, on appelle ça le mitage pavillonnaire. \\
Pour décrire ce phénomène, Donzelot utilise le terme de sécession. 


Pour les ménages précarisés sont relégués dans des banlieues à l'extérieur de la ville et qui sont symboliquement fortement dévalorisées auquel on a attribué des zones d'anomie sociale. Certains appellent ces zones les territoires perdus de la République. Cependant, ce ne sont pas des ghettos car il n'y a pas d'institutions duplicatives. 


Lapeyronnie a écrit en 2008 et considère qu'il y a bien un effet ghetto dans les quartiers et qu'il est à entendre de deux manières. À la fois, il serait défini par l'extérieur, et une vision de l'intérieur qui considère que leur quartier est devenu, au fil du temps, un ghetto. C'est donc un étiquetage au sens de Becker. \\
Pour Lapeyronnie, le ghetto existe dès qu'on le considère comme une marge. Ce qu'il constate, c'est que, dans ces quartiers, il y a une culture de la consommation de masse qui finalement, attise les déceptions, les frustrations. De plus, ces quartiers ont souvent été populaires, donc dans des emplois industriels et créent une différence entre ceux qui ont un emploi et ceux qui n'ont en pas. \\
Ceux ayant un emploi regrette que le quartier soit devenu ghetto, les autres estimes qu'il n'y avait rien d'autres à faire. \\
Il y a un sentiment partagé de distance par rapport aux institutions républicaines. Le ghetto n'est pas un endroit anomique. Les habitants de ghettos reconnaissent l'État mais considère qu'il n'est pas pour eux. Une fois qu'on est stigmatisé, soit l'individu se sent stigmatisé soit il retourne le stigmate en le transformant en fierté. Le ghetto est donc un contre monde. Dans tous les cas, ces "ghettos urbains" sont des formes paroxystiques. 


\subsection{F/USA: des constellations socio-spatiales distinctes}

Préteceille essaye d'appliquer les modèles américains au monde social Français. Il se demande si la ségrégation sociale a augmenté à Paris. Il répondra à la fois par l'affirmative et la négative. \\
Il dit que les plus ségrégués des classes populaires, ce sont les ouvriers alors que globalement les employés et professions intermédiaires s'en sortent mieux. En réalité, la catégorie la plus ségrégué sur un territoire, ce sont les classes de la très grande bourgeoisie. \\
Il se demande si la ségrégation ethno-raciale existe en France comme aux USA. En France, la ségrégation ethno-raciale a effectivement augmenté depuis les années 80 mais restent faibles. Les plus ségrégués sont les immigrés d'origine maghrébines, subsaharienne et Turque.


Le premier indice de morphologie est la taille. Le ghetto de Chicago a 400 000 habitants, plusieurs centaines de kilomètres carrés. Le quartier noir de New York, c'est plus d'un million d'habitants. \\
En France, les quartiers les plus massifs au sens des indices de ségrégation, sont de l'ordre de un dixième par rapport aux ghettos américains. \\
Le deuxième indice est la structure. En France, les quartiers sont ouverts, restent mélangés, les institutions ne sont pas dupliqués. Aux USA, les ghettos sont relativement unifiés. \\
Un troisième indice est la mobilité, et, en France, on sort des quartiers ségrégués. \\
Donc, oui, il existe en France des quartiers discriminés, mais ce ne sont pas des espaces clos, ce ne sont pas des ghettos.


Un autre indicateur est la violence, on fait le présupposé de la violence dans les quartiers ségrégués français. \\
Il existe certes des phénomènes des violences, mais ils n'ont rien à voir avec les faits américains. \\
Un autre indicateur est le rôle de l'État. Aux USA, l'État n'est pas aussi présent que dans les quartiers français. Au pire, dans certains quartiers, l'État recule. Les USA n'investissent pas dans les ghettos contrairement à la France. 


\section{Des différents types de ségrégations en France}

Quand on parle de ségrégations en France, on parle de classe sociale car elle existe en France et donc on l'enquête. Mais aussi parce qu'il y a une influence marxiste. Cela fait que dans la sociologie européenne, l'espace urbain est le produit historique des rapports de classes. \\

\subsection{Les "beaux quartiers": l'entre soi réservé aux classes supérieures}

En France, les quartiers ségrégués ne sont pas les plus discriminés. Si on regarde comment les individus bougent dans un quartier, on se rend compte que la classe la plus enfermé sur elle même est la grande bourgeoisie (chef d'entreprises, cadres supérieurs à fort niveau de patrimoine, profession des arts et du spectacle, hautes fonctions publiques). \\
Le XVIe, le VIIIe, sont par exemple des arrondissements d'entre soi de grande bourgeoisie. Il y a sur ces territoires très peu de mouvement pendulaire. La construction de Bercy à l'Est a provoqué un mécontentement des hauts fonctionnaires habitant dans l'ouest, le transport fluvial a donc été remis en fonction car ils ne prennent pas le métro ni le bus. \\
Les mobilités professionnels, à part cette exception, sont très rares. \\
Les Pinçon-Charlot ont écrit "Les ghettos du Gotha" consacrés à la grande bourgeoisie. Ils ont passés toute une carrière auprès de la grande bourgeoisie et montrent donc comment, de génération en génération, ils restent sur les mêmes espaces et comment, cette appropriation du territoire est un élément de se maintenir en haut de la hiérarchie sociale car la haute bourgeoisie a les moyens de clore des espaces.


Concernant la stratégie matrimoniale, elle n'est pas spécialement contrôlée dans les classes populaires alors que dans la haute bourgeoisie, les mariages sont arrangés, ou presque. On explique aux femmes quels sont les bons partis. Des rallyes sont organisés pour se faire rencontrer les individus. C'est une manière de clore les groupes sociaux. \\
C'est encore plus vrai dans la noblesse dans le sens strict du terme où les nobles se marient toujours entre eux. \\
Les Pinçon-Charlot montrent qu'en partant du Louvre, on peut dessiner un axe du pouvoir qui passe par l'arc de triomphe et l'arche de la défense. \\
De fait, il existe une stratégie de la haute bourgeoisie pour annexer les espaces et les territoires pour asseoir leur position de pouvoir. 

\subsection{Les quartiers populaires: une cohabitation préservée ?}

La deuxième classe la plus ségrégué est la classe ouvrière pour des raisons inverses aux classes bourgeoises. \\
Les quartiers populaires sont des quartiers très ancrés dans l'histoire des territoires. Les populations des quartiers ouvriers sont liés aux mutations économiques. Le nombre d'ouvriers a fortement décru du fait de la désindustrialisation et la mondialisation. Une des tendances lourdes qui a transformé la classe populaire ouvrière, c'est la tertiarisation de l'économie avec des trajectoires différenciés. \\
Une partie des quartiers populaires ouvriers sont devenus précaire et paupérisés car n'ont pas pu suivre le mouvement économique. C'est dans lequel se concentre les plus précaires et les plus instables professionnellement. \\
Dans les quartiers les plus précarisés, seuls 52\% des chômeurs ouvriers y habitent. Il n'y pas d'uniformité dans les quartiers précarisés. La majorité des ouvriers habitent dans d'autres types d'espace que dans des "ghettos d'ouvriers". 44\% des chômeurs ouvriers habitent dans des espaces socialement mixte (avec classes moyennes). \\
Plus de la moitié des ouvriers habitent donc sur des territoires mixtes.


Ces quartiers ont une image mixte, personne ne les qualifierait de ghetto. On quitte ces quartiers que lorsque l'on change de classes sociales. L'évolution et la situation immobilière des classes populaires est donc très diversifié contrairement aux classes de haute bourgeoisie. \\
De fait, la relégation dans des territoires tout à fait appauvris est assez faibles. \\
Il existe très peu de quartiers sinistrés.

\subsection{Les classes moyennes périurbaines: séparatisme social ou déstabilisation ?}

Les classes moyennes ont une place centrale dans le travail de Danzelot. \\
Elles peuvent être chassées du centre ville par la rénovation de ceux-ci et vont donc allez dans les zones périurbaines. Cela fait bouger les frontières des villes. \\
Les classes moyennes font-elle exprès de se différenciés (à la fois des classes paupérisées et aussi de la haute bourgeoisie) ou est-ce une paupérisation à venir ? \\
Le pavillon de banlieue, permet de se différencié des plus pauvre au sens où c'est une maison, pas un appartement, sans être un logement de la haute. Il y a dans les zones périurbaines des techniciens, des enseignants, des professionnels de la santé etc. Ce sont des espaces très hétérogène car, de fait, ils sont près de quartiers bien plus ouvriers. \\
Il y a des pratiques différenciés dans les zones périurbaines, notamment les déplacements pendulaires en voiture ou en RER. Ces déplacements permettent de connaître comment les classes sociales utilisent l'espace. À chaque fois qu'il y a une mobilisation sur un transport, il va se jouer la frontière des classes sociales sur la gestion de l'espace. Les tracés sont l'enjeu de luttes.

\section{Conclusion}

Sur l'ensemble des phénomènes qui sont de ségrégations, ce qui est en creux comme question, c'est la question de la mixité sociale, de la mixité urbaine. C'est un enjeux fort des classes. \\
Toutes les politiques publiques de mixités n'engendrent pas nécessairement un rapprochement entre les classes sociales. Elles accentuent même parfois les tensions. \\
Pour repérer ce débat, il suffit de repérer le débat autour de la délinquance qui a toujours été la manière d'appréhendé la mixité. 








\end{document}
