\documentclass[12pt, a4paper, openany]{book}

\usepackage[latin1]{inputenc}
\usepackage[T1]{fontenc}
\usepackage[francais]{babel}
\date{}
\title{Cours de Sociologie (UFR Amiens)}
\pagestyle{plain}

\begin{document}
\maketitle

\chapter{Genre et socialisation}

\section{La socialisation}

Muriel Daron dit de la socialisation: "C'est l'ensemble des processus par lesquels un individu est construit par la société globale et locale dans laquelle il vit". Ce sont donc des processus au cours desquels l'individu intériorise des façons de faire, de penser, et d'être qui sont situées socialement. \\
Elle dit aussi: "La socialisation, c'est la façon dont la société forme et transforme les individus". \\
La socialisation est dynamique, elle n'est jamais achevée, elle continue tout au long de la vie. La socialisation a donc un caractère très dynamique. \\
La socialisation n'est pas de l'ordre de l'éducation ou l'inculcation, elle se fait de manière inconsciente. \\
La notion d'habitus est très proche de la notion de socialisation. Bourdieu, dans "La domination masculine" parle d'habitus masculin et d'habitus féminin. 

\subsection{La socialisation de genre}

On peut isoler trois dimensions principales:
\begin{itemize}
\item La dimension de l'apprentissage des rôles de sexe ;
\item La dimension de la ségrégation sexuée ;
\item L'identité de genre que développe les enfants. 
\end{itemize}

Il y a autant de socialisation de genre qu'il y a d'instances de socialisation, pareil pour les espaces de socialisation.

\section{La socialisation de genre primaire: la famille}

\subsection{Les parents comme modèle sexués}

Les deux parents offrent deux modèles d'identification distincts dans la famille. Les mères, dans l'éducation des enfants, ont plutôt des activités verbales ou visuelles ; alors que les pères se chargent de l'activité ludique ponctuelle, plutôt physique, psychomoteur. \\
Par ailleurs, les pères et mères ne s'adressent pas de la même manière aux enfants: les pères parlent davantage comme des adultes à leurs enfants tandis que les mères parlent davantage à ceux-ci avec un "langage bébé". Ce serait donc plutôt le père qui font le pont avec l'extérieur. \\
Pour des enfants un peu plus âgés, une sociologue a fait une enquête sur l'argent de poche: le rapport à l'argent serait sexué. Pour les plus jeunes, l'argent est d'abord donné par la mère par de petites fractions, de main en main. C'est ensuite le père qui prend le relais pour des sommes plus importantes et rationalisés. On notera aussi que ce sont plutôt les mères qui s'occupent des devoirs. \\
Les parents ne font pas la même chose et constituent donc un modèle d'identification différent. 

\subsection{Des comportements parentaux différenciés}

Un des premiers ouvrages dédiés à cette question s'intitule "Du côté des petites filles", de Elena Bellotti, en 1973. Elle montrait que dès la grossesse, la conduite n'est pas identique suivant le sexe de l'enfant. Cette différence de conduite tend à valoriser les garçons. \\
Par exemple, les femmes allaitent plus souvent les garçons que les filles, de plus, les parents tolèrent que les garçons soient voraces. Les cris des garçons sont plus tolérés etc. \\

\subsection{Des environnements matériels différenciés}

Ces différences sexuées s'incarnent aussi dans l'environnement des enfants: on va offrir des environnement différents aux filles et aux garçons. \\ 
Alors que l'égalité entre homme et femme adulte s'améliore, les différenciations sexués se font de plus en plus, et de plus en plus jeunes. \\
Si on s'intéresse aux vêtements, on se rend compte que la différenciation entre petites filles et petits garçons est très récente sur une perspective longue: sous l'ancien régime, cette différenciation n'existait pas, que ce soit garçon ou fille, ils étaient habillés exactement de la même manière. Ce n'est qu'au début du XXe siècle que les habits des garçons et filles ont commencés à se différencier. \\ 
Cela est pareil pour les couleurs, la différenciation est là aussi très récente vu que l'essentiel des vêtements étaient blancs. La différenciation date des années 60. \\
Pour ce qui est des jouets, ce sont des agents périphériques de socialisation qui véhiculent des normes de genre. Souvent, pour un enfant, jouer, c'est imiter des adultes. Les jeux ritualisent les rôles sexués et accentue les différences. Au XVIIIe et XIXe siècle, fille comme garçon jouent à la poupée. C'est au XIXe que l'on va commencer à manufacturé les jouets et donc créer des différences. Différences que l'on connaît aujourd'hui.

\section{La socialisation de genre à l'école}

Malgré une idéologie d'égalité entre le sexes, les enquêtes d'observation menées en classe, montrent une différence dans le comportement des enseignants. \\
Les enseignants interagissent plus avec les garçons qu'avec les filles. Les élèves garçons sont davantage perçus comme des individualités, et les filles, plus indifférenciées. Les sociologues ont aussi montrés, que, sur le contenu des interactions, les enseignants adressent davantage de remarques sur le fond aux garçons et sur la forme aux filles. \\
Les enseignants attribuent plus la réussite des filles à leurs efforts et celle des garçons à leurs qualités intellectuelles. \\
Les enseignants ont aussi des attentes stéréotypés sur les sexes: ils s'attendent à ce que les garçons soient dissipés et que les filles soient sages. Ils ont aussi un comportement différencié en matière d'orientation: ils proposent plus d'orientation en matière scientifique aux garçons qu'aux filles. 

\subsection{La réussite scolaire des filles}

"Allez les filles" est un ouvrage de Baudelot qui montre que les filles ont de meilleurs résultats, de plus, il y a plus d'étudiantes que d'étudiants. \\
Les explications avancées dans l'ouvrage sont que la socialisation primaire des filles les place en meilleure condition de conformité avec l'univers scolaire. De plus, comme ce sont les mères qui aident principalement aux devoirs, l'identification à la mère par les filles permettraient une plus grande aisance dans la scolarité. Les filles seraient aussi plus sensible aux attentes institutionnelles. Elles sont moins absentes, plus assidues, etc. \\
Jamais les orientations ne sont à la hauteur des réussite. Les différences d'orientation ne sont pas dus à une différence de réussite. Pour comprendre cette différence, il faut enquêter à un palier d'orientation (seconde), et on remarque que si les résultats sont semblables, les filles ont moins de culture scientifique personnelle. Il y a donc une différence dans l'auto-évaluation: à même niveau, les garçons s'estiment bon, les filles, non, et n'envisage donc pas de passage dans une filière scientifique. \\
Malgré la mixité scolaire, les modèles de sexe continuent à prévaloir. "Il est exceptionnel que les filles issus d'une catégorie sociale donnée dépasse le niveau des garçons de la catégorie immédiatement supérieure". 

\section{Pratiques culturelles}

\subsection{Le sport}

Mennesson a fait une enquête sur les femmes boxeuses et footballeuse dans le top niveau chez les amateurs. \\
Elle va chercher à comprendre comment elles ont réussi à transgresser les normes de genre et s'imposer dans un sport masculin. Elle va se rendre compte que la majorité des enquêtés ont constitués des dispositions sexuées inversés au cours de l'enfance et de l'adolescence dans la socialisation sportive et la participation au groupe de père masculin. Elles ont donc été socialisés par les hommes (père, frère) de leur famille où le sport a une certaine importance. \\
Ces filles ont donc généralement refusé d'intégré le groupe de filles. Cette socialisation spécifique s'inscrit dans des configurations familiales spécifiques: par exemple, une fratrie de filles où la plus jeune endosse le rôle du garçon pour répondre aux attentes du père d'avoir une descendance masculine (modèle du garçon manquant). Le deuxième modèle est au sein de familles nombreuses et la socialisation d'une fille serait confiée à ses frères les plus proches en âge. \\
Ces sportives intègrent plus ou moins le modèle de "garçon manqué" suivant l'intensité et la précocité de la socialisation sportive. 

\subsection{La lecture}

Les produits culturels disposent des stéréotypes sexués. Des études montrent que, au niveau des romans, les livres d'enfants etc. les stéréotypes sont fortement présents. Les femmes ont tendance à être peu présentes, passives, et les hommes à être les héros. \\
Detrez travaille sur la lecture des mangas, et montre que les garçons s'identifient parfaitement aux shonen seulement, alors que les filles montreront qu'elles lisent shojo comme shonen. Les lecteurs de mangas, une fois qu'ils ont affirmés leur identité sexué, vont prendre plus de distance et lire aussi des shojos: leur discours devient moins stéréotypé. 




\chapter{Genre et travail}

\section{Travail domestique et division sexuée du travail}

\subsection{Qu'est-ce que le travail domestique ?}

On appelle ça aussi le travail de reproduction: cela recouvre un certain nombre de tâches, qui sont relative au soins aux personnes et qui sont réalisés de manière quotidienne et gratuite, dans le cadre de la famille afin d'assurer l'entretien et le bien être des différents membres de cette famille. \\
On inclut dans le travail domestique le travail parental: soins aux enfants et aux adultes dépendants (personnes âgées) mais aussi les tâches administratives: courrier, impôts, etc. \\
Christine Delphy a publié un recueil d'articles des années 70 et 80 dans "L'ennemi principal - Tome 1: économie politique du patriarcat" qui montre que les tâches domestiques sont surtout caractérisé par leur gratuité. En effet, un grand nombre de tâches domestique pourrai être marchandes.


\subsection{Une répartition des tâches domestiques toujours inégalitaire}

La mesure du travail domestique date des années 70: on manque donc d'éléments pour comparé à long terme. \\
L'INSEE fait les premières mesure en 1981. Deux sociologues vont travaillées sur cette enquête et diront du travail domestique: "Tout activité réalisée au domicile et qui a un substitut marchand".  \\
En 1974, une femme inactive accomplit 40h de travail domestique par semaine, ce qui représente une part significative du PIB. Les sociologues montrent qu'une femme qui se dit inactive travaille donc. Elles montrent aussi que la mise en couple augmente le travail domestique des femmes mais a un effet nul sur les maris. Quand il y a un ou des enfants, cela augmente encore le travail domestique des femmes, ça a un impact nul sur les maris qui, au contraire, augmente leur temps de travail salarié. \\
La participation des femmes au marché du travail ne s'accompagne pas d'une redistribution égalitaire des temps de travail domestiques: une femme à temps plein travaille environ 28h par semaine aux travaux domestiques contre 16h pour les hommes. \\
Ces pratiques changent peu et lentement, par jour, les femmes consacrent en moyenne 3h30 aux travaux domestiques contre 2h pour les hommes. Les évolutions sont très lente car le temps domestique des hommes a augmenté de 6 minutes entre 1974 et 1999, et de 1 minute entre 1999 et 2010. 


Les hommes et les femmes ne consacrent donc pas les mêmes temps mais ne font pas du tout les mêmes choses. \\
Les hommes s'adonnent au bricolage, jardinage, soin aux animaux. Les femmes au ménage, aux courses, à la cuisine, le linge, prendre en charge les adultes dépendants. Le fait de s'occuper des enfants est assez partagé. \\
Même si les femmes des classes supérieures sont déchargés de certaines tâches, ce sont tout de même elles qui vont s'en occuper dans le sens où ce sont elles qui vont recruter des personnes pour s'occuper du ménage, du linge etc. \\
Il existe donc des tâches féminisées et d'autres qui sont plutôt masculines. 

\subsection{Division sexuée du travail}


D'après Danièle Kergoat: il y a deux principes  organisateurs de la division sexuée du travail: le principe de séparation (tâche masculine et tâche féminine) et le principe hiérarchique (les tâches masculines sont d'une valeur supérieure aux féminines). \\
Il est à noté que ce qui est considéré comme travail masculin ou féminin varie fortement selon les sociétés, mais la séparation est toujours présente. \\
"La division sexuelle du travail a pour caractéristiques l'assignation prioritaire des hommes à la sphère productive et des femmes à la sphère reproductive ainsi que simultanément, la captation par les hommes des fonctions à forte valeur sociale ajoutée". \\
La spécificité des tâches féminines est leur invisibilité alors que les tâches masculines sont visibles et valorisés. \\
Ce sont les femmes ou les filles aînées qui réalisent, dans les classes populaires, la majorité du travail administratif.


\section{Un accès inégal à l'emploi}

L'analyse de la division sexuée du travail est essentiel pour comprendre les inégalités de genre dans la sphère professionnel, d'abord parce que les femmes sont contraintes dans l'accès à l'emploi et d'autre part parce que cette division organise l'accès à l'emploi: les femmes et les hommes n'occupent pas les mêmes métiers, et, quand c'est le cas, on se rend compte qu'ils ne s'approprient pas le métier de la même manière et n'ont pas les même carrières. \\
Il y a un inégal accès à l'emploi et une inégalité dans l'exercice concret des métiers. 


\subsection{L'accroissement du salariat des femmes}

Les femmes travaillent depuis longtemps (évident si on prend en compte le travail domestique) mais c'est vrai aussi si on parle de travail marchand. Les conjointes d'agriculteurs, de marchands, etc réalisent un travail marchand non reconnu. \\
On recense 11,6 millions d'hommes actifs et 5 millions de femmes actives à la fin du XIXe siècle. Le salariat féminin existe donc depuis pas mal de temps. Si on ajoute à ces 5 millions les femmes qui font du travail marchand non reconnu, on arrive à 8,1 millions de femmes actives. De ce fait, l'écart n'est donc pas si important que ça. \\
Il y a une croissance très importante au XXe siècle.


Le taux d'activité des femmes croît très rapidement dans les années 60 et 70, et continue de progresser dans les années 80. Pour les femmes entre 24 et 57 ans, on passe de 59,9\% à 82,3\% de femmes qui travaillent. \\
Entre 1954 et 1982, le taux d'activité des mères de deux enfants passent de 24,7\% à 57,7\%, en 1997, on est 74\%, et enfin, en 2007, à 84\%. \\
D'après Margaret Maruani, le modèle dominant n'est plus le choix de la maternité ou du travail, ni le modèle de l'alternance (alterner entre période professionnel et maternage), mais le modèle dominant est le modèle du cumul dans lequel les maternités n'induisent pas d'interruption de l'activité professionnel. \\
Actuellement en France, les mères de deux enfants sont majoritairement actives ; ce n'est qu'au troisième enfant que le taux d'activité baisse. \\
Sur la même période, quand on s'intéresse au taux d'activité des hommes, il est très peu sensible à leur situation familiale. Il est de 96\% sans enfant, 98\% avec enfants.  


\subsection{Un accès à l'emploi qui demeure inégalitaire}

Il demeure des importantes disparités d'insertion avec notamment la question du temps partiel. Le temps partiel est une forme d'emploi féminin: en 2012, 30,2\% des femmes travaillent à temps partiel contre 6,9\% des hommes et 80\% des emplois à temps partiel sont occupés par des femmes. Dans un contexte de crise économique, cette forme d'emploi a été encouragée par les pouvoirs publiques. Le commerce, les services à la personne etc. se sont emparées de cette forme de travail pour gérer la main d'oeuvre. La plupart du temps, le temps partiel est imposé par l'employeur. \\
Seul 34\% des emplois à temps partiel féminin ont été choisis par les femmes pour des raisons familiales. C'est important car le temps partiel imposé concerne souvent des femmes jeunes, peu diplômée, peu qualifiée et en contrat précaire ; c'est une source importante de sous emploi. \\
Le temps partiel choisi concerne des femmes plus qualifiées et souvent dans la fonction publique. 


Pendant très longtemps, on avait un chômage des femmes qui était plus important que celui des hommes. Depuis quelques années, l'écart est très serré, voir même tend à s'inversé. L'écart n'est donc plus significatif. \\
Il y a de forte disparité selon l'âge, les diplômes ou la PCS. \\
Il peut y avoir une sous évaluation du chômage féminin car la frontière entre inactivité et chômage est plus poreuse: les femmes préféreront déclarer qu'elles sont femmes au foyer plutôt que chômeuse et l'inverse pour les hommes. 


\section{Un marché du travail segmenté: métiers de femmes et métiers d'hommes}

\subsection{Des métiers différents et inégalement prestigieux}

L'accroissement du salariat des femmes s'est produit au moment de la tertiarisation de l'économie. Les femmes se sont donc retrouvés concentrés dans le secteur tertiaire. Les femmes ont principalement pris des emplois dans des secteurs qui étaient déjà anciennement féminisés: éducation, santé, social. \\
De ce fait, on a 80\% de la population active féminine qui est concentré dans 5 secteurs d'activités: éducation, santé, action sociale, service aux particuliers, service aux entreprises. \\
À l'inverse, certains secteurs sont très masculins: l'industrie, le BTP, les transports. \\
C'est la ségrégation horizontale du marché du travail qui se double d'une ségrégation verticale: les hommes et les femmes ne sont pas représentés de la même façon dans les échelons.


Les métiers du "care" sont les métiers relatifs aux soins et à la prise en charge des jeunes enfants et des personnes dépendantes. Ces métiers permettent de montrer qu'il y a un déplacement des métiers féminins plutôt qu'une véritable remise en question de la division sexuée du travail. \\
Il y a une continuité de ces métiers avec le rôle des femmes dans la sphère privée. Ces métiers sont peu reconnus socialement et économiquement. \\
Ces métiers montrent un mécanisme: le déni de qualification. C'est à dire que les compétences des femmes acquises dans le travail domestique sont valorisées par les employeurs mais que ces compétences sont considérées comme des qualités féminines (comme quelque chose d'inné) et non pas comme des qualifications qui auraient fait l'objet d'un apprentissage. Ce déni de qualification entraîne notamment le fait que les métiers "féminins" sont bien moins payés. \\
Hoschild parle de travail émotionnel: il y a beaucoup de métiers où il y a une part importante d'émotions, faire paraître les émotions fait parti du travail (empathie etc.). Ce travail représente un coup psychique pour les femmes. Il est à la fois attendu de le faire et en même temps non reconnu dans le travail. Les femmes sont beaucoup moins protégées des sollicitations émotionnelles et sexuelles. 


\section{Métiers mixtes et carrières sexuées}

La mixité s'accompagne d'une recomposition des différences plutôt qu'une disparition.

\subsection{Féminisation des professions}

On a souvent associé féminisation et dévalorisation, à tort. Dans un certain nombre de cas, il y a effectivement dévalorisation mais liée à d'autres facteurs. \\
Plus un secteur d'emploi est ancien, moins il se féminise facilement. Ce qui facilite la féminisation d'une profession, c'est quand l'entrée est régi par un diplôme. Pour ce qui est des secteurs techniques, l'insertion des femmes a été plus facile dans les secteurs récents comme l'électronique où il n'y a pas de tradition ouvrière masculine à contrario de l'industrie, du bâtiment etc. \\
Les pionnières dans un domaine sont très souvent sur-sélectionnées socialement et scolairement. Souvent, celles-ci utilisent des techniques de "neutralisation", en euphémisant leur apparence de femme. Cependant, pour les femmes suivantes, c'est beaucoup plus variable, voire même les pousse à mettre en valeur leur féminité. Par exemple, les femmes chirurgiens font face à des pratiques sexistes très importantes et vont donc marqués leur féminité de manière symbolique (maquillage par exemple). 

\subsection{Mixité et différenciation sexuée}

La mixité est ici au sens de coexistence des deux sexes ne signifie pas égalité, car on constate que au sein de professions mixtes, il tend à se redéfinir des spécialités féminines et masculines. \\
Si on prend l'exemple de la magistrature qui était une discipline masculine (les femmes n'avaient pas le droit de passer le concours), mais cependant, une sociologue française a montrée que la discipline est mixte et paritaire, mais les femmes et les hommes n'aspirent pas à la même fonction. La fonction à laquelle aspire le plus les magistrates est de devenir juge des enfants, pour les magistrats, ils souhaitent surtout aller au parquet. 72\% des juges des enfants sont des femmes, cependant, elles sont minoritaires au parquet. \\
Chez les médecins, l'analyse est la même que chez les magistrats. Les femmes sont surtout présentes en pédiatrie, dermatologie etc.


Isabelle Bertaux Wianne a travaillée sur des petits indépendants: artisans, commerçants etc. qui ont la spécificité de s'installer en couple. Elle s'est intéressée à qui faisait quoi. Les conditions d'installation sont asymétriques, c'est souvent un projet de l'homme dans la continuité de sa formation alors que pour les femmes, c'est un projet étranger car souvent elles vont accepter le projet de leur époux et abandonner leur projet antérieur. \\
Ces femmes ont souvent un niveau scolaire plus important que leur mari, mais en même temps, elles anticipent leur valeur sur le marché du travail (la situation des femmes est plus incertaine et moins rémunératrice), mais ces femmes ont surtout intériorisée que la dimension professionnel est très important pour un homme. Dans ces cas, l'homme garde le statut principal, alors que la femme a le statut secondaire (femme de boulanger, de boucher etc.). \\
Celles qui n'adhèrent pas à ce modèle, ce sont celle qui ont un emploi régulier et bien payée alors que celle qui adhère sont des femmes de milieu populaire qui voient une possibilité d'ascension sociale. 


Dans le secteur de l'imprimerie, un secteur très ancien, il n'y avait que des hommes. En 1969, une imprimerie embauche des femmes pour être claviste sur de nouvelles machines. Le syndicat des ouvriers typographes négocie avec la direction que les femmes n'aient pas accès à la machine qui compose le texte (la plus importante de l'usine). Ce que montre les sociologues, c'est que peu à peu, les techniques évoluent (avec l'arrivée de l'informatique), mais à chaque fois que des changements techniques sont introduits, les ouvriers négocient des différences avec la direction. \\
Dans les années 80, les sociologues décrivent la situation suivante: les ouvriers qualifiés saisissent le texte et le corrige et font 180 lignes à l'heure. Les femmes saisissent le texte bien plus rapidement (300 lignes à l'heure) mais théoriquement, elles ne font pas de travail de correction et n'ont pas le droit à plus de 5\% d'erreurs. La différence de traitement entre tâche et salaire devient trop évidente, une grève des femmes permettra d'égaliser la situation. \\
Cette histoire dans une imprimerie nous montre réellement les conséquences de l'arrive de la mixité sexuée dans un secteur masculin.


La différenciation sexuée varie très fortement selon les professions et les organisations. Catherine Marry a travaillée sur les ingénieurs et montre qu'il y a une banalisation des femmes dans ce milieu. Mais selon les secteurs, il y a un maintient de la différenciation sexuée. \\
D'habitude, les stéréotypes agissent contre les femmes, mais parfois, ils jouent en faveur des femmes. Dans la police par exemple, le fait que le recours à la violence soit vu comme masculin joue dans l'intérêt des femmes car elles accèdent plus facilement à des postes de direction. 


\subsection{Le plafond de verre: Ségrégation verticale}

Les femmes et les hommes ne gravissent pas les échelons de la même manière: c'est la ségrégation verticale. Il y a une concentration des femmes dans les métiers et les postes du bas de l'échelle hiérarchique. \\
Ce qu'on appelle le plafond de verre, c'est le fait quand dans une profession donnée, les femmes sont de moins en moins nombreuse au fur et à mesure qu'on s'élève dans la hiérarchie. Laufelr écrit que le plafond de verre, c'est l'ensemble des obstacles visibles et invisibles qui séparent les femmes du sommet des hiérarchies professionnelles et organisationnelles. \\
Ce concept est particulièrement utilisé dans l'étude des carrières des femmes cadres et diplômé qui, dans le public comme dans le privé, n'accèdent pas à des postes de direction. \\
Les femmes représentent 58\% des salariés du public mais seulement 37,5\% des cadres et 12,5\% des cadres supérieurs. Il n'y a que 16\% de femmes qui sont chef d'entreprise. 


Les facteurs qui forment ce plafond de verre sont à la fois de l'ordre de la discrimination directe ou indirecte ainsi que de l'auto-exclusion. \\
Cécile Guillaume et une autre sociologue ont fait une étude sur une grosse entreprise et montre qu'il y a un plafond de verre qui résulte de la combinaison de cinq facteurs. \\
Le premier facteur est la politique de gestion des ressources humaines dans l'entreprise. Les RH favorisent la promotion de cadres issus de certaines grandes écoles d'ingénieurs dans lesquelles les femmes demeurent très minoritaires. \\
Le deuxième facteur, ce sont les freins à la mobilité géographique des femmes. La progression dans la carrière passe par la mobilité géographique et les femmes hésitent ou renoncent à cette mobilité car pense avoir des difficultés à l'imposer à leur conjoint. \\
Troisième facteur, les salariés qui aspirent à une promotion doivent faire preuve d'un investissement total qui est difficilement compatible avec des charges familiales. \\
Quatrième facteur, il y a un certain calendrier de promotion et donc rater une promotion du à une maternité par exemple, handicapent les femmes qui deviennent trop âgé pour être promu. \\
Cinquième facteur, les réseaux de sociabilité et de cooptation qui permettent de grimper dans la hiérarchie sont très marqués par des réseaux masculins. Les femmes n'ont donc pas accès à ces réseaux. \\
Les normes pratiques de ces organisations sont la source de discrimination sans qu'il y ait la volonté de discriminations. C'est de la discrimination indirecte. 


Il existe des processus qui relèvent d'une réelle discrimination. Un argument mobilisé est que les femmes seraient soumise à la charge domestique ou encore les grossesses. \\
Une enquête a été réalisée sur ce sujet. Cette perception discrimine les femmes dans l'accession aux postes supérieures. Battagliola a mené cette enquête et remarque l'absence de promotion des femmes à ancienneté égale avec les hommes. \\
La direction du centre justifie cet écart par la moindre disponibilité des femmes en raison de leurs charges familiales. De nombreuses femmes ont fait preuve d'une grande disponibilité dans leur travail et ont, pour autant, eu aucune promotion. 

\subsection{Carrière masculine et arrangement conjugaux}

On va s'appuyer sur l'enquête de Catherine Mary sur les ingénieurs et réalise qu'il y a une corrélation statistique entre la réussite professionnel des hommes ingénieurs et l'importance de leur descendance. Elle donne deux explications: cette plus grande réussite renvoie à la division du travail domestique qui existe dans les couples d'ingénieurs de familles nombreuses, ces pères sont dégagés de toute contrainte domestique par des épouses qui sont très diplômée mais femme au foyer. \\
La deuxième explication est le renvoie aux normes et aux stéréotypes de genre des employeurs qui associent aux pères de familles stabilisés le sens des responsabilité et l'autorité et qui donc, parmi les hommes ingénieurs, seront favorisés pour monter dans la hiérarchie.


Dans l'exemple de la haute fonction publique, le sociologue Jacquemart a écrit "J'ai une femme exceptionnelle" et montre que la conciliation entre vie pro et vie perso n'est un problème que pour les femmes, les hommes, eux, parviennent à s'investir dans leur vie personnelle sans avoir à résoudre des problèmes de conciliation. Les femmes de la haute fonction publique articule calendrier des postes et calendrier familial: elles font en sorte d'avoir une seule grossesse par poste. \\
Dans les entretiens, on voit que ces femmes ont du mal à se plier à l'amplitude horaire et la mobilité géographique que leur poste requiert. Elles ne postulent donc pas dans des postes plus élevés par crainte de séparation conjugale. \\
Les hommes eux, ne se sente pas de devoir planifier les naissances en fonction de leur calendrier pro, et d'ailleurs, dans les entretiens, les hommes font très rarement référence à leur vie domestique, familiale dans leur récit de carrière. Ils décrivent des arrangements conjugaux qui leur permet d'investir pleinement la sphère professionnel, d'être disponible, d'être mobile géographiquement car la vie familiale est prise en charge par leur conjointe. \\
"Pour l'instant, elle reste à la maison car elle a eu un deuxième enfant. Elle a eu une petite interruption de trois ans." Les entretiens montrent l'état de la situation familiale et les arrangements conjugaux chez les hauts-fonctionnaires. Ces modèles conjugaux sont majoritaire parmi les enquêtés, quelque soit la génération ou la classe sociale d'origine. \\
Sur 95 enquêtés, il y a deux couples qui ont un modèle égalitaire. C'est un modèle de couple à double carrière. 


\chapter{Genre et politique}

\section{La République masculine: histoire de l'exclusion politique des femmes}

\subsection{Les femmes en politique sous l'ancien régime}

À l'époque moderne, après le Moyen-Âge, les femmes disposent d'un certain nombre d'accès aux politiques: les femmes monarques, les femmes dans les assemblées provinciales, les femmes aristocrates chef de guerre, les femmes de classe populaire émeutière. \\
L'accès des femmes au trône est limité par rapport à celui des hommes et il est spécifique. Il existe une loi salique qui progressivement est considéré comme l'une des lois fondamentales du Royaume. Pour autant, il y a tout de même des cas où les femmes règnent: c'est le cas d'une régence par exemple (De Médicis). \\
Sous l'ancien régime, certaines femmes peuvent siéger dans les assemblées provinciales et élire des députés aux États-Généraux. Les femmes qui peuvent siéger sont les abbesses du clergé, les héritières des fiefs de la noblesse, les membres des corporations féminines dans le tiers État. Ces femmes siègent car on leur reconnaît un statut indépendamment de leur sexe.





\end{document}
