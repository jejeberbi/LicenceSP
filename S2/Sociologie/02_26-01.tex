\documentclass[12pt, a4paper, openany]{book}

\usepackage[latin1]{inputenc}
\usepackage[T1]{fontenc}
\usepackage[francais]{babel}
\date{}
\title{Cours de Sociologie (UFR Amiens)}
\pagestyle{plain}

\begin{document}
\maketitle

\chapter{Genre et socialisation}

\section{La socialisation}

Muriel Daron dit de la socialisation: "C'est l'ensemble des processus par lesquels un individu est construit par la société globale et locale dans laquelle il vit". Ce sont donc des processus au cours desquels l'individu intériorise des façons de faire, de penser, et d'être qui sont situées socialement. \\
Elle dit aussi: "La socialisation, c'est la façon dont la société forme et transforme les individus". \\
La socialisation est dynamique, elle n'est jamais achevée, elle continue tout au long de la vie. La socialisation a donc un caractère très dynamique. \\
La socialisation n'est pas de l'ordre de l'éducation ou l'inculcation, elle se fait de manière inconsciente. \\
La notion d'habitus est très proche de la notion de socialisation. Bourdieu, dans "La domination masculine" parle d'habitus masculin et d'habitus féminin. 

\subsection{La socialisation de genre}

On peut isoler trois dimensions principales:
\begin{itemize}
\item La dimension de l'apprentissage des rôles de sexe ;
\item La dimension de la ségrégation sexuée ;
\item L'identité de genre que développe les enfants. 
\end{itemize}

Il y a autant de socialisation de genre qu'il y a d'instances de socialisation, pareil pour les espaces de socialisation.

\section{La socialisation de genre primaire: la famille}

\subsection{Les parents comme modèle sexués}

Les deux parents offrent deux modèles d'identification distincts dans la famille. Les mères, dans l'éducation des enfants, ont plutôt des activités verbales ou visuelles ; alors que les pères se chargent de l'activité ludique ponctuelle, plutôt physique, psychomoteur. \\
Par ailleurs, les pères et mères ne s'adressent pas de la même manière aux enfants: les pères parlent davantage comme des adultes à leurs enfants tandis que les mères parlent davantage à ceux-ci avec un "langage bébé". Ce serait donc plutôt le père qui font le pont avec l'extérieur. \\
Pour des enfants un peu plus âgés, une sociologue a fait une enquête sur l'argent de poche: le rapport à l'argent serait sexué. Pour les plus jeunes, l'argent est d'abord donné par la mère par de petites fractions, de main en main. C'est ensuite le père qui prend le relais pour des sommes plus importantes et rationalisés. On notera aussi que ce sont plutôt les mères qui s'occupent des devoirs. \\
Les parents ne font pas la même chose et constituent donc un modèle d'identification différent. 

\subsection{Des comportements parentaux différenciés}

Un des premiers ouvrages dédiés à cette question s'intitule "Du côté des petites filles", de Elena Bellotti, en 1973. Elle montrait que dès la grossesse, la conduite n'est pas identique suivant le sexe de l'enfant. Cette différence de conduite tend à valoriser les garçons. \\
Par exemple, les femmes allaitent plus souvent les garçons que les filles, de plus, les parents tolèrent que les garçons soient voraces. Les cris des garçons sont plus tolérés etc. \\

\subsection{Des environnements matériels différenciés}

Ces différences sexuées s'incarnent aussi dans l'environnement des enfants: on va offrir des environnement différents aux filles et aux garçons. \\ 
Alors que l'égalité entre homme et femme adulte s'améliore, les différenciations sexués se font de plus en plus, et de plus en plus jeunes. \\
Si on s'intéresse aux vêtements, on se rend compte que la différenciation entre petites filles et petits garçons est très récente sur une perspective longue: sous l'ancien régime, cette différenciation n'existait pas, que ce soit garçon ou fille, ils étaient habillés exactement de la même manière. Ce n'est qu'au début du XXe siècle que les habits des garçons et filles ont commencés à se différencier. \\ 
Cela est pareil pour les couleurs, la différenciation est là aussi très récente vu que l'essentiel des vêtements étaient blancs. La différenciation date des années 60. \\
Pour ce qui est des jouets, ce sont des agents périphériques de socialisation qui véhiculent des normes de genre. Souvent, pour un enfant, jouer, c'est imiter des adultes. Les jeux ritualisent les rôles sexués et accentue les différences. Au XVIIIe et XIXe siècle, fille comme garçon jouent à la poupée. C'est au XIXe que l'on va commencer à manufacturé les jouets et donc créer des différences. Différences que l'on connaît aujourd'hui.

\section{La socialisation de genre à l'école}

Malgré une idéologie d'égalité entre le sexes, les enquêtes d'observation menées en classe, montrent une différence dans le comportement des enseignants. \\
Les enseignants interagissent plus avec les garçons qu'avec les filles. Les élèves garçons sont davantage perçus comme des individualités, et les filles, plus indifférenciées. Les sociologues ont aussi montrés, que, sur le contenu des interactions, les enseignants adressent davantage de remarques sur le fond aux garçons et sur la forme aux filles. \\
Les enseignants attribuent plus la réussite des filles à leurs efforts et celle des garçons à leurs qualités intellectuelles. \\
Les enseignants ont aussi des attentes stéréotypés sur les sexes: ils s'attendent à ce que les garçons soient dissipés et que les filles soient sages. Ils ont aussi un comportement différencié en matière d'orientation: ils proposent plus d'orientation en matière scientifique aux garçons qu'aux filles. 

\subsection{La réussite scolaire des filles}

"Allez les filles" est un ouvrage de Baudelot qui montre que les filles ont de meilleurs résultats, de plus, il y a plus d'étudiantes que d'étudiants. \\
Les explications avancées dans l'ouvrage sont que la socialisation primaire des filles les place en meilleure condition de conformité avec l'univers scolaire. De plus, comme ce sont les mères qui aident principalement aux devoirs, l'identification à la mère par les filles permettraient une plus grande aisance dans la scolarité. Les filles seraient aussi plus sensible aux attentes institutionnelles. Elles sont moins absentes, plus assidues, etc. \\
Jamais les orientations ne sont à la hauteur des réussite. Les différences d'orientation ne sont pas dus à une différence de réussite. Pour comprendre cette différence, il faut enquêter à un palier d'orientation (seconde), et on remarque que si les résultats sont semblables, les filles ont moins de culture scientifique personnelle. Il y a donc une différence dans l'auto-évaluation: à même niveau, les garçons s'estiment bon, les filles, non, et n'envisage donc pas de passage dans une filière scientifique. \\
Malgré la mixité scolaire, les modèles de sexe continuent à prévaloir. "Il est exceptionnel que les filles issus d'une catégorie sociale donnée dépasse le niveau des garçons de la catégorie immédiatement supérieure". 

\section{Pratiques culturelles}

\subsection{Le sport}

Mennesson a fait une enquête sur les femmes boxeuses et footballeuse dans le top niveau chez les amateurs. \\
Elle va chercher à comprendre comment elles ont réussi à transgresser les normes de genre et s'imposer dans un sport masculin. Elle va se rendre compte que la majorité des enquêtés ont constitués des dispositions sexuées inversés au cours de l'enfance et de l'adolescence dans la socialisation sportive et la participation au groupe de père masculin. Elles ont donc été socialisés par les hommes (père, frère) de leur famille où le sport a une certaine importance. \\
Ces filles ont donc généralement refusé d'intégré le groupe de filles. Cette socialisation spécifique s'inscrit dans des configurations familiales spécifiques: par exemple, une fratrie de filles où la plus jeune endosse le rôle du garçon pour répondre aux attentes du père d'avoir une descendance masculine (modèle du garçon manquant). Le deuxième modèle est au sein de familles nombreuses et la socialisation d'une fille serait confiée à ses frères les plus proches en âge. \\
Ces sportives intègrent plus ou moins le modèle de "garçon manqué" suivant l'intensité et la précocité de la socialisation sportive. 

\subsection{La lecture}

Les produits culturels disposent des stéréotypes sexués. Des études montrent que, au niveau des romans, les livres d'enfants etc. les stéréotypes sont fortement présents. Les femmes ont tendance à être peu présentes, passives, et les hommes à être les héros. \\
Detrez travaille sur la lecture des mangas, et montre que les garçons s'identifient parfaitement aux shonen seulement, alors que les filles montreront qu'elles lisent shojo comme shonen. Les lecteurs de mangas, une fois qu'ils ont affirmés leur identité sexué, vont prendre plus de distance et lire aussi des shojos: leur discours devient moins stéréotypé. 








\end{document}
