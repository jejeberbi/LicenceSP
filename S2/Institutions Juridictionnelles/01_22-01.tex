\documentclass[12pt, a4paper, openany]{book}

\usepackage[latin1]{inputenc}
\usepackage[T1]{fontenc}
\usepackage[francais]{babel}
\date{}
\title{Cours d'institutions juridictionnelles (UFR Amiens)}
\pagestyle{plain}

\begin{document}

\chapter{Introduction}

On peut distinguer deux approches de l'administration: approche organique selon laquelle l'administration est un ensemble d'organes ou d'organisme (l'administration fiscale, de l'environnement etc.), dans cette approche, une institution en est une créée par une décision unilatérale de la personne publique. Dans un État unitaire comme la France, toutes les institutions naissent de la volonté de l'État. \\
Il existe aussi l'approche matérielle. L'administration est une fonction, une mission, une action ou encore une activité. Dans ce cadre, l'institution administrative est investi d'une mission d'intérêt général. Cette mission donne à l'institution administrative toute une série de privilèges qu'on dit "exorbitant du droit commun", c'est à dire qui n'existe pas en droit privé. Ces privilèges sont propre à l'institution et ça la caractérise par rapport à une institution privée. \\
Les institutions administratives sont composés par l'ensemble des organismes qui sous l'autorité du gouvernement, participent à l'exécution des tâches d'intérêt général qui incombe à l'État. Ces institutions peuvent être définis par leurs fonctions (sens matérielle) ou ce qu'elles sont physiquement (sens organique): ce sont des critères cumulatifs. \\
L'administration au sens d'administration publique se distingue de l'administration privée. Toutes les entreprises disposent d'une administration dont le but est d'assuré le suivi des affaires en cours et assurer le fonctionnement de l'ensemble des activités. La différence est le but visé. Le but de l'entreprise et celui de l'État n'est pas le même. \\
Il est à noter que l'intérêt général visé par les institutions publiques est une notion subjective et donc évolutive. Elle est différente selon les époques, le pays etc. L'intérêt général se définirait comme le bien commun de la société dans son ensemble, l'intérêt de la collectivité nationale: l'intérêt général s'oppose parfois donc à l'intérêt particulier.


La fonction administrative inclut deux ensembles de missions exercés dans l'intérêt général. Une fonction de règlement autonome ou d'application des lois ; une fonction de prestation où l'administration fournit biens et services aux membres de la collectivité. \\
Pour réaliser les missions dont elle est chargée, l'administration peut mettre en oeuvre des moyens de droit privé. Mais elle doit souvent mettre en oeuvre des moyens de droit public désigné par l'expression de prérogatives de puissance publique. \\
Si un contentieux intervient à l'utilisation de ces prérogatives, c'est un juge spécial: le juge administratif qui intervient. Du point de vue organique, l'administration a connu un important élargissement au cours du XXe siècle. En principe, la gestion des activités administratives est confiée à des personnes de droit public. Mais du fait de l'extension et la diversification de ces missions, le nombre d'organismes gérant ces missions a augmenté. \\
Cela a eu des conséquences sur le droit applicable sur les relations au sein des institutions administratives. Les personnes privées participent largement à la gestion des activités administratives. Certaines de ces personnes privées sont crées spontanément par les institutions publiques pour profiter de la souplesse du droit privé. \\
Des personnes privées peuvent être associées à la réalisation des missions des institutions (État avec les écoles privées par exemple). Cela peut aller encore plus loin puisque des personnes privées peuvent se voir confier elle même la gestion d'une mission de service public. 


La notion d'institutions administratives est plus étroite que celle d'administration car ne recouvre que les personnes de droit public. \\
Ces organismes de droit public sont très nombreux et ils ont des formes juridiques variées. Certains ont la personnalité morale alors que d'autres ne l'ont pas. \\
Il ne faut pas confondre les personnes publiques avec les autorités publiques ou administratives qui sont des personnes physiques qui relèvent du droit public (préfet, maire, ministre etc.). 


Les différents types de personnes morales de droit public:
\begin{itemize}
\item L'État ;
\item les collectivités territoriales, qui se définissent par un territoire et la généralité de leurs fonctions (commune, département, région, à statut particulier et collectivité d'outre-mer), l'article 72 de la Constitution fixe les collectivités mais le législateur peut en créer de nouvelles ;
\item les établissements publics, qui se définissent comme un ensemble de moyens, affectés à un service public spécialisé doté de la personnalité morale publique. La création de ces établissements correspond à des besoins très divers, ils sont rattachés à une collectivité territoriale et ont une spécialité ;
\item les institutions publiques spécialisés, certaines personnes publiques ne sont ni des établissements publics ni des collectivités. La banque de France a été qualifiée de personne morale par le juge ;
\item les groupements d'intérêt public, leurs activités ont un caractère d'intérêt général et l'État effectue un contrôle sur leur création et leur administration. 
\end{itemize}

\part{Les juridictions administratives}

\chapter{L'autonomie de la juridiction administrative}

\section{L'existence de deux ordres de juridiction}

En France, l'organisation juridictionnelle se caractérise par une dualité, elle tient à l'existence d'un ordre juridictionnelle administratif autonome à l'ordre judiciaire. Les deux ordres forment deux ensembles distinct et hiérarchisés de juridiction. \\
Un ordre de juridiction est un ensemble de juridiction constituant une hiérarchie unique et soumise au contrôle d'une cour suprême. \\
Les juridictions judiciaires ont compétences pour régler des litiges opposant les particuliers entre eux et pour assurer le respect des lois. Les juridictions administratives ont compétence pour connaître des litiges opposant des administrés à l'administration ou encore les administrations entre elles. \\
On notera que les États anglo-saxons n'ont pas ce système de dualité et ont donc une unité de juridictions. 

\subsection{Les causes de la dualité}

L'apparition d'une juridiction administrative autonome a deux causes principales: d'une part, une volonté d'une mise à l'écart des tribunaux judiciaires. \\
Dès l'ancien régime, le Roi a tenté d'affirmé son autorité et celle de ses intendants face au pouvoir judiciaire des anciens parlements provinciaux. Déjà en 1641, l'édit de St-Germain interdisait aux juges de se mêler des affaires de l'État, de l'administration ou du Gouvernement. Cependant, ce texte n'a pas eu beaucoup d'effets. Les révolutionnaires, attachés au pouvoir des autorités élus supprime les parlements d'ancien régime dont ils se méfient parce que ces parlements, même si ils ont résisté à l'absolutisme royal, se montrent conservateur.


D'une autre part, c'est la conception rigide de la règle de séparation des pouvoirs qui en elle même n'interdit pas aux tribunaux judiciaires de statuer sur les contentieux de l'administration. Or, pour les révolutionnaires, juger l'administration, c'est aussi administrer. \\
Pour les révolutionnaires, la séparation des pouvoirs implique la séparation des autorités administratives et judiciaires. Ils feront donc une loi et un décret en ce sens: la loi des 16 et 24 août 1790 ainsi que le décret du 16 fructidor an III. 


\subsection{La garantie constitutionnelle du dualisme juridictionnel}

La Constitution de 1958 ne traite pas de l'ordre juridique administratif donc ne garantie pas le dualisme. \\
Ce dualisme pouvait donc être remis en cause par une Loi. Cependant, le CC a rendu deux décisions importantes pour consacré l'autonomie du juge administratif. \\
La première est une décision du 22 Janvier 1980 dite "validation d'actes administratif" fait de la juridiction administrative et donc le dualisme un Principe Fondamental Reconnu par les Lois de la République. \\
L'article 64 consacre l'indépendance des institutions juridictionnelles. 


Une décision de 23 Janvier 1987 consacre la compétence des juridictions administratives. \\
Le principe de séparation des autorités administratives et judiciaires, n'a pas, en lui même, valeur constitutionnelle. L'annulation ou la réformation des décisions prises par les autorités publiques dans les prérogatives de puissance publique fait partie des principes fondamentaux reconnus par les lois de la République: le juge administratif est exclusivement compétent pour y prononcer. \\
Le législateur peut, dans l'intérêt d'une bonne administration de la justice, unifier les règles de compétence juridictionnelle au sein de l'ordre juridictionnel intéressé: le législateur peut décider que l'ensemble d'un contentieux sera de la compétence du juge administratif ou du juge judiciaire. \\
En l'espèce, le contentieux de la concurrence a été unifié au juge judiciaire. \\
Cette décision est importante car consacre la protection constitutionnelle du domaine de compétence du juge administratif. En outre, l'article 161-1 de la Constitution, issu de la loi constitutionnelle de 2008 consacre indirectement le Conseil d'État comme juridiction suprême de l'ordre juridictionnel. 


\section{La structuration progressive de la juridiction administrative}

La juridiction administrative s'est progressivement séparée de l'administration elle même. 

\subsection{L'évolution vers l'indépendance}

Fin XVIIIe siècle, début XIXe, les tribunaux judiciaires ne pouvaient pas juger les affaires administratives. On a donc considérés que ceux-ci devaient être jugés au sein de l'administration. \\
Il y a deux recours: recours devant l'administration ou recours contentieux. L'administré doit faire un recours devant l'administration si il y a problème: l'administration se retrouve juge et parti, il peut y avoir un doute sur la partialité de l'administration. Les réclamations remontent la hiérarchie, les litiges sont réglés par le Ministre en tant que supérieur hiérarchique de l'ensemble des agents publics. On appelle cela le ministre juge. \\
Le chef de l'État est censé prononcer lui même les décisions et ne les rends définitive que par sa signature. Lorsque le Conseil d'État est créé par l'article 52 de la Constitution de l'an VIII, en 1799, c'est lui qui prépare les décisions et solutions en tant que conseil du Gouvernement, et donc des ministres. Le chef de l'État suit toujours les projets élaborés par le Conseil d'État d'autant qu'une commission du contentieux au sein du CE est créée en 1806, chargée justement de la préparation de ces solutions. \\
On appelle cela la justice retenue: c'est le CE qui fait les décisions, mais celles-ci sont retenues par le Chef de l'État car c'est lui qui les rends définitives.


La loi du 24 Mai 1872 va permettre le passage à une justice déléguée. Cette loi donne au CE le pouvoir de juger souverainement et de manière indépendante au nom du peuple français. \\
Le CE est donc depuis, un juge à part entière. \\
Cette loi crée par ailleurs le tribunal des conflits avec une composition strictement paritaire de membres du CE et de la C.Cas, qui a pour rôle de résoudre les conflits de compétence entre les juridictions judiciaires et administratives. Sa mission est également de prévenir un déni de justice dans le cas de contrariété de décisions définitives rendus dans le même litige par une juridiction de chacun des deux ordres (pour éviter que les deux ordres s'estiment incompétents). \\
Cependant, le système du Ministre juge continuait de s'appliquer: les recours administratifs devaient être présentés devant le ministre. Le CE jouait une sorte d'appel car la décision du Ministre pouvait être contesté.


Le juge de droit commun a compétence pour connaître de toutes les affaires qui n'entrent pas dans les compétences d'attribution d'une autre juridiction.


En 1889 est rendu l'arrêt Cadot où l'intervention préalable du Ministre juge est supprimé. Le CE devient juge de droit commun en premier et dernier ressort des contentieux administratifs. \\
La loi du 28 pluviôse, an VIII, crée l'administration préfectorale et les conseils de préfecture. Ces conseils sont des juridictions qui interviennent au niveau départemental et ont pour mission de conseiller les préfets mais à la différence du Conseil d'État, ces conseils de préfecture ont dès leur création des fonctions juridictionnelles. Ces conseils sont chargés des impôts directs, des travaux publics, des domaines nationaux et de la voirie publique. \\
Il existait de nombreuses critiques concernant ces conseils de préfecture, notamment concernant le manque d'indépendance de leur juge: ils ont donc été réformés plusieurs fois et leur champ de compétence a été étendue. \\
Le CE était très encombrée, et avant 1953, il ne parvenait plus à remplir sa mission (25 000 affaires en attente et 8 ans d'attente en moyenne pour la résolution d'une affaire). \\
En 1953, ont été créés les tribunaux administratifs qui remplacent les conseils de préfecture. Ces tribunaux deviennent les juges de droit commun des contentieux administratifs. \\
Malgré la création de ces tribunaux, le CE a conservé de nombreuses compétences en premier ressort. Cette réforme, en 1953, s'est montré vite insuffisante: l'encombrement du CE a encore augmenté dans les années 70-80. \\
Un autre niveau de juridiction sera créé: une loi du 31 Décembre 1987 crée les cours administratives d'appel auquel de larges domaines de compétences sont attribués. Néanmoins, le CE va conservé des compétences en appel. \\
La structure de l'ordre administratif s'est donc rapprochée, dans sa forme, de l'ordre judiciaire. Du fait du nombre croissant de recours devant le juge administratif, le nombre de recours continue d'augmenter: en 1976, il y a 25 000 recours, en 2005, il y en a 156 000 en 2005. \\
Des réformes sont intervenues: notamment le développement d'instance à juge unique qui ont permis d'améliorer le fonctionnement de la juridiction administrative, et de réduire les délais de jugement. 


\subsection{Les spécificités de la juridiction administrative} 

Habituellement, dans l'ordre judiciaire, les tribunaux sont composés de magistrats soumis à un statut particulier garanti par la Constitution. Les juridictions administratives sont, elles, composées de juges soumis à des modes de recrutement et à un déroulement de carrière particulier. Les 300 fonctionnaires du CE ne sont pas magistrat même si ils exercent. \\
Les tribunaux judiciaire ne font que trancher des litiges alors que les juridictions administratives jouent le rôle de conseiller juridique du Gouvernement et des administrations. \\
Enfin, les règles de procédure sont distinctes de celle du juge judiciaire. À juge spécial, droit spécial. Chaque juridiction applique un droit différent. \\
Quand se sont développées les juridictions administratives s'est posé la question du droit à appliquer à l'administration: un droit privé ou un droit spécial. \\ 
Le tribunal des conflits, dans l'arrêt Blanco de 1873, a affirmé l'autonomie du droit administratif. Dans cette affaire, un enfant a été blessé par un petit wagon d'une manufacture de tabac. Le père a saisi un tribunal en dommages et intérêts dirigés contre l'État qui est considéré comme responsable des fautes des ouvriers. L'administration considérait que l'ordre judiciaire était incompétent, le tribunal des conflits a eu à se prononcer. Celui-ci a affirmé que le droit privé ne s'appliquait pas à l'État, dès lors qu'il est impossible d'assimiler l'État à un simple particulier. Il y a donc non seulement un juge spécial mais aussi un droit spécial. \\
Il arrive parfois que l'administration soit jugée par l'ordre judiciaire qui lui applique le droit privé. 


\subsection{Une organisation juridictionnelle critiquée}

Une des critiques les plus importantes est que l'existence d'un juge spécifique pour l'administration est source de complexité pour les justiciables. Ceux-ci ignorent en général la répartition des compétences entre les juridictions et saisissent en général un juge incompétent, ce qui est source de retard. \\
Cette dualité peut aussi être la source d'un ralentissement de la procédure. Le juge judiciaire peut être amené à saisir le juge administratif d'une question préjudicielle, c'est à dire de régler un point de droit qui n'est pas de sa compétence, et qui détermine l'issue d'un procès judiciaire. Ceci provoque un ralentissement de la procédure. \\
Une autre critique est que l'administration dispose d'un privilège de juridiction: elle est jugée par une juridiction spécifique. Certains ont pu considérés que ce privilège était défavorable aux administrés, jugés selon des règles différentes et donc moins avantageuses pour lui. Le juge administratif s'est employé à compenser ce caractère inégalitaire du droit administratif au bénéfice des administrés. \\
Une dernière critique est qu'on reproche aux juges administratifs de ne pas être indépendant. Ce sont des fonctionnaires recrutés normalement par la voie de l'ENA, ce ne sont pas des magistrats formés par l'ENM. Cette critique doit être relativisé pour plusieurs raisons: l'indépendance des juges administratifs a valeur constitutionnelle ; le statut des membres des cours administrative leur garantit l'inamovibilité (ils ne peuvent pas être révoqués sans procédure spéciale ou mutés sans leur consentement). 



\chapter{Les principes de fonctionnement}

Les juridictions administratives sont composées d'une juridiction administrative de droit commun qui a deux degrés de juridiction et des juridictions administratives spécialisées. Toutes ces juridictions relèvent de l'autorité du conseil d'État comme juge de cassation. 


\section{Répartition des compétences entre les juridictions administratives}

La structure générale des juridictions administratives résultent des traditions et de l'histoire. Elle reflète aussi la recherche d'une certaine efficacité et aussi une proximité avec les requérants et les justiciables.  \\
Il y a deux grandes catégories de juridictions: les juridictions de droit commun et les juridiction spécialisée, le tout sous la houle du CE.

\subsection{Les juridictions de droit commun}

\subsubsection{Les tribunaux administratifs}

Ils ont succédé, en 1953, aux conseils de préfecture, on en compte 31 en métropole, 11 en Outre Mer. Chaque tribunal comprend un président et plusieurs conseillers. \\
Les tribunaux importants comprennent plusieurs chambres composées d'au moins 3 conseillers en vertu du principe du caractère collégial des formations de jugement. \\
Une loi du 6 janvier 1986 a donné un véritable statut législatif à leurs membres désormais considérés comme des magistrats à part entière dont l'indépendance est proclamée. \\
Depuis 1990, les TA sont rattachés au CE et non plus, comme auparavant, au ministère de l'intérieur. Les TA statuent en premier ressort et sont juges du droit commun. Tous les litiges relèvent de leur compétence sauf si une disposition spéciale a attribuée compétence à une autre juridiction. \\
Dans certaines procédures, ils doivent statuer en premier et dernier ressort. Le tribunal territorialement compétent est, en principe celui dans le ressort duquel se trouve le siège de l'autorité dont l'acte est attaquée. Pour éviter l'encombrement du TA de Paris, des exceptions ont été prévues. \\
Les TA peuvent être consultés par les préfets. 



\section{}








\end{document}
