\documentclass[12pt, a4paper, openany]{book}

\usepackage[latin1]{inputenc}
\usepackage[T1]{fontenc}
\usepackage[francais]{babel}
\date{}
\title{Cours d'institutions juridictionnelles (UFR Amiens)}
\pagestyle{plain}

\begin{document}

\chapter{Introduction}

On peut distinguer deux approches de l'administration: approche organique selon laquelle l'administration est un ensemble d'organes ou d'organisme (l'administration fiscale, de l'environnement etc.), dans cette approche, une institution en est une créée par une décision unilatérale de la personne publique. Dans un État unitaire comme la France, toutes les institutions naissent de la volonté de l'État. \\
Il existe aussi l'approche matérielle. L'administration est une fonction, une mission, une action ou encore une activité. Dans ce cadre, l'institution administrative est investi d'une mission d'intérêt général. Cette mission donne à l'institution administrative toute une série de privilèges qu'on dit "exorbitant du droit commun", c'est à dire qui n'existe pas en droit privé. Ces privilèges sont propre à l'institution et ça la caractérise par rapport à une institution privée. \\
Les institutions administratives sont composés par l'ensemble des organismes qui sous l'autorité du gouvernement, participent à l'exécution des tâches d'intérêt général qui incombe à l'État. Ces institutions peuvent être définis par leurs fonctions (sens matérielle) ou ce qu'elles sont physiquement (sens organique): ce sont des critères cumulatifs. \\
L'administration au sens d'administration publique se distingue de l'administration privée. Toutes les entreprises disposent d'une administration dont le but est d'assuré le suivi des affaires en cours et assurer le fonctionnement de l'ensemble des activités. La différence est le but visé. Le but de l'entreprise et celui de l'État n'est pas le même. \\
Il est à noter que l'intérêt général visé par les institutions publiques est une notion subjective et donc évolutive. Elle est différente selon les époques, le pays etc. L'intérêt général se définirait comme le bien commun de la société dans son ensemble, l'intérêt de la collectivité nationale: l'intérêt général s'oppose parfois donc à l'intérêt particulier.


La fonction administrative inclut deux ensembles de missions exercés dans l'intérêt général. Une fonction de règlement autonome ou d'application des lois ; une fonction de prestation où l'administration fournit biens et services aux membres de la collectivité. \\
Pour réaliser les missions dont elle est chargée, l'administration peut mettre en oeuvre des moyens de droit privé. Mais elle doit souvent mettre en oeuvre des moyens de droit public désigné par l'expression de prérogatives de puissance publique. \\
Si un contentieux intervient à l'utilisation de ces prérogatives, c'est un juge spécial: le juge administratif qui intervient. Du point de vue organique, l'administration a connu un important élargissement au cours du XXe siècle. En principe, la gestion des activités administratives est confiée à des personnes de droit public. Mais du fait de l'extension et la diversification de ces missions, le nombre d'organismes gérant ces missions a augmenté. \\
Cela a eu des conséquences sur le droit applicable sur les relations au sein des institutions administratives. Les personnes privées participent largement à la gestion des activités administratives. Certaines de ces personnes privées sont crées spontanément par les institutions publiques pour profiter de la souplesse du droit privé. \\
Des personnes privées peuvent être associées à la réalisation des missions des institutions (État avec les écoles privées par exemple). Cela peut aller encore plus loin puisque des personnes privées peuvent se voir confier elle même la gestion d'une mission de service public. 


La notion d'institutions administratives est plus étroite que celle d'administration car ne recouvre que les personnes de droit public. \\
Ces organismes de droit public sont très nombreux et ils ont des formes juridiques variées. Certains ont la personnalité morale alors que d'autres ne l'ont pas. \\
Il ne faut pas confondre les personnes publiques avec les autorités publiques ou administratives qui sont des personnes physiques qui relèvent du droit public (préfet, maire, ministre etc.). 


Les différents types de personnes morales de droit public:
\begin{itemize}
\item L'État ;
\item les collectivités territoriales, qui se définissent par un territoire et la généralité de leurs fonctions (commune, département, région, à statut particulier et collectivité d'outre-mer), l'article 72 de la Constitution fixe les collectivités mais le législateur peut en créer de nouvelles ;
\item les établissements publics, qui se définissent comme un ensemble de moyens, affectés à un service public spécialisé doté de la personnalité morale publique. La création de ces établissements correspond à des besoins très divers, ils sont rattachés à une collectivité territoriale et ont une spécialité ;
\item les institutions publiques spécialisés, certaines personnes publiques ne sont ni des établissements publics ni des collectivités. La banque de France a été qualifiée de personne morale par le juge ;
\item les groupements d'intérêt public, leurs activités ont un caractère d'intérêt général et l'État effectue un contrôle sur leur création et leur administration. 
\end{itemize}

\part{Les juridictions administratives}

\chapter{L'autonomie de la juridiction administrative}

\section{L'existence de deux ordres de juridiction}

En France, l'organisation juridictionnelle se caractérise par une dualité, elle tient à l'existence d'un ordre juridictionnelle administratif autonome à l'ordre judiciaire. Les deux ordres forment deux ensembles distinct et hiérarchisés de juridiction. \\
Un ordre de juridiction est un ensemble de juridiction constituant une hiérarchie unique et soumise au contrôle d'une cour suprême. \\
Les juridictions judiciaires ont compétences pour régler des litiges opposant les particuliers entre eux et pour assurer le respect des lois. Les juridictions administratives ont compétence pour connaître des litiges opposant des administrés à l'administration ou encore les administrations entre elles. \\
On notera que les États anglo-saxons n'ont pas ce système de dualité et ont donc une unité de juridictions. 

\subsection{Les causes de la dualité}

L'apparition d'une juridiction administrative autonome a deux causes principales: d'une part, une volonté d'une mise à l'écart des tribunaux judiciaires. \\
Dès l'ancien régime, le Roi a tenté d'affirmé son autorité et celle de ses intendants face au pouvoir judiciaire des anciens parlements provinciaux. Déjà en 1641, l'édit de St-Germain interdisait aux juges de se mêler des affaires de l'État, de l'administration ou du Gouvernement. Cependant, ce texte n'a pas eu beaucoup d'effets. Les révolutionnaires, attachés au pouvoir des autorités élus supprime les parlements d'ancien régime dont ils se méfient parce que ces parlements, même si ils ont résisté à l'absolutisme royal, se montrent conservateur.


D'une autre part, c'est la conception rigide de la règle de séparation des pouvoirs qui en elle même n'interdit pas aux tribunaux judiciaires de statuer sur les contentieux de l'administration. Or, pour les révolutionnaires, juger l'administration, c'est aussi administrer. \\
Pour les révolutionnaires, la séparation des pouvoirs implique la séparation des autorités administratives et judiciaires. Ils feront donc une loi et un décret en ce sens: la loi des 16 et 24 août 1790 ainsi que le décret du 16 fructidor an III. 


\subsection{La garantie constitutionnelle du dualisme juridictionnel}

La Constitution de 1958 ne traite pas de l'ordre juridique administratif donc ne garantie pas le dualisme. \\
Ce dualisme pouvait donc être remis en cause par une Loi. Cependant, le CC a rendu deux décisions importantes pour consacré l'autonomie du juge administratif. \\
La première est une décision du 22 Janvier 1980 dite "validation d'actes administratif" fait de la juridiction administrative et donc le dualisme un Principe Fondamental Reconnu par les Lois de la République. \\
L'article 64 consacre l'indépendance des institutions juridictionnelles.




\end{document}
