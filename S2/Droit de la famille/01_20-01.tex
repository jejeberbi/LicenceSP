\documentclass[12pt, a4paper, openany]{book}

\usepackage[latin1]{inputenc}
\usepackage[T1]{fontenc}
\usepackage[francais]{babel}
\date{}
\title{Cours de Droit Civil (UFR Amiens)}
\pagestyle{plain}

\begin{document}

\chapter{Introduction Générale du droit de la famille}

Pour parler d'un droit de la famille, il faut d'abord parler de la famille: on pourrai voir la famille dans l'histoire, dans la sociologie, ou encore dans l'affectif. Est-ce que la famille contemporaine est la même que la famille du Moyen-âge ? La notion de famille a infiniment évolué. \\
Au moment où on promulgue le Code Civil en 1804, la famille est un couple d'un homme et d'une femme mariés. L'idée du mariage est proche de l'idée religieuse: on se marie pour la vie et au delà. Le divorce est peu ancré et apparaît définitivement en 1884 dans le Code Civil. On parle d'enfants légitime quand les parents sont mariés. Pour les autres, le code parlera d'enfants naturels. \\
Aujourd'hui, le mariage existe encore certes, mais concerne les couples hétéro comme les couples homos. De plus, d'autres modèles conjugaux se sont créés: le PACS notamment mais aussi le concubinage qui a été introduit dans le Code. Le divorce existe bien sûr, mais a été fortement facilité.  La distinction enfant légitime/naturel a été supprimée. Ensuite, devant la facilité à se marié et à divorcé, c'est devenu de plus en plus courant de ne pas passer sa vie avec une seule personne (entre 40 et 50\% des mariages finissent par un divorce). Des familles recomposés sont aussi apparus. \\
Il y a donc aujourd'hui beaucoup de possibilités de famille. Certains parlent péjorativement de "plat à la carte". Il y a donc pluralité de situations: l'enjeu de la loi est donc de cadré le champ des possibles, il n'y a pas de modèle. La fonction du législateur serait donc de veiller à ce que il n'y ait pas de situations déséquilibrées. Le législateur cherche donc à encadré les champs de liberté pour pas qu'il y ait de situations désavantageuse pour quelqu'un.


Le droit de la famille est un droit vivant, mais aussi très technique. Parler du droit de la famille à notre époque, c'est surtout parlé des couples et donc aussi de leurs enfants si ils en ont. 

\part{Les couples}

Aujourd'hui, il y a trois types de couples: les couples mariés, les couples Pacsés, ainsi que les couples en concubinage.

\chapter{Les couples mariés}

\section{Le mariage}

\subsection{La validité du mariage}

\subsubsection{Les conditions de validités}

Les conditions de fond sont relatives à l'état des personnes ainsi qu'au consentement.


Les conditions relatives à l'État des personnes: \\
Éléments d'ordre individuel: le sexe et l'état de santé.


Dans la version de 1804, on ne dit pas explicitement que le mariage est l'union d'un homme et d'une femme. Le mariage est cependant l'union d'un homme et d'une femme: c'est évident car malgré la révolution française, malgré la laïcisation du Droit Civil (et donc du droit de la famille), le mariage dans la conception du Code Civil est pratiquement la résultante de la conception religieuse catholique. Puisque les choses sont implicites, l'article 144 ancien du C.Civ nous dit "L'homme avant 18 ans révolus, la femme avant 15 ans révolus, ne peuvent contracter mariage", c'est le seul qui nous permettent de deviner que le mariage a la conception d'une union d'un homme et d'une femme. \\
La loi du 17 Mai 2013 est une loi qui va aboutir à écarter l'idée que le mariage repose sur la différenciation des sexes. Depuis cette loi, l'union peut être d'un homme et d'une femme, de deux hommes ou deux femmes. Le Conseil Constitutionnel a du vérifier la loi ; dans sa décision du 17 Mai 2013 dit que si les lois avant 2013 voyaient le mariage comme l'union d'un homme et d'une femme, cette règle ne peut constituer un principe fondamental de la République au sens du préambule de 1946. Le sexe est donc indifférent.


Concernant l'état de santé, jusqu'à la loi du 20 Décembre 2007 (loi relative à la simplification du Droit), les futurs époux avaient obligation de produire en vu du mariage un certificat prénuptial (ce certificat avait été instauré pendant le régime de Vichy) et qui apparaissait dans l'ancien article 63, alinéa 2 du C.Civ. Si l'un des époux a un problème de santé important et le dissimule à l'autre, alors si l'on considère que l'autre conjoint n'aurait pas accepté le mariage en connaissance de cause, celui-ci peut invoquer l'article 180, alinéa 2.


Concernant l'âge, en 1804, le mariage était autorisé à 15 ans pour une femme et 18 ans pour l'homme (art. 144 ancien, voir ci dessus). Il faut attendre la loi du 4 avril 2006 pour qu'on modifie ce principe. Cette loi est intéressante car elle avait pour objectif de renforcé la prévention et la répression des violences au sein du couple ou commises contre les mineurs donc contre les mariages forcés (à l'égard des femmes mineurs). L'âge est donc de 18 ans, peu importe le sexe. \\
Cependant, l'article 145 pose une exception en cas de motifs graves. Il faut une dispense d'âge accordée par le procureur de la République. Il faut obtenir, à côté de la dispense d'âge, l'autorisation des parents du futur époux mineur. On remarquera que la Loi ne précise pas les motifs graves, le procureur apprécie donc ceux-ci. Traditionnellement, on considère comme motif grave le fait qu'une femme soit enceinte, mais il est à précisé que le procureur sera très vigilant. 


I- Les éléments d'ordre familial


1. Les autorisations au mariage


a. Le mariage d'un mineur incapable \\
Le mineur en droit français est celui ou celle qui a moins de 18 ans. Cependant, il faut distinguer les mineurs émancipés. À partir de 16 ans révolus, avec le consentement parental, le mineur peut être considéré comme un mineur émancipé. La conséquence de cet état est donné à l'article 413-6, alinéa premier "Le mineur émancipé est capable, comme un majeur, de tous les actes de la vie civile. § Il doit néanmoins pour se marier ou se donner en adoption, observer les mêmes règles que si il n'était point émancipé". \\
Pour le consentement parental, le C.Civ prévoit toute une série d'hypothèse. Le point de départ est donné à l'article 148 du C.Civ. Même article qui pose la première hypothèse du C.Civ: si les parents ne sont pas d'accord, le consentement est donné. Si l'un des parents est mort ou incapable de manifester sa volonté, on voit l'article 149 qui dit que le consentement de l'un suffit. \\
Si les deux parents sont dans l'impossibilité de manifester leur volonté, ce sera aux aïeux de donner leur accord (art. 150). Un aïeul est un ascendant en ligne directe paternelle ou maternelle au deuxième, troisième et quatrième degré de parenté. \\
Si il n'y a ni père, ni mère ni aïeux pouvant exprimer leur volonté, on fait appel au conseil de famille et obtenir son consentement. 


b. Le mariage d'un majeur incapable \\
Il faut partir de l'art. 425 du C.Civ. Toute personne dans l'incapacité d'exprimer sa volonté bénéficie d'une protection juridique. \\
Trois variantes de protection juridique: sauvegarde de justice ; curatelle ; tutelle. \\
La sauvegarde de justice est prévu par les article 433 et suivants du C.Civ. Cette sauvegarde de justice va permettre de protéger la personne en question sans la dessaisir de ses prérogatives juridiques. Par exemple une personne très âgée, légèrement fragilisée, va pouvoir continuer d'agir comme elle le veut mais pourra obtenir des annulations de vente par exemple ou des meilleures conditions contractuelles. En matière de mariage, cependant, la sauvegarde de justice n'impose ni ne donne rien de plus à la personne. \\
Lorsque l'on a une personne placée sous curatelle, elle est altérée légèrement de ses capacités mentales. Les articles 440 et suivants du C.Civ dispose de la curatelle. La personne sous curatelle a besoin d'être assistée ou contrôlée dans les actes importants de la vie civile. L'article 460 du C.Civ dispose que le mariage d'une personne sous curatelle ne se fait que sur autorisation du curateur ou à défaut, du juge des tutelles. \\
Lorsqu'une personne est placée sous tutelle, la personne souffre de troubles graves affectant son discernement et ses capacités intellectuelles. Cette personne, nous dit le C.Civ à l'article 440, a besoin d'être représentée dans les actes de la vie civile. Le tuteur va représenter la personne. Le mariage d'une personne en tutelle, n'est permis qu'avec l'autorisation du juge ou du conseil de famille si il a été constitué. Même si par principe, rien ne fait obstacle au mariage, en pratique, une difficulté va se posé: se rendre compte si oui ou non la personne placée sous tutelle prend conscience de la portée de son engagement. 


2. Les prohibitions au mariage


a. La bigamie \\
Dans un dictionnaire, la bigamie se définit comme "l'état d'une personne qui, étant engagé dans les liens du mariage, en a contracté un autre avant la dissolution du précédent". La polygamie, au même titre que la bigamie est prohibée en droit français. \\
Il faut se reporter à l'art. 147 du C.Civ qui dispose "on ne peut contracter un second mariage avant la dissolution du premier". La polygamie est poursuivi sur le terrain du pénal: art 433-20 du Code Pénal: "le fait pour une personne engagée dans les liens du mariage d'en contracté un autre avant la dissolution du précédent est puni d'un an d'emprisonnement et 45 000€ d'amendes". La loi pénale punit l'officier public qui aurait, en toute connaissance de cause, procédé à un tel mariage, des mêmes peines. 


b. L'inceste \\
L'inceste, dans le dictionnaire juridique Cornu, désigne tout à la fois une union prohibé et des relations sexuelles entre proches parents. \\
Dans certaines situations, en particulier quand l'un des partenaires est majeur et l'autre mineur, le code pénal envisage des poursuites. \\
Sur le mariage entre proche civil, il faut distinguer deux zones: une première zone où il y a inceste absolu (zone d'horreur) et une deuxième zone, dit inceste relatif (zone d'accommodement). 


L'article 161 dispose que les ascendants et descendants ne peuvent pas se marier. C'est la zone d'horreur. \\
Le C.Civ pense à tout car prohibe aussi le mariage en ligne collatéral (art. 162). \\
Concernant le droit de la filiation si il y a un enfant incestueux, il faut regarder l'article 310-2 qui dispose que l'on ne peut pas établir de filiation qui ferait apparaître publiquement l'origine incestueuse. Une filiation est une reconnaissance car en reconnaissant l'enfant, on établit un lien de filiation. Si dans le cadre d'un inceste, on laisserait les deux parents reconnaître l'enfant, on aurai la preuve publique apparente que l'enfant est né d'un inceste. L'enfant sera donc condamné juridiquement à n'avoir qu'un seul parent. 


Il y a deux cas dans la zone d'accommodement, un premier où l'on revient à l'article 161 du C.Civ. Le PR peut autorisé des mariages pour des causes graves en alliés de ligne directe lorsque la personne qui a créé l'alliance est décédé (art 164). \\
À partir du moment où il y a un mariage, il y a une famille avec laquelle on a un lien car alliance. Ce sont juridiquement des alliés. Il y a des liens d'alliance direct (beaux-parents) et des alliés à différents degrés.  Le mariage avec les alliés direct est interdit sauf quand la personne qui a créée l'alliance est décédée et que le PR a donnée son autorisation. Si la personne qui a formée l'alliance n'est pas décédé, le PR ne peut pas donner sa dispense à cette règle. \\ 
Le législateur a voulu donner une certaine souplesse au nom de ce qu'on appelle la paix des familles. \\
La deuxième situation d'accommodement est entre un oncle/tante et le neveu/nièce où le Président peut donner son consentement en cas de causes graves. 


II. Les conditions relatives au consentement


A. Consentement intègre


1. Chacun des époux a-t-il consenti au mariage ?


Le point de départ est l'article 146 du C.Civ: "Il n'y a point de mariage lorsqu'il n'y a point de consentement". \\
Lors d'une célébration, l'officier public pose la question de prendre ou non pour époux la personne présente. Le "oui" rituel donné par les époux est donc la preuve du consentement. Une fois que l'officier a ce double consentement, il les déclare unis par les liens du mariage. \\
Pour les personnes sous tutelle ou sous curatelle, c'est qu'elles ont une atteinte quant à leurs facultés mentales. Il leur faut donc un consentement particulier que nous avons vu plus haut. Cependant, reste un problème: est-ce que, au moment où l'officier public pose la question, la personne aura assez de discernement pour répondre ? 


Une première série de difficulté peut être levée si l'officier de l'état civil se rend compte de quelque chose. Si une des deux personnes n'est pas en état d'exprimer son consentement comme être soul/drogué voir autres. Si c'est apparent, l'officier de l'état civil peut refuser le mariage. \\
La difficulté persiste si le consentement n'est que d'apparence. Il y a deux règles qui doivent nous aider lorsqu'il y a problème. Il y a problème une fois la cérémonie faite et que quelqu'un demande la nullité du mariage car il n'y a pas eu consentement. Si il y a défaut de consentement, le mariage peut être purement et simplement annulée. \\
Il y a deux règles jurisprudentielles: c'est à celui qui conteste la validité du mariage d'apporter la preuve du défaut de consentement. Ce ne sont donc pas aux personnes mariées de prouver qu'ils avaient toute leur lucidité. Cela résulte d'une jurisprudence constante rendu par la Civ 1ère le 2 Décembre 1992. Deuxième règle donnée le 30 Novembre 1965: une fois que la preuve est apportée, les juges du fond ont un pouvoir d'appréciation souverain.


Deuxième série de difficulté: le mariage in extremis, aux portes de la mort, le mariage d'un mourant. \\
On peut se demander si la personne en fin de vie consent effectivement à un mariage. Le C.Civ parle de cette situation, mais seulement au niveau de la célébration si l'un des époux est mourant. L'article 75, en assouplissant les règles de célébration pour un mourant, autorise implicitement le mariage d'un mourant. \\
La difficulté dans ce mariage in extremis est que, puisque la personne est mourante, elle aurait réellement des problèmes à exprimer ses volontés. Dans une décision du 22 Janvier 1968, la C.Cas nous dit: "Si, lors de la cérémonie du mariage, l'un des époux ne peut parler, il appartient au juge de relever et d'interpréter les signes (larmes, regards, attitude), par lesquels cet époux a entendu affirmer sa volonté". 


2. Chacun des époux a-t-il consenti librement au mariage ?


Maintenant, l'hypothèse est différente. Si quelqu'un, pour n'importe quel raison, force une personne à se marier: le consentement est d'apparence mais est néanmoins forcé. \\
On doit donc consentir au mariage de façon totalement libre. Lorsque l'on pense au consentement libre, il faut se référer à deux articles: d'abord, le 146, mais aussi l'article 180. \\
Deux affirmations nous intéresses: si il n'y a pas de consentement libre, le mariage peut être attaqué. Et si il y a eu contrainte, le mariage est nul. \\
Le consentement qui n'a pas été obtenu librement est un consentement qui a pu être obtenu avec violence, ou par contrainte. On entend violence, violence physique ou violence morale, la contrainte morale. \\
La crainte révérencielle est une veille notion, qui existait déjà en droit romain: le père avait droit de vie ou de mort sur les enfants dans la Rome antique. La crainte révérencielle vient donc du respect pour ses ascendants, le respect dû (voir art. 371). \\
La personne qui s'est mariée par crainte de décevoir ses parents peut faire annuler son mariage, ce qui n'était pas le cas avant la loi du 4 Avril 2006 et ce qui n'est toujours pas le cas dans les contrats (art. 1114). 


B. Un consentement réel 


Si il n'y a pas de violence, de pression, et si il y a lucidité, la vie pourrait être belle, mais on va être confronté à deux types de problèmes. \\
Premier type de problème: si deux personnes disent oui mais sans se comporter en époux, pour, par exemple, bénéficier d'un avantage, et font donc, semblant d'être époux. \\
Deuxième type de problème: les époux se marient et l'un des deux épouse l'autre car il y a chez l'autre un élément déterminant de son consentement qui finalement est faux. 





\subsubsection{Les sanctions des conditions de validité}


\subsection{Les effets du mariage}









\end{document}
