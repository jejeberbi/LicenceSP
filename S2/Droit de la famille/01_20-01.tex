\documentclass[12pt, a4paper, openany]{book}

\usepackage[latin1]{inputenc}
\usepackage[T1]{fontenc}
\usepackage[francais]{babel}
\date{}
\title{Cours de Droit Civil (UFR Amiens)}
\pagestyle{plain}

\begin{document}

\chapter{Introduction Générale du droit de la famille}

Pour parler d'un droit de la famille, il faut d'abord parler de la famille: on pourrai voir la famille dans l'histoire, dans la sociologie, ou encore dans l'affectif. Est-ce que la famille contemporaine est la même que la famille du Moyen-âge ? La notion de famille a infiniment évolué. \\
Au moment où on promulgue le Code Civil en 1804, la famille est un couple d'un homme et d'une femme mariés. L'idée du mariage est proche de l'idée religieuse: on se marie pour la vie et au delà. Le divorce est peu ancré et apparaît définitivement en 1884 dans le Code Civil. On parle d'enfants légitime quand les parents sont mariés. Pour les autres, le code parlera d'enfants naturels. \\
Aujourd'hui, le mariage existe encore certes, mais concerne les couples hétéro comme les couples homos. De plus, d'autres modèles conjugaux se sont créés: le PACS notamment mais aussi le concubinage qui a été introduit dans le Code. Le divorce existe bien sûr, mais a été fortement facilité.  La distinction enfant légitime/naturel a été supprimée. Ensuite, devant la facilité à se marié et à divorcé, c'est devenu de plus en plus courant de ne pas passer sa vie avec une seule personne (entre 40 et 50\% des mariages finissent par un divorce). Des familles recomposés sont aussi apparus. \\
Il y a donc aujourd'hui beaucoup de possibilités de famille. Certains parlent péjorativement de "plat à la carte". Il y a donc pluralité de situations: l'enjeu de la loi est donc de cadré le champ des possibles, il n'y a pas de modèle. La fonction du législateur serait donc de veiller à ce que il n'y ait pas de situations déséquilibrées. Le législateur cherche donc à encadré les champs de liberté pour pas qu'il y ait de situations désavantageuse pour quelqu'un.


Le droit de la famille est un droit vivant, mais aussi très technique. Parler du droit de la famille à notre époque, c'est surtout parlé des couples et donc aussi de leurs enfants si ils en ont. 

\part{Les couples}

Aujourd'hui, il y a trois types de couples: les couples mariés, les couples Pacsés, ainsi que les couples en concubinage.

\chapter{Les couples mariés}

\section{Le mariage}

\subsection{La validité du mariage}

\subsubsection{Les conditions de validités}

Les conditions de fond sont relatives à l'état des personnes ainsi qu'au consentement.


Les conditions relatives à l'État des personnes: \\
Éléments d'ordre individuel: le sexe et l'état de santé. \\
Dans la version de 1804, on ne dit pas explicitement que le mariage est l'union d'un homme et d'une femme. Le mariage est cependant l'union d'un homme et d'une femme: c'est évident car malgré la révolution française, malgré la laïcisation du Droit Civil (et donc du droit de la famille), le mariage dans la conception du Code Civil est pratiquement la résultante de la conception religieuse catholique. Puisque les choses sont implicites, l'article 144 ancien du C.Civ nous dit "L'homme avant 18 ans révolus, la femme avant 15 ans révolus, ne peuvent contracter mariage", c'est le seul qui nous permettent de deviner que le mariage a la conception d'une union d'un homme et d'une femme. \\
La loi du 17 Mai 2013 est une loi qui va aboutir à écarter l'idée que le mariage repose sur la différenciation des sexes. Depuis cette loi, l'union peut être d'un homme et d'une femme, de deux hommes ou deux femmes. Le Conseil Constitutionnel a du vérifier la loi ; dans sa décision du 17 Mai 2013 dit que si les lois avant 2013 voyaient le mariage comme l'union d'un homme et d'une femme, cette règle ne peut constituer un principe fondamental de la République au sens du préambule de 1946. 


\subsubsection{Les sanctions des conditions de validité}


\subsection{Les effets du mariage}









\end{document}
