\documentclass[12pt, a4paper, openany]{book}

\usepackage[latin1]{inputenc}
\usepackage[T1]{fontenc}
\usepackage[francais]{babel}
\date{}
\title{Cours de Droit Civil (UFR Amiens)}
\pagestyle{plain}

\begin{document}

\chapter{Introduction Générale du droit de la famille}

Pour parler d'un droit de la famille, il faut d'abord parler de la famille: on pourrai voir la famille dans l'histoire, dans la sociologie, ou encore dans l'affectif. Est-ce que la famille contemporaine est la même que la famille du Moyen-âge ? La notion de famille a infiniment évolué. \\
Au moment où on promulgue le Code Civil en 1804, la famille est un couple d'un homme et d'une femme mariés. L'idée du mariage est proche de l'idée religieuse: on se marie pour la vie et au delà. Le divorce est peu ancré et apparaît définitivement en 1884 dans le Code Civil. On parle d'enfants légitime quand les parents sont mariés. Pour les autres, le code parlera d'enfants naturels. \\
Aujourd'hui, le mariage existe encore certes, mais concerne les couples hétéro comme les couples homos. De plus, d'autres modèles conjugaux se sont créés: le PACS notamment mais aussi le concubinage qui a été introduit dans le Code. Le divorce existe bien sûr, mais a été fortement facilité.  La distinction enfant légitime/naturel a été supprimée. Ensuite, devant la facilité à se marié et à divorcé, c'est devenu de plus en plus courant de ne pas passer sa vie avec une seule personne (entre 40 et 50\% des mariages finissent par un divorce). Des familles recomposés sont aussi apparus. \\
Il y a donc aujourd'hui beaucoup de possibilités de famille. Certains parlent péjorativement de "plat à la carte". Il y a donc pluralité de situations: l'enjeu de la loi est donc de cadré le champ des possibles, il n'y a pas de modèle. La fonction du législateur serait donc de veiller à ce que il n'y ait pas de situations déséquilibrées. Le législateur cherche donc à encadré les champs de liberté pour pas qu'il y ait de situations désavantageuse pour quelqu'un.


Le droit de la famille est un droit vivant, mais aussi très technique. Parler du droit de la famille à notre époque, c'est surtout parlé des couples et donc aussi de leurs enfants si ils en ont. 

\part{Les couples}

Aujourd'hui, il y a trois types de couples: les couples mariés, les couples Pacsés, ainsi que les couples en concubinage.

\chapter{Les couples mariés}

\section{Le mariage}

\subsection{La validité du mariage}

\subsubsection{Les conditions de validités}

§1. Les conditions de fond.


Les conditions relatives à l'État des personnes: \\
Éléments d'ordre individuel: le sexe et l'état de santé.


Dans la version de 1804, on ne dit pas explicitement que le mariage est l'union d'un homme et d'une femme. Le mariage est cependant l'union d'un homme et d'une femme: c'est évident car malgré la révolution française, malgré la laïcisation du Droit Civil (et donc du droit de la famille), le mariage dans la conception du Code Civil est pratiquement la résultante de la conception religieuse catholique. Puisque les choses sont implicites, l'article 144 ancien du C.Civ nous dit "L'homme avant 18 ans révolus, la femme avant 15 ans révolus, ne peuvent contracter mariage", c'est le seul qui nous permettent de deviner que le mariage a la conception d'une union d'un homme et d'une femme. \\
La loi du 17 Mai 2013 est une loi qui va aboutir à écarter l'idée que le mariage repose sur la différenciation des sexes. Depuis cette loi, l'union peut être d'un homme et d'une femme, de deux hommes ou deux femmes. Le Conseil Constitutionnel a du vérifier la loi ; dans sa décision du 17 Mai 2013 dit que si les lois avant 2013 voyaient le mariage comme l'union d'un homme et d'une femme, cette règle ne peut constituer un principe fondamental de la République au sens du préambule de 1946. Le sexe est donc indifférent.


Concernant l'état de santé, jusqu'à la loi du 20 Décembre 2007 (loi relative à la simplification du Droit), les futurs époux avaient obligation de produire en vu du mariage un certificat prénuptial (ce certificat avait été instauré pendant le régime de Vichy) et qui apparaissait dans l'ancien article 63, alinéa 2 du C.Civ. Si l'un des époux a un problème de santé important et le dissimule à l'autre, alors si l'on considère que l'autre conjoint n'aurait pas accepté le mariage en connaissance de cause, celui-ci peut invoquer l'article 180, alinéa 2.


Concernant l'âge, en 1804, le mariage était autorisé à 15 ans pour une femme et 18 ans pour l'homme (art. 144 ancien, voir ci dessus). Il faut attendre la loi du 4 avril 2006 pour qu'on modifie ce principe. Cette loi est intéressante car elle avait pour objectif de renforcé la prévention et la répression des violences au sein du couple ou commises contre les mineurs donc contre les mariages forcés (à l'égard des femmes mineurs). L'âge est donc de 18 ans, peu importe le sexe. \\
Cependant, l'article 145 pose une exception en cas de motifs graves. Il faut une dispense d'âge accordée par le procureur de la République. Il faut obtenir, à côté de la dispense d'âge, l'autorisation des parents du futur époux mineur. On remarquera que la Loi ne précise pas les motifs graves, le procureur apprécie donc ceux-ci. Traditionnellement, on considère comme motif grave le fait qu'une femme soit enceinte, mais il est à précisé que le procureur sera très vigilant. 


I- Les éléments d'ordre familial


1. Les autorisations au mariage


a. Le mariage d'un mineur incapable \\
Le mineur en droit français est celui ou celle qui a moins de 18 ans. Cependant, il faut distinguer les mineurs émancipés. À partir de 16 ans révolus, avec le consentement parental, le mineur peut être considéré comme un mineur émancipé. La conséquence de cet état est donné à l'article 413-6, alinéa premier "Le mineur émancipé est capable, comme un majeur, de tous les actes de la vie civile. § Il doit néanmoins pour se marier ou se donner en adoption, observer les mêmes règles que si il n'était point émancipé". \\
Pour le consentement parental, le C.Civ prévoit toute une série d'hypothèse. Le point de départ est donné à l'article 148 du C.Civ. Même article qui pose la première hypothèse du C.Civ: si les parents ne sont pas d'accord, le consentement est donné. Si l'un des parents est mort ou incapable de manifester sa volonté, on voit l'article 149 qui dit que le consentement de l'un suffit. \\
Si les deux parents sont dans l'impossibilité de manifester leur volonté, ce sera aux aïeux de donner leur accord (art. 150). Un aïeul est un ascendant en ligne directe paternelle ou maternelle au deuxième, troisième et quatrième degré de parenté. \\
Si il n'y a ni père, ni mère ni aïeux pouvant exprimer leur volonté, on fait appel au conseil de famille et obtenir son consentement. 


b. Le mariage d'un majeur incapable \\
Il faut partir de l'art. 425 du C.Civ. Toute personne dans l'incapacité d'exprimer sa volonté bénéficie d'une protection juridique. \\
Trois variantes de protection juridique: sauvegarde de justice ; curatelle ; tutelle. \\
La sauvegarde de justice est prévu par les article 433 et suivants du C.Civ. Cette sauvegarde de justice va permettre de protéger la personne en question sans la dessaisir de ses prérogatives juridiques. Par exemple une personne très âgée, légèrement fragilisée, va pouvoir continuer d'agir comme elle le veut mais pourra obtenir des annulations de vente par exemple ou des meilleures conditions contractuelles. En matière de mariage, cependant, la sauvegarde de justice n'impose ni ne donne rien de plus à la personne. \\
Lorsque l'on a une personne placée sous curatelle, elle est altérée légèrement de ses capacités mentales. Les articles 440 et suivants du C.Civ dispose de la curatelle. La personne sous curatelle a besoin d'être assistée ou contrôlée dans les actes importants de la vie civile. L'article 460 du C.Civ dispose que le mariage d'une personne sous curatelle ne se fait que sur autorisation du curateur ou à défaut, du juge des tutelles. \\
Lorsqu'une personne est placée sous tutelle, la personne souffre de troubles graves affectant son discernement et ses capacités intellectuelles. Cette personne, nous dit le C.Civ à l'article 440, a besoin d'être représentée dans les actes de la vie civile. Le tuteur va représenter la personne. Le mariage d'une personne en tutelle, n'est permis qu'avec l'autorisation du juge ou du conseil de famille si il a été constitué. Même si par principe, rien ne fait obstacle au mariage, en pratique, une difficulté va se posé: se rendre compte si oui ou non la personne placée sous tutelle prend conscience de la portée de son engagement. 


2. Les prohibitions au mariage


a. La bigamie \\
Dans un dictionnaire, la bigamie se définit comme "l'état d'une personne qui, étant engagé dans les liens du mariage, en a contracté un autre avant la dissolution du précédent". La polygamie, au même titre que la bigamie est prohibée en droit français. \\
Il faut se reporter à l'art. 147 du C.Civ qui dispose "on ne peut contracter un second mariage avant la dissolution du premier". La polygamie est poursuivi sur le terrain du pénal: art 433-20 du Code Pénal: "le fait pour une personne engagée dans les liens du mariage d'en contracté un autre avant la dissolution du précédent est puni d'un an d'emprisonnement et 45 000€ d'amendes". La loi pénale punit l'officier public qui aurait, en toute connaissance de cause, procédé à un tel mariage, des mêmes peines. 


b. L'inceste \\
L'inceste, dans le dictionnaire juridique Cornu, désigne tout à la fois une union prohibé et des relations sexuelles entre proches parents. \\
Dans certaines situations, en particulier quand l'un des partenaires est majeur et l'autre mineur, le code pénal envisage des poursuites. \\
Sur le mariage entre proche civil, il faut distinguer deux zones: une première zone où il y a inceste absolu (zone d'horreur) et une deuxième zone, dit inceste relatif (zone d'accommodement). 


L'article 161 dispose que les ascendants et descendants ne peuvent pas se marier. C'est la zone d'horreur. \\
Le C.Civ pense à tout car prohibe aussi le mariage en ligne collatéral (art. 162). \\
Concernant le droit de la filiation si il y a un enfant incestueux, il faut regarder l'article 310-2 qui dispose que l'on ne peut pas établir de filiation qui ferait apparaître publiquement l'origine incestueuse. Une filiation est une reconnaissance car en reconnaissant l'enfant, on établit un lien de filiation. Si dans le cadre d'un inceste, on laisserait les deux parents reconnaître l'enfant, on aurai la preuve publique apparente que l'enfant est né d'un inceste. L'enfant sera donc condamné juridiquement à n'avoir qu'un seul parent. 


Il y a deux cas dans la zone d'accommodement, un premier où l'on revient à l'article 161 du C.Civ. Le PR peut autorisé des mariages pour des causes graves en alliés de ligne directe lorsque la personne qui a créé l'alliance est décédé (art 164). \\
À partir du moment où il y a un mariage, il y a une famille avec laquelle on a un lien car alliance. Ce sont juridiquement des alliés. Il y a des liens d'alliance direct (beaux-parents) et des alliés à différents degrés.  Le mariage avec les alliés direct est interdit sauf quand la personne qui a créée l'alliance est décédée et que le PR a donnée son autorisation. Si la personne qui a formée l'alliance n'est pas décédé, le PR ne peut pas donner sa dispense à cette règle. \\ 
Le législateur a voulu donner une certaine souplesse au nom de ce qu'on appelle la paix des familles. \\
La deuxième situation d'accommodement est entre un oncle/tante et le neveu/nièce où le Président peut donner son consentement en cas de causes graves. 


II. Les conditions relatives au consentement


A. Consentement intègre


1. Chacun des époux a-t-il consenti au mariage ?


Le point de départ est l'article 146 du C.Civ: "Il n'y a point de mariage lorsqu'il n'y a point de consentement". \\
Lors d'une célébration, l'officier public pose la question de prendre ou non pour époux la personne présente. Le "oui" rituel donné par les époux est donc la preuve du consentement. Une fois que l'officier a ce double consentement, il les déclare unis par les liens du mariage. \\
Pour les personnes sous tutelle ou sous curatelle, c'est qu'elles ont une atteinte quant à leurs facultés mentales. Il leur faut donc un consentement particulier que nous avons vu plus haut. Cependant, reste un problème: est-ce que, au moment où l'officier public pose la question, la personne aura assez de discernement pour répondre ? 


Une première série de difficulté peut être levée si l'officier de l'état civil se rend compte de quelque chose. Si une des deux personnes n'est pas en état d'exprimer son consentement comme être soul/drogué voir autres. Si c'est apparent, l'officier de l'état civil peut refuser le mariage. \\
La difficulté persiste si le consentement n'est que d'apparence. Il y a deux règles qui doivent nous aider lorsqu'il y a problème. Il y a problème une fois la cérémonie faite et que quelqu'un demande la nullité du mariage car il n'y a pas eu consentement. Si il y a défaut de consentement, le mariage peut être purement et simplement annulée. \\
Il y a deux règles jurisprudentielles: c'est à celui qui conteste la validité du mariage d'apporter la preuve du défaut de consentement. Ce ne sont donc pas aux personnes mariées de prouver qu'ils avaient toute leur lucidité. Cela résulte d'une jurisprudence constante rendu par la Civ 1ère le 2 Décembre 1992. Deuxième règle donnée le 30 Novembre 1965: une fois que la preuve est apportée, les juges du fond ont un pouvoir d'appréciation souverain.


Deuxième série de difficulté: le mariage in extremis, aux portes de la mort, le mariage d'un mourant. \\
On peut se demander si la personne en fin de vie consent effectivement à un mariage. Le C.Civ parle de cette situation, mais seulement au niveau de la célébration si l'un des époux est mourant. L'article 75, en assouplissant les règles de célébration pour un mourant, autorise implicitement le mariage d'un mourant. \\
La difficulté dans ce mariage in extremis est que, puisque la personne est mourante, elle aurait réellement des problèmes à exprimer ses volontés. Dans une décision du 22 Janvier 1968, la C.Cas nous dit: "Si, lors de la cérémonie du mariage, l'un des époux ne peut parler, il appartient au juge de relever et d'interpréter les signes (larmes, regards, attitude), par lesquels cet époux a entendu affirmer sa volonté". 


2. Chacun des époux a-t-il consenti librement au mariage ?


Maintenant, l'hypothèse est différente. Si quelqu'un, pour n'importe quel raison, force une personne à se marier: le consentement est d'apparence mais est néanmoins forcé. \\
On doit donc consentir au mariage de façon totalement libre. Lorsque l'on pense au consentement libre, il faut se référer à deux articles: d'abord, le 146, mais aussi l'article 180. \\
Deux affirmations nous intéresses: si il n'y a pas de consentement libre, le mariage peut être attaqué. Et si il y a eu contrainte, le mariage est nul. \\
Le consentement qui n'a pas été obtenu librement est un consentement qui a pu être obtenu avec violence, ou par contrainte. On entend violence, violence physique ou violence morale, la contrainte morale. \\
La crainte révérencielle est une veille notion, qui existait déjà en droit romain: le père avait droit de vie ou de mort sur les enfants dans la Rome antique. La crainte révérencielle vient donc du respect pour ses ascendants, le respect dû (voir art. 371). \\
La personne qui s'est mariée par crainte de décevoir ses parents peut faire annuler son mariage, ce qui n'était pas le cas avant la loi du 4 Avril 2006 et ce qui n'est toujours pas le cas dans les contrats (art. 1114). 


B. Un consentement réel 


Si il n'y a pas de violence, de pression, et si il y a lucidité, la vie pourrait être belle, mais on va être confronté à deux types de problèmes. \\
Premier type de problème: si deux personnes disent oui mais sans se comporter en époux, pour, par exemple, bénéficier d'un avantage, et font donc, semblant d'être époux. \\
Deuxième type de problème: les époux se marient et l'un des deux épouse l'autre car il y a chez l'autre un élément déterminant de son consentement qui finalement est faux. 


1. Le cas du mariage simulé


On parlera de mariage simulé dès l'instant où le mariage est, certes, célébré (il y a consentement intègre) mais il n'existe pas de réel intention matrimoniale: le fait de se marier réellement avec l'intention de respecter et de mettre en oeuvre les règles inhérentes au mariage. \\
Dans la pratique, on a deux expressions. Elles sont intéressantes car nous montre deux hypothèses. On parle parfois de mariage blanc, ou encore de mariage gris dans le langage courant. \\
Le mariage blanc évoque cette situation de mariage simulé avec accord des deux époux: les deux sont d'accord pour simuler le fait d'être mariés. En revanche, dans le mariage gris, l'un des deux a l'intention de se comporter en conjoint, l'autre conjoint n'en a pas l'intention et le cache.


Dans la situation du mariage blanc, la jurisprudence considère que dans certains cas, il y a possibilité d'annuler le mariage, quand les deux époux ont été d'accord pour simulé. \\
L'on peut rajouter dans l'interprétation de l'article 146: "Il n'y a pas de mariage lorsqu'il n'y a point de consentement simulé". \\
La jurisprudence nous dit que ce qui est important lorsque les époux consentent au mariage, c'est de savoir quels sont les buts véritables qu'ils poursuivent. Si le but véritable poursuivi par les époux est de ne pas se comporter en personnes mariées, de créer une communauté de vie et de résidence: et bien il y aura nullité du mariage sur la base de l'article 146 du C.Civ. \\
Si les époux ont comme but à la fois de se comporter en époux mais aussi un autre but: donner la nationalité à un des époux par exemple ; cela ne pose aucun problème pour les juges. \\
La question essentielle est donc, pour le juge, si les époux veulent se comporter comme tel. Cette question est apprécié souverainement par les juges du fond: Civ. 1ère, 12 Novembre 1998 "La détermination des buts véritables poursuivis par les époux relève de l'appréciation souveraine des juges du fond". 


Dans la situation du mariage gris, il faut se dire que dans cette situation, contrairement à la situation précédente: il y a un conjoint coupable et un conjoint innocent. La victime va donc essayer d'obtenir la nullité du mariage ; la société peut la demander aussi. \\
Le conjoint innocent pourra invoquer deux articles: l'article 146 mais aussi l'erreur visé à l'article 180. 


2. Le consentement donné par erreur


En partant de l'exemple du mariage gris, lorsque l'époux innocent a consenti au mariage, il était fondamentalement important pour lui que les règles soient respectées et le mariage mis en oeuvre. \\
Dans la mesure où celui-ci a été trompé par le conjoint coupable, il peut se poser la question si l'époux innocent a commis une erreur. Si il avait connu les intentions réelles, il ne se serait pas marié: voilà pourquoi on peut parler d'erreur. Quand l'erreur est admise, elle entraîne la nullité du mariage. \\
Si il y a eu erreur sur la personne ou dans les qualités essentielles de celle-ci, il peut y avoir annulation (art. 180). \\
L'art. 180 va aboutir à des situations où l'erreur sera prise en compte.


a. L'erreur indifférente


Une erreur indifférente est une situation où, certes, il y a erreur, mais elle n'est pas admise par les tribunaux. \\
La difficulté provient du fait que lorsque l'on parle de mariage, on parle d'amour donc de séduction. Il y a des personnes franches et d'autres qui ne le sont pas vraiment voir pas du tout. Il existe une formule datant de la fin du XVIe et début du XVIIe, formule de Loysel (1536-1617) qui rédigeait les coutumes et essayait de rédiger des formules plus facile à retenir. Il dit donc "On fait de mariage trompe qui peux". Dans certains cas, cette formule garde toute sa pertinence. \\
Il existe donc de nombreuses erreurs qui ne sont pas suffisamment graves pour annuler un mariage. Tout ce qui relève donc de la séduction ne ont pas des erreurs suffisantes pour annuler un mariage. \\
En revanche, il existe des situations qui vont à l'encontre de la formule de Loysel où un époux a trompé l'autre sur des qualités qui sont considérées comme importantes. 


b. Les erreurs prises en compte


Si il y a erreur dans la personne ou sur une des qualités fondamentales de la personne, le mariage peut être annulé. C'est ce que nous dis l'art. 180, alinéa 2. \\
Depuis la loi du 11 Juilet 1975, on distingue deux types d'erreur dans le mariage: d'une part, l'erreur dans la personne, d'autre part, l'erreur de qualité sur la personne. \\
Dans la version de 1804 ou dans la version de 1975, on parle d'erreur dans la personne. Mais ce n'est que depuis 1975 que l'on parlera d'erreurs sur les qualités essentielles sur la personne. La loi de 1975 a donc augmenté le nombre de cas où l'on pouvait annuler le mariage. \\
La jurisprudence ancienne est une jurisprudence basé sur l'ancien alinéa 2 de l'art. 180, c'est à dire, basé uniquement sur la notion d'erreur dans la personne. Lorsque l'on parle d'erreur dans la personne, on envisage l'erreur sur l'identité physique: la substitution d'une personne à une autre (jumeaux/jumelles) ; Carbonnier dit "La substitution d'une femme à une autre, femme sous le voile". \\
Erreur sur l'identité civile: c'est de se donner un état civil qui ne correspond pas à la réalité, notamment où la fausse identité civile est flatteuse pour celui qui l'utilise (quand on s'attribue une qualité de noble par exemple). Autre hypothèse: un des deux conjoints utiliserait un nom pour faire croire qu'il est d'une famille illustre (fils d'un grand sportif, d'un chanteur etc.).


L'affaire Berthon est une affaire célèbre. Mr Berthon avait, à 17 ans, été condamné pour complicité d'assassinat à 15 ans de travaux forcés. Pour bonne conduite, il fut libéré 12 ans plus tard, donc à ses 29 ans. Il rencontre là une demoiselle Zoé avec qui il se marie. Berthon n'a jamais dis à Zoé qu'il était bagnard. Celle-ci l'apprend un peu plus tard et veut donc engager une action en nullité du mariage. \\
Il va y avoir un très long procès: deux passages devant la C.Cas et donc un arrêt rendu en assemblé plénière (les Chambres réunis à l'époque), le 24 Avril 1862,\footnote{DP 1862, I, p.53} cette décision va aboutir au final à refuser de prononcer la nullité du mariage invoqué par l'épouse. En effet, celle-ci, considérait que l'erreur qu'elle avait commise était d'avoir épousé sans le savoir un ancien bagnard, pouvait entraîner la nullité car celle-ci disait avoir commis une erreur dans la personne. \\
La C.Cas ne l'entendra pas ainsi. Elle va s'appuyer sur l'ancien art. 180 du C.Civ et va interpréter restrictivement l'alinéa 2 ancien. Ce n'est pas une erreur dans la personne car ce n'est ni une erreur sur l'identité physique, ni une erreur sur l'identité civile. "Attendu que si la nullité ainsi établi ne doit pas être restreinte au cas unique de l'erreur provenant d'une substitution frauduleuse de personnes au moment de la célébration ; attendu que si elle peut également recevoir son application quand l'erreur procède de ce que l'un des époux s'est fait agréé en se présentant comme membre d'une famille qui n'est pas la sienne et s'est attribué les conditions d'origine et la filiation qui appartient un autre, cette erreur n'est admissible qu'autant qu'elle a pour résultat d'amener l'une des parties à épousé une personne autre que celle à qui elle croyait s'unir". \\
En l'espèce, la C.Cas va considéré qu'il n'y a pas d'erreur sur l'identité physique, ni sur l'identité civile. L'application est donc très restrictive. \\
Avant que la loi du 11 Juillet 1975, les tribunaux vont, petit à petit, commencer à parler d'erreur sur les qualités essentielles sur la personne. L'article 180, alinéa 2 dans sa nouvelle rédaction est donc le fruit d'un changement de jurisprudence. La loi du 11 Juillet 1975 va donc consacrer l'erreur sur la sur les qualités essentielles de la personne.


On a plus d'exemples concernant des juridictions de première ou seconde instance que de la C.Cas concernant les erreurs sur les qualités essentielles. La situation dans laquelle un époux avait épousé une prostitué a été une cause de nullité de mariage, car si l'époux avait su qu'il épousait une prostitué, il ne l'aurait pas épousé. Cela découle d'une décision donnée par le TGI de Paris le 13 Février 2001. \\
Autre exemple: nullité si impossibilité d'avoir des relations sexuelles normales ; nullité si erreur sur la qualité de l'état de santé (séropositivité) etc. \\
En revanche, n'a pas été considéré comme qualité essentielle la virginité de l'épouse alors même que cela avait revêtu un caractère important pour l'époux et que l'épouse avait menti à cet égard avant le mariage (CA Douai, 2008).


Lorsqu'un juge a à se prononcer sur la nullité du mariage pour erreur au sens de l'alinéa 2 de l'article 180, l'erreur sur l'identité civile: l'arrêt Berthon l'explique très bien ; sur les qualités essentielles, elle peut jouer un rôle mais il faut ajouter des conditions: il faut que l'erreur soit déterminante du consentement. Cela signifie que celui qui souhaite annulé le mariage n'aurai pas consenti au mariage en ayant connaissance de cette qualité. Deuxième condition: l'erreur doit être excusable par le juge: ce n'est pas parce que je considère que telle qualité est essentielle que je n'ai pas moi même à faire des vérifications de base. Il faut que celui qui consent au mariage procède à un minimum de vérifications. \\
L'annulation fait que le mariage n'est jamais censé avoir lieu. \\
Dans certains cas, à côté de l'erreur, il y aura aussi le divorce, ce que parfois on obtient pas sur le terrain de l'annulation, on l'obtiendra sur le terrain du divorce. 


§2. Les conditions de forme


I. Les aspects de publications liés au mariage (publication des bans)


A. Formalité avant la publication


La publication du mariage est subordonné à l'accomplissement de certaines formalités: il s'agit de constituer le dossier de mariage. \\
C'est l'article 63 alinéa 2 qui évoque ces formalités. Il faut les copies intégrales des actes de naissance, des documents d'identité de chacun des époux et l'indication des noms, prénoms, domicile et profession des témoins. Dans certains cas, l'officier de l'état civil peut demander à auditionner les deux futurs époux. Si il n'y a aucun doute particulier, on dispensera les futurs époux de cette audition. 


B. La publication


La publication se fait à la mairie du lieu de mariage (article 63 alinéa premier), cette publication se fait pendant 10 jours et le mariage ne peut pas être célébré avant la fin de la publication (premier jour non compris). \\
Cette publication énoncera les noms, prénoms, professions, domicile des futurs époux ainsi que le lieu où le mariage devra être célébré. 


C. La dispense de publication


Il faut envisager l'exception de l'article 169 du C.Civ où le procureur de la République peut dispenser de la publication des bans ou alors il peut dispenser du délai de 10 jours pour causes graves. \\
En première cause grave, on peut imaginer une personnalité publique se mariant. 


II. La cérémonie de mariage


A. Les formes de la cérémonie


1. Les formes


D'après l'article 165, le mariage sera célébré publiquement. Les portes de la mairie doivent donc être ouvertes. \\
La cérémonie est républicaine et célébré par l'officier de l'état civil de la commune. La notion de cérémonie républicaine a été rajouté lors de la loi pour le mariage pour tous pour rappelé que le mariage est laïque et fait au nom de l'État. 


2. Les dates et lieu


Pour la date, le mariage se tient au minimum 10 jours après publication des bans (premier jour non compris). \\
Pour le lieu, il faut voir l'article 74 du C.Civ, le mariage est célébré par principe dans la commune où l'un des époux habite (depuis au moins un mois) ou dans la commune où leurs parents habitent. \\
L'exception est donnée à l'article 75 alinéa 2 in limine. En cas d'empêchement grave, le procureur peut requérir l'officier de l'état civil du lieu du mariage pour aller célébrer la cérémonie dans la résidence de l'un des époux ou des parents. L'art. 75 considère aussi l'exception en cas de mariage in extremis. 


3. La cérémonie


Voir article 75. 




\subsubsection{Les sanctions des conditions de validité}


§1. L'opposition à mariage


I. Les conditions de l'opposition


A. Quels sont les personnes qui peuvent faire opposition à mariage ?


Il faut se reporter aux article 172 à 175-1 du C.Civ. On voit que le législateur a bâti une liste de personnes pouvant faire opposition. Il y a quatre séries de personnes pouvant faire opposition. \\ 
L'article 172 est dans l'hypothèse de la bigamie. Le conjoint qui découvrirait que son conjoint veut se marier avec un autre peut faire opposition. \\
L'article 173, alinéa premier dispose que le père ou la mère peut faire opposition à un mariage. À défaut de père et de mère, ce sont les aïeuls qui peuvent former une opposition au mariage de leurs descendants. L'opposition est possible même si les contractants au mariage sont majeurs. \\
L'article 174 dispose que si et seulement si il n'y a pas d'ascendant, on va voir les collatéraux: frères, soeurs, cousins, etc. Les collatéraux ne peuvent former opposition que dans deux cas: si le consentement du conseil de famille est requis (art. 159) et qu'il n'a pas été obtenu ; si le futur époux a un problème psychologique lui empêchant de donner un consentement libre. \\
L'article 175-1 dispose que le ministère public peut former opposition mais seulement dans les cas où il pourrait demander la nullité du mariage. \\
On notera que l'opposition vise à empêcher la célébration du mariage, la nullité vise à annuler un mariage déjà célébré. 


B. Quelles formes prends l'opposition ?


Pour faire opposition, il faut faire "acte d'opposition" et il faut donc nécessairement faire appel à un huissier de justice. Cet acte est un acte authentique. Cet acte doit être signifié (signification d'huissier) par l'huissier, ce qui veut dire que cet acte va être porté à la connaissance des futurs époux, mais il va être aussi signifié à l'officier de l'état civil où le mariage doit être célébré. \\
L'acte d'opposition doit contenir, d'après l'article 176, alinéa premier, les motifs de l'opposition et le texte de loi sur lequel le motif repose. Si nous sommes dans l'hypothèse où un des futurs époux doit obtenir une autorisation des parents plus une autorisation du procureur, les parents peuvent faire opposition ou le procureur si une des deux autorisations manquent.


II. Les effets de l'opposition


L'objectif premier d'une opposition est d'empêcher la célébration d'un mariage. 


A. Temps 1: le mariage ne peut être célébré


En cas d'opposition, le mariage ne peut pas être célébré, voir article 68 du C.Civ, avant qu'on lui ait remis la mainlevée. Si l'officier d'état civil passe outre et marie quand même les époux en question, sur le plan civil, l'officier peut être condamné à payer une amende ainsi que des dommages et intérêts. 


B. Temps 2: la demande de mainlevée de l'opposition


Soit les futurs époux renoncent, devant l'opposition, à leur projet de mariage, l'objectif de l'opposition a été atteint. \\
Soit les époux contestent le bien fondé de l'opposition. Si ils le conteste, ils doivent donc intenter une action en mainlevée. Cette demande en mainlevée doit être formée devant un TGI. L'article 177 dispose que le TGI a 10 jours pour statuer. Si il y a appel, là aussi la Cour d'Appel doit statuer dans les 10 jours. \\
Il y a deux possibilités: soit la mainlevée est accordée, soit elle est refusée. Si elle est accordée, le mariage peut avoir lieu, on dit qu'il y a eu jugement ou arrêt de mainlevée. \\
Si il y a jugement de mainlevée, alors l'opposition était infondé, si il y a préjudice lié à cette opposition, les futurs époux peuvent engager une action en responsabilité civile contre ceux qui ont fait opposition. On distingue deux situations: une première dans laquelle même si il y a préjudice, les époux ne peuvent pas demander de dommages et intérêts et une autre où ils peuvent. Tout dépend de la catégorie qui a fait opposition. \\
Si l'opposition provient des parents dans le sens strict (des ascendants) ou si l'opposition provient du ministère public, il ne peut y avoir de dommages et intérêts. Si l'opposition provient des collatéraux, ils peuvent payer dommages et intérêts. Tout cela est prévu à l'article 179, alinéa premier. \\
Si il y a eu une première opposition, et que cette première opposition des ascendants a été levée, l'article 173, alinéa 2, nous dit que le jeu est terminé "Aucune nouvelle opposition formée par un ascendant n'est recevable ni ne peut retarder la célébration". 


§2. La nullité


I. La mise en oeuvre


La nullité, à la différence de l'opposition est encore plus redoutable, car si l'opposition empêche un mariage, l'action en nullité quand elle aboutit, conduit à un effacement pur et simple du mariage. 


A. La distinction nullité relative et nullité absolu


1. La nullité absolu


La nullité absolu intervient quand un intérêt général d'ordre public a été transgressé. La nullité absolu sanctionne donc un comportement grave attentatoire aux intérêts supérieurs de la société. \\
Le comportement de nullité sanctionne ce qui est considéré comme le plus grave. À ce qui est le plus grave correspond un régime juridique particulier. \\
Quand on parle de nullité absolu, toute personne qui a intérêt peut engager une action, c'est souvent le parquet, visant à anéantir l'acte ou la situation juridique. \\
Quand on est en matière de nullité absolu, on peut intenter une action pendant 30 ans. On dit, en d'autres terme que lorsqu'il y a action en nullité absolu, l'action est prescrite au bout de 30 ans. \\
Un acte nul de nullité absolu ne peut pas être confirmé (refait). 


2. La nullité relative


Elle est considérée comme une sanction moins grave que la nullité absolu alors même que la conséquence est identique. Lorsqu'il y a nullité relative, il y aussi effacement rétroactive de la situation juridique. \\
La nullité relative va mettre en jeu la violation d'un intérêt, mais pas d'un intérêt collectif, mais la violation d'un intérêt individuel. \\
En matière de nullité relative, le cercle des personnes qui peuvent agir est beaucoup plus limité que le cercle qui peuvent agir en nullité absolu. En nullité relative, ne peut agir que les personnes dont l'intérêt particulier est protégé par la Loi ; il n'y a pas de procureur de la République qui peut agir. \\
La durée de prescription de la nullité relative est généralement de 5 ans. Mais il peut être raccourci. \\
Quand on est sur le terrain de la nullité relative, il faut toujours regarder ce que dit la Loi. \\
En matière de nullité relative, on peut confirmé le contrat nul. \\
Ne pas faire une formalité peut entraîné la nullité relative du contrat.


B. Les cas de nullité


Il y a 10 cas de nullité. Il y a 7 cas de nullité absolu et 3 cas de nullité relative. 


1. Les cas de nullité absolu


5 cas relatifs à la violation d'une règle de fond (règle substantielle). 
2 cas relatifs à la violation d'une règle de forme. 

\begin{enumerate}
\item Mariage où l'un des époux n'est pas pubère (art. 184 et 144, attention à l'interprétation) ;
\item Défaut total de consentement (quelqu'un qui ne serait pas en état d'exprimer vraiment son consentement), mariage simulé (mariage blanc et gris) ;
\item Défaut de présence (on ne peut pas se marier par procuration) (art. 146-1) ; 
\item Inceste, absolu (art. 161, 162), et relatif (art. 163) sans autorisation ;
\item Bigamie (art. 147) ; 
\item Défaut de célébration publique (on parle de mariage clandestin) (art. 191), il doit y avoir volonté de fraude ;
\item Défaut d'officier d'État civil compétent (art. 191), compétence territoriale et compétence matérielle (parler de compétence matérielle, c'est se demander si la personne qui agit, peut agir matériellement, a-t-elle la possibilité juridique d'intervenir en ce domaine ?). 
\end{enumerate}

Les époux, le ministère public ou toute personne ayant intérêt peuvent demander la nullité d'un mariage devant le TGI (TGI qui a compétence exclusive du mariage). \\
L'attaque du mariage doit se faire dans les 30 ans après la célébration de celui-ci. Quand le délai pour intenter une action est dépassé, on dit que l'action est éteinte ou que l'action est prescrite.  \\
Dans l'article 191, on rajoute les ascendants comme pouvant attaquer un mariage. 


Un arrêt de la C.Cas, Civ. 7 Aout 1883, est l'arrêt dit de l'affaires des mariages de Montrouge, où un officier d'État civil qui avait procédé au mariage n'avait pas eu délégation du maire en bonnes formes. Formellement parlant, l'officier n'avait pas valable délégation. La question était de savoir si juridiquement parlant, l'officier avait la compétence. La C.Cas a tranché que l'incompétence matérielle de l'officier d'État civil ne constitue qu'un cas de nullité facultative laissé à l'appréciation des juges. \\
Le juge n'a donc pas annulé le mariage car si la délégation n'était pas formellement bonne, il n'y avait pas d'intentions de fraudes. \\
À contrario, lorsque le défaut de compétence matérielle cache une volonté délibérée de fraude, le juge ne peut que prononcer la nullité absolu. 


2. Les cas de nullité relative


Premier cas: défaut de consentement libre. Sur ce premier cas, on se rapporte à l'article 146 mais surtout l'article 180 du C.Civ. N'est pas libre le consentement qui a été donné sous la contrainte, ou la crainte révérencielle. \\
Dans cette hypothèse, les époux peuvent agir devant le TGI ou bien le ministère public. \\
L'action peut se faire dans les cinq ans à compter de la célébration du mariage. 


Deuxième cas: l'erreur dans la personne ou sur les qualités essentielles de la personne (art. 180). \\
Dans ce cas, seul l'époux victime de l'erreur peut agir devant le TGI dans les 5 ans à compter de la célébration du mariage. 


Troisième cas: le mineur qui se marie sans obtenir les autorisations nécessaires (parents, ascendant, conseil de famille) (art. 182). \\
Dans ce cas, ceux dont le consentement était nécessaire peuvent attaquer le mariage ainsi que l'époux qui avait besoin de ce consentement. \\
Le temps de prescription est là aussi 5 ans. Il y a la mise en oeuvre de l'article 183 qui vise à restreindre d'une certaine manière le délai où on a les possibilités d'agir en nullité. \\
Si les parents acceptent tacitement le mariage, ils ne peuvent plus agir en nullité. Concernant l'époux, il doit s'écouler 5 ans à partir du moment où il n'a plus besoin du consentement.



II. Les conséquences




\subsection{Les effets du mariage}


















\end{document}
