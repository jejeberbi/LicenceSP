\documentclass[12pt, a4paper, openany]{book}

\usepackage[latin1]{inputenc}
\usepackage[T1]{fontenc}
\usepackage[francais]{babel}
\date{}
\title{Cours de Science Politique (UFR Amiens)}
\pagestyle{plain}

\begin{document}

\chapter{Les citoyens et la politique}

Pour comprendre la relation qu'il y a entre les citoyens et la politique, on a l'habitude de recourir au concept (créé par Daniel Gaxie) de politisation. La politisation, dans le sens commun, est le fait de rendre quelque chose politique. \\
Le concept, lui, a un autre sens et possède deux dimensions. C'est le degré de connaissance de la politique d'abord, le degré d'intérêt pour la chose politique ensuite. Cette politisation, Gaxie en parle dans "Le cens caché". Cette politisation serait donc un cens caché, même dans notre démocratie, la politisation bloque l'accès à l'opinion et à son expression comme le ferait un impôt censitaire. \\
Le modèle théorique de la démocratie repose sur la participation active de tous les citoyens. Or, la sociologie montre qu'une part limité de la population seulement a une connaissance spécifique de la politique et une part limité seulement s'intéresse à la politique. \\
Nous allons nous intéressé surtout aux notions fondamentales qui permettent de comprendre la nature des rapports que les citoyens entretiennent avec la politique. Nous allons aussi examiner les facteurs qui creusent l'inégalité des citoyens en terme de politisation. On parlera aussi, à la suite de Pierre Bourdieu, de compétence sociale et/ou politique ainsi que du sentiment de cette compétence. 


L'avènement du suffrage universel masculin en 1848 vient un postulat qui peut se résumer ainsi: le citoyen est intéressé par la politique et il est compétent pour y prendre part à travers ses choix électoraux. Le suffrage repose donc sur un idéal normatif: celui d'un citoyen actif et éclairé sur les enjeux politiques. Cette activité et cet intérêt est censé se manifesté par le vote qui devient une obligation morale et démocratique. \\
La démocratie est fondée sur le double présupposé de la politisation de l'ensemble des citoyens et de la participation de l'ensemble des citoyens. \\
Or, dans les régimes démocratiques, tout le monde ne participe pas forcément à la chose publique, comme le prouve l'abstention. La progression très forte de l'abstention dans les démocraties modernes, la non inscription sur les listes électorales, la désertion des urnes en sommes, traduisent des formes d'exclusion sociales et des formes d'exclusions politiques qui prennent racine dans un processus de socialisation politique.


\section{La socialisation politique}

La socialisation politique que tous les individus subissent ou vivent dans leur histoire, participent à la formation, à la structuration des comportements politiques. C'est au cours de ce processus de socialisation que l'on se familiarise avec la politique, que l'on se forme des idées politiques et que l'on se forge des identités politiques. 

\subsection{Les mécanismes de la socialisation politique}

\subsubsection{La socialisation en général}

"La socialisation est le processus autoritaire par lequel l'individu intègre les normes de son groupe." \\
La socialisation est un processus qui se déroule tout au long de la vie et pendant lequel un individu apprend et intériorise les normes et les valeurs de la société à laquelle elle appartient et par lequel il construit son identité sociale. \\
Annick Percheron parle d'instance de socialisation, mais aussi d'agences de socialisation. Les comportements sont définis socialement (déterminisme social). \\
La famille, l'école, le travail, l'appartenance à une génération, les médias etc. contribuent à la socialisation. La socialisation est autoritaire car se fait de manière inconsciente. \\
La socialisation politique correspond à un processus d'apprentissage par les individus des règles qui organisent le champs politique mais aussi de valeurs et de préférences politiques. Les auteurs reconnaissent deux types de socialisation: la socialisation primaire (enfance et adolescence) et socialisation secondaire (adulte). \\
C'est la socialisation secondaire qui va permettre à l'adulte de s'intégré à d'autres groupes autre que la famille, et qui pourra peut être s'inscrire en rupture par rapport à la socialisation primaire. Le plus souvent, la socialisation secondaire s'inscrit dans le prolongement de la socialisation primaire.


De plus, la socialisation est un processus interactif, ce n'est pas une simple transmission de normes et de valeurs mais c'est aussi une acquisition, on est actif au cours du processus de socialisation. \\
La politiste Française Anne Muxel montre dans ses travaux que la socialisation politique peut suivre deux logiques inverses, concurrentes. Première logique d'identification par laquelle les individus intègrent les normes et les valeurs politiques de leurs parents et des générations passées. C'est aussi la logique de l'héritage. La deuxième logique est d'expérimentation, logique par laquelle les individus font oeuvre d'une certaine autonomie vis à vis de leurs parents, de leur génération passée et oeuvre également d'innovation. C'est ainsi qu'elle montre comment les variations économiques, politiques et historiques rendent la socialisation de chaque génération unique par rapport à celle des générations précédentes ou suivantes. \\
C'est ainsi que Anne Muxel invente le concept d'effet de générations. 


\subsubsection{La pluralité des instances de socialisation politique}

C'est Annick Percheron qui a montré la pluralité des instances de socialisations et l'influence de celles-ci sur notre préférence politique. Ces instances sont la famille, l'école, les groupes de pairs ou encore les médias. \\
Dès l'âge de 8 ans, les enfants sont familiarisés avec un langage élémentaire politique. Par des associations d'idées, ils peuvent exprimer leur sympathie ou non sur des idées, des hommes politiques etc. La famille joue un rôle important dans la socialisation politique primaire. \\
Cette socialisation politique s'opère dès le plus jeune âge. La proximité affective favorise la transmission et la reproduction des idéologies. \\
Dans les années 50, des travaux américains sur les élections présidentielles (menés par l'université du Michigan) ont établis le rôle central de la socialisation familiale et ont aussi établis la précocité des préférences partisanes des enfants. On parle d'identification partisane que l'on pourrai définir comme la loyauté durable à un parti politique (Républicains ou Démocrates aux USA). \\
Les travaux de l'université de Michigan, qui utilisent cette maxime: "Dis moi pour qui ton père votait, je te dirai pour qui tu votes". Se fondant sur des sondages d'opinions, ces chercheurs se sont rendus compte que le facteur le plus important du point de vue statistique pour expliquer une orientation lors d'une élection présidentielle n'est autre que le simple fait de se déclarer démocrate ou républicain. \\
La majorité des électeurs ne connaissent rien à la politique entendu comme la gestion des affaires de l'État: ils semblent donc simplement voter en faisant confiance à une tradition de vote soit républicaine soit démocrate. Cette tradition est très largement liée à une transmission familiale. \\
Cette vision fait des électeurs de simples supporters et contredit même l'idéal démocratique.


Dans les années 70, Annick Percheron reprend ce modèle, le prolonge et l'affine. \\
Percheron montre que les jeunes manifestent entre 10 et 14 ans une proximité idéologique avec la droite ou la gauche proche de celles de leurs parents. Ceci est d'autant plus vrai que les préférences politiques des parents sont fortes et visibles. \\
La continuité entre les choix politiques des parents et des enfants apparaît de manière très nette lorsque l'on se place sur l'axe bipolaire droite/gauche. \\
Percheron parle de filiation politique alors que l'université de Michigan parlait d'identification partisane.


Les résultats de son enquête publiés en 1989, indiquent que certains facteurs renforcent cette filiation politique. Il y a trois facteurs. \\
Le premier facteur est qu'un fort taux d'intérêt pour la politique de la part des parents se traduit par 74\% de reproduction parfaite des enfants. Il y a 31\% de reproduction parfaite quand les parents ne manifestent pas d'intérêt pour la politique. \\
Le deuxième facteur est l'homogénéité des choix politiques des parents, c'est à dire que le fait d'avoir deux parents de même conviction politique augmente la reproduction parfaite. Les chiffres sont beaucoup plus faible en cas d'hétérogénéité des idées politiques des parents. \\
Le troisième et dernier facteur est le niveau d'instruction des parents: plus ils ont un niveau élevé, plus le taux de reproduction sera élevé. Toutes les familles n'ont pas les mêmes capacités à organiser la transmission. Dans la famille se transmet une forme de compétence politique par apprentissage des parents envers les enfants et aussi un sentiment de compétence politique. \\
La période décisive serait entre 10-11 ans et 16 ans. Cependant, elle ne la considère pas comme une donnée stabilisée pour le reste de la vie. On peut observer une différence entre Percheron et les chercheurs de l'université de Michigan, car, pour elle, lors du passage à l'âge adulte, la socialisation politique est loin d'être terminée, elle reste un phénomène en voie de formation. \\
La socialisation politique familiale s'effectue selon trois modalités: par imprégnation ou familiarisation ; inculcation ; modes indirects de transmission (délégation à d'autres instances le travail de socialisation: choix d'inscrire en école privé ou publique et toute action affectant l'enfant). \\
Première étape: processus d'accommodation: processus par lequel l'individu modifie ses représentations et ses comportements pour s'adapter à son environnement. Deuxième étape: assimilation, l'individu va se voir assigné une identité politique dans laquelle il va se reconnaître. L'utilisation du terme "je suis" au lieu de "je vote tel chose" ou "je me reconnais dans tel parti" montre cette identification. \\
Muxel confirmera l'impact de la famille sur la socialisation politique en 2010. On notera d'ailleurs que les mères transmettent davantage leur préférence politique que les père. Le champ politique étant fortement masculinisé, cela fait donc un paradoxe. Aujourd'hui, on s'aperçoit que quand on s'élève dans les classes sociales, on constate une homogénéisation des comportements électoraux et politique entre femme et homme. En revanche, plus on descend, plus on trouve des comportements différenciés. Dans les milieux sociaux populaires, les hommes ont du mal à reconnaître leur incompétence politique: Bourdieu dit: "les hommes sont dominés par leur domination", c'est le masquage d'une réalité sociale. \\
La famille transmet un cadre idéologique général et des valeurs qui vont structurer les comportements politiques des enfants. La famille fournit donc un outillage idéologique.

\subsubsection{L'école}

L'école joue aussi ce rôle. C'est un espace de socialisation politique et latent. \\
Certains enseignements sont ouvertement dirigés vers l'objectif de la socialisation politique: l'histoire, l'éducation civique, la philosophie etc. L'école essaye donc de former les élèves à la politique, aux institutions, à l'histoire politique du pays etc. \\
De manière plus latente, l'école contribue à la socialisation politique car c'est un espace de vie en commun et où coexiste des pouvoirs institutionnels ainsi que des pouvoirs informels (dans les groupes d'élèves etc.). C'est aussi un espace d'expérimentation d'exercice ou de domination à un pouvoir. C'est aussi un lieu qui comporte des règles, des devoirs ainsi que des droits. L'école possède des règles du vivre ensemble qu'elle essaye de faire appliqué. \\
L'école est aussi le lieu de l'apprentissage de la démocratie via l'élection des délégués de classe notamment. Il y a liberté de candidature mais aussi pluralité. De plus, l'enseignant explique que le délégué ne sera pas là pour défendre sa personne mais qu'il sera le représentant et le porte parole du groupe. On apprend donc la démocratie à l'école ainsi que la pratique du vote. \\
À la différence de la famille, l'école transmet des connaissances politiques formelles. L'impact de l'école en terme de choix politique est très faible. \\
Les relations entre élèves peuvent avoir une certaine importance dans la socialisation politique. Muxel montre par exemple que le lieu de l'établissement n'est pas tout à fait neutre. On ne parle pas de la politique de la même manière en milieu rural ou en milieu urbain etc. \\
Muxel montre que l'héritage politique se transmet d'autant plus facilement que l'enfant vit dans un environnement social conforme aux préférences politique des parents. Il peut donc y avoir des contradictions.

\subsubsection{L'appartenance à une génération}

La vie politique dans les années 70, du fait des événements politiques de Mai 68, est beaucoup plus polarisé sur un clivage gauche/droite que peut ne l'être la vie politique actuellement. \\
Les effets de générations peuvent donc entraînés des ruptures entre la génération précédente et la suivante. 

\subsubsection{Les médias}

Les médias jouent surtout un rôle lors de la socialisation politique secondaire. \\
On peut d'abord considéré que les médias fixent l'agenda de l'actualité politique. En pesant sur le calendrier, sur le rythme, sur les enjeux, sur la mise en scène du débat politique. Ils contribuent à sélectionner les enjeux légitimes, qui méritent de faire l'objet d'un débat, ils disent à l'électeur potentiel ce qui est digne d'être posé en problème politique. C'est ce qu'on appelle l'effet d'agenda. L'effet de socialisation des médias est donc très présent mais relativement indirect. \\
On dit que c'est indirect car les médias imposent des préoccupations sans que le récepteur ait son mot à dire quant à l'importance que lui accorde à ces sujets. Il y a une imposition de problématique (les sondages jouent aussi ce rôle là). \\
Les médias influencent aussi la culture politique en construisant le sens des enjeux qu'ils mettent en avant. Par plusieurs moyens: le traitement répété de certains problèmes sociaux, leur traduction visuelle, la répétition de certains points de vue, contribuent à imposer la signification politique à certaines questions et à orienter les critères du jugement politique. \\
La question de l'influence des médias, notamment sur le choix électoral, a été posé en 1940 par une équipe de chercheurs (université de Columbia) aux États-Unis, réunis autour de Paul Lazarfeld. Son enquête vise à comprendre comment les électeurs forment leurs choix électoraux. Il se pose donc bien sûr la question du poids des médias dans son ouvrage "The people's choice". \\
Il montre que l'influence directe des médias sur les choix électoraux sur les gens étaient assez limités, et, en fait, les médias viendraient plutôt renforcer les convictions politiques déjà existantes et les modifierait que rarement. \\
Ses recherches ressortent au contraire que ce sont avant tout l'origine et le milieu social d'appartenance qui détermine le vote et non pas ou très peu les médias voire les campagnes électoral, d'où l'expression: "On pense politiquement comme on est socialement". \\
En 1940, il remarque que seul 8\% des électeurs changent leur intention de vote entre le début de la campagne et la fin de la campagne car les individus ont tendance à ne retenir que ce qui est conforme à leurs opinions. \\
Cependant, à long terme, à force d'exposition, de répétition, les médias contribuent à forger des représentations politiques, des problèmes etc. \\


\subsubsection{Le milieu professionnel}

Le milieu professionnel est évidemment un lieu de socialisation politique car il est un lieu de socialisation avant tout. Mais surtout, c'est aussi un lieu relativement homogène au niveau politique: rare sont les patrons à militer au NPA, les électeurs de gauche sont essentiellement fonctionnaire etc. \\
En usines, l'espace ouvrier est (c'est moins vrai aujourd'hui) marqué politiquement par les luttes sociales, le syndicalisme etc. 


\subsubsection{Conclusion}

Le processus de socialisation est multidimensionnelle et continu tout au long de la vie. Ce processus dépend des groupes d'appartenance qui définit l'individu. 

\section{La compétence et l'intérêt pour la politique}

La démocratie est fondée sur l'idéal d'un citoyen actif éclairé: l'idée selon laquelle le citoyen est informé lorsqu'il va voter des enjeux. Le citoyen renvoie donc à l'idée d'un citoyen compétent. Le vote doit manifester l'intérêt pour la politique mais aussi la préférence, et mise donc sur sa connaissance politique. 


On distingue deux principes de participation politique: la participation conventionnelle (le vote) et la participation politique non conventionnelle (pratiques qui tendent à remettre en cause l'ordre établi). Cette distinction postule une forme de normalité sociale du vote par rapport à d'autres formes de participations politiques. \\
Les échelles de participation vont de l'inscription sur les listes électorales à la lecture régulière de la presse, la participation à des meetings voire l'adhésion à un parti. \\
Le rapport à la politique est rarement celui de l'électeur intéressé par la politique. 


\subsection{Un rapport distendu à la politique...}

Philip Converse dès les années 1960, réalise une étude auprès d'électeurs américains, il observe que l'électorat américain se caractérise par un faible niveau de conceptualisation idéologique. Il démontre que seulement 12\% des électeurs interrogés mobilisent des critères d'ordre idéologiques. \\
Les électeurs américains se caractérisent aussi par une faible reconnaissance des principes idéologiques. Ils se caractérisent aussi par un faible niveau de cohérence des attitudes politiques. \\
Il y a aussi une absence de stabilité dans le temps des attitudes politiques sur les enjeux. \\
Tout cela concoure à mettre en évidence l'incompétence politique des citoyens. 


Les études menées en France montrent que la part des personnes intéressées par la politique et proche d'un parti politique est limitée. En 2006, on apprend que 48\% des Français interrogés déclarent s'intéressé à la politique. \\
L'identification à des camps politiques ne cessent de diminuer depuis les années 1970. Toujours en 2006, 30\% des Français refusent de se positionner sur un axe gauche/droite, 60\% considèrent même que la distinction gauche/droite est dépassée. \\
L'indifférence citoyenne est un enjeu majeur pour les démocraties modernes. Cette indifférence se manifeste par une abstention montante.


\subsection{...révélateur d'un "cens caché"}

Gaxie montre que si les barrières économiques à l'entrée des bureaux de vote ont été levées, celle-ci subsiste sous une autre forme: une forme socio-culturelle. Un cens en aurait caché un autre. Il existe des exclusions du jeu politique qui sont plus subtil. \\
Pour Gaxie, la politisation est inégalement distribué dans l'espace social. Il existe des inégalités sociales en matière de politisation. \\
Dans son ouvrage, Gaxie exclu l'idée d'une compétence universelle pour la politique. Il démontre aussi que la politisation est soumise à des facteurs socio-culturel qui altèrent de manière important le comportement électoral, l'intérêt pour la politique et la capacité qu'ont les citoyens à exprimer, à produire une opinion politique. Cette capacité est inégalement distribuée. \\
C'est le niveau d'étude atteint par les citoyens qui montre des écarts de cette capacité ainsi qu'avec la position socio-professionnelle.


Gaxie distingue la compétence politique et le sentiment de posséder celle-ci dans le prolongement de Bourdieu. \\
La première dimension est la dimension subjective, la compétence que l'on s'auto-attribue. \\
La deuxième dimension est la dimension objective: la détention d'un ensemble de savoir faire, de connaissances. \\
Les individus issus des milieux populaires vont avoir plus tendance que les autres à ne pas se sentir habilité à intervenir en matière politique. À l'inverse, les membres des CSP+ s'estime plus compétent même si ils ne le sont pas réellement. 


La professionnalisation de la politique a entraîné la domination du champ politique. Les professionnels ont imposés des règles spécifiques, un langage particulier. Le citoyen ne peut appréhender le jeu du champ politique que si il possède au préalable une compétence spécifique. \\
Il faut prendre en compte le fait que la compétence politique n'est pas le seul outil dont dispose les citoyens pour se repérer dans le jeu politique et pour l'évaluer. Ils ont possibilité de mobiliser d'autres jugements fondés sur d'autres critères que le connaissance technique savante et spécialisée des enjeux politiques. \\
Ces critères peuvent être une considération affective, symbolique. Des chercheurs américains appellent cela des "shorts-cuts", ce qui donnerai en français "Raccourci cognitif" qui mobilisent peu d'informations mais qui permettent malgré tout de produire ou d'émettre des jugements, des opinions. \\
Ces raccourcis sont des signes auquel on se réfère pour indiquer d'autres informations que l'on juge alors inutile de vérifier. \\
Dans un article de 1993, "Le vote désinvesti", Gaxie a réalisé une enquête avec ses étudiants lors d'une élection municipal de 1989. Il montre que finalement, les votes peuvent être dénués de toute dimension spécifiquement politique. \\
Parmi les électeurs les moins politisés, certains n'ont retenu de la candidature de Olivier Besancenot que sa profession de jeune facteur. \\
Ce type d'évaluation peut être considéré comme une autre forme de compétence et de rationalité politique: les citoyens les moins politisés tentent de se saisir de l'espace politique, de se l'approprier. Le sociologue du politique ne peut donc pas les ignorer. Cependant, force est de constat que la compétence politique spécifique, technique, savante, reste largement déterminante dans les probabilités de s'intéressé à la politique. \\
La compétence politique comme toute autre forme de compétence spécialisée se mesure dans la capacité à reproduire le langage spécifique des professionnel à faire fonctionner les catégories et à interpréter les logiques que mettent en oeuvre les principaux acteurs du jeu politique. La compétence politique s'inscrit donc dans une logique cumulative: ceux qui ont la plus grande compétence politique ont le plus de chances d'accumuler toujours plus de compétence. 




\section{L'abstention}



\chapter{Les modèles explicatifs du vote}







\chapter{Les partis politiques}






\chapter{L'action collective}







\chapter{L'action publique}






\end{document}
