\documentclass[12pt, a4paper, openany]{book}

\usepackage[latin1]{inputenc}
\usepackage[T1]{fontenc}
\usepackage[francais]{babel}
\date{}
\title{Cours de Droit Constitutionnel (UFR Amiens)}
\pagestyle{plain}

\begin{document}

\chapter{Introduction: Les origines de la Vème république}

La Constitution du 4 Octobre 1958 est la Constitution du 5ème régime républicain qu'a connu la France depuis 1789. \\
La première république court du 21 Septembre 1792 au 18 Mai 1804. La deuxième république court du 4 Novembre 1848 au 2 Décembre 1851. La troisième République court du 4 Septembre 1870 jusqu'au 10 Juillet 1940. La 4ème république court du 23 Avril 1944 (constitution d'octobre 1946) au 4 Octobre 1958. \\
La Constitution de la 5ème république est en vigueur depuis presque 60 ans, ce qui en fait, depuis 1791, le deuxième régime le plus long après la 3ème République. Ce qui explique cette durée, c'est la souplesse, ce n'est pas un isolement des pouvoirs, mais de collaboration mais aussi l'instauration d'une véritable autorité d'État, qui transite par l'autorité du pouvoir exécutif: l'autorité de l'État passe entièrement par l'autorité du Chef de l'État. La constitution reconnaît clairement la suprématie du Président sur les autres pouvoirs d'où le fait que certains parlent de monarchie républicaine. \\
La 5ème a aussi su s'adapter et évoluer à son environnement, environnement social ou international. La procédure de révision est une procédure efficace d'adaptation de la constitution. Elle a été révisée 24 fois. La Constitution tel que nous la connaissons n'a donc plus grand chose à voir avec le texte de 1958. En 1958 apparaissait par exemple un titre sur les colonies, qui n'a plus lieu aujourd'hui. Une partie de la Constitution est aussi dédié à l'Union Européenne, partie qui n'existait pas en 1958. \\
La dernière révision est du 23 Juillet 2008. Sur un total de 106 articles, 39 articles sont modifiés et 9 articles sont ajoutés. 

\section{Le contexte inhérent à la mise en place de la Constitution du 4 Octobre 1958}

Le constat de la mise en place de la 5ème République est un constat rapide: il y a un concours de circonstance ; en effet, la 5ème République naît de l'impotence constitutionnelle antérieure, notamment de l'incapacité de la 4ème République à résoudre les immenses défis qui se présentaient à lui. \\
En effet, la 4ème République connaît une incapacité institutionnelle d'une part et une incapacité politique d'autre part.


Sur le plan politique, la 4ème n'a pas su résoudre les défis qui se présentaient à elle: la reconstruction après la seconde guerre mondiale (en 1958, la reconstruction n'est pas terminée: 5 millions de personnes vivent dans des bidonvilles ou sont sans abris) ; le défi de la décolonisation (conflit d'Indochine, colonies d'Afrique du Nord, Maroc puis Algérie). En 1958, la France se retrouve en pleine guerre en Algérie: c'est ce conflit qui va précipiter la chute de la IVème. \\
Sur le plan économique, la France est en retard, c'est principalement une économie de subsistance constitué de micro-entreprise: confine à l'artisanat plutôt qu'à l'industrie là où l'Allemagne ou encore les USA ont déjà fait naître les premières multinationales. \\
Quatrième défi: la France a eu de nombreux problèmes à se positionner dans le cadre des prémices de la Guerre Froide mais aussi de l'Europe qui débute sa construction. En 1958, la France ne dispose donc d'aucun leadership (diplomatique ou politique).


Sur le plan institutionnel, la constitution de 1946 a reproduit les défauts de la troisième République: entre 1946 et 1958, la durée moyenne des gouvernements est de moins de 6 mois. Conséquence: faiblesse de l'exécutif et donc fondamentalement, faiblesse de l'État. \\
L'État en 1958 est donc une institution à la dérive confronté au conflit Algérien qui fait rage depuis 1954. Ce conflit oppose les partisans d'une Algérie Française et ceux qui veulent la construction d'un État indépendant. Sur place, et notamment à Alger, le conflit radicalise les positions d'une partie importante de l'armée Française, traumatisée de ses échecs militaires antérieurs (notamment en Indochine). Il n'est donc pas question de subir une défaite supplémentaire. \\
Ces militaires sont soutenus par une grande majorité des Français installés en Algérie et qui ont des intérêts sur place, notamment des intérêts économiques. Ils sont installés depuis plus de 120 ans. Le contexte est donc particulièrement tendu. Il se tend encore lorsque Pierre Phlimlin est pressenti comme futur président du conseil le 13 Mai 1958. Celui-ci est un Homme politique très ouvert aux négociations en vue de la préparation d'une indépendance avec le FLN. Cette perspective d'investiture engendre une insurrection en Algérie et la prise du Gouvernement Général à Alger par les militaires. \\
L'armée procède à un putsch local en Algérie en y mettant évidemment en place un régime autoritaire, autocentré (une junte). Ceci prend officiellement le nom de comité de salut public. Le Gouvernement Français ne réagit pas, car il ne peut réagir. L'assemblé ne peut pas non plus faire grand chose: les institutions sont paralysées. \\
Sur place, le Général Salan prend la tête de la Junte. Il est un fidèle du Général de Gaulle et lance à ce dernier un appel pour former un Gouvernement. Celui-ci ne répond pas officiellement et réprouve même ce coup de force, mais il affirme être prêt à assumer ses responsabilités politiques comme il l'a déjà fait dans d'autres contextes de crise. \\
Le 24 Mai 1958, plusieurs bataillons de parachutistes arrivent en Corse et y instaure un comité de salut public. L'armée prépare donc une opération de parachutage à Paris prévu dans la nuit du 27 au 28 Mai 1958 appelée "Résurrection". L'armée prépare donc un coup d'État. \\
Le Gouvernement Phlimlin démissionne donc. René Coty appelle De Gaulle à formé un Gouvernement. Son Gouvernement est investi le 1er Juin 1958 par 329 voix contre 224. Son rôle sera de préparé l'établissement d'un nouveau régime, avec un objectif ultime: restaurer l'autorité de l'État. En ce sens, De Gaulle propose au Parlement une loi constitutionnelle permettant d'établir une nouvelle République: elle est votée le 3 Juin 1958. \\ 
Cette loi donne au Gouvernement un mandat: "Élaborer puis proposer au pays par la voie du référendum les changements indispensables aux institutions".

\section{Les bases constitutionnelles de la Vème République}

Ces bases se trouvent dans deux sources intrinsèquement liées: la loi du 3 Juin 1958 qui porte les empreintes des idées constitutionnelles de De Gaulles.

\subsection{Les bases intellectuelles}

Les bases reposent sur la pensée politique du Général de Gaulle, mais aussi sur les idées de ses principaux collaborateurs: Michel Debré (juriste), René Capitant (juriste et professeur de droit constitutionnel): deux juristes fondamentalement opposés au régime parlementaire à la Française. Ils reprendront des mécanismes qui permettent la longévité et donc rejetteront les mécanismes qui ont conduit aux incapacités des régimes antérieurs. \\
En 1946, De Gaulle avait déjà donné ses idées constitutionnelles dans un discours donné à Bayeux. Pour lui, le vote définitif de la loi et le vote définitif du budget appartient à une assemblée parlementaire élu au suffrage universel direct. Il est donc pour une démocratie représentative. \\
Il est aussi favorable à un bicaméralisme, qui doit être inégalitaire et technique: inégalitaire dans le sens où la primauté doit être accordée à l'assemblée la plus légitime, cette assemblée est seule capable de censurée le Gouvernement. La seconde chambre serait donc cantonnée la fonction purement législative. Technique car la seconde chambre doit être recruté autrement que le Sénat républicain traditionnel: elle devrait être recruté sur une base sociale et professionnel donc représentant de la société civile et non délégué de partis politique. \\
De Gaulle souhaite établir la responsabilité politique du Gouvernement et de ses membres devant l'assemblée élue au suffrage universel. Il souhaite aussi que le parlementarisme soit rationalisé pour avoir une stabilité politique. \\
Il souhaite aussi une véritable autorité de l'État donc par la reconnaissance de l'autorité du Chef de l'État donc d'une véritable présidence de la République duquel émanerait le Gouvernement et son chef. \\
En d'autres termes, de Gaulle souhaite un régime parlementaire dualiste où le Gouvernement est responsable devant la chambre et devant le chef de l'État. Le Président dans cette perspective n'est pas un président inactif ou effacé: il est un arbitre engagé, un président qui gouverne et fait gouverner, s'assure du fonctionnement régulier des autres institutions. Pour cela il doit être titulaire du pouvoir réglementaire, il doit diriger le conseil des ministres, il doit pouvoir dissoudre l'assemblée élue au pouvoir universel. Le Président doit aussi pouvoir concentré entre ses mains l'ensemble des pouvoirs en cas de péril de la nation (dictature constitutionnelle): c'est ce qui manquait au président Lebrun en 1940 ; sur cela, De Gaulle s'inspire de la Constitution de 1804 qui permettait à l'empereur de concentré entre ses mains ces pouvoirs exceptionnelles. L'inspiration est impériale. Les Gaullistes n'hésitent pas à chercher les mécanismes efficaces dans toutes les constitutions. \\
Il est évident qu'un Président avec une palette de pouvoirs aussi large ne peut émané du Parlement: sa légitimité doit venir d'un électorat. 


Ce discours va fortement influencé la loi constitutionnelle du 3 Juin 1958.

\subsection{Les principes directeurs}

Juridiquement, la loi Constitutionnelle du 3 Juin 1958 est un peu particulière: son objet est fondamentalement de réviser la constitution. Il s'agit d'une loi de révision au titre de l'article 90 de la constitution de 1946. Or, cette loi permet de fonder juridiquement la constitution de la Vème République. \\
Cette loi procède en trois temps:
\begin{itemize}
\item La loi organise une procédure de révision spécifique par dérogation à l'article 90 afin de rendre la procédure plus simple. Le premier paragraphe de l'article unique de cette loi dispose "Par dérogation aux dispositions de son article 90, la Constitution sera révisée par le Gouvernement investi le 1er Juin 1958 et ce dans les formes suivantes...".
\item Le paragraphe deux de cette loi dispose "Le Gouvernement de la République établit un projet de loi Constitutionnelle mettant en oeuvre les principes ci-après...". On utilise donc la procédure de révision pour détruire la Constitution. Le vide constitutionnel permet donc l'établissement d'un nouveau texte. En ce sens, le pouvoir constituant dérivé mute en un pouvoir constituant originaire permettant l'établissement de la Vème République. L'intérêt de ce tour de passe-passe est de permettre le respect des formes légales mais aussi de permettre au pouvoir constituant dérivé de conditionné matériellement l'établissement de la future constitution en fixant des principes directeurs qui vont devoir être retranscrit: il y a donc assurance d'une continuité constitutionnelle. 
\item La loi ne détermine pas un blanc au bénéfice de De Gaulle. La loi du 3 Juin fixe des limites à l'écriture d'une nouvelle constitution. 
\end{itemize}

Les principes sont: \\
a. Seul le suffrage universel est la source du pouvoir, c'est du suffrage universel ou des instances élues par lui que dérive le pouvoir législatif et le pouvoir exécutif. \\
b. "Le pouvoir exécutif et le pouvoir législatif doivent être effectivement séparés de sorte à ce que le Gouvernement et le Parlement assument chacun, pour sa part et sous sa responsabilité la plénitude de leurs attributions". Il y a donc une séparation effective des pouvoirs et d'équilibre des pouvoirs. "Le Gouvernement aura pour charge de déterminer et de conduire la politique de la Nation". Le Parlement a pour responsabilité le vote de la loi, du budget et le contrôle politique de l'action du Gouvernement. \\
c. "Le Gouvernement doit être responsable devant le Parlement". Le régime doit donc être parlementaire, certes, mais on en déduit que le parlementarisme doit être rationalisé. \\
d. "L'autorité judiciaire doit demeurer indépendante pour être à même d'assurer le respect des libertés essentielles tels qu'elles sont définis par le préambule de la Constitution de 1946 et par la Déclaration des Droits de l'Homme à la quelle il se réfère". La Constitution devra donc définir les libertés fondamentales avec des textes existant déjà antérieurement: DDHC de 1789 et la préambule de la Constitution du 27 Octobre 1946. En ce sens, la future constitution doit déterminer une constitution matérielle et donc assurer la pérennité de la tradition constitutionnelle française. La protection des libertés est de plus, confiée aux juges, qui sont indépendants. 
e. "La Constitution doit permettre d'organiser les rapports de la République avec les peuples qui lui sont associés". La futur Constitution doit instaurer une communauté entre la métropole et les peuple des TOM qui constituaient en 1958 les partis colonisés de l'État Français. 


On notera que cette loi est silencieuse sur la question du Chef de l'État, un blanc est donc laissé à De Gaulle qui pourra donner toute l'importance qu'il souhaite à l'institution présidentielle. 

\subsection{Le projet de Constitution}

Le projet de Constitution représente en quelque sorte la fusion des principes directeurs et des idées constitutionnelles du Gouvernement. Parmi ces idées figurent un point essentiel qui est d'établir ou de restaurer l'autorité de l'État. Cette autorité n'est pas seulement organique pour les Gaullistes: pour eux, l'autorité doit être personnelle et fonctionnelle ; cela implique la reconnaissance d'un pouvoir politique au Chef de l'État. On retrouve donc les germes de ce qu'on appelle le présidentialisme. Le présidentialisme est irréductible à l'idée que les pouvoirs puissent être effectivement séparés. \\
Dans l'écriture du projet, il fallait donc inscrire en premier lieu que le Président de la République est la clé de voûte des institutions nouvelles. C'est immédiatement après le titre I consacré à la souveraineté que vient le titre II consacré au Président. Dans la Constitution de la IVe République, l'institution présidentielle arrivait au Ve titre. \\
Dans le projet de Constitution, la souveraineté est nationale mais appartient au peuple qui peut l'exprimer par le référendum et par la voie de ses représentants. Le premier représentant du peuple est le Président de la République et non plus les parlementaires. \\
Le Président est la clé de voûte du système institutionnel, le Président assure lui même le fonctionnement régulier des autres pouvoirs publiques. Il le fait par deux mécanismes: l'arbitrage d'abord (droit de dissolution de l'organe parlementaire) ; nomination ensuite (nomme aux principaux emplois civils et militaires: généraux, préfets, consuls, ambassadeurs, Gouvernement). Il a un pouvoir très étendu. Il dispose aussi de la capacité à faire appel directement au peuple ; d'adopter des décrets ; l'initiative et la maîtrise de la procédure de révision de la Constitution ; il peut concentrer les pouvoirs exceptionnels ; il est politiquement irresponsable devant les chambres ; il a donc immunité politique mais aussi juridique, le Président n'est responsable que devant le peuple, c'est le Gouvernement en contresignant qui est responsable, or il n'y a pas de contreseing pour certains pouvoirs, le Président est donc protégé. \\
Le pouvoir du Parlement est important, mais néanmoins réduit. Il exerce la fonction législative (voter la loi donc) et contrôle l'action du Gouvernement: c'est donc très classique. Cependant, la Constitution limite la capacité législative du Parlement en donnant des domaines dans lesquels le Parlement peut légiférer. Les autres domaines sont légiférés par le pouvoir exécutif via le pouvoir réglementaire. La Loi adopté par le Parlement a vocation à être contrôlée par une juridiction spéciale: le Conseil Constitutionnel. La fonction législative est donc désacralisé. De Gaulle désacralise la Loi dans la mesure où c'est l'acte adopté par les institutions tenus par les Partis Politiques. Or, il est opposant au phénomène partisan. \\
Ensuite, le Gouvernement dispose de la maîtrise de la procédure législative. Il dispose de la capacité d'obliger le Parlement à l'adoption d'un texte législatif. Le projet prévoit que la responsabilité gouvernementale soit difficile à mettre en oeuvre. Il est difficile pour le parlement de censuré un Gouvernement. 


Traits caractéristique du présidentialisme: le projet rend possible la concentration des pouvoirs entre les mains de l'exécutif et du parti qui le soutient.


Le projet de Constitution est approuvé au référendum du 28 Septembre 1958 avec plus de 79,2\% de oui et un taux de participation de 84\% du corps électoral. C'est la communauté Française qui adopte la Constitution et non pas seulement la métropole. Cependant, un territoire est contre: la Guinée, en conséquence, elle obtiendra son indépendance. \\
Le texte est promulgué le 4 Octobre 1958 par le premier président en exercice de la Vème République: René Coty. La Constitution est publiée au JORF le 5 Octobre 1958 et entre en vigueur à partir du 7 Octobre 1958. \\
À partir de là, tout va très vite. De Gaulle se présente à la présidence et est élu par un collège électoral élargi. Il est donc élu au suffrage universel indirect le 21 Décembre 1958, et prend ses fonctions le 20 Janvier 1959. 

\part{Le Droit constitutionnel des institutions sous la Vème République}

Quels sont les conditions de la régularité de l'action des institutions politiques ? \\
Ces institutions fonctionnent-elles séparément ? Y-a-t-il une séparation effective des pouvoirs ? \\
Nous verrons les institutions dans l'ordre qu'ils sont donnés dans la Constitution. On verra le statut des institutions, mais aussi leurs fonctions, ainsi que leurs relations avec les autres institutions. \\
Tous les régimes français se sont définis soit par une domination du pouvoir exécutif (Consulat, Empire etc.) soit par une domination du pouvoir législatif (IIIème République, IVe etc.) soit les pouvoirs se sont isolés (Directoire, IIe République). 

\chapter{Le Président de la République}

La Présidence de la République n'est pas une institution nouvelle, elle a été créée en 1848 et le premier Président élu était Louis Napoléon Bonaparte. \\
La Présidence sous la Vème République est une institution centrale, majeure, tant sur le plan Constitutionnel, tant sur le plan politique. Le Président est au centre d'un système institutionnel qu'il fait fonctionner pour la réalisation du programme sur la base duquel il a été élu. \\
En ce sens, le Président est véritablement le Chef de l'État et participe directement au gouvernement (au sens de l'autorité) de l'État. \\
La Vème République a connu 9 Président de la République dont 7 ont étés élus au suffrage universel:

\begin{itemize}
\item René Coty (4 Oct 1958 - 7 Janvier 1959) 
\item Charles de Gaulle (7 Janvier 1959 - 1969) \\
Il est le premier Président élu au suffrage universel direct en 1965. Il démissionne en 1969 suite à un échec de son référendum. 
\item Alain Poher, qui a remplacé De Gaulle après sa démission. Poher a donc assuré l'intérim présidentiel. \\
Ce même Poher sera de nouveau intérimaire lors du décès du Président Pompidou.
\item Georges Pompidou est le deuxième Président élu (1969 - 1974, mort en fonction).
\item Valéry Giscard D'Estaing (1974 - 1981).
\item François Mitterrand (1981 - 1995). 
\item Jacques Chirac (1995 - 2007).
\item Nicolas Sarkozy (2007 - 2012).
\item François Hollande (2012 - ...).
\end{itemize}


\section{L'institution présidentielle}

\subsection{L'élection du Président de la République}

L'élection du Président est un instant politique majeur car cette élection structure l'espace politique Français. Les principaux partis politiques (qu'on appelle parti de Gouvernement) ont tous pour priorité de remporter la compétition que représente l'élection présidentielle. \\
À cet égard, le mode constitutionnel d'élection conditionne fortement cette structuration. Le second tour n'oppose que deux candidats. Cette opposition conduit nécessairement à une bipolarisation de la vie politique. Le gagnant gouverne, le perdant est destiné à devenir l'opposition. \\
L'article 6, §1 dispose: "Le Président de la République est élu pour 5 ans au suffrage universel direct". Il est donc élu par l'ensemble du corps électoral, au titre de l'article 3 de la Constitution. On notera aussi que la circonscription électorale est unique. Ce qui affirme donc sa légitimité face donc à des parlementaires. \\
Cette forme de dévolution des pouvoirs est donc en rupture avec les deux Républiques différentes où le Président émanait du Parlement. Ce n'est pourtant pas une novation, le Président de la IIe République était lui aussi élu (au titre de son article 46). 

\subsubsection{La candidature à la Présidence}

Les conditions d'accès à la candidature sont définis par le code électoral, c'est donc la Loi qui fixe la condition. \\
Il existe des conditions de fonds et des conditions de forme.


Sur les conditions de fond, le candidat doit disposer de la nationalité Française, avoir au moins 18 ans, il doit ne pas avoir porté atteinte à la dignité morale de la fonction (donc ne pas avoir été condamné à des crimes). \\
Concernant les formes, le candidat doit déclarer sa situation patrimoniale (tant sur le plan mobilier que immobilière). Il présente ensuite une déclaration établissant ses comptes de campagnes sachant qu'en France, il existe un remboursement des frais électoral, il y a cependant un plafond à ne pas dépasser: en 2012 il était d'environ 16 millions pour le premier tour, 20 millions pour le second tour. Ces comptes sont vérifiés par une commission. \\
Autre condition de forme: le candidat doit déposer auprès du Conseil Constitutionnel une déclaration de candidature qui doit être parrainé par au moins 500 élus nationaux ou locaux recueillis dans au moins 30 départements différents. Cette règle du parrainage a pour objectif principal d'éviter les candidatures dites "fantaisistes". Cette règle est cependant peu efficace: en 2002, il y a 16 candidats et donc l'éclatement des suffrages. Il y avait 8 candidats de gauche en 2002. Les candidatures fantaisistes n'ont pas non plus été empêchés car l'humoriste Coluche a réussi à réunir les parrainages en 1981.


\subsubsection{Le scrutin}

Le Président de la République a été élu au scrutin universel indirect jusqu'à la révision constitutionnelle du 6 Novembre 1962. \\
Avant, le Président était élu par un collège de grands électeurs composé des députés, des sénateurs, des conseillers généraux (maintenant départementaux), des membres des assemblées des peuples et territoires d'outre mer ainsi que des représentants des conseils municipaux dont tous les maires. Le collège était composé d'environ 90 000 élus. \\
La légitimité démocratique du Président était donc de second degré. \\
À partir de la réforme constitutionnelle de 1962 et de la première élection de 1965, le scrutin est universel et direct. \\
De plus, le scrutin est uninominal, majoritaire, à deux tours. Les candidatures sont donc personnelles, opposant des personnalités politiques, qui ont vocation à être des leaders. Le candidat doit obtenir la majorité absolu des suffrages exprimés. Le candidat ayant eu le plus de suffrages exprimés est Jacques Chirac face à Lepen: 82,2\%. Le candidat en ayant obtenu le moins est VGE face à Mitterrand le 19 Mai 1974: 50,7\% des suffrages exprimés. \\
La majorité des suffrages est calculée à partir des suffrages exprimés et non du total des inscrits. Les bulletins blancs et nul ne sont pas pris en compte. C'est contestable sur le plan démocratique. La loi du 21 Février 2014 fait que les bulletins blancs sont décomptés à part et son annexés au procès verbal du bureau de vote, mais ne servent toujours pas à compter la masse électorale: les suffrages exprimés. \\
Dans l'hypothèse ou aucun candidat n'a obtenu la majorité absolu au premier tour, un second tour est organisé 15 jours plus tard entre les deux candidats arrivés en tête. À ce jour, aucun candidat n'a eu de majorité dès le premier tour. Même De Gaulle a fait un second tour en 1965 face à François Mitterrand. De Gaulle avait eu 44\% des suffrages au premier tour. \\
La Constitution prévoit un encadrement des dates du scrutin: c'est le Gouvernement qui fixe la date du scrutin, mais l'article 7, alinéa 3, fixe une période de cette élection, il dispose: "L'élection du nouveau Président a lieu 20 jours au moins, 35 jours au plus avant l'expiration des pouvoirs du Président en exercice". Le Gouvernement a donc une marge de manoeuvre de 15 jours. La date est déterminé à partir de l'entrée en fonction ; à cette date on ajoute la durée du mandat, on obtient alors la date d'expiration des pouvoirs. \\ 
Le premier tour a toujours lieu un dimanche, le second tour est organisé "le 14ème jour suivant". 

\subsubsection{Les conséquences de l'élection}

Pour parler des conséquences de l'élection, il faut parler du tournant qu'a constitué le référendum de 1962. Date à laquelle la France était en pleine guerre d'Algérie. Des groupes d'activistes, favorables à l'Algérie Française et en particulier l'OAS, préparent dès le début des années 60, une série d'attentat contre le Président en exercice: De Gaulle. Il échappe de peu à un attentat de 1961, dans la banlieue parisienne, au Petit Clamard. \\
Vu le danger auquel est soumis la fonction présidentielle, il pense à renforcer la légitimité présidentielle de la fonction. Il estime que les fonctions présidentielles sont d'une telle nature, à tel point exposé aux risques d'attentat, qu'il faut pouvoir donner au peuple le pouvoir d'élire lui même le Chef de l'État. \\
La réforme est adopté le 6 Novembre 1962 et la première conséquence est la présidentialisation du régime politique. La pratique politique et le tournant de 1962 ont permis l'émergence du présidentialisme: une situation qui a tendance à concentrer le pouvoir de l'État entre les mains du Président en raison de la conjonction de trois facteurs: son mode d'élection, l'étendue des attributions confiées, la disposition d'une majorité parlementaire. Ces trois facteurs combinés permettent d'assurer à la fois la primauté et le dirigisme du Président, ce sont les deux caractéristiques du présidentialisme. Ces deux caractéristiques remettent en cause la séparation des pouvoirs tel que voulu par les constituants. \\
Il y a donc une instrumentalisation des pouvoirs: le Président, élu par le peuple, bénéficie du soutien d'une majorité parlementaire en vue de la réalisation de son programme politique. Par ailleurs, le Président, premier représentant du peuple, dispose du moyen gouvernemental et administratif de faire conduire sa politique: le Gouvernement est lui même instrumentaliser à la conduite de ce programme. Cette volonté d'instrumentaliser explique pourquoi les élections législatives suivent toujours de quelques semaines l'élection présidentiel. Ceci permet un effet d'enchaînement logique: un président élu acquiert ensuite une chambre de la même couleur politique que lui: le Président a alors un soutien parlementaire lui permettant son présidentialisme. Le Président a une chambre d'enregistrement, qui, par ailleurs, il peut dissoudre à volonté (article 12 de la Constitution). \\
Ce phénomène de présidentialisme explique l'instauration du quinquennat en France, pour qu'il s'accorde avec la durée du mandat législatif. L'instrumentalisation est donc encore renforcée, et les risques de cohabitation sont largement réduits.


Dans cette perspective, le Président apparaît clairement comme le chef de la majorité parlementaire, davantage que le Premier Ministre. \\
De Gaulle déclarait en 1964: "Une Constitution c'est un esprit des institutions et une pratique". L'esprit de la Constitution a été fondamentalement revu. \\
La Constitution organise à priori un régime parlementaire où le Parlement vote loi et budget, contrôle l'action du Gouvernement, Gouvernement qui est politiquement responsable à contrario du Président. \\
Dans le cadre de ce régime parlementaire, le Gouvernement est central et intermédiaire entre le Parlement et le Président. Le Gouvernement possède donc un certain pouvoir, avec à sa tête un premier Ministre qui a des attributions large. Ce premier ministre devrait représenter, au même titre que le Président, une tête de l'État. \\
L'exécutif est dit bicéphale: cela fonctionne sur le principe d'une dyarchie. Le Président surveille le fonctionnement, le Premier Ministre mène. En cas de fait majoritaire, l'assemblé nationale est le simple faire valoir de la volonté du Président de même que le Premier Ministre qui exécute au quotidien le programme politique du Président.


De Gaulle déclarait en 1964: "Le Président est évidemment le seul à détenir et à déléguer l'autorité de l'État". C'est la même position que Sarkozy prendra en 2007: "Je serai un Président qui gouverne". Cela montre à quel point le Président est le chef de l'exécutif mais aussi de l'État. \\
"C'est le président qui est, et qui est seul, le représentant de la Nation" De Gaulle. Il est à noter que cette déclaration est contraire à la lettre de l'article 3 de la Constitution ("ses représentants"). \\
François Mitterrand dénoncera un coup d'État permanent, et pourtant, il déclare dans un message adressé au parlement le 8 Février 1981: "J'ai dis que mes engagements constitueraient la charte de l'action gouvernementale. J'ajouterai, puisque le suffrage universel s'est prononcé une deuxième fois, qu'ils sont devenus la charte de votre action législative." \\
Il n'y a plus de contre pouvoir: le Parlement n'a plus qu'à devenir la chambre d'enregistrement du Gouvernement. \\
Le fait majoritaire a eu pour effet de transformer en profondeur l'esprit de la Constitution. Mais encore faut-il qu'il y ait fait majoritaire. La prépondérance présidentielle est strictement proportionné au soutien parlementaire. Un Président qui n'a pas le soutien à l'assemblée est un président diminué. Cela correspond à un phénomène qui s'appelle la cohabitation. Dans cette hypothèse, il est contraint de nommer premier ministre le leader de la majorité parlementaire. \\
Trois épisodes de cohabitation: en 1986, entre Mitterrand et Chirac ; en 1993 entre Mitterrand et Balladur ; entre Chirac et Jospin en 1997. \\
Paradoxalement, une cohabitation offre une lecture plus proche de l'esprit de la cohabitation que pendant une période de fait majoritaire. Le Président se contente d'arbitrer sans être lui même un acteur de la vie politique (article 5) et le Premier Ministre gouverne: dans le sens où le PM et le Gouvernement ont la capacité réglementaire et la capacité de conduire la politique de la Nation.


\subsection{Le statut du Président de la République}

Le mot statut renvoie à l'état de l'organe. Le statut renvoie aux règles qui déterminent la durée de l'exercice des attributions présidentielles et aux règles qui fixent les conditions de l'exercice de ces attributions.


\subsubsection{Le mandat présidentiel}

Le mandat présidentiel est fixé par la Constitution. Il est doublement limité. Il est limité par sa durée d'une part, et par son renouvellement d'autre part. \\
Concernant la durée, il est fixé par la Constitution, il s'agit d'un quinquennat depuis le référendum constituant de 2000. Ce délai est en rupture avec la tradition parlementaire Française: depuis 1873 le septennat était la durée du mandat présidentiel. \\
Depuis la loi constitutionnelle du 23 Juillet 2008, l'article 6 alinéa 2 dispose "Nul ne peut exercer plus de deux mandats consécutifs". Cette réforme est très  inspirée de la tradition nord américaine. Il en résulte donc que le Président en exercice ne peut être réélu qu'une seule fois. Le mot consécutif est important: rien ne s'oppose à ce qu'un Président exerce plus de deux mandats dès lors que ceux-ci ne se succèdent pas. Rien ne s'oppose à ce qu'un ancien Président se présente  à nouveau après avoir fait deux mandats, laissé un mandat, puis se représente ensuite. \\
Différents arguments justifient cette réforme: cela favorise la possibilité d'une alternance politique ; limite l'effet d'usure du pouvoir présidentiel, effet qui a été patent à l'occasion des seconds mandats du Président Mitterrand et Chirac (respectivement 12 et 14 ans de Présidence). Comme tout réforme, celle-ci est également contestable: dans l'hypothèse d'un président politiquement efficace pour la promotion des intérêts de la nation et qui est empêché de se représenter ; cette réforme ne prémunit pas d'une situation à la Russe, l'hypothèse de l'élection d'un président factice qui nomme l'ancien Président Premier Ministre.


\subsubsection{La cessation des fonctions}

La cessation normale des fonctions intervient avec le terme du mandat présidentiel. De manière exceptionnelle, la cessation peut résulter d'une vacance de la présidence ou d'empêchement définitif de Président. \\
Première hypothèse: la démission. De Gaulle a démissionné en 1969 à la suite de la perte d'un référendum. \\
Deuxième hypothèse: le décès du Président en exercice. Le Président Pompidou est mort des suites d'une très longue maladie en 1974.


Troisième hypothèse: l'empêchement définitif du Président en fonction. L'empêchement correspond à une situation d'invalidité ne permettant plus au Président d'exercer la plénitude de ses responsabilités: par exemple, un AVC qui invaliderait le Président. La Constitution prévoit une procédure à l'article 7 alinéa 4, qui dispose "L'empêchement est constaté par le Conseil Constitutionnel à la majorité absolu de ses membres". Il décide à ce titre du caractère provisoire ou définitif de l'empêchement. \\ 
Si c'est définitif, le CC le constate et le Président en exercice est remplacé par de nouvelles élections. \\
Dans ce cas, le pouvoir n'est pas vacant, mais suspendu. L'intérim est alors nécessaire et est lui même provisoire dans la mesure où l'intérim exercé par le Président du Sénat ne peut excéder 35 jours. \\
Ni l'état de santé du Président de Pompidou ni celui du Président Mitterrand n'ont abouti à la constatation d'un empêchement. \\


\subsubsection{L'intérim et la suppléance}

L'intérim présidentiel est assuré par le Président du Sénat. On a connu deux situations d'intérim présidentiel: en 1969 et 1974, l'un et l'autre étant exercé par le même Président Alain Poher. \\
L'article 5 de la Constitution fixe la limite de l'intérim qui ne peut excéder 35 jours. Cela signifie donc aussi que l'empêchement provisoire ne peut durer plus de 35 jours. \\
Cet intérim est justifié par la nécessité de gérer les affaires courantes: si il n'y a plus de Président, il y a toujours une Présidence, pour réunir et présider le conseil des ministres, pour recevoir les Chefs d'État étranger, etc. \\
Confier cet intérim au Président du Sénat serait une anomalie démocratique, car l'intérimaire peut exercer l'étendue des pouvoirs présidentiels pendant plus d'un mois. Or, celui-ci dirige l'institution parlementaire la moins légitime. Par ailleurs, sur le plan politique, le Sénat, pour des raisons de sociologie politique, est majoritairement à droite depuis 1958 du fait de son mode d'élection, à cause de la surreprésentation des communes rurales. \\
Le sénat n'a connu qu'une seule alternance depuis 1958, il a été tenu 2 ans par la gauche de 2012 à 2014. \\
Le Président intérimaire aurait donc la capacité d'entraver l'action d'un gouvernement de gauche. \\
La Constitution donne des bornes à l'exercice de l'intérim: l'article 7 détermine une impossibilité d'exercer deux des principaux pouvoirs attribués au Président: le référendum et la dissolution. \\
Le dernier alinéa de l'article 7 prévoit une cristallisation du fonctionnement régulier des pouvoirs publics: le Gouvernement ne peut engager sa responsabilité devant le Parlement (article 49). De plus, aucune révision de la Constitution ne peut être entreprise. 


La suppléance consiste en un remplacement momentané et ne concerne que de courtes absences en dehors de l'hypothèse de l'empêchement. Ces remplacements peuvent consister en un voyage diplomatique, courte maladie, brève hospitalisation. \\
La suppléance est exercée par le Premier Ministre dans une logique de collaboration des pouvoirs des deux têtes de l'exécutif. \\
L'article 21 de la Constitution détermine cette compétence primo-ministérielle en fixant deux conditions à la suppléance: la suppléance ne concerne que la présidence des conseils et en particulier, le PM peut suppléer le Président dans la présidence du Conseil des Ministres ; la suppléance ne peut intervenir que sur la base d'une délégation expresse de pouvoir décidée par le Président: si le Premier Ministre préside le conseil des ministres, l'ordre du jour est tout de même fait par le Président. 

\subsubsection{L'irresponsabilité du Président}

La responsabilité est une exigence fondamentale dans un État de droit. Elle conduit à ce que l'auteur d'un fait préjudiciable doit en assumer les conséquences légales. Cette responsabilité, de manière générale, peut être de trois nature: elle peut être civile, c'est à dire pour un auteur d'un acte dommageable d'assumer les conséquences du préjudice subi par la victime, c'est l'idée de la réparation via des dommages et intérêts ; elle peut être ensuite pénale, il s'agit pour l'auteur d'un fait répréhensible (infraction) d'en assumer les conséquences sociales via le mécanisme de la répression ; elle peut être enfin politique, les autorités légitimes doivent agir dans la confiance des autorités dont elles procèdent (élection: l'élu doit agir dans la confiance de ses électeurs, etc.), la défiance doit donc conduire à la démission de l'autorité qui n'a plus la confiance de l'autorité dont elle procède.


1. Le régime général de la responsabilité présidentielle


L'article 67, alinéa premier de la Constitution fixe ce régime général. Sa rédaction résulte de la loi Constitutionnelle du 23 Février 2007 à la toute fin du mandat de Chirac. Il dispose "Le Président de la République n'est pas responsable des actes accomplis en cette qualité sous réserve des dispositions des articles 53-2 et 68." \\
L'article 67 fixe donc la règle de l'irresponsabilité présidentielle. À cette règle sont assorties deux exceptions.


a. La question de l'irresponsabilité politique


En application du principe de la séparation des pouvoirs, le PR est politiquement irresponsable devant le Parlement. Il en résulte que le PR en exercice n'a pas à recueillir la confiance du Parlement pour exercer ses attributions à la différence du Gouvernement. Ce mécanisme constitutionnel de l'irresponsabilité est typique du régime parlementaire. Cette irresponsabilité politique du chef de l'État découle du principe de l'inviolabilité du monarque constitutionnel. \\
Le PR étant élu au suffrage universel direct ne saurait donc être politiquement responsable que devant le peuple. Il peut lui signifier sa défiance politique de plusieurs façons: il peut décider de ne pas réélire le Président en cas de nouvelle candidature ; il peut décider d'élire une majorité parlementaire hostile au Président en cours de mandat (cohabitation) ; en émettant un vote négatif à l'occasion du référendum organisé par le PR.


b. La question de la responsabilité civile et pénale (responsabilité légale)


Sur le plan civil et pénal, l'expression utilisé par l'article 67 alinéa premier de la Constitution signifie que l'institution présidentielle bénéficie d'une immunité, qui est la règle selon laquelle une personne ne peut être ni poursuivi ni condamné en raison de la spécificité de ses fonctions. \\
Cette immunité est fondée sur le principe selon lequel le PR est garant au titre de l'article 5 de la Constitution, de la continuité de l'État. L'immunité est relative: il y a deux voies qui permettent d'engager la responsabilité pénale du PR. \\
La première voie est déterminée par l'article 53-2: c'est la voie internationale. Cet article reconnaît la compétence de la CPI à l'égard de la commission des crimes les plus graves par les autorités de l'État (crime de guerre, crime contre l'humanité, crime de génocide). \\
La deuxième voie est prévue à l'article 68 de la Constitution. Le PR peut être destitué en cas de manquement à ses devoirs manifestement incompatible avec l'exercice de son mandat. La destitution est prononcée par le Parlement réuni en Haute Cour. \\
Le PR en raison de cette immunité n'est pas un citoyen ordinaire. Le principe d'égalité devant la justice n'est pas applicable au cours de son mandat. Il faut ajouter à cela que le PR, dans les rares hypothèses où il peut être poursuivi, celui-ci bénéficie de ce qu'on appelle un privilège de juridiction. 


2. L'engagement de la responsabilité du Président


Les conditions d'engagement de la responsabilité sont complexes et se recoupent. Elles tiennent au moment où les actes fautifs ou délictuels se sont produit. Il faut s'intéresser aussi au moment de l'engagement des poursuites et spécialement si il est ou non en fonction. Il faut aussi voir la nature des actes: les actes commis peuvent-ils être attachés ou non à la fonction présidentielle ? (actes relevant de la fonction ou actes relevant du personnel).


Pour les actes commis avant l'exercice de la fonction présidentielle, ils sont des actes évidemment détachables de la fonction présidentielle. En ce sens, les actes doivent pouvoir faire l'objet d'une procédure civile ou pénale ordinaire. Cette hypothèse s'est vérifiée avec Chirac à propos de faits susceptible de recevoir une qualification pénale: détournement de fonds public, faits qui ont été commis antérieurement à son entrée en fonction. Le CC, dans une décision du 22 Janvier 1999: "Statut de la CPI", a pourtant estimé qu'il était nécessaire de protéger la fonction présidentielle pour tout l'exercice du mandat présidentiel. En d'autres termes, le PR, pendant son mandat ne peut être poursuivi devant les tribunaux ordinaires. La C.Cas dans un arrêt du 10 Octobre 2001, dit Breisacher, dit que le PR bénéficie d'un privilège de juridiction: il ne peut être ni poursuivi ni entendu comme témoin devant les juridictions pénales. Il ne pourrait être poursuivi que devant la haute cour. \\
La C.Cas précise qu'il est nécessaire alors que la prescription qui s'attache aux infractions pénales soient suspendus tout le temps de l'exercice du mandat. Le délai de prescription reprend son court le premier jour consécutif de la fin du mandat présidentiel. C'est ce qu'on appelle une fiction juridique. \\
Cette solution explique les poursuites lancées à l'encontre de Chirac à la fin de son mandat et sa condamnation en 2011 par le Tribunal Correctionnel de Paris à deux ans de prison avec sursis.


Pour les actes commis pendant le mandat, il faut opérer une distinction entre les actes détachables de la fonction, et ceux inhérente à la fonction. \\
Le PR bénéficie de l'immunité pénale et du privilège de juridiction. Il ne peut être poursuivi que devant la haute cour et pour une unique infraction: le manquement à ses devoirs. Il s'agirai de crimes contre l'État, comme par exemple la trahison, la conspiration ou l'intelligence avec des puissances étrangères. Les autres qualifications pénales ne concernent donc pas le PR comme une injure par exemple. \\
Pour les actes commis pendant le mandat et sans lien avec la fonction présidentielle, pendant la durée du mandat, le PR bénéficie encore du privilège de juridiction. À l'échéance du mandat, le PR peut être poursuivi devant les juridictions ordinaires, il redevient un justiciable de droit commun. \\
Dans tous les cas, le PR peut se soumettre à une procédure ordinaire mais uniquement si il y consent, comme par exemple en 2007 où Sarkozy à consenti à réaliser une procédure de divorce. 


3. La procédure de destitution


La procédure de destitution en cas de manquement à ses devoir, est fixée à l'article 68 de la Constitution. Procédure révisée en 2007 et précisé par la Loi organique du 24 Novembre 2014 avec une entrée en vigueur au 1er janvier 2016. \\
La Haute Cour est convoquée sur la proposition de l'une des deux chambres. La Loi organique encadre fortement l'initiative: elle doit être déposée par au moins un dixième des parlementaires. La proposition ne peut faire l'objet que d'une seule lecture (pas de navette parlementaire): si une des deux chambres la refuse, c'est la fin de la procédure. La proposition doit être motivée par des raison de fait et des raisons de droit, mais les parlementaires ne disposent d'aucun droit d'amendement. \\
Si elle est saisi par une proposition conjointe, la Haute Cour est présidée par le Président de l'Assemblée Nationale. Elle ne comporte aucun magistrat professionnel. Elle est composée de l'ensemble des parlementaires: environ 1000 membres ; c'est donc une justice politique. \\
Toutes les décisions prises par la Haute Cour, la destitution elle même, doivent être prises à la majorité des deux tiers. Ces décisions sont à effet immédiat.  


Du fait de son encadrement, cette procédure est impraticable. \\
La procédure a été très fortement politisée sur les voeux du Président Chirac qui a initié la Loi Constitutionnelle de 2007. Avant 2015, le Président bénéficiait d'une immunité et d'un privilège de juridiction, mais devant une institution qui fonctionnait sur des règles plus objectives et juridiques et moins encadrée: la Haute Cour de Justice, composée de 24 juges accompagné de magistrats professionnel et de conseiller à la C.Cas, 12 députés et 12 Sénateurs ; les décisions se prenaient à la majorité absolue. \\
La qualification pénale retenu de "manquement à ses devoirs" est particulièrement flou. Rien n'empêche l'utilisation de cette procédure de destitution par un Parlement hostile, et non pas à des fins pénales mais à des fins politiques. Il ne s'agit d'ailleurs pas de condamner le PR, mais de le destituer. \\
L'infraction pénale antérieurement retenue qui était la haute trahison était beaucoup plus précise. 


\section{Fonction du Président de la République}

La présidence sert la finalité centrale de la Constitution de 1958: la restauration de l'autorité de l'État. \\
Cette finalité centrale place le Président au dessus de toutes les autres institutions. Les assemblées et le Gouvernement sont chargés ensemble de produire la volonté nationale sous la forme de Loi et de décrets. \\
Cette prédominance fonctionnelle du chef de l'État est aussi assurée par la diversité et l'intensité des prérogatives qui lui sont confiées. L'article 5 de la Constitution qualifie le Président d'arbitre du bon fonctionnement des institutions. Cette image de l'arbitrage est une image trompeuse dès lors que le Président participe au jeu politique et assume et assure une fonction gouvernementale au sens politique du terme. \\
Le PR est la clé de voûte de la fonction de gouverner en France. Pour cela, il dispose de nombreuses attributions qui, en droit, peuvent être classées en deux catégories: les pouvoirs dispensés du contreseing, les attributions soumises à l'obligation du contreseing.

\subsection{Les attributions dispensés du contreseing}

Les compétences parmi les plus importantes sont dispensées de la contresignature du Premier Ministre. C'est un coup de force juridique, car, dans un régime parlementaire, le PR est politiquement irresponsable devant les chambres, et la tête du pouvoir exécutif est en principe le Chef du Gouvernement. Il résulte de cette irresponsabilité que les actes du Président doivent être contresignés par les membres du Gouvernement. Le défaut de contresignature constitue ce qu'on appelle un vice de formes. L'acte est donc juridiquement nul. Il est donc important que le Gouvernement marque son approbation aux actes présidentiels. Dans un État de droit, il n'est pas d'actes commis sans responsabilités. La fonction du contreseing permet au PM d'endosser devant le Parlement la responsabilité politique des actes du Chef de  l'État puisqu'il est irresponsable. \\
Or, le choix de la Vème République a été en rupture avec un système parlementaire classique: certains actes présidentiels échappent à toute responsabilité politique. C'est une façon d'assurer la primauté du pouvoir présidentiel qui adopte des actes par sa seule signature. \\
Les attributions non soumis à contreseing sont donnés à l'article 19 de la Constitution qui dispose "Les actes du Président de la République autre que ceux prévus aux article 8 (premier alinéa), 11, 12, 16, 18, 54, 56 et 61 sont contresignés par le Premier Ministre, et le cas échéant, par les Ministres Responsables". \\ 
Démission et nomination du Premier Ministre, référendum législatif, dissolution, pouvoirs exceptionnels, pouvoir de communication du Président avec le Parlement, la saisine du CC, la nomination de trois membres du CC et de son président sont dispensés de contreseing. 

\subsubsection{La nomination du Premier Ministre et la démission du Gouvernement}

La nomination du PM est à l'article 8, alinéa premier de la Constitution qui dispose "Le Président de la République nomme le Premier Ministre et met fin à ses fonctions par présentation par celui-ci de la démission du Gouvernement". \\
La procédure de nomination du PM reflète très bien l'idée que le pouvoir exécutif est placé sur un plan d'inégalité organique: l'un des organes exécutif procède de l'autre. Ce pouvoir de nomination est ce que l'on appelle un pouvoir discrétionnaire dans le sens où le PR dispose de toute la marge d'appréciation nécessaire pour choisir. Le PM n'a pas à aller chercher la confiance de l'AN pour être nommé par le PR. Le PM n'existe que par la seule volonté du Président. Il y a donc une idée de concentration des pouvoirs exécutifs. 










\end{document}
