\documentclass[12pt, a4paper, openany]{book}

\usepackage[latin1]{inputenc}
\usepackage[T1]{fontenc}
\usepackage[francais]{babel}
\date{}
\title{Cours de Droit Constitutionnel (UFR Amiens)}
\pagestyle{plain}

\begin{document}

\chapter{Introduction: Les origines de la Vème république}

La Constitution du 4 Octobre 1958 est la Constitution du 5ème régime républicain qu'a connu la France depuis 1789. \\
La première république court du 21 Septembre 1792 au 18 Mai 1804. La deuxième république court du 4 Novembre 1848 au 2 Décembre 1851. La troisième République court du 4 Septembre 1870 jusqu'au 10 Juillet 1940. La 4ème république court du 23 Avril 1944 (constitution d'octobre 1946) au 4 Octobre 1958. \\
La Constitution de la 5ème république est en vigueur depuis presque 60 ans, ce qui en fait, depuis 1791, le deuxième régime le plus long après la 3ème République. Ce qui explique cette durée, c'est la souplesse, ce n'est pas un isolement des pouvoirs, mais de collaboration mais aussi l'instauration d'une véritable autorité d'État, qui transite par l'autorité du pouvoir exécutif: l'autorité de l'État passe entièrement par l'autorité du Chef de l'État. La constitution reconnaît clairement la suprématie du Président sur les autres pouvoirs d'où le fait que certains parlent de monarchie républicaine. \\
La 5ème a aussi su s'adapter et évoluer à son environnement, environnement social ou international. La procédure de révision est une procédure efficace d'adaptation de la constitution. Elle a été révisée 24 fois. La Constitution tel que nous la connaissons n'a donc plus grand chose à voir avec le texte de 1958. En 1958 apparaissait par exemple un titre sur les colonies, qui n'a plus lieu aujourd'hui. Une partie de la Constitution est aussi dédié à l'Union Européenne, partie qui n'existait pas en 1958. \\
La dernière révision est du 23 Juillet 2008. Sur un total de 106 articles, 39 articles sont modifiés et 9 articles sont ajoutés. 

\section{Le contexte inhérent à la mise en place de la Constitution du 4 Octobre 1958}

Le constat de la mise en place de la 5ème République est un constat rapide: il y a un concours de circonstance ; en effet, la 5ème République naît de l'impotence constitutionnelle antérieure, notamment de l'incapacité de la 4ème République à résoudre les immenses défis qui se présentaient à lui. \\
En effet, la 4ème République connaît une incapacité institutionnelle d'une part et une incapacité politique d'autre part.


Sur le plan politique, la 4ème n'a pas su résoudre les défis qui se présentaient à elle: la reconstruction après la seconde guerre mondiale (en 1958, la reconstruction n'est pas terminée: 5 millions de personnes vivent dans des bidonvilles ou sont sans abris) ; le défi de la décolonisation (conflit d'Indochine, colonies d'Afrique du Nord, Maroc puis Algérie). En 1958, la France se retrouve en pleine guerre en Algérie: c'est ce conflit qui va précipiter la chute de la IVème. \\
Sur le plan économique, la France est en retard, c'est principalement une économie de subsistance constitué de micro-entreprise: confine à l'artisanat plutôt qu'à l'industrie là où l'Allemagne ou encore les USA ont déjà fait naître les premières multinationales. \\
Quatrième défi: la France a eu de nombreux problèmes à se positionner dans le cadre des prémices de la Guerre Froide mais aussi de l'Europe qui débute sa construction. En 1958, la France ne dispose donc d'aucun leadership (diplomatique ou politique).


Sur le plan institutionnel, la constitution de 1946 a reproduit les défauts de la troisième République: entre 1946 et 1958, la durée moyenne des gouvernements est de moins de 6 mois. Conséquence: faiblesse de l'exécutif et donc fondamentalement, faiblesse de l'État. \\
L'État en 1958 est donc une institution à la dérive confronté au conflit Algérien qui fait rage depuis 1954. Ce conflit oppose les partisans d'une Algérie Française et ceux qui veulent la construction d'un État indépendant. Sur place, et notamment à Alger, le conflit radicalise les positions d'une partie importante de l'armée Française, traumatisée de ses échecs militaires antérieurs (notamment en Indochine). Il n'est donc pas question de subir une défaite supplémentaire. \\
Ces militaires sont soutenus par une grande majorité des Français installés en Algérie et qui ont des intérêts sur place, notamment des intérêts économiques. Ils sont installés depuis plus de 120 ans. Le contexte est donc particulièrement tendu. Il se tend encore lorsque Pierre Phlimlin est pressenti comme futur président du conseil le 13 Mai 1958. Celui-ci est un Homme politique très ouvert aux négociations en vue de la préparation d'une indépendance avec le FLN. Cette perspective d'investiture engendre une insurrection en Algérie et la prise du Gouvernement Général à Alger par les militaires. \\
L'armée procède à un putsch local en Algérie en y mettant évidemment en place un régime autoritaire, autocentré (une junte). Ceci prend officiellement le nom de comité de salut public. Le Gouvernement Français ne réagit pas, car il ne peut réagir. L'assemblé ne peut pas non plus faire grand chose: les institutions sont paralysées. \\
Sur place, le Général Salan prend la tête de la Junte. Il est un fidèle du Général de Gaulle et lance à ce dernier un appel pour former un Gouvernement. Celui-ci ne répond pas officiellement et réprouve même ce coup de force, mais il affirme être prêt à assumer ses responsabilités politiques comme il l'a déjà fait dans d'autres contextes de crise. \\
Le 24 Mai 1958, plusieurs bataillons de parachutistes arrivent en Corse et y instaure un comité de salut public. L'armée prépare donc une opération de parachutage à Paris prévu dans la nuit du 27 au 28 Mai 1958 appelée "Résurrection". L'armée prépare donc un coup d'État. \\
Le Gouvernement Phlimlin démissionne donc. René Coty appelle De Gaulle à formé un Gouvernement. Son Gouvernement est investi le 1er Juin 1958 par 329 voix contre 224. Son rôle sera de préparé l'établissement d'un nouveau régime, avec un objectif ultime: restaurer l'autorité de l'État. En ce sens, De Gaulle propose au Parlement une loi constitutionnelle permettant d'établir une nouvelle République: elle est votée le 3 Juin 1958. \\ 
Cette loi donne au Gouvernement un mandat: "Élaborer puis proposer au pays par la voie du référendum les changements indispensables aux institutions".

\section{Les bases constitutionnelles de la Vème République}

Ces bases se trouvent dans deux sources intrinsèquement liées: la loi du 3 Juin 1958 qui porte les empreintes des idées constitutionnelles de De Gaulles.

\subsection{Les bases intellectuelles}

Les bases reposent sur la pensée politique du Général de Gaulle, mais aussi sur les idées de ses principaux collaborateurs: Michel Debré (juriste), René Capitant (juriste et professeur de droit constitutionnel): deux juristes fondamentalement opposés au régime parlementaire à la Française. Ils reprendront des mécanismes qui permettent la longévité et donc rejetteront les mécanismes qui ont conduit aux incapacités des régimes antérieurs. \\
En 1946, De Gaulle avait déjà donné ses idées constitutionnelles dans un discours donné à Bayeux. Pour lui, le vote définitif de la loi et le vote définitif du budget appartient à une assemblée parlementaire élu au suffrage universel direct. Il est donc pour une démocratie représentative. \\
Il est aussi favorable à un bicaméralisme, qui doit être inégalitaire et technique: inégalitaire dans le sens où la primauté doit être accordée à l'assemblée la plus légitime, cette assemblée est seule capable de censurée le Gouvernement. La seconde chambre serait donc cantonnée la fonction purement législative. Technique car la seconde chambre doit être recruté autrement que le Sénat républicain traditionnel: elle devrait être recruté sur une base sociale et professionnel donc représentant de la société civile et non délégué de partis politique. \\
De Gaulle souhaite établir la responsabilité politique du Gouvernement et de ses membres devant l'assemblée élue au suffrage universel. Il souhaite aussi que le parlementarisme soit rationalisé pour avoir une stabilité politique. \\
Il souhaite aussi une véritable autorité de l'État donc par la reconnaissance de l'autorité du Chef de l'État donc d'une véritable présidence de la République duquel émanerait le Gouvernement et son chef. \\
En d'autres termes, de Gaulle souhaite un régime parlementaire dualiste où le Gouvernement est responsable devant la chambre et devant le chef de l'État. Le Président dans cette perspective n'est pas un président inactif ou effacé: il est un arbitre engagé, un président qui gouverne et fait gouverner, s'assure du fonctionnement régulier des autres institutions. Pour cela il doit être titulaire du pouvoir réglementaire, il doit diriger le conseil des ministres, il doit pouvoir dissoudre l'assemblée élue au pouvoir universel. Le Président doit aussi pouvoir concentré entre ses mains l'ensemble des pouvoirs en cas de péril de la nation (dictature constitutionnelle): c'est ce qui manquait au président Lebrun en 1940 ; sur cela, De Gaulle s'inspire de la Constitution de 1804 qui permettait à l'empereur de concentré entre ses mains ces pouvoirs exceptionnelles. L'inspiration est impériale. Les Gaullistes n'hésitent pas à chercher les mécanismes efficaces dans toutes les constitutions. \\
Il est évident qu'un Président avec une palette de pouvoirs aussi large ne peut émané du Parlement: sa légitimité doit venir d'un électorat. 


Ce discours va fortement influencé la loi constitutionnelle du 3 Juin 1958.

\subsection{Les principes directeurs}

Juridiquement, la loi Constitutionnelle du 3 Juin 1958 est un peu particulière: son objet est fondamentalement de réviser la constitution. Il s'agit d'une loi de révision au titre de l'article 90 de la constitution de 1946. Or, cette loi permet de fonder juridiquement la constitution de la Vème République. \\
Cette loi procède en trois temps:
\begin{itemize}
\item La loi organise une procédure de révision spécifique par dérogation à l'article 90 afin de rendre la procédure plus simple. Le premier paragraphe de l'article unique de cette loi dispose "Par dérogation aux dispositions de son article 90, la Constitution sera révisée par le Gouvernement investi le 1er Juin 1958 et ce dans les formes suivantes...".
\item Le paragraphe deux de cette loi dispose "Le Gouvernement de la République établit un projet de loi Constitutionnelle mettant en oeuvre les principes ci-après...". On utilise donc la procédure de révision pour détruire la Constitution. Le vide constitutionnel permet donc l'établissement d'un nouveau texte. En ce sens, le pouvoir constituant dérivé mute en un pouvoir constituant originaire permettant l'établissement de la Vème République. L'intérêt de ce tour de passe-passe est de permettre le respect des formes légales mais aussi de permettre au pouvoir constituant dérivé de conditionné matériellement l'établissement de la future constitution en fixant des principes directeurs qui vont devoir être retranscrit: il y a donc assurance d'une continuité constitutionnelle. 
\item La loi ne détermine pas un blanc au bénéfice de De Gaulle. La loi du 3 Juin fixe des limités à l'écriture d'une nouvelle constitution. 
\end{itemize}



\part{Le Droit constitutionnel des institutions sous la Vème République}

Quels sont les conditions de la régularité de l'action des institutions politiques ? \\
Ces institutions fonctionnent-elles séparément ? Y-a-t-il une séparation effective des pouvoirs ? \\
Nous verrons les institutions dans l'ordre qu'ils sont donnés dans la Constitution. On verra le statut des institutions, mais aussi leurs fonctions, ainsi que leurs relations avec les autres institutions.

\chapter{Le Président de la République}

\section{L'institution présidentielle}

\subsection{L'élection du Président de la République}

\subsection{Le statut du Président de la République}






\end{document}
