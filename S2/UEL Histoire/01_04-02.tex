\documentclass[12pt, a4paper, openany]{book}

\usepackage[latin1]{inputenc}
\usepackage[T1]{fontenc}
\usepackage[francais]{babel}
\date{}
\title{Cours d'Histoire (UFR Amiens)}
\pagestyle{plain}

\begin{document}
\maketitle

\section{L'histoire du Parlement Français}

En France, le Parlement a une histoire qui date de plus de 200 ans. \\
Il est apparu en 1789: ce sont les États généraux convoqués par Louis XVI en Mai 1789. Ces États-généraux se constitue en Assemblée Nationale le 17 Juin 1789: à côté de la souveraineté royale, il y a la souveraineté nationale. Ils se réunissent à Versailles jusqu'en Novembre 1789 car en Octobre, les Parisiens viennent chercher Louis XVI. \\
Le Parlement siégera près des Tuileries dans une salle qui s'appelle "La salle du Manège". Jusqu'en 1795, le Parlement est monocaméral. En 1795, le Parlement devient bicaméral, avec le conseil des 500 et le conseil des Anciens. \\
Le conseil des 500 déménagera au Palais Bourbon et le conseil des Anciens ira au Palais du Luxembourg. \\
En 1799, après son Coup d'État, Bonaparte modifie les institutions: il y a désormais deux assemblées: le corps législatif et le Sénat. Sous la période Bonaparte, le Parlement est écrasé par le premier consul (qui deviendra ensuite empereur).


Les Parlements se développeront sous les monarchies censitaires entre 1814 et 1848. C'est pendant cette période que l'expérience du Parlement s'enracine dans le Parlement et voit son pouvoir croître: il y a désormais la Chambre des députés (élus au suffrage censitaire) et la Chambre des Pairs (aristocrates nommés par le Roi). \\ 
Des coutumes qui ont encore lieu aujourd'hui vont se créer comme le principe de la commission, l'oratoire (principe de l'éloquence). \\
À partir de 1830, ils auront l'initiative législative qui jusque là appartenait au Roi. De plus, on notera que le Gouvernement a une responsabilité politique devant le Parlement, c'est donc un régime parlementaire. Le parlementarisme est marqué par l'aménagement de deux grandes salles: l'hémicycle du Palais Bourbon (construit entre 1828 et 1832) mais aussi l'hémicycle du Palais du Luxembourg (édifié entre 1836 et 1842).


Avec la Révolution de 1848, le Parlement va connaître des aléas. Le Parlement deviendra monocaméralisme et sera élu au suffrage universel masculin.\\
Louis-Napoléon Bonaparte comme son Oncle n'aime pas le Parlement et va donc l'écraser mais gardera deux chambres qui auront les mêmes noms que sous le premier Empire. \\
Le Parlement est dépouillé de ses pouvoirs sous le second empire. Dans les années 1860, Napoléon redonne quelques pouvoirs au Parlement. \\
Il faut attendre l'installation définitive de 1870 de la République pour que le Parlement retrouve ses pouvoirs.


La Constitution de la IIIe va créer deux chambres (Chambre des députés et Sénat) avec un suffrage universel direct pour les députés et indirect pour les Sénateurs. \\
Pour la première fois, il y a un article dans la Constitution qui établit le régime parlementaire: "Les ministres sont responsables de la politique du Gouvernement devant les chambres". \\
En réaction au Régime de Napoléon, le PR va perdre beaucoup de pouvoir au profit du Parlement. Le régime parlementaire est institutionnalisé. \\
L'hégémonie du Parlement est créée et les Gouvernements sont instable. \\
Concernant les commissions, elles sont spécialisées et permanentes. Les groupes politiques apparaissent mais il n'existe pas de discipline partisane. \\
Le Parlement veille à ce que le PR n'ait pas trop de pouvoir et élisent des personnalités de second plan et le poussent à la démission si il devient trop populaire comme en 1924 où le Président Millerand doit démissionner. \\
Le Parlement a tellement de pouvoir et un travail tellement lent que certaines personnalités demandent une réhabilitation du pouvoir exécutif. C'est cela qui conduit à la création d'une grande vague anti-parlementaire.


Au lendemain de la seconde guerre mondiale, la Chambre des députés devient l'AN et le Sénat devient le Conseil de la République. Le principe de la IVe reste le même que la IIIe République: hégémonie du Parlement. 


\section{Le Parlement de la Ve République}

Le Sénat siège au Palais du Luxembourg. Depuis la révision constitutionnelle de 2008, le nombre de Sénateur est limité à 348, leur mandat est de 6 ans, renouvelable par moitié tous les 3 ans. Pour être Sénateur, il faut avoir 24 ans révolus et être élu par les grands électeurs (Parlementaires, Conseillers régionaux et départementaux, Conseillers municipaux élus par leur conseil), le vote est obligatoire et le scrutin se fait soit à la proportionnelle (plus de trois sénateurs dans un département) soit majoritaire à deux tours. \\
Sous la Vème, on ne peut pas être parlementaire et membre du Gouvernement. On ne peut pas non plus être Sénat et Député ou Parlementaire et membre du CC. Un Sénateur ne peut pas cumuler plus d'un mandat local avec son mandat de Sénateur. \\
Les sénateurs jouissent d'une immunité qui les protège de toutes poursuites judiciaires, sauf si le Sénat lève son immunité parlementaire ou si il est pris en flagrant délit.


L'AN siège au Palais Bourbon. \\
La tribune du Palais Bourbon est particulier car il y a un bas relief de Lemot représentant l'Histoire et la Renommée (vestige du Conseil des 500). \\
Pour être éligible, c'est ici 18 ans, l'élection se fait dans le cadre de circonscription dans un suffrage uninominal à deux tours. \\
Un député a les même incompatibilités et la même immunité. 


Au XIXe siècle, les députés étaient des grands propriétaires terriens, surtout jusqu'en 1848, dû au suffrage censitaire. Avec la IIIe République, la domination était plutôt bourgeoise: un quart des députés viennent de la petite et moyenne bourgeoisie. \\
Dans l'entre deux guerres, il a existé ce qu'on appelle la République des avocats car ils étaient surreprésentés dans les institutions. À partir de 1936, ce sont les professeurs qui avaient cette place. C'est donc plus largement les fonctionnaires qui investissent l'Assemblée au milieu du XXe siècle. \\
Si on regarde les députés par CSP, il y a une majorité de fonctionnaires, de professions libérales, de cadres et d'ingénieurs. Il y a peu ou pas d'ouvriers, très peu d'agriculteurs etc. \\
Il en est de même au Sénat. La majorité sont des retraités, sinon les groupes dominants sont les fonctionnaires, cadres et professions libérales: les mêmes qu'à l'AN.


Concernant la féminisation du Parlement, il est récent et reste très minoritaire car elles peuvent voter et se présenter seulement depuis 1944. Tout de suite, un certain nombre de femmes sont élue (entre 5 et 7\%), puis chute ensuite jusqu'à la fin des années 70 où on retrouve les chiffres de la Libération, chiffres qui se sont ensuite accrus jusqu'en 2012 où 27\% des femmes sont des députés. \\
Au Sénat, le phénomène est le même pour arriver aujourd'hui à 25\%. 


L'âge moyen des députés est de 55 ans, celui des Sénateurs est de 66 ans. \\
Il faut noter que cet âge croît, en particulier depuis le début des années 80. Avec l'arrivée des Socialistes au pouvoir, les Hommes politiques se sont renouvelés, mais ils ne se sont pas renouvelés depuis. \\
À l'AN seul 10\% des parlementaires ont moins de 40 ans. Au Sénat, les moins de 40 ans ne représentent même pas 1\%. La France peine à renouveler son personnel parlementaire: par rapport aux autres pays, le nombre de seniors parlementaires en France est bien plus élevé que dans d'autres pays. Il y a 9 fois plus de députés de plus de 60 ans que de députés de moins de 40 ans en France. 

\section{Les pouvoirs du Parlement Français}

Le Parlement vote la Loi (Art. 24 de la Constitution). La loi la plus importante chaque année est la loi de finance (souvent votée à l'automne pour l'année d'après). Ils doivent voter l'ensemble des dépenses de l'État, chapitre par chapitre, pareil pour les recettes (239 milliards d'euros de recettes en 2015 et un peu plus de 300 milliards de dépenses). \\
Deuxième loi importante: le financement de la sécurité sociale. La branche maladie est déficitaire alors que les autres sont plutôt équilibré. \\
Ces lois occupent un quart du travail des parlementaires. 


Le Parlement contrôle l'action du Gouvernement. Il existe des votes de confiance, ou bien de censure. Seuls les députés peuvent censurés le Gouvernement. \\
Le vote de confiance a, en général, lieu au début d'un Gouvernement. \\
La motion de censure n'a eu lieu qu'une fois en Octobre 1962. \\
Les parlementaires contrôlent aussi l'action du Gouvernement en créant des commissions d'enquête sur l'action du Gouvernement. 


Les parlementaires peuvent également proposer l'organisation d'un référendum. Ils autorisent aussi la déclaration de guerre ou autorise la prolongation d'interventions militaires au delà de quatre mois. Ils peuvent aussi saisir le Conseil Constitutionnel. Le Parlement peut également se constituer en haute cour et destituer le Président en cas de manquement à ses devoirs. Il peut également modifier la Constitution en congrès.


Chaque assemblée élit son président et définit son règlement intérieur. Bartolone est le Président de l'AN et Larcher est Président du Sénat. \\
Le Président de l'AN préside le congrès de Versailles alors que le Président du Sénat assure l'intérim de la fonction présidentiel. \\
Les deux chambres sont réunies en sessions ordinaires de Octobre à Juin. Pendant cette session, le Parlement ne siège pas tous les jours. Au maximum, il siège 120 jours. Le Gouvernement peut cependant rajouter des jours si nécessaire. Le PR peut convoquer le Parlement en session extraordinaire. \\
Les débats publics du Sénat et de l'AN ont le droit d'être connus de tous les citoyens (publicité des débats). Les séances sont ouvertes au public et un compte rendu intégral est rendu tous les jours des débats de la semaine précédente. De plus, les séances sont filmées. \\
Il peut arriver que les parlementaires se réunissent en comité secret. C'est arrivé entre 1914 et 1918. \\
Le vote d'un parlementaire est individuel mais peut recevoir une procuration. Les votes peuvent se faire à main levé, en assis levé, de manière électronique (scrutin ordinaire), ou encore appelé à la tribune pour mettre une enveloppe dans une urne. \\
Une fois par semaine, les parlementaires questionnent le Gouvernement. C'est une séance spécialement suivi par les parlementaires. \\
Les parlementaires se réunissent dans des groupes politiques (il faut 15 députés pour former un groupe), il y a actuellement 6 groupes parlementaires. Au Sénat, il y a également 6 groupes, mais il faut 10 Sénateurs pour former un groupe. \\
Les parlementaires siègent dans des commissions permanentes selon leur spécialité (Affaire culturelle et éducation ; Affaires économiques ; Affaires étrangères ; Affaires sociales ; Défense nationale et forces armées ; Développement durable et aménagement du territoire ; Finances ; Lois). Les Parlementaires peuvent créer des commissions extraordinaires.


\section{Comment produit-on une Loi ?}

Il faut d'abord l'initiative de la Loi: le fait de pouvoir proposer une loi. L'art 39 de la Constitution donne l'initiative au PM et aux membres du Parlement. La plupart des lois votées viennent du Gouvernement. \\
Quand une loi est proposé par le PM, on l'appelle Projet de loi. Quand il s'agit d'un Parlementaire, on parle de proposition de Loi. Les Parlementaires ne peuvent pas proposer n'importe quel type de lois. Les Parlementaires ne peuvent pas proposer de lois qui feraient baisser les recettes ou augmenter les dépenses (art. 40). \\
Le PM se fait aider du Conseil d'État pour rédiger correctement ses projets de Lois. Quand elle est rédigée, elle est adoptée en Conseil des Ministres. La Loi adopté en Conseil des Ministres est alors déposé sur le bureau d'une des deux assemblées, la plupart du temps à l'AN.


Pour être adoptée, une Loi doit être adoptée dans les mêmes termes entre l'AN et le Sénat. \\
Premièrement, le projet ou la proposition est examiné en commission. Les parlementaires amendent, donne des avis etc. \\
Une fois l'étude en commission terminée, le texte sera étudié en séance publique. Ils vont d'abord en discuter article après article et voter chaque article. À la fin, ils votent l'ensemble de la Loi (vote solennel). On notera qu'il existe un phénomène d'obstruction. Le Gouvernement peut notamment utiliser l'article 49, alinéa 3, pour faire adopter la Loi plus rapidement et notamment faire face à l'obstruction ou faire face à une majorité  incertaine. Depuis 2008, l'article 49, alinéa 3 ne peut être utilisé qu'une fois par session. \\
Une fois adopté en première lecture, le texte passe devant le Sénat, où, de la même manière, le projet est examiné en commission puis en séance publique. À la fin, le Sénat adopte le texte. Si le texte est le même que celui de l'AN, la Loi est adoptée. Mais si ce n'est pas le même, le texte va faire la navette entre chaque assemblée. \\
Si aucun texte commun, le Gouvernement peut avoir recours à la commission mixte paritaire (7 députés et 7 sénateurs qui sont nommés entre eux) pour produire un texte commun. Si un texte commun est trouvé, le projet est adopté. Sinon, l'AN statue définitivement (dans deux cas sur trois, la CMP aboutit à un accord). Plus de 70\% des Lois ont été adoptées selon des procédures normales. Dans 20\% des cas, les projets sont adoptés en CMP. Dans 10\% des cas seulement, l'AN statue définitivement. 


Une fois la Loi adopté, la Loi peut être transmise au CC si le PR, le Président du Sénat, le Président de l'AN ou 60 députés ou sénateurs le souhaitent. \\
Le PR promulgue ensuite la Loi au JORF et devient donc ensuite appliqué. Des décrets d'application peuvent compléter la Loi. 


\chapter{La décentralisation et les collectivités locales}

La décentralisation est un processus qui transfère des compétences administratives de l'État central aux collectivités locales (communes, départements, régions). 


\section{Histoire de la décentralisation}

Pendant très longtemps, jusque dans les années 80, la France fonctionnait sur un modèle de centralisation. La monarchie était très centralisé, la Révolution n'a entraîné aucune rupture dans ce domaine ni l'empire. \\
Sous l'ancien régime, il existait des autorités locales, mais c'était plutôt de la déconcentration que de la décentralisation: de plus les autorités étaient très différentes selon les régions. \\
Après la Révolution, il y a une uniformité qui se crée: le Pays est divisé en Département et en Commune, mais le contrôle de l'État est resté fort sur ces collectivités ; il s'est même renforcé avec la création par Napoléon en 1800 du Préfet. \\
Depuis le début du XIXe siècle, on a assisté à deux évolutions: les institutions des collectivités locales sont devenus de plus en plus élus et non nommé par le Gouvernement et leurs compétences se sont élargis, restant sous l'autorité du Gouvernement.


À la fin de la seconde guerre mondiale, on a pris conscience des méfaits de la centralisation. En 1947, Jean-François Gravier fait prendre conscience que la France était un pays trop centralisé. L'État a alors mis en place une politique d'aménagement du territoire, mais cela n'a pas aboutit à la décentralisation. \\
En 1969, De Gaulle propose la décentralisation au référendum qui est rejeté. \\
C'est Mitterrand qui, le 2 Mars 1982, crée un vrai processus de décentralisation. Cette loi prévoit que ce n'est plus le préfet qui administre une région mais le Conseil général. Depuis 1982, il y a trois niveaux: commune, département, région. \\
Cette loi a été complété car la décentralisation a créé un consensus, la droite a poursuivi le processus, en 1999, une loi est votée sur l'intercommunalité. 


Au début du XXIe siècle, une nouvelle étape est en cours. En 2003, la France a connu le début de l'acte 2, mené par Raffarin, qui a fait adopté une réforme de la Constitution qui inscrit dans celle-ci la décentralisation. \\
Le développement de la décentralisation a mis en évidence un certain nombre de difficultés. Il y a eu par exemple, une embauche croissante de fonctionnaire se répercutant par une hausse d'impôts locaux. De plus, toutes les régions/communes se sont mis à s'occuper de tout, n'importe comment, sans se coordonner. \\
En 2007, le Gouvernement de Sarkozy cherche à rationaliser la décentralisation. Ils voteront d'ailleurs une loi créant le conseiller territorial mais qui sera abrogé en 2012 par la nouvelle majorité. \\
Cette nouvelle majorité vote en 2013 la division du nombre de canton sans diviser le nombre d'élus. En Janvier 2014, une loi est votée pour permettre aux grandes villes de France de s'unir avec leurs banlieues pour créer des métropoles. En Janvier 2015, le nombre des régions sont réduites, il y a depuis plus que 13 Régions métropolitaines. La dernière loi est la loi NOTR d'août 2015 qui précise les compétences de chaque collectivité territoriale pour éviter les chevauchements ; les départements ont vu leurs compétences limités, les intercommunalités sont devenuss plus importante avec la région. \\

\section{La Commune}

La Commune est née sous la Révolution Française, c'est, depuis, la cellule administrative de base du pays. S'est posée la question du nombre de communes à créer. La solution qui s'est imposée est de transformer les paroisses en communes, il y a donc en France, 36 800 communes. 70\% des communes ont moins de 700 habitants et 6\% seulement ont plus de 5000 habitants. Les aires urbaines regroupent la majorité de la population (80\%). \\
Les communes n'ont jamais vraiment été remises en cause. Longtemps, le Maire était choisi par le préfet ou le Gouvernement, mais depuis le 5 Avril 1884, tous les conseillers municipaux sont élus au suffrage universel et le maire était élu par ce conseil. Paris était une exception jusqu'en 1977 avec un préfet de police à sa tête.  


Depuis 1971, il existe une loi qui permet aux communes de fusionner entre elles. Depuis 2015, la fusion des communes a été encouragée par le Gouvernement, il y a eu 700 fusions rien que cette année 2015. \\
L'intercommunalité est la coopération entre les communes essentiellement par la création des EPCI (Établissement Public de Coopération Intercommunale), ce sont soit des syndicats de commune, soit des communautés de communes, soit des communautés d'agglomération, soit des métropoles. 


Toute collectivité locale a un organe qui délibère et un organe qui exécute. Pour la commune, le Conseil Municipal est l'organe délibératif, élu au suffrage universel pour 6 ans. Le conseil évolue selon la démographie de la commune, le minimum étant de 7 membres et le maximum de 69 membres. \\
Le mode de scrutin varie selon la taille de la commune. Dans les communes de plus de 1000 habitants, le scrutin se fait par liste avec deux tours. Elles ne doivent pas être panachés (on ne peut pas rayer des candidats), et il y a une parité imposée. Si au premier tour, une liste a obtenu un quart des inscrits et la majorité absolus, ils ont la moitié des sièges. Le deuxième tour se fait avec toute liste qui a obtenu 10\%. Les sièges se répartissent à la proportionnelle. \\
Les dernières élections ont eu lieu en 2014, les prochaines auront donc lieux en 2020. Les débats du Conseil Municipal sont publics. 


Le Maire est l'organe exécutif de la commune, il exécute les décisions délibérés lors du conseil. Il est élu à la majorité absolu par le conseil municipal. \\
Le Maire est accompagné d'adjoint, ils ont chacun leurs spécialités. \\
Le Maire est non seulement élu du conseil municipal mais est aussi agent de l'État car chaque commune doit faire vivre des services publics comme l'état civil. Il officie en tant qu'officier d'état pour les naissances, décès, mariage. \\
Le conseil municipal vote le budget et doit toujours être en équilibre. Il y a deux postes dans ce budget: les dépenses de fonctionnement et les dépenses d'investissement. \\
L'État nomme les maîtres, instituteurs, etc. mais ce sont les communes qui s'occupent des écoles. \\
Dans les recettes, il y a les impôts locaux et la dotation de l'État. \\
Le conseil municipal doit de plus en plus penser à l'intercommunalité.

\section{Le département}

Le département, comme la commune, a été créé avec la Révolution. \\
Les départements sont créés via des indications géographiques ou utile (on devait pouvoir aller au chef lieu du département à cheval dans la journée). Il y avait 89 départements, aujourd'hui 101. \\
Les départements devaient être supprimés, mais finalement, le gouvernement a reculé sur ce point. 


L'organe délibératif du département est le conseil départemental. C'est depuis 2015 que les conseils généraux ont été remplacés en conseil départementaux. Les mandats de leur conseiller sont de 6 ans. \\
Les électeurs élisent deux conseillers départementaux par canton. \\
Le nombre de conseiller par département varie selon la population. \\
Le président du conseil départemental est élu par le conseil départemental. Il a des vices présidents pour l'aider. 


C'est par le vote de son budget annuel que le Conseil Départemental use de ses compétences. \\
La loi NOTR parle de "Solidarité sociale et territoriale". Le département verse donc le RSA, les allocations d'autonomie, gère divers services sociaux. Le budget social représente donc la très grande partie du budget. \\
Il gère aussi les collège, le patrimoine départemental, la sécurité incendie et des éléments de l'aménagement du territoire (eau, tourisme, équipement en zone rural). 


\section{La région}

Contrairement à la commune et aux départements qui sont un héritage de la Révolution, la Région serai plutôt un héritage des provinces de l'ancien régime. \\
C'est après la seconde guerre mondiale que l'on a créé les régions pour lutter contre la centralisation parisienne. \\
C'est un décret pris en 1956 qui a créé la première  carte régionale dite régions économiques, avec 21 régions qui deviendront vite 22 régions (séparation de la Corse et la PACA). \\ 
De Gaulle a ensuite créé les CODER, des commissions qui sont l'ancêtre du conseil régional et qui devait aider les préfets à développer les régions. 


L'organe délibératif est le conseil régional depuis 1986. Les conseillers régionaux sont élus au suffrage universel pendant 6 ans. C'est un scrutin par liste départemental proportionnel. \\
Le Président du Conseil Régional est l'organe exécutif. \\
La Région a pour compétence le développement économique et durable. Les régions ont encore à géré les transports non urbains (TER). Les régions s'occupent de la formation professionnel, et donc de la construction et gestion des lycées. La région s'occupe des aérodromes et ports maritimes régionaux.







\end{document}
