\documentclass[12pt, a4paper, openany]{book}

\usepackage[utf8x]{inputenc}
\usepackage[T1]{fontenc}
\usepackage[francais]{babel}
\date{}
\title{Cours d'Histoire (UFR Amiens)}
\pagestyle{plain}

\begin{document}
\tableofcontents

\section{L'histoire du Parlement Français}

En France, le Parlement a une histoire qui date de plus de 200 ans. \\
Il est apparu en 1789: ce sont les États généraux convoqués par Louis XVI en Mai 1789. Ces États-généraux se constitue en Assemblée Nationale le 17 Juin 1789: à côté de la souveraineté royale, il y a la souveraineté nationale. Ils se réunissent à Versailles jusqu'en Novembre 1789 car en Octobre, les Parisiens viennent chercher Louis XVI. \\
Le Parlement siégera près des Tuileries dans une salle qui s'appelle "La salle du Manège". Jusqu'en 1795, le Parlement est monocaméral. En 1795, le Parlement devient bicaméral, avec le conseil des 500 et le conseil des Anciens. \\
Le conseil des 500 déménagera au Palais Bourbon et le conseil des Anciens ira au Palais du Luxembourg. \\
En 1799, après son Coup d'État, Bonaparte modifie les institutions: il y a désormais deux assemblées: le corps législatif et le Sénat. Sous la période Bonaparte, le Parlement est écrasé par le premier consul (qui deviendra ensuite empereur).


Les Parlements se développeront sous les monarchies censitaires entre 1814 et 1848. C'est pendant cette période que l'expérience du Parlement s'enracine dans le Parlement et voit son pouvoir croître: il y a désormais la Chambre des députés (élus au suffrage censitaire) et la Chambre des Pairs (aristocrates nommés par le Roi). \\ 
Des coutumes qui ont encore lieu aujourd'hui vont se créer comme le principe de la commission, l'oratoire (principe de l'éloquence). \\
À partir de 1830, ils auront l'initiative législative qui jusque là appartenait au Roi. De plus, on notera que le Gouvernement a une responsabilité politique devant le Parlement, c'est donc un régime parlementaire. Le parlementarisme est marqué par l'aménagement de deux grandes salles: l'hémicycle du Palais Bourbon (construit entre 1828 et 1832) mais aussi l'hémicycle du Palais du Luxembourg (édifié entre 1836 et 1842).


Avec la Révolution de 1848, le Parlement va connaître des aléas. Le Parlement deviendra monocaméralisme et sera élu au suffrage universel masculin.\\
Louis-Napoléon Bonaparte comme son Oncle n'aime pas le Parlement et va donc l'écraser mais gardera deux chambres qui auront les mêmes noms que sous le premier Empire. \\
Le Parlement est dépouillé de ses pouvoirs sous le second empire. Dans les années 1860, Napoléon redonne quelques pouvoirs au Parlement. \\
Il faut attendre l'installation définitive de 1870 de la République pour que le Parlement retrouve ses pouvoirs.


La Constitution de la IIIe va créer deux chambres (Chambre des députés et Sénat) avec un suffrage universel direct pour les députés et indirect pour les Sénateurs. \\
Pour la première fois, il y a un article dans la Constitution qui établit le régime parlementaire: "Les ministres sont responsables de la politique du Gouvernement devant les chambres". \\
En réaction au Régime de Napoléon, le PR va perdre beaucoup de pouvoir au profit du Parlement. Le régime parlementaire est institutionnalisé. \\
L'hégémonie du Parlement est créée et les Gouvernements sont instable. \\
Concernant les commissions, elles sont spécialisées et permanentes. Les groupes politiques apparaissent mais il n'existe pas de discipline partisane. \\
Le Parlement veille à ce que le PR n'ait pas trop de pouvoir et élisent des personnalités de second plan et le poussent à la démission si il devient trop populaire comme en 1924 où le Président Millerand doit démissionner. \\
Le Parlement a tellement de pouvoir et un travail tellement lent que certaines personnalités demandent une réhabilitation du pouvoir exécutif. C'est cela qui conduit à la création d'une grande vague anti-parlementaire.


Au lendemain de la seconde guerre mondiale, la Chambre des députés devient l'AN et le Sénat devient le Conseil de la République. Le principe de la IVe reste le même que la IIIe République: hégémonie du Parlement. 


\section{Le Parlement de la Ve République}

Le Sénat siège au Palais du Luxembourg. Depuis la révision constitutionnelle de 2008, le nombre de Sénateur est limité à 348, leur mandat est de 6 ans, renouvelable par moitié tous les 3 ans. Pour être Sénateur, il faut avoir 24 ans révolus et être élu par les grands électeurs (Parlementaires, Conseillers régionaux et départementaux, Conseillers municipaux élus par leur conseil), le vote est obligatoire et le scrutin se fait soit à la proportionnelle (plus de trois sénateurs dans un département) soit majoritaire à deux tours. \\
Sous la Vème, on ne peut pas être parlementaire et membre du Gouvernement. On ne peut pas non plus être Sénat et Député ou Parlementaire et membre du CC. Un Sénateur ne peut pas cumuler plus d'un mandat local avec son mandat de Sénateur. \\
Les sénateurs jouissent d'une immunité qui les protège de toutes poursuites judiciaires, sauf si le Sénat lève son immunité parlementaire ou si il est pris en flagrant délit.


L'AN siège au Palais Bourbon. \\
La tribune du Palais Bourbon est particulier car il y a un bas relief de Lemot représentant l'Histoire et la Renommée (vestige du Conseil des 500). \\
Pour être éligible, c'est ici 18 ans, l'élection se fait dans le cadre de circonscription dans un suffrage uninominal à deux tours. \\
Un député a les même incompatibilités et la même immunité. 


Au XIXe siècle, les députés étaient des grands propriétaires terriens, surtout jusqu'en 1848, dû au suffrage censitaire. Avec la IIIe République, la domination était plutôt bourgeoise: un quart des députés viennent de la petite et moyenne bourgeoisie. \\
Dans l'entre deux guerres, il a existé ce qu'on appelle la République des avocats car ils étaient surreprésentés dans les institutions. À partir de 1936, ce sont les professeurs qui avaient cette place. C'est donc plus largement les fonctionnaires qui investissent l'Assemblée au milieu du XXe siècle. \\
Si on regarde les députés par CSP, il y a une majorité de fonctionnaires, de professions libérales, de cadres et d'ingénieurs. Il y a peu ou pas d'ouvriers, très peu d'agriculteurs etc. \\
Il en est de même au Sénat. La majorité sont des retraités, sinon les groupes dominants sont les fonctionnaires, cadres et professions libérales: les mêmes qu'à l'AN.


Concernant la féminisation du Parlement, il est récent et reste très minoritaire car elles peuvent voter et se présenter seulement depuis 1944. Tout de suite, un certain nombre de femmes sont élue (entre 5 et 7\%), puis chute ensuite jusqu'à la fin des années 70 où on retrouve les chiffres de la Libération, chiffres qui se sont ensuite accrus jusqu'en 2012 où 27\% des femmes sont des députés. \\
Au Sénat, le phénomène est le même pour arriver aujourd'hui à 25\%. 


L'âge moyen des députés est de 55 ans, celui des Sénateurs est de 66 ans. \\
Il faut noter que cet âge croît, en particulier depuis le début des années 80. Avec l'arrivée des Socialistes au pouvoir, les Hommes politiques se sont renouvelés, mais ils ne se sont pas renouvelés depuis. \\
À l'AN seul 10\% des parlementaires ont moins de 40 ans. Au Sénat, les moins de 40 ans ne représentent même pas 1\%. La France peine à renouveler son personnel parlementaire: par rapport aux autres pays, le nombre de seniors parlementaires en France est bien plus élevé que dans d'autres pays. Il y a 9 fois plus de députés de plus de 60 ans que de députés de moins de 40 ans en France. 

\section{Les pouvoirs du Parlement Français}

Le Parlement vote la Loi (Art. 24 de la Constitution). La loi la plus importante chaque année est la loi de finance (souvent votée à l'automne pour l'année d'après). Ils doivent voter l'ensemble des dépenses de l'État, chapitre par chapitre, pareil pour les recettes (239 milliards d'euros de recettes en 2015 et un peu plus de 300 milliards de dépenses). \\
Deuxième loi importante: le financement de la sécurité sociale. La branche maladie est déficitaire alors que les autres sont plutôt équilibré. \\
Ces lois occupent un quart du travail des parlementaires. 


Le Parlement contrôle l'action du Gouvernement. Il existe des votes de confiance, ou bien de censure. Seuls les députés peuvent censurés le Gouvernement. \\
Le vote de confiance a, en général, lieu au début d'un Gouvernement. \\
La motion de censure n'a eu lieu qu'une fois en Octobre 1962. \\
Les parlementaires contrôlent aussi l'action du Gouvernement en créant des commissions d'enquête sur l'action du Gouvernement. 


Les parlementaires peuvent également proposer l'organisation d'un référendum. Ils autorisent aussi la déclaration de guerre ou autorise la prolongation d'interventions militaires au delà de quatre mois. Ils peuvent aussi saisir le Conseil Constitutionnel. Le Parlement peut également se constituer en haute cour et destituer le Président en cas de manquement à ses devoirs. Il peut également modifier la Constitution en congrès.


Chaque assemblée élit son président et définit son règlement intérieur. Bartolone est le Président de l'AN et Larcher est Président du Sénat. \\
Le Président de l'AN préside le congrès de Versailles alors que le Président du Sénat assure l'intérim de la fonction présidentiel. \\
Les deux chambres sont réunies en sessions ordinaires de Octobre à Juin. Pendant cette session, le Parlement ne siège pas tous les jours. Au maximum, il siège 120 jours. Le Gouvernement peut cependant rajouter des jours si nécessaire. Le PR peut convoquer le Parlement en session extraordinaire. \\
Les débats publics du Sénat et de l'AN ont le droit d'être connus de tous les citoyens (publicité des débats). Les séances sont ouvertes au public et un compte rendu intégral est rendu tous les jours des débats de la semaine précédente. De plus, les séances sont filmées. \\
Il peut arriver que les parlementaires se réunissent en comité secret. C'est arrivé entre 1914 et 1918. \\
Le vote d'un parlementaire est individuel mais peut recevoir une procuration. Les votes peuvent se faire à main levé, en assis levé, de manière électronique (scrutin ordinaire), ou encore appelé à la tribune pour mettre une enveloppe dans une urne. \\
Une fois par semaine, les parlementaires questionnent le Gouvernement. C'est une séance spécialement suivi par les parlementaires. \\
Les parlementaires se réunissent dans des groupes politiques (il faut 15 députés pour former un groupe), il y a actuellement 6 groupes parlementaires. Au Sénat, il y a également 6 groupes, mais il faut 10 Sénateurs pour former un groupe. \\
Les parlementaires siègent dans des commissions permanentes selon leur spécialité: \\
\begin{itemize}
\item Affaire culturelle et éducation 
\item Affaires économiques 
\item Affaires étrangères
\item Affaires sociales
\item Défense nationale et forces armées
\item Développement durable et aménagement du territoire
\item Finances
\item Lois 
\end{itemize}
Les Parlementaires peuvent créer des commissions extraordinaires.


\section{Comment produit-on une Loi ?}

Il faut d'abord l'initiative de la Loi: le fait de pouvoir proposer une loi. L'art 39 de la Constitution donne l'initiative au PM et aux membres du Parlement. La plupart des lois votées viennent du Gouvernement. \\
Quand une loi est proposé par le PM, on l'appelle Projet de loi. Quand il s'agit d'un Parlementaire, on parle de proposition de Loi. Les Parlementaires ne peuvent pas proposer n'importe quel type de lois. Les Parlementaires ne peuvent pas proposer de lois qui feraient baisser les recettes ou augmenter les dépenses (art. 40). \\
Le PM se fait aider du Conseil d'État pour rédiger correctement ses projets de Lois. Quand elle est rédigée, elle est adoptée en Conseil des Ministres. La Loi adopté en Conseil des Ministres est alors déposé sur le bureau d'une des deux assemblées, la plupart du temps à l'AN.


Pour être adoptée, une Loi doit être adoptée dans les mêmes termes entre l'AN et le Sénat. \\
Premièrement, le projet ou la proposition est examiné en commission. Les parlementaires amendent, donne des avis etc. \\
Une fois l'étude en commission terminée, le texte sera étudié en séance publique. Ils vont d'abord en discuter article après article et voter chaque article. À la fin, ils votent l'ensemble de la Loi (vote solennel). On notera qu'il existe un phénomène d'obstruction. Le Gouvernement peut notamment utiliser l'article 49, alinéa 3, pour faire adopter la Loi plus rapidement et notamment faire face à l'obstruction ou faire face à une majorité  incertaine. Depuis 2008, l'article 49, alinéa 3 ne peut être utilisé qu'une fois par session. \\
Une fois adopté en première lecture, le texte passe devant le Sénat, où, de la même manière, le projet est examiné en commission puis en séance publique. À la fin, le Sénat adopte le texte. Si le texte est le même que celui de l'AN, la Loi est adoptée. Mais si ce n'est pas le même, le texte va faire la navette entre chaque assemblée. \\
Si aucun texte commun, le Gouvernement peut avoir recours à la commission mixte paritaire (7 députés et 7 sénateurs qui sont nommés entre eux) pour produire un texte commun. Si un texte commun est trouvé, le projet est adopté. Sinon, l'AN statue définitivement (dans deux cas sur trois, la CMP aboutit à un accord). Plus de 70\% des Lois ont été adoptées selon des procédures normales. Dans 20\% des cas, les projets sont adoptés en CMP. Dans 10\% des cas seulement, l'AN statue définitivement. 


Une fois la Loi adopté, la Loi peut être transmise au CC si le PR, le Président du Sénat, le Président de l'AN ou 60 députés ou sénateurs le souhaitent. \\
Le PR promulgue ensuite la Loi au JORF et devient donc ensuite appliqué. Des décrets d'application peuvent compléter la Loi. 


\chapter{La décentralisation et les collectivités locales}

La décentralisation est un processus qui transfère des compétences administratives de l'État central aux collectivités locales (communes, départements, régions). 


\section{Histoire de la décentralisation}

Pendant très longtemps, jusque dans les années 80, la France fonctionnait sur un modèle de centralisation. La monarchie était très centralisé, la Révolution n'a entraîné aucune rupture dans ce domaine ni l'empire. \\
Sous l'ancien régime, il existait des autorités locales, mais c'était plutôt de la déconcentration que de la décentralisation: de plus les autorités étaient très différentes selon les régions. \\
Après la Révolution, il y a une uniformité qui se crée: le Pays est divisé en Département et en Commune, mais le contrôle de l'État est resté fort sur ces collectivités ; il s'est même renforcé avec la création par Napoléon en 1800 du Préfet. \\
Depuis le début du XIXe siècle, on a assisté à deux évolutions: les institutions des collectivités locales sont devenus de plus en plus élus et non nommé par le Gouvernement et leurs compétences se sont élargis, restant sous l'autorité du Gouvernement.


À la fin de la seconde guerre mondiale, on a pris conscience des méfaits de la centralisation. En 1947, Jean-François Gravier fait prendre conscience que la France était un pays trop centralisé. L'État a alors mis en place une politique d'aménagement du territoire, mais cela n'a pas aboutit à la décentralisation. \\
En 1969, De Gaulle propose la décentralisation au référendum qui est rejeté. \\
C'est Mitterrand qui, le 2 Mars 1982, crée un vrai processus de décentralisation. Cette loi prévoit que ce n'est plus le préfet qui administre une région mais le Conseil général. Depuis 1982, il y a trois niveaux: commune, département, région. \\
Cette loi a été complété car la décentralisation a créé un consensus, la droite a poursuivi le processus, en 1999, une loi est votée sur l'intercommunalité. 


Au début du XXIe siècle, une nouvelle étape est en cours. En 2003, la France a connu le début de l'acte 2, mené par Raffarin, qui a fait adopté une réforme de la Constitution qui inscrit dans celle-ci la décentralisation. \\
Le développement de la décentralisation a mis en évidence un certain nombre de difficultés. Il y a eu par exemple, une embauche croissante de fonctionnaire se répercutant par une hausse d'impôts locaux. De plus, toutes les régions/communes se sont mis à s'occuper de tout, n'importe comment, sans se coordonner. \\
En 2007, le Gouvernement de Sarkozy cherche à rationaliser la décentralisation. Ils voteront d'ailleurs une loi créant le conseiller territorial mais qui sera abrogé en 2012 par la nouvelle majorité. \\
Cette nouvelle majorité vote en 2013 la division du nombre de canton sans diviser le nombre d'élus. En Janvier 2014, une loi est votée pour permettre aux grandes villes de France de s'unir avec leurs banlieues pour créer des métropoles. En Janvier 2015, le nombre des régions sont réduites, il y a depuis plus que 13 Régions métropolitaines. La dernière loi est la loi NOTR d'août 2015 qui précise les compétences de chaque collectivité territoriale pour éviter les chevauchements ; les départements ont vu leurs compétences limités, les intercommunalités sont devenuss plus importante avec la région. \\

\section{La Commune}

La Commune est née sous la Révolution Française, c'est, depuis, la cellule administrative de base du pays. S'est posée la question du nombre de communes à créer. La solution qui s'est imposée est de transformer les paroisses en communes, il y a donc en France, 36 800 communes. 70\% des communes ont moins de 700 habitants et 6\% seulement ont plus de 5000 habitants. Les aires urbaines regroupent la majorité de la population (80\%). \\
Les communes n'ont jamais vraiment été remises en cause. Longtemps, le Maire était choisi par le préfet ou le Gouvernement, mais depuis le 5 Avril 1884, tous les conseillers municipaux sont élus au suffrage universel et le maire était élu par ce conseil. Paris était une exception jusqu'en 1977 avec un préfet de police à sa tête.  


Depuis 1971, il existe une loi qui permet aux communes de fusionner entre elles. Depuis 2015, la fusion des communes a été encouragée par le Gouvernement, il y a eu 700 fusions rien que cette année 2015. \\
L'intercommunalité est la coopération entre les communes essentiellement par la création des EPCI (Établissement Public de Coopération Intercommunale), ce sont soit des syndicats de commune, soit des communautés de communes, soit des communautés d'agglomération, soit des métropoles. 


Toute collectivité locale a un organe qui délibère et un organe qui exécute. Pour la commune, le Conseil Municipal est l'organe délibératif, élu au suffrage universel pour 6 ans. Le conseil évolue selon la démographie de la commune, le minimum étant de 7 membres et le maximum de 69 membres. \\
Le mode de scrutin varie selon la taille de la commune. Dans les communes de plus de 1000 habitants, le scrutin se fait par liste avec deux tours. Elles ne doivent pas être panachés (on ne peut pas rayer des candidats), et il y a une parité imposée. Si au premier tour, une liste a obtenu un quart des inscrits et la majorité absolus, ils ont la moitié des sièges. Le deuxième tour se fait avec toute liste qui a obtenu 10\%. Les sièges se répartissent à la proportionnelle. \\
Les dernières élections ont eu lieu en 2014, les prochaines auront donc lieux en 2020. Les débats du Conseil Municipal sont publics. 


Le Maire est l'organe exécutif de la commune, il exécute les décisions délibérés lors du conseil. Il est élu à la majorité absolu par le conseil municipal. \\
Le Maire est accompagné d'adjoint, ils ont chacun leurs spécialités. \\
Le Maire est non seulement élu du conseil municipal mais est aussi agent de l'État car chaque commune doit faire vivre des services publics comme l'état civil. Il officie en tant qu'officier d'état pour les naissances, décès, mariage. \\
Le conseil municipal vote le budget et doit toujours être en équilibre. Il y a deux postes dans ce budget: les dépenses de fonctionnement et les dépenses d'investissement. \\
L'État nomme les maîtres, instituteurs, etc. mais ce sont les communes qui s'occupent des écoles. \\
Dans les recettes, il y a les impôts locaux et la dotation de l'État. \\
Le conseil municipal doit de plus en plus penser à l'intercommunalité.

\section{Le département}

Le département, comme la commune, a été créé avec la Révolution. \\
Les départements sont créés via des indications géographiques ou utile (on devait pouvoir aller au chef lieu du département à cheval dans la journée). Il y avait 89 départements, aujourd'hui 101. \\
Les départements devaient être supprimés, mais finalement, le gouvernement a reculé sur ce point. 


L'organe délibératif du département est le conseil départemental. C'est depuis 2015 que les conseils généraux ont été remplacés en conseil départementaux. Les mandats de leur conseiller sont de 6 ans. \\
Les électeurs élisent deux conseillers départementaux par canton. \\
Le nombre de conseiller par département varie selon la population. \\
Le président du conseil départemental est élu par le conseil départemental. Il a des vices présidents pour l'aider. 


C'est par le vote de son budget annuel que le Conseil Départemental use de ses compétences. \\
La loi NOTR parle de "Solidarité sociale et territoriale". Le département verse donc le RSA, les allocations d'autonomie, gère divers services sociaux. Le budget social représente donc la très grande partie du budget. \\
Il gère aussi les collège, le patrimoine départemental, la sécurité incendie et des éléments de l'aménagement du territoire (eau, tourisme, équipement en zone rural). 


\section{La région}

Contrairement à la commune et aux départements qui sont un héritage de la Révolution, la Région serai plutôt un héritage des provinces de l'ancien régime. \\
C'est après la seconde guerre mondiale que l'on a créé les régions pour lutter contre la centralisation parisienne. \\
C'est un décret pris en 1956 qui a créé la première  carte régionale dite régions économiques, avec 21 régions qui deviendront vite 22 régions (séparation de la Corse et la PACA). \\ 
De Gaulle a ensuite créé les CODER, des commissions qui sont l'ancêtre du conseil régional et qui devait aider les préfets à développer les régions. 


L'organe délibératif est le conseil régional depuis 1986. Les conseillers régionaux sont élus au suffrage universel pendant 6 ans. C'est un scrutin par liste départemental proportionnel. \\
Le Président du Conseil Régional est l'organe exécutif. \\
La Région a pour compétence le développement économique et durable. Les régions ont encore à géré les transports non urbains (TER). Les régions s'occupent de la formation professionnel, et donc de la construction et gestion des lycées. La région s'occupe des aérodromes et ports maritimes régionaux.

\chapter{Les institutions de l'Union Européenne}

Le poids de la contrainte européenne est aujourd'hui une évidence. Ce n'est pas une contrainte au sens évident du terme car la France est entrée volontairement dans l'UE. \\
En 1983, les Socialistes de Mitterrand ont arrêté la politique de distribution des richesses pour garder un franc fort. En 1993, un grand débat a eu lieu du le traité de Maastricht comme en 2005 avec le projet de Constitution Européenne. 

\section{La construction Européenne: une oeuvre récente}

Les premières initiatives ont été prises au lendemain de la seconde guerre mondiale lors des années 1950. \\
Les dirigeants d'un certain nombre de pays Européens ont voulu assuré définitivement la paix Européenne notamment entre l'Angleterre, l'Allemagne et la France. \\
En 1947 commence la guerre froide et il fallait protéger l'Europe face aux soviétiques. \\
Dans ce double objectif de paix et de protection, les premiers pas sont faits. Il est à noter que l'idée d'une construction Européenne n'est pas neuve. \\
La construction se fait sur 6 pays: Benelux, Allemagne, France, Italie. La première réalisation européenne est la CECA. L'idée avait été lancé par le Ministre Français Robert Schuman le 9 Mai 1950. Le 9 Mai est donc pour cette raison, la journée de l'Europe. \\
En Avril 1951 est signée un traité. L'Europe naît à 6 dans le cadre de la CECA.


La deuxième étape va être proposé dans le domaine de la défense. Le Chancelier Allemand de l'époque Konrad Adenauer va réfléchir à cela. \\
Il est à noter que les principaux courants politiques de l'époque: socialiste et chrétiens démocrates sont les courants favorables à la construction européenne. \\
En 1954 est lancée la CED (Communauté Européenne de Défense) mais ce projet va échouer à cause des députés Français. 


À partir de 1957, il y a le temps de la CEE, qui va se construire entre 1957 et 1992. \\
Le traité de Rome est signé le 25 Mars 1957 qui instaure la CEE entre les 6 Pays fondateurs Européens. L'idée de la CEE a été proposée par Guy Mollet, et consiste à baisser les droits de douanes entre les pays membres. Le RU refuse d'y entrer.


De Gaulle essaye de réaliser une Union politique qui sera un échec. 


Dans les années 70, la CEE va connaître un élargissement. Le RU, le Danemark et l'Irlande vont entre dans la CEE. \\
En 1981, la Grèce et en 1986, l'Espagne et le Portugal rentre dans la CEE. \\
Dans les années 80, la construction Européenne a connu une nouvelle relance grâce au couple franco-allemand de Mitterrand et Helmut Kohl. Ce couple a toujours été le moteur de la construction Européenne. \\
Ce couple Mitterrand-Kohl va conduire aux accords de Schengen (signés en 1985) ; à l'acte unique Européen (signé en Février 1986) qui vise à la libre circulation des Hommes, des capitaux, et des services ; le traité de Maastricht est signé en 1992 (et le référendum a lieu en 1993 et le oui obtient 51\%) et crée l'Union Européenne, une citoyenneté européenne, une union monétaire européenne, renforce le Parlement Européen. 


En 1995, on passe à une UE à 15 puis à 25 en 2005 et désormais 28 aujourd'hui. \\
La monnaie commune est introduite en 1999 et est en vigueur depuis 2002. Il y a aujourd'hui 19 pays sur les 28 qui ont adopté la monnaie commune. \\
En 2004 est lancée l'idée d'un projet de constitution pour l'Europe. Il est rédigé par un conseil d'experts venant de tous les pays présidé par VGE. Ce projet sera refusé par les Français le 29 Mai 2005. Les Néerlandais le refuseront aussi. \\
Néanmoins, il faut adapter les textes de l'Union et en 2007 est signé le traité de Lisbonne. 


\section{La communautarisation croissante de la politique nationale depuis 1957}

La Constitution reconnaît que l'ordre juridique européen s'impose à l'ordre juridique Français. \\
En Juin 1992, cette primauté a été réaffirmé par le Conseil Constitutionnel. \\
De plus, il y a eu depuis la naissance de la CEE et de l'UE, un élargissement des compétences de l'UE. L'UE a pour compétence exclusive l'Union douanière, la concurrence, la politique commerciale, la politique monétaire, la pêche. L'Europe a tenté aussi de s'attribuer une compétence d'affaires étrangères, mais c'est un échec car l'Europe n'arrive pas à avoir une politique étrangère commune. \\
Il y a des compétences communes: politique sociale, énergie, recherche etc. \\
L'UE peut avoir des compétences qui complètent celle des États: culture, éducation etc. \\
Il y a un principe clé: le principe de subsidiarité. Principe selon lequel les responsabilités doivent être prise au niveau le plus bas possible. C'est à dire que dans les domaines qui ne relèvent pas des compétences exclusives de l'UE, l'UE n'intervient que si les États ne font pas mieux. 

\section{Le fonctionnement principal des institutions}

Le Conseil Européen est un des plus importants. Il a été créé en 1974. Ce conseil est constitué de tous les chefs d'État ou de Gouvernement de tous les pays européens. Le Conseil Européen est présidé par un Président nommé par les membres du conseil (actuellement Donald Tsk). Ce conseil décide des grandes orientations politiques de l'UE. Les décisions doivent être appliqués par la commission européenne. \\
Les sommets européens ont lieu tous les six mois ou plus si nécessaire. 


La commission européenne est une réunion de 28 commissaires (sorte de Ministre), il y en a un par État et ont tous été choisis par les Chefs d'État ou de Gouvernement. Junker est l'actuel Président de la commission. Ce Président et ces 28 commissaires ne peuvent siéger que si ils ont la confiance du Parlement Européen. \\
La commission siège à Bruxelles et siège en permanence. 


Le Parlement européen siège à Strasbourg et/ou Bruxelles. Les députés européens sont élus au suffrage universel tous les 5 ans. \\
Les élections européennes sont hélas souvent peu mobilisatrice.  \\
Le Parlement européen compte actuellement 751 députés, dont le nombre par pays est proportionnel à sa population. Le Parlement européen vote des règlements ou des directives. Il vote le budget de l'Europe et surveille l'action de la commission. \\
Comme dans l'AN, les députés se réunissent par groupe politique. Le groupe le plus nombreux est le Parti populaire européen (CDU allemand, UMP Français). La France et l'Allemagne sont les deux pays les plus représentés au Parlement. Le deuxième groupe est socialiste (Socialiste français et SPD allemand). Le troisième groupe est celui des conservateurs et réformiste (droite moins modéré mais moins extrême). Les démocrates et libéraux sont un groupe centriste. La gauche unitaire est un groupe d'extrême gauche (communistes français, Podemos espagnol). Il y a un groupe écologiste. Enfin, il y a deux groupes d'extrême droite: l'Europe de la liberté (souverainiste), et l'Europe des nation (députés du FN ou de la Ligue du Nord Italien). \\
Martin Schultz est le Président du Parlement. Simone Veil a été en 1979 la première présidente du Parlement Européen. 


Le conseil des ministres européen est la réunion des ministres des 28 pays d'Europe réunis sur une question précise. C'est un conseil spécialisé qui se réuni ponctuellement: en général tous les quinze jours mais ce ne sont jamais les mêmes. \\
Finance, Agriculture et pêche, compétitivité, affaires étrangères, affaires générales (intérieur), éducation jeunesse culture et sport, emploi et politique sociale, environnement, justice, transport et énergie sont 10 thèmes récurrents du conseil des ministres. \\
Ce Conseil est dirigé pendant 6 mois par un pays européen (le ministre de ce pays préside le conseil). \\
Pour prendre les décisions, il faut qu'il y ait soit consensus, soit majorité qualifié (3/5). 

\section{Conclusion}

L'UE est accusée aujourd'hui de déficit démocratique et d'être une institution technocratique. Ce qui relativement faux vu qu'un grand nombre d'institution sont à l'image des élections dans les États membres. C'est cependant une institution complexe. 

\chapter{Droite(s), Gauche(s) ... et Centre(s)}

Dans son dernier livre "Ce pays qui aime les idées", un auteur britannique étudie la vie politique française et dit que la spécificité de la vie politique Française, c'est de diaboliser ou sacraliser certaines choses et ne pas s'intéresser à la complexité en les réduisant. \\
La notion de droite et de gauche en est un parfait exemple. 

\section{Droite et Gauche, notions fruit d'une histoire}

L'apparition de droite et de gauche date de la révolution Française. Le vendredi 28 Aout 1789, en pleins débats pour l'écriture de la Constitution, on débat de l'idée si le Roi aura un droit de veto ou non. Se placeront à droite du bureau du Président de l'assemblée les députés favorables au veto, et à la gauche de ce bureau les députés qui soutiennent la souveraineté nationale. \\
Tout au long du 19e siècle, droite et gauche seront des notions parlementaires. On parlera du côté droit de la chambre des députés ou du côté gauche. À ce moment, les députés de droites sont ultra-royalistes. Dans les débats politiques en dehors du Parlement, on parle des rouges (héritiers de la révolution) et des blancs (royaliste). \\
À partir du début du 20e siècle, les notions de droite et de gauche vont entrer dans les notions de politique générale. Aux élections législative de 1906, les candidats ne se qualifient plus comme Républicains ou Conservateurs mais comme de Gauche ou de Droite. Les notions seront très utilisés en 1920 et en 1930. \\
Après la seconde guerre mondiale, la notion de gauche est très utilisée à contrario de la droite qui a peur de s'identifier au Régime de Vichy: ils se définiront comme conservateur, modéré, indépendants, mais surtout Gaulliste. Depuis les années 1980, le terme est revenu et revendiqué. 


Ce qui fait la différence entre la gauche et la droite a beaucoup évolué dans l'histoire. Ce sont des positions relatives, on est à droite ou à gauche de quelque chose. \\
"Appartiennent à la coalition de droite, tout ce qui s'appuie directement ou indirectement sur l'église et le château" Siegfried. \\
Jusqu'à la fin du 19e siècle, la droite sont les royaliste, la gauche les Républicains. \\
À gauche, il y a une idéalisation du peuple et une volonté d'égalité entre toutes les classes sociales, alors qu'à droite on cherche à conserver une inégalité sociale avec la crainte que la gauche détruise l'identité française. \\
À gauche, on soutiendra la souveraineté nationale, à droite, au contraire, on sera attiré par un chef capable de sauver le pays, l'homme providentiel. \\
Les symboles ne sont pas les mêmes, la gauche exalte la révolution, les lieux de manifestation populaire, des personnages comme Jean Jaurès, la commune de Paris etc. La droite va plutôt exalter les invalides, des grands hommes etc.

\section{Des divisions au sein de chaque camp}

Gauche ou droite se partage bien sûr la haine du camp d'en face. Il existe aussi une haine à l'intérieur des camps eux mêmes. \\
René Rémond, dans "Les droites en France" démontre l'existence de plusieurs courants à droite. Selon les époques, il y a toujours existé plusieurs courants de droite. \\
Il distingue trois courants: la droite légitimiste qui naît sous la restauration, c'est la droite contre révolutionnaire, qui souhaite défendre les anciennes hiérarchies sociales, cette droite est anti libéral, anti démocratique, attaché à une identité fantasmé de la France. C'est une droite qui dénonce ce que sont les ennemis du Pays, il y en a pour eux, à l'intérieur comme à l'extérieur. C'est une droite aussi xénophobe et antisémite. Depuis la fin du 19e siècle, c'est l'extrême droite. \\
La droite orléaniste: celle-ci accepte une partie de l'héritage de la révolution française, le libéralisme notamment, mais qui est conservatrice par rapport à l'ordre social. C'est une droite plus laïque que les légitimistes. Rémond existe que le lointain héritier, c'est la droite Giscardien. \\
La droite bonapartiste, une droite qui accepte l'héritage démocratique mais pas l'héritage libéraliste. C'est une droite qui aime le grand homme, patriotique. Rémond y voit comme lointain successeur le courant Gaulliste. 


Julliard, dans "Les gauches françaises" (2012), explique, comme Rémond, les différents courants. Il isole quatre courants. \\
La gauche libérale d'abord, née sous la Révolution, défend la liberté face au pouvoir absolu du Roi. À partir de 1830, la droite va s'y rallier, il existera alors une gauche et une droite libérale. Cette gauche libérale a été incarné par le courant du radicalisme, qui accepte le développement des libertés, de l'économie de marché, défend la séparation des pouvoirs, qui défend l'idée que l'État ne doit pas être trop fort et qui donc soutient la décentralisation. \\
La gauche jacobine, une gauche qui défend plus l'égalité que la liberté, qui revendique un État fort, puissant, quitte à restreindre les libertés pour faire progresser l'égalité, anti-cléricale. \\
La gauche collectiviste, qui souhaite l'abolition de la propriété privé, et créé une propriété collective. Les socialiste (entre le 19e et 20e siècle) et les communistes (à partir de 1920) ont incarnés ce courant. \\
La gauche libertaire, très minoritaire dans notre histoire, qui est une gauche qui a fait parlé d'elle qu'à certains moments. Cette gauche appelle à l'effondrement de l'État, de la République, du pouvoir en général. Ce sont des anarchistes. 


\section{Deux questions}

\subsection{Et le centre ?}

Dès la Révolution française, on a vu apparaître une gauche et une droite, on a vu se constituer entre les deux quelque chose qu'il fallait appeler le centre, qui ne voulait pas sauver l'ancien régime mais qui était effrayé par les gauches. On les a appelés les "impartiaux" ou "le Marais" (partie basse de l'hémicycle). Ces hommes du centre étaient très moqués, étaient accusés de ne pas avoir de programmes ni d'identité politique. \\
Le centre est en fait un compromis entre l'ancien régime et la nouvelle société voulu par la gauche.\\
Le centre a survécu jusqu'aujourd'hui mais il n'a jamais vraiment été indépendant de la gauche ou de la droite. Il y a deux ailes dans le centre: le centre gauche et le centre droit. \\
Au début du 20e siècle s'est constitué dans le centre la démocratie chrétienne. \\
Le centre a été très souvent au pouvoir car souvent allié à la majorité, tantôt à droite, tantôt à gauche. \\
Depuis 1962, notre vie politique est marqué par une bipolarisation. Mais même si il y a bipolarisation, la majorité est toujours ou la droite et le centre ou la gauche et le centre. En dépit de la bipolarisation, le centre existe toujours.   

\subsection{Et aujourd'hui ?}

Ce qui sépare aujourd'hui la gauche et la droite n'est pas la même chose. Des courants nés à gauche comme le libéralisme se sont déplacés à droite par exemple. \\
Globalement, les historiens sont d'accords pour dire qu'il y un mouvement de droitisation, la droite est devenu extrême, le mouvement libéral est devenu de droite, les radicaux qui étaient extrême sont aujourd'hui au centre. \\
D'où une question qui se pose: ces notions de gauche et droite ont elles encore un sens ? \\
Des thèmes traditionnellement marqués dans un camp ont, soit disparu (collectivisation), soit basculé dans l'autre camp (libéralisme), soit tous les camps s'y sont ralliés (décentralisation). Au référendum de 2005, la gauche socialiste et la droite modéré appelaient à voter oui alors que la droite, l'extrême droite et l'extrême gauche appelaient au non. \\
Concrètement, les clivages s'effacent, mais, symboliquement, ces divisions subsistent. Aujourd'hui, droite et gauche deviennent une distinction symbolique majeur pour les français. Dans le discours politique, ces symboles sont très utilisés pour raviver le clivage car la population demanderai une certaine lutte, un besoin d'avoir un ennemi plutôt que de tâcher de résoudre de manière rationnelle les problèmes politiques. La politique c'est donc aussi de l'émotion, des symboles etc.


\chapter{Des cultures politiques}

\section{Qu'est-ce qu'une culture politique ?}

Culture politique est une expression qui a émergé dans les années 80. Elle a été explicité par Serge Berstein qu'il définit comme "Un ensemble de représentation et de valeurs partagés par un groupe suffisamment large de la société et qui explique son comportement politique". Ce donc des visions du monde, des idées philosophiques, politiques, voire même religieuse. Il y a également des références historiques dont ceux partageant la culture se réclame. \\
Il y a, dans la culture politique, une idée de la société, et donc son langage associé (la notion de la lutte des classes pour les communistes bien sûr) mais aussi ses symboles (marteau et faucille, rose etc.). \\
Les cultures politiques peuvent naître, se transformer et mourir. Elles naissent toutes dans un contexte de crise en apportant des réponses à celle-ci. Ensuite, ces cultures se transforment pour s'adapter à l'évolution de la société et donc peut mourir si elle ne s'adapte pas. \\
La culture politique se transmet, par la famille, l'école etc. (phénomène de politisation). \\
La composition de tout cela contribue à donner une identité politique à chacun.

\section{Une confrontation ancienne entre culture politique}

Plusieurs cultures politiques se sont constitués et ont évolués jusqu'à aujourd'hui. \\
Ce qui est spécifique en France, c'est que, depuis le XIXe siècle, les cultures politiques s'affrontent alors que dans d'autres pays, il y a une culture politique et non pas "des". En Angleterre, il y a une culture politique anglaise et il y a un consensus sur la société, de même aux États-Unis ce qui n'est jamais arrivé en France. \\
La France n'est pas marqué par l'habitude du compromis, du consensus.

\section{Les cultures politiques}

\subsection{La culture libérale}

Les libéraux rejettent tout ce qui est conflictuel et cherchent le compromis. VGE parlait de "décrispation". \\
Elle a dominé dans la première moitié du XIXe siècle en défendant l'idée libérale de la Révolution avec comme penseur Benjamin Constant ou Alexis de Tocqueville. \\
Après 1848, la culture libérale a rallié la démocratie et la République. À partir de 1870, c'est cette culture qui est à l'origine de la IIIe République. \\
La culture libérale est très attaché au droit comme moyen de protéger les libertés. Elle accorde donc une très grande importance aux institutions et donc développer une bonne Constitution. C'est la raison pour laquelle le modèle constitutionnel des libéraux c'est un régime parlementaire bicaméral. \\
Les libéraux défendent les libertés de chaque individu et se méfient donc de l'État et défendront donc la décentralisation. 


Dans le domaine social, les libéraux vont valoriser la concertation, la discussion, la tolérance et non pas la manifestation. Ils sont plutôt réformistes et non pas révolutionnaire. \\
Ils encouragent le développement de la culture personnelle.


C'est donc une culture politique qui n'est pas idéologique mais pragmatique qui a pour modèle une société apaisé où la classe moyenne domine. Les valeurs qui permettent de faire advenir ces classes, ce sont le travail, l'effort, l'épargne. \\
La culture libérale Française ne prône pas de libéralisme économique. Elle estime au contraire que l'État a son rôle à jouer dans l'économie pour éviter que seules les puissances de l'argent aient le pouvoir économique. C'est donc un entre deux entre l'ultra-libéralisme et l'étatisme.


La culture libérale n'a jamais eu de grands mouvements ou de grands partis mais a eu des notables, notamment VGE récemment, Pinnet plus anciennement. 


\subsection{La culture socialiste}

Elle est née au milieu du XIXe siècle par des penseurs révoltés par la misère ouvrière (Saint-Simon, Fourier, Proudhon). Les socialistes ont en commun l'idée d'abolir la propriété privé, l'objectif étant d'arriver à une société égalitaire. \\
Cette culture s'est développé à la fin du XIXe sous l'influence de Karl Marx. La SFIO est né en 1905, fortement influencé par Jean Jaurès qui reste dans la mémoire des socialistes comme un grand homme. Dans les années 30, Léon Blum en est le leader, suivi ensuite de Guy Mollet.


La culture a toujours été tendu entre deux traditions contradictoires qui l'ont fait naître. La tradition révolutionnaire et la tradition Républicaine. Il y a contradiction entre les deux car lorsqu'on est Républicain, c'est l'urne qui donne le pouvoir alors que quand on est révolutionnaire, c'est le combat qui le donne. \\
Pour résoudre cette contradiction, la culture socialiste a toujours vécu en tenant un discours révolutionnaire mais en tenant des politiques réformistes. \\
Le premier lieu où les socialistes ont eu le pouvoir: les mairies, a été fortement marqué par ce mouvement réformiste notamment dans le Nord de la France. On peut citer Augustin Laurent et Pierre Mauroy à Lille, ou encore Guy Mollet à Arras. \\
Au niveau national, lorsque les socialistes ont été au pouvoir, ils n'ont jamais fait de révolutions mais ont été profondément réformistes.


Lors du congrès de Bad-Godesberg (en 1959), le SPD Allemand (les socialistes allemands) a abandonné le discours révolutionnaire. \\
En France, certains se sont dis social-démocrate notamment Michel Roccard et François Hollande. Globalement, en France, au PS, le discours révolutionnaire a été abandonné. \\
Si, en France, les socialistes ont continués à tenir ce discours, c'est parce qu'il y a, à la gauche du PS, le PCF qui tenait un discours révolutionnaire et ne souhaitait pas paraître moins révolutionnaire. 

\subsection{La culture nationaliste}

Depuis le milieu des années 1980, on a vu croître le courant de l'extrême droite. Cela marque la résurrection d'une culture forgé dans la deuxième moitié du XIXe siècle. \\
Cette culture est issu de la vieille droite contre-révolutionnaire, légitimiste qui souhaitait défendre la Monarchie et l'Église. À la fin du XIXe, cette culture était presque morte qu'un penseur du nom de Charles Maurras (vers 1900) a mélangé ce qu'il restait de cette culture aux idées nationalistes pour faire naître la culture nationaliste. \\
Alors que les Républicains étaient très profondément patriotes (de manière défensive), les nationalistes avaient eux une vision extrême qui les poussait au bellicisme, à la haine de tout ce qui n'était pas Français. \\
C'est donc un courant xénophobe et raciste. Selon Maurras, il y a trois ennemis intérieurs (Juif, Franc-maçon, protestant) et un ennemi extérieur (l'étranger). 


Le programme politique des nationalistes est entièrement tourné vers le passé en créant une société idéalisé autour du Roi, de l'Église, de l'ancien régime. \\
Leur combat politique implique une grande violences verbale mais aussi physiques (créations de ligues). L'objectif de ce courant n'est pas de faire élire des députés pour arriver au pouvoir mais de faire tomber la République par le désordre. \\
Cette culture croît qu'un chef doit prendre la tête du pays pour éviter la décadence et se placer en opposition au système soviétique. Leur modèle de régime est donc celui de Franco, de Mussolini ou de Hitler.


Elle a cru trouver son grand homme avec Pétain (c'est le cas de Maurras). \\
Après la seconde guerre mondiale, cette culture a émergé à nouveau au moment de la guerre d'Algérie et a même porté un candidat aux élections de 1965 (Jean-Louis Tixier Vignancour). Enfin, cette culture s'est effondrée mais a réapparu en 1984 pour les élections européennes avec Jean-Marie Le Pen et le FN. \\
Leur mode d'action politique est le populisme et qui propose pour renverser les élites l'arrivée au pouvoir d'un chef autoritaire. Ils utilisent aussi des slogans simplistes et de mensonges. 


\subsection{La culture politique libertaire}

C'est une culture qui a peu d'influence aujourd'hui. Elle est essentiellement composé du mouvement anarchiste. \\
Elle est née au XIXe siècle. Elle a pour caractéristique d'être radicale, souhaite rompre avec toutes formes d'États, de religions, de la société libérale et avec l'économie capitaliste. \\
"Ni Dieu ni maître" est un slogan anarchiste. Ce mouvement agit par la propagande par le fait, des attentats donc comme l'assassinat de Président, de Tsar voir de l'impératrice d'Autriche. \\
En 1900-1914, la CGT est un syndicat d'anarcho-syndicaliste. \\
Cette culture a connu une dernière mouvance en 1968 avec la contestation des institutions. 


C'est une culture politique qui reste marquée par la société du XIXe et une différenciation ouvrier/patrons alors que la société s'est transformée en une société de classes moyennes, d'où le fait qu'elle soit en état de mort.



\end{document}
