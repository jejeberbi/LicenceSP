\documentclass[12pt, a4paper, openany]{book}

\usepackage[latin1]{inputenc}
\usepackage[T1]{fontenc}
\usepackage[francais]{babel}
\date{}
\title{Cours d'Histoire (UFR Amiens)}
\pagestyle{plain}

\begin{document}
\maketitle

\section{L'histoire du Parlement Français}

En France, le Parlement a une histoire qui date de plus de 200 ans. \\
Il est apparu en 1789: ce sont les États généraux convoqués par Louis XVI en Mai 1789. Ces États-généraux se constitue en Assemblée Nationale le 17 Juin 1789: à côté de la souveraineté royale, il y a la souveraineté nationale. Ils se réunissent à Versailles jusqu'en Novembre 1789 car en Octobre, les Parisiens viennent chercher Louis XVI. \\
Le Parlement siégera près des Tuileries dans une salle qui s'appelle "La salle du Manège". Jusqu'en 1795, le Parlement est monocaméral. En 1795, le Parlement devient bicaméral, avec le conseil des 500 et le conseil des Anciens. \\
Le conseil des 500 déménagera au Palais Bourbon et le conseil des Anciens ira au Palais du Luxembourg. \\
En 1799, après son Coup d'État, Bonaparte modifie les institutions: il y a désormais deux assemblées: le corps législatif et le Sénat. Sous la période Bonaparte, le Parlement est écrasé par le premier consul (qui deviendra ensuite empereur).


Les Parlements se développeront sous les monarchies censitaires entre 1814 et 1848. C'est pendant cette période que l'expérience du Parlement s'enracine dans le Parlement et voit son pouvoir croître: il y a désormais la Chambre des députés (élus au suffrage censitaire) et la Chambre des Pairs (aristocrates nommés par le Roi). \\ 
Des coutumes qui ont encore lieu aujourd'hui vont se créer comme le principe de la commission, l'oratoire (principe de l'éloquence). \\
À partir de 1830, ils auront l'initiative législative qui jusque là appartenait au Roi. De plus, on notera que le Gouvernement a une responsabilité politique devant le Parlement, c'est donc un régime parlementaire. Le parlementarisme est marqué par l'aménagement de deux grandes salles: l'hémicycle du Palais Bourbon (construit entre 1828 et 1832) mais aussi l'hémicycle du Palais du Luxembourg (édifié entre 1836 et 1842).


Avec la Révolution de 1848, le Parlement va connaître des aléas. Le Parlement deviendra monocaméralisme et sera élu au suffrage universel masculin.\\
Louis-Napoléon Bonaparte comme son Oncle n'aime pas le Parlement et va donc l'écraser mais gardera deux chambres qui auront les mêmes noms que sous le premier Empire. \\
Le Parlement est dépouillé de ses pouvoirs sous le second empire. Dans les années 1860, Napoléon redonne quelques pouvoirs au Parlement. \\
Il faut attendre l'installation définitive de 1870 de la République pour que le Parlement retrouve ses pouvoirs.


La Constitution de la IIIe va créer deux chambres (Chambre des députés et Sénat) avec un suffrage universel direct pour les députés et indirect pour les Sénateurs. \\
Pour la première fois, il y a un article dans la Constitution qui établit le régime parlementaire: "Les ministres sont responsables de la politique du Gouvernement devant les chambres". \\
En réaction au Régime de Napoléon, le PR va perdre beaucoup de pouvoir au profit du Parlement. Le régime parlementaire est institutionnalisé. \\
L'hégémonie du Parlement est créée et les Gouvernements sont instable. \\
Concernant les commissions, elles sont spécialisées et permanentes. Les groupes politiques apparaissent mais il n'existe pas de discipline partisane. \\
Le Parlement veille à ce que le PR n'ait pas trop de pouvoir et élisent des personnalités de second plan et le poussent à la démission si il devient trop populaire comme en 1924 où le Président Millerand doit démissionner. \\
Le Parlement a tellement de pouvoir et un travail tellement lent que certaines personnalités demandent une réhabilitation du pouvoir exécutif. C'est cela qui conduit à la création d'une grande vague anti-parlementaire.


Au lendemain de la seconde guerre mondiale, la Chambre des députés devient l'AN et le Sénat devient le Conseil de la République. Le principe de la IVe reste le même que la IIIe République: hégémonie du Parlement. 


\section{Le Parlement de la Ve République}

Le Sénat siège au Palais du Luxembourg. Depuis la révision constitutionnelle de 2008, le nombre de Sénateur est limité à 348, leur mandat est de 6 ans, renouvelable par moitié tous les 3 ans. Pour être Sénateur, il faut avoir 24 ans révolus et être élu par les grands électeurs (Parlementaires, Conseillers régionaux et départementaux, Conseillers municipaux élus par leur conseil), le vote est obligatoire et le scrutin se fait soit à la proportionnelle (plus de trois sénateurs dans un département) soit majoritaire à deux tours. \\
Sous la Vème, on ne peut pas être parlementaire et membre du Gouvernement. On ne peut pas non plus être Sénat et Député ou Parlementaire et membre du CC. Un Sénateur ne peut pas cumuler plus d'un mandat local avec son mandat de Sénateur. \\
Les sénateurs jouissent d'une immunité qui les protège de toutes poursuites judiciaires, sauf si le Sénat lève son immunité parlementaire ou si il est pris en flagrant délit.


L'AN siège au Palais Bourbon. \\
La tribune du Palais Bourbon est particulier car il y a un bas relief de Lemot représentant l'Histoire et la Renommée (vestige du Conseil des 500). \\
Pour être éligible, c'est ici 18 ans, l'élection se fait dans le cadre de circonscription dans un suffrage uninominal à deux tours. \\
Un député a les même incompatibilités et la même immunité. 


Au XIXe siècle, les députés étaient des grands propriétaires terriens, surtout jusqu'en 1848, dû au suffrage censitaire. Avec la IIIe République, la domination était plutôt bourgeoise: un quart des députés viennent de la petite et moyenne bourgeoisie. \\
Dans l'entre deux guerres, il a existé ce qu'on appelle la République des avocats car ils étaient surreprésentés dans les institutions. À partir de 1936, ce sont les professeurs qui avaient cette place. C'est donc plus largement les fonctionnaires qui investissent l'Assemblée au milieu du XXe siècle. \\
Si on regarde les députés par CSP, il y a une majorité de fonctionnaires, de professions libérales, de cadres et d'ingénieurs. Il y a peu ou pas d'ouvriers, très peu d'agriculteurs etc. \\
Il en est de même au Sénat. La majorité sont des retraités, sinon les groupes dominants sont les fonctionnaires, cadres et professions libérales: les mêmes qu'à l'AN.


Concernant la féminisation du Parlement, il est récent et reste très minoritaire car elles peuvent voter et se présenter seulement depuis 1944. Tout de suite, un certain nombre de femmes sont élue (entre 5 et 7\%), puis chute ensuite jusqu'à la fin des années 70 où on retrouve les chiffres de la Libération, chiffres qui se sont ensuite accrus jusqu'en 2012 où 27\% des femmes sont des députés. \\
Au Sénat, le phénomène est le même pour arriver aujourd'hui à 25\%. 


L'âge moyen des députés est de 55 ans, celui des Sénateurs est de 66 ans. \\
Il faut noter que cet âge croît, en particulier depuis le début des années 80. Avec l'arrivée des Socialistes au pouvoir, les Hommes politiques se sont renouvelés, mais ils ne se sont pas renouvelés depuis. \\
À l'AN seul 10\% des parlementaires ont moins de 40 ans. Au Sénat, les moins de 40 ans ne représentent même pas 1\%. La France peine à renouveler son personnel parlementaire: par rapport aux autres pays, le nombre de seniors parlementaires en France est bien plus élevé que dans d'autres pays. Il y a 9 fois plus de députés de plus de 60 ans que de députés de moins de 40 ans en France. 









\end{document}
