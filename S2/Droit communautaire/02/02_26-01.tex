\documentclass[12pt, a4paper, openany]{book}

\usepackage[utf8x]{inputenc}
\usepackage[T1]{fontenc}
\usepackage[francais]{babel}
\date{\today}
\title{Cours de droit communautaire (UFR Amiens)}
\pagestyle{plain}

\begin{document}

La CECA est le laboratoire des idées de la construction européenne. Quatre organes seront gardés: haute autorité (commission), conseil des ministres, assemblée, cour de justice. \\
La CEE a pour ambition d'élaborer une législation économique d'ensemble. Le traité de Rome a donc pour ambition de créer une économie européenne. La CEE marque ainsi la diminution des pouvoirs de la haute autorité au profit de la commission que l'on connaît aujourd'hui. Cependant, certains pouvoirs de la haute autorité ont été donnés au Conseil qui possède désormais une majorité de pouvoir décisionnel. Le Conseil prend d'abord ses décisions à l'unanimité puis à la majorité qualifié. La commission aura pour principale tache d'exécuter les décisions du conseil. \\
Cependant, la commission possède toujours le monopole de l'initiative. C'est la commission qui propose les textes législatifs: elle propose les textes de sa propre initiative, ou à la demande du conseil, ou maintenant, à la demande du Parlement. \\
L'assemblée qui deviendra le Parlement Européen donnera la caution démocratique au système mais dispose à l'origine très peu de pouvoir: à l'origine son pouvoir n'est que consultatif. L'histoire de la construction européenne est donc l'histoire de la montée en puissance du Parlement européen. Il est en principe aujourd'hui co-législateur avec le Conseil.

\subsubsection{Le fonctionnalisme}

Le fonctionnalisme consiste à mener à bien des objectifs concrets ayant la vertu d'entraîner la poursuite d'autres objectifs qui lui sont liés. \\
Si l'on revient sur la déclaration Schuman, il est écrit "L'Europe ne se fera pas d'un coup, ni dans une construction d'ensemble". Il faut comprendre qu'on ne commence pas par le haut en créant un État fédéral Européen. La déclaration parle de réalisations concrètes entraînant des solidarités de faits. \\
Par exemple, le fait de prévoir la libre circulation des biens entraîne le besoin de se mettre d'accord sur les règles de sécurité. Le sigle CE désigne toute la politique de normalisation conduite au niveau Européen. \\
La politique de protection du consommateur a aussi été harmonisé. Au sein de l'UE existe un partage de compétence, entre l'État et l'UE.


Le pari du fonctionnalisme est, qu'à force de mettre des compétences en commun, les États ressentiraient la nécessité progressive de rajouter des compétences à exercer en commun. \\
Il faut souligner l'importance du projet avorté de la CED (Communauté Européenne de Défense) en 1954 qui consistait à résoudre le problème du réarmement Allemand face à l'URSS. L'idée était donc d'incorporer la futur armée allemande à une armée Européenne. Cependant, la France n'a pas ratifié le traité et le projet à été abandonné. \\
Avec cet échec, on ne peut pas imaginer plus régalien que la défense et la politique étrangère donc nécessairement un véritable État européen. \\
L'échec de la CED a eu deux conséquences: dès lors qu'a été marqué le refus définitif de créer une construction Européenne. George Bidault, Président du Conseil sous la IVe République disait: "Il s'agira de faire l'Europe, sans défaire la France". L'UE a donc une forme d'État fédéral inversé: la souveraineté vient des membres et non de l'entité centrale. \\
La deuxième conséquence est que cela a confirmé la pertinence de l'approche fonctionnaliste. Le projet Européen se conçoit alors comme un dépassement progressif de la souveraineté de l'État sans remettre en cause son essence, c'est à dire des transferts successifs de compétences avec la problématique du point de rupture. \\
En tout état de cause, les États demeurent souverain car ils peuvent partir comme le montre l'organisation d'un référendum sur la sortie de l'Union au Royaume-Uni. 


La mission originelle de la CEE est l'intégration des marchés nationaux. Dans le traité de Rome, il y a une expression avancée du concept d'intégration régionale économique. Dans le processus de création d'un espace économique régional, on distingue plusieurs étapes:
\begin{itemize}
\item La zone de libre échange ;
\item L'union douanière ;
\item Le marché commun ;
\item L'union économique et monétaire ;
\item L'union politique.
\end{itemize}

Dans le traité de Rome, on pose une union douanière à l'intérieur de laquelle se pose le marché commun. L'union douanière se distingue de la zone de libre échange, en ce que cette dernière limite le processus de désarmement douanier aux seuls échanges entre les États partis. Chacun des États partis gardant leur autonomie vis à vis des États tiers. \\
L'AELE (Association Européenne de Libre Échange) comprend la Suisse, l'Islande, la Norvège, le Liechtenstein. La position de la Suisse est très particulière car elle a adopté plus de 100 accords bilatéraux avec l'UE pour assurer une zone économique intégrée. Les trois autres pays se sont regroupés au sein de l'Espace Européen Économique (EEE) qui a la particularité d'avoir adopté les règles du marché commun de l'UE. 


On considère que la zone de libre échange est moins efficace que l'union douanière car si on laisse la compétence aux États de fixer leur tarif douanier extérieur, cela a pour conséquence que toutes les marchandises entrent dans la ZLE par le pays le moins cher. Une fois entré dans la zone, la marchandise circule librement. \\
En revanche, dans le cadre d'une union douanière, les partenaires conviennent de constituer une seule entité commerciale par rapport au reste du monde, ce qui se traduit par la mise en place d'un tarif extérieur commun. La théorie économique enseigne qu'une union douanière entraîne une augmentation du trafic entre les membres et une diminution corrélative avec les pays tiers. Les pays membres échangent entre eux deux tiers de leurs marchandises. \\
Le tarif douanier commun a été définitivement mis en place en 1968. Il a fallu 11 ans aux pays membres pour mettre en place le tarif douanier commun. Ce domaine est désormais une compétence exclusive de l'UE. C'est la raison pour laquelle il revient à l'Union de négocier les tarifs douaniers avec l'OMC. La politique commerciale extérieure est donc la première politique commune. \\
Cela explique pourquoi c'est la commission qui négocie le TTIP. Il se pose cependant deux problèmes: la normalisation des produits différentes, l'agriculture à base d'OGM etc. et la justice où serait mise en place des tribunaux spéciaux d'arbitrage qui peuvent conduire à des conflits d'intérêt. 


Le traité de Rome crée également le marché commun. Aujourd'hui, ce marché commun est appelé marché intérieur. C'est la grande ambition de la CEE en 1957. Il vise à fondre les économies nationales dans une économie européenne, fusionner les marchés donc, comme si c'en était un seul. \\
Le marché commun ne s'assure pas seulement de la libre circulation des biens, il promeut aussi la libre circulation des facteurs de production: le travail et le capital. Le facteur travail vise la libre circulation des personnes, qui a donc originellement une seule raison économique. L'espace Schengen est la libre circulation des personnes sans distinction de la personne comme agent économique ; tout le monde peut donc circuler dans cet espace. \\
La libre circulation économique est le droit de circuler dans un autre État membre, dans les mêmes conditions que les nationaux dans le but de travailler: on parle de libre circulation des travailleurs. Les indépendants ont aussi le droit de s'établir dans d'autres pays membres: liberté d'établissement. \\
L'exercice des activités professionnels est donc facilité. 


L'idée de réaliser un marché commun en Europe est un mélange entre pragmatisme et utopie. Les producteurs peuvent importer et exporter librement. Les entreprises peuvent s'établir et exercer des activités transnationales. Pour les investisseurs, il peuvent investir où ils le souhaitent. Pour les consommateurs, c'est la possibilité d'accéder aux produits en provenance des autres États membres dans les mêmes conditions que les produits nationaux. \\
À l'époque, le marché commun repose sur un dessin très ambitieux. Le but principal est la pacification du continent et de présager une autre réalité: l'unité sociale et politique de l'Europe. Dans l'esprit des créateurs du marché commun, celui-ci précède forcément l'union politique de l'Europe, le meilleur gage de la préservation de la paix. \\
Le rapprochement entre les États est une réussite: il n'y a pas eu de conflits armés depuis la seconde guerre mondiale (d'où l'attribution du prix Nobel de la paix à l'Union en 2012). Il y a eu un conflit: l'ex-Yougoslavie dont les nouveaux pays sont soit membres soit candidats pour être membre de l'UE. 


L'union économique et monétaire a été réalisée par le traité de Maastricht en 1992. À cet égard, on peut déjà observé d'une part que tous les États de l'Union ne participent pas à l'Euro. Seulement 19 États membres sont à l'Euro. \\
Au niveau de l'union économique, il s'agit essentiellement de contraintes concernant la discipline budgétaire. Il n'existe pas aujourd'hui, au sein de l'union, une harmonisation de la fiscalité. \\
La seule harmonisation a été sur la TVA en mettant en place des fourchettes. \\
Le droit du travail est toujours une compétence des États.\\
En 2016, l'UE est donc une union économique et monétaire imparfaite. 


\part{L'identité de l'Union Européenne}

On s'accorde sur des éléments identitaires de l'Union en ce qui concerne les personnes qui la constituent, les valeurs sur laquelle l'Union est fondée et les buts que l'Union poursuit sous forme de différenciations intégratives.

\chapter{Les personnes qui constituent l'UE}

L'appartenance à l'Union pour un État est conditionné. Ces conditions sont importantes car nous renseignes ce qui est important aux yeux de l'Union. S'intéresser aux personnes, c'est également envisager l'Union dans sa relation avec les personnes physiques: les citoyens de l'UE. En vertu de l'article 20, §1 du traité sur le fonctionnement de l'UE (TFUE), la citoyenneté de l'Union est attribuée à toutes personnes ayant la nationalité d'un État membre. Celle-ci n'a pas pour seul objet d'apporter des droits et des libertés à ses destinataires, elle a une ambition politique: identifier les individus membres du corps politique qu'entend formé l'Union. \\
La communauté Européenne participe donc à un objectif commun qui est l'affirmation d'une identité européenne, aussi bien vis à vis des États membres que vis à vis de l'extérieur. \\
Il y a donc une idée de double appartenance: actuellement, la nation, dans le futur, au sein de l'Union.


\section{L'appartenance de l'État à l'UE}

\subsection{Devenir membre de l'UE}

La CEE a toujours été ouverte aux autres pays d'Europe que les pays fondateurs. Depuis, l'UE a connu des élargissements successifs. On distingue plusieurs vagues d'adhésion. Une première vague lorsque la Grande Bretagne a compris que le marché commun était viable avec l'Irlande, le Danemark et la Norvège qui adhérèrent en 1973 (sauf la Norvège en raison d'un mauvais référendum). La Grèce a ensuite adhéré en 1981, l'Espagne et le Portugal en 1986. Ensuite, la vague du nord: 1995: Autriche, Finlande, Suède avec encore un référendum négatif en Norvège. Vague de l'est en 2004 et 2007: Pologne, République Tchèque, Slovaquie, Hongrie, Slovénie, Chypre, Malte, Letonie, Lituanie, Estonie ; Bulgarie, Roumanie. \\
La Croatie intègre l'UE en Juillet 2013. \\
Il y a aujourd'hui 28 États membres de l'UE. 


Cet élargissement de l'UE montre un certain succès de l'UE. Des négociations sont actuellement en cours avec les pays des Balkans et la Turquie. L'UE pourrait donc passer à plus de 30 membres. \\
La question de la dilution de l'Union se pose: un grand nombre d'États pourrait paralyser l'Union. \\
Il y a une différence de vision entre l'Europe espace (vision pacifique et prospère prôné par le Royaume-Uni) et l'Europe puissance (c'est à dire l'Europe politique).  

\subsubsection{La procédure d'adhésion}

La procédure d'adhésion est prévue à l'article 49 du TUE (Traité sur l'Union Européenne), qui distingue deux temps: un premier temps institutionnel, puis un second temps interétatique qui concerne l'accord d'adhésion. \\
"L'État demandeur adresse sa demande au Conseil, lequel se prononce à l'unanimité après avoir consulté la commission et après approbation du Parlement". \\
Il y a un accord d'adhésion entre l'État demandeur et les États membres. Cet accord est soumis à la ratification par tous les États membres. Ratification selon leurs règles constitutionnelles. \\
La pratique va générer quelques ajustements: le Parlement doit donner son approbation à la candidature, or, celui-ci souhaite savoir ce que contient l'accord avant de voter. \\
La commission va intervenir en amont pour discuter avec l'État candidat de la possibilité d'adhérer. Puis, après l'accord du Conseil, la commission va intervenir pour négocier les conditions de l'accord d'adhésion. 


Trois phases:
\begin{itemize}
\item Candidature puis négociations ;
\item Avis de la commission à la fin des négociations ;
\item Vote du conseil à l'unanimité avec avis conforme du Parlement.
\end{itemize}


Concernant le droit français, il a été inscrit à l'article 88-5 qu'une nouvelle adhésion requerrait un référendum ou un vote aux trois cinquième de chaque assemblée. \\
En Irlande, tous les traités Européens ne peuvent être adoptés que par référendum.  

\subsubsection{Les conditions de fonds}

Si on regarde l'article 49 relatif à l'adhésion, il est écrit que "tout État européen qui respecte les valeurs visés à l'article 2 et s'engage à les promouvoir peut adhérer". \\
Il faut donc: être un État européen, respecter les valeurs de l'article 2 (démocratie, État de droit, respect des droits fondamentaux) et les promouvoir. \\
Ce n'est qu'avec la candidature de la Turquie et éventuellement de l'Ukraine que le critère d'identification géographique est apparu. Mais ce critère géographique est insuffisant: aucune autorité ne peut définir où s'arrête le continent Européen. Le critère géographique est sur déterminé par le critère politique comme en Ukraine où ne sera établi qu'un accord d'association. \\
On ne sait pas ce qu'est être européen donc au final, la seule condition est sur les valeurs, d'après les textes.


En réalité, les conditions sont plus précises et se sont décalés au fur et à mesure des élargissements. La première vague a fait apparaître la condition de la reprise totale du droit de l'Union (la condition de l'acquis communautaire). La vague du Sud montrera qu'il est nécessaire de remplir la condition de démocratie. La vague de l'est a permis de systématiser les critères de Copenhague. Le quatrième critère est la capacité d'absorption de l'Union (l'intégration ne doit pas mettre en péril l'intégration politique de l'Union).


i. Dès la première vague est mis sur la table la condition de la reprise de l'acquis communautaire: les dispositions des traités et tous les actes pris par les institutions formant le droit de l'Union. Ils doivent prendre la totalité de ce que l'Union a déjà fait. Être candidat s'inscrit dans le temps car la commission doit s'assurer que le droit de l'État candidat reprend bien l'acquis communautaire et nécessite souvent une grande transformation du droit de l'État candidat. \\
Les modalités sont des mesures de transitions: pendant un certain temps, l'État peut avoir son droit non conforme avec l'acquis communautaire. 


ii.



iii.



iv. 


\subsection{Se retirer de l'UE}



\section{La citoyenneté}


\chapter{Les valeurs de l'UE}








\chapter{Les buts de l'UE}












\end{document}
