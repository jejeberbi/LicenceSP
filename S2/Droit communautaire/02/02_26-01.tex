\documentclass[12pt, a4paper, openany]{book}

\usepackage[utf8x]{inputenc}
\usepackage[T1]{fontenc}
\usepackage[francais]{babel}
\date{\today}
\title{Cours de droit communautaire (UFR Amiens)}
\pagestyle{plain}

\begin{document}

La CECA est le laboratoire des idées de la construction européenne. Quatre organes seront gardés: haute autorité (commission), conseil des ministres, assemblée, cour de justice. \\
La CEE a pour ambition d'élaborer une législation économique d'ensemble. Le traité de Rome a donc pour ambition de créer une économie européenne. La CEE marque ainsi la diminution des pouvoirs de la haute autorité au profit de la commission que l'on connaît aujourd'hui. Cependant, certains pouvoirs de la haute autorité ont été donnés au Conseil qui possède désormais une majorité de pouvoir décisionnel. Le Conseil prend d'abord ses décisions à l'unanimité puis à la majorité qualifié. La commission aura pour principale tache d'exécuter les décisions du conseil. \\
Cependant, la commission possède toujours le monopole de l'initiative. C'est la commission qui propose les textes législatifs: elle propose les textes de sa propre initiative, ou à la demande du conseil, ou maintenant, à la demande du Parlement. \\
L'assemblée qui deviendra le Parlement Européen donnera la caution démocratique au système mais dispose à l'origine très peu de pouvoir: à l'origine son pouvoir n'est que consultatif. L'histoire de la construction européenne est donc l'histoire de la montée en puissance du Parlement européen. Il est en principe aujourd'hui co-législateur avec le Conseil.

\subsubsection{Le fonctionnalisme}

Le fonctionnalisme consiste à mener à bien des objectifs concrets ayant la vertu d'entraîner la poursuite d'autres objectifs qui lui sont liés. \\
Si l'on revient sur la déclaration Schuman, il est écrit "L'Europe ne se fera pas d'un coup, ni dans une construction d'ensemble". Il faut comprendre qu'on ne commence pas par le haut en créant un État fédéral Européen. La déclaration parle de réalisations concrètes entraînant des solidarités de faits. \\
Par exemple, le fait de prévoir la libre circulation des biens entraîne le besoin de se mettre d'accord sur les règles de sécurité. Le sigle CE désigne toute la politique de normalisation conduite au niveau Européen. \\
La politique de protection du consommateur a aussi été harmonisé. Au sein de l'UE existe un partage de compétence, entre l'État et l'UE.


Le pari du fonctionnalisme est, qu'à force de mettre des compétences en commun, les États ressentiraient la nécessité progressive de rajouter des compétences à exercer en commun. \\
Il faut souligner l'importance du projet avorté de la CED (Communauté Européenne de Défense) en 1954 qui consistait à résoudre le problème du réarmement Allemand face à l'URSS. L'idée était donc d'incorporer la futur armée allemande à une armée Européenne. Cependant, la France n'a pas ratifié le traité et le projet à été abandonné. \\
Avec cet échec, on ne peut pas imaginer plus régalien que la défense et la politique étrangère donc nécessairement un véritable État européen. \\
L'échec de la CED a eu deux conséquences: dès lors qu'a été marqué le refus définitif de créer une construction Européenne. George Bidault, Président du Conseil sous la IVe République disait: "Il s'agira de faire l'Europe, sans défaire la France". L'UE a donc une forme d'État fédéral inversé: la souveraineté vient des membres et non de l'entité centrale. \\
La deuxième conséquence est que cela a confirmé la pertinence de l'approche fonctionnaliste. Le projet Européen se conçoit alors comme un dépassement progressif de la souveraineté de l'État sans remettre en cause son essence, c'est à dire des transferts successifs de compétences avec la problématique du point de rupture. \\
En tout état de cause, les États demeurent souverain car ils peuvent partir comme le montre l'organisation d'un référendum sur la sortie de l'Union au Royaume-Uni. 


La mission originelle de la CEE est l'intégration des marchés nationaux. Dans le traité de Rome, il y a une expression avancée du concept d'intégration régionale économique. Dans le processus de création d'un espace économique régional, on distingue plusieurs étapes:
\begin{itemize}
\item La zone de libre échange ;
\item L'union douanière ;
\item Le marché commun ;
\item L'union économique et monétaire ;
\item L'union politique.
\end{itemize}

Dans le traité de Rome, on pose une union douanière à l'intérieur de laquelle se pose le marché commun. L'union douanière se distingue de la zone de libre échange, en ce que cette dernière limite le processus de désarmement douanier aux seuls échanges entre les États partis. Chacun des États partis gardant leur autonomie vis à vis des États tiers. \\
L'AELE (Association Européenne de Libre Échange) comprend la Suisse, l'Islande, la Norvège, le Liechtenstein. La position de la Suisse est très particulière car elle a adopté plus de 100 accords bilatéraux avec l'UE pour assurer une zone économique intégrée. Les trois autres pays se sont regroupés au sein de l'Espace Européen Économique (EEE) qui a la particularité d'avoir adopté les règles du marché commun de l'UE. 


On considère que la zone de libre échange est moins efficace que l'union douanière car si on laisse la compétence aux États de fixer leur tarif douanier extérieur, cela a pour conséquence que toutes les marchandises entrent dans la ZLE par le pays le moins cher. Une fois entré dans la zone, la marchandise circule librement. \\
En revanche, dans le cadre d'une union douanière, les partenaires conviennent de constituer une seule entité commerciale par rapport au reste du monde, ce qui se traduit par la mise en place d'un tarif extérieur commun. La théorie économique enseigne qu'une union douanière entraîne une augmentation du trafic entre les membres et une diminution corrélative avec les pays tiers. Les pays membres échangent entre eux deux tiers de leurs marchandises. \\
Le tarif douanier commun a été définitivement mis en place en 1968. Il a fallu 11 ans aux pays membres pour mettre en place le tarif douanier commun. Ce domaine est désormais une compétence exclusive de l'UE. C'est la raison pour laquelle il revient à l'Union de négocier les tarifs douaniers avec l'OMC. La politique commerciale extérieure est donc la première politique commune. \\
Cela explique pourquoi c'est la commission qui négocie le TTIP. Il se pose cependant deux problèmes: la normalisation des produits différentes, l'agriculture à base d'OGM etc. et la justice où serait mise en place des tribunaux spéciaux d'arbitrage qui peuvent conduire à des conflits d'intérêt. 


Le traité de Rome crée également le marché commun. Aujourd'hui, ce marché commun est appelé marché intérieur. C'est la grande ambition de la CEE en 1957. Il vise à fondre les économies nationales dans une économie européenne, fusionner les marchés donc, comme si c'en était un seul. \\
Le marché commun ne s'assure pas seulement de la libre circulation des biens, il promeut aussi la libre circulation des facteurs de production: le travail et le capital. Le facteur travail vise la libre circulation des personnes, qui a donc originellement une seule raison économique. L'espace Schengen est la libre circulation des personnes sans distinction de la personne comme agent économique ; tout le monde peut donc circuler dans cet espace. \\
La libre circulation économique est le droit de circuler dans un autre État membre, dans les mêmes conditions que les nationaux dans le but de travailler: on parle de libre circulation des travailleurs. Les indépendants ont aussi le droit de s'établir dans d'autres pays membres: liberté d'établissement. \\
L'exercice des activités professionnels est donc facilité. 


L'idée de réaliser un marché commun en Europe est un mélange entre pragmatisme et utopie. Les producteurs peuvent importer et exporter librement. Les entreprises peuvent s'établir et exercer des activités transnationales. Pour les investisseurs, il peuvent investir où ils le souhaitent. Pour les consommateurs, c'est la possibilité d'accéder aux produits en provenance des autres États membres dans les mêmes conditions que les produits nationaux. \\
À l'époque, le marché commun repose sur un dessin très ambitieux. Le but principal est la pacification du continent et de présager une autre réalité: l'unité sociale et politique de l'Europe. Dans l'esprit des créateurs du marché commun, celui-ci précède forcément l'union politique de l'Europe, le meilleur gage de la préservation de la paix. \\
Le rapprochement entre les États est une réussite: il n'y a pas eu de conflits armés depuis la seconde guerre mondiale (d'où l'attribution du prix Nobel de la paix à l'Union en 2012). Il y a eu un conflit: l'ex-Yougoslavie dont les nouveaux pays sont soit membres soit candidats pour être membre de l'UE. 


L'union économique et monétaire a été réalisée par le traité de Maastricht en 1992. À cet égard, on peut déjà observé d'une part que tous les États de l'Union ne participent pas à l'Euro. Seulement 19 États membres sont à l'Euro. \\
Au niveau de l'union économique, il s'agit essentiellement de contraintes concernant la discipline budgétaire. Il n'existe pas aujourd'hui, au sein de l'union, une harmonisation de la fiscalité. \\
La seule harmonisation a été sur la TVA en mettant en place des fourchettes. \\
Le droit du travail est toujours une compétence des États.\\
En 2016, l'UE est donc une union économique et monétaire imparfaite. 


\part{L'identité de l'Union Européenne}

On s'accorde sur des éléments identitaires de l'Union en ce qui concerne les personnes qui la constituent, les valeurs sur laquelle l'Union est fondée et les buts que l'Union poursuit sous forme de différenciations intégratives.

\chapter{Les personnes qui constituent l'UE}

L'appartenance à l'Union pour un État est conditionné. Ces conditions sont importantes car nous renseignes ce qui est important aux yeux de l'Union. S'intéresser aux personnes, c'est également envisager l'Union dans sa relation avec les personnes physiques: les citoyens de l'UE. En vertu de l'article 20, §1 du traité sur le fonctionnement de l'UE (TFUE), la citoyenneté de l'Union est attribuée à toutes personnes ayant la nationalité d'un État membre. Celle-ci n'a pas pour seul objet d'apporter des droits et des libertés à ses destinataires, elle a une ambition politique: identifier les individus membres du corps politique qu'entend formé l'Union. \\
La communauté Européenne participe donc à un objectif commun qui est l'affirmation d'une identité européenne, aussi bien vis à vis des États membres que vis à vis de l'extérieur. \\
Il y a donc une idée de double appartenance: actuellement, la nation, dans le futur, au sein de l'Union.


\section{L'appartenance de l'État à l'UE}

\subsection{Devenir membre de l'UE}

La CEE a toujours été ouverte aux autres pays d'Europe que les pays fondateurs. Depuis, l'UE a connu des élargissements successifs. On distingue plusieurs vagues d'adhésion. Une première vague lorsque la Grande Bretagne a compris que le marché commun était viable avec l'Irlande, le Danemark et la Norvège qui adhérèrent en 1973 (sauf la Norvège en raison d'un mauvais référendum). La Grèce a ensuite adhéré en 1981, l'Espagne et le Portugal en 1986. Ensuite, la vague du nord: 1995: Autriche, Finlande, Suède avec encore un référendum négatif en Norvège. Vague de l'est en 2004 et 2007: Pologne, République Tchèque, Slovaquie, Hongrie, Slovénie, Chypre, Malte, Letonie, Lituanie, Estonie ; Bulgarie, Roumanie. \\
La Croatie intègre l'UE en Juillet 2013. \\
Il y a aujourd'hui 28 États membres de l'UE. 


Cet élargissement de l'UE montre un certain succès de l'UE. Des négociations sont actuellement en cours avec les pays des Balkans et la Turquie. L'UE pourrait donc passer à plus de 30 membres. \\
La question de la dilution de l'Union se pose: un grand nombre d'États pourrait paralyser l'Union. \\
Il y a une différence de vision entre l'Europe espace (vision pacifique et prospère prôné par le Royaume-Uni) et l'Europe puissance (c'est à dire l'Europe politique).  

\subsubsection{La procédure d'adhésion}

La procédure d'adhésion est prévue à l'article 49 du TUE (Traité sur l'Union Européenne), qui distingue deux temps: un premier temps institutionnel, puis un second temps interétatique qui concerne l'accord d'adhésion. \\
"L'État demandeur adresse sa demande au Conseil, lequel se prononce à l'unanimité après avoir consulté la commission et après approbation du Parlement". \\
Il y a un accord d'adhésion entre l'État demandeur et les États membres. Cet accord est soumis à la ratification par tous les États membres. Ratification selon leurs règles constitutionnelles. \\
La pratique va générer quelques ajustements: le Parlement doit donner son approbation à la candidature, or, celui-ci souhaite savoir ce que contient l'accord avant de voter. \\
La commission va intervenir en amont pour discuter avec l'État candidat de la possibilité d'adhérer. Puis, après l'accord du Conseil, la commission va intervenir pour négocier les conditions de l'accord d'adhésion. 


Trois phases:
\begin{itemize}
\item Candidature puis négociations ;
\item Avis de la commission à la fin des négociations ;
\item Vote du conseil à l'unanimité avec avis conforme du Parlement.
\end{itemize}


Concernant le droit français, il a été inscrit à l'article 88-5 qu'une nouvelle adhésion requerrait un référendum ou un vote aux trois cinquième de chaque assemblée. \\
En Irlande, tous les traités Européens ne peuvent être adoptés que par référendum.  

\subsubsection{Les conditions de fonds}

Si on regarde l'article 49 relatif à l'adhésion, il est écrit que "tout État européen qui respecte les valeurs visés à l'article 2 et s'engage à les promouvoir peut adhérer". \\
Il faut donc: être un État européen, respecter les valeurs de l'article 2 (démocratie, État de droit, respect des droits fondamentaux) et les promouvoir. \\
Ce n'est qu'avec la candidature de la Turquie et éventuellement de l'Ukraine que le critère d'identification géographique est apparu. Mais ce critère géographique est insuffisant: aucune autorité ne peut définir où s'arrête le continent Européen. Le critère géographique est sur déterminé par le critère politique comme en Ukraine où ne sera établi qu'un accord d'association. \\
On ne sait pas ce qu'est être européen donc au final, la seule condition est sur les valeurs, d'après les textes.


En réalité, les conditions sont plus précises et se sont décalés au fur et à mesure des élargissements. La première vague a fait apparaître la condition de la reprise totale du droit de l'Union (la condition de l'acquis communautaire). La vague du Sud montrera qu'il est nécessaire de remplir la condition de démocratie. La vague de l'est a permis de systématiser les critères de Copenhague. Le quatrième critère est la capacité d'absorption de l'Union (l'intégration ne doit pas mettre en péril l'intégration politique de l'Union).


i. Dès la première vague est mis sur la table la condition de la reprise de l'acquis communautaire: les dispositions des traités et tous les actes pris par les institutions formant le droit de l'Union. Ils doivent prendre la totalité de ce que l'Union a déjà fait. Être candidat s'inscrit dans le temps car la commission doit s'assurer que le droit de l'État candidat reprend bien l'acquis communautaire et nécessite souvent une grande transformation du droit de l'État candidat. \\
Les modalités sont des mesures de transitions: pendant un certain temps, l'État peut avoir son droit non conforme avec l'acquis communautaire. 


ii. Le conseil Européen de Copenhague a déclaré que la démocratie représentative et le respect des droits de l'Homme constitue deux conditions essentielles à l'UE. Cette condition s'est posé d'elle même à la lumière de l'élargissement de l'UE aux pays du sud: Espagne, Grèce, Portugal. \\
Concernant la Grèce, il y avait dès 1967 un accord d'association. Accord gelé pendant la dictature des colonels. À la fin de cette dictature, la Grèce engage une procédure et intègre l'UE en 1981. \\
Portugal: révolution oeillets ; Espagne: mort de Franco. \\
Le problème de ces pays était leur retard de leur développement pour les intégrer dans la communauté européenne. L'Espagne posait un soucis car était un pays agricole, or la PAC (Politique Agricole Commune) rendait la CEE excédentaire sur certains produits agricoles. Il a donc fallu attendre 10 ans avant qu'ils intègrent vraiment l'UE. Ils sont rentrés dans l'Union le 1er Janvier 1986. \\
À partir du moment où il y avait un retour à la démocratie, il n'y avait plus vraiment d'obstacles à l'adhésion de ces pays.


iii. Les critères de Copenhague sont relatifs aux pays de l'est. À la suite de l'implosion du bloc soviétique, le Conseil Européen de Copenhague de 1993 a assurée aux pays d'Europe de l'est qu'ils pouvaient intégrer l'UE sous trois conditions: la première condition est l'État de droit évidemment ; la deuxième condition est économique: économie de marché viable pouvant s'insérer dans l'espace concurrentiel du marché intérieur ; la troisième condition est l'acquis: la capacité des pays à assurer les obligations en tant que membre de l'UE. \\
Une des particularités de l'intégration de ces pays est qu'ils ont mis 15 ans à rentrer dans l'UE. Cela leur a été bénéfique car ils ont transformé progressivement leurs État d'un point de vue économique (passer d'une économie dirigiste à une économie de marché) et d'un point de vue politique (intégration de l'acquis européen). \\
Il y a eu des aides de l'UE de la part du BERD notamment de sorte à restructurer les anciens États communistes. \\
La Roumanie et la Bulgarie ont mis plus de temps à rentrer car ils ont eu un dysfonctionnement de l'appareil d'État qui était fortement corrompu. \\
S'agissant de la Croatie, dernier pays à être rentrer dans l'Union, on leur a imposé une collaboration totale avec le TPIY.


iv. Un quatrième critère existe et pourrait prendre de l'importance dans l'avenir: c'est la capacité d'absorption de l'Union. \\
Juridiquement, on ne sait pas ce que ça veut dire. \\
Sur une dimension économique: il s'agit d'absorber l'économie d'État en retard économiquement sans déséquilibrer le marché intérieur. Typiquement, le coût du travail étant différent, il y a possibilité d'une concurrence conflictuelle. \\
La deuxième dimension est la dimension institutionnelle. Dans la commission, il y a un commissaire par État, et de 12 commissaires en 1992, on est aujourd'hui à 28 commissaires. Le problème est le même avec les parlementaires ou au conseil. \\
Il y a donc eu une succession de modification des traités. Le premier acte important est l'acte unique européen en 1986. En 1992, il y a Maastricht qui lance l'Euro et la citoyenneté européenne. Ensuite intervient le grand élargissement. \\
Il y a le traité d'Amsterdam en 1997, qui posera l'espace de liberté, de sécurité et de justice, mais sera un échec sur la réforme des institutions. \\
Le traité de Nice de 2001 réforme les institutions. \\
Un traité institue une Constitution en 2004. Celui-ci a été rejeté en 2005 par la France et les Pays-Bas. Lorsqu'il a été rejeté, il a été nécessaire de le reprendre en partie et d'en faire le traité de Lisbonne de 2007. \\


Le critère de la capacité d'absorption a pris de l'ampleur avec la perspective d'intégrer la Turquie. Dès 1963, les accords d'Ankara sont signés, ce qui montre la volonté de la Turquie d'intégrer l'UE. La procédure est très longue, plus de 50 ans de discussions, 29 ans depuis la candidature officielle. Il faudra attendre 1999 pour que l'UE prenne acte de la candidature de la Turquie. Les négociations commenceront officiellement en 2005. \\
Lorsqu'il y a officiellement des négociations, est fait référence à la capacité d'absorption d'un nouvel État membre. Il ne s'agit pas seulement de la dimension économique de la Turquie. La capacité d'absorption vise surtout la dimension politique. En 2005, la France et l'Allemagne ont émis des réticences concernant les conséquences de la puissance de la Turquie (puissance diplomatique et démographique). Accepter la Turquie au sein de l'Union, ça serait accepter un pays aussi grand que l'Allemagne. \\
La Turquie pose la question entre deux visions de l'UE: l'Europe-espace ou l'Europe-puissance. Le critère de capacité d'absorption pose donc en fait le problème de la vision de l'Europe. La Turquie augmenterait-elle la puissance de l'Union ou la diminuerait-elle ?


Aujourd'hui, il y a quatre candidatures d'adhésion: trois pays en ex-Yougoslavie: l'ancienne République de la Macédoine ; la Serbie ; le Monténégro. La Turquie étant bien sûr le dernier pays. \\
De plus, le Gouvernement Islandais a décidé de retirer sa candidature. \\
Quand on regarde ces pays, leur demande d'adhésion n'ont pas le même sens. Pour les trois pays de l'ex-Yougoslavie, il y aura une forte valeur symbolique puisque l'Union sera complète. Ces adhésions sont tournés vers le passé alors que l'adhésion de la Turquie est tourné vers l'avenir.  

\subsection{Se retirer de l'UE}

Le traité de Rome a été signé sans date butoir. Cette absence de date ne signifie pas que les membres ne peuvent pas se retirer. \\
Depuis le traité de Lisbonne, il existe une procédure dédiée au retrait d'un État membre. 

\subsubsection{La procédure de retrait}

La procédure est minimaliste, il faut se référer à l'article 50 du traité sur l'UE. L'État membre doit notifier le conseil Européen. Ensuite, un accord sur les modalités de retrait doit être conclu entre l'État et le conseil à la majorité qualifiée avec l'approbation du Parlement Européen. \\
Le retrait sera effectif à la date fixé dans l'accord, ou, à défaut, deux ans après sa signature. Le traité prévoit que dans le délai de deux ans, l'État puisse proroger ce délai avec l'accord à l'unanimité du conseil. \\
La notification de l'État qui souhaite partir est unilatéral. Il suffit d'une majorité qualifiée au conseil. Il n'est pas nécessaire que chaque État membre donne son accord selon son processus constitutionnel. \\
Aujourd'hui, chaque État membre est libre de quitter l'Union.


Cette procédure est une sécurité juridique et permet de cadrer les choses. Le juriste doit prévoir des choses à l'avance. Cela permet de trouver plus facilement des accords. \\
On ne peut pas aujourd'hui, empêcher un État membre de quitter l'Union au nom du droit des peuples à disposer d'eux même. \\
Si il y a un réel désir de sécession, cela peut se finir par un rapport de force. Le Droit Européen permet donc ici d'empêcher tout recours à la force en prévoyant le cas et une procédure spécifique. \\
L'Union repose sur deux choses essentielles, le pacifisme et le volontarisme. Le droit de retrait est donc cohérent avec cela. 

\subsubsection{Le référendum britannique}

Le Royaume-Uni ne fait pas parti des 6 Pays fondateurs. Très rapidement, le RU va manifester son intention de rejoindre l'UE. En 1963, De Gaulle s'opposera à l'intégration du RU dans la CEE. Il faudra attendre Pompidou pour que le RU intègre la CEE en 1973. \\
En 1975, la Grande-Bretagne organise un référendum sur son maintien dans la CEE, qui se soldera par un oui à 67\%. 


David Cameron, en Janvier 2013 a déclaré qu'il organiserai un référendum sur la sortie de l'UE avant fin 2017 si il était réélu. \\
Ce référendum apparaît de l'extérieur comme une question qui n'a rien à voir avec l'Europe mais plutôt de politique interne où Cameron a cherché à rallier son aile droite à lui. C'est un risque car lui même a dis qu'il voterai pour le maintien dans l'UE. \\
Le Royaume-Uni tire son avantage au niveau de l'Union au niveau du marché intérieur. En effet, la balance commerciale de la Grande Bretagne est positive. De plus, ils sont en dehors de l'Euro, ce qui les avantage pour le développement de la City. \\
Les négociations cherchent aujourd'hui à éviter un Brexit en donnant des concessions au Royaume-Uni pour que Cameron obtiennent le maintien de son pays dans l'UE. \\
Cameron se plaint d'une inflation législative: il y aurait trop de textes de Loi. L'Union devrait donc engager une procédure de simplification du droit et agir seulement quand il serait prouvé qu'il serait plus efficace d'agir au niveau de l'Union (principe de subsidiarité). Cameron cherche aussi à donner plus de pouvoir aux parlements nationaux, ce que le traité de Lisbonne a déjà fait. \\
La commission Européenne initie des textes qu'elle a désormais obligation de transmettre à tous les parlements nationaux qui donnent un avis sur la proposition. Si un certain nombre de parlements sont contre, la commission doit retirer son texte, ou le modifier, ou le garder mais doit le justifier (on appelle ça la procédure du carton jaune). \\
Si on souhaite donner plus de pouvoir aux parlements nationaux, il faudrait leur donner le pouvoir de donner un "carton rouge", ce qui nécessiterai de réviser les traités.


Concernant la zone Euro, le RU demande à pouvoir se protéger des décisions prises au sein de la zone euro si les décisions sont négatives pour les pays hors zone euro. \\
Les britanniques voudraient pouvoir priver pendant 4 ans les autres ressortissants des pays européens de certaines prestations sociales quand ils s'installent au Royaume-Uni. Sur cette question, cela est réglementée par la directive 2004-38CE, donc la demande de Cameron pourrait entraîner la modification de celle-ci. Une modification de traité ne serait donc pas nécessaire. \\
On oppose citoyen européen mobile aux citoyens sédentaires. Si les mobiles faisaient vraiment des abus, il serait possible à l'État de prendre des mesures dérogatoires. Si les mesures dérogatoires peuvent se prendre un cas bien précis, sur vérification de la commission, cela peut être acceptable pour l'Union dans le cadre de la lutte contre les "touristes sociaux".


Si jamais l'accord de négociation devait toucher un traité, ce serai lourd car chaque pays devrait lancer une procédure constitutionnelle pour ratifier le nouveau traité et donc lancer un référendum dans certains pays. \\
La marge de négociation est donc très étroite. \\
Si le RU sort, il existe deux voies d'intégration à l'AELE, le modèle Suisse et une intégration totale de la zone économique (EEE). Le modèle Suisse, c'est plus de 130 accords bilatéraux donc très distant avec l'UE. \\
Être membre de l'EEE, c'est être soumis aux règles du marché intérieur sans participer à leur élaboration et en payant une cote part en plus. Ce qui est bien en deçà de la situation actuelle du RU. \\
On peut noter que la sortie de pays Eurosceptique pourrai permettre d'engager une Europe plus fédérale. 

\section{La citoyenneté}

\subsection{Le statut du citoyen de l'Union}

\subsubsection{La signification du statut du citoyen}

Le manque de légitimité au sein des institutions Européennes s'appelle le déficit démocratique. Ce déficit ne saurait être réglée si il n'existe pas de peuple Européen (un demos européen). Il existes des peuples Européen mais pas un peuple Européen. \\
Cette critique doit être relativisé: si il y aurai un peuple Européen, nous n'aurions pas besoin d'inventer l'intégration Européenne. Il ne faut pas oublier d'où on part. Au départ, la formation d'une communauté Européenne ne nécessitait pas de légitimité. \\
C'est en raison du succès du fonctionnalisme qui a créée la nécessité de créer des institutions politiques et démocratiques. L'entreprise européenne a pour but de faire advenir le peuple Européen. La construction de l'Union s'inscrit dans le temps, on ne peut donc pas reprocher à l'Union ce qui n'existait pas, et ce qui n'est pas encore advenu. \\
L'Union n'a pas vocation à créer un super État Européen (fédéral ou non) à court et moyen terme. 


Si il n'y a pas de peuple Européen, les peuples sont présents dans les institutions à travers l'importance du critère démographique. La proportion des parlementaires est directement liée à la démographie du pays. De plus, le poids de chaque État au Conseil est corrélé avec sa population. \\
On retrouve aux article 9 du TUE et 20 §1 du TFUE qu'est citoyen de l'Union toute personne ayant la nationalité d'un État membre. La citoyenneté de l'Union s'ajoute à la citoyenneté nationale et ne la remplace pas. Selon l'article 14, §2 du TUE, le Parlement est représentant de citoyens de l'Union. On peut parler d'un citoyenneté hors sol, abstraite, défini essentiellement dans le domaine juridique. \\
Le titulaire de la citoyenneté possède donc un statut avec des droits et obligations. Il n'existe aujourd'hui pas vraiment d'obligation pour les citoyens Européens. \\
La nationalité ou la citoyenneté c'est un statut exprimant le lien juridique liant cette personne à une entité politique regroupant une collectivité humaine. \\
Le statut comprenant les droits propres du citoyen ou du national le distingue de l'étranger à la communauté qui ne bénéficie donc pas des droits propres du citoyen ou du national. On parle alors de statut différencié. \\
La citoyenneté de l'Union doit donc être vu dans sa représentation classique: elle a pour but d'individualiser le statut de citoyens Européens en opposition aux ressortissants d'État tiers à l'Union. \\
Le but final de la citoyenneté Européenne est la même que la citoyenne d'une nation: créer un sentiment d'appartenance, c'est ce qu'on appelle la solidarité. \\
C'est pour ça que la citoyenneté a un caractère constitutionnel.


L'arrêt Rottmann dont l'enjeu était la protection de l'existence de citoyen de l'Union. En l'espèce, un Autrichien poursuivi pour escroquerie s'échappe et se rend en Allemagne et là demande la nationalité Allemande en gardant secret ses déboires judiciaires en Autriche, ce qui a pour effet de lui faire perdre sa nationalité Autrichienne. \\
Quand les autorités Allemandes se rendent compte qu'il a dissimulé ses déboires judiciaires, l'Allemagne lui retire sa nationalité. Ayant perdu ses deux nationalités, il devient apatride et perd donc le statut de citoyen de l'Union. Rottmann conteste sa perte de nationalité devant un juge allemand car deviendrai apatride. Le juge allemand saisi la CJUE pour savoir si il est encore citoyen Européen. \\
La CJUE va décidé de contrôler la proportionnalité du retrait de la nationalité. Un contrôle de proportionnalité suffit à regarder si la justification est raisonnable. Elle va considéré que, de 1. à partir du moment où il y a fraude, la décision de retrait est justifié, cependant la conséquence de l'apatridie est très grave et donc de 2. à invité l'Autriche à réintégré Rottmann. \\
La CJUE a donc décidé que la citoyenneté Européenne n'était plus la chose des État mais bien de l'Union, l'Union pouvant protéger les nationaux contre leur propre État. 


Une décision du CC le 23 Janvier 2015, saisi d'une QPC, concernait en l'espèce un franco-marocain déchu de sa nationalité pour terrorisme. La question concernait le principe d'égalité: il y avait une différence entre une personne née Française et une personne naturalisé. Le CC répondra que les citoyens sont égaux peu importe leur mode d'acquisition de la nationalité. Or, la différence de traitement est expliqué par la gravité des faits. Cette différence, selon un auteur, relève d'un point de vue colonialiste de la nationalité. \\
Le droit de l'Union s'appliquera et il y aura donc un contrôle de proportionnalité concernant la déchéance d'un binational. Souvent, dans ces cas là, l'UE laisse une marge d'appréciation aux États (principe de subsidiarité). 

\subsection{Les droits du statut de citoyen}

\subsubsection{Les droits politiques du citoyen de l'Union}

Ces droits datent du traité de Maastricht qui a institué la citoyenneté Européenne. Le premier droit est le droit de vote et d'éligibilité aux élections du Parlement Européen ainsi qu'aux élections municipales dans l'État membre où il réside. Si un ressortissant d'un État membre vit dans un autre État membre, il dispose du droit de vote aux élections Européennes ainsi qu'aux élections municipales. \\
Un ressortissant de l'Union peut bénéficier sur le territoire d'un pays tiers de la protection des autorités diplomatiques ou consulaires de tout État membre dans les mêmes conditions qu'un national. \\
Un citoyen Européen dispose aussi de droits administratifs: le droit d'adresser des pétitions au Parlement Européen (concernant un sujet relevant de la compétence de l'Union), le droit de recourir au médiateur (organe de contrôle non juridictionnel, droit à la médiation, élu 5 ans par le Parlement) européen, le droit de s'adresser aux institutions de l'UE dans une des langues du traité et de recevoir une réponse dans cette même langue. 




\subsubsection{}


\chapter{Les valeurs de l'UE}








\chapter{Les buts de l'UE}












\end{document}
