\documentclass[12pt, a4paper, openany]{book}

\usepackage[latin1]{inputenc}
\usepackage[T1]{fontenc}
\usepackage[francais]{babel}
\date{}
\title{Cours de droit communautaire ou droit de l'Union Européenne (UFR Amiens)}
\pagestyle{plain}

\begin{document}

% Jacqué - Droit institutionnel de l'UE - Éditions Dalloz
% Blumann - Droit institutionnel de l'UE - Éditions Litec

\chapter{Introduction}

\section{Distinction: OI de coopération et OI d'intégration}

\subsection{Coopération}

Au lendemain de la seconde Guerre Mondiale naissent en Europe plusieurs Organisations Internationales (OI), l'OTAN notamment, en réponse au bloc soviétique et à son expansionnisme. \\ 
Sur le plan économique, il y a OECE (Organisation Européenne de Coopération Économique) qui avait notamment pour rôle de mettre en place le plan Marshall. L'OECE existe encore sous le nom aujourd'hui de l'OCDE (Organisation pour la Coopération et le développement Économique). \\
Le Conseil de l'Europe a été créé pour le rapprochement des peuples européens.


Leur point commun est d'être toutes des OI de coopération. Ces OI ont pour rôle de favoriser la coordination des activités des États partis dans un domaine spécifique et dans la poursuite d'un intérêt commun. Cela doit se faire nécessairement sans transfert de souveraineté: cela implique deux choses, l'armature institutionnelle est simple, l'organe central décisionnel est formé des représentants des États, et les décisions sont prises à l'unanimité (et en cas de vote à la majorité, les États peuvent mettre un véto). \\
Le CE compte 47 pays Européens en plus des 28 membres de l'UE. Le CE comprend deux organes: le comité des ministres et une assemblée consultative formée des représentants des parlementaires des pays. C'est le comité qui prend des recommandations aux deux tiers des voix: il n'y a pas de pouvoir de coercition. Ce comité peut approuver des conventions internationales.  \\
Le CE est incapable de produire une norme qui ne soit pas la volonté des États. Soit le comité adoptera une recommandation (soft-law), soit il approuve une convention où l'État donnera son consentement. \\
Les destinataires des normes émises par le comité sont les seuls États partis et non les individus/particuliers. Pour les conventions, c'est la loi nationale qui fixe les modalités précises de la réception de cette convention ou la convention elle même (la Charte Sociale Européenne ne bénéficie pas d'une juridiction distincte comme la CEDH. Cependant, les individus peuvent invoquer son article 6 qui donne droit une protection en cas de licenciement). 

\subsection{Intégration}

À ces OI de coopération, on oppose les OI d'intégration comme l'UE. \\
L'UE a été créée par le traité de Maastricht en 1992. Avant 1992, il y avait la CEE issu du traité de Rome. \\
L'OI d'intégration est caractérisée par un transfert de souveraineté des États à l'OI qui exerce ces compétences sans dépendre des États. On remarquera les quatre traits suivants:
\begin{enumerate}
\item Les États sont souvent représentés mais la majorité qualifié peut imposer sa norme aux autres (on notera certains sujets dits "sensible" qui ont besoin d'unanimité comme la fiscalité directe) ;
\item D'autres organes qui ne représentent pas les États peuvent participer à la prise de décision (le parlement Européen donne en principe son accord pour l'adoption des directives et règlements de l'Union) ;
\item Les normes adoptés sont en principes contraignantes pour les États ;
\item Les pouvoirs de l'OI s'exerce immédiatement sans passer par les intermédiaires des gouvernements nationaux, au profit ou à la charge des particuliers.
\end{enumerate}


Le droit de l'UE est présenté comme le droit de l'intégration. Ce terme vient d'un ouvrage publié en 1970 publié par Pescatorof "Le droit de l'intégration: émergence d'un phénomène nouveau dans les relations internationales selon les communautés". \\
Selon lui, le droit de l'intégration a deux aspects: intégration indépendant des États ou intégration institutionnelle (institutions adoptant elle même des normes, non prévus par le traité) ; intégration normative: la primauté de la norme de l'Union et effet direct de la norme de l'Union.

\subsubsection{La primauté} 
 
La primauté signifie que toute norme nationale contraire aux normes de l'Union est inopposable à la pleine application de la règle de l'Union. On parle aussi de prévalence. La primauté assure l'effectivité de l'application uniforme du droit de l'Union dans tous les États membres. Cela est vital car si les normes nationales (même constitutionnelles) peuvent s'opposer au droit de l'Union, le droit ne serait plus uniforme donc mis en péril. Les institutions Européennes revendiquent la primauté de leur Droit, mais les nations revendiquent la primauté de leur droit national. \\
En Droit Français, le droit de l'Union bénéficie d'une immunité constitutionnelle. Le Conseil Constitutionnel a prévu une réserve de constitutionnalité ou contre limite constitutionnelle qui dit qu'une règle de l'Union ne peut pas être contraire à une règle ou un principe inhérent à l'identité constitutionnelle de la France. Il n'y a jamais eu en France de conflit de constitutionnalité avec le droit de l'Union. 

\subsubsection{L'effet direct}

L'effet direct est l'aptitude d'une norme à produire elle même des effets dans l'ordre interne de l'État en ce qu'elle confère directement un droit dans le patrimoine juridique des particuliers. Pour être d'effet direct, il faut que la norme soit auto-exécutoire donc que la norme soit suffisamment précise. \\
L'Union interfère dans la relation entre l'État et sa population. Cette relation n'est plus exclusive car l'Union est capable de créer un lien direct avec les ressortissants des États membres. Chaque ressortissant des États membre est un citoyen de l'UE. \\
Si le droit national est contraire aux normes internationales, l'État est responsable devant les autres États mais le particulier peut se prévaloir de la norme internationale dans le cadre de l'UE. \\

\subsection{Ordre juridique}

Le Droit de l'UE est aussi efficace qu'un Droit étatique. La CJUE s'assure de l'application des normes Européennes dans les États membres.  \\
L'ordre juridique Européen est sophistiqué. Il faut entendre comme système juridique, selon Hart, des règles primaires, donc des règles de comportement (faire, ne pas faire, faculté de faire) et des règles secondaires qui recouvrent ce qui touche à la création, à l'application et à la validité des normes primaires. Ce sont des normes sur la norme. \\
L'ensemble juridique est l'ensemble des normes qui régissent une entité, cette entité peut être étatique, infra-étatique (droit canonique) ou encore supra-étatique (droit de l'Union). Le droit de l'Union possède des normes primaires et secondaires. \\
Seul l'ordre juridique de l'État est complet. Il maîtrise entièrement l'expression du droit: l'extinction et la création des normes. L'application et la sanction des normes. L'État a la compétence de la compétence (souveraineté), son domaine d'actions est illimité puisqu'il se donne lui même sa compétence. C'est aussi lui qui a la force nécessaire (la violence publique légitime). \\
L'ordre juridique de l'Union est incomplet: l'Union n'est compétente pour agir que dans la mesure des transferts de compétence consentis par les États dans les traités: on parle de principes d'attribution des compétences de l'Union. Par conséquent, l'Europe n'a pas la compétence de la compétence (la souveraineté). \\
On peut décrire l'Union comme une puissance normative qui produit beaucoup de règles normatives. Mais elle ne dispose pas de la force matérielle pour faire exécuter ces règles normatives. L'exécution des normes de l'Union se fait donc par l'État même quand ces normes sont contre les États membres. \\
Même si l'ordre juridique de l'Union est incomplet, il est le plus efficace au Monde. Jean Monnet a été le concepteur de la méthode d'intégration communautaire, en sachant que cette méthode d'intégration communautaire marque la spécificité de l'intégration de l'UE. 

\section{Méthode d'intégration communautaire}

\subsection{La pensée de Jean Monnet}
Durant la première guerre mondiale, Jean Monnet était en contact avec les décideurs britannique en tant qu'officier en charge du ravitaillement. Pendant la seconde, il a été le promoteur auprès de Roosevelt pour que les USA s'engagent en guerre. Il a aussi été un fondateur de la SDN. \\
Monnet a un objectif: la paix qui ne peut passer que par la réconciliation de la France et l'Allemagne. \\
De la SDN, il a appris que toute entreprise de rapprochement des peuples qui reposerait sur une OI composé essentiellement de représentants d'États disposant de droit de véto est voué à la paralysie dès qu'une question fait apparaître des divergences d'intérêts entre les États. \\
Dans ses mémoires, il explique très clairement qu'il a en tête de neutralisé ce droit de véto et donc de neutralisé la souveraineté des États. On distinguera la souveraineté-légitimité et la souveraineté-puissance. La souveraineté désigne le caractère suprême d'une puissance, qui n'est soumise à aucune autre et qui a le pouvoir de s'imposer unilatéralement. \\
L'État dispose de la souveraineté en disposant de la violence publique légitime. La souveraineté de l'État signifie donc qu'il est maître chez lui (pouvoir de commandement suprême): c'est la souveraineté interne et d'autre part il n'est soumis à aucune autre puissance à l'extérieur (souveraineté externe ou souveraineté-liberté). En relations internationales, cela conduit au droit de véto: personne ne peut vouloir pour l'État ou contre l'État, c'est pour ça que Jean Monnet ne croit pas que le Conseil de l'Europe qui a pour but premier d'établir la paix en Europe disposera des moyens d'atteindre cet objectif car le Conseil de l'Europe ne dispose pas des moyens de surmonter les non vouloir des États.

\subsection{La méthode d'intégration communautaire}
L'acte fondateur de l'UE est la déclaration Schuman du 9 Mai 1950. Il est ministre des affaires étrangères en France. Il proposera avec le chancelier Allemand la création du CECA ayant pour but la mise en commun de la production de charbon et d'acier en Europe. \\
Le traité instituant la CECA a été signé par 6 États: France, Allemagne, Italie, Belgique, Pays-Bas, Luxembourg, signé en 1951, entré en vigueur en 1952: le traité était actif pour 50 ans. \\
La déclaration Schuman a été rédigé par Jean Monnet qui a aussi imaginé la CECA, la première communauté Européenne puis inspirera ensuite la CEE ainsi que la CEEA (tout deux créés en 1957). On parle souvent des communautés européennes. \\
L'acte fondateur de l'Europe semble une expression exagérée pour un traité qui ne parle que de charbon et d'acier (la CECA), cependant, ce sont deux ressources qui sont loin d'être anodines: le charbon est la source d'énergie principale des industries, et l'acier utile à l'armement. Cela est donc fortement symbolique car l'objectif est la paix entre l'Allemagne et la France. \\
La CECA pose les bases de la méthode d'intégration communautaire. Cette méthode repose sur deux piliers: une forme de supra-nationalité par la construction d'institutions indépendantes des États ; ainsi que le fonctionnalisme.

\subsubsection{Supra-nationalité}

Les États par un traité international opère un transfert de compétence déterminé en faveur de l'OI et donc de ses organes. L'OI dispose donc de compétences d'attribution. L'État est État membre et ses pouvoirs sont donnés dans le traité: il est impliqué dans les processus décisionnels mais uniquement dans les procédures prévues par le traité. C'est là que s'opère le dépassement de la souveraineté prévu par Jean Monnet: il y a une forme divisibilité de la souveraineté matérielle de l'État. L'État accepte d'exercer en commun, avec d'autres États, certaines compétences compétence qui peuvent être régaliennes (la monnaie unique). \\
Pourtant, la souveraineté est censé être indivisible: c'est vrai pour l'essence de la souveraineté. En revanche, la souveraineté peut être divisée concernant l'exercice de la compétence. Là est le dépassement de la souveraineté. \\
C'est précisément ce que prévoit la CECA. Monnet parlera de fusion partielle car ça ne concerne que le domaine de compétence transféré à l'OI.


La structure de la CECA est original et révolutionnaire: elle met en place un quadripartisme institutionnel: une haute autorité qui est la préfiguration de ce que sera la Commission Européenne ; un conseil des ministres ; une assemblée composée de parlementaires nationaux ayant uniquement un pouvoir consultatif ; une cour de justice qui veillera au respect des règles du traité. \\
Le plus étonnant est la haute autorité car elle concentre le pouvoir décisionnel. Les membres sont désignés par les États, mais ensuite, ils ne représentent pas du tout les États. Le premier président de la CECA était Jean Monnet. Cette haute autorité est particulière car prend elle même les décisions relatives à l'organisation du marché du charbon et de l'acier. Elle fixe par exemple le niveau de production ou s'assure que la concurrence loyale et non faussée est respecté. Pour contester une décision de la haute autorité, il faut s'adresser à la cour de justice qui verra si la haute autorité a respecté le traité. \\
Cela permet de contrebalancé l'instabilité et l'inertie de la diplomatie traditionnelle. La CECA est l'exact contraire du Conseil de l'Europe. \\
Cela introduit une légitimité nouvelle dans les relations internationales: la compétence poursuivant un but d'intérêt général. \\
L'organisation de la CECA est l'exact opposé de la politique démocratique où les candidats et donc les élus qui conduisent des politiques tournés vers le court terme. Le but de Jean Monnet est donc de créer une gouvernance basée sur la raison. \\



\end{document}
